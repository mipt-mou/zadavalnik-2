\documentclass[russian,10pt]{article}

\usepackage[intlimits]{amsmath}
\usepackage{amsthm,amsfonts}
\usepackage{amssymb}
\usepackage{mathrsfs}
%\usepackage{graphicx}
\usepackage[final]{graphicx,epsfig} 
\usepackage{longtable}
\usepackage{indentfirst}
\usepackage[utf8]{inputenc}
\usepackage[T2A]{fontenc}
\usepackage[russian,english]{babel}
\usepackage[usenames]{color}
\usepackage{esint}




\pdfpagewidth 14cm
\pdfpageheight 20cm

\textwidth 11cm
\textheight 16.5cm
\oddsidemargin 1.5cm % 1.84cm
\topmargin 1.5cm % 1.8cm
\footskip 1cm

\hoffset -1in
\voffset -1in
\headheight 0pt
\headsep 0pt

\hyphenpenalty=300
%\tolerance=800
\binoppenalty=10000

\clubpenalty=10000 \widowpenalty=10000  % подавление "висячих строк"
\tolerance=2000  % терпимость к жидким строкам
\righthyphenmin=2  % минимальное число символов при переносе* 

% Adjust first page number according to real document position in the book.
\setcounter{page}{1}

% Dot after section number
\makeatletter
% In section title
\def\@seccntformat#1{\csname the#1\endcsname.\quad}
\makeatother

%\tolerance = 2000

% To place author above title
\def\maketitle{
  \begin{center}


\thispagestyle{empty} 
\begin{center}

МИНИСТЕРСТВО ОБРАЗОВАНИЯ И НАУКИ \\
РОССИЙСКОЙ ФЕДЕРАЦИИ \\

 $ $ \\

Московский физико-технический институт \\
(государственный университет) \\
 
  $ $ \\ $ $ \\ $ $ \\
  $ $ \\
  $ $  \\
\smallskip
\smallskip
\premierAuthors, \\
\autresAuthors \\
 
 $ $ \\ $ $ \\
 
\textbf{СТОХАСТИЧЕСКИЙ АНАЛИЗ \\ В ЗАДАЧАХ} \\
 
 $ $ \\
  $ $ \\
  $ $ \\
Учебно-методическое пособие \\
 
  $ $ \\ $ $ \\ $ $ \\ $ $ \\
  \smallskip
   Часть 2
 $ $ \\ $ $ \\ $ $ \\ $ $ \\ $ $ \\ $ $ \\
 
Москва--Долгопрудный 2015


\end{center}

\newpage
  \end{center}
}

\renewcommand{\refname}{Литература}

%\sloppy
%\DeclareGraphicsRule{*}{eps}{*}{}
\DeclareMathOperator{\diam}{diam}

\graphicspath{{images/}}
\newcommand{\imgh}[3]{\begin{figure}[!h]\center{\includegraphics[width=#1]{#2}}\caption{#3}\label{Fig:#2}\end{figure}}

% Perfectly typesetted tilde for url links.
%\def\urltilde{\kern -.15em\lower .7ex\hbox{\~{}}\kern .04em}

\addto{\captionsrussian}{
	\renewcommand{\proofname}{\bf Решение. }
}

\usepackage{wasysym}
\usepackage{verbatim}

\theoremstyle{definition}
%\newtheorem{problem}{\noindent\normalsize\bfЗадача \No\!\!}
\newtheorem{problem}{\noindent\normalsize\bf{}}[section]
\renewcommand{\theproblem}{\arabic{problem}}
\newtheorem*{example}{\noindent\normalsize\bf{}Пример}
\newtheorem*{definition}{\noindent\normalsize\bf{}Определение}
\newtheorem*{remark}{\noindent\normalsize\bf{}Замечание}
\newtheorem{theorem}{\noindent\normalsize\bf{}Теорема}
\newtheorem*{lemma}{\noindent\normalsize\bf{}Лемма}
\newtheorem*{suite}{\noindent\normalsize\bf{}Следствие}


\newtheorem*{ordre}{\noindent\normalsize\bf{}Указание}

\newenvironment{solution}{\begin{proof}\vspace{1em}} {\end{proof} \vspace{2em}}



%\newcommand{\fixme}[1]{\textcolor{red}{\noindent\normalsize\frownie{}[#1]}} %for displaying red texts

\newcommand{\fixme}[1]{}


\usepackage{enumitem}

\RequirePackage{enumitem}
\renewcommand{\alph}[1]{\asbuk{#1}} % костыль для кирилической нумерации 

\setenumerate[1]{label=\alph*), fullwidth, itemindent=\parindent, 
  listparindent=\parindent} 
\setenumerate[2]{label=\arabic*), fullwidth, itemindent=\parindent, 
  listparindent=\parindent, leftmargin=\parindent}

\newcommand{\rg}{\ensuremath{\mathrm{rg}}}
\newcommand{\grad}{\ensuremath{\mathrm{grad}}}
\newcommand{\diag}{\ensuremath{\mathrm{diag}}}
\newcommand{\const}{\ensuremath{\mathop{\mathrm{const}}}\nolimits}
\newcommand{\Var}{\ensuremath{\mathop{\mathbb{D}}}\nolimits}
\newcommand{\Exp}{\ensuremath{\mathrm{{\mathbb E}}}}
\newcommand{\PR}{\ensuremath{\mathrm{{\mathbb P}}}}
\newcommand{\Be}{\ensuremath{\mathrm{Be}}}
\newcommand{\Po}{\ensuremath{\mathrm{Po}}}
\newcommand{\Bi}{\ensuremath{\mathrm{Bi}}}
\newcommand{\Ker}{\ensuremath{\mathrm{Ker}}}
\newcommand{\Real}{\ensuremath{\mathrm{Re}}}
\newcommand{\Lin}{\ensuremath{\mathrm{Lin}}}
\newcommand{\Gl}{\ensuremath{\mathrm{Gl}}}
\newcommand{\mes}{\ensuremath{\mathrm{mes}}}
\newcommand{\cov}{\ensuremath{\mathrm{cov}}}
\newcommand{\I}{\ensuremath{\mathrm{I}}}
\newcommand{\N}{\ensuremath{\mathcal{N}}}
\newcommand{\KL}{\ensuremath{\mathcal{KL}}}

\usepackage{url}
\makeatletter
\g@addto@macro{\UrlBreaks}{\UrlOrds}
\makeatother

\begin{document}

\selectlanguage{russian}

\maketitle

\thispagestyle{empty} 

УДК 519.21
 
 $ $ \\ $ $ \\
 
\premierAuthors, \autresAuthors. Стохастический анализ в задачах: Учебно-методическое пособие. Часть 1 / МФТИ. М.--Д., 2015, изд. 3-е, доп.

$ $ \\ 
 
\begin{abstract}

Содержит программу, список литературы и задачи одноименного курса, читаемого сотрудниками кафедры Математических основ управления студентам факультета управления и прикладной математики Московского физико-технического института. Задачи могут быть использованы в качестве упражнений на семинарских занятиях, сдачах заданий, экзаменах, а также при самостоятельном освоении курса. В новое издание добавлен ряд задач, отражающих также опыт преподавания вероятностных дисциплин в Независимом московском университете и опыт наших коллег из ПреМоЛаб МФТИ, преподающих вероятностные дисциплины в ведущих научных центрах Германии и Франции. Данный сборник задач разрабатывался с учетом направлений подготовки: студентов МФТИ, студентов НМУ, магистров факультета Компьютерных наук ВШЭ и магистров факультета Прикладной математики БФУ им. И.~Канта.  

 $ $ \\ $ $ \\ $ $ \\ $ $ \\ $ $ 

\noindent рецензенты: д.ф.-м.н. А.В. Колесников \\
\noindent проф. Факультета математики НИУ ВШЭ \\
\noindent д.ф.-м.н. А.Н. Соболевский \\ 
\noindent проф. Независимого московского университета, \\ зам. директора ИППИ РАН 

\end{abstract}


  
 \newpage 

\thispagestyle{empty} 
\begin{center}

{ОБЯЗАТЕЛЬНАЯ ЧАСТЬ ПРОГРАММЫ УЧЕБНОГО КУРСА  \\ 
«Теория вероятностей»  }
 
\end{center}
 
 {\small
 
Интуитивные предпосылки теории вероятностей. Множество элементарных исходов опыта, событие. Классическое и статистическое определение вероятности. Математическое определение вероятности. Алгебра и сигма-алгебра событий, минимальная сигма алгебра. Аксиомы теории вероятностей и следствия из них. Вероятностное пространство.

Теорема непрерывности вероятности. Теорема сложения вероятностей. Зависимые и независимые события. Условная вероятность события. Формула полной вероятности. Формула Байеса. Леммы Бореля--Кантелли. Закон ``0-1'' Колмогорова.

Случайная величина как измеримая функция. Функция распределения случайной величины. Дискретные и непрерывные случайные величины. Плотность распределения вероятностей. Формула включений-исключений.

Конкретные распределения случайных величин. Схема Бернулли, геометрическое и биномиальное распределение. Простейший поток событий и распределение Пуассона. Показательное, равномерное, нормальное, log-нормальное и отрицательно-биномиальное распределения. Бета-распределение и гамма-распределение.

Случайный вектор. Функция распределения случайного вектора. Зависимые и независимые случайные величины, условные законы распределения. Функции случайных величин. Невырожденное функциональное преобразование случайного вектора.
Интеграл Стилтьеса. Математическое ожидание и дисперсия случайной величины. Моменты случайной величины. Условное математическое ожидание. Корреляционная матрица случайного вектора. Коэффициент корреляции двух случайных величин.

Характеристическая функция и ее свойства. Связь моментов случайной величины с ее характеристической функцией. Разложение характеристической функции в ряд.
Сходимость последовательностей случайных величин с вероятностью единица (почти наверное), порядка $p$ (в среднем квадратичном), по вероятности, по распределению. Соотношение между различными типами сходимости. 

Неравенство Чебышева. Закон больших чисел. Критерий Колмогорова. Теоремы Хинчина и Чебышева.  Усиленный закон больших чисел. Теорема Колмогорова и Бореля. Оценивание скорости сходимости частоты к вероятности в схеме Бернулли. Неравенство Бернштейна. 

Интегральная и локальная теоремы Myавра--Лапласа. Дискретная поправка. Теорема Линдберга. Центральная предельная теорема для одинаково распределенных случайных величин. Центральная предельная теорема в форме Натана. Условие Ляпунова. Теорема Гливенко.
}


\newpage

\section{Введение}

В основу предлагаемого сборника задач по теории вероятностей положены задачи (в том числе повышенной сложности), предлагавшиеся в разные годы студентам факультета управления и прикладной математики (ФУПМ) МФТИ и Независимого московского университета на семинарах, сдачах заданий и экзаменах. Главными отличительными особенностями пособия являются: а) широкий спектр представленного материала, б) отражение ряда современных направлений развития теории вероятностей и в) нацеленность на приложения.

По замыслу авторов, предлагаемый сборник задач отчасти демонстрирует роль курса как “вероятностного фундамента” для ряда других дисциплин: прикладной статистики, стохастических дифференциальных уравнений, эффективных алгоритмов, экспериментальной экономики (финансовой математики) и др. Несмотря на широкий спектр представленных тем, основной  акцент делается на формирование у читателей геометрической интуиции, восходящей к Пуанкаре, которая позволяет с единых позиций понять многообразие асимптотических результатов стохастической теории как проявление одного общего принципа концентрации меры. 

Большое внимание в сборнике уделяется различным приемам доказательства предельных теорем – асимптотических результатов теории вероятностей. Для этого прежде всего используется аппарат производящих функций и теории функций комплексного переменного.

Более половины представленных в сборнике задач не являются стандартными. Для таких задач даны указания, комментарии (замечания), ссылки на публикации. 

Желающих более глубоко изучить представленные темы отсылаем к списку литературы, приведенному в конце сборника задач. 

Следует отметить, что за последние десять лет заметно возросло значение для выпускников Физтеха глубоких знаний вероятностных дисциплин. Это обусловлено множеством причин.

Прежде всего, это вызвано широким распространением задач анализа больших массивов данных (machine learning, data mining). Подтверждением служит взаимная востребованность студентов ФУПМ и Школы анализа данных компании Яндекс, большая популярность среди  студентов ФУПМ кафедры интеллектуальных систем (заведующий кафедрой член-корреспондент РАН К.В. Рудаков, Вычислительный центр РАН) и кафедры предсказательного моделирования (заведующий кафедрой академик РАН А.П. Кулешов, Институт проблем передачи информации РАН). 
%Во многом именно для таких студентов и написаны разделы \ref{bayes}, \ref{zb4}, \ref{MK}, \ref{stats}, ССЫЛКА[Оптимизация и стохастика]. %3 – 6, 9, 10.

Другая не менее важная причина -- разработка эффективных (приближенных, рандомизированных) алгоритмов решения сложных задач. 
%Этому посвящены разделы \ref{MK}, \ref{information}, \ref{CS}, ССЫЛКА[Оптимизация и стохастика].
Сложно, например, представить себе современного специалиста по моделированию, который бы не использовал методы Монте-Карло. Также сложно представить себе специалиста в области computer science, которому не приходилось бы применять рандомизированные алгоритмы и подвергать алгоритмы вероятностному анализу (например, для оценки сложности в среднем). 

Еще одна причина  связана с тем, что приложение вероятностных методов к анализу и разработке экономических моделей для части студентов ФУПМ является фундаментом в работе на базовых кафедрах. Например, заведующие базовыми кафедрами ФУПМ член-коррерспондент РАН И.Г. Поспелов (Вычислительный центр РАН) и  член-коррерспондент РАН Ю.С. Попков (Институт системного анализа РАН) активно работают в направлении разработки вероятностных моделей экономических агентов и вероятностного анализа агломерационных моделей. %Этому посвящен раздел ССЫЛКА[Марковские модели макросистем].

Наконец, можно заметить, что задачи анализа больших компьютерных, социальных, транспортных сетей в последнее время выходят на передний план во многих приложениях. Огромную роль в изучении таких сетей играют вероятностные модели, некоторые из которых будут приведены в предлагаемом сборнике задач. 
%Разделы \ref{hard}, \ref{genF}, \ref{combinatorics}, ССЫЛКА[Марковские модели макросистем] сборника посвящены изучению подобных сетей, в частности, различным предельным переходам.

Важную роль в подготовке настоящего сборника задач сыграла Лаборатория структурных методов анализа данных в предсказательном моделировании (ПреМоЛаб), открытая на базе ФУПМ МФТИ в 2011 году. В частности, благодаря этой лаборатории, у студентов есть возможность посмотреть на сайте www.mathnet.ru, www.premolab.ru (семинары: Стохастический анализ в задачах, Математический кружок и др.) видеозаписи выступлений ведущих ученых, посвященные ряду нестандартных задач из этого сборника. 

Мы благодарны нашим коллегам Д.В. Беломестному, Я.И.~Белопольской, М.Л.~Бланку, Н.Д.~Введенской, В.В.~Веденяпину, Д.П.~Ветрову, А.М.~Вершику, К.В.~Воронцову, В.В.~Вьюгину, В.В.~Высоцкому, А.В.~Калинкину, М.Н.~Вялому, М.С.~Гельфанду, Э.Х.~Гимади, Г.К.~Голубеву, А.Б.~Дайняку, П.Е.~Двуреченскому, Н.Х.~Ибрагимову, М.И.~Исаеву, Г.А.~Кабатянскому, А.В.~Колесникову, А.В.~Леонизову, Г.~Лугоши, Ю.В.~Максимову, В.А.~Малышеву, В.Д.~Мильману, В.В.~Моттлю, Т.А.~Нагапетяну, А.В.~Назину, Ю.Е.~Нестерову, А.С.~Немировскому, В.И.~Опойцеву, Ф.В.~Петрову, С.А.~Пирогову, Б.Т.~Поляку, И.Г.~Поспелову, А.М.~Райгородскому, В.Н.~Разжевайкину, М.А.~Раскину, В.Г.~Редько, A.Е.~Ромащенко, А.В. Савватееву, А.Н.~Соболевскому, А.~Содину, В.Г.~Спокойному, Й.~Стоянову, У.~Сэндхольму, С.П.~Тарасову, И.О.~Толстихину, М.Ю.~Хачаю, О.С.~Федько, Ю.А.~Флёрову, А.X.~Шеню, оказавшим заметное влияние на формирование различных разделов этого учебного пособия, и А.А. Шананину, во многом способствовавшему развитию на ФУПМ базового цикла вероятностных дисциплин. Также отметим большую помощь старшекурсников, аспирантов ФУПМ и участников нашего стохастичексого семинара в НМУ и Физтехе в вычитке этого сборника задач (в особенности, Д.~Бабичева, А.~Балицкого, Ф.~Гончарова, Ю.~Дорна, Е.~Клочкова, А.~Макарова, Е.~Молчанова, Н.~Животовского, М.~Панова, Л.~Прохоренкову(Остроумову), А.~Суворикову, Д.~Петрашко, М.~Широбокова).

\begin{center}
\textbf{
Список обозначений
}
\end{center}

\noindent $\langle \Omega, \mathcal{F}, \PR \rangle$ -- вероятностное пространство ($\Omega$ -- множество исходов, $\mathcal{F}$ -- $\sigma$-алгебра, $\PR$ -- вероятностная мера); \\
$p(x), \; f(x), \; f_X(x)$ -- плотность распределения случайной величины $X$;
$\Exp X$ -- математическое ожидание случайной величины $X$; \\
$\Exp_p X$ -- математическое ожидание $X$ с плотностью распределения $p$; \\
$\Exp_\PR X$ -- математическое ожидание $X$ по мере $\PR$; \\
$\Var X$ -- дисперсия случайной величины $X$; \\
$\N(m,\sigma^2)$ -- нормальное распределение; \\
$\Po(\lambda)$ -- распределение Пуассона; \\
$\Dir(\alpha_1,...,\alpha_n)$ -- распределение Дирихле;\\
$\Beta(\alpha,\beta)$ -- бета-распределение;\\
$\Be(p)$ -- распределение Бернулли; \\
$R[a,b], [a,b]$ -- равномерное распределение; \\
$\Phi(x)$ -- функция стандартного нормального распределения $\N(0,1)$; \\
$\mathrm{Exp}(x)$ -- показательное распределение; \\
$\overset{d}{\longrightarrow}$ $\left(\mathop{\longrightarrow}\limits_{n\to\infty}^{d}\right)$ -- сходимость по распределению, в ряде случаев $n\to\infty$ опущено во избежание громоздких обозначений; \\
$\overset{p}{\longrightarrow}$ -- сходимость по вероятности; \\
$\overset{\text{п.н.}}{\longrightarrow}$ -- сходимость с вероятностью 1;\\
$[x^{n}]\varphi(x)$ -- коэффициент при $x^n$ в разложении в степенной ряд функции $\varphi(x)$;\\
с.в. -- случайная величина; \\
з.б.ч. -- закон больших чисел; \\
х.ф. -- характеристическая функция; \\
ц.п.т. -- центральная предельная теорема; \\
ПФ -- производящая функция; \\
ЭПФ -- экспоненциальная производящая функция;\\
$\langle \cdot, \cdot\rangle$ -- скалярное произведение; \\
Индикаторная функция:
\[
\I(\text{true}) = [\text{true}] = 1, \quad \I(\text{false}) = [\text{false}] = 0;
\]
Простая выборка с плотностью распределения $p$:
\[
X_1,\ldots,X_n \sim p(X);
\]
В схожих ситуациях используется символ $\in$ вместо $\sim$ (например, $X \in \N(m, \sigma^2)$), если подразумевается принадлежность случайной величины  семейству распределений, $\sim$ также обозначает пропорциональность; \\ 
Ненормированная плотность распределения:
\[
p(x) \propto g(x), \quad p(x) = \frac{g(x)}{\int g(x) dx};
\]
Математическое ожидание по заданной переменной:
\[
\Exp_X h(X,Y) = \int_{-\infty}^{+\infty} h(x,Y) dF(x); 
\]
Дивергенция Кульбака--Лейблера для распределений $\PR_1$ и $\PR_2$ с общим носителем~$\Omega$ (соответствующие плотности распределений обозначены как $p_1$ и $p_2$):
\[
\KL(\PR_1 \Vert \PR_2) = \int_{\Omega} \log \left( \frac{p_1(x)}{p_2(x)} \right) p_1(x) dx;
\]
Дивергенция Кульбака--Лейблера для распределения $\PR_\theta$, зависящего от параметра (соответствующая плотность распределения обозначена как $p(x| \theta)$):
\[
\KL(\theta_1 \Vert \theta_2) = \KL(\PR_{\theta_1} \Vert \PR_{\theta_2});
\]
$A \triangle B = A \setminus B + B \setminus A$ -- симметрическая разность; \\
$\log = \ln$; \\
$\mathcal{B}$ -- борелевская сигма алгебра;






\thispagestyle{empty} 
\tableofcontents


\input{standart_tasks}


\section{Задачи повышенной сложности}
\label{hard}

\begin{problem}
Докажите, что при $n\to\infty$ 
$$
X_n\xrightarrow{L_2} X \,\Rightarrow\, X_n\xrightarrow{L_1}X \, \Rightarrow\, X_n\xrightarrow{p}X 
\, \Leftarrow\, X_n\xrightarrow{\text{ п.н. }}X , 
$$
$$
X_n\xrightarrow{p}X \, \Rightarrow\, X_n\xrightarrow{d}X . 
$$
С помощью контрпримеров покажите, что никакие другие стрелки импликации в эту схему в общем случае добавить нельзя. 
При каких дополнительных условиях можно утверждать, что 
$$
X_n\xrightarrow{\text{ п.н. }}X  \, \Rightarrow\, X_n\xrightarrow{L_1}X ?
$$
Кроме того, показать, что 
$$
X_n\xrightarrow{p} X \; (n\to\infty) \,\Leftrightarrow\, \rho_P(X_n,X)={\mathbb E}\Bigl( \frac{|X_n-X|}{1+|X_n-X|}\Bigr)
\xrightarrow{} 0 , 
$$
то есть сходимость по вероятности метризуема.
\end{problem}

\begin{remark} (См. А.Н. Ширяев [\ref{chiraiev}], Т. 1, а также главу 6 из \cite{Gupta}, и \cite{stoianov})
Выполнена импликация 
$$
X_n\xrightarrow{\text{ п.н. }}X  \, \Rightarrow\, X_n\xrightarrow{L_1}X , 
$$
т.е. возможен предельный переход под знаком математического ожидания, если семейство с.в. $\{ X_n\}$ является равномерно интегрируемым: 
$$
\sup\limits_n {\mathbb E}\bigl[ |X_n|\cdot {\mathbb I}_{\{ |X_n|>c\}} \bigr] \rightarrow 0, \quad c \to +\infty; 
$$

Отметим также, что сходимость по распределению также как и по вероятности метризуемы в отличие от сходимости п.н. (см. задачу \ref{limpnnero} из  раздела \ref{standart}). 

\end{remark}

\begin{problem}\Star(Теорема Колмогорова о трех рядах) 
Пусть $\left\{ {\xi _n } 
\right\}_{n\in {\rm N}} $ -- последовательность независимых с.в. 
Покажите, что для сходимости ряда $\sum\limits_{n=1}^\infty {\xi _n } $ с 
вероятностью 1 необходимо, чтобы для любого $c>0$ сходились ряды 
$\sum\limits_{n=1}^\infty {\Exp\xi _n^c } $, $\sum\limits_{n=1}^\infty {\Var\xi 
_n^c } $, $\sum\limits_{n=1}^\infty {\PR\left( {\left| {\xi _n } \right|\ge c} 
\right)} $, где $\xi _n^c =\xi _n I\left( {\xi _n \le c} \right)$, и 
достаточно, чтобы эти ряды сходились при некотором $c>0$.
\end{problem}

\begin{problem}(Интеграл Лебега--Стилтьеса)
Для неотрицательной случайной величины $\xi$ определим ее математическое $\Exp \xi$ ожидание посредством интеграла Лебега $\int \xi(\omega) d\PR(\omega)$, определяемого как:
\[
\lim \limits_{n \to \infty} \left(
  \sum \limits_{k = 1}^{n 2^n} \frac{k-1}{2^n} \PR\left(  \frac{k-1}{2^n} \leq \xi <  \frac{k}{2^n}  \right) + n \PR(\xi \geq n). 
\right)
\] 

\noindent Пусть $\xi: \Omega \to \mathbb{R}$ -- произвольная случайная величина и $\xi^{+} = \max (\xi, 0)$, $\xi^{-} = - \min (\xi, 0)$. Тогда при условии $\Exp \xi^{+} < \infty$ или $\Exp \xi^{-} < \infty$ оператор $\Exp$ определяется как 
$\Exp \xi = \Exp \xi^{+} - \Exp \xi^{-}$. 

Докажите следующие теоремы и вытекающие из них свойства:

\begin{enumerate}
\item Для случайной величины $\xi \geq 0$ существует последовательность \textit{простых} (индикатор измеримого множества) случайных величин, таких что $\xi_{n+1} \geq \xi_n$ и $\xi_n \mathop{\longrightarrow} \limits^{\text{п.н.}} \xi$. Следствие: $\forall \xi \geq 0 \;\; \exists \; \Exp \xi$.
\item Если  $\xi_{n+1} \geq \xi_n \geq 0$ и $\xi_n \mathop{\longrightarrow} \limits^{\text{п.н.}} \xi$, то $\Exp \xi_n \to \Exp \xi$. 
\begin{enumerate}
\item[Следствие 1:] Путь $\xi_n \geq 0$, тогда 
\[ \Exp\left(\sum \limits_{n=1}^{\infty} \xi_n \right) = \sum \limits_{n=1}^{\infty} \Exp \xi_n. \]
\item[Следствие 2:] Пусть $\Exp \vert \xi \vert < \infty$ и $\PR(A) \to 0$, тогда $\Exp (\vert \xi \vert I_A) \to 0$. 
\end{enumerate}
\item Путь $\xi_n \geq 0$, тогда $\Exp (\lim \limits_{n\to\infty} \inf \xi_n) \leq \lim \limits_{n\to\infty} \inf \Exp \xi_n$. 
\item Путь $\vert \xi_n \vert \leq Y$ и $\Exp Y < \infty$, тогда $\Exp (\lim \limits_{n\to\infty} \inf \xi_n) \leq \lim \limits_{n\to\infty} \inf \Exp \xi_n  \leq \lim \limits_{n\to\infty} \sup \Exp \xi_n  \leq \Exp (\lim \limits_{n\to\infty} \sup \xi_n)$.
\begin{enumerate}
\item[Следствие 1:] Путь $\xi_n \geq 0$, $\vert \xi_n \vert \leq Y$, $\Exp Y < \infty$ и $\xi_n \mathop{\longrightarrow} \limits^{\text{п.н.}} \xi$, тогда $\Exp \vert \xi \vert < \infty$, $\Exp \xi_n \to \Exp \xi$ и $\Exp \vert \xi_n - \xi \vert \to 0$.
\item[Следствие 2:] Путь $\vert \xi_n \vert \leq Y$, $\Exp Y^p < \infty$, $p >0$ и $\xi_n \mathop{\longrightarrow} \limits^{\text{п.н.}} \xi$, тогда $\Exp \vert \xi \vert^p < \infty$ и $\Exp \vert \xi_n - \xi \vert^p \to 0$. 
\end{enumerate}
\end{enumerate} 

\end{problem}

\begin{remark}
Положим $g(x): \mathbb{R} \to \Upsilon$ -- борелевская функция ($\forall A \in \mathcal{B}(\Upsilon): \;  g^{-1}(A) \in \mathcal{B}(\mathbb{R}) $). Если существует один из интегралов 
\[
\int \limits_{A} g(x) dF_{\xi}(x) \quad \text{или} \quad \int \limits_{\xi^{-1}(A)} g(\xi(\omega)) d\PR(\omega),
\]
то существует и другой, и они совпадают. Интеграл  $\int \limits_{\mathbb{R}} g(x) dF_{\xi}(x)$ называется интегралом \textit{Лебега--Стилтьеса}.\\
\indent Также стоит заметить, что в формулировке предлагается разложить случайную величину на положительную $\xi^+$ и отрицательную $\xi^-$ части. По сути это эквивалентно разложению меры вероятностной меры $\PR$ на аналогичные части. На первый взгляд неочевидно, почему такое разложение существует и каждая из компонент является вероятностной сигма-мерой. На этот счет есть известная теорема о разложении меры -- т.Хана (Hahn Decomposition), с которой можно ознакомиться например в книге Богачева В.И. - Основы Теории Меры т.2.
\end{remark}




\begin{problem}
Симметричную монету независимо бросили $n$ раз. Результат бросания записали в виде последовательности нулей и единиц. 
\begin{enumerate}
\item Покажите, что с вероятностью стремящейся к единице при $n\to \infty $ длина максимальной подпоследовательности  из подряд идущих единиц $l_n$ лежит в промежутке $(\log_2 \sqrt{n} ,\; \log_2 n^{2} )$; 

\item\Star Покажите, что серия из гербов длины $\log_{2} n$ наблюдается с вероятностью, стремящейся к единице при $n\to\infty$;

\item\DStar Верно ли, что  $\frac{ l_n}{ \log_2 n } \overset{\text{п.н.}}{\longrightarrow} 1$?
\end{enumerate}
\end{problem}

 \begin{ordre}
 В пункте а) для нахождения нижней оценки разобьем всю последовательность бросаний на участки длиной $\log \sqrt{n}$. Оцените вероятность того, что хотя бы один из участков состоит полностью из единиц.    
 \end{ordre}

\begin{remark}
 См. Erdos P., Renyi A. On a new law of large numbers // Journal Analyse Mathematique. 1970. V. 23. P. 103--111.
 
 \verb| http://www.renyi.hu/|$\sim$\verb|p_erdos/1970-07.pdf.| 
\end{remark}


\begin{problem}(Закон 0 и 1)
Пусть $(\Omega,\Xi,{\mathbb P})$ --- вероятностное пространство, $\xi_1,\xi_2,\ldots$ --- некоторая последовательность независимых с.в. 
Обозначим $\Xi_n^{\infty}=\sigma(\xi_{n},\xi_{n+1},\ldots)$ --- $\sigma$-алгебру, порожденную с.в. $\xi_{n},\xi_{n+1},\ldots$ и пусть 
$$
{\mathcal X}=\bigcap\limits_{n=1}^{\infty} \Xi_{n}^{\infty} . 
$$
Поскольку пересечение $\sigma$-алгебр есть снова $\sigma$-алгебра, то ${\mathcal X}$ --- есть $\sigma$-алгебра. Эту $\sigma$-алгебру 
будем называть <<хвостовой>> или <<остаточной>>, в связи с тем, что всякое событие $A\in{\mathcal X}$ не зависит от значений с.в. 
$\xi_1,\xi_2,\ldots,\xi_n$ при любом конечном $n$, а определяется лишь <<поведением бесконечно далеких значений последовательности 
$\xi_1,\xi_2,\ldots$ >>. \\
\noindent С помощью задачи $\ref{SigmaAlgebra}$ раздела \ref{standart} докажите справедливость следующего утверждения: \\

Пусть $\xi_1,\xi_2,\ldots$ --- последовательность независимых в совокупности с.в. и $A\in{\mathcal X}$ 
(событие $A$ принадлежит <<хвостовой>> $\sigma$-алгебре). Тогда ${\mathbb P}(A)$ может принимать лишь два значения $0$ или $1$. 
\end{problem}

\begin{ordre}
Идея доказательства состоит в том, чтобы показать, что каждое <<хвостовое>> событие $A$ не зависит от самого себя и, значит, 
${\mathbb P}(A\cap A)={\mathbb P}(A)\cdot {\mathbb P}(A)$, т.е. ${\mathbb P}(A)={\mathbb P}^2(A)$, откуда 
${\mathbb P}(A)=0$ или $1$. 

Полученный результат, в частности, означает, что $\sum  \xi_k$ сходится или расходится с вероятностью 1.  

Существует также обобщение последнего правила -- закон Хьюитта и Сэвиджа (см. А.Н. Ширяев Т.2 \cite{21}), где под ${\mathcal X}$ подразумевается $\sigma$-алгебра перестановочных событий:
\[
\mathcal{X} = \{ \xi^{-1}(B), \; B \in \mathcal{B}(\mathbb{R}^{\infty}): \;
\PR (\xi^{-1}(B) \triangle (\pi\xi)^{-1}(B)) = 0, \; \forall \pi \}, 
\] 
где $\pi$ -- перестановка конечного числа элементов.
\end{ordre}

\begin{problem}
Сто паровозов выехали из города по однополосной линии, каждый с постоянной скоростью. Когда движение установилось, то из-за того, что быстрые догнали идущих впереди более медленных, образовались караваны (группы, движущиеся со скоростью лидера). Найдите математическое ожидание и дисперсию числа караванов. Скорости различных паровозов независимы и одинаково распределены, а функция распределения скорости непрерывна.
\end{problem}

\begin{problem}
Согласно законам о трудоустройстве в городе \textit{М}, наниматели обязаны предоставить всем рабочим выходной, если хотя бы у одного из них день рождения, и принимать на службу рабочих независимо от их дня рождения. За исключением этих выходных рабочие трудятся весь год из 365 дней. Предприниматели хотят максимизировать среднее число человеко-дней в году. Сколько рабочих трудятся на фабрике в городе \textit{М}?

\end{problem}

\begin{problem}
В каждую $i$-ю единицу времени живая клетка получает случайную дозу облучения $X_i$, причем $\{ X_i\}_{i=1}^{t}$ имеют 
одинаковую функцию распределения $F_X(x)$ и независимы в совокупности $\forall t \in \mathbb{N}$. Получив интегральную дозу облучения, 
равную $\nu$ ($\nu \gg \Exp X_1$), клетка погибает. Обозначим через $T$ -- случайную величину, равную времени жизни клетки.
Оценить среднее время жизни клетки ${\mathbb E}T$. 
\end{problem}

\begin{ordre}

Докажите \textit{тождество Вальда}: 
$$
{\mathbb E}S_T={\mathbb E}X_1\cdot {\mathbb E}T, \; \text{где} \; S_T = \sum_{i=1}^{T}X_i,    
$$

\noindent введя вспомогательную случайную величину

$$
Y_j=\begin{cases}
1, &\text{ если }\quad X_1+\ldots +X_{j-1}=S_{j-1}<\nu, \\
0, &\text{ в остальных случаях }, 
\end{cases}
$$
 
\noindent и записав $S_{T}$ в следующем виде: $S_T = \sum_{i=1}^T Y_i X_i$. 

\end{ordre}

\begin{problem}\Star(Теорема Дуба)
\label{sec:doob}
Пусть $Y_0,\dots,Y_n$ -- последовательность случайных величин, являющаяся мартингалом (см. замечание).
Показать, что для  \textit{момента остановки}
$\tau = \inf \{k\leq n: Y_k\geq \lambda \}$  ($\tau = n$, если $\max_{k\leq n}Y_k <\lambda$), верно
\begin{equation*}
\mathbb{E}{Y_{\tau}} = \mathbb{E}Y_n.
\end{equation*}
\end{problem}
\begin{remark}
Определение мартингала (см. А.Н. Ширяев [\ref{chiraiev}] Т.2). Пусть задано вероятностное пространство $(\Omega,\mathcal{F},\mathbb{P})$  с семейством ($\mathcal{F}_n$) $\sigma$-алгебр $\mathcal{F}_n$, $n\geq 0$ таких, что $\mathcal{F}_0\subseteq\dots\subseteq\mathcal{F}_n\subseteq \mathcal{F}$. 
Пусть $Y_0,\dots,Y_n$ ~--- последовательность случайных величин, заданных на $(\Omega,\mathcal{F},\mathbb{P})$, для каждого $n\geq 0 $ величины $Y_n$ являются $\mathcal{F}_n$-измеримыми. Мартингалом назвается  такая  последовательность $Y = (Y_n,\mathcal{F}_n)$, что 
\begin{equation*}
\mathbb{E}|Y_n|\leq\infty,
\end{equation*}
\begin{equation*}
\mathbb{E}(Y_{n+1}|\mathcal{F}_n)=Y_n \quad \text{почти наверное}.
\end{equation*}

Для доказательства теоремы Дуба необходимо показать, что 
%\begin{equation*}
%\mathbb{E}(Y_{\tau}|\mathcal{F}_n) = Y_n,
%\end{equation*}
%\end{remark}
%то есть
 для любого $A\in \mathcal{F}_n$ 
\begin{equation*}
\int_{A} Y_{\tau} d\mathbb{P} = \int_{A}Y_{n}d\mathbb{P}.
\end{equation*}
\end{remark}

\begin{problem}[Мартингалы и теорема о баллотировке]
Пусть $S_n =\xi _1 +....+\xi _n $, где с.в. $\left\{ {\xi _k } \right\}_{k\in {\mathbb N}} $ -- 
независимы и одинаково распределены \[\xi _k =\left\{ {\begin{array}{l}
 1,\quad p=1 \mathord{\left/ {\vphantom {1 2}} \right. 
\kern-\nulldelimiterspace} 2, \\ 
 -1,\;\,p=1 \mathord{\left/ {\vphantom {1 2}} \right. 
\kern-\nulldelimiterspace} 2. \\ 
 \end{array}} \right.\] Покажите, что тогда

$\PR\left( {S_1 >0,...,S_n >0\left| {S_n =a-b} \right.} 
\right)=\frac{a-b}{a+b},$ где $a>b$ и $a+b=n$.
\end{problem}

\begin{remark}
Проинтерпретируем этот результат: $\xi_k =1$ будем 
интерпретировать как голос, поданный на выборах за кандидата $A$, 
$\xi _k =-1$ -- за кандидата $B$. Тогда $S_n $ есть разность числа 
голосов, поданных за кандидатов $A$ и $B$, если в голосовании 
приняло участие $n$ избирателей, а $\PR\left( {S_1 >0,...,S_n >0\left| {S_n 
=a-b} \right.} \right)$ есть вероятность того, что кандидат $A$ все 
время был впереди кандидата $B$, при условии, что $A$ в общей 
сложности собрал $a$ голосов, а $B$ собрал $b$ голосов и $a>b$, 
$a+b=n$. 
\end{remark}

\begin{problem}(Закон арксинуса см. А.Н. Ширяев [\ref{chiraiev}] Т.2)
В условиях замечания к предыдущей задаче 
найдите при 20 бросаниях с какой вероятностью один из игроков никогда не будет впереди,  будет впереди не более одного раза.
\end{problem}

\begin{remark} 
Воспользуйтесь \textit{законом арксинуса} 

$\PR\left( {k_n <xn} \right)\simeq 
\frac{2}{\pi }\arcsin \sqrt x $, где с.в. $k_n =\left| {\left\{ 
{k=1,...,n:\;\;S_k \ge 0} \right\}} \right|$.
\end{remark}

\begin{problem}(Случайные блуждания)
Заблудившийся грибник оказался в центре леса имеющего форму круга с радиусом 5 км. Грибник движется по следующим правилам: пройдя 100 метров в случайно выбранном направлении север-юг-запад-восток (на это у грибника уходит 2 минуты), он решает снова случайно выбрать направление движения и т.д. Оцените математическое ожидание времени, через которое грибник выйдет из леса. Как изменится ответ, если выбор направления движения грибника осуществляется равновероятно на $[0,2\pi]$? 
\end{problem}

\begin{problem} 
Покажите, что последовательность дискретных с.в. $\left\{ 
{\xi _n } \right\}_{n\in {\mathbb N}} $, принимающих значения в ${\mathbb N}$ 
сходится по распределению к дискретной с.в. $\xi $ тогда и только тогда, 
когда при $n\to \infty $ для любого $k\in {\mathbb N}$ $\PR\left\{ {\xi _n =k} 
\right\}\to \PR\left\{ {\xi =k} \right\}$.
\end{problem}

\begin{problem}(Сходимость по моментам) 
Пусть $\left\{ {F_n \left( x \right)} \right\}_{n\in {\mathbb N}}$ -- последовательность функций  распределения, 
имеющих все моменты 
\[M_{n,k} =\int {x^kdF_n \left( x \right)} <\infty. \] 
Пусть для всех $k\in {\mathbb N}$ имеют место следующие сходимости: $M_{n,k} \to 
M_k \ne \pm \infty $. Тогда (по т.Хелли) существуют такая подпоследовательность $\left\{ 
{F_{n_m } \left( x \right)} \right\}_{m\in {\mathbb N}} $ и функция 
распределения $F\left( x \right)$ (с моментами $\left\{ {M_k } 
\right\}_{k\in {\mathbb N}} )$, что $F_{n_m } \left( x \right)\to F\left( x 
\right)$ в точках непрерывности $F\left( x \right)$.\\

\begin{enumerate}
\item Покажите, что если моменты $\left\{ {M_k } \right\}_{k\in {\mathbb N}} $ 
однозначно определяют функцию $F\left( x \right)$ (достаточным условием для 
этого будет существование такого $\chi >0$, что 
$ M_k^{1/k} / k \leq \chi $, $k\in {\mathbb N})$, то в качестве подпоследовательности можно брать 
саму последовательность.

\item  Проинтерпретируйте полученный результат с точки зрения сходимости по 
распределению соответствующих с.в.
\end{enumerate}
\end{problem}
\begin{remark}
См., например, Сачков В.Н. Вероятностные методы в комбинаторном анализе. М.: Наука, 1978.
\end{remark}

\begin{problem}

Покажите, что все моменты распределения
$$p_{\lambda } \left(x\right)=\frac{1}{24} e^{-x^{{1\mathord{\left/ {\vphantom {1 4}} \right. \kern-\nulldelimiterspace} 4} } } \left(1-\lambda \sin x^{{1\mathord{\left/ {\vphantom {1 4}} \right. \kern-\nulldelimiterspace} 4} } \right),\; x\ge 0$$
при любом значении параметра $\lambda \in \left[0,1\right]$ совпадают.

\begin{remark}

Достаточное \textit{условие Карлемана} того, что моменты однозначно определяют распределение случайной величины $\xi$, имеет вид (см. \cite{stoianov}):
$$
\sum _{n=0}^{\infty }\left(\Exp|\xi|^{2n} \right)^{{-1\mathord{\left/ {\vphantom {-1 \left(2n\right)}} \right. \kern-\nulldelimiterspace} \left(2n\right)} } =\infty.
$$

\end{remark}

\end{problem} 

\begin{problem}
Приведите примеры таких случайных величин $X$, $Y$ и $Z$, что вероятностные распределения сумм $X+Y$ и $X+Z$ совпадают, но распределения случайных величин различны.
\end{problem}
\begin{ordre}
Удобнее сначала подобрать соответствующие характеристические функции.
\end{ordre}
\begin{remark}
См. книгу Г. Секея [\ref{sekei}]. Эту же книгу можно рекомендовать и по двум следующим задачам.
\end{remark}

\begin{problem}
Приведите примеры независимых одинаково распределенных случайных величин $X$ и $Y$, для которых распределение суммы  $X+Y$ не однозначно определяет распределение $X$ и $Y$.
\end{problem}


\begin{problem}
Пусть 
  $X$ стохастически меньше, чем $Y$, т. е. 
 $$\forall t \; F_X(t) \leq F_Y(t), \exists t_0: \; F_X(t_0) < F_Y(t_0).$$
 Парадоксально, но может так случиться, что 
  $\PR(X > Y) \geq 0.99$. Приведите пример таких с.в. $X$ и $Y$.
\end{problem}

\begin{problem}(Парадокс транзитивности \cite{2013}, \cite{book12})
Будем говорить, что случайная величина $X$ больше по вероятности случайной величины $Y$, если ${\mathbb P}(X>Y)>{\mathbb P}(X\le Y)$. 
Пусть известно, что для случайных величин $X$, $Y$, $Z$ выполнена следующая цепочка равенств: 
$$
{\mathbb P}(X>Y)={\mathbb P}(Y>Z)=\alpha>\frac{1}{2} . 
$$
Верно ли, что $X$ больше по вероятности $Z$ и почему? 
\end{problem}




\begin{problem}

Требуется определить, начиная с какого этажа брошенный с балкона 100-этажного здания стеклянный шар разбивается. В наличии имеется два таких шара. Предложить метод нахождения граничного этажа, минимизирующий математическое ожидание числа бросков. Рассмотреть случай большего числа шаров.  

\end{problem}

\begin{problem}\Star
На подоконнике лежит $N$ помидоров. Вечером $i$-го дня ($1 \leqslant i \leqslant N$) портится один помидор. Каждое утро человек съедает один (случайно выбранный) свежий помидор из оставшихся. Таким образом, каждый помидор либо испортился, либо был съеден.

\begin{enumerate}
\item Получите рекуррентную формулу для математического ожидания количества съеденных помидоров от числа $N$.
\item Найдите асимптотическую оценку количества съеденных помидоров при $N \rightarrow \infty$.
\end{enumerate}

\end{problem}

\begin{problem}
\begin{enumerate}
\item Имеется монетка (несимметричная). Несимметричность монетки заключается в том, что либо орел выпадает в два раза чаще решки; 
либо наоборот (априорно, до проведения опытов, оба варианта считаются равновероятными). Монетку бросили $10$ раз. Орел выпал $7$ раз. 
Определите апостериорную вероятность того, что орел выпадает в два раза чаще решки (апостериорная вероятность считается с учетом 
проведенных опытов; иначе говоря, это просто условная вероятность). 

\item Определите апостериорную вероятность того, что орел выпадает не менее чем в два раза чаще решки. Если несимметричность 
монетки заключается в том, что либо орел выпадает не менее чем в два раза чаще решки; либо наоборот (априорно оба варианта считаются 
равновероятными). 
\end{enumerate}
\end{problem}





\begin{problem}(Распределение Гумбеля или двойное экспоненциальное распределение)
\label{gumbel}
Пусть $\varsigma _{1} ,...,\varsigma _{n} $ -- независимые одинаково распределенные случайные величины, и существуют такие константы $\alpha ,T>0$, что
\[\mathop{\lim }\limits_{y\to \infty } e^{{y\mathord{\left/ {\vphantom {y T}} \right. \kern-\nulldelimiterspace} T} } \left[1-\PR\left(\varsigma _{k} <y\right)\right]=\alpha. \] 
Покажите, что
\[
\max \left\{\varsigma _{1},\ldots,\varsigma _{n} \right\}-T\ln \left(\alpha n\right)\xrightarrow[{n\to \infty }]{d} \xi,  
\]
\noindent где $\PR\left(\xi <\tau \right)=\exp \left\{-e^{-{\tau \mathord{\left/ {\vphantom {\tau  T}} \right. \kern-\nulldelimiterspace} T} } \right\}.$

\end{problem}


\begin{problem}\Star(Logit-распределение или распределение Гиббса)
\label{gibbs}
Пусть известно, что $\xi_{1} ,\ldots,\xi_{n}$~--- независимые одинаково распределенные по \textit{закону Гумбеля} случайные величины. Пусть  $x_{k} =C_{k} +\xi _{k} ,$ $k=1,...,n$, где $C_{k} $ -- не случайные величины. Положим $q_n=\arg \mathop{\max }\limits_{k=1,\ldots,n} \left\{C_{k} +\xi _{k} \right\}$. Тогда с.в. $q_n$ имеет распределение:
\[\PR\left(q_n=k\right)=\frac{e^{{C_{k} \mathord{\left/ {\vphantom {C_{k}  T}} \right. \kern-\nulldelimiterspace} T} } }{\sum _{l=1}^{n}e^{{C_{l} \mathord{\left/ {\vphantom {C_{l}  T}} \right. \kern-\nulldelimiterspace} T} }  } , \quad k=1,\ldots,n.\] 
Докажите, что результат останется верным, если $n \to \infty$, a также $\xi_k$ -- независимые одинаково распределенные, причем 
\[
\mathop{\lim }\limits_{y\to \infty } e^{{y\mathord{\left/ {\vphantom {y T}} \right. \kern-\nulldelimiterspace} T} } \left[1-\PR\left(\xi _{k} <y\right)\right]=\alpha \geq 0. 
\]
\end{problem}
\begin{remark}
См. монографию Andersen S.P., de Palma A., Thisse J.-F. Discrete choice theory of product differentiation. MIT Press, Cambridge, 1992; см. также \cite{222}.
\end{remark}


\begin{problem}(Рекорды)
Пусть $X_1 ,X_2 ,\ldots $ -- независимые 
случайные величины с одной и той же плотностью распределения вероятностей 
$p(x)$. Будем говорить, что наблюдается рекордное значение в момент времени 
$n>1$, если $X_n >\max \left[ {X_1 ,...,X_{n-1} } \right]$. Докажите 
следующие утверждения.

\begin{enumerate}
\item Вероятность того, что рекорд зафиксирован в момент времени $n$, 
равна $1/n$;

\item Математическое ожидание числа рекордов до момента времени $n$ 
равно 
\[
\sum\limits_{1<k\le n} {\frac{1}{k}} \sim \ln n;
\]

\item Пусть $Y_n $ --- случайная величина, принимающая значение $1$, если 
в момент времени $n$ зафиксирован рекорд, и значение $0$ -- в противном случае. 
Тогда случайные величины $Y_1 ,Y_2 ,\ldots$ независимы в совокупности;

\item Дисперсия числа рекордов до момента времени $n$ равна
\[
\sum\limits_{1<k\le n} {\frac{k-1}{k^2}} \sim \ln n;
\]

\item Если $T$ -- момент появления первого рекорда после момента времени $1$, то $\Exp T= \infty$.
\end{enumerate}
\end{problem}

\begin{ordre}
См. \cite{4}.
\end{ordre}

\begin{remark}
Приведем для справки следующую теорему. Пусть $\eta_0,\eta_1,\dots$ ~--- последовательность независимых случайных величин с одной и той же непрерывной функцией распределения. Для каждого $n\in \mathbb{Z}_{+}$ по случайным величинам $\eta_0,\eta_1,\dots,\eta_n$ построим вариационный ряд 
$$\eta_{0,n}\leq \eta_{1,n}\leq\dots\leq\eta_{n,n}.$$
Рекордные моменты $\{\nu(n),n\in\mathbb{Z}_{+}\}$ определяются следующим образом: $\nu(0) = 0$ и
$$\nu(n+1)=\min\{j>\nu(n): \eta_j>\eta_{j-1,j-1}\},\quad n\in \mathbb{Z}_{+}.$$
Верен следующий результат (см. Якымив А.Л. Вероятностные приложения тауберовых теорем МАИК Наука, 2005, с. 256):
$$\mathbb{P}\{\nu(n)>t\}\equiv \frac{t^{-1}\ln^{n-1}(t)}{(n-1)!}.$$

\medskip

Для доказательства приведенной теоремы существенно используются тауберовы теоремы.
Тауберовыми теоремами называют теоремы, выводящие из асимптотических свойств производящих функций и преобразований Лапласа функций и последовательностей  (а также других интегральных преобразований) асимптотики этих функиций и последовательностей (то есть эти теоремы обратные к абелевым). 
\end{remark}

\begin{problem}
Зрелый индивидуум производит потомков согласно производящей функции $h(s)$.
Предположим, что все начинается с $k$ индивидуумов, достигших зрелости. Каждый потомок достигает зрелости с вероятностью $p$. Докажите, что производящая функция числа зрелых индивидуумов в следующем поколении равна $H_k(s) = [h(ps + 1 - p)]^{k}$.   
\end{problem}

\begin{problem}(Распределение Коши)
Радиоактивный источник испускает 
частицы в случайном направлении (при этом все направления равновероятны). 
Источник находится на расстоянии $d$ от фотопластины, которая представляет 
собой бесконечную вертикальную плоскость.

\begin{enumerate}
\item При условии, что частица попадает в плоскость, покажите, что 
горизонтальная координата точки попадания (если начало координат выбирается 
в точке, ближайшей к источнику) имеет плотность распределения:
\[
p\left( x \right)=\frac{d}{\pi \left( {d^2+x^2} \right)}.
\]
Это распределение известно как \textit{распределение Коши}.

\item Можно ли вычислить среднее (математическое ожидание) этого 
распределения?

\item Предположим, что параметр $d$ неизвестен, но есть $n \gg 1$ независимых реализаций рассматриваемой с.в. $X_1, \ldots, X_n$. Предложите способ оценивания $d$. То есть необходимо указать функцию от выборки (статистику) $\widehat d_n=\widehat d_n(X^n)$,
значение которой будет рассматриваться в качестве приближения к неизвестному истинному значению~d. 
\begin{remark}
К такого вида оценке, как правило, предъявляются следующие требования:
\begin{enumerate}
\item Состоятельность: $\widehat d_n \rightarrow d$ почти наверное или по вероятности;
\item Несмещенность: $\Exp(\widehat d_n) = d$.
\end{enumerate}

Для сравнения между собой различных оценок одного и того же параметра выбирают некоторую \textit{функцию риска}, которая измеряет отклонение оценки от истинного значения параметра.
Чаще всего в качестве функции риска рассматривают дисперсию статистики. 

См. Косарев Е.Л. Методы обработки экспериментальных данных. --М.: Физматлит, 2008.
\end{remark}

\end{enumerate}
\end{problem}

\begin{problem}
\label{condExp1}
Предположим, что с.в. $X\in L_2$, это означает ${\mathbb E}(X^2)<\infty$. Докажите, что 
\begin{equation*}
\label{UMO}
\| X-{\mathbb E}(X|Y_1,\ldots,Y_n)\|_{2}=\min\limits_{\varphi\in H} \| X-\varphi(Y_1,\ldots,Y_n)\|_{2} , 
\end{equation*}
где $H$ --- подпространство пространства $L_2$ всевозможных борелевских функций $\varphi(Y_1,\ldots,Y_n)\in L_2$; 
${\mathbb E}(X|Y_1,\ldots,Y_n)$ --- условное математическое ожидание с.в. $X$ относительно $\sigma$-алгебры, порожденной с.в. 
$Y_1,\ldots,Y_n$, часто говорят просто относительно с.в. $Y_1,\ldots,Y_n$; 
$$
\| X\|_{2}=\sqrt{\langle X,X\rangle}=\sqrt{{\mathbb E}(X\cdot X)}=\sqrt{{\mathbb E}(X^2)} . 
$$
Обобщите утверждение задачи на случай, когда $X$ -- случайный вектор.
\end{problem}


\begin{ordre}
Покажите, что $(X-{\mathbb E}(X|H)) \bot \xi,\; \forall\xi\in H$, т.е. ${\mathbb E}(\cdot|H)$ 
является проектором на подпространство $H$ в $L_2$. Детали см., например, в учебнике Розанов Ю.А. Теория вероятностей. Случайные процессы. Математическая статистика. М.: Наука, 1985.
\end{ordre}

\begin{remark}
Данную задачу и последующие следует сравнить с \ref{scalar_prod} и \ref{scalar_prod_1} 
из раздела \ref{standart}.
\end{remark}

\begin{problem}
Используя соотношение из задачи \ref{condExp1}  в качестве определения условного математического ожидания, докажите справедливость основных свойств математического ожидания, в частности
\[
\Exp(\Exp(X|Y)) = \Exp(X). 
\]
\end{problem}

\begin{ordre}
$\langle 1, X \rangle = \langle 1, \Exp(X|Y) \rangle$.
\end{ordre}


\begin{problem}
\label{condExp3}
Докажите, что если в условиях предыдущей задачи вектор $(X,Y_1,\ldots,Y_n)^T$ --- является нормальным случайным вектором (без ограничения 
общности можно также считать, что $(Y_1,\ldots,Y_n)^T$  --- невырожденный нормальный случайный вектор), то в качестве $H$ можно взять 
подпространство всевозможных линейных комбинаций с.в. $Y_1,\ldots,Y_n$. Т.е. можно более конкретно сказать, на каком именно 
классе борелевских функций достигается минимум в задаче $\ref{UMO}$. 
\end{problem}

\begin{ordre}
Будем искать 
${\mathbb E}(X|Y_1,\ldots,Y_n)$ в виде 
$$
\label{Gauss}
{\mathbb E}(X|Y_1,\ldots,Y_n)=c_1 Y_1+\ldots +c_n Y_n . 
$$
Докажите следующие утверждения:

\begin{enumerate}
\item $X-c_1 Y_1-\ldots-c_n Y_n, Y_1,\ldots, Y_n$ -- независимы.
\item $X-c_1 Y_1-\ldots-c_n Y_n$ ортогонален подпространству $H$ пространства $L_2$ всевозможных борелевских функций $\varphi(Y_1,\ldots,Y_n)\in L_2$.
\end{enumerate}
 
\end{ordre}


\begin{comment}
\begin{problem}
Некто обладает одной облигацией, которую намеревается продать в один из последующих четырех дней, в которых цена облигации 
принимает различные значения, априори неизвестные, но становящиеся известными в начале каждого дня продаж. Предполагается, что 
цены облигации независимы и их перестановки по торговым дням равновозможны. Какова стратегия продавца, состоящая в выборе дня 
продажи облигации и гарантирующая максимальную вероятность того, что он продаст облигацию в день ее наибольшей цены? 
\end{problem}

\begin{ordre}
Рассмотреть следующие возможные стратегии и сравнить вероятности продажи облигации в день наибольшей цены: 
\begin{enumerate}
\item на первом шаге (в первый день торгов) запомним имевшую место цену облигации, не продавая ее, а затем продадим 
облигацию в тот день, когда ее цена окажется большей цены, зафиксированной в первый день, или (когда такого дня не окажется) в 
последний (четвертый) день, независимо от цены этого дня (стратегия $S_1$); 

\item не продавая облигацию в первом и втором торговых днях, зафиксируем  максимальную цену из двух, имевших место для этих дней, 
и продадим облигацию в третьем торговом дне, если цена облигации в нем будет выше, чем указанная зафиксированная максимальная цена, 
или, в противном случае, в четвертом дне (стратегия $S_2$). 
\end{enumerate}
\end{ordre}
\end{comment}

\begin{problem}
Ведущий приносит два одинаковых конверта и говорит, что в них лежат деньги, причем в одном вдвое больше, чем в другом. Двое участников берут конверты и тайком друг от друга смотрят, сколько в них денег. Затем один говорит другому: «Махнемся не глядя?» (предлагая поменяться конвертами). Стоит ли соглашаться?
\end{problem}
\begin{remark}
См. А. Шень. Вероятность: примеры и задачи; Г. Секей \cite{book12} (эту книгу полезно посмотреть и в следующей задаче). В книге Г. Секея поднимаются в связи с аналогичной задачей вопросы о аксиоматике теории вероятностей. Отметим, что аксимоматика А.Н. Колмогорова, построеная на теории меры не единственный способ ввести случайные величины. Интересные материалы на эту тему имеются в статье D. Mumford'а "На заре эры стохастичности" в сборнике "Математика: граница и перспективы. Под ред. Д.В. Аносова и А.Н. Паршина. М. Фазис, 2005" (также отметим в этом сборнике статью W.T. Gowers'а и аналогичный сборник "Математические события XX века. М.: Фазис, 2003"). Также в этой связи можно отметить книжку Janes E.T. Probability theory. The logic of science. Cambridge University Press, 2003 и Кановей В.Г., Любецкий В.А. Современная теория множеств: абсолютно неразрешимые классические проблемы. М.: МЦНМО, 2013.  См. также задачу \ref{banah_tar}.
\end{remark}

\begin{problem}(Спящая красавица)
В воскресенье с красавицей обговаривается схема эксперимента, согласно которой в вечером в воскресенье красавица усыпляется. Далее подкидывается симметричная монетка. Если монетка выпадает орлом, то красавицу будят в понедельник (потом снова дают снотворное), затем будят еще раз во вторник (потом снова дают снотворное). Если решкой, то будят только в понедельник (потом снова дают снотворное). В среду красавицу пробуждают окончательно в любом случае. Снотворное стирает красавице память в том смысле, что она помнит правила, оговоренные с ней в воскресенье, но не помнит сколько раз уже ее будили и не знает какой сегодня день недели. Каждый раз когда красавицу будят ей предлагают оценить вероятность того, что монетка выпала решкой. Что может ответить красавица?
\end{problem}

\begin{problem}(Как играть в проигрышную игру \cite{book12}) 
Предположим, что в некоторой игре четное число розыгрышей $2n$. Игрок  $A$ в одном розыгрыше выигрывает с вероятностью 0.45, $B$ --  с вероятностью 0.55. Чтобы выиграть в игру, игрок должен набрать более половины всех очков. Если у  $A$ есть возможность выбрать  $2n$, то, как ни странно $2n=2$ не является лучшим выбором. Найдите оптимальное $2n$.    
\end{problem}

\begin{problem}(Задача о лабиринте, А.Н. Соболевский)
В лесах дремучих стоит дом не дом, чертог не чертог, а дворец зверя лесного, чуда морского, весь в огне, в серебре и золоте и каменьях самоцветных. Красная девица входит на широкий двор, в ворота растворенные, и находит там три двери, а за ними три горницы красоты несказанной, а в каждой из тех горниц еще по три двери, ведущие в горницы краше прежних.
Походив по горницам, красная девица начинает догадываться, что дворец построен ярусами: двери со двора ведут в горницы первого яруса, из тех -- в горницы второго яруса, и так далее. В каждой горнице есть вход и три выхода, ведущие в три горницы следующего яруса. В горницах последнего, $n$-го яруса растворены окна широкие во сады диковинные, плодовитые, а в садах птицы поют и цветы растут.
Вернувшись на широкий двор и отдохнув, красная девица видит, что произошла перемена: ворота, через которые она вошла, и часть дверей внутри дворца сами собой затворились,
да не просто так, а каждая дверь с вероятностью 1/3 независимо от других. Немного обеспокоенная, красная девица начинает метаться из горницы в горницу сквозь оставшиеся незатворенными двери в поисках выхода. Покажите, что при больших $n$ вероятность того, что
она сможет добраться до окон, растворенных в сады, близка к $(9 - \sqrt{27}) / 4 \approx 0.95$.
\end{problem}


 \begin{problem}(Задача М. Гарднера о разборчивой невесте)
 В аудитории находится невеста, которая хочет выбрать себе жениха. За дверью выстроилась очередь из $N$ женихов. Относительно любых 
 двух женихов невеста может сделать вывод, какой из них для неё предпочтительнее. Таким образом, невеста задает на множестве женихов 
 отношение порядка (естественно считать, что если $A$ предпочтительнее $B$, а $B$ предпочтительнее $C$, то $A$ предпочтительнее $C$). 
 Предположим, что все $N!$ вариантов очередей равновероятны и невеста об этом знает (равно, как и число $N$). Женихи запускаются 
 в аудиторию по очереди. Невеста видит каждого из них в первый раз! Если на каком-то женихе невеста остановится (сделает свой выбор), 
 то оставшаяся очередь расходится. Невеста хочет выбрать наилучшего жениха (исследуя $k$–го по очереди жениха, невеста лишь может 
 сравнить его со всеми предыдущими, которых она уже просмотрела и пропустила). Оцените (при $N\to\infty$) вероятность того, что невесте 
 удастся выбрать наилучшего жениха, если она придерживается следующей стратегии: просмотреть (пропустить) первых по очереди $[N/e]$ 
 кандидатов и затем выбрать первого кандидата, который лучше всех предыдущих (впрочем, такого кандидата может и не оказаться, тогда, 
 очевидно, невеста не смогла выбрать наилучшего жениха). 
 \end{problem}
 \begin{remark}
 Можно показать, что описанная стратегия будет асимптотически оптимальной. Популярное изложение имеется у С.М. Гусейн-Заде. Однако полезно обратить внимание, что эта задача является ярким примером целого направления: оптимальной остановки стохастических процессов, перетекающего в оптимальное управление процессами Маркова. Наиболее полезным для приложений (особенно в области "Исследование операций") во всем этом является распространение принципа динамического программирования Беллмана на стохастический случай ({\itпринцип Вальда--Беллмана}). Об этом можно прочитать, например, у Дынкина--Юшкевича, Ширяева, Аркина--Евстигнеева. Для введения можно посмотреть книги Е.С. Вентцель или Ю.А. Розанова \cite{6}, см. также задачу \ref{m_field_games} из раздела \ref{macrosystems}. Часть студентов ФУПМ также столкнется с этим в курсе стохастических дифференциальных уравнений. В таком контексте полезно будет посмотреть книгу Б. Оксендаля Стохастические дифференциальные уравнения: Введение в теорию и приложения -- М.: Мир, 2003. 
 \end{remark}
 
 \begin{problem}(Биржевой парадокс)
Рассмотрим любопытный экономический пример. Пусть имеется начальный капитал $K_1$, который требуется увеличить. Для этого имеются две возможности: вкладывать деньги в надежный банк и покупать на бирже акции некоторой компании. Пусть $u$ -- доля капитала, вкладываемая в банк, а $v$ -- доля капитала, расходуемая на приобретение акций ($0 \leq u + v \leq 1$). Предположим, что банк гарантирует $b \times 100 \%$ годовых, а акции приносят $X \times 100 \%$ годовых, где $X$ -- случайная величина с математическим ожиданием $m_X > b > 0$.  Таким образом, через год капитал составит величину $K_2 = K_1 (1 + b u + Xv)$. Очевидно, что, если придерживаться стратегии, максимизирующей средний доход за год, то выгодно присвоить следующие значения $u = 0$, $v = 1$. 

Рассмотрите прирост капитала $K_{t+1}$   за  $t$ лет, считая $X_1, \ldots,  X_t$ независимыми с.в.  Покажите, что при ежегодном вложении капитала в акции  
\[
\Exp(K_t) \rightarrow \infty, \; \text{при} \; t  \rightarrow \infty,
\]
\noindent но при этом, в случае $\Exp\left[ \log (1 + X) \right] < 0$     

\[
K_t \overset{\text{п.н.}}{\longrightarrow}  0, \; \text{при} \; t  \rightarrow \infty.
\]
Приведите пример такой случайной величины $X$.

\end{problem}

\begin{ordre}
Воспользуйтесь усиленным законом больших чисел, введя замену $K_{t+1} = K_1 e^{t Y_t}$, где  $Y_t = \frac{1}{t} \sum \limits_{i=1}^{t}\log(1+X_i)$. 
\end{ordre}

\begin{remark}
Рассмотренный парадокс хорошо иллюстрирует особенность поведения случайной последовательности, которая в отличие от детерминированной может сходиться в разных смыслах к разным значениям. 

Причина рассмотренного биржевого парадокса -- неудачный выбор критерия оптимальности. В качестве альтернативы могут быть выбраны следующие критерии:

\begin{enumerate}
\item \textit{Логарифмическая стратегия} или \textit{стратегия Келли}:
\[
Y_t \rightarrow \lambda(u) = \Exp\left[ \log (1 + b u + X (1-u)) \right],
\]
\[
\lambda(u) \rightarrow \max \limits_{ 0 \leq u \leq 1}
\]
\[
\Updownarrow
\]
\[
\Exp\left[ \log (K_{t+1}) \right] \rightarrow \max \limits_{ u_1 \ldots u_t }.
\]

\item Вероятностный критерий:
\[
\mathbb{P}_\phi (u_1 \ldots u_t ) = \mathbb{P} (K_{t+1} \geq -\phi) \rightarrow \max \limits_{ u_1, \ldots, u_t }.
\]
Используя метод \textit{динамического программирования} (см. D.P. Bertsekas. Dynamic Programming and Optimal Control, Vol. 1.  Athena Scientific, Belmont, Massachusetts, 1995), покажите, что оптимальное управление удовлетворяет следующему соотношению: 

\[
u_t(K_t) = \begin{cases}
\begin{array}{cc}
0, & \phi \geq - K_t(1+b), \\
1, & \phi < - K_t(1+b).
\end{array}\end{cases}
\]

\end{enumerate}

Подобный тип управления (стратегии) созвучен исторической практике накопления капитала: в эпоху первичного накопления люди зачастую серьезно рисковали ради денег, но как только они накапливали сумму, достаточную для безбедного существования при ее вложении хотя бы в банк, необходимость в риске для них отпадала. Более детальное рассмотрение задачи см. в Ю.С. Кан, А.И Кибзун. Задачи стохастического программирования с вероятностными критериями. --  М.: ``ФИЗМАТЛИТ'', 2009. 
\end{remark} 
 


\begin{problem}(Равновесие Нэша)
Один игрок прячет (зажимает в кулаке) одну или две монеты достоинством 10 рублей. Другой игрок должен отгадать, сколько денег у первого спрятано. Если отгадывает, то получает деньги, если нет -- платит 15 рублей. Каковы  должны быть стратегии игроков при многократном повторении игры?

\end{problem}

\begin{problem}
В аудитории находится 100 человек (игроков). Каждого просят написать целое число от 1 до 100. Победителем окажется тот участник, который написал число, наиболее близкое к 2/3 от среднего арифметического всех чисел. Требуется найти  \textit{наилучший ответ} при фиксированных стратегиях соперников: каждый соперник, просчитав на $X \sim \mathrm{Po}(2)$  хода  вперед, выбирает наугад (равномерно) число от 1 до $100 \cdot(2/3)^X$. К примеру, возможен следующий ход мыслей соперника: поскольку все догадались, что не стоит писать число большее $100 \cdot(2/3)$, то не стоит писать число большее $100 \cdot(2/3)^2$.
\end{problem}

\begin{remark}
Стратегия (действие игрока $i$ в зависимости от своего типа $t_i \in T_i$) $s_i^*: T_i, S_{-i} \rightarrow A_i$ называется \textit{наилучшим ответом} на заданные стратегии соперников $s_{-i}(t_{-i})$, если она является решением задачи максимизации ожидаемого выигрыша рассматриваемого игрока. При этом усреднение производится по всем неизвестным переменным, относящимся к соперникам (типам соперников $t_{-i}$): 
 \[
 \underset{t_{-i}}{\sum} u_i(s_i^*, s_{-i}(t_{-i}), t_i, t_{-i}) \cdot  \PR(t_{-i} | t_i) = \]\[ \underset{a_i \in A_i}{\max} \underset{t_{-i}}{\sum} u_i(a_i, s_{-i}(t_{-i}), t_i, t_{-i}) \cdot \PR(t_{-i} | t_i), 
 \]
 \noindent где $u_i$ -- выигрыш игрока $i$  при заданных действиях и типах всех игроков, $A_i$ -- множество действий игрока $i$, $\PR(t_{-i} | t_i)$ -- представление о типах остальных игроков при известном своем типе. 
 
Профиль стратегий $\{s_i^*\}$ -- есть множество \textit{равновесных} стратегий (\textit{равновесие по Нэшу}), где каждая стратегия является наилучшим ответом на фиксированные стратегии соперников. Поиск оптимальной стратегии зачастую подразумевает поиск равновесной стратегии. Равновесие по Нэшу является неподвижной точкой отображения $S_1 \times \ldots \times S_n \rightarrow S_1^* \times \ldots \times S_n^*$, где $S_i$ -- множество отображений $T_i \rightarrow A_i$, $S_i^*$ -- множество наилучших ответов $i$-го игрока, причем профиль стратегий $(s_1, \ldots, s_n)$ отображается во множество ответов каждого из игроков (имеет место многозначное отображение). Такая точка не всегда существует, но всегда существует ее аналог в случае, когда игрок может смешивать несколько стратегий (т.е. смешанной стратегией будет дискретное распределение $(p_1,\ldots,p_{|S_i|}),\; \PR(s_i = k) = p_k$). 

См. также книгу R.B. Myerson. Game Theory, Harvard University Press, 1997. 
\end{remark}

 \begin{comment}
\begin{problem}[сублинейный приближенный вероятностный алгоритм для 
матричных игр; Григориадис -- Хачиян, 1995]
Рассматривается симметричная 
антагонистическая игра двух лиц X и Y. Смешанные стратегии X и Y будем 
обозначать соответственно $\vec {x}$ и $\vec {y}$. При этом $x_k $ - 
вероятность того, что игрок X выберет стратегию с номером k, аналогично 
определяется $y_k $. Таким образом, $\vec {x},\vec {y}\in S=\left\{ {\vec 
{x}\in {\mathbb R}^n:\;\;\vec {e}^T\vec {x}=1,\;\vec {x}\ge \vec {0}} \right\}$, 
где $\vec {e}=\left( {1,...,1} \right)^T$. Выигрыш игрока X: $V_X \left( 
{\vec {x},\vec {y}} \right)=\vec {y}^TA\vec {x}$, а выигрыш игрока Y: $V_Y 
\left( {\vec {x},\vec {y}} \right)=-\vec {y}^TA\vec {x}$ (игра 
антагонистическая). Каждый игрок стремится максимизировать свой выигрыш, при 
заданном ходе оппонента. Равновесием Нэша (в смешанных стратегиях) 
называется такая пара стратегий $\left( {\vec {x}^\ast ,\;\vec {y}^\ast } 
\right)$, что
\[
\vec {x}^\ast \in \mbox{Arg}\mathop {\max }\limits_{\vec {x}\in S} \vec 
{y}^{\ast T} A\vec {x},
\quad
\vec {y}^\ast \in \mbox{Arg}\mathop {\min }\limits_{\vec {y}\in S} \vec 
{y}^TA\vec {x}^\ast .
\]
Ценой игры называют $\mathop {\max }\limits_{\vec {x}\in S} \mathop {\min 
}\limits_{\vec {y}\in S} \vec {y}^TA\vec {x}=\mathop {\min }\limits_{\vec 
{y}\in S} \mathop {\max }\limits_{\vec {x}\in S} \vec {y}^TA\vec {x}=\vec 
{y}^{\ast T}A\vec {x}^\ast $. Поскольку, по условию, игра также симметричная, 
то $A=-A^T$ - матрица$n\times n$. С помощью стандартной редукции можно 
свести к этому случаю общий случай произвольной матричной игры. В 
рассматриваемом же случае цена игры (выигрыш игроков в положении равновесия 
Нэша) есть 0, а множества оптимальных стратегий игроков совпадают. Требуется 
найти с точностью $\varepsilon >0$ положение равновесия Нэша (оптимальную 
стратегию), т.е. требуется найти такой вектор $\vec {x}$, что $A\vec {x}\le 
\varepsilon \vec {e}$, $\vec {x}\in S$. Покажите, считая элементы матрицы 
$A$ равномерно ограниченными, скажем, единицей, что приводимый ниже алгоритм 
находит с вероятностью не меньшей $1 \mathord{\left/ {\vphantom {1 2}} 
\right. \kern-\nulldelimiterspace} 2$ (вместо $1 \mathord{\left/ {\vphantom 
{1 2}} \right. \kern-\nulldelimiterspace} 2$ можно взять любое положительное 
число меньшее единицы) такой $\vec {x}$ за время ${\rm O}\left( {\varepsilon 
^{-2}n\log ^2n} \right)$, т.е. в определенном смысле даже не вся матрица (из 
$n^2$ элементов) просматривается. Отметим также, что в классе 
детерминированных алгоритмов, время работы растет с ростом $n$ не медленнее 
чем $\sim n^2$ (эта нижняя оценка получается из информационных соображений). 
Другими словами, никакой детерминированный алгоритм не может также 
асимптотически быстро находить приближенно равновесие Нэша. Точнее говоря, 
описанный ниже вероятностный алгоритм дает почти квадратичное ускорение по 
сравнению с детерминированными.

\underline {\textbf{Алгоритм}}

\begin{enumerate}
\item \textbf{Инициализация:} $\vec {x}=\vec {U}=\vec {0}$, $\vec {p}={\vec {e}} \mathord{\left/ {\vphantom {{\vec {e}} n}} \right. \kern-\nulldelimiterspace} n$, $t=0$.
\item \textbf{Повторить:}
\item \textbf{Счетчик итераций: }$t:=t+1$.
\item \textbf{Датчик случайных чисел:} выбираем $k\in \left\{ {1,...,n} \right\}$ с вероятностью $p_k $.
\item \textbf{Модификация }$\vec {X}$\textbf{: }$X_k :=X_k +1$.
\item \textbf{Модификация }$\vec {U}$\textbf{:} $U_i :=U_i +a_{ik} $, $i=1,...,n$.
\item \textbf{Модификация }$\vec {p}$\textbf{:} $p_i :={p_i \exp \left( {\varepsilon {a_{ik} } \mathord{\left/ {\vphantom {{a_{ik} } 2}} \right. \kern-\nulldelimiterspace} 2} \right)} \mathord{\left/ {\vphantom {{p_i \exp \left( {\varepsilon {a_{ik} } \mathord{\left/ {\vphantom {{a_{ik} } 2}} \right. \kern-\nulldelimiterspace} 2} \right)} {\left( {\sum\limits_{j=1}^n {p_j \exp \left( {\varepsilon {a_{jk} } \mathord{\left/ {\vphantom {{a_{jk} } 2}} \right. \kern-\nulldelimiterspace} 2} \right)} } \right)}}} \right. \kern-\nulldelimiterspace} {\left( {\sum\limits_{j=1}^n {p_j \exp \left( {\varepsilon {a_{jk} } \mathord{\left/ {\vphantom {{a_{jk} } 2}} \right. \kern-\nulldelimiterspace} 2} \right)} } \right)}$, $i=1,...,n$.
\item \textbf{Критерий останова:} если ${\vec {U}} \mathord{\left/ {\vphantom {{\vec {U}} t}} \right. \kern-\nulldelimiterspace} t\le 
\varepsilon \vec {e}$, то останавливаемся и печатаем ${\vec {x}=\vec {X}} \mathord{\left/ {\vphantom {{\vec {x}=\vec {X}} t}} \right. \kern-\nulldelimiterspace} t$.
\end{enumerate}
\textbf{Указание.} Покажите, что с вероятностью не меньшей, чем $1 
\mathord{\left/ {\vphantom {1 2}} \right. \kern-\nulldelimiterspace} 2$ 
алгоритм остановится через $t^\ast =4\varepsilon ^{-2}\ln n$ итераций. Для 
этого введите $P_i \left( t \right)=\exp \left( {{\varepsilon U_i \left( t 
\right)} \mathord{\left/ {\vphantom {{\varepsilon U_i \left( t \right)} 2}} 
\right. \kern-\nulldelimiterspace} 2} \right)$ и $\Phi \left( t 
\right)=\sum\limits_{i=1}^n {P_i \left( t \right)} $. Покажите, что

$M\left[ {\left. {\Phi \left( {t+1} \right)} \right|\vec {P}\left( t \right)} 
\right]=\Phi \left( t \right)\sum\limits_{i,k=1}^n {p_i \left( t \right)} 
p_k \left( t \right)\exp \left( {{\varepsilon a_{ik} } \mathord{\left/ 
{\vphantom {{\varepsilon a_{ik} } 2}} \right. \kern-\nulldelimiterspace} 2} 
\right)$ и $\exp \left( {{\varepsilon a_{ik} } \mathord{\left/ {\vphantom 
{{\varepsilon a_{ik} } 2}} \right. \kern-\nulldelimiterspace} 2} \right)\le 
1+{\varepsilon a_{ik} } \mathord{\left/ {\vphantom {{\varepsilon a_{ik} } 
2}} \right. \kern-\nulldelimiterspace} 2+{\varepsilon ^2} \mathord{\left/ 
{\vphantom {{\varepsilon ^2} 6}} \right. \kern-\nulldelimiterspace} 6$.

Используя это и кососимметричность матрицы $A$, покажите
\[
M\left[ {\Phi \left( {t+1} \right)} \right]\le M\left[ {\Phi \left( t 
\right)} \right]\left( {1+{\varepsilon ^2} \mathord{\left/ {\vphantom 
{{\varepsilon ^2} 6}} \right. \kern-\nulldelimiterspace} 6} \right).
\]
Следовательно, $M\left[ {\Phi \left( t \right)} \right]\le n\exp \left( 
{{t\varepsilon ^2} \mathord{\left/ {\vphantom {{t\varepsilon ^2} 6}} \right. 
\kern-\nulldelimiterspace} 6} \right)$ и $M\left[ {\Phi \left( {t^\ast } 
\right)} \right]\le n^{5 \mathord{\left/ {\vphantom {5 3}} \right. 
\kern-\nulldelimiterspace} 3}$. Отсюда по неравенству Маркова имеем, что 
($n\ge 8)$
\[
P\left( {\Phi \left( {t^\ast } \right)\le n^2} \right)\ge P\left( {\Phi 
\left( {t^\ast } \right)\le 2n^{5 \mathord{\left/ {\vphantom {5 3}} \right. 
\kern-\nulldelimiterspace} 3}} \right)\ge 1 \mathord{\left/ {\vphantom {1 
2}} \right. \kern-\nulldelimiterspace} 2.
\]
Тогда $P\left( {{\varepsilon U_i \left( {t^\ast } \right)} \mathord{\left/ 
{\vphantom {{\varepsilon U_i \left( {t^\ast } \right)} 2}} \right. 
\kern-\nulldelimiterspace} 2\le 2\ln n,\;i=1,...,n} \right)\ge 1 
\mathord{\left/ {\vphantom {1 2}} \right. \kern-\nulldelimiterspace} 2$. 
Откуда уже следует, что $P\left( {\vec {x}\left( {t^\ast } \right)\le 
\varepsilon \vec {e}} \right)\ge 1 \mathord{\left/ {\vphantom {1 2}} \right. 
\kern-\nulldelimiterspace} 2$.
\end{problem}
 \end{comment}

\begin{problem}(Аукцион первой цены)
Два участника аукциона конкурируют за покупку некоторого объекта. Ценности объекта    $v_1$ и $v_2$ для участников являются независимыми случайными величинами, равномерно распределенными на отрезке [0, 1]. Участник  имеет точную информацию о своем значении $v_i$, но не знает $v_j$. Участники делают ставки из диапазона [0, 1] одновременно и независимо друг от друга. В данном аукционе побеждает тот, кто поставил большую ставку. При равенстве ставок бросается жребий. Каждый обязан заплатить по средней ставке, даже если ему объект не достается! Отказаться от участия в этом аукционе нельзя.
\begin{enumerate}
\item Выпишите функции выигрыша игороков в данной игре;
\item Найдите наилучший ответ игрока в данной игре в классе квадратичных стратегий: $b_i(v_i) = cv_i^2$, где $c > 0$. Зависит ли наилучший ответ от стратегии соперника в данном случае?
\item Покажите, что в этой игре нет других (кроме найденных в пункте б)) оптимальных решений (наилучших ответов) с гладкими  возрастающими симметричными стратегиями. \textit{Симметричными} стратегиями является пара $(b_1(v_1), b_2(v_2)) = (b(v_1), b(v_2))$.
\end{enumerate}
\end{problem}

\begin{ordre}
Стратегией игрока $i$ в данной игре является функция $b_i: [0, 1] \rightarrow [0, 1]$ равная величине ставки при ценности объекта  $v_i$.
\end{ordre}


\begin{problem}(Распределения канторовского типа)
Рассмотрим в сумме $\sum_{k=1}^\infty 2^{-k} X_{k}$, где $X_{k} $ -- взаимно независимые с.в., имеющие распределение Бернулли с параметром ${1\mathord{\left/ {\vphantom {1 2}} \right. \kern-\nulldelimiterspace} 2} $, только слагаемые с четными номерами, или, что с точностью до множителя 3 (в дальнейшем потребуется для удобства) есть $Y=3\sum _{s=1}^{\infty }4^{-s} X_{s}  $. Покажите, что функция распределения $F(x)=\PR\left(Y\le x\right)$ является сингулярной (когда не оговаривается относительно какой меры, подразумевается, что относительно \textit{меры Лебега}, т.е. равномерной).


\begin{ordre}
Можно рассматривать $Y$ как выигрыш игрока, который получает $3\cdot 4^{-k} $, когда $k$-е бросание симметричной монеты дает в результате решку. Ясно, что полный выигрыш лежит между 0 и $3\left(4^{-1} +4^{-2} +\ldots \right)=1$. Если первое подбрасывание монеты привело к решке, то полный выигрыш $\ge {3\mathord{\left/ {\vphantom {3 4}} \right. \kern-\nulldelimiterspace} 4} $, тогда как в противоположном случае $Y\le 3\left(4^{-2} +4^{-3} +\ldots \right)=4^{-1} $. То есть неравенство ${1\mathord{\left/ {\vphantom {1 4}} \right. \kern-\nulldelimiterspace} 4} <Y<{3\mathord{\left/ {\vphantom {3 4}} \right. \kern-\nulldelimiterspace} 4} $ не может быть осуществлено ни при каких обстоятельствах, значит $F(x)={1\mathord{\left/ {\vphantom {1 2}} \right. \kern-\nulldelimiterspace} 2} $ в интервале $x\in \left({1\mathord{\left/ {\vphantom {1 4}} \right. \kern-\nulldelimiterspace} 4} ,{3\mathord{\left/ {\vphantom {3 4}} \right. \kern-\nulldelimiterspace} 4} \right)$. Чтобы определить, как ведет себя функция распределения на интервале $x\in \left(0,{1\mathord{\left/ {\vphantom {1 4}} \right. \kern-\nulldelimiterspace} 4} \right)$, покажите, что на этом интервале график отличается только преобразованием подобия $F(x)=(1/2)F(4x)$.

\end{ordre}

\begin{remark}
Пример, когда свертка двух сингулярных распределений имеет непрерывную плотность: с.в. $X=\sum _{k=1}^{\infty }2^{-k} X_{k}  $ имеет равномерное распределение на интервале $\left(0;1\right)$. Обозначим сумму членов ряда с четными и нечетными номерами через $U$ и $V$ соответственно. Ясно, что $U$ и $2V$ имеют одинаковое распределение и их распределение относится к канторовскому типу.
\end{remark}

\end{problem}


\begin{problem}
Напомним, что \textit{сингулярными} мерами называются меры, функции распределения $F(x)$ которых непрерывны, но точки их роста ($x$ -- точка роста $F(x)$, если для любого $\varepsilon >0$ выполняется: $F(x+\varepsilon )-F(x-\varepsilon )>0$) образуют множество нулевой меры Лебега. Покажите, что мера, соответствующая функции Кантора, сингулярна по отношению к мере Лебега.

\end{problem}

\begin{remark} (Соболевский А.Н. Конкретная теория вероятностей)
Любая вероятностная мера может быть представлена в виде суммы абсолютно непрерывной, дискретной и сингулярной мер. 

Сингулярные распределения вероятности возникают в эргодической теории и математической статистической физике как инвариантные меры диссипативных динамических систем, обладающих т.н. ``странными аттракторами''. Количественное изучение таких мер относится к геометрической теории меры и известно под названием ``фрактальной геометрии''. Основные импульсы развития этой дисциплины исходили из работ К. Каратеодори, Ф. Хаусдорфа, А. Безиковича 1920-х годов, а позднее -- Б. Мандельброта и многочисленных физиков, которые занимались ``динамическим хаосом'' в 1980-х годах (П. Грассбергер, И. Прокачча, Дж. Паризи, У. Фриш). Подробнее о мультифрактальных мерах см., например, книги: Е. Федер, Фракталы -- М.: УРСС, 2014; K. Falconer. Fractal Geometry: Mathematical Foundations and Applications, John Wiley  Sons, 1990; Я.Б. Песин, Теория размерности и динамические системы -- М.-Ижевск: Ин-т компьютерных исследований, 2002.

\end{remark} 


\begin{problem}[Модель Эрдёша--Реньи]
\label{sec:erdRenyi}
 Пусть есть конечное множество (в дальнейшем множество вершин) $V$, $\xi _{vv'} $ -- независимые с.в., занумерованные парами $\left\{v,v'\right\}\in V\times V$, $\vert V \vert = N$, $\xi _{vv'} \in \Be(p)$. 
Таким образом, можно задать абстрактный случайный граф на фиксированном множестве вершин. Покажите, что 
 
\begin{enumerate}
\item  При $p=\frac{1}{N^{1+\varepsilon } } $, $\varepsilon >0$ среднее число не изолированных вершин в случайном графе есть $o\left(N\right)$.
 
\item  При $p=\frac{1}{N^{1-\varepsilon } } $, $\varepsilon >0$ с вероятностью, близкой к 1, существует связная компонента порядка $N$ ($N \gg 1$).
\end{enumerate}
\end{problem}
 \begin{remark}
 См. брошюру В.А. Малышева \cite{27}, аналогично для следующей задачи.
 \end{remark}
 
 
\begin{problem} 
Рассматривается конфигурация спинов $\omega =\left\{x_{mn} \right\}$ (где $x_{mn} \in Be(p) $ -- независимые с.в.) на двумерной решетке $\left\{(m,n) \in {\mathbb Z}^{2} \right\}$. Вершину $(m,n)$ назовем занятой, если $x_{mn} =1$. Соединим ребром все соседние (находящиеся на расстоянии 1) занятые вершины. Получится случайный граф $G=G\left(\omega \right)$. Назовем кластером графа $G$ максимальное подмножество $A$ вершин решетки такое, что для любых двух $v,v'\in A$ существует связывающий их путь по ребрам графа $G$. Докажите, что существует такое $0<\bar{p}<1$, что при $p<\bar{p}$ все кластеры конечны с вероятностью 1, а при $p>\bar{p}$ с положительной вероятностью есть хотя бы один бесконечный кластер.
 
 
\begin{ordre}
Покажите, что при достаточно малых значениях $p$ вероятность события, что все кластеры конечны, равна 1. Покажите, что вероятность того, что кластер, содержащий начало координат и имеющий не менее $N$ вершин, не превосходит $\left(Cp\right)^{N} \mathop{\to }\limits_{N\to \infty } 0$, где $C$ -- некоторая константа. А значит и событие: бесконечный кластер содержит начало координат -- имеет нулевую вероятность.
\end{ordre}
 
\end{problem}

\begin{problem}

На некоторой реке имеется 6 островов (см. Рис. \ref{Fig:graphs_bridges.pdf}), соединенных между собой системой мостов. Во время летнего наводнения часть мостов была разрушена. При этом каждый мост разрушается с вероятностью ${1\mathord{\left/ {\vphantom {1 2}} \right. \kern-\nulldelimiterspace} 2} $, независимо от других мостов. Какова вероятность того, что после наводнения можно будет перейти с одного берега на другой, используя не разрушенные мосты? См. A.S. Cherny, The Kolmogorov student's competitions on probability theory, MSU.

\imgh{70mm}{graphs_bridges.pdf}{Схема мостов}

\end{problem}

\imgh{70mm}{guk.jpg}{``Иллюстрация к задаче Перколяция''. Анастасия Ковылина, 2014.}

\begin{problem}\Star(Перколяция)
В квадратном пруду (со стороной равной 1) 
выросли (случайным образом) $N\gg 1$ цветков лотоса, имеющих форму круга 
радиуса $r>0$. Назовем $r_N $ -- \textit{радиусом перколяции}, если с вероятностью не меньшей 0.99 не 
любящий воду жук сможет переползти по цветкам лотоса с северного берега на 
южный, не замочившись. Покажите, что $r_N \sim \dfrac{C}{\sqrt{N}}$. Оцените $C$.



\end{problem}

\begin{remark} 
См. Кестен Х. Теория просачивания для математиков. -- М.: Мир, 1986, Grimmett G. Percolation. -- М.:  Springer, 1999.
\end{remark}



\begin{problem}
В квадратной таблице задан процесс окрашивания ячеек: с вероятностью $p$ ячейка независимо окрашивается в черный цвет, иначе остается белой. Исследуем в такой системе размеры связных компонент из черных ячеек. При небольших значениях $p$ образуется много отделимых клеточных областей, в то время как при $p \simeq 1$ c большой вероятностью получится несколько  компонент соизмеримых со всей таблицей. Установлено, что значение $p_c = 0.5927462\ldots$ является критическим, при превышении которого в результате раскраски  среднее значение размера компоненты растет вместе с увеличением размера таблицы (таблица достаточно большая). Определим $\pi(s)$ как плотность распределения площадей компонент, где площадь одной клетки равна $a$. Допустим, что $\pi(s)$ зависит только от параметров $a$ и $\Exp s$. Так как исследуемая плотность не должна зависеть от масштаба, то 
\[
\pi(s) = c_1 f\left( \frac{s}{a}, \frac{a}{\Exp s} \right).
\]      
Изменим площадь одной клетки $a \to a/b$, тогда также ввиду независимости от масштаба в области ($p < p_c$) плотность должна принять аналогичный вид  
\[
\pi(s) = c_2(b) f\left( \frac{s}{a/b}, \frac{a/b}{\Exp s} \right) = c_2 f\left( \frac{sb}{a}, \frac{a}{\Exp sb} \right) .
\]
Приблизившись к значению $p_c$, где $\Exp s \to \infty$, получим соотношение
\[
\pi(s) = c_2 f\left( \frac{sb}{a}, 0 \right) = \frac{c_2}{c_1} \pi(sb).
\]
Докажите, что распределение со свойством $\pi(s) = g(b) \pi(sb)$  является степенным ($\pi(s) \propto s^{-\beta}$).
\end{problem}

\begin{remark}
В контексте этой и последующих задач рекомендуем ознакомится с работами M.E.J. Newman, Power laws, Pareto distributions and Zipf’s law, Contemporary Physics, 2005; M. Mitzenmacher
A Brief History of Generative Models for Power Law and Lognormal Distributions, Internet Mathematics, vol 1, No. 2, 2004.
\end{remark}

\begin{problem}
\label{depression}
В популяции каждый индивидуум имеет  порог чувствительности к депрессии, являющийся независимой от других индивидуумов с.в. 
На популяцию периодически нагоняют депрессию заданного уровня, после чего индивидуумы с порогом ниже уровня покидают популяцию, а на их место генерируются новые индивидуумы. Используя рассуждения предыдущей задачи, установите степенное распределение для количества заменяемых индивидуумов за одно воздействие депрессии.     
\end{problem}

\begin{problem}
Докажите, что итерационные интервалы между рекордными значениями при генерации независимых одинаково распределенных случайных величин имеют степенное распределение $p(x) \propto \frac{1}{x}.$ Элемент в последовательности называется  рекордным если все   предшествующие элементы последовательности меньше него.
\end{problem}

\begin{problem}
Типичность возникновения степенного распределения в ряде естественных манипуляций (к примеру, суммирование или умножение членов последовательности) со случайными величинами частично обусловлена схожестью с лог-нормальным распределением (см. задачу \ref{lognorm} из раздела \ref{zb4}) с плотностью
\[
p(x) = - \frac{(\ln x)^2}{2 \sigma^2} + \left( \frac{m}{\sigma^2} - 1 \right) \ln x  - \frac{m^2}{2 \sigma^2}.
\]    
При каких значениях $ \sigma$ распределения  похожи на большем участке значений $x$?
Покажите, что для произведения случайных величин характерна сходимость к лог-нормальному распределению. Докажите это же свойство для длин отрезков, получающихся в результате деления единичного отрезка с равномерным многократным выбором точек раздела (см. задачи \ref{permutation} и \label{permutation1} из раздела \ref{zb4}).     
\end{problem}


\begin{problem}[Growth model with preferential attachment]
\label{prefattach}
Рассмотрим следующую модель роста графа: пусть в каждый момент времени в 
графе добавляется новая вершина, и с вероятностью $\delta $ добавляется новое 
ребро, соединяющее новую вершину со случайно выбранной имеющейся вершиной 
(вероятность выбора пропорциональна степени вершины).



Покажите, что в этой модели справедлив степенной закон для степени вершин, а 
именно, что функция плотности распределения для степени вершины в точке $d$ 
равна $2{\delta ^2/d^3}$.
\begin{remark}
Обозначим через $d_i (t)$ -- среднее значение степени i-й вершины в момент 
времени $t$. Покажите, что для $d_i (t)$ справедливо следующее 
дифференциальное уравнение:
\[
\frac{\partial }{\partial t}d_i (t)=\frac{d_i (t)}{2t}
\]
с условием $d_i (i)=\delta $. Откуда следует, что $d_i (t)=\delta \sqrt 
{t/i} $.
Детали см. J. Hopcroft, R. Kannan, Computer science Theory for the Information Age.
\end{remark}
\end{problem}


\begin{problem}(Preferential attachment)
\label{pref_attach}
Типичная модель интернета (см. задачу \ref{hnmgraph} из раздела \ref{combinatorics})
представляет собой  процесс генерации направленного бинарного графа с кратными ребрами и петлями, инициализируемый одной вершиной с петлей, в котором добавление новой  вершины $t$ в граф происходит согласно правилу:  c вероятностью $\alpha$ вершина $t$ проводит исходящее ребро к вершине из списка $\{1,\ldots,t\}$, выбранной равновероятно; с вероятностью $1- \alpha$ вершина из списка выбирается не равновероятно, а пропорционально количеству входящих ребер. 
Введем переменную $X_d = X_d(t)$ -- количество вершин степени $d$ при количестве вершин в графе равном $t$.  Покажите, что справедливо тождество (при $\triangle t = 1$)
\[
\frac{\triangle X_d}{\triangle t} = \frac{\alpha (X_{d-1} - X_d) + (1- \alpha) ( (d-1)X_{d-1}  - dX_d )}{t},
\]     
Переходя к непрерывному параметру $t$ с использованием замены $X_d(t) = t c_d $ (предположение о неизменной доле вершин степени $d$ при больших $t$), установите соотношение
\[
\frac{c_d}{c_{d-1}} \sim 1- \left( \frac{2 - \alpha}{1 - \alpha} \right) \left( \frac{1}{d} \right), 
\quad c_d \sim   d^{ -\frac{2 -\alpha}{1 -\alpha}},
\]
то есть имеет место степенное распределение доли вершин степени~$d$.
\end{problem}

\begin{remark}
В контексте данной задачи см. работу M. Mitzenmacher 
A Brief History of Generative Models for Power Law and Lognormal Distributions,
Internet Mathematics, vol 1, No. 2, 2004. 
\end{remark}



 
\begin{problem}(Обобщенная схема размещений)
\noindent
\begin{enumerate}
\item  Пусть для целочисленных  неотрицательных с.в. $\eta _1,\ldots,\eta _N$ существуют независимые одинаково распределенные с.в. 
$\xi_1,\ldots,\xi_N$ такие, что
\[ 
\PR\left( {\eta _1 =k_1,\ldots,\eta _N =k_N } \right) =  \]\[ \PR\left( {\left. {\xi _1 
=k_1 ,\ldots,\xi _N =k_N } \right|\xi _1 +\ldots+\xi _N =n} \right) \;\;\;\; (*) \]
 
Введем независимые одинаково распределенные с.в. $\xi _1^{\left( r \right)} 
$,{\ldots},$\xi _N^{\left( r \right)} $, где $r$ целое неотрицательное число 
и
\[
\PR\left( {\xi _1^{\left( r \right)} =k} \right)=\PR\left( {\left. {\xi _1 =k} 
\right|\xi _1 \ne r} \right),
\quad
k=0,1,\ldots
\]
Пусть $p_r = \PR\left( {\xi _1 =r} \right)$ и $S_N =\xi _1 +\ldots+\xi _N$,
$S_N^{\left( r \right)} =\xi _1^{\left( r \right)} +\ldots+\xi _N^{\left( r \right)} $. Пусть $\mu _r \left( {n,N} \right)$ -- число с.в. $\eta _1$,{\ldots},$\eta_N $, принявших значение $r$. Покажите, что с.в. типа $\mu_r \left( {n,N} \right)$ можно изучать с помощью \textit{обобщенной схемы размещений}: для любого $k=0,\ldots,N$
\[
\PR\left( {\mu _r \left( {n,N} \right)=k} \right)=C_n^k p_r^k \left( {1-p_r } 
\right)^{N-k}\frac{\PR\left( {S_{N-k}^{\left( r \right)} =n-kr} 
\right)}{\PR\left( {S_N =n} \right)}.
\]
Напомним, что в классической схеме размещений $n$ различных частиц по $N$ 
различным ячейкам было доказано, что распределение заполнений ячеек $\eta _1 
$,{\ldots},$\eta _N $ имеет вид (\textit{полиномиальное распределение}):
\[
\PR\left( {\eta _1 =k_1 ,\ldots,\eta _N =k_N } \right)=\frac{n!}{k_1!\ldots k_N! N^n},
\]
где $k_1,\ldots,k_N$ -- неотрицательные целые числа, $\sum_{i} k_i =n$. Если положить $\xi_1,\ldots,\xi_N \in \Po \left( \lambda 
\right)$ -- независимые с.в. ($\lambda >0$ -- произвольно), то получим (*).
 
\item ** Дан случайный граф (модель Эрдеша--Реньи) $G\left( {n,\;p} \right)$. Пусть 
$p=c\frac{\ln n}{n}$. Покажите, что при $n\to\infty$ и $c>1$ граф $G\left( {n,\;p} \right)$ 
почти наверное связен, а при $c>1$ почти наверное не связен.
 
 
\end{enumerate}
\begin{remark}
См. монографию Колчин В.Ф. Случайные графы. -- М.:~Физматлит, 2004. Пункт б) подробно разобран в монографии Алона--Спенсера \cite{15}. Мы также рекомендуем смотреть популярные тексты А.М.~Райгородского по тематике случайных графов и их приложений, см. также раздел
\ref{combinatorics}.
\end{remark}
\end{problem}


\begin{problem}(Устойчивые системы большой размерности; В.И. Опойцев)
Из курсов функционального анализа и вычислительной математики хорошо известно, что если спектральный радиус матрицы 
$A=\| a_{ij}\|_{i,j=1}^{n}$ меньше единицы, $\rho(A)<1$, то итерационный процесс $x^{k+1}=A x^k + b$ 
(СОДУ $\dot{x}=-x+A x+ b$), вне зависимости от точки старта $x^0$, 
сходится к единственному решению уравнения $x^*=Ax^*+ b$. 
Скажем, если $\| A\|=\max\limits_{i} \sum\limits_j |a_{ij}|<1$, то и $\rho(A)<1$ (обратное, конечно, не верно). Предположим, что 
существует такое $\varepsilon>0$, что 
$$
\frac{1}{n}\sum\limits_{i,j} |a_{ij}|<1-\varepsilon . 
\quad (S)
$$
Очевидно, что отсюда не следует: $\rho(A)<1$. 
Тем не менее, введя на множестве матриц, удовлетворяющих условию $(S)$, равномерную меру, покажите, что относительная мера тех матриц 
(удовлетворяющих условию $(S)$), для которых спектральный радиус не меньше единицы, стремится к нулю 
с ростом $n$ ($\varepsilon$ --- фиксировано и от $n$ не зависит). 
\end{problem}
\begin{ordre}

1. Покажите, что  достаточно рассматривать матрицы с неотрицательными элементами. 

2.  Покажите, что достаточно доказать утверждение задачи на множестве матриц, удовлетворяющих условию 
$$
\frac{1}{n}\sum\limits_{i,j} a_{ij}=1-\varepsilon . 
\quad (SE)
$$

3. Далее положим $a_{ij}\in \mathrm{Exp} \bigl( n/(1-\varepsilon)\bigr)$ --- независимые одинаково распределенные случайные величины. Покажите, что при $n\to\infty$ распределение элементов случайной матрицы $A=\| a_{ij}\|_{i,j=1}^n$ 
будет сходиться к равномерному распределению на множестве матриц, удовлетворяющих ($SE$). 

4. Введя обозначения
$P_n={\mathbb P}(\| A\|\ge 1)\ge {\mathbb P}(\rho(A)\ge 1)$, воспользуйтесь неравенством Чебышёва

$$
P_n\le n {\mathbb P}\Bigl( \sum\limits_j a_{1j}\ge 1 \Bigr)=n {\mathbb P}\Bigl( X\ge 1 \Bigr)\le 
n {\mathbb P}\Bigl( |X-(1-\varepsilon)|\ge \varepsilon \Bigr)=
$$
$$
=n {\mathbb P}\Bigl( |X-{\mathbb E}X|\ge \varepsilon \Bigr)\le \frac{n}{\varepsilon^4} {\mathbb E}(X-{\mathbb E}X)^4=
O\Bigl(\frac{1}{n}\Bigr) \xrightarrow{n\to\infty} 0 . 
$$
\end{ordre}




\begin{problem}(Вероятностное доказательство формулы Эйлера \cite{7})
Пусть $X$ -- 
целочисленная случайная величина с распределением

\[\PR\left( {X=n} \right)=\frac{1}{\varsigma (s)n^s},\;\]
\noindent где $\varsigma 
(s)=\sum\limits_{n\in {\mathbb N}} {n^{-s}} ,\quad s>1$.

Пусть $1<p_1 <p_2 <p_3 <\ldots $ -- простые числа, и пусть $A_k $ -- событие 
= {$X$ делится на $p_k ${\}}.

\begin{enumerate}

\item Найдите $\PR(A_k)$ и покажите, что события $A_1 ,A_2 ,\ldots $ 
независимы;

\item Покажите, что
\begin{center}
$\prod\limits_{k=1}^\infty {(1-p_k ^{-s})} =\frac{1}{\varsigma (s)}$ (формула 
Эйлера).
\end{center}

\end{enumerate}
\end{problem}

\begin{problem}\Star(Статистика теоретико-числовых функций)
\label{sec:z_func_riman}
Довольно часто 
вероятностные соображения (например, независимость) используются в теории 
чисел не совсем строго, но зато весьма часто они позволяют угадать 
правильный ответ. Поясним сказанное, пожалуй, наиболее популярным примером (Дирихле, 1849 г.) 
из книги  Марка Каца 
``Статистическая независимость в теории вероятностей, анализе и теории 
чисел''. М.: ИЛ, 1963.

Пусть $A$ -- некоторое множество положительных целых чисел. Обозначим через 
$A\left( n \right)$ количество тех его элементов, которые содержатся среди 
первых $n$ чисел натурального ряда. Если существует предел $\mathop {\lim 
}\limits_{n\to \infty } {A\left( n \right)} \mathord{\left/ {\vphantom 
{{A\left( n \right)} n}} \right. \kern-\nulldelimiterspace} n=\PR\left( A 
\right)$, то он называется плотностью $A$. К сожалению, вероятностная мера 
$\PR\left( A \right)$ не является вполне аддитивной (счетно-аддитивной).

Рассмотрим целые числа, делящиеся на простое число $p$. Плотность множества 
таких чисел, очевидно, равна $1 \mathord{\left/ {\vphantom {1 p}} \right. 
\kern-\nulldelimiterspace} p$. Возьмем теперь множество целых чисел, которые 
делятся одновременно на $p$ и $q$ ($q$ -- другое простое число). Делимость 
на $p$ и $q$ эквивалентна делимости на $pq$, и, следовательно, плотность 
нового множества равна $1 \mathord{\left/ ({\vphantom {1 {pq}}}) \right. 
\kern-\nulldelimiterspace} {pq}$. Так как $1 \mathord{\left/ ({\vphantom {1 
{pq}}}) \right. \kern-\nulldelimiterspace} {pq}=\left( {1 \mathord{\left/ 
{\vphantom {1 p}} \right. \kern-\nulldelimiterspace} p} \right)\cdot \left( 
{1 \mathord{\left/ {\vphantom {1 q}} \right. \kern-\nulldelimiterspace} q} 
\right)$, то мы можем истолковать это так: ``события'', заключающиеся в 
делимости на $p$ и $q$, независимы. Это, конечно, выполняется для любого 
количества простых чисел.

Поставим теперь задачу посчитать долю несократимых дробей или, другими 
словами, ``вероятность'' несократимости дроби (фиксируется знаменатель дроби 
$n$, а затем случайно, с равной вероятностью $1 \mathord{\left/ {\vphantom 
{1 n}} \right. \kern-\nulldelimiterspace} n$ выбирается любое число от 1 до 
$n$ в качестве числителя, и подсчитывается доля случаев, в которых 
полученная дробь оказывалась несократимой) в следующем (чезаровском) смысле 
(здесь и далее индекс $p$ может пробегать только простые числа):
\[
\mathop {\lim }\limits_{N\to \infty } \frac{1}{N}\sum\limits_{n=1}^N 
{\frac{\# \left\{ {k<n:\;\;\text{Н.О.Д.}\left( {n,k} \right)=1} \right\}}{n}} 
=\mathop {\lim }\limits_{N\to \infty } \frac{1}{N}\sum\limits_{n=1}^N 
{\frac{\phi \left( n \right)}{n}} = \]\[\mathop {\lim }\limits_{N\to \infty } 
\frac{1}{N}\sum\limits_{n=1}^N {\prod\limits_p {\left( {1-\frac{\rho _p 
\left( n \right)}{p}} \right)} } ,
\]
где $\phi \left( n \right)$ -- функция Эйлера, $\rho _p \left( n 
\right)=\left\{ {\begin{array}{l}
 1,\quad n\mbox{ делится на }p, \\ 
 0,\quad \mbox{иначе}. \\ 
 \end{array}} \right.$.

\noindent Согласно введенному выше определению плотности:

\[
\Exp\left\{ {\prod\limits_{p\le p_k } {\left( {1-\frac{\rho _p \left( n 
\right)}{p}} \right)} } \right\}=\prod\limits_{p\le p_k } {\Exp\left\{ {\left( 
{1-\frac{\rho _p \left( n \right)}{p}} \right)} \right\}} 
=\prod\limits_{p\le p_k } {\left( {1-\frac{1}{p^2}} \right)} .
\]
С учетом этого хочется написать следующее:
\[
\mathop {\lim }\limits_{N\to \infty } \frac{1}{N}\sum\limits_{n=1}^N 
{\frac{\phi \left( n \right)}{n}} =\Exp\left\{ {\frac{\phi \left( n 
\right)}{n}} \right\}=\Exp\left\{ {\prod\limits_p {\left( {1-\frac{\rho _p 
\left( n \right)}{p}} \right)} } \right\}\mathop =\limits^? 
\]
\[
\mathop =\limits^? \prod\limits_p {\Exp\left\{ {\left( {1-\frac{\rho _p \left( 
n \right)}{p}} \right)} \right\}} =\prod\limits_p {\left( {1-\frac{1}{p^2}} 
\right)} =\frac{1}{\varsigma \left( 2 \right)}=\frac{6}{\pi ^2}.
\]
Будь введенная вероятностная мера, по которой считается это математическое 
ожидание, счетно-аддитивной, то можно было бы поставить точку, получив 
ответ. Однако, это не так. Несмотря на правильность ответа, 
приведенное выше рассуждение не может считаться доказательством. Впрочем, 
часто вероятностные рассуждения удается пополнить, используя их в качестве 
основы. Так в разобранном нами примере все сводится к обоснованию равенства 
``$?$''.

Легко понять, что полученный ответ несет определенную информацию о 
статистических свойствах функции Эйлера.

В теории чисел такого типа задачи занимают крайне важное место. Достаточно 
сказать, что гипотеза Римана ``на миллион'' о распределении нетривиальных 
нулей дзета-функции Римана $\varsigma \left( z \right)$ равносильна 
следующему свойству функции Мёбиуса $\mu(n)$:

\begin{center}
$\left| {\sum\limits_{n=1}^N {\mu \left( n \right)} } \right|\le \sqrt N $ 
(Одлыжко--Риэль),
\end{center}
где $\mu(n)$ определяется уравнением
\[
\sum\limits_{d \vert n} \mu(d) = 
\begin{cases}
1, & n = 1; \\
0, & n > 1.
\end{cases}
\quad
\Leftrightarrow
\quad
\mu(n) = 
\begin{cases}
(-1)^r, & n = p_1\ldots p_r; \\
0, & n \;  \mod \; p^2 = 0.
\end{cases}
\]
Что в свою очередь (Х.М. Эдвардс) в определенном смысле ``завязано'' на 
случайности последовательности $\left\{ {\mu \left( n \right)} 
\right\}_{n\in {\mathbb N}} $.\\


\noindent Используя указанный выше формализм, найдите долю чисел натурального ряда, 
``свободных от квадратов'', т.е. не делящихся на квадрат любого простого 
числа.

\end{problem}

\begin{remark} 
Известный российский математик Владимир Игоревич Арнольд 
последние десять лет жизни активно развивал описанное направление, которое 
он называл ``Экспериментальной математикой'' (помимо популярных книжек и 
статей, осталось и несколько видеолекций на эту тему на mathnet.ru
с выступлениями на семинаре МИАНа, в летней школе Современная математика и 
на мехмате). Получая схожим образом ``ответы'', их далее можно проверять, 
ставя численные эксперименты. При современных возможностях вычислительных 
машин, можно отслеживать логарифмические функции в асимптотике (т.е. 
проверять гипотезы с логарифмами), но не с повторными логарифмами, которые 
также как и в случайных процессах встречаются в теории чисел. Таким образом, 
у В.И. Арнольда получалось довольно много теорем (десятки, а возможно, даже сотни). 
Часть теорем, конечно, была известна ранее (см., например, книгу Карацубы 
А.А. Основы аналитической теория чисел. М.: Наука, 1975), но удавалось 
получать и новые формулировки.

Следующий пример, взятый из другой книги М. Каца \cite{20}, демонстрирует, что отмеченным выше 
способом можно получить и неверный результат.

\begin{example}[Из журнала Nature, 1940] Из изложенного выше следует, что количество целых чисел, не превосходящих $N$ и не делящихся ни на одно из простых 
чисел $p_1 $, $p_2 $, {\ldots}, $p_k $, равно приблизительно 
$N\prod\limits_{j=1}^k {\left( {1-\frac{1}{p_j }} \right)} $. Рассмотрим 
теперь количество целых чисел, не превосходящих $N$ и не делящихся ни на одно из 
простых чисел, меньших $\sqrt N $. Такими числами могут быть только простые 
числа, лежащие между $\sqrt N $ и $N$, число которых
\[
\pi \left( N \right)-\pi \left( {\sqrt N } \right)\sim N\prod\limits_{p_j 
<\sqrt N } {\left( {1-\frac{1}{p_j }} \right)} .
\]
Но из теории чисел известно, что $\pi \left( N \right)\sim N \mathord{\left/ 
{\vphantom {N {\ln N}}} \right. \kern-\nulldelimiterspace} {\ln N}$ и
\[
\prod\limits_{p_j <\sqrt N } {\left( {1-\frac{1}{p_j }} \right)} \sim {\exp 
\left( {-\gamma } \right)} \mathord{\left/ {\vphantom {{\exp \left( {-\gamma 
} \right)} {\ln \sqrt N }}} \right. \kern-\nulldelimiterspace} {\ln \sqrt N 
}={2\exp \left( {-\gamma } \right)} \mathord{\left/ {\vphantom {{2\exp 
\left( {-\gamma } \right)} {\ln N}}} \right. \kern-\nulldelimiterspace} {\ln 
N},
\]
где $\gamma $ -- константа Эйлера. Следовательно, $2\exp \left( {-\gamma } 
\right)=1$. Пришли к неверному соотношению!
\end{example}

Интересно в этой связи отметить также вероятностный способ получения 
правильной асимптотической формулы $\pi \left( N \right)\sim N 
\mathord{\left/ {\vphantom {N {\ln N}}} \right. \kern-\nulldelimiterspace} 
{\ln N}$ для количества простых чисел, не превосходящих $N$, приведенный в книге 
Куранта Р., Роббинса Г. Что такое математика. М.: МЦНМО, 2007, см. также \cite{2013}.

В заключение заметим, что применение вероятностных соображений в теории 
чисел продолжает привлекать ведущих математиков и по сей день (см. 
Я.Г. Синай Статистические свойства функции Мебиуса, Автомат. и телемех., 2013, № 10, 6–14, а также Т. Тао Структура и случайность -- М.: МЦНМО, 2013).
\end{remark}

\begin{problem}\Star(Парадокс Банаха--Тарского)
\label{banah_tar}
С. Банах показал, что если предполагать только аддитивность меры, то в одно- и двумерном пространствах любое ограниченное множество становится измеримым (имеет длину и площадь). Таким образом, в одно- и двумерном случаях равномерное распределение можно задать на любом (ограниченном) множестве, если от вероятности требовать только аддитивность. Приведите пример, показывающий, что в трехмерном пространстве это сделать невозможно. 
\end{problem}
\begin{remark}
Базируясь на аксиоме выбора, шар в трехмерном пространстве допускает такое разбиение на конечное число непересекающихся множеств, из которых можно составить передвижением (как твердых тел = перенос + поворот) два шара того же радиуса (см., например, В. Босс, Т. 6, 12, 16 \cite{2013}).
\end{remark}

\begin{problem}
Найдите вероятность $q_n$ того, что случайная $(0,1)$-матрица размера $n\times n$ является невырожденной над полем 
$GF_2=\{ 0,1\}$. Доказать, что существует $\lim\limits_{n\to\infty} q_n=q>0$. 
\end{problem}


\begin{problem}\Star(Закон Вигнера \cite{7})

Пусть $\xi _{ij}^{n} $, $1\le i,j\le n$, $n=1,2,\ldots $ -- совокупность одинаково распределенных случайных величин, удовлетворяющих условиям:

\begin{enumerate}
\item \label{zero}  при каждом $n$ случайные величины $\xi _{ij}^{n} $, $1\le i\le j\le n$ независимы;

\item \label{first}  матрица $\left(A^{n} \right)_{ij} =\frac{1}{2\sqrt{n} } \left(\xi _{ij}^{n} \right)$ симметричная, т.е. $\xi _{ij}^{n} =\xi _{ji}^{n} $;

\item \label{second} случайная величина $\xi _{ij}^{n} $ имеет симметричное распределение, т.е. для всякого борелевского множества $B\in B({\mathbb R})$ выполнено равенство $\PR\left(\xi _{ij}^{n} \in B\right)=\PR\left(\xi _{ij}^{n} \in -B\right)$;

\item \label{third} все моменты с.в. $\xi _{ij}^{n} $ конечны, т.е. $\Exp\left[\left(\xi _{ij}^{n} \right)^{k} \right]<\infty $ при всех $k\ge~1$, причем дисперсия равна единице, $\Var\left[\xi _{ij}^{n} \right]=1$.
\end{enumerate}

Рассмотрим дискретную вероятностную меру $\mu ^{n} $ собственных значений $\lambda _{1}^{(n)} ,\ldots ,\lambda _{n}^{(n)} $ случайной матрицы $A^{n} $: для всякого борелевского множества $B\in B({\mathbb R})$
\begin{center}
$\mu ^{n} (B)=\frac{1}{n} \sum _{i=1}^{n}I\left\{\lambda _{i}^{(n)} \in B\right\} $. 
\end{center}
Ясно, что такая мера сама по себе случайна, так как зависит от собственных значений случайной матрицы. Пусть $L_{k}^{n} $ -- $k$-й момент меры $\mu ^{n} $ матрицы $A^{n} $, т.е. 
\begin{center}
$L_{k}^{n} =\int _{-\infty }^{\infty }x^{k} \mu ^{n} (dx) =\frac{1}{n} \sum _{i=1}^{n}\left(\lambda _{i}^{(n)} \right)^{k}  $ (также случайная величина).
\end{center}
Докажите, что $\mathop{\lim }\limits_{n\to \infty } \Exp\left[L_{k}^{n} \right]=m_{k} $, $\mathop{\lim }\limits_{n\to \infty } \Var\left[L_{k}^{n} \right]=0$, где
\begin{center}
 $m_{k} =\frac{2}{\pi } \int _{-1}^{1}\lambda ^{k} \sqrt{1-\lambda ^{2} } d\lambda  $,
\end{center}
то есть случайные меры собственных значений в некотором смысле сходятся к неслучайной мере на действительной прямой с плотностью, задаваемой \textit{полукруговым законом Вигнера}: 

\[p(\lambda )=\left\{\begin{array}{cc} {\frac{2}{\pi } \int _{-1}^{1}\sqrt{1-\lambda ^{2} } d\lambda , } & {-1\le \lambda \le 1;} \\ {0,} & {\text{иначе.}} \end{array}\right. \] 

\noindent Применив неравенство Чебышёва, покажите, что $L_{k}^{n} \mathop{\to }\limits_{n\to \infty }^{p} m_{k} .$


\end{problem}



\begin{remark}
Если $\eta_{n} $ -- последовательность мер, моменты которых сходятся к соответствующим моментам меры $\eta $, то (при дополнительных условиях на рост моментов мер $\eta_{n} $) сами эти меры слабо сходятся к мере $\eta $.

См. по данной тематике М.Л. Мета Случайные матрицы -- М.:  МЦНМО, 2012, а также T. Tao, Topics in random matrix theory.Graduate Studies in Mathematics, vol. 132, 2012. 
\end{remark}



\begin{ordre} (См. Коралов--Синай \cite{7}). Покажите, что 

\[\Exp\left[L_{k}^{n} \right]=\frac{1}{n} \Exp\left[\sum _{i=1}^{n}\left(\lambda _{i}^{(n)} \right)^{k}  \right]=\frac{1}{n} \Exp\left[\mathrm{tr}\left[\left(A^{n} \right)^{k} \right]\right] = \]\[ \frac{1}{n} \left(\frac{1}{2\sqrt{n} } \right)^{k} \Exp\left[\sum _{i_{1} ,\ldots ,i_{k} =1}^{n}\xi _{i_{1} i_{2} }^{n} \xi _{i_{2} i_{3} }^{n} \cdots \xi _{i_{k} i_{1} }^{n}  \right].\] 
\noindent где $\lambda _{i}^{(n)} $ -- собственные значения матрицы $A^{n} $ и $\mathrm{tr}\left[\left(A^{n} \right)^{k} \right]$ -- след ее $k$-й степени.
С учетом условий \[\Exp\left[\xi _{i_{1} i_{2} }^{n} \xi _{i_{2} i_{3} }^{n} \cdots \xi _{i_{k} i_{1} }^{n} \right]=\prod_{J_k} \Exp\left[\left(\xi _{ij}^{n} \right)^{p(i.j)} \right], \]\[J_k = \{ i,j: i \leq j, \;
\sum p(i,j) = k\}. \] причем в силу условия \ref{second} нечетные моменты равны нулю: $$\Exp\left[\left(\xi _{ij}^{n} \right)^{p(i.j)} \right]=0,$$ если $p(i,j)$ -- нечетное. Таким образом, для четных моментов $k=2s$ имеем:

\[\Exp\left[L_{2r}^{n} \right]=\frac{1}{2^{2r} n^{r+1} } \Exp\left[\sum _{i_{1} ,\ldots ,i_{2r} =1}^{n}\xi _{i_{1} i_{2} }^{n} \xi _{i_{2} i_{3} }^{n} \cdots \xi _{i_{2r} i_{1} }^{n}  \right]= \]\[ \frac{1}{2^{2r} n^{r+1} } \sum _{i_{1} ,\ldots ,i_{2r} =1}^{n}\prod_{J_r} \Exp\left[\left(\xi _{ij}^{n} \right)^{2p(i.j)} \right]  .\] 

 Задачу можно свести к комбинаторному подсчету соответствующих путей (сопоставленных ненулевым слагаемым $\xi _{i_{1} i_{2} }^{n} \xi _{i_{2} i_{3} }^{n}, \ldots, \xi _{i_{2r} i_{1} }^{n} $) на множестве $\left\{1,2,\ldots ,n\right\}$, где каждое ребро (петли при этом не запрещаются) проходится четное число раз (без учета направления). Совокупность таких путей можно разбить на два класса: пути, накрывающие дерево из $r$ ребер, каждое из которых проходится дважды и остальные пути, для которых либо есть петли, либо циклы, либо есть ребра, через которые путь проходит по крайней мере четыре раза. Для путей второго класса (с учетом условия \ref{third}) получите оценку: \[
 \Exp\left[
    \sum  \limits_{ i_{1} < \ldots < i_{2r} } \xi _{i_{1} i_{2} }^{n} \xi _{i_{2} i_{3} }^{n} \cdots           \xi _{i_{2r} i_{1} }^{n}  
\right]\le C_{r} n^{r} ,
 \] \noindent где $C_{r} $ -- некоторая константа.

Как следствие, вклад таких путей в искомое математическое ожидание стремится к нулю при $n\to \infty $: \[\mathop{\lim }\limits_{n\to \infty } \frac{1}{2^{2r} n^{r+1} } C_{r} n^{r} k_{r} =0.\] Для путей первого класса в силу условия \ref{third} на дисперсию получите $\Exp \left(\xi _{i_{1} i_{2} }^{n} \xi _{i_{2} i_{3} }^{n} \cdots \xi _{i_{2r} i_{1} }^{n}\right) =1$. Покажите, что число путей первого класса равно \[\frac{n!(2r)!}{(n-r-1)!r!(r+1)!}. \] Для этого сопоставьте таким путям неотрицательные траектории одномерного симметричного случайного блуждания: $\left(\omega _{0} \omega _{1} \ldots \omega _{2r} \right)$, где $\omega _{0} =\omega _{2r} =0$, $\omega _{i} \ge 0$ при всех $i=1,\ldots ,2r$. Каждой фиксированной такой траектории $\left(\omega _{0} \omega _{1} \ldots \omega _{2r} \right)$ соответствует $n(n-1)\ldots (n-r)$ допустимых путей первого класса. Число неотрицательных траекторий с начальной и конечной нулевой точкой равно $\frac{(2r)!}{r!(r+1)!} $ (см. задачу \ref{sec:katalan} из раздела \ref{genF}).
Итак, 

\[\mathop{\lim }\limits_{n\to \infty } \Exp\left[L_{2r}^{n} \right]=\mathop{\lim }\limits_{n\to \infty } \frac{1}{2^{2r} n^{r+1} } \frac{n!(2r)!}{(n-r-1)!r!(r+1)!} =\frac{(2r)!}{2^{2r} r!(r+1)!} =m_{2r} .\] 
Утверждение касательно дисперсии $L_{k}^{n} $ доказывается аналогично.

\end{ordre}

\begin{problem}
На автобусных остановках обычно указывается интервал движения автобуса $m$, т.е. среднее время между двумя последовательными прибытиями автобусов. Предположим, что эти интервалы независимы и одинаково распределены с математическим ожиданием $m$ и дисперсией $\sigma^2$. Покажите, что если пассажир приходит на остановку в случайный момент времени, то математическое ожидание времени ожидания автобуса будет равно $\left(m^2+\sigma^2\right)/2m$.

Если известна функция распределения с.в., равной интервалу между последовательными прибытиями автобусов, то какое время оптимально стоять автобусу на остановке? Время зависит от текущей реализации этой с.в.,    а оптимальность необходимо понимать в смысле минимизации среднего времени ожидания автобуса. Решите задачу для показательного распределения.
\end{problem}
\begin{remark}
Следует заглянуть в [\ref{sekei}], аналогично для следующей задачи.
\end{remark}

\begin{problem}
Продавец газет заказывает ежедневно $N$ газет. С каждой проданной газеты он получает прибыль $b$ и теряет $c$ на каждой газете, оставшейся непроданной.  Каким нужно выбрать  $N$, чтобы максимизировать ожидаемую прибыль, если число покупателей за день имеет распределение $\Po(\lambda)$? 
\end{problem}

\section{Байесовские методы}
\label{bayes}

\begin{problem}
Охранная система представлена в виде Байесовской модели, изображенной на Рис. \ref{Fig:bayes1.png}. 
\imgh{50mm}{bayes1.png}{Модель охранной системы.}
Пусть $t$ обозначает факт срабатывания тревоги, $v$ -- наличие вора $\in \text{Be}(s)$, $e$ -- было ли землетрясение, $r$ -- наличие радиосообщения. Величины $t$, $v$, $e$, $r$ имеют распределение Бернулли. Переменная $s$ -- статистика по криминогенной активности -- принимает значения из отрезка [0, 1]. Значения $\PR(t = 1|v,e)$, $\PR(r = 1|e)$, $\PR(e = 1)$ и $\PR(v = 1)$ заданы, причем $\PR(t = 1|v=0,e=0) = 0$, $\PR(r = 1|e = 0) = 0$.
Проведите расчет $\PR(v = 1|t = 1, s = s_0)$ и $\PR(v = 1 | t = 1, s=s_0, r = 1)$. 
\end{problem}

\begin{remark}
Генеративные вероятностные модели часто представляются в виде интуитивно понятных  \textit{графических моделей}. Графическая модель -- это граф, определяющий зависимости между случайными величинами. Вершинам соответствуют случайные величины, ребрам -- зависимости между ними.

Наблюдаемые величины (значения которых известны) закрашиваются, скрытые (значения которых надо найти) -- остаются незакрашенными. Стрелка из вершины $A$ в $B$  обозначает зависимость $B$ от $A$ (часто под такого рода зависимостью подразумевается, что $A$ является параметрами распределения с.в. $B$).

Для обозначения повторяющихся величин с одинаковым распределением используются прямоугольники, которые могут быть вложенными. В одном из углов прямоугольника обычно указывается количество повторений случайных величин, расположенных внутри него. 

\end{remark}

\begin{problem}
Радиоактивный источник излучает за одну секунду $n \in \Po(s)$ частиц, где $s$ -- неизвестная интенсивность излучения. Прибор, регистрирующий частицы, имеет погрешность $\theta = 0.9$, т.е. количество зарегистрированных частиц за одну секунду $c \in \mathrm{Bin}(\theta,n)$. Покажите, что 
\[
\PR(c|\theta,s) = \Po(s\theta),
\]
\[
\PR(n-c|c,\theta,s) = \Po(s(1-\theta)). 
\]
Предположим, что за первую секунду прибор зарегистрировал $c_1 = 10$ частиц, а за вторую -- $c_2 = 16$ частиц (см. Рис. \ref{Fig:bayes3.png}). Докажите формулы для условного распределения и математического ожидания $n_1$: 
\[
\PR(n_1 | c_1, c_2, \theta, s) = \int_{0}^{\infty} p(n_1|c_1,\theta,s) p(s|c_1,c_2,\theta) ds = 
\]
\[
= C_{n_1+c_2}^{c_1+c_2} \left(\frac{2\theta}{1+\theta}\right)^{c_1+c_2+1} \left(\frac{1-\theta}{1+\theta}\right)^{n_1 - c_1},  
\]
\[
\Exp(n_1 | c_1, c_2, \theta, s) = \frac{c_1 + c_2 + 1 - \theta}{2\theta} + \frac{c_1 - c_2}{2} = 131.5.
\]
Найдите 90\% доверительный интервал для $n_1 | c_1, c_2, \theta, s$, воспользовавшись соотношением
\[
\PR(|X - \Exp X| > t \sqrt{\Var X}) \leq \frac{1}{t^2}.
\]
\imgh{60mm}{bayes3.png}{Модель радиоактивного источника.}
\end{problem}


\begin{problem}[Распределение Дирихле]
\label{dir}

Ознакомьтесь с распределениями Дирихле, бета и гамма, приведенными в замечании. Покажите, что если $(X_1, \ldots, X_k)$, $X_i \in \text{Г}( \gamma_i, 1)$ -- гамма распределенные независимые с.в. и $S = \sum \limits_{i=1}^k X_i$, то 
\[\tag{1}
(X_1 / S , \ldots, X_k / S) \in \mathrm{Dir}(\gamma_1, \ldots, \gamma_{k}).
\] 
Предложите способ генерации вектора с распределением Дирихле ($\gamma_i \in \mathbb{N}$), имея в распоряжении генератор с.в. с равномерным распределением на отрезке $[0, 1]$. 

Покажите, что каждая компонента вектора с распределением Дирихле имеет бета-распределение $X_i/S \in \mathrm{Beta}(\gamma_i, \sum \limits_{j \neq i} \gamma_j)$. 

Положим, что вектор $(\theta_1,\ldots,\theta_k)$ априорно имеет распределение (1), выборка $Y_1,\ldots, Y_n$ сгенерирована из дискретного распределения $\mathrm{Cat}(\theta_1,\ldots,\theta_k)$, т.е. $\PR(Y_i = m | \theta_1,\ldots,\theta_k) = \theta_m$. Докажите справедливость формулы для плотности апостериорного распределения $\theta_1,\ldots,\theta_k$: 
\[
p(\theta_1,\ldots,\theta_k \vert Y_1,\ldots, Y_n, \gamma_1, \ldots, \gamma_{k}) = 
\]
\[
=\mathrm{Dir}\left(\gamma_1 + \sum_{i=1}^n \delta_1(Y_i), \ldots, \gamma_{k}+ \sum_{i=1}^n \delta_k(Y_i)\right),
\]
где $\delta_k(x) = [x = k]$.
Убедитесь, что данная функция принимает максимальное значение при 
\[\tag{2}
\theta_m \propto \gamma_m - 1 + \sum_{i=1}^n \delta_m(Y_i).
\]

\end{problem}

\begin{remark}

Говорят, что вектор $X$ принадлежит распределению \textit{Дирихле} $\mathrm{Dir}(\gamma_1, \ldots, \gamma_{k})$, если: 
\[
f_{X}(x_1, \ldots, x_k) = \frac{\text{Г}(\gamma_1 +  \ldots + \gamma_{k})}{\text{Г}(\gamma_1)  \ldots  \text{Г}(\gamma_{k})} x_1^{\gamma_1-1} \ldots x_k^{\gamma_k-1},
\] 
\noindent где  $(x_1, \ldots, x_k) \in \{x_i \geq 0, \; \sum x_i  = 1 \}$, $\gamma_i \geq 0$, $\text{Г}$ -- гамма функция. 

Говорят, что вектор $X$ принадлежит \textit{бета} распределению \\  $\mathrm{Beta}(\gamma_1, \gamma_2)$, если: 

\begin{center}
$f_{X}(x) = \frac{\text{Г}(\gamma_1 + \gamma_2)}{\text{Г}(\gamma_1) \text{Г}(\gamma_2)} x^{\gamma_1-1} (1 - x)^{\gamma_2-1}$, где  $x \in [0, 1]$.
\end{center} 


Допускается также обозначение следующего вида \\ $ X \in \mathrm{Dir}(\gamma_1, \ldots, \gamma_{k+1})$, если 
\[
f_{X}(x_1, \ldots, x_k) = \frac{\text{Г}(\gamma_1 +  \ldots + \gamma_{k+1})}{\text{Г}(\gamma_1)  \ldots  \text{Г}(\gamma_{k+1})} x_1^{\gamma_1-1} \ldots x_k^{\gamma_k-1} (1 - x_1 - \ldots - x_k)^{\gamma_{k+1}-1},
\] 


\noindent где  $(x_1, \ldots, x_k) \in \{x_i \geq 0, \; \sum x_i \leq 1 \}$.

Говорят, что с.в. $X$ принадлежит \textit{гамма} распределению
\\  $\text{Г}(\alpha, \lambda)$, если: 
\begin{center}
$f_{X}(x) = \frac{x^{\alpha-1} e^{-x/\lambda}}{\lambda^\alpha\text{Г}(\alpha)}$,  где  $x \geq 0$.
\end{center}

Распределение Дирихле $\mathrm{Dir}(\gamma_1, \ldots, \gamma_{k})$ может быть интерпретировано следующим образом. Его функция плотности возвращает вероятность того, что вероятности $A_1,\ldots,A_K$ -- несовместных  событий равны соответственно $x_1, \ldots, x_K$, $\sum_i x_i = 1$ при условии, что $i$-е событие наблюдалось $\gamma_i-1$ раз. 

Отметим два свойства распределения Дирихле, делающего его удобным для использования в качестве априорного, при оценке параметров дискретного распределения (см. выражение (2)). Первое, семейство распределений Дирихле является сравнительно большим среди всех возможных распределений на симплексе, что является своего рода страховкой от неправильного выбора типа априорного распределения. Второе, имея выборку, сгенерированную из дискретного распределения, апостериорное распределение допускает простое аналитическое вычисление. 

Обратите внимание, что в выражении (2)  при $\gamma_i < 1$, $i = \overline{1,k}$ априорное распределение способствует разреживанию (увеличению количества нулевых компонент) $\theta_1,\ldots,\theta_k$, при  $\gamma_i = 1$, $i = \overline{1,k}$ априорное распределение является равномерным, а при  $\gamma_i > 1$, $i = \overline{1,k}$ распределение способствует сглаживанию $\theta_1,\ldots,\theta_k$.

Распределение Дирихле приобрело широкое практическое применение в области  тематического моделирования текстов (см. задачу \ref{sec:them_modeling}), машинного перевода, построения моделей в экологии, байесовской проверки гипотез, точечного оценивания и оценки доверительных интервалов. 
\end{remark}

\begin{problem}[Упорядоченное распределение Дирихле]
Пусть $X_{(k)}$ является $k$-ой порядковой статистикой в наборе $X_1, \ldots, X_n$ (*) случайных величин с функцией распределения $F_X(x)$. Пусть $U_k = F_X(X_{(k)}).$ Покажите, что $U \in \mathrm{Dir^*}(1, \ldots, 1)$ (см. замечание).

Сравните доказанное утверждение с задачей \ref{sec:ordered_seq} раздела \ref{standart}. 
Покажите, что если $\mathbb{P}(X \in [X_{(i-1)}, X_{(i)}]) = W_i = U_i - U_{i-1}$, где $X_{(0)} = -\infty$, $X_{(n+1)} = +\infty$, то 
$W \in \mathrm{Dir}(1, \ldots, 1)$.
 
Имея в распоряжении набор векторов (*), где $X_i \in \mathbb{R}^m$, эмпирическая оценка плотности распределения в точке $x$ может быть найдена следующим образом:
\[
\widehat{f}_X(x) = \frac{\mathbb{E}U_k}{V_{r_k}},
\]    
где $r_k$ -- расстояние от $x$ до $X_{(k)}$, $V_{r_k}$ -- объем шара радиуса $r_k$ в пространстве $\mathbb{R}^m$. Докажите, что  
\[
U_k \in \mathrm{Beta}(k, n-k+1), 
\]
\[
\widehat{f}_X(x) \xrightarrow{p} f_X(x), \quad
n \to \infty.
\]    


\end{problem}


\begin{remark}

Говорят, что вектор $Y$ принадлежит \textit{упорядоченному распределению Дирихле} $\mathrm{Dir^*}(\gamma_1, \ldots, \gamma_{k+1})$, если: 
\[
f_{Y}(y_1, \ldots, y_k) = \]
\[ \frac{\text{Г}(\gamma_1 +  \ldots + \gamma_{k+1})}{\text{Г}(\gamma_1)  \ldots  \text{Г}(\gamma_{k+1})} y_1^{\gamma_1-1}(y_2 - y_1)^{\gamma_2-1} \ldots (y_k - y_{k-1})^{\gamma_k-1} (1 - y_k)^{\gamma_{k+1}-1},
\] 

\noindent где  $(y_1, \ldots, y_k) \in \{ 0 \leq y_1 \leq \ldots \leq  y_k \leq 1 \}$, $\gamma_i \geq 0$.

Причем, если $X \in \mathrm{Dir}(\gamma_1, \ldots, \gamma_{k+1})$ и $Y_i = \sum \limits_{j=1}^{i} X_j$, $i = \overline{1,k}$ то 
\[
Y \in \mathrm{Dir^*}(\gamma_1, \ldots, \gamma_{k+1}).
\]

\end{remark}


\begin{problem} 
\label{po_dir}
Распределение Пуассона--Дирихле получается из упорядоченного вектора с распределением Дирихле предельным переходом, описанным в замечании. Докажите, что величины $\theta_{(i)}$ с вероятностью 1 убывают с экспоненциальной скоростью:
\[
\theta_{(i)} \propto e^{- i/\lambda}.
\]
Такое распределение часто возникает при моделировании разделения большого числа элементов из $S$ на большое число видов. Предположим, что из $S$, была произведена выборка размера $n$. Докажите, что вероятность того, что все элементы выборки имеют один и тот же вид равна
\[
\frac{\lambda \text{Г}(\lambda) \text{Г}(n) }{\text{Г}(\lambda+n)}.
\]
\end{problem}

\begin{remark}
Пусть $\theta_1,\ldots,\theta_n \in \mathrm{Dir}(\gamma_1,\ldots,\gamma_n)$, при этом
выполнены условия
\[
\max_i\gamma_i \to 0,
\quad n \to \infty,
\] 
\[
\lambda_n = \sum_{i=1}^n \gamma_i \to \lambda. 
\]
Введем обозначения
\[
s_0 = 0, \quad s_j = \gamma_1 + \ldots + \gamma_j, \quad j = \overline{1,n}.
\]
Распределение Дирихле может быть сконструировано из $Y_i \in \text{Г}(\gamma_i, 1)$ по правилу $\theta_i = Y_i/\sum_j Y_j$. Известно, что $Y_i$ безгранично делимы, т.е. могут быть разбиты на сколь угодно большое число гамма распределенных случайных величин $Y_{i1},\ldots,Y_{ip}$. Частичные суммы величин $\sum_j Y_j$ приближаются в пределе к случайному процессу с независимыми приращениями $g$: 
\[
g(s_m) = \sum_{i=1}^{m} Y_i, \quad \theta_j = \frac{g(s_j) - g(s_{j-1})}{g(s_n)}. 
\]       
Из вида характеристической функции 
\[
\phi_{g(s)}(\mu) = \exp \left(  
-s \int_{0}^{\infty} \left(1 - e^{-\mu z} \right) \frac{e^{-z}}{z} dz
\right)
\]
\[
\gamma(dz) = \frac{e^{-z}}{z} dz
\]
заключаем, что $g(s)$ является субординатором на множестве $s \in [0, \lambda]$ (см. задачу \ref{subord} из раздела \ref{geom}). Величины скачков процесса $g(s)$ образуют пуассоновский процесс $J(t)$ на множестве $t \in (0, \infty)$ c переменной интенсивностью
\[
\lambda(z) = \lambda \frac{e^{-z}}{z}.
\]   
Обозначим за $J_{(1)} \geq J_{(2)} \geq J_{(3)} \geq \ldots$ упорядоченные величины скачков. Последовательность $\theta_{(1)} \geq \theta_{(2)} \geq \theta_{(3)} \geq \ldots$, где 
\[
\theta_{(i)}  = \frac{J_{(i)}}{g(\lambda)},
\]
имеет распределение \textit{Пуассона--Дирихле}.

\end{remark}

\begin{problem}[Сопряженные распределения]
\label{bayes_def}
Сформулируем задачу байесовского вывода. Пусть известно распределение $p(X|\theta)$ (\textit{правдоподобие выборки} $X$). Требуется найти: 
\begin{enumerate}
\item $p(\theta|X) = \frac{p(X|\theta)p(\theta)}{p(X)}$ -- апостериорное распределение;
\item $p(x'|X) = \int_\Theta p(x'|\theta)p(\theta|X) d\theta$ -- предсказание нового $x'$.
\end{enumerate} 

Ключевым моментом является выбор наиболее разумного $p(\theta)$ -- \textit{априорного распределения}. Выбор осуществляется, исходя из дополнительной информации о параметре $\theta$ и необходимости введения регуляризации (отдание предпочтения более простым функциям $p(X | \theta)$, которые могут быть построены более точно при ограниченном размере выборки $|X|$). 

Решение  задач Байесовского вывода существенно упрощается при использовании \textit{сопряженных семейств} распределений (см. М. Де Гроот. Оптимальные статистические решения. -- М.: Мир, 1974).  Семейство распределений $\lbrace p(\theta|\alpha), \; \alpha \in \Omega \rbrace$ называется \textit{сопряженным} к семейству правдоподобий $\lbrace p(X|\theta) , \; \theta \in \Theta \rbrace$, если $\forall 
\alpha \; \exists \alpha'$ такой, что апостериорное распределение $p(\theta|X, \alpha) = p(\theta|\alpha')$ (т.е.  апостериорное распределение имеет такой же вид, что и априорное, но с другими параметрами). Именно  сопряженное распределение разумно взять в качестве априорного, что, в частности, упростит вычисление нормировочного интеграла: 
\[
p(X) = \int \limits_{\theta} p(X|\theta)p(\theta) d \theta.
\] 
 

Рассмотрим несколько примеров сопряженных семейств. Покажите, что:

\begin{enumerate}
\item Сопряженным к распределению Бернулли является \textit{бета}  распределение (см. замечание к задаче \ref{dir}).
Если $x_1, \ldots, x_n$ -- реализация выборки из распределения Бернулли, 
априорное распределение есть $\mathrm{Beta}(\alpha, \beta)$, то апостериорное распределение будет иметь параметры:
\[
\alpha + \sum \limits_{i = 1}^n x_i,  \; \; \beta + n - \sum \limits_{i = 1}^n x_i.  
\]


\item Пусть $x_1, \ldots, x_n$ -- реализация одного элемента выборки из мультиномиального  распределения с неизвестным вектором параметров $\theta$ длины $n$. Допустим, что  априорное распределение  $\theta$ есть распределение Дирихле (см. замечание к задаче \ref{dir}) $\mathrm{Dir}(\alpha_1, \ldots, \alpha_n)$. Тогда апостериорное распределение есть  $\mathrm{Dir}(\alpha_1 + x_1, \ldots, \alpha_n + x_n)$.  

\item Если $x_1, \ldots, x_n$ -- реализация выборки из распределения Пуассона с неизвестным значением среднего $\theta$, 
априорное распределение есть гамма-распределение  $\text{Г}(\alpha, \beta)$, то апостериорное распределение будет иметь параметры:
\[
\alpha + \sum \limits_{i = 1}^n x_i,  \; \; \beta + n.  
\]

\item Если $x_1, \ldots, x_n$ -- реализация выборки из экспоненциального распределения  с неизвестным параметром $\theta$, 
априорное распределение есть гамма-распределение  $\text{Г}(\alpha, \beta)$, то апостериорное распределение будет иметь параметры:
\[
\alpha + n,  \; \; \beta + \sum \limits_{i = 1}^n x_i.  
\]


\item Если $x_1, \ldots, x_n$ -- реализация выборки из нормального распределения  с неизвестным  значением среднего $m$ и заданной мерой точности $r = 1/\sigma^2$, 
априорное распределение $m$ есть  $N(\mu, 1/\tau)$, то апостериорное распределение есть $N(\mu', 1/\tau')$, где:
\[
\mu' = \frac{\tau \mu + n r \overline{x}}{\tau + n r},  \; \; 
\tau' = \tau + n r.  
\]

\item Если $x_1, \ldots, x_n$ -- реализация выборки из нормального распределения  c заданным  значением среднего  $m$ и неизвестной мерой точности $r = 1/\sigma^2$. Пусть априорное распределение  $r$ есть гамма-распределение  $\text{Г}(\alpha, \beta)$. Тогда апостериорное распределение будет иметь параметры:
\[
\alpha + \frac{n}{2},  \; \; \beta + \frac{1}{2} \sum \limits_{i = 1}^n (x_i - m)^2.  
\]

\item Если $x_1, \ldots, x_n$ -- реализация выборки из нормального распределения  c неизвестным значением среднего  $m$ и неизвестной мерой точности $r = 1/\sigma^2$ (см. Рис. \ref{Fig:model.pdf}). Пусть условное априорное распределение $m$ при фиксированном $r$ есть  $N(\mu, 1/(\tau r) )$; априорное распределение  $r$ есть гамма-распределение  $\text{Г}(\alpha, \beta)$. Тогда апостериорное совместное распределение $m$ и $r$ имеет  следующий вид:
\[
m \vert r \in N \left( \frac{\tau \mu + n \overline{x}}{\tau + n}, (\tau + n) r  \right),  
\]
\[
r \in \text{Г} \left( \alpha + \frac{n}{2}, \beta + \frac{1}{2} \sum \limits_{i = 1}^n (x_i - \overline{x})^2 + \frac{1}{2} \frac{\tau n (\overline{x} - \mu)^2}{\tau + n} \right),
\]
\[
p(m, r | X, \mu, \tau, \alpha, \beta) = \const \cdot p(r | \alpha, \beta) p(m | r, \mu, \tau) p(X | m, r) =
\]
\[
p(m| r, X, \mu, \tau) p(r | X, \mu, \tau, \alpha, \beta).
\]

\imgh{80mm}{model.pdf}{Графическая модель генерации выборки из нормального распределения с неизвестными параметрами.}

\end{enumerate} 

\end{problem}



\begin{problem}[Экспоненциальное семейство распределений]
\label{EF}
Многие стандартные распределения принадлежат экспоненциальному семейству (см. замечание), например, нормальное, гамма, бета, Бернулли, Дирихле. 
Пусть с.в. $Y \in EF$. Докажите следующие свойства $EF$:

\[
\varphi_{T(x)}(t) = \Exp e^{\langle t, T(x) \rangle} = \frac{Z(\theta + t)}{Z(\theta)};
\]
\[
\nabla \log Z(\theta) = \Exp T(x);
\quad
\nabla^2 \log Z(\theta) = \cov (T(x), T(x));
\]
для выборки $X_1,\ldots,X_n \in p(x|\theta)$ оценка максимума правдоподобия может быть представлена как 
\[
\nabla \log Z(\widehat{\theta}_{ML}) = \frac{1}{n} \sum_{i=1}^{n} T(X_i);
\]
\[
\mathcal{KL}(\nu_2 \Vert \nu_1) = \nabla d(\nu_2) (\nu_2 - \nu_1) - [d(\nu_2) - d(\nu_1)];
\]
сопряженное распределение (см. задачу \ref{bayes_def}) на параметры $\theta$ может быть вычислено по формуле
\[
p(\theta | \alpha_1, \alpha_2) \propto e^{ \langle \theta, \alpha_1 \rangle - \alpha_2 \log Z(\theta)}.
\]

\end{problem}

\begin{remark}
C.в. $X \in EF$, если ее плотность распределения имеет следующий вид (canonical   parametrization):
\[
p(x | \nu) = h(x)e^{\langle x, \nu \rangle - d(\nu)}. 
\]
Возможно также другое представление:
\[
p(x | \theta) = \frac{h(x)}{Z(\theta)} e^{ \langle \theta, T(x) \rangle}, 
\]
где $T(x)$ -- \textit{достаточная статистика} распределения $p(x | \theta)$.
\end{remark}


\begin{problem}[Процесс Дирихле]
\label{DP}
Рассмотрим графическую модель разделения смеси распределений (mixture model). Предполагается, что каждый элемент выборки $X = \{x_1, \ldots, x_n\}$ имеет скрытую компоненту из вектора $Z = \{z_1, \ldots, z_n\}$, обозначающую компоненту смеси, к которой относится элемент. Процесс генерации выборки устроен таким образом (см. Рис. \ref{Fig:DP.pdf}):
генерируется априорное дискретное распределение компонент смеси $(\pi_1, \ldots, \pi_K) \in \mathrm{Dir}(\alpha/K, \ldots, \alpha/K)$; в таком случае $\PR(z_i = m) = \pi_m$; $j$-я ($j \in \overline{1,K}$) компонента смеси имеет параметры $\theta_j$, которые сгенерированы из произвольного распределения $F(\lambda)$; $x_i$ генерируется из распределения $i$-й компоненты с параметрами $\theta_i$. 

Как правило, в таких задачах требуется оценить скрытые переменные $Z$, $\pi$, $\theta$ по известной выборке $X = \{x_1, \ldots, x_n\}$ и заданных значениях параметров $\alpha$, $\lambda$. 

\imgh{60mm}{DP.pdf}{Dirichlet Process Mixture Model.}

В случае, когда $K$ -- число компонент дискретного распределения не задано, вместо распределения Дирихле для конечного $K$ уместно воспользоваться процессом Дирихле DP, где $K = \infty$ (см. замечание).    

Пусть $\theta_1, \ldots, \theta_{n+1}$ -- независимые случайные величины, у которых распределение является случайной мерой $G \in \mathrm{DP}(H, \alpha)$, $A \subset \Theta$. Докажите, что
\[
\PR(\theta_{n+1} \in A | \theta_1, \ldots, \theta_n) = \frac{1}{n+\alpha}\left(\alpha H(A) + \sum \limits_{i=1}^{n} \delta_{\theta_i}(A) \right),
\]
Пусть $m$ -- число уникальных величин среди $\theta_1, \ldots, \theta_n$. Докажите следующие выражения:
\[
\Exp(m | n) = \alpha (\psi(n + \alpha)  - \psi(\alpha)) \simeq \alpha \ln \left(1 + \frac{n}{\alpha} \right) \quad \text{при} \; \alpha, n \gg 0,
\]
\[
\Var(m | n) = \alpha (\psi(n + \alpha)  - \psi(\alpha)) + \alpha^2 (\psi'(n + \alpha)  - \psi'(\alpha)) \simeq \]\[ \alpha \ln \left(1 + \frac{n}{\alpha} \right) \quad \text{при} \; n > \alpha \gg 0,
\]
где $\psi(x) = \frac{\text{Г'(x)}}{\text{Г(x)}}$ -- digamma функция.
\end{problem}

\begin{remark}
Пусть $H$ -- некоторое распределение с носителем $\Theta$, $\alpha$ -- положительное число. $G$, случайная вероятностная мера (распределение), называется   \textit{процессом Дирихле} $\mathrm{DP}(H, \alpha)$ с базовым распределением $H$ и параметром концентрации $\alpha$, если для любого конечного измеримого разбиения пространства элементарных исходов $\Theta$: $A_1, \ldots, A_r$ выполнено
\[
(G(A_1), \ldots, G(A_r)) \in \mathrm{Dir}(\alpha H(A_1), \ldots, \alpha H(A_r)).
\]

Процесс Дирихле обладает следующими свойствами:
\begin{enumerate}
\item $\forall A \subset \Theta$: $\Exp \left[G(A)\right] = H(A)$;
\item $\forall A \subset \Theta$: $\Var \left[G(A)\right] = \frac{H(A)(1-H(A))}{\alpha+1}$;
\item Пусть $\theta_1, \ldots, \theta_n$ -- независимые случайные величины со случайным распределением $G \in \mathrm{DP}(H, \alpha)$, тогда
\[
\PR(G(A_1), \ldots, G(A_r) | \theta_1, \ldots, \theta_n) \in \mathrm{Dir}(\alpha H(A_1) + n_1, \ldots, \alpha H(A_r) + n_r),
\]
где $n_j = | i: \theta_i \in A_j |$. 
Что также можно записать в другом виде:
\[
G | \theta_1, \ldots, \theta_n \in \mathrm{DP} \left(\alpha + n, \frac{\alpha}{\alpha+n} H + \frac{n}{\alpha+n} \cdot \frac{\sum \delta_{\theta_i}}{n} \right),
\]
где $\frac{\sum \delta_{\theta_i}}{n}$ -- смесь распределений $\delta_{\theta_1}, \ldots, \delta_{\theta_n}$, в котором $\PR(\theta = \theta_i) = 1$ при $\theta \in \delta_{\theta_i}$.   
\end{enumerate}

\end{remark}

\begin{problem}[Stick-breaking construction]
\label{stick}

Рассмотрим альтернативное определение процесса Дирихле (см. замечание к задаче \ref{DP}). Пусть $J_k$ -- $k$-й по счету скачек гамма процесса $g(s)$, $s \in [0,\alpha]$ (см. замечание к задаче \ref{po_dir}), нормированное значение скачка обозначим как $\pi_k = J_k/g(\alpha)$. Введем также последовательность с.в. $v_k: \Theta \to \Theta$ с вероятностной мерой $H: \mathcal{B}(\Theta) \to [0,1]$, где $\mathcal{B}(\Theta)$ -- все измеримые подмножества $\Theta$. Докажите, что случайная мера $G: \mathcal{B}(\Theta) \to [0,1]$, такая что
\[
\forall A \in \mathcal{B}(\Theta): \; G(A) = \sum_{k = 1}^{\infty} \pi_k \delta_{v_k}(A),
\]
является процессом Дирихле $DP(H, \alpha)$.

Для программной реализации процесса Дирихле, как правило, используется генеративная модель ``ломания палки'':
\[
\beta_k \in \mathrm{Beta}(\alpha, 1),
\quad
\pi_k = \beta_k \prod \limits_{l=1}^{k-1} (1-\beta_l), 
\quad
\theta_k \in F(\lambda) = H,
\]
\[
G \in \sum \limits_{k = 1}^{\infty} \pi_k \delta_{\theta_k}.
\]
Докажите, что $G \in \mathrm{DP}(H, \alpha)$.

\end{problem}

\begin{ordre}
См. статью T. Ferguson. A Bayesian analysis of some nonparametric problems. 1973. 
\end{ordre}

\begin{problem}
Для процесса Дирихле $G \in DP(H, \alpha)$ (см. альтернативное определение в задаче \ref{stick}) и измеримой функции $Z: \Theta \to \mathbb{R}$ докажите справедливость следующего утверждения. Если $\int_{\Theta} |Z(\theta)| dH (\theta) < \infty$, то с вероятностью единица 
$\int_{\Theta} |Z(\theta)| dG (\theta) < \infty$ и 
\[
\Exp \int_{\Theta} Z(\theta) dG (\theta)  = \int_{\Theta} Z(\theta) d \Exp G (\theta) =  \int_{\Theta} Z(\theta) dH (\theta).
\]
\end{problem}

\begin{problem}
\label{Fx_bayes}
Требуется по выборке $X_1,\ldots,X_n \in \mathbb{R}$ оценить функцию распределения $F_X(t)$. Априорно предполагается, что $F_X$ соответствует вероятностной мере $H: \mathcal{\mathbb{R}} \to [0,1]$. Предлагается оценивать функцию $F_X$ как 
\[
\widehat{F}_X(t) = \Exp G((-\infty, t) | X_1,\ldots,X_n),
\]
где $G((-\infty, t) | X_1,\ldots,X_n)$ -- с.в., соответствующая условной случайной мере $G | X_1,\ldots,X_n$ (см. замечание к задаче \ref{DP}).
Покажите, что
\[
\widehat{F}_X(t) = \frac{\alpha}{\alpha+n} H((-\infty, t)) + \frac{n}{\alpha+n}\frac{1}{n} \sum_{i=1}^{n} [X_i < t].
\]
Аналогично, для оценки $\Exp X$ предлагается использовать оценку
\[
\widehat{m}_X = \Exp \int_{-\infty}^{+\infty} x dG(x).
\]
Покажите, что
\[
\widehat{m}_X = \frac{\alpha}{\alpha+n} \Exp_{H} X + \frac{n}{\alpha+n}\frac{1}{n} \sum_{i=1}^{n} X_i.
\]

\end{problem}


\begin{problem}
Требуется по выборкам $X_1,\ldots,X_n \in \mathbb{R}$ и $Y_1,\ldots,Y_m \in \mathbb{R}$ оценить вероятность
\[
\bigtriangleup = \PR(X_1 < Y_1) = \int_{-\infty}^{+\infty} F_X(t)dF_Y(t).
\]
Априорно предполагается, что $F_X$ соответствует вероятностной мере $H_X: \mathcal{\mathbb{R}} \to [0,1]$, $F_Y$ -- вероятностной мере $H_Y$. Предлагается оценивать $\bigtriangleup$ как (см. задачу \ref{Fx_bayes}) 
\[
\widehat{\bigtriangleup} = \int_{-\infty}^{+\infty} \widehat{F}_X(t)d\widehat{F}_Y(t).
\]
Покажите, что
\[
\widehat{\bigtriangleup} = p_x p_y \bigtriangleup_0 + p_x(1-p_y)\frac{1}{m} \sum_{i=1}^m H_X(Y_i) + (1-p_x)p_y\frac{1}{n} \sum_{i=1}^n H_Y(X_i) + 
\]
\[
+(1-p_x)(1-p_y) \frac{1}{m n} \sum_{i,j} [X_i < Y_i],
\]
где
\[
p_x = \frac{\alpha_x}{\alpha_x+n},
\quad
p_y = \frac{\alpha_y}{\alpha_y+m},
\quad
\bigtriangleup_0 = \int_{-\infty}^{+\infty} H_X(t)dH_Y(t).
\]
\end{problem}

\begin{problem}[Вариационный вывод для DP]
Для поиска максимума функции правдоподобия $\log p(X | Z, \theta, \pi, \alpha, \lambda)$ в задаче \ref{DP} можно воспользоваться вариационным выводом (см. задачу \ref{varinf} из раздела \ref{CS}), введя более простую параметрическую модель распределения скрытых переменных $q(Z, \theta, \pi)$ и приближая $q(Z, \theta, \pi)$ к $p(Z, \theta, \pi| X, \alpha, \lambda)$ в смысле расстояния $\KL(q \Vert p)$.   

Приведем генеративную модель из задачи \ref{DP}, в которой в качестве процесса Дирихле используется Stick-breaking construction (см. задачу \ref{stick}), а также конкретный пример распределения $F(\lambda) = \mathrm{Dir}(\lambda_1,\ldots,\lambda_M)$: 
\[
\beta_k \in \mathrm{Beta}(\alpha, 1),
\;
\pi_k = \beta_k \prod \limits_{l=1}^{k-1} (1 - \beta_l), 
\;
z_i \in \mathrm{Cat}(\pi),
\;
\theta_k \in \mathrm{Dir}(\lambda_1,\ldots,\lambda_M),
\]
\[
\beta = \Vert \beta_k \Vert_{K \times 1}, \quad
Z = \Vert z_i \Vert_{n \times 1}, \quad
\theta = \Vert \theta_{km} \Vert_{K \times M}.
\]


В качестве примера распределения $x$ в каждой из компонент смеси расмматривается дискретное распределение с параметрами $\theta_k$, т.е.
\[
x_i|z_i \in \mathrm{Cat}(\theta_{z_i}) \quad 
\Leftrightarrow 
\quad
\PR(x_i = m | z_i = k) = \theta_{k m},  
\]
\[
m = \overline{1,M}, \quad k = \overline{1,K}, \quad
i = \overline{1,n}.
\]
Будем искать $q(Z, \beta, \theta)$ в следующем факторизованном виде
\[
q(Z, \beta, \theta) = \prod \limits_{k=1}^K q(\beta_k | \gamma_k) \prod \limits_{k=1}^K q(\theta_k | \tau_k) \prod \limits_{i=1}^n q(z_i | \phi_i),
\]
$\gamma_k$ -- Beta параметры для распределений с.в. $\beta_k$, $\tau_k$ -- параметры распределения $F$ аналогичные $\lambda$, $\phi_i$ -- параметры дискретного распределения с.в. $z_i$, $K \sim \alpha \ln n$. 
Покажите, что параметры $q$, соответствующие минимуму $\KL(q \Vert p)$, удовлетворяет следующей системе уравнений:
\[
\gamma_{k1} = \alpha + \sum_{i = 1}^n q(z_i = k), \quad
\gamma_{k2} = 1 + \sum_{i = 1}^n q(z_i > k), 
\]
\[
\tau_{km} = \lambda_m + \sum_{i = 1}^n [x_i = m] q(z_i = k),
\]
\[
\phi_{ik} \propto \exp \left( \Exp_q \log (\beta_k) + \sum_{l < k} \Exp_q \log (1 - \beta_l) + \sum_{i = 1}^n \Exp_q \log (\theta_{kx_i}) 
\right),
\]
где
$$
q(z_i = k) = \phi_{i,k}, \quad q(z_i > k) = \sum \limits_{j = k+1}^K \phi_{i,j},
$$
$$
\Exp_q \log (\beta_k) = \psi(\gamma_{k1}) - \psi(\gamma_{k1} + \gamma_{k2}), 
$$
$$
\Exp_q \log (1- \beta_k) = \psi(\gamma_{k2}) - \psi(\gamma_{k1} + \gamma_{k2}), 
$$
$$
\Exp_q \log (\theta_{km}) = \psi(\tau_{km}) - \psi\left(\sum_{j}\tau_{kj}\right). 
$$

\end{problem}






\begin{problem}[Тематическое моделирование]
\label{sec:them_modeling}
Вероятностная тематическая модель представляет собой средство для выявления тем в коллекции текстовых документов, описывая каждую тему $z \in Z$ дискретным распределением на множестве терминов $W$, a каждый документ $d \in D$  — дискретным распределением на множестве тем. Предполагается, что порядок терминов в документах не важен, и коллекция является  выборкой из дискретного распределения $\PR(d, w, z)$, где  $\Omega = D \times W \times Z$,  $z$ -- скрытая переменная темы термина $w$ в документе $d$. Коллекция представляется матрицей частот $F  
 = \Vert \widehat{p}_{wd} \Vert_{W\times D}$. Предполагается также, что
\[
\PR(d, w, z) = \PR(w|d,z)\PR(z|d)\PR(d) = \PR(w|z)\PR(z|d)  \PR(d) 
= \phi_{wz} \theta_{zd} \; \PR(d), 
\]  
\[
 \PR(d, w) = \PR(d) \sum_z \phi_{wz} \theta_{zd} . 
\]
Обозначим за $n_{dw}$ число слов $w$ в документе $d$.
Поиск максимума правдоподобия модели для заданной выборки $X = \{(d,w, n_{dw}): d \in D, w \in W\}$  
\[
 L(\Phi, \Theta) = \log  \prod_{d,w} \bigg[\PR(d, w)\bigg]^{n_{dw}}
\] 
 равносилен минимизации расстояния $\mathcal{KL}(F \Vert \Phi \Theta)$, где стохастическое матричное разложение $\Phi \Theta$ не единственно и определено с точностью до невырожденного преобразования $\Phi S^{-1}  S \Theta$.
Ввиду наличия неопределенности в выборе решения, наряду с правдоподобием имеет смысл максимизировать одновременно набор критериев $R_i(\Phi, \Theta)$, выражающих априорные предположения на счет $\phi_z$ и $\theta_d$ как столбцов матриц $\Phi, \Theta^T$ и называемых \textit{регуляризаторами}. Предлагается в качестве решения задачи многокритериальной  оптимизации максимизировать
\[
L(\Phi, \Theta) + R(\Phi, \Theta) = L(\Phi, \Theta) + \sum_i \tau_i R_i(\Phi, \Theta).
\]
Покажите, что шагами EM-алгоритма (см. задачу \ref{em} в разделе \ref{CS}) для решения данной задачи будут

Expectation step: вычислить распределение по темам для каждого термина в документе  
\[
q_{dw}(z) = \frac{\phi_{wz} \theta_{zd}}{\sum_s \phi_{ws} \theta_{sd}};
\]
Maximization step:  найти оптимальные  $\Phi, \Theta$

\[
\phi_{wz} \propto \left( 
\sum_d n_{dw} q_{dw}(z)  + \phi_{wz}\frac{\partial R}{\partial \phi_{wz}}
\right)_{+}, \;
\theta_{zd} \propto \left( 
\sum_w n_{dw} q_{dw}(z)  + \theta_{zd}\frac{\partial R}{\partial \theta_{zd}}
\right)_{+}.
\]
Докажите, что регуляризатор вида
\[
R(\Theta) = - \sum_{d=1}^{D}\sum_{k=1}^{K} \alpha_k \log(\theta_{dk})
\]
соответствует следующему априорному предположению на счет распределения $\theta_d$
\[
\theta_d \in \mathrm{Dir}(\alpha_1 + 1,\ldots,\alpha_K + 1),
\quad
d \in \overline{1,D}, \; K = |T|.
\]
При каком значении $\alpha_1,\ldots,\alpha_K$ данный регуляризатор будет способствовать увеличению количества нулевых компонент $\theta_d$?
\end{problem}

\begin{problem}[PAC-Байесовские неравенства]
Ознакомьтесь с постановкой задачи классификации. 
Для правила (алгоритма) классификации $h \in H$ и набора классифицируемых объектов $X_1, \ldots, X_n$ пусть выборка $X_1^h, \ldots, X_n^h$ есть значения функции потерь  с $X_i^h \in [0,1]$ и $\Exp[X_1^h] = \mu^h$, $\Var[X_1^h]=(\sigma^h)^2$. 

Воспользовавшись неравенством Хефдинга и указанием к задаче, докажите следующее утверждение.
Для любого не зависящего от выборок распределения $p$ на $H$ ($h \sim p$), $\lambda \geq 0$, $\delta \in (0, 1)$ и произвольного априорного распределения $q$ на $H$ с вероятностью не меньше $1 - \delta$  справедливо:
\[\tag{1}
\Exp_q \left( 
\frac{1}{n} \sum_{i=1}^n X_i^h - \mu^h
\right) \leq
\frac{\mathcal{KL} (q \Vert p) + \ln \frac{1}{\delta}  }{n \lambda} + \frac{\lambda}{8}.
 \]
Воспользовавшись неравенством Бернштейна, докажите утверждение, аналогичное предыдущему.
 \[\tag{2}
\Exp_q \left( 
\frac{1}{n} \sum_{i=1}^n X_i^h - \mu^h
\right) \leq
\frac{\mathcal{KL} (q \Vert p) + \ln \frac{1}{\delta}  }{n \lambda} + \lambda (e-2) \Exp_p (\sigma^h)^2.
 \]


\end{problem}

\begin{ordre}
Докажите следующие утверждения.

Для любого измеримого отображения $f : H \to \mathbb{R}$ и любых пар распределений $q$ и $p$ на $H$ справедливо:
\[
\Exp_q f(h) \leq  \mathcal{KL} (q \Vert p)  + \ln \left( \Exp_p e^{f(h)} \right).
\]

Случайная величина $X$ определена на множестве $D$.  Рассмотрим c.в.  $h \sim p$, $h \in H$, не зависящую от $X$,   измеримое отображения $f : H \times D \to \mathbb{R}$. Тогда $\forall \delta \in (0, 1)$ и произвольного распределения $q$ на $H$   с вероятностью не меньше $1 - \delta$ справедливо следующее:
\[
\Exp_q f(h, X) \leq \mathcal{KL} (q \Vert p) + \ln \frac{1}{\delta} + \ln \left( \Exp_p \Exp_X e^{f(h,X)} \right).
\]

\end{ordre}

\begin{remark}
Минимизация по $\lambda$, фигурирующем в выражениях (1) и (2), приводит к зависимости вида $\lambda = \lambda(q)$. Для получения оценки, выполняющейся одновременно для всех $q$, рассмотрим конечный набор значений $\lambda_i$, и для каждого $q$ будем брать $i$, соответствующее минимуму в выражениях (1) и (2). Т.о., если определить $\lambda_i$ как $c^i \lambda_{\min}$, где $\lambda_{\min}$ соответствует $\mathcal{KL} (q \Vert p) = 0$, то получим следующие \textit{PAC-Байесовские неравенства} Хефдинга и Бернштейна.
\[
\Exp_q \left( 
\frac{1}{n} \sum_{i=1}^n X_i^h - \mu^h
\right) \leq
\frac{1+c}{2}
\sqrt{
\frac{\mathcal{KL} (q \Vert p) + \ln \frac{1}{\delta} + \varepsilon(q) }{2 n} 
},
\]
\[
\varepsilon(q) = \frac{\ln 2}{2 \ln c} \left(
1 + \frac{\mathcal{KL} (q \Vert p) }{\ln (1/\delta) }
\right);
\]
при условии
$\mathcal{KL} (q \Vert p) \leq n  (e-2) \Exp_p (\sigma^h)^2  -  \ln \frac{\nu}{\delta}$ выполнено неравенство
\[
\Exp_q \left( 
\frac{1}{n} \sum_{i=1}^n X_i^h - \mu^h
\right) \leq
(1 + c)
\sqrt{
\frac{ (e-2) \Exp_p (\sigma^h)^2 (\mathcal{KL} (q \Vert p) + \ln \frac{\nu}{\delta})  }{n} 
},
 \]
 \[
 \nu  = \left\lceil \frac{1}{\ln c} \ln \left( \frac{(e-2)n}{4\ln(1/\delta)} \right) \right\rceil + 1.
 \]   
   
Данные неравенства позволяют с хорошей точностью оценить отличие среднего риска от эмпирического риска, по сравнению с $VC$-оцениванием. Оптимизация  по $q$ позволяет настраивать гиперпараметры алгоритма классификации.     
\end{remark}





\section{Производящие и характеристические \\ функции}
\label{genF}

Помимо источников литературы, указанных в тексте задач, рекомендуется ознакомиться со следующими книгами:
\begin{itemize}
 \item Айгнер М., Комбинаторная теория. -- М.: Мир, 1982. — 561 с.
 \item Гульден Я., Джексон Д. Перечислительная комбинаторика. -- М.: Наука, 1990 -- 504 с.
 \item Рыбников К.А, Комбинаторный анализ. Очерк истории. -- М.: МехМат, 1996. -- 125с.
 \itemСтенли Р. Перечислительная комбинаторика. --  М.: Мир, 1990. -- 440 с.
\end{itemize}

\begin{problem}(Счастливые билеты)
Трамвайные билеты имеют шестизначные номера. Билет называют счастливым, если 
сумма его первых трех цифр равна сумме трех последних. Вычислите приближенно 
вероятность того, что Вам достанется счастливый билет (предполагается, что 
появление билетов равновероятно).

\end{problem}

\begin{ordre}
Покажите, что число счастливых билетов совпадает с числом билетов, 
сумма цифр у которых равна 27. Запишите \textit{производящую функцию (ПФ)} для 
последовательности $\left\{ {a_k } \right\}_{k=0}^{54} $, где $a_k $ -- число 
билетов с суммой цифр равной $k$. Для нахождения коэффициента ПФ $a_{27} $ 
воспользуйтесь \textit{теоремой Коши} (ТФКП). Для оценки полученного интеграла примените 
\textit{метод стационарной фазы}. (см. книгу Федорюк М.В. Метод перевала. – М.: Наука, 1977, а также \cite{lando}. Последнюю книгу можно также рекомендовать и для решения последующих двух задач).
\end{ordre}

\begin{problem}[Числа Каталана]
\label{sec:katalan}
Пусть в очереди в столовой МФТИ за булочками по цене 10 рублей 50 копеек стоят $2n$ студентов. Пусть у $n$ человек нет пятидесятикопеечной монеты, но есть рублевая, 
а у $n$ человек -- есть монета в 50 копеек. Пусть изначально касса пуста. 
Найдите вероятность события, что никто из студентов в очереди не будет ждать 
свою сдачу. Считайте, что все способы расстановки студентов в очереди равновероятны.
\end{problem}

\begin{ordre}
Задачу можно проинтерпретировать в терминах правильных скобочных структур, описываемых числами Каталана. Если обозначить левую скобку буквой $a$, а правую --- $b$, то можно переписать правильные скобочные структуры в виде «слов» в алфавите $\left\{a,b\right\}$ (\textit{язык Дика}). Несложно показать, что «некоммутативный производящий ряд», перечисляющий слова языка (этот ряд представляет собой просто формальную сумму всех слов языка, вкдючая пустое слово $\lambda$, выписанных в порядке возрастания длины):

\[D(a,b)=\lambda +ab+aabb+abab+aaabbb+aababb+\ldots \] 
удовлетворяет уравнению

\[D(a,b)=\lambda +aD(a,b)bD(a,b).\] 

Перейдите от некоммутативного производящего ряда к обычному, сделав подстановку $a=x$, $b=x$, $\lambda =x^{0} =1$.

Можно искать ПФ языка Дика с помощью \textit{теоремы Лагранжа}, связывающую ее с ПФ подъязыка его неразложимых слов.

\textbf{Определение. }Слово $w=\beta _{1} \ldots \beta _{m} $ языка $L$ называется \textit{неразложимым} в $L$, если никакое его непустое подслово 
$$\beta _{i}\beta _{i+1}\dots \beta _{i+l} , 1\le i,\; i+l\le m,\; l\ge 0$$, отличное от самого слова $w$, не принадлежит языку $L$.

В частности, пустое слово в любом языке, содержащим его, неразложимо.

Несложно проверить, что язык Дика удовлетворяет нижеперечисленным свойствам:

1) пустое слово входит в язык $L$;

2) начало всякого неразложимого слова не совпадает с концом другого или того же самого неразложимого слова;

3) если между любыми двумя буквами любого слова языка $L$ вставить слово языка $L$, то получится слово языка $L$;

4) если из любого слова языка $L$ выкинуть подслово, входящее в язык $L$, то получится слово языка $L$.

Обозначим через $n(y)=n_{0} +n_{1} y+n_{2} y^{2} +\ldots $ ПФ для числа неразложимых слов языка $L$.

\textbf{Теорема.} ПФ $l(x)$ для языка $L$, удовлетворяющего свойствам 1)--4), и ПФ $n(x)$ для подъязыка неразложимых слов в нем связаны между собой \textit{уравнением Лагранжа}

$$l(x)=n\left(xl(x)\right).$$

Приведем уравнение к классическому виду. Положим $xl(x)=\tilde{l}(x)$. Тогда уравнение Лагранжа примет вид:

$$\tilde{l}(x)=xn(\tilde{l}(x)).$$



Неразложимые слова в языке Дика --- это $\lambda $ и $ab$. Отсюда немедленно получаем уравнение $l(x)=1+\left(xl(x)\right)^{2} $ на ПФ для языка Дика.

\textit{Замечание:} Уравнение Лагранжа -- функциональное уравнение, связывающее между собой  ПФ для числа слов в языке и числа неразложимых слов в нем. Оказывается, если одна из функций известна, то оно всегда разрешимо. (см.  указания к следующей задаче об уточнении приведенной здесь теоремы Лагранжа)

\end{ordre}


\begin{problem}[Остовные деревья в полном графе]

Пусть имеется полный граф с $n$ вершинами $\{ 1,2,\ldots ,n\} $. Каждое из $C_n^2=\frac{n(n-1)}{2} $ ребер графа с вероятностью $1/2 $ удаляется. Найдите вероятность того, что полученный после удаления ребер граф будет остовным деревом.

\end{problem}

\begin{ordre}

Обозначим $t_{n} $ -- число остовных деревьев на пронумерованных вершинах $\{ 1,2,\ldots ,n\} $. Ясно, что искомая в задаче вероятность есть $t_n/2^{C_n^2}$.

Выделим одну вершину и посмотрим на те связные компоненты или блоки, на которое разобьется остовное дерево, если проигнорировать все ребра, проходящие через выделенную вершину. Если невыделенные вершины образуют $m$ компонент размеров $k_{1} ,k_{2} ,\ldots ,k_{m} $, то их можно соединить с выделенной вершиной $k_{1} k_{2} \cdot\ldots\cdot k_{m} $ способами.

Такие рассуждения приводят к рекуррентному соотношению

$$ t_{n} = \sum _{m>0}\frac{1}{m!}  \; \sum _{\sum_{i=1}^m k_i = n-1}\left(\begin{array}{c} {n-1} \\ {k_{1} ,k_{2} ,\ldots ,k_{m} } \end{array}\right) \, k_{1} k_{2} \cdot\ldots\cdot k_{m} \, t_{k_{1} } t_{k_{2} }\cdot\ldots\cdot t_{k_{m} }, $$

\noindent при любом $n>1$. 

Теперь обозначим $u_{n} =nt_{n} $, тогда рекуррентное соотношение примет следующий вид:

$$ \frac{u_{n} }{n!} =\sum _{m>0}\frac{1}{m!}  \; \sum _{\sum_{i=1}^m k_i = n-1}\frac{u_{k_{1} } }{k_{1} !} \frac{u_{k_{2} } }{k_{2} !} \cdots \frac{u_{k_{m} } }{k_{m} !}  ,\quad n>1.$$

Обозначим за $U(x)$ \textit{экспоненциальную производящую функцию (ЭПФ)} для последовательности $\left\{u_{n} \right\}$ (то есть $U(x)=\sum _{n=0}^{\infty }\frac{u_{n} }{n!} x^{n}  $). Таким образом,

$$U(x)=xe^{U(x)}.$$

Для нахождения явной формулы для этой последовательности, можно воспользоваться следующим уточнением \textit{теоремы Лагранжа}.

\textbf{Теорема.} Пусть функции $\varphi =\varphi (x)$ ($\varphi (0)=0$) и $\psi =\psi (z)$ связаны между собой уравнением Лагранжа 

$$\varphi (x)=x\psi \left(\varphi (x)\right).$$ 

Тогда коэффициент при $x^{n} $ в функции $\varphi $ равен коэффициенту при $z^{n-1} $ в разложении $\frac{1}{n} \psi ^{n} (z)$: $[x^{n}]\varphi(x) = [z^{n-1}]\frac{1}{n} \psi ^{n} (z)$.

\end{ordre}

\begin{problem}[Задача с марками, \cite{13}]
\label{laplas}
Пусть Вы хотите собрать коллекцию из $N$ 
марок, для этого Вы каждый день покупаете конверт со случайной маркой 
(марки появляются на купленных конвертах равновероятно).
\begin{enumerate}
\item Введем дискретную с.в. $X$, равную номеру впервые купленной Вами 
повторной марки. Найдите математическое ожидание с.в. $X$.

\item Покажите, что распределение случайной величины $X$ имеет асимптотически 
\textit{ распределение Релея } при $n=t\sqrt N $ ($N\gg 1)$:
$$ \PR \left ( {X>t\sqrt N } \right ) \sim e^{-\frac{t^2}{2}},
\quad \PR\left ( {X=t\sqrt N } \right ) \sim \frac{1}{\sqrt N }te^{-\frac{t^2}{2}}.$$
\item Определить математическое ожидание номера купленной Вами марки, которая 
станет недостающей в собранной Вами коллекции марок. 
\end{enumerate}
\end{problem}

\begin{ordre}

а) Покажите, что  
$$\PR \left ( X>n \right ) =\frac{n!}{N^n} \left[ {z^n} \right]\left( {1+z} 
\right)^N=n! \; \left[ {z^n} \right] \left( {1+\frac{z}{N}} \right)^N,$$
$$\Exp X
=
\sum\limits_{n=0}^{\infty} 
n! \; \left[ z^n \right] \left( 1+\frac{z}{N} \right)^N
=
\int\limits_0^{\infty} e^{-t} \left( 1+\frac{t}{N} \right)^N dt. $$

Далее для приближенного вычисления интеграла  
$$\int\limits_0^\infty 
{e^{-t}\left( {1+\frac{t}{N}} \right)^Ndt} =N\int\limits_0^\infty 
{e^{N\left( {\ln (1+u)-u} \right)}du} $$
 воспользуемся \textit{методом Лапласа}:
$$
 \int\limits_0^\infty {f(u)e^{NS(u)}du} \approx f(u_0 ) e^{NS(u_0 
)}\int\limits_{u_0 -\delta }^{u_0 +\delta } {e^{\frac{NS''(u_0 )(u-u_0 
)^2}{2}}} du $$
$$\approx \sqrt {2\pi } \frac{f(u_0 )e^{NS(u_0 )}}{\sqrt {-NS''(u_0 )} }, $$
где $u_0 $ -- единственная точка максимума вещественнозначной функции 
$S(u)$ на полубесконечном интервале $(0, +\infty )$. Основная идея 
асимптотического представления интеграла Лапласа заключается в представлении 
функции $S(u)$ в окрестности точки максимума $u_0 $ в виде ряда Тейлора.

б) Согласно \textit{теореме Коши} (из курса ТФКП): 
$$ \PR(X>n)=n!\left[ {z^n} \right]\left( {1+\frac{z}{N}} 
\right)^N=\frac{1}{n!}\frac{1}{2\pi i }\oint\limits_{\vert z\vert =\rho } 
{\left( {1+\frac{z}{N}} \right)^N\frac{dz}{z^{n+1}}}. $$
Перейдя к полярным координатам, получите, что:
$$
\frac{1}{2i\pi }\oint\limits_{\vert z\vert =\rho } {\left( {1+\frac{z}{N}} 
\right)^N \frac{dz}{z^{n+1}}}  =\frac{1}{2\pi }\int\limits_{-\pi }^\pi {\left( 
{1+\frac{\rho e^{i\theta }}{N}} \right)^N\frac{d\theta }{\rho ^ne^{in\theta 
}}} $$
$$=\frac{1}{2\pi }\int\limits_{-\pi }^\pi {e^{f(\rho e^{i\theta })}d\theta 
} ,
$$
где $f(z)=N\ln \left( {1+\frac{z}{N}} \right)-n\ln z$, $z=\rho 
e^{i\theta }$;

Воспользуйтесь \textit{методом перевала}: разложите в ряд Тейлора функцию $f$ в окрестности седловой 
точки: $$f(\rho e^{i\theta })=f(\rho )-\frac{1}{2}\beta (\rho )\theta 
^2+O(\theta ^3)$$ для $\left| \theta \right|<\delta $ (разложение Тейлора), 
где $$\beta (\rho )=\rho ^2\left. {\left[ {\left( {\frac{d}{dz}} 
\right)^2f(z)} \right]} \right|_{z=\rho }. $$

Замените интеграл по всей окружности $\vert z\vert =\rho $ на интеграл по ее 
части: 
$$
\int\limits_{-\delta }^\delta {e^{f(\rho e^{i\theta })}d\theta } 
\approx e^{f(\rho )}\int\limits_{-\delta }^\delta {e^{-\frac{1}{2}\beta 
(\rho )\theta ^2}d\theta } $$
$$=\frac{e^{f(\rho )}}{\sqrt {\beta (\delta )} 
}\int\limits_{-\delta \sqrt {\beta (\rho )} }^{\delta \sqrt {\beta (\rho )} 
} {e^{-\frac{1}{2}u^2} du}\mathop \to \limits_{\beta (\rho )\to \infty } 
\frac{e^{f(\rho )}}{\sqrt {\beta (\delta )} }\int\limits_{-\infty }^\infty 
{e^{-\frac{1}{2}u^2}du}  $$
$$=\sqrt {2\pi }\frac{e^{f(\rho )}}{\sqrt {\beta 
(\delta )} }.
$$

в) Пусть $Y$ -- дискретная с.в., равная номеру купленной Вами 
марки, которая станет недостающей в собранной Вами коллекции марок.

Тогда $$\PR \left ( Y\le n \right ) =\frac{n!}{N^n}\left[ {z^n} \right](e^z-1)^N=n!\left[ 
{z^n} \right](e^{\frac{z}{N}}-1)^N,$$ а следовательно 
$$\Exp Y=\sum\limits_{n=0}^\infty {n!\left[ {z^n} \right]\left( 
{e^z-(e^{\frac{z}{N}}-1)^N} \right)} =\int\limits_0^\infty {\left[ 
{1-(1-e^{-\frac{t}{N}})^N} \right]} dt.$$
Сделав замену переменных 
$y=1-e^{-\frac{t}{N}}$, получите, что $$\Exp Y=N\left( {1+\frac{1}{2}+\cdots 
+\frac{1}{N}} \right). $$

\end{ordre}

\begin{problem}[Урновая схема, \cite{13}]

Рассмотрим случайное размещение $n$ различных шаров по $m$ различным урнам. 
Пусть случайные величины $MIN$, $MAX$ -- размер наименее или наиболее заполненной 
урны в случайном размещении. Получите функции распределения этих случайных 
величин, а именно $$\PR \left ( MAX\le l\right ) =\frac{n!\left[ {z^n} \right]e_l 
(z)^m}{m^n}=n!\left[ {z^n} \right]e_l \left( {\frac{z}{m}} \right)^m,$$

$$\PR\left ( MIN>l\right ) =n!\left[ {z^n} \right]\left( {e^{\frac{z}{m}}-e_l \left( 
{\frac{z}{m}} \right)} \right)^m,
$$
где $e_l (z)=1+z+\frac{z^2}{2!}+\cdots 
+\frac{z^l}{l!}$.

\end{problem}



\begin{problem}[Задача о циклах в случайной перестановке, \cite{13}]
\\
\begin{enumerate}
\item Найдите математическое ожидание числа циклов длины $r$ в 
случайной перестановке длины $n$.

\item Найдите математическое ожидание числа циклов в 
случайной перестановке длины $n$.

\item (Сто заключенных) 
В коридоре находятся 100 человек, у каждого свой номер (от 1 до 100). Их по одному заводят в комнату, в которой 
находится комод со 100 выдвижными ящиками. В ящики случайным образом 
разложены карточки с номерами (от 1 до 100). Каждому разрешается заглянуть в 
не более чем 50 ящиков. Цель каждого -- определить, в каком ящике находится 
его номер. Общаться и передавать друг другу информацию запрещается. 
Предложите стратегию, которая с вероятностью не меньшей $0.3$ (в 
предположении, что все $100!$ способов распределения карточек по ящикам 
равновероятны) приведет к выигрышу всей команды. Команда выигрывает, если 
все 100 участников верно определили ящик с карточкой своего номера.



\end{enumerate}
\end{problem}
\begin{ordre}
 ЭПФ для последовательностей: 
 
---  
числа циклов длины $n$:
\[C(z)=\sum _{n=0}^{\infty }\frac{(n-1)!}{n!} z^{n}  =\log \frac{1}{1-z}; \]

--- числа множеств на $n$ элементах:
\[S(z)=\sum _{n=0}^{\infty }\frac{1}{n!} z^{n}  =e^{z}; \]

--- числа перестановок на $n$ элементах:
\[P(z)=\sum _{n=0}^{\infty }\frac{n!}{n!} z^{n}  =\frac{1}{1-z}. \]

Перестановка есть не что иное, как совокупность циклов:
\[P(z)=S\left(C(z)\right).\]

Если в такой функции ставить метку (переменную $u$) для циклов длины $r$, получим, что ``двойная''  ЭПФ, перечисляющая число циклов длины $r$ в перестановке длины $n$:

$$\exp \left\{(u-1)\frac{z^{r} }{r} +\log \frac{1}{1-z} \right\}=\frac{\exp \left\{(u-1)\frac{z^{r} }{r} \right\}}{1-z} .$$

Заметим, что $\left[z^{n} \right]\left. \frac{\partial }{\partial u} \left(\frac{\exp \left\{(u-1)\frac{z^{r} }{r} \right\}}{1-z} \right)\right|_{u=1} $ -- среднее (математическое ожидание) число циклов длины $r$ в перестановке длины $n$.

\textbf{Стратегия для ста заключенных} Каждый человек вначале открывает ящик под номером, равным его собственному номеру, затем -- под номером, который указан на карточке, лежащей в ящике, и т.д. Среднее число циклов длины $r$ в случайной 
перестановке -- $1/r$. Тогда среднее число циклов длины большей $n/2$
есть $\sum\limits_{k=n \mathord{\left/ {\vphantom {n 2}} \right. 
\kern-\nulldelimiterspace} 2}^n {\frac{1}{k}} $. Это и есть вероятность 
существования цикла длины большей $n/2$. Поэтому вероятность успеха команды -- 
есть $1-\sum\limits_{k=51}^{100} {\frac{1}{k}} \approx 0,31$ (для сравнения, если 
произвольно открывать ящики, то вероятность успеха будет 
$2^{-100}\approx 10^{-30}$). В случае, когда карточки 
разложены не случайным образом, то следует сделать случайной нумерацию 
ящиков, и далее следовать описанной выше стратегии.
\end{ordre}

\begin{problem}[Задача о беспорядках, \cite{lando}]
\label{permloop}
Группа из $n$ фанатов выигрывающей футбольной команды на радостях 
подбрасывают в воздух свои шляпы. Шляпы возвращаются в случайном порядке ---
по одной к каждому болельщику. Какова вероятность того, что никому из 
фанатов не вернется своя шляпа? Найдите математическое ожидание и дисперсию числа шляп, вернувшихся совим хозяевам.
\end{problem}

\begin{ordre}
Формально задача сводится к подсчету числа беспорядков $d_{n} $ на множестве из $n$ элементов (перестановка $\pi $ элементов множества $\left\{1,2,\ldots ,n\right\}$ называется  \textit{беспорядком}, если $\pi (k)\ne k$ ни при каких $k=1,\ldots ,n$).

Получите ЭПФ для числа беспорядков: $$D(x)=\sum _{n=0}^{\infty }\frac{d_{n} }{n!} x^{n}  =\frac{e^{-x} }{1-x}. $$

Пусть $d_{n,k} $ -- число перестановок на множестве из $n$ элементов, оставляющих на месте ровно $k$ элементов (то есть число неподвижных точек равно $k$), тогда $d_{n,0} =d_{n} $. Более того, $d_{n,k} =C_{n}^{k} d_{n-k} $. Из правила суммы, несложно получить:

$$ n!=\sum _{k=0}^{n}d_{n,k}  =\sum _{k=0}^{n}C_{n}^{k} d_{n-k}  =\sum _{k=0}^{n}C_{n}^{n-k} d_{n-k}  =\sum _{k=0}^{n}C_{n}^{k} d_{k}.$$

То есть получилась биномиальная свертка двух рядов:
$$
\sum _{n=0}^{\infty }\frac{n!}{n!} x^{n}  =\sum _{n=0}^{\infty }\frac{d_{n} }{n!} x^{n}  \sum _{n=0}^{\infty }\frac{1}{n!} x^{n}.
$$
Для нахождения среднего и дисперсии числа шляп, вернувшихся к своим хозяевам, удобно воспользоваться указанием к предыдущей задаче о подсчете среднего и дисперсии циклов длины 1 в слуайной перестановке длины $n$.

Сравните подход с помощью производящих функций для этой задачи с вашими решениями задач \ref{sec:clubok} и \ref{sec:latters} из раздела стандартных задач (раздел \ref{standart}).
\end{ordre}


\begin{problem}[Ожерелья]

Найдите вероятность того, что случайная раскраска ожерелья из $n$ бусин в $k$ различных цветов имеет ровно $m\le n$ бусин первого цвета. Ожерелья, получающиеся одно из другого с помощью плоского поворота, считаются эквивалентными. Положите $n=7$, $k=2$, $m=3$.

\end{problem}

\begin{remark}
Ознакомиться с классическими работами по теории перечисления (Пойа, Дж.К. Рота) можно по книгам

Перечислительные задачи комбинаторного анализа / Сборник переводов под редакцией Г. П. Гаврилова.  — М.: Мир, 1979. -- 362 с.

Краснов М.Л. и др. Вся высшая математика. Учебник. Т.7. -- М.: КомКнига. -- 2006. -- 208 с.
\end{remark}

\begin{ordre}

Сопоставим каждой раскраске функцию $f$ как отображение из множества пронумерованных бусин в нераскрашенном ожерелье в множество пронумерованных бусин в раскрашенном ожерелье. Основной нюанс в решении комбинаторных задач такого типа заключается в том, что некоторые функции (раскраски) оказываются эквивалентными, так как получаются одна из другой с помощью некоторой подстановки (в данном случае задающей поворот в плоскости): $f_{1} \sim f_{2} $, если найдется подстановка $g\in G$, что $f_{1} (g)=f_{2} $, т.е. $\forall i=1,\ldots ,n$   $f_{1} \left(g\left(d_{i} \right)\right)=f_{2} \left(d_{i} \right).$

Для решения этой задачи воспользуемся подходом Пойа.
\textit{Цикловым индексом подстановки} называют одночлен
\[x_{1} ^{k_{1} } x_{2} ^{k_{2} } \ldots x_{n} ^{k_{n} } ,\] 
где $\left(k_{1} ,\; k_{2} ,\; \ldots ,\; k_{n} \right)$ -- тип подстановки, т.е. подстановка представима в виде $k_{1} $ цикла длины 1, $k_{2} $ циклов длины 2,  и т.д.

\textit{Цикловым индексом группы подстановок} $G$ называют среднее арифметическое цикловых индексов ее элементов:
\[P_{G} (x_{1} ,x_{2} ,\ldots ,x_{n} )=\frac{1}{|G|} \sum _{g\in G}x_{1} ^{k_{1} } x_{2} ^{k_{2} } \ldots x_{n} ^{k_{n} }  .\] 
Покажите, что цикловой индекс группы поворотов $C_n$ равен: $$P_{G} (x_{1} ,x_{2} ,\ldots ,x_{n} )=\frac{1}{n} \sum _{j=1}^{n}\left(x_{\frac{n}{(n,j)} } \right)^{(n,j)}  ,$$ где $(n,j)$ --- наибольший общий делитель $n$ и $j$. Каждому цвету $r_{i} $ $i=1,\ldots ,k$ придадим некоторый \textit{вес} $w(r_{i} )$. \textit{Весом функции} $f$ назовем произведения весов полученной раскраски:
\[W(f)=\prod _{i=1}^{n}w(f(d_{i} )).\] 
Ясно, что эквивалентные функции имеют одинаковый вес:
\[f_{1} \sim f_{2} \Rightarrow W\left(f_{1} \right)=W\left(f_{2} \right).\] 
Весом класса эквивалентности называется вес любой функции из этого класса; если $F$ -- класс эквивалентности и $f\in F$, то $W(F)=W(f)$. Заметим, что и у неэквивалентных функций могут совпадать веса.

\textbf{Теорема Пойа (1937 г.)}
Сумма весов классов эквивалентности рана
$$\sum _{F}W(F) =P_{G} \left(\sum _{i=1}^{k}w(r_{i} ) ,\; \sum _{i=1}^{k}w^{2} (r_{i} ) ,\; \sum _{i=1}^{k}w^{3} (r_{i} ) ,\; \ldots ,\sum _{i=1}^{k}w^{n} (r_{i} ) \right),$$ где $P_{G} $ -- цикловой индекс группы подстановок $G$.

\end{ordre}

\textbf{Следствие}. 
Число классов эквивалентности равно 
$$P_{G} \left(k,\; k,\; k,\; \ldots ,k\right).$$ 

Согласно следствию теоремы Пойа число различных ожерелий равно $$P_{G} (k,k,\ldots ,k)=\frac{1}{n} \sum _{j=1}^{n}k^{(n,j)}  =\frac{1}{n} \sum _{s|n}k^{s} \phi \left(\frac{n}{s} \right), $$ где $\phi \left(l\right)$ -- число взаимно простых делителей числа $l$, не превосходящих $l$ (\textit{функция Эйлера}), обозначение $s|n$ -- $s$ является делителем $n$.

Для подсчета числа ожерелий с ровно $m$ бусинами первого цвета, положите вес этого цвета $w(r_{1} )=x$, веса остальных цветов -- единицей. В ПФ (полученной согласно теореме Пойа) возьмите коэффициент при $x^{m} $.



\begin{problem}[Изомеры органических молекул]

Рассмотрим математическую модель органической молекулы: в центре тетраэдра поместим атом углерода $C$, в вершинах тетраэдра равновероятно помещаются метил $\left(CH_{3} \right)$, этил $\left(C_{2} H_{5} \right)$, водород $\left(H\right)$ и хлор $\left(Cl\right)$. Найдите вероятность того, что случайная молекула заданной структуры окажется метаном $CH_{4} $. 

\end{problem}


\begin{ordre}
 Воспользуйтесь подходом Пойа, описанным в указаниях предыдущей задачи. Покажите, что цикловой индекс группы вращения тетраэдра $P_{G} (x_{1} ,x_{2} ,x_{3} ,x_{4} )=\frac{1}{12} \left(8x_{1} x_{3} +3x_{2} ^{2} +x_{1} ^{4} \right).$
\end{ordre}

\begin{comment}
\begin{problem}[Marching cubes]
Сколько различных раскрасок вершин кубов в 2 цвета можно получить с учетом вращений и отражений?
\end{problem}

\begin{remark}
Сокращение числа раскрасок используется для сжатого хранения и визуализации изоповерхности на трехмерном скалярном поле при помощи алгоритма marching cubes: вместо хранения массива цветов всех вершин трехмерной сетки достаточно хранить номер раскраски каждого куба сетки (вершина считается окрашенной в положительный цвет, если значение поля в ней не меньше значения поля на изоповерхности; в отрицательный цвет, если значение поля в ней меньше значения поля на изоповерхности).  
\end{remark}

\end{comment}

\begin{remark}
Для решения последующих трех задач рекомендуется ознакомится подходом Егорычева вычисления комбинаторных сумм, изложенных в  книгах

Леонтьев В.К. Избранные задачи комбинаторного анализа. -- М.: МГТУ, -- 2001. -- 184 с.

Егорычев Г.П. Интегральное представление и вычисление комбинаторных сумм. -- Новосибирск, Наука,  1977. -- 284 с.
\end{remark}

\begin{problem}
Обозначим через $E^n$ -- множество бинарных последовательностей длины $n$, или множество вершин 
единичного $n$-мерного куба, а через $E_k^n $ -- $k$-ый слой куба $E^n$, то есть 
подмножество точек $E^n$, имеющих ровно $k$ единичных координат. Пусть 
$X= \langle x, y \rangle$ -- случайная величина, где $ 
x \in E_p^n $, $ y \in E_q^n $ -- независимые и равномерно 
распределенные на $E_p^n $ и $E_q^n $ соответственно векторы. Обозначим 
через $a_{p,q} (k)=\PR\left\{ {X=k} \right\}$. Доказать следующие утверждения:

\begin{enumerate}
\item $\sum\limits_{k=0}^n {a_{p,q} (k)z^k} =\frac{1}{2\pi i}C_n^p 
\oint\limits_{\left| u \right|=\rho } 
{\frac{(1+zu)^p(1+u)^{n-p}}{u^{q+1}}du}; $

\item $a_{p,q} (k)=\frac{C_p^k C_{n-p}^{q-k} }{C_n^q };$

\item $\Exp X=\frac{pq}{n};$

\item $\Var X=\frac{pq}{n(n-1)}\left( {n+\frac{pq}{n}-(p+q)} \right) .$
\end{enumerate}

\end{problem}

\begin{problem}
Пусть $\xi$ -- случайная величина, равномерно распределенная на множестве всех пар векторов 
$(x,y)\in \{ 0,1\}^n\otimes \{ 0,1\}^n$, равная $ \langle x,y \rangle=\sum\limits_{k=1}^{n} x_k y_k$. Найдите: 
$$
{\mathbb P}(\xi=k), \; {\mathbb E}\,\xi, \; \Var \xi . 
$$
\end{problem}

\begin{problem}
Пусть $\xi$ -- с.в., равномерно распределенная на множестве бинарных матриц ($\{0,1\}^{m\times n}$) порядка 
$m\times n$ и равная числу нулевых столбцов матрицы. Доказать, что 
\begin{enumerate}
\item $P_k(m,n)={\mathbb P}(\xi=k)=C_n^k\cdot \left.\bigl( 2^m -1\bigr)^{n-k}\right/ 2^{m\cdot n}$; 

\item ${\mathbb E}\xi=\left. n\right/2^m$; 
\item если $2^m-1=\alpha\cdot n$, где $\alpha$ не зависит от $n$, то 
$$
\lim\limits_{n\to\infty} P_k(m,n)=e^{-\lambda}\, \frac{\lambda^k}{k!} ,\; \text{ где } \lambda=\alpha^{-1} .
$$
\end{enumerate}
\end{problem}

\begin{comment}
\begin{ordre}

Рассмотрим следующие случайные величины: 
$$
\xi_i=\begin{cases}
1, & \text{ $i$-й столбец нулевой}, \\
0, & \text{ иначе }.
\end{cases} 
$$
Тогда $\xi_i\in\Be$, $\xi=\xi_1+\ldots +\xi_n$. 

Имеет место следующее мультипликативное свойство:
$$
\psi_{\xi}(z)=\bigl[\psi_{\xi_i}(z)\bigr]^n
$$

\end{ordre}
\end{comment}

\begin{problem}
Может ли функция $\varphi(t)=\begin{cases}1,\quad t\in[-T,T]\\
0,\quad t\notin[-T,T] \end{cases}$ -- 
быть \textit {характеристической функцией} (х.ф.) некоторой с.в.? Изменится ли ответ, если <<чуть-чуть>> размазать (сгладить) разрывы функции 
$\varphi(t)$ в точках $t=\pm T$? 
\end{problem}

\begin{comment}
\begin{ordre}
Характеристическая функция обладает следующими свойствами:
\begin{fixme}
Iterate the properties 
\end{fixme}
\end{ordre}
\end{comment}


\begin{problem}
Будут ли функции~$\cos(t^2)$, $\exp(-t^4)$, $\arcsin((\cos t)/2) / \arcsin(1/2)$ характеристическими для каких-нибудь случайных величин?
\end{problem}

\begin{problem}

Докажите, что выпуклая линейная комбинация х.ф. есть х.ф.  ``смеси'' слагаемых. Т.е. если $\phi_{k}(t) = \Exp \exp (i\xi_k t)$, $\sum_k a_k = 1$, $a_k \geq 0$,  то $\sum_k a_k \phi_{k}(t)$ есть х.ф. случайной величины
\[
\xi = \sum_k [z = k] \xi_k,
\]
где $z$ не зависящая от $\{\xi_k\}$ c.в. с распределением $\PR(z=k) = a_k$.
\end{problem}


\begin{problem}
Случайные величины $X_1, X_2$ независимы и имеют распределение Коши $\text{Ca}(x_1, d_1), \; \text{Ca}(x_2, d_2)$, где плотность распределения  $\text{Ca}(x_1, d_1)$ определяется как    
\[
f(x) = \frac{d}{\pi(d^2 + (x-x_1)^2)}.
\]
Докажите, что распределением $X_1 + X_2$ является $\text{Ca}(x_1+x_2, d_1+d_2)$.
\end{problem}

\begin{ordre}
Убедитесь, что характеристическая функция $X_1$ имеет вид $\varphi(t) = e^{ix_1t - d|t|}$.
\end{ordre}

\begin{problem}(Квадратичные формы)
$A$ -- симметричная ($n \times n$)-матрица, $X \in \N(0, S)$, покажите что для квадратичной формы $Q = X^TAX$ справедливы равенства 
\[
\Exp e^{tQ} =\det (I - 2t A S)^{-1/2},
\] 
\[
\Exp Q = \text{tr}(AS), \quad \Var Q = 2 \text{tr}((AS)^2),
\]
\[
\Exp [ (WX +m)^{T} A (WX +m) ] = m^T A m + \text{tr}(W^TAWS),
\]
где $W$ и $m$ -- неслучайные матрицы.
\end{problem}


\begin{problem}
Пусть $\varphi_{\xi}$~-- х.ф. абсолютно непрерывной случайной величины~$\xi$ с плотностью~$p_{\xi}$. Рассмотрим $f_1 =\text{Re}(\varphi_{\xi})$ и~$f_2 = \text{Im}(\varphi_{\xi}) $. Существуют ли случайные величины $\eta_1,\,\eta_2$, для которых $f_1,\,f_2$~являются их х.ф.? 
\end{problem}



\begin{problem}
Реализуем $m$~раз схему из $n$~испытаний Бернулли с вероятностью успеха~$p$. Считаем, что все реализации схем взаимно независимы. На выходе получим $m$~случайных векторов $x_1,\,\dots,\,x_m$ с координатами $0$~и~$1$. Некоторые из этих векторов могут совпадать. Скажем, что векторы $x_i,\,x_j,\,x_k$ образуют прямой угол с вершиной в~$x_k$, если~$vx_i-x_k,\,x_j - x_k) = 0$ (помимо обычных прямых углов, под это определение попадают и <<вырожденные>>, т.\,е. образованные совпадающими векторами). Найдите математическое ожидание числа прямых углов во множестве~$\{x_1,\,\dots,\,x_m\}$.
\end{problem}






\begin{problem} 
Найдите вероятность того, что пара случайно выбранных из $\{ 0,1\}^n$ векторов является ортогональной 
\begin{enumerate}
\item над полем $\mathbb{F}_2=\{ 0,1\}$; 

\item над полем действительных чисел. 
\end{enumerate}
\end{problem}

\begin{comment}
\begin{ordre}
a) Пусть 
$$
\xi_{x,y}=\begin{cases}
1, &\text{ если } (x,y)=0,\\
0, &\text{ если } (x,y)=1.
\end{cases}
$$
Для искомой вероятности тогда имеем 
$$
P=\frac{1}{2^n\cdot 2^n}\sum\limits_{x,y}\xi_{x,y}
$$
\end{ordre}
\end{comment}


\begin{problem}[об оценке хвостов]
Пусть $\Psi (z)=\Exp z^X$ --- ПФ случайной величины  $X$. Докажите, что
\[
{\rm\PR}(X\le r)\le x^{-r}\Psi (x),\mbox{ для }0<x\le 1;
\]
\[
{\rm \PR}(X\ge r)\le x^{-r}\Psi (x),\mbox{ для }x\ge 1.
\]
\begin{remark}
Для решения этой и последующих шести задач можно рекомендовать книгу \cite{29}
\end{remark}
\end{problem}









\begin{problem}[Игра У. Пенни, 1969]
Авдотья и Евлампий играют в игру: они 
бросают монету до тех пор, пока не встретится РРО или РОО. Если первой 
появится последовательность РРО, выигрывает Авдотья, если РОО -- Евлампий. Будет ли игра честной (одинаковы ли вероятности выигрыша у обоих игроков)?
\end{problem}

\begin{ordre}

Справедливы следующие равенства:
\[
\mbox{1+N(Р+О)=N+А+В},
\]
\[
\mbox{NРРО=А},
\]
\[
\mbox{NРОО=В+АО},
\]
где $A$ -- конфигурации, выигрышные для Авдотьи, $B$ --  конфигурации, выигрышные для Евлампия, $N$ -- конфигурации последовательностей, для которых ни один из игроков не выиграл.
Заменяя Р и О на $\frac{1}{2}$, найдите значения $A$ и $B$ - вероятностей выигрыша Авдотьи и Евлампия.
\end{ordre}


\begin{problem}
Теперь трое игроков: Авдотья, Евлампий и Компьютер. Играют пока 
не выпадет одна из следующих последовательностей: А=РРОР, В=РОРР, С=ОРРР. Каковы шансы каждого выиграть?
\end{problem}




\begin{problem}
Рассмотрим следующую игру: первый игрок называет одну из восьми комбинаций ООО, ООР, ОРО, РОО, ОРР, РОР, РРО, РРР, второй игрок называет другую, потом они бросают симметричную монету до тех пор, пока в последовательности орлов (О) и решек (Р) -результатов бросания -- не появится одна из выбранных комбинаций. Тот, чья комбинация появится, выиграл. Будет ли игра честной? Найдите для каждой комбинации, выбранной первым игроком, выгодные комбинации второго игрока.
\end{problem}

\begin{ordre} 
Для решения этой задачи, см. также \ref{book12}.Например, если первый игрок выбрал ОРО, то выбираю ООР, второй игрок выигрывает с вероятностью ${2\mathord{\left/ {\vphantom {2 3}} \right. \kern-\nulldelimiterspace} 3} $. Действительно, пусть $p_{OO} $, $p_{OP} $, $p_{PO} $, $p_{PP} $ вероятности выигрыша первого игрока при последних значениях ОО, ОР, РО, РР (в предположении, что раньше никто не выиграл). Их можно найти из системы уравнений:

\[\left\{\begin{array}{l} {p_{OO} =\frac{1}{2} p_{OO} +\frac{1}{2} \cdot 0,} \\ \\ {p_{OP} =\frac{1}{2} \cdot 1+\frac{1}{2} p_{PP} ,} \\ \\ {p_{PO} =\frac{1}{2} p_{OO} +\frac{1}{2} p_{OP} ,} \\ \\ {p_{PP} =\frac{1}{2} p_{PO} +\frac{1}{2} p_{PP} .} \end{array}\right. \] 

Откуда получаем:

\[p_{OO} =0, \; p_{OP} =\frac{2}{3}, \;  p_{PO} =\frac{1}{3}, \; p_{PP} =\frac{1}{3} .\] 

То есть общая вероятность выигрыша первого игрока равна ${1\mathord{\left/ {\vphantom {1 3}} \right. \kern-\nulldelimiterspace} 3}$.
\end{ordre}

\begin{problem}
В вершине пятиугольника $ABCDE$ 
находится яблоко, а на расстоянии двух ребер, в вершине $C$, находится 
червяк. Каждый день червяк переползает в одну из двух соседних вершин с 
равной вероятностью. Так, через один день червяк окажется в вершине $B$ или 
$D$ с вероятностью $1 \mathord{\left/ {\vphantom {1 2}} \right. 
\kern-\nulldelimiterspace} 2$. По прошествии двух дней червяк может снова 
оказаться в $C$, поскольку он не запоминает своих предыдущих положений. 
Достигнув вершины $A$, червячок останавливается пообедать.

\begin{enumerate}
\item Чему равны математическое ожидание и дисперсия числа дней прошедших до обеда?
\item Какую оценку дает неравенство Чебышёва для вероятности $p$ того, что это число дней будет 100 или больше?
\item Что позволяют сказать о величине $p$-оценки из задачи ``об оценке хвостов''.
\end{enumerate}
\end{problem}


\begin{problem}
Пять человек стоят в вершинах пятиугольника $ABCDE$ и 
бросают друг другу диски фрисби. У них имеется два диска, которые в 
начальный момент находятся в соседних вершинах. В очередной момент времени 
диски бросают либо налево, либо направо с одинаковой вероятностью. Процесс 
продолжается до тех пор, пока обе тарелки не окажутся в одной вершине.

\begin{enumerate}
\item Найдите математическое ожидание и дисперсию числа пар бросков.
\item Найдите ``замкнутое'' выражение через числа Фибоначчи для вероятности того, что игра продлится более 100 шагов.
\end{enumerate}
\end{problem}


\begin{problem}
Обобщите предыдущую задачу на случай $m$-угольника и найдите 
математическое ожидание и дисперсию числа пар бросков до столкновения 
дисков. Докажите, что если $m$ нечетно, то ПФ случайной величины  для числа бросаний 
представима в следующем виде:\footnote{ Воспользуйтесь подстановкой $z=1 
\mathord{\left/ {\vphantom {1 {\cos ^2\theta }}} \right. 
\kern-\nulldelimiterspace} {\cos ^2\theta }$.}
\[
G_m (z)=\prod\limits_{k=1}^{(m-1)/2} {\frac{p_k z}{1-q_k z}} ,
\]
где
\[
p_k =\sin ^2\frac{(2k-1)\pi }{2m},
\quad
q_k =\cos ^2\frac{(2k-1)\pi }{2m}.
\]
\end{problem}




\begin{problem}[Загадочный случайный суп]
Студент, решивший отобедать в 
столовой, может обнаружить в своей тарелке с супом случайное число $N$ 
инородных частиц со средним $\mu $ и конечной дисперсией. С 
вероятностью $p$ выбранная частица является мухой, иначе это таракан; типы 
разных частиц независимы. Пусть $F$ -- количество мух и $S$ -- количество тараканов.

\begin{enumerate}
\item Покажите, что ПФ с.в. $F$ удовлетворяет равенству
\[
\psi _F (s)=\psi _N (ps+1-p).
\]

\item Предположим, что с.в. $N$ имеет пуассоновское 
распределение с параметром $\mu$:  $N\in \Po\left( \mu \right))$. Покажите, что $F$ имеет пуассоновское распределение с 
параметром $p\mu $, а с.в. $F$ и $S$ независимы. Покажите, что
\[
\psi _N (s)=\psi _N^2 \Bigl( {\frac{1}{2}(1+s)} \Bigr).
\]
\end{enumerate}

\end{problem}


\begin{problem}[Соболевский А.Н.]
На каждой упаковке овсянки печатается купон одного из $k$ различных цветов. Считая, что при отдельной покупке купон каждого цвета может встретиться с равной вероятностью и различные покупки независимы, требуется найти производящую функцию распределения вероятности, математическое ожидание и дисперсию минимального числа упаковок, которые придется купить для того, чтобы собрать купоны всех $k$ цветов.
\end{problem}

\begin{ordre}
Воспользуйтесь вспомогательными вероятностями: $p(n)$ -- вероятность разделить $n$ одинаковых объектов на $k$ групп так, что в каждой группе будет не менее  одного объекта, $f(m)$ -- вероятность того, что минимальное число купленных  упаковок равно $m$.
\[
p(n) = \sum_{m = k}^{n} f(m), 
\]
\[
P(z) = \sum_{n=k}^{\infty} p(n) z^n = \sum_{n=k}^{\infty} \sum_{m = k}^{n} f(m) z^n = 
\sum_{n=m}^{\infty} \sum_{m = k}^{\infty} f(m) z^m z^{n-m} = 
\]
\[=
F(z) \frac{1}{1-z}.
\]    

См. также задачу \ref{laplas} из этого раздела и указание к ней.
\end{ordre}


\begin{problem}[Задача о предельной форме диаграмм Юнга]
\label{limung}
\textit{Разбиением} $\lambda $ натурального числа $n$ называется набор натуральных 
чисел $(\lambda _1 ,\ldots ,\lambda _N )$, для которого $\lambda _1 \ge 
\ldots \ge \lambda _N >0$ и $\lambda _1 +\ldots +\lambda _N =n$. 
\textit{Диаграммой Юнга} разбиения $\lambda =(\lambda _1 ,\ldots ,\lambda _N )$ называется 
подмножество первого квадранта плоскости, состоящее из единичных 
квадратиков. Квадратики размещаются по строкам, выровненным по левому краю, 
причем число квадратиков в $i$-ой строке равно $\lambda _i $.
Множество всех диаграмм Юнга, 
соответствующих натуральному числу $n$ (или иначе множество всех разбиений 
натурального числа $n$) обозначим через $\mathcal P_n $.
Введем равномерную меру $\mu _n $ на $\mathcal P_n$, то есть $$\mu _n (\lambda 
)=\frac{1}{p(n)}$$ для всех $\lambda \in \mathcal P_n$, где  $p(n)=|\mathcal P_n|$ -- число разбиений натурального $n$ (число диаграмм Юнга, 
соответствующих натуральному $n$).

Пусть $r_k (\lambda )$ -- число слагаемых в разбиении $\lambda$, равных $k$ (иначе говоря, выполнено равенство $n = \sum_k k r_k (\lambda )$). Число слагаемых в разбиении $\lambda$, больших или равных $\lceil t \rceil$ (где $t$ -- неотрицательное действительное чтсло), обозначим через $\phi _\lambda (t)=\sum\limits_{k\ge t} {r_k (\lambda )} $. Заметим, что $\phi _\lambda (t)$, $t \ge 0$ -- 
ступенчатая функция, непрерывная справа, замыкание внутренности подграфика которой и есть диаграмма Юнга, соответствующая разбиению $\lambda $. Сделаем шкалирование (скейлинг) функции $\phi _\lambda (t)$ с множителем $a$: 
$$\tilde {\phi }_\lambda (t)=\frac{a}{n}\sum\limits_{k\ge at} {r_k (\lambda 
)} =\frac{a}{n}\phi _\lambda (at).$$ Заметим, что после шкалирования 
$$\int\limits_0^\infty {\tilde {\phi }_\lambda (t)dt} =1.$$

\textbf{Теорема (А.М.Вершик).} Пусть $a_n =\sqrt n $, тогда для любого $\varepsilon >0$ 
существует $n_\varepsilon $, что для любого $n>n_\varepsilon $
\[
\mu _n \left\{ {\lambda :\quad \mathop {\sup }\limits_t \left| {\tilde {\phi 
}_\lambda (t)-C(t)} \right|<\varepsilon } \right\}>1-\varepsilon ,
\]
где $C(t)=\int\limits_t^\infty {\frac{e^{-\sqrt {\varsigma (2)} 
u}}{1-e^{-\sqrt {\varsigma (2)} u}}du} =-\frac{\sqrt 6 }{\pi }\ln 
(1-e^{-\left( {\pi \mathord{\left/ {\vphantom {\pi {\sqrt 6 }}} \right. 
\kern-\nulldelimiterspace} {\sqrt 6 }} \right)t})$.

Другими словами, типичная (в смысле равномерной меры) диаграмма Юнга после 
шкалирования $\tilde {\phi }_\lambda (t)=\frac{1}{\sqrt n }\sum\limits_{k\ge 
\sqrt n t} {r_k (\lambda )} $ с ростом $n$ имеет \textit {предельную форму (limit shape)}, заданную 
с помощью $C(t)$ или в более симметричной форме: $e^{-\frac{\pi }{\sqrt 6 
}x}+e^{-\frac{\pi }{\sqrt 6 }y}=1$.

Интерпритация этой задачи с точки зрения статистической физики следующая.  $\mathcal P_n $ -- множество всех состояний системы (Бозе-частиц) 
с постоянной энергией $n$, каждое состояние $\lambda \in \mathcal P_n $ однозначно 
характеризуется набором значений $r_k (\lambda )$ (чисел заполнений $k$-го 
уровня энергии). Заметим, что у Бозе-частиц нет ограничений на значения $r_k 
(\lambda )$, в отличие от Ферми-частиц, где $r_k (\lambda )\le 1$. $\mathcal P_n $ в 
статистической физике называется каноническим ансамблем (равномерная 
мера $\mu _n $ соответствует тому, что все состояния равновероятны). Теорема утверждает, что  шкалированная функция распределения по уровням энергии у системы Бозе-частиц (с полной энергией $n$),  имеет в пределе вид функции $C(t)$ (иначе, теорема утверждает, что существует предельное распределение энергий частиц, позволяющее делать заключение типа: какая предельная
доля общей энергии приходится на частицы с данными энергиями).

\begin{remark}
С детальным доказательством теоремы можно ознакомиться в статье А.М. Вершика ``Статистическая механика комбинаторных разбиений и их предельные конфигурации'', Функциональный анализ и его приложения. -- 1996. -- Т. 30. -- Вып. 2. -- С. 19-39.
\end{remark}

Согласно приведенной выше статьи, далее описаны только основные шаги и идеи доказательства. Читателю требуется с учетом приведенных ниже указаниям обосновать результат о предельной форме диаграмм Юнга.

\begin{ordre} Заметим, что для обоснования существования предельной формы шкалированных диаграмм Юнга, необходимо для начала проверить выполнимость следующих пределов:
$$
\mathop {\lim }\limits_{n\to 
\infty } \mathbb E_{n}\tilde {\phi }_\lambda (t) = C(t),
$$
$$\mathop {\lim }\limits_{n\to 
\infty }\mathbb E_{n } \left[ {\left( {\tilde {\phi }_\lambda (t)-C(t)} 
\right)^2} \right]=0,$$
что на самомо деле не так просто.

Основная идея обоснования утверждения теоремы заключается, во-первых, в переходе к пространству всех диаграмм Юнга $\mathcal P=\bigcup\limits_n {\mathcal P_n} $ с семейством (по $x$) мер $\mu _x $:
$$\mu _x (\lambda )=\frac{x^{\sum\limits_i {\lambda _i } 
}}{F(x)}=\frac{x^{n(\lambda )}}{F(x)}, \text{ где } F(x)=\sum\limits_{n=0}^\infty 
{p(n)x^n},$$
которые в свою очередь индуцируют на пространстве $\mathcal P_n $ исходную равномерную меру  $\mu _n $,
во-вторых, обосновании утверждения указанной теоремы для семейства введенных мер $\mu _x $ при $x \to 1-0$ (что существенно проще, см. далее). Представление меры $\mu_x$ как выпуклой комбинации мер $\mu_n$ (см. утверждение 3 ниже) позволяют делать заключение, что тот же предел (найденный для мер  $\mu _x $ при $x \to 1-0$) будет и для мер $\mu _n $ при $n \to \infty$

С точки зрения статистической физики $\mathcal P$ 
соответствует (большому) каноническому ансамблю с мерой Гиббса $\mu _x 
(\lambda )=\frac{x^{n(\lambda )}}{F(x)}$, где $F(x)=\sum\limits_\lambda 
{x^{\sum\limits_i {\lambda _i } }} =\sum\limits_{n=0}^\infty 
{p(n)x^{n(\lambda )}} $ -- статистическая сумма. Покажите, что для 
статистической суммы $F(x)$ (или иначе производящей функции последовательности $p(n)$) справедлива формула Эйлера: 
$$F(x)=\sum\limits_{n=0}^\infty {p(n)x^n} =\prod\limits_{k=1}^\infty 
{(1-x^k)^{-1}}.$$

Переход от малого к большому каноническому ансамблю упрощает решение задачи в силу  мультипликативности мер $\mu _x $ (см. утверждение 2 
ниже).

Проверьте справедливость следующих \textbf{утверждений}:

1) $\left. {\mu _x (\lambda )} \right|_{\mathcal P_n } \equiv \frac{\mu _x (\lambda 
)}{\mu _x (\mathcal P_n)}=\mu _n (\lambda )$, то есть мера на $\mathcal P_n $, индуцированная 
мерой $\mu _x $ совпадает с равномерной мерой $\mu _n $.

2) $r_1 (\lambda ),\;r_2 (\lambda ),\ldots$ как случайные величины на $\mathcal P$ 
независимы относительно мер $\mu _x $.

\textbf{Указание}: Проверьте, что $\mu _x \left( {\lambda :\;r_k (\lambda 
)=s} \right)=x^{ks}(1-x^k)$.

3) Мера $\mu _x $ является выпуклой комбинацией мер $\mu _n $.

\textbf{Указание}: $\mu _x =\sum\limits_{n=0}^\infty 
{\frac{x^np(n)}{F(x)}\mu _n } $.

Согласно описанной выше схемы (идеи) доказательства, проверим выполниммость
$$\mathop {\lim }\limits_{x\to 1-0} \mu_{x} \left\{ {\lambda :\quad \mathop {\sup }\limits_t \left| {\tilde {\phi }_\lambda (t)-C(t)} 
\right|<\varepsilon } \right\}=1.$$
Для этого нужно, во-первых,
$$
\mathop {\lim }\limits_{x\to 1-0} \mathbb E_{x}\tilde {\phi }_\lambda (t)=C(t),
$$
во-вторых,
$$\mathop {\lim }\limits_{x\to 
1-0 } \mathbb E_{x } \left[ {\left( {\tilde {\phi }_\lambda (t)-C(t)} 
\right)^2} \right]=0.$$
Заметим, что в отличие от случая, когда усреднение берется по мере $\mu_n$, $n(\lambda)$ -- случайная величина, воспользоваться линейностью математического ожидания, например, при вычислении $\mathbb E_x [n(\lambda)]^{-1/2} \sum_{k\ge t\sqrt {n(\lambda)}} r_k(\lambda)$ нельзя. Преодоление этой трудности аналогично выбору перевального контура в методе Лапласа: вместо предела $x\to 1-0$ выберем последовательность $x_n$ такую, что с одной стороны $x_n \to 1-0$ при ${n\to\infty}$, с другой стороны мера $\mu_{x_n}$ концентрируется на  $\mathcal P_n$. Для этого будем оптимизировать (по $x$) вероятность того, что случайное (по мере $\mu_x$) разбиение дает фиксированное исходное значение $n$:
$$ \mu_{x} \left\{ \sum_k kr_k(\lambda) = n \right \} = p(n)\frac{x^n}{F(x)}\to\max(x), $$
так что  $x=x_n$ является корнем уравнения
$x\frac{d}{dx}\left[ {\ln 
F(x)} \right]=n$ (покажите это).
Последнее уравнение можно интерпретировать немного иначе, а именно выбирать значение $x=x_n$ из совпадения математического ожидания (по мере $\mu_x$) случайной величины $n(\lambda)$ с исходным значением $n$:
$$ \mathbb E_x \left[ {\sum\limits_k {kr_k (\lambda )} } 
\right]=n. $$
Покажите, что последнее также эквивалентно $x\frac{d}{dx}\left[ {\ln 
F(x)} \right]=n$. Это уравнение имеет единственное решение $x_n \in (0,1)$ и 
при этом $\mathop {\lim }\limits_{n\to \infty } \mathbb E_{x_n } \left[ {\left( 
{\frac{n(\lambda )}{n}-1} \right)^2} \right]=0$. Таким образом, получаем 
$$
\mathbb E_{x_n} \tilde {\phi }_\lambda (t) = \frac{1}{\sqrt n} \sum_{k\ge t\sqtr n} \mathbb E_{x_n} r_k,
$$
что является интегральной суммой для $C(t)$.

Вырожденность предельной формы обосновывается \[\mathop {\lim }\limits_{n\to 
\infty } E_{x_n } \left[ {\left( {\tilde {\phi }_\lambda (t)-C(t)} 
\right)^2} \right]=0.\]

Таким образом, обосновывается
$$\mathop {\lim }\limits_{n\to \infty} \mu _{x_n} \left\{ {\lambda :\quad 
\mathop {\sup }\limits_t \left| {\tilde {\phi }_\lambda (t)-C(t)} 
\right|<\varepsilon } \right\}=1.$$

\begin{comment}
Обозначим 
через $p(n)$ число разбиений натурального $n$ (число диаграмм Юнга, 
соответствующих натуральному $n)$. Условимся считать $p(0)=1$. 

Покажите, что производящая функция последовательности $p(n)$ равна 
$$F(x)=\sum\limits_{n=0}^\infty {p(n)x^n} =\prod\limits_{k=1}^\infty 
{(1-x^k)^{-1}}\text{ (формула Эйлера).}$$
Множество всех диаграмм Юнга, 
соответствующих натуральному числу $n$ (или иначе множество всех разбиений 
натурального числа $n)$ обозначим через $\mathcal P_n $. 

Обозначим через $\phi _\lambda (t)=\sum\limits_{k\ge t} {r_k (\lambda )} $ 
ступенчатую функцию, непрерывную справа, замыкание внутренности подграфика 
которой и есть диаграмма Юнга, соответствующая разбиению $\lambda $. Сделаем 
шкалирование (скейлинг) функции $\phi _\lambda (t)$ с множителем $a$: 
$\tilde {\phi }_\lambda (t)=\frac{a}{n}\sum\limits_{k\ge at} {r_k (\lambda 
)} =\frac{a}{n}\phi _\lambda (at)$. Заметим, что после шкалирования 
$\int\limits_0^\infty {\tilde {\phi }_\lambda (t)dt} =1$. Преобразование, 
сопоставляющее каждой диаграмме $\lambda \in \mathcal P_n $ шкалированную ее границу 
$\tilde {\phi }_\lambda (t)$, обозначим через $\tau _a $. Введем ${\cal D}$ 
{\-} пространство мер с двумя возможными атомами (``зарядами'') в $0$ и 
$\infty $ и непрерывной мерой на $(0,\infty )$, заданной плотностью $p\in 
L_+^1 ({\rm R}_+)$ (неотрицательная монотонно невозрастающая функция), то есть 
$$
{\cal D} = \{ (\alpha_0, \alpha_{\infty}, p(\cdot) ): \alpha_0, \alpha_{\infty} \ge 0, p\in 
L_+^1 ({\rm R}_+), \alpha_0 + \alpha_{\infty} + \int\limits_0^{\infty} {p(t)dt} = 1 \}.
$$
${\cal D}$ является компактом.

Заметим, что образ $\mathcal P_n$ при отображении $\tau _a $ лежит в ${\cal D}$ (это 
меры с нулевыми зарядами $\alpha _0 ,\alpha _\infty $и кусочно-постоянной 
плотностью).

Введем равномерную меру $\mu _n $ на $\mathcal P_n$, то есть $\mu _n (\lambda 
)=\frac{1}{p(n)}$ для всех $\lambda \in \mathcal P_n$.

Обозначим через $\tau _a^\ast (\mu _n )$ образ равномерной меры $\mu _n $ на 
$\mathcal P_n$ при отображении $\tau _a $, то есть $(\tau _a^\ast \mu _n )(A)=\mu _n 
(\tau _a^{-1} A)$, для любого $A\in {\cal D}$.

Следующая теорема говорит о том, что существует последовательность $a_n $, 
что последовательность образов равномерной меры на множестве диаграмм $\tau 
_{a_n }^\ast (\mu _n )$ слабо сходится к вырожденной мере с нетривиальным 
носителем: $\delta _C $, где $C\in {\cal D}$ (не является зарядом) 
называется \textit{предельной формой} (limit shape).

\textbf{Теорема.} Пусть $a_n =\sqrt n $, тогда для любого $\varepsilon >0$ 
существует $n_\varepsilon $, что для любого $n>n_\varepsilon $
\[
\mu _n \left\{ {\lambda :\quad \mathop {\sup }\limits_t \left| {\tilde {\phi 
}_\lambda (t)-C(t)} \right|<\varepsilon } \right\}>1-\varepsilon ,
\]
где $C(t)=\int\limits_t^\infty {\frac{e^{-\sqrt {\varsigma (2)} 
u}}{1-e^{-\sqrt {\varsigma (2)} u}}du} =-\frac{\sqrt 6 }{\pi }\ln 
(1-e^{-\left( {\pi \mathord{\left/ {\vphantom {\pi {\sqrt 6 }}} \right. 
\kern-\nulldelimiterspace} {\sqrt 6 }} \right)t})$.

Другими словами, типичная (в смысле равномерной меры) диаграмма Юнга после 
шкалирования $\tilde {\phi }_\lambda (t)=\frac{1}{\sqrt n }\sum\limits_{k\ge 
\sqrt n t} {r_k (\lambda )} $ с ростом $n$ имеет предельную форму, заданную 
с помощью $C(t)$ или в более симметричной форме: $e^{-\frac{\pi }{\sqrt 6 
}x}+e^{-\frac{\pi }{\sqrt 6 }y}=1$.

Доказательство этой теоремы непростое. Ограничимся обоснованием лишь 
основных идей.

Рассмотрим множество всех диаграмм Юнга $\mathcal P=\bigcup\limits_n {\mathcal P_n} $. Введем 
семейство (по $x)$ мер $\mu _x $ на множестве $\mathcal P$: 

$$\mu _x (\lambda )=\frac{x^{\sum\limits_i {\lambda _i } 
}}{F(x)}=\frac{x^{n(\lambda )}}{F(x)}, \text{ где } F(x)=\sum\limits_{n=0}^\infty 
{p(n)x^n}.$$

Идея перехода к множеству $\mathcal P$ и мерам $\mu _x $ является стандартным в 
статистической физике. Представим $n(\lambda )=\sum\limits_k {kr_k (\lambda 
)} $, где $r_k (\lambda )$ -- число слагаемых в разбиении $\lambda $, 
равных $k$. Тогда $\mathcal P_n $ -- множество всех состояний системы (Бозе-частиц) 
с постоянной энергией $n$, каждое состояние $\lambda \in \mathcal P_n $ однозначно 
характеризуется набором значений $r_k (\lambda )$ (чисел заполнений $k$-го 
уровня энергии). Заметим, что у Бозе-частиц нет ограничений на значения $r_k 
(\lambda )$, в отличие от Ферми-частиц, где $r_k (\lambda )\le 1$. $\mathcal P_n $ в 
статистической физике называется микроканоническим ансамблем (равномерная 
мера $\mu _n $ соответствует тому, что все состояния равновероятны). $\mathcal P$ 
соответствует (большому) каноническому ансамблю с мерой Гиббса $\mu _x 
(\lambda )=\frac{x^{n(\lambda )}}{F(x)}$, где $F(x)=\sum\limits_\lambda 
{x^{\sum\limits_i {\lambda _i } }} =\sum\limits_{n=0}^\infty 
{p(n)x^{n(\lambda )}} $ -- статистическая сумма. Указанная выше теорема 
дает предельное распределение энергий Бозе-частиц.

Переход от малого к большому каноническому ансамблю упрощает решение задачи 
(прежде всего в силу мультипликативности мер $\mu _x $, см. утверждение 2 
ниже).

Проверьте справедливость следующих \textbf{утверждений}:

1) $\left. {\mu _x (\lambda )} \right|_{\mathcal P_n } \equiv \frac{\mu _x (\lambda 
)}{\mu _x (\mathcal P_n)}=\mu _n (\lambda )$, то есть мера на $\mathcal P_n $, индуцированная 
мерой $\mu _x $ совпадает с равномерной мерой $\mu _n $.

2) $r_1 (\lambda ),\;r_2 (\lambda ),\ldots$ как случайные величины на $\mathcal P$ 
независимы относительно мер $\mu _x $.

Проверьте, что $\mu _x \left( {\lambda :\;r_k (\lambda 
)=s} \right)=x^{ks}(1-x^k)$.

3) Мера $\mu _x $ является выпуклой комбинацией мер $\mu _n $.

 $\mu _x =\sum\limits_{n=0}^\infty 
{\frac{x^np(n)}{F(x)}\mu _n } $.

Описанные выше утверждения, а также ``тауберовые'' аргументы позволяют 
доказать слабую эквивалентность малого и большого канонического ансамбля:
\[
\mathop {\lim }\limits_{x\to 1\!-\!0} \tau^* \mu _x =\mathop {\lim 
}\limits_{n\to \infty } \tau^* \mu _n .
\]
Таким образом, теперь доказательство теоремы сводится к обоснованию того, 
что для шкалированых диаграмм Юнга с множителем $\sqrt n $ существует 
вырожденный слабый предел (предельная конфигурация $C)$:
\[
w\!-\!\lim\limits_{x\to 1\!-\!0} \mu _x =\delta _C ,
\]
то есть $$\mathop {\lim }\limits_{x\to 1} \mu _x \left\{ {\lambda :\quad 
\mathop {\sup }\limits_t \left| {\tilde {\phi }_\lambda (t)-C(t)} 
\right|<\varepsilon } \right\}=1.$$
Для этого нужно найти средние мер $\mu _x $ на ${\cal D}$ и оценить дисперсию при $x\to 1\!-\!0$. 
Аналогично выбору перевала в методе Лапласа, выберем последовательность $x_n 
$ так, что для $x=x_n  \quad \mathbb E_x \left[ {\sum\limits_k {kr_k (\lambda )} } 
\right]=n$.
Покажите, что последнее уравнение эквивалентно $x\frac{d}{dx}\left[ {\ln 
F(x)} \right]=n$. Это уравнение имеет единственное решение $x_n \in (0,1)$ и 
при этом $\mathop {\lim }\limits_{n\to \infty } \mathbb E_{x_n } \left[ {\left( 
{\frac{n(\lambda )}{n}-1} \right)^2} \right]=1$. Здесь использовалось 
свойство мультипликативности мер $\mu _x $, другими словами, независимость 
чисел заполнения.
Осталось заметить, что $$\mathop {\lim }\limits_{n\to \infty } \mathbb E_{x_n } 
\left[ {\frac{1}{\sqrt n }\sum\limits_{k\ge t\sqrt n } {kr_k (\lambda )} } 
\right]=C(t).$$
Вырожденность предельной меры обосновывается \[\mathop {\lim }\limits_{n\to 
\infty } E_{x_n } \left[ {\left( {\tilde {\phi }_\lambda (t)-C(t)} 
\right)^2} \right]=0.\]
\end{comment}

Стоит отметить, что предельную форму диаграмм Юнга можно найти и с помощью 
следующих рассуждений.
Пусть в первом квадранте на целочисленной решетке отмечены пары точек $(x_i 
,y_i )$, так что
\[
0=x_1 <x_2 <\ldots <x_k ,\quad y_1 <y_2 <\ldots <y_k =0.
\]
Число возможных диаграмм Юнга, продолженная граница которых проходит через 
заданные пары точек, приближенно выражается 
\[
\exp \left\{ {\sum\limits_i {\left( {\Delta _i x+\Delta _i y} \right)H\left( 
{\frac{\Delta _i x}{\Delta _i x+\Delta _i y}} \right)} } \right\},
\]
где $\Delta _i x=x_{i+1} -x_i $, $\Delta _i y=y_{i+1} -y_i $, $H(p)=-p\ln 
p-(1-p)\ln (1-p)$.

Соедините указанные точки ломаной, продлив ее осями за концевые точки. 
Перейдите в новую систему координат поворотом старой против часовой стрелки 
на $45^o$ и сделав растяжение в $\sqrt 2 $ раза. Ломаная теперь задается как 
график функции $u=f(t)$. Теперь с ростом числа точек
\[
\exp \left\{ {\sum\limits_i {\left( {\Delta _i x+\Delta _i y} \right)H\left( 
{\frac{\Delta _i x}{\Delta _i x+\Delta _i y}} \right)} } \right\}\to \exp 
\left\{ {\int {H\left( {{f}'(t)} \right)dt} } \right\}.
\]
Для нахождения предельной формы нужно найти минимум функционала (с 
энтропийным лагранжианом) при ограничениях $f(t)\ge \vert t\vert $, $\vert 
{f}'\vert \le 1$, $\int {\left( {f(t)-\vert t\vert } \right)dt} =1$. Затем 
вернуться в исходную систему координат.
\end{ordre}
\end{problem}

\begin{problem}\DStar(Статистика выпуклых ломаных \cite{7})
\label{convcurve}
\imgh{50mm}{curves1.pdf}{к задаче "Статистика выпуклых ломаных" ($c=3$).}
Рассмотрим на плоскости выпуклую ломаную $\Gamma $, выходящую из нуля, у которой вершины являются целыми точками и угол наклона каждого звена неотрицателен и не превосходит ${\pi \mathord{\left/ {\vphantom {\pi  2}} \right. \kern-\nulldelimiterspace} 2} $ ($\Gamma $ -- график кусочно-линейной функции $x_2 = \Gamma(x_1)$). Выпуклость понимается в том смысле, что наклон последовательных звеньев ломаной строго возрастает. Пространство всех таких ломаных обозначим через $\Pi $, а через $\Pi (m_{1} ,m_{2} )$ -- множество ломаных, оканчивающихся в точке $(m_{1} ,m_{2} )$. 

Введем равномерную меру на пространстве $\Pi (m_{1} ,m_{2} )$:
\[P_{m_{1} ,m_{2} } (\Gamma )=\frac{1}{\left|\Pi (m_{1} ,m_{2} )\right|} =\frac{1}{N(m_{1} ,m_{2} )} .\] 

Рассмотрим детерминированную кривую $t_2 = L_{c}(t_1) $, $t_1 \in [0,1]$, заданную уравнением $\left(ct_{1} +t_{2} \right)^{2} =4ct_{2} $ (см. Рис. \ref{Fig:curves1.pdf}). Заметим, что точки $(0;0)$ и $(1;c)$ являются соответственно левым и правым концом кривой, также для этой кривой справедливо \[\left. \left({dt_{2} \mathord{\left/ {\vphantom {dt_{2}  dt_{1} }} \right. \kern-\nulldelimiterspace} dt_{1} } \right)\right|_{t_{1} =0} =0, \; \left. \left({dt_{2} \mathord{\left/ {\vphantom {dt_{2}  dt_{1} }} \right. \kern-\nulldelimiterspace} dt_{1} } \right)\right|_{t_{1} =1} =\infty .\] 

Покажите справедливость закона больших чисел для выпуклых ломаных $\Gamma\in \Pi(m_{1} ,m_{2} )$:

\textbf{Теорема (Я.Г.Синай).} Для любого $\delta > 0$ справедливо
$$
\lim_{m_{1},m_2 \to \infty, \frac{m_2}{m_1}\to c} P_{m_{1} ,m_{2} } \left \{ \max_{t_1\in[0;1]}\left | \frac{1}{m_1}\Gamma(m_1 t_1) - L_c(t_1) \right |\le \delta \right\} = 1.
$$


%Возьмем произвольное $\delta >0$ и построим полосу 
%\[U_{\delta ,m_{1} } =\left\{x=(x_{1} ,x_{2} ):\; \left|x_{2} -m_{1} L_{c} \left({x_{1} \mathord{\left/ {\vphantom {x_{1}  m_{1} }} \right. \kern-\nulldelimiterspace} m_{1} } \right)\right|\le \delta m_{1} \right\}.\] 
%Покажите, что для любого $\delta >0$ вероятность $P_{m_{1} ,m_{2} } \left\{\Gamma \subset U_{\delta ,m_{1} } \right\}$ стремится к 1 при $m_{1} \to \infty $, $m_{2} \to \infty $, $\frac{m_2}{m_1}\to c$. 

Иными словами, в результате масштабного преобразования (скейлинга) $t_{1} ={x_{1} \mathord{\left/ {\vphantom {x_{1}  m_{1} }} \right. \kern-\nulldelimiterspace} m_{1} } $ и $t_{2} ={x_{2} \mathord{\left/ {\vphantom {x_{2}  m_{1} }} \right. \kern-\nulldelimiterspace} m_{1} } $ форма случайных ломаных становится при $m_{1} \to \infty $ детерминированной.

\end{problem}

\begin{ordre} 

Согласно работе Я.Г.Синай "Вероятностный подход к анализу статистики выпуклых ломаных". Функциональный анализ и его приложения. -- 1994. -- Т. 28, Вып. 2. -- С. 41-48. (либо см. книгу \cite{7}) рассмотрим подход к исследованию асимптотических вероятностных свойств выпуклых ломаных, основанный на понятиях статистической физики -- микроканонического и большого канонического ансамблей. Согласно такой точки зрения, распределение вероятностей $P_{m_{1} ,m_{2}}$ на $\Pi(m_{1} ,m_{2})$ вводится как условное распределение, индуцированное подходящей мерой $Q$ (см. ниже), заданной на
пространстве $\Pi$. При этом те или иные свойства ломаных (в частности, закон
больших чисел) устанавливаются сначала по отношению к $Q$, а затем переносятся
на случай $P_{m_{1} ,m_{2}}$  с помощью соответствующей локальной предельной теоремы.

Сначала рассмотрим множество $X$ всех пар взаимно простых целых чисел, и пусть $C_{0} (X)$ -- пространство финитных функций на $X$  с неотрицательными целыми значениями. Нетрудно понять, что каждой такой функции естественным образом отвечает некоторая выпуклая (конечнозвенная) ломаная, и наоборот. Введем на $C_{0} (X)$ мультипликативную статистику $Q = Q_{z_1,z_2}$ (параметры $z_1$ и $z_2$, $0<z_1,z_2<1$, определим ниже) -- распределение случайного поля $\nu = \nu(\cdot)$  на $X$ с независимыми значениями при разных $x = (x_1, x_2)\in X$  и распределением:
\[Q_{z_{1} ,z_{2} } (\nu )=\prod _{x=(x_{1} ,x_{2} )\in X}\left[\left(z_{1}^{x_{1} } z_{2}^{x_{2} } \right)^{\nu (x)} \left(1-z_{1}^{x_{1} } z_{2}^{x_{2} } \right)\right] =\] 

\[\prod _{x=(x_{1} ,x_{2} )\in X}\left(z_{1}^{x_{1} } z_{2}^{x_{2} } \right)^{\nu (x)}  \prod _{x=(x_{1} ,x_{2} )\in X}\left(1-z_{1}^{x_{1} } z_{2}^{x_{2} } \right) .\] 
То есть каждая случайная величина $\nu (x)$ имеет геометрическое распределение с параметрами $z_{1}^{x_{1} } z_{2}^{x_{2} } $.

Проверьте, что на пространстве $P_{m_{1} ,m_{2}}$ введенная выше мера $Q_{z_1,z_2}$ индуцирует распределение, не зависящее от параметров $z_1$ и $z_2$, в данном случае равномерное:
\[Q_{z_{1} ,z_{2} } \left(\nu |\Pi (m_{1} ,m_{2} )\right)=\frac{1}{N(m_{1} ,m_{2} )} =P_{m_{1} ,m_{2} } (\nu ),\]

С точки зрения статистической физики, распределение $Q_{z_1,z_2}$ играет роль большого канонического распределения Гиббса, а распределение $P_{m_{1} ,m_{2}}$ -- микроканонического распределения. Если какое-либо событие по отношению к распределению $Q_{z_{1} ,z_{2} } $ имеет малую вероятность, гораздо меньшую, чем вероятность $Q_{z_{1} ,z_{2} } \left(\Pi (m_{1} ,m_{2} )\right)$, то оно имеет и малую вероятность по отношению к распределению $P_{m_{1} ,m_{2} } $. Основная идея заключается в том, чтоб подобрать такие значения параметров $z_{1} ,z_{2} \in \left[0,1\right]$, при которых вероятность $Q_{z_{1} ,z_{2} } \left(\Pi (m_{1} ,m_{2} )\right)$ приняла бы возможно большее значение, то есть чтобы распределение $Q_{z_{1} ,z_{2} }$ концентрировалось на пространстве $\Pi(m_{1} ,m_{2})$:
$$Q_{z_{1} ,z_{2} } \left(\Pi (m_{1} ,m_{2} )\right) = Q_{z_{1} ,z_{2} } \left \{ \sum _{x\in X}\nu (x)x_{1} = m_1, \sum _{x\in X}\nu (x)x_{2} = m_2 \right \} =$$
$$\frac{N(m_1,m_2)z_1^{m_1}z_2^{m_2}}{\prod_{x=(x_{1} ,x_{2} )\in X}(1-z_{1}^{x_{1} } z_{2}^{x_{2} })^{-1}}\to\max(z_{1} ,z_{2} \in \left[0,1\right]).$$
Покажите, что такой выбор согласуется с интуитивным выбором из фиксации "в среднем" по мере $Q_{z_{1} ,z_{2} }$ правого конца случайной ломаной:
\[\begin{array}{l} {\Exp_{z_{1} ,z_{2} } \left(\sum _{x\in X}\nu (x)x_{1}  \right)=\sum _{x\in X}\frac{x_{1}z_1^{x_1}z_2^{x_2}}{1-z_1^{x_1}z_2^{x_2}}=m_{1} ,} \\ {\Exp_{z_{1} ,z_{2} } \left(\sum _{x\in X}\nu (x)x_{2}  \right)=\sum _{x\in X}\frac{x_{2}z_1^{x_1}z_2^{x_2}}{1-z_1^{x_1}z_2^{x_2}}=m_{2}, } \end{array}\] 
\noindent 
что в свою очередь эквивалентно системе уравнений (см. предыдущую задачу):
\[\begin{array}{l} {z_1 \frac{d}{dz_1} \ln \left (\prod_{x=(x_{1} ,x_{2} )\in X}}(1-z_{1}^{x_{1} } z_{2}^{x_{2} })^{-1}\right ) =m_{1} ,} \\ {\z_2 \frac{d}{dz_2} \ln \left ( \prod_{x=(x_{1} ,x_{2} )\in X}}(1-z_{1}^{x_{1} } z_{2}^{x_{2} })^{-1} \right)=m_{2}. } \end{array}\] 

Детали нахождения $z_1$ и $z_2$ см. в приведенной выше литературе. Здесь приводится только результат, что $z_{1}$ и $z_{2} $ равны соответственно 
\[1-\left(\frac{\zeta (3)c}{\zeta (2)m_{1} } \right)^{\frac{1}{3}} (1+o(1)), \; \; 1-\left(\frac{\zeta (3)}{\zeta (2)c^{2} m_{1} } \right)^{\frac{1}{3}}  (1+o(1)).\] 
Дзета-функция Римана появляется в связи с тем, что рассматриваются только пары взаимно простых чисел -- пространство $X$ (см. задачу \ref{sec:z_func_riman} из раздела \ref{hard}).


\begin{comment}
\textbf{Лемма.} Пространство $\Pi $ находится во взаимно однозначном соответствии с пространством $C_{0} (X)$ финитных функций $\nu (x)$, заданных на $X$ - множестве пар взаимно простых положительных чисел, включая пары $(0,1)$, $(1,0)$, и принимающих целые неотрицательные значения. 

Введем на пространстве $\Pi $ (с точки зрения статистической механики представляющего большой канонический ансамбль) распределение вероятностей $Q_{z_{1} ,z_{2} } $, зависящее от двух параметров $z_{1} ,z_{2} \in \left[0;1\right]$, и играющее роль большого канонического распределения Гиббса. Или в силу леммы зададим $Q_{z_{1} ,z_{2} } $ на пространстве функций $C_{0} (X)$:

\[Q_{z_{1} ,z_{2} } (\nu )=\prod _{x=(x_{1} ,x_{2} )\in X}\left[\left(z_{1}^{x_{1} } z_{2}^{x_{2} } \right)^{\nu (x)} \left(1-z_{1}^{x_{1} } z_{2}^{x_{2} } \right)\right] =\] 

\[\prod _{x=(x_{1} ,x_{2} )\in X}\left(z_{1}^{x_{1} } z_{2}^{x_{2} } \right)^{\nu (x)}  \prod _{x=(x_{1} ,x_{2} )\in X}\left(1-z_{1}^{x_{1} } z_{2}^{x_{2} } \right) .\] 

Так каждая случайная величина $\nu (x)$ имеет по отношению к распределению $Q_{z_{1} ,z_{2} } $ геометрическое распределение с параметрами $z_{1}^{x_{1} } z_{2}^{x_{2} } $ и при разных $x$ случайные величины $\nu (x)$ взаимно независимы. Несложно вывести, что

\[Q_{z_{1} ,z_{2} } \left(\Pi (m_{1} ,m_{2} )\right)=\sum _{}Q_{z_{1} ,z_{2} } (\nu ) =N(m_{1} ,m_{2} )z_{1}^{m_{1} } z_{2}^{m_{2} } \prod _{x\in X}\left(1-z_{1}^{x_{1} } z_{2}^{x_{2} } \right) ,\] 
\noindent где суммирование ведется по тем $\nu $, для которых $\sum _{x\in X}\nu (x)x_{1}  =m_{1} $ и $\sum _{x\in X}\nu (x)x_{2}  =m_{2} $.

При этом

\[Q_{z_{1} ,z_{2} } \left(\nu |\Pi (m_{1} ,m_{2} )\right)=\frac{1}{N(m_{1} ,m_{2} )} =P_{m_{1} ,m_{2} } (\nu ),\] 
\noindent то есть равномерное распределение по отношению к распределению $Q_{z_{1} ,z_{2} } $ является микроканоническим распределением. Отсюда будет следовать, что если какое-либо событие по отношению к распределению $Q_{z_{1} ,z_{2} } $ имеет малую вероятность, гораздо меньшую, чем вероятность $Q_{z_{1} ,z_{2} } \left(\Pi (m_{1} ,m_{2} )\right)$, то оно имеет и малую вероятность по отношению к распределению $P_{m_{1} ,m_{2} } $. Основная идея заключается в том, чтоб подобрать такие значения параметров $z_{1} ,z_{2} \in \left[0;1\right]$, при которых вероятность $Q_{z_{1} ,z_{2} } \left(\Pi (m_{1} ,m_{2} )\right)$ приняла бы возможно большее значение. Для этого выберем параметры, удовлетворяющие уравнениям:

\[\begin{array}{l} {\Exp_{z_{1} ,z_{2} } \left(\sum _{x\in X}\nu (x)x_{1}  \right)=m_{1} ,} \\ {\Exp_{z_{1} ,z_{2} } \left(\sum _{x\in X}\nu (x)x_{2}  \right)=m_{2} .} \end{array}\] 

В результате алгебраических преобразований получаем, что при $m_{2} =cm_{1} $ значения $z_{1} ,z_{2} $ равны соответственно 

\[1-\left(\frac{\zeta (3)c}{\zeta (2)m_{1} } \right)^{\frac{1}{3}} (1+o(1)), \; \; 1-\left(\frac{\zeta (3)}{\zeta (2)c^{2} m_{1} } \right)^{\frac{1}{3}}  (1+o(1)).\] 

Дзета-функция Римана появляется в связи с тем, что рассматриваются только пары взаимно простых чисел -- пространство $X$ (см. задачу \ref{sec:z_func_riman} из раздела \ref{hard}).
\end{comment}

Зафиксируем последовательность $0<\tau _{1} <\tau _{2} <\cdots <\tau _{N} $. Впоследствии $\tau _{1} \to 0$, $\tau _{N} \to \infty $, $\mathop{\max }\limits_{j} (\tau _{j+1} -\tau _{j} )\to 0$. Введем случайные величины

\[\zeta _{k}^{(j)} (\nu )=\sum _{x\in X:\; \tau _{j} \le {\frac{x_2}{x_1}} \le \tau _{j+1} }x_{k} \nu (x) ,\quad k=1,2.\] 

\noindent $\zeta _{1}^{(j)} (\nu )$ и $\zeta _{2}^{(j)} (\nu )$ есть приращение по осям $x_{1} $, $x_{2} $ соответственно той части ломаной, где тангенс угла наклона звеньев заключен между $\tau _{j} $ и $\tau _{j+1} $. По отношению к распределению вероятностей $Q_{z_{1} ,z_{2} } $ случайные величины $\zeta _{k}^{(j)} (\nu )$ взаимно независимы при разных $j$. Их математические ожидания:

\[\Exp_{z_{1} ,z_{2} } \zeta _{k}^{(j)} (\nu )=\sum _{\tau _{j} \le {\frac{x_2}{x_1}} \le \tau _{j+1} }x_{k} \frac{z_{1}^{x_{1} } z_{2}^{x_{2} } }{1-z_{1}^{x_{1} } z_{2}^{x_{2} } }  .\] 

С учетом полученных выше параметров $z_{1} ,z_{2} $ это соотношение можно переписать в дифференциальном виде, решением которого и является детерминированная кривая (см. рис.)

\[\left(ct_{1} +t_{2} \right)^{2} =4ct_{2} .\] 

Далее нужно воспользоваться законом больших чисел по отношению к распределениям $Q_{z_{1} ,z_{2} } $ и $P_{m_{1} ,m_{2} } $.

\end{ordre}
\begin{remark}
Помимо уже приведенной выше литературы см. статью А.М.Вершик ``Предельная форма выпуклых целочисленных многоугольников и близкие вопросы'', Функциональный анализ и его приложения. -- 1994. -- Т. 28, Вып.1. -- С. 16–25.
\end{remark}
\begin{comment}

\begin{problem}
\label{sub_exp}
Характеристические функции с.в. $\xi_k$ ограничены согласно выражениям
\[
\log \Exp e^{\lambda \xi_k} \leq \frac{q_k^2 + \lambda^2}{2}, 
\quad |\lambda| < g.
\]
Покажите, что для суммы $S = \sum_k c_k \xi_k$ при условиях $\sum_k c_k = 1$, $\sum_k e^{-q_k} \leq 1$  выполнены следующие утверждения:

\begin{enumerate}
\item если $g = \infty$, то
\[
\log \Exp e^{S} \leq H_1 = \sum_k c_k q_k,
\]
\[
\forall x \geq \frac{1}{2}:  \; \PR(S \geq H_1 + \sqrt{2x}) \leq e^{-x}.
\]

\item если $g < \infty$, то $\forall |\lambda| < g$
\[
\log \Exp e^{\lambda S} \leq \frac{H_2 + \lambda^2}{2}, 
\quad H_2 = \sum_k c_k q_k^2,
\]
\[
\forall x \geq \frac{1}{2}:  \; \PR(S \geq H(x)) \leq e^{-x}, 
\quad H(x) = H_1 + \sqrt{2x} + \frac{g^{-2} x + 1}{g} H_2.  
\]
\item если $g^2 \geq H_2 + 1$, то
\[
\Exp S \leq H_1 + \frac{H_2}{g} + 3,
\quad
\Exp S^2 \leq \left(H_1 + \frac{H_2}{g} + 4 \right)^2.
\]

\end{enumerate}

\end{problem}

\begin{ordre}
\begin{enumerate}
\item Воспользуйтесь неравенством Гельдера в виде 
\[
\log \Exp e^{\sum_k a_k \zeta_k} \leq \sum_k a_k \log \Exp e^{\zeta_k},
\]  
а также соотношением
\[
\PR \left(\sum_k \zeta_k \geq 0 \right) \leq \sum_k \PR (\zeta_k \geq 0).
\]
\item Введем вспомогательную переменную
\[
z_k(\lambda_k) = 
\begin{cases}
\frac{x+q_k}{g} + \frac{g}{2} + \frac{q^2_k}{2g}, \quad \lambda_k \geq g, \\
q_k + \sqrt{2x}, \quad \lambda_k < g.
\end{cases}
\]
Убедитесь, что 
\[
\PR(\xi_k \geq z_k) \leq e^{-x}, 
\quad
\sum_k c_k z_k \leq H(x).
\]
\item Примените результат задачи \ref{mom_ineq} из раздела \ref{standart}.
\end{enumerate}
\end{ordre}

\end{comment}







\section{Предельные теоремы}
\label{zb4}


\begin{problem}
\label{contin}
Пусть для любой $f(\cdot) \in C^{\infty}\left(\mathbb{R}\right)$, обладающей равномерно ограниченными (для  $\forall i > 0$, $|f^{(i)}| <  \infty$) непрерывными производными,  при $n\to\infty$ выполнено соотношение 
 \[
 \int f(x) dF_n(x) \rightarrow \int f(x) dF(x),
 \]
\noindent где $F_n$, $F$ ~--- функции распределения с.в. $X_n$ и $X$ соответственно. Докажите, что в таком случае имеет место сходимость по распределению  $X_n$ к $X$, то есть при $n \to\infty$
\begin{equation*}
F_n(x) \to F(x)\quad\text{ в точках непрерывности }F(x).
\end{equation*}
 
\end{problem}
\begin{remark}
Если бы мы рассмотрели в качестве класса функций $f$ непрерывные ограниченные функции на метрическом пространстве $S$, то получили бы в точности сходимость по распределению (распределения задаются на классе борелевских множеств метрического пространства $S$, в некоторых учебниках такую сходимость называют слабой сходимостью, что, пожалуй, точнее отражает суть дела). 

См. Биллингсли П. Сходимость вероятностных мер. М.: Наука, 1977, 352 с.
\end{remark}

\begin{problem}

Объясните, почему при $n\to\infty$ нет слабой сходимости последовательности с.в. $\xi_n$  следующего вида:
\begin{equation*}
\mathbb{P} (\xi_n = -n) = \mathbb{P} (\xi_n = n) = \frac{1}{2}. 
\end{equation*}
Имеет ли место сходимость соответствующей последовательности функций распределений? Если ответ положительный, то будет ли функцией распределения предельная функция? Будет ли поточечно сходиться соответствующая последовательность характеристических функций? Приведите пример, когда имеет место поточечная сходимость характеристичесих функций, но нет слабой сходимости (сходимости по распределению) соответствующих случайных величин.
\end{problem}


\begin{problem}\Star(Теорема Леви--Крамера)
\label{levi_kramer}
Пусть $\left\{ {\xi _n } 
\right\}_{n\in { \mathbb{N}}} $ последовательность с.в., а $\left\{ {F_n \left( x 
\right)} \right\}_{n\in { \mathbb{N}}} $ и $\left\{ {\phi _n \left( t \right)} 
\right\}_{n\in { \mathbb{N}}} $ соответствующие последовательности функций 
распределений и характеристических функций. Покажите, что верны следующие утверждения.
\begin{enumerate}
\item Если существует с.в. $\xi $ с характеристической функцией $\phi 
\left( t \right)$ и такой функцией распределения $F\left( x \right)$, что при $n\to\infty$  
$F_n \left( x \right)\to F\left( x \right)$ в точках непрерывности $F\left( 
x \right)$, то (при $n\to\infty$) $\phi _n \left( t \right)\to \phi \left( t \right)$ 
равномерно на каждом конечном интервале.
\item Если существует такая непрерывная в нуле функция $\phi \left( t 
\right)$, что $\phi _n \left( t \right)\to \phi \left( t \right)$ при $n\to\infty$, то 
существует такая с.в. $\xi $ с характеристической функцией $\phi \left( t 
\right)$ и функцией распределения $F\left( x \right)$, такой что (при $n\to\infty$)  $F_n \left( x 
\right)\to F\left( x \right)$ в точках непрерывности $F\left( x \right)$, то 
есть $\xi _n \buildrel d \over \longrightarrow \xi $. Более того, при $n\to\infty$  
$F_n \left( x \right)$ сходится к $ F\left( x \right)$ равномерно на любом конечном или 
бесконечном множестве точек непрерывности функции $F\left( x \right)$, а 
также $\phi _n \left( t \right)\to \phi \left( t \right)$ равномерно на 
каждом конечном интервале.
\end{enumerate}

\end{problem}
\begin{remark}
См. книгу Сачков В.Н. Вероятностные методы в комбинаторном анализе. М.:~Наука, 1978, а также \cite{Gupta,21,stoianov}.
\end{remark}

\begin{problem}
Пусть с.в. $X_k$ нормально распределены и существует такая с.в. $X$, что $X_n \to X$ в $L_2$, т.е. $\mathbb{E} \left[|X_n-X |^2\right] \to 0$ при $n\to\infty$. Покажите, что с.в. $X$  нормально распределена.
\end{problem}

\begin{problem}[Теорема Линдберга]
Пусть   
$
\xi_{1}, \ldots , \xi_{n}
$, $n\in\mathbb{N}$ 
 независимые случайные величины, причем  $\Exp(\xi_{i}) = 0$ и $\Var(\xi_{i}) = \sigma^2_{i} < \infty$. Отметим, что вероятностное пространство может меняться в зависимости от $n$. Введем обозначения: 
 \[
 S_n = \xi_{1} + \ldots + \xi_{n},
 \]
 \[
 s^2_n = \sigma^2_{1} + \ldots + \sigma^2_{n}.
 \]
  
Покажите, что если $\forall \epsilon > 0$ 
   \begin{equation} \label{lindberg}
  \frac{1}{s_n^2} \underset{k=1}{\overset{n}{\sum}} \underset{x \geq \epsilon s_n}{\int} x^2 dF_{\xi_{k}}(x) \rightarrow 0, \: n \rightarrow \infty,
  \end{equation} 
  
  \noindent то при $n\to\infty$
  $$
  \frac{S_n}{s_n} \overset{d}{\longrightarrow} N(0, 1).
  $$
  Покажите также, что из условия \eqref{lindberg} следует следующее условие на частичную дисперсию:
  $$
    \max_{j\leq n}\dfrac{\sigma_j^2}{s_n^2} \rightarrow 0, \, n\rightarrow \infty
  $$
\end{problem}

\begin{ordre}[Л. Биллингсли]
Используя результат задачи \ref{contin}, необходимо доказать, что при $n\to\infty$
\[
\Exp\bigg[f\bigg(\frac{S_n}{s_n}\biggr)\biggr] \rightarrow \Exp(f(N)),
\] 
где $N$ --- нормально распределенная случайная величина с математическим ожиданием 0 и дисперсией 1.

Зафиксируем произвольную $f \in C^{\infty}, |f^{(i)}| <  \infty, \forall i > 0$.
Используя формулу Тейлора, покажите что
\begin{equation}\label{tayl_lind}
f(x + h_1) - f(x+h_2)  - \bigg[f'(x)(h_1 - h_2) + \frac{1}{2} f''(x)(h_1^2 - h_2^2)\biggr] \leq g(h_1) + g(h_2),
\end{equation}
\noindent где 
$$g(h) = \underset{x}{\sup}\bigg| f(x+h) - f(x) - f'(x)h - \frac{1}{2} f'(x)h^2\biggr|.$$

Последовательно заменяя $\xi_{i}$ в сумме $S_n$ на случайные величины $\eta_{k} \in N(0, \sigma_{k}^2)$, получим последовательность 

\[
\Exp\bigg[f\bigg(\frac{S_n}{s_n}\biggr)\biggr],\,
\ldots,\,
\Exp\bigg[f\bigg(\frac{\xi_{1} + \ldots + \eta_{n}}{s_n}\biggr)\biggr],\,
\ldots,\,
\Exp[f(N)].
\]

Модуль разности первого и последнего членов данной последовательности можно оценить сверху суммой модулей разности пар последовательно идущих элементов.  Введя вспомогательную переменную $\beta_{k} = \underset{i<k}{\sum}\xi_{k} + \underset{i>k}{\sum}\eta_{k}$, докажите, используя \eqref{tayl_lind}, следующее неравенство
\[
\biggr|\Exp\bigg[f\bigg(\frac{S_n}{s_n}\biggr)\biggr] - \Exp(f(N))\biggr| \leq \underset{k=1}{\overset{n}{\sum}} \Exp \bigg[g\bigg(\frac{\xi_{k}}{s_n}\biggr)\biggr] + \underset{k=1}{\overset{n}{\sum}} \Exp \bigg[g\bigg(\frac{\eta_{nk}}{s_n}\biggr)\biggr].
\] 

Далее, воспользовавшись свойством функции $g$ 
$$\exists K: g(h) < K \min(h^2, |h|^3)$$
и разбив математическое ожидание на интегралы по множеству  $(x \geq \epsilon s_n)$ и его дополнению, установите сходимость правой части неравенства к нулю.  
\end{ordre}

\begin{remark}

Классический вариант центральной предельной теоремы (ц.п.т.) может быть получен как следствие теоремы Линдберга \cite{5}: пусть $\xi_{i}$, $i=1,2,\dots$  независимые и одинаково распределенные случайные величины с нулевым математическим ожиданием и конечной дисперсией, тогда при $n\to\infty$
\[
\frac{1}{\sigma \sqrt{n}} \underset{k=1}{\overset{n}{\sum}} \xi_{k} \overset{d}{\longrightarrow}  N(0, 1).
 \]
 Отметим также, что с помощью описанной схемы можно оценить и скорость сходимости в ц.п.т.
 
Можно показать, что если 
\[
\lim_{n\to\infty}\max_{1\le k \le n} \Exp\left[\xi_{k}^2\right]=0\text{ или }\lim_{n\to\infty}\max_{1\le k \le n} \PR\left(|\xi_{k}|\ge\epsilon\right)=0,
\]
то условие (\ref{lindberg}) является не только достаточным, но и необходимым, для выполнения ц.п.т. Несложно проверить, что условие (\ref{lindberg}) влечет первое из этих условий, из которого, в свою очередь, следует второе.

См. Биллингсли П. Сходимость вероятностных мер. М.: Наука, 1977, 352 с.
\end{remark}




\begin{comment}
\begin{remark} 

\noindent $F_{\xi _{n} } (x)=\left\{\begin{array}{cc} {0,} & {x\le -n} \\ {{\raise0.7ex\hbox{$ 1 $}\!\mathord{\left/ {\vphantom {1 2}} \right. \kern-\nulldelimiterspace}\!\lower0.7ex\hbox{$ 2 $}} ,} & {-n<x\le n} \\ {1,} & {x>n} \end{array}\right. $ сходятся к функции $G(x)\equiv \frac{1}{2} $.

\end{remark} 
\end{comment}
\begin{problem}
Пусть $\xi _{1} ,\xi _{2} ,...$  независимые одинаково распределенные с.в. с конечной ненулевой дисперсией. \mbox{Обозначим $S_{n} =\sum _{i=1}^{n}\xi _{i}$}. Выяснить при каких значениях $c$ имеет или не имеет место сходимость при $n\to\infty$
$$
\PR\left(\frac{S_{n} }{n} \leq c\right)\to I\left(\Exp\xi _{1} \leq c\right).
$$

\begin{remark}
См. A.S. Cherny, The Kolmogorov student's competitions  on probability theory.
\end{remark}

\end{problem}


\begin{problem}
Пусть имеются независимые одинаково распределенные случайные величины $X_i$, $i=1,\dots,n$:

\[X_{i} =\left\{\begin{array}{cc} {1,} & {p;} \\ {-1,} & {q=1-p.} \end{array}\right. \] 

Покажите, что верна {\it локальная предельная теорема} (см. \cite{2}) для суммы независимых случайных величин $X_i$: равномерно по всем $x=O\left(\sqrt{n} \right)$ таким, что $(p-q)n+x$ целое неотрицательное число, выполнено (при $n\to\infty$)
\[\PR\left\{\sum _{i=1}^{n}X_{i} =(p-q)n+x \right\}\approx \frac{1}{\sqrt{2\pi npq} } \exp \left\{-\frac{x^{2} }{2npq} \right\}.\] 
\end{problem}
\begin{ordre} Покажите, используя \textit{теорему Коши} из курса ТФКП, что
$$\PR\left\{S_{n} =\left\lfloor \alpha n\right\rfloor \right\}=\frac{1}{2\pi i}     \ointctrclockwise _{|z|=\rho }\frac{(pz+qz^{-1} )^{n} }{z^{\left\lfloor \alpha n\right\rfloor +1} } dz$$
$$=\frac{1}{2\pi i} \ointctrclockwise  _{|z|=\rho }e^{n\left[\ln (pz+qz^{-1} )-\alpha \ln z\right]} e^{\left[\alpha n-\left\lfloor \alpha n\right\rfloor -1\right]\ln z} dz.$$


Примените \textit{метод перевала} для аппроксимации полученного интеграла. Выберите радиус окружности $\rho $ так, чтобы точка перевала находилась на пересечении этой окружности с положительной вещественной осью: $z=\rho $; для этого найдите максимум функции $\ln (pz+qz^{-} )-\alpha \ln z$. Замените интеграл по всей окружности на интеграл по $\delta $-дуге, содержащей точку перевала \cite{27}.
\end{ordre}


\begin{comment}
\begin{problem}

Доказать локальную предельную теорему:

\noindent Пусть $0<p<1$ и $X_{i} $, $i=1,...,n$ - независимые случайные величины, имеющие распределение:
\[X_{i} =\left\{\begin{array}{cc} {1,} & {p,} \\ {-1,} & {q=1-p;} \end{array}\right. \] 
Тогда равномерно по всем $x=O\left(\sqrt{n} \right)$ таким, что $(p-q)n+x$ целое неотрицательное число
\[\PR\left\{\sum _{i=1}^{n}X_{i} =(p-q)n+x \right\}\sim \frac{1}{\sqrt{2\pi npq} } \exp \left\{-\frac{x^{2} }{2npq} \right\}\] 
при $n\to \infty $. 

\begin{remark}
Воспользоваться формулой Стирлинга
\[
n! \sim \sqrt{2 \pi n} \frac{n^n}{e^n} 
\]
\end{remark}

\begin{remark}
Пусть $n=2k$ и $p=\frac{1}{2} $, тогда вероятность того, что число единиц в точности рано числу минус единиц мало (но не экспоненциально мало):
\[\PR\left\{\sum _{i=1}^{2k}X_{i} =k \right\}\sim \frac{1}{\sqrt{\pi k} } \] 
\end{remark}

\end{problem}

\end{comment}


\begin{problem}\Star(Коралов--Синай \cite{7})
Пусть задано гильбертово пространство
$\rm H = L^2\left( {{\rm R},{\rm B},\mu _G } \right)$, где $\mu _G$ --  
гауссовская мера, со скалярным произведением 
$$
\langle {f,g} 
\rangle=\int\limits_{-\infty }^{+\infty } 
{f(x)g(x)d\mu_G(x)}.
$$ 

Пусть $\xi _i $, $i=1,\dots,n$ независимые с.в. с нулевым математическим ожиданием и 
плотностью распределения 
$$
p_h =\frac{1}{\sqrt {2\pi } }\left( {1+h(x)} 
\right)e^{-\frac{x^2}{2}},
$$ где такой $h$ -- элемент $\rm H$, что
\begin{enumerate}
\item величина $\left\| h \right\|$ достаточно мала, 
где 
$$\left\| h \right\|^2=\frac{1}{\sqrt {2\pi } }\int\limits_{-\infty }^{+\infty } 
{h^2(x)e^{-\frac{x^2}{2}}dx},$$

\item $\langle h(x),\mathbbm{1} \rangle=0$,

\item $\langle {h(x),x} \rangle=0$.

\end{enumerate}
Покажите, что последовательность с.в. 
$$
\zeta _n 
=2^{-\frac{n}{2}}\sum\limits_{i=1}^{2^n} {\xi _i } 
$$ 
сходится 
по распределению к нормальной с.в. с нулевым математическим ожиданием и 
дисперсией 
$$
\sigma ^2(p_h )=\frac{1}{\sqrt {2\pi } }\int\limits_{-\infty 
}^{+\infty } {x^2p_h (x)dx}.$$

\end{problem}
\begin{remark} 
Для решения задачи можно воспользоваться методом ренормгруппы. Заметьте, что
$$\zeta _{n+1} =\frac{1}{\sqrt 2 }\left( {\zeta '_n 
+\zeta ''_n } \right),$$
где 
$$
\zeta '_n 
=2^{-\frac{n}{2}}\sum\limits_{i=1}^{2^n} {\xi _i },\quad 
\zeta ''_n 
=2^{-\frac{n}{2}}\sum\limits_{i=2^n+1}^{2^{n+1}} {\xi _i }
$$
независимые 
одинаково распределенные случайные величины, поэтому 
$$
p_{n+1} (x)=Tp_n (x),
$$ 
где $p_n (x)$~--- плотность распределения $\zeta _n $, а $T$~--- нелинейный 
оператор (действующий в пространстве плотностей), такой что 
$$
Tp(x)=\sqrt 2 
\int\limits_{-\infty }^{+\infty } {p(\sqrt 2 x-u)p(u)du}.
$$ 
В этих терминах 
в задаче нужно показать, что 
\[
T^n p_h (x)\mathop \to \limits_{n\to \infty } 
\frac{1}{\sqrt {2\pi } \sigma (p_h )}{\rm e}^{-\frac{x^2}{2\sigma ^2(p_h )}}.
\] 

Затем рассмотрите нелинейный оператор $\tilde{L}$ на пространстве ${\rm H}$, связанный с 
оператором $T$ следующим образом
$$
T p_h (x)=\frac{1}{\sqrt {2\pi } }\left( 
{1+\tilde{L}\left( {h(x)} \right)} \right){\rm e}^{-\frac{x^2}{2}}.
$$ 
Линеаризуйте 
оператор $\tilde {L}$, показав, что $\tilde {L}h=Lh+O\left( {\left\| h 
\right\|^2} \right)$, где 
\[L(h)(x)=\frac{2}{\sqrt \pi }\int\limits_{-\infty 
}^{+\infty } {e^{-\left( {\frac{x^2}{2}-\sqrt 2 xu+u^2} \right)}h(u)du} .
\]

Далее покажите, что линейный оператор $L$ имеет полное множество собственных 
векторов 
$$h_k (x)=e^{\frac{x^2}{2}}\left( {\frac{d}{dx}} 
\right)^ke^{-\frac{x^2}{2}},\quad k\ge 0,
$$
с собственными значениями $\lambda 
_k =2^{1-\frac{k}{2}}$, $k\ge 0$. 

Пусть ${\rm H}_k $~--- одномерное 
подпространство, натянутое на $h_k (x)$. Тогда $L$~--- сжимающий оператор на 
${\rm H} \setminus \left( {{\rm H}_0 \oplus {\rm H}_1 \oplus {\rm H}_2 } \right)$.


Покажите, что $\tilde {L}$~--- сжимающий оператор на ${\rm H} \setminus \left( {{\rm 
H}_0 \oplus {\rm H}_1 } \right)$, имеющий единственную неподвижную точку 
\[f_h (x)=\frac{1}{\sigma (p_h )}e^{\frac{x^2}{2}-\frac{x^2}{2\sigma ^2(p_h 
)}}-1.
\]
\end{remark}



\begin{problem}
В игре в рулетку колесо разделено на 38 равных секторов: 18 красных, 18 белых и два сектора (0 и 00) зеленого цвета. Пусть ставка игрока на каждом шаге равна 1000 рублей. Обозначим за $X_{i} $ выигрыш в $i$-ой игре. Тогда $X_{1} ,X_{2} ,X_{3} ,...$ ~--- независимые с.в., имеющие распределение: 
\[X_{i} =\left\{\begin{array}{cc} {+1,} & p = 18/38, \\ {-1,} & p = 20/38. \end{array}\right. \] 
Пусть сыграно $n=19^2=361$ партий. С помощью ц.п.т.  и  неравенства Берри--Эссена оцените погрешность приближения выигрыша.
\end{problem}
\begin{remark}
Приведем формулировку теоремы  Берри--Эссена \cite{19}.
\label{sec:BerryEssen}
 Пусть $\xi_1, \xi_2\dots$ независимые одинаково распределенные с.в., причем $\mathbb{E}\xi_i = m$, 
 $\mu^3={\mathbb E}|\xi_i - {\mathbb E}\xi_i|^3<\infty$, $\sigma^2=\mathbb D \xi_i$.
Близость с.в. $\frac{\sum_{i=1}^{n}\xi_i-nm}{\sigma\sqrt{n}}$ к стандартной нормально распределенной с.в. (согласно ц.п.т.) в смысле 
близости их функций распределения определяется неравенством Берри--Эссена 
$$
\sup\limits_x \left| {\mathbb P}\Bigl( \frac{\sum_{i=1}^{n}\xi_i-nm}{\sigma\sqrt{n}}<x \Bigr) - \Phi(x) 
\right| \le \frac{C_0 \mu^3}{\sigma^3 \sqrt{n}} , 
$$
где $0.4<C_0<0.7056$,
$\Phi(x)=\int_{-\infty}^x \frac{e^{-t^2/2}}{\sqrt{2\pi}}\, dt$.
\end{remark}

\begin{problem}[Петербургский парадокс \cite{19, book12}] 
\label{piter}
Рассмотрим игру в орлянку: игрок делает ставку и подкидывает симметричную монету, если выпадает «орел», то игрок забирает двойную ставку, иначе теряет все свои деньги. Броски повторяются независимо в каждой игровой серии, изначально игрок имеет в бане сумму равную 1 и в каждой серии разыгрывает все деньги, что у него есть.\\
\indent Пусть $X_{1} ,X_{2} ,X_{3} ,...$  --- независимые с.в., обозначают состояние банка игрока после 1,2,3.. игр соответственно. Несложно заметить, что распределение $X_i$ выглядит как $\PR\left(X_{i} =2^{k} \right)=2^{-k} $, $k=1,2,3,...$. То есть, если в игре в орлянку $k$ раз выпал «орел», то выигрыш будет $2^{k}$. \\
\indent Справедливой ценой за игру называют математическое ожидание выигрыша. Но здесь $\Exp X_{i} =\infty $, однако, для этого нужно играть бесконечное число раз и иметь бесконечно много денег. Покажите, что 
$$
\frac{S_{n} }{n\log _{2} n} \mathop{\to }\limits^{p} 1\quad \text{при~} n\to \infty ,
$$ 
где $S_{n} =\sum _{k=1}^{n}X_{k}$. Проинтерпретируйте этот результат, введя цену за $n$ игр.

\begin{remark} 

Пусть для каждого $n$ с.в. $X_{nk} $, $1\le k\le n$ независимы. Пусть также $b_{n} >0$ с $b_{n} \to \infty $ и $\bar{X}_{nk} =X_{nk}  I \left\{X_{nk} \le b_{n} \right\}$. Предположим, что выполняются условия:
\begin{enumerate}
\item 
$
\sum _{k=1}^{n}\PR\left\{\left|X_{nk} \right|>b_{n} \right\} \mathop{\to }\limits_{n\to \infty } 0;
$ 
\item 
$
\frac{1}{b_{n} ^{2} }\sum _{k=1}^{n}\Var \bar{X}_{nk}   \mathop{\to }\limits_{n\to \infty } 0.
$ 
\end{enumerate}
Тогда верно
$$\frac{1}{b_{n} }\left(\sum _{k=1}^{n}X_{nk}  -\sum _{k=1}^{n}\Exp \bar{X}_{nk}  \right) \mathop{\to }\limits^{p} 0\quad \text{при~} n\to \infty. 
$$

\noindent Для решения задачи положите $X_{nk} =X_{k} $. В качестве $b_{n} >0$ возьмите $b_{n} =2^{m(n)} $, где $m(n)$ -- целое число, которое можно представить в виде $m(n)=\log _{2} n+K(n)$, $K(n)\to \infty $ при $n\to \infty $. Например, если \mbox{$K(n)\le \log_2 (\log_2 n)$,} то результатом применения приведенного результата будет $\frac{S_{n} }{n\log _{2} n} \mathop{\to }\limits^{p} 1$ при $n\to \infty $.

\end{remark} 

\end{problem}

\begin{problem}
Случайная величина (размер выигрыша) принимает значение $2^{k} -1$ с вероятностью 
$$p_{k} =\frac{1}{2^{k} k(k+1)} \quad \text{для~} k=1,2,3,...$$ 
и значение\textit{ $-1$} с вероятностью 
$$p_{0} =1-\sum _{k=1}^{\infty }p_{k} .$$ 
Проверьте, что математическое ожидание выигрыша равно нулю. Применив теорему из предыдущей задачи, покажите, что при $n\to \infty $  для суммарного размера выигрыша за $n$ партий ($S_{n} $) справедливо
$$\frac{S_{n} }{n \log _{2} n}\mathop{\to }\limits^{p} 1.$$

\begin{remark}  
Положите в замечании к задаче \ref{piter} $b_{n} =2^{m(n)} $, где $$m(n)=\min \left\{m\in \mathbb{N}:\; 2^{-m} \frac{1}{\sqrt{m^{3}}} \le n^{-1} \right\}.$$ Следует также обратиться к книге \cite{19}.
\end{remark} 

\end{problem}


%\begin{comment}
\begin{problem}[MAX-устойчивые распределения: Гумбеля, Фреше, Вейбулла]
Распределение $G\left(x\right)$ называется max-устойчивым, если для любых $n=1,2,...$ существуют $a_{n} >0$ и $b_{n} \in {\mathbb R}$, такие что $G^{n} \left(a_{n} x+b_{n} \right)=G\left(x\right)$.
Пусть есть независимые одинаково распределенные с.в. $X_{1} ,...,X_{n} $ с распределением $F\left(x\right)$. Обозначим $X_{\left(n\right)} =\max \left\{X_{1} ,...,X_{n} \right\}$. Распределение такой с.в. $F_{X_{\left(n\right)} } \left(x\right)=\left[F\left(x\right)\right]^{n} $.
Обозначим также за $x_{(n)}$ решение уравнения $$\mathbb{P}(X>x_{(n)}) = 1-F(x_{(n)}) = 1/n$$ (так называемое, характеристическое наибольшее значение).
\begin{enumerate}
\item Пусть при $x\to\infty$ функция распределения $X_i$ имеет вид $F(x) = 1-Ax^{-\alpha}$, $\alpha>0$. Покажите, что функция распределения случайной величины $Y = X_{(n)}/x_{(n)}$ поточечно стремится к пределу $F_{Y}(y) = {\rm e}^{{-y}^{-\alpha}}$ при $y\geq 0$, $F_{Y}(y) = 0$ при $y<0$ (распределение Фреше).
%\item  Пусть $\mathop{\lim }\limits_{x\to \infty } e^{\alpha x} \left(1-F\left(x\right)\right)=\beta $, где $\alpha ,\beta >0$ и $x\in {\mathbb R}$. Покажите, что $X_{\left(n\right)} -\frac{1}{\alpha } \ln \left(\beta n\right)\mathop{\to }\limits^{d} \chi $, где $\chi $ имеет распределение Гумбеля: $\PR\left(\chi \le x\right)=e^{-e^{-\alpha x} } $, $x\in {\mathbb R}$.
\item Пусть случайные величины $X_i$ ограничены сверху ($X_i\leq 0$) и в окрестности нуля при $x\to\infty$ функция распределения $X_i$ имеет вид $F(x) = 1-A|x|^{\alpha}$, $\alpha>0$ при $x\leq 0$, $F(x)=1$ при $x>0$. Покажите, что функция распределения случайной величины $Y = X_{(n)}/x_{(n)}$ стремится к пределу $F_{Y}(y) = {\rm e}^{{-|y|}^{\alpha}}$ при $y\leq 0$, $F_{Y}(y) = 1$ при $y>0$ (распределение Вейбулла).
%\item  Пусть $\mathop{\lim }\limits_{x\to \infty } x^{\alpha } \left(1-F\left(x\right)\right)=\beta $, где $\alpha ,\beta >0$ и $x\in {\mathbb R}_{+} $. Покажите, что $X_{\left(n\right)} \left(\beta n\right)^{-\frac{1}{\alpha } } \mathop{\to }\limits^{d} \eta $, где $\eta $ имеет распределение Фреше: $\PR\left(\eta \le x\right)=e^{-x^{-\alpha } } $, $x>0$.
\item Пусть при $x\to\infty$ функция распределения $X_i$ имеет вид $F(x) = 1- {\rm e}^{-\lambda x}$. Покажите, что функция распределения случайной величины $Y = X_{(n)}-x_{(n)}$ стремится к пределу $F_{Y}(y) = {\rm e}^{-{\rm e}^{-\lambda y}}$ (распределение Гумбеля).
%\item  Пусть $\mathop{\lim }\limits_{x\to \infty } \left(c-x\right)^{\alpha } \left(1-F\left(x\right)\right)=\beta $, $F\left(c\right)=1$, где $\alpha ,\beta >0$, $c\in {\mathbb R}$ и $x\in {\mathbb R}$. Покажите, что $\left(X_{\left(n\right)} -c\right)\left(\beta n\right)^{\frac{1}{\alpha } } \mathop{\to }\limits^{d} \gamma $, где $\gamma $ имеет распределение Вейбулла: $\PR\left(\gamma \le x\right)=e^{-(-x)^{-\alpha } } $, $x<0$.

\item Покажите, что распределения Гумбеля, Фреше, Вейбулла являются max-устойчивыми.

\item Покажите, что если $X_i$, $i=1,2,\dots$ --- независимые одинаково распределенные стандартные
\\
1)  гауссовские случайные величины, то верно
\[
\lim_{n\to\infty}\PR\left\{2m_n[\max_{i=1,\dots,n}{X_i} - m_n]\leq z \right\}= \exp(-\exp(-z)),
\]
\noindent
где 
\[
  m_n = \left[2\log\frac{n+1}{\sqrt{8\pi}\log(n+1)}\right]^{1/2};
\]
2) лапласовские случайные величины, то верно
\[
\lim_{n\to\infty}\PR\left\{\max_{i=1,\dots,n}{X_i} - \log(n/2)\leq z \right\}= \exp(-\exp(-z)).
\]
%\item случайные величины с распределением Коши, %то верно
%\begin{equation}
%\lim_{n\to\infty}\PR\left\{\frac{\pi}{4}\max_{i=1,\dots,n}{X_i} \leq z \right\}= \exp(-z^{-1}).
%\end{equation}
\end{enumerate}

\begin{remark}
Рассмотрим следующий пример (см. Лагутин М.\,Б. Наглядная математическая статистика Издательство: М.: БИНОМ. Лаборатория знаний, 2007). Пусть $X_1,\dots,X_n$~--- независимые одинаково распределенные случайные величины с распределением 
$$
F(x) = \left(1-\frac{1}{\ln x}\right)I\left(x>\exp(1)\right),  
$$
Обозначим $X_{(n)} = \max\{X_1,\dots,X_n\}$. Оценим $\gamma = P(X_{(n)}>10^7)$ при $n=4$. Из независимости и одинаковой распределенности $X_i$ 
$$P(X_{(n)}\leq x) = [F(x)]^n.$$
Поскольку $\ln 10 \approx 2.3$, а $(1-\epsilon)^n\approx 1-\epsilon n$ при малых $\epsilon$, получаем
$$\gamma = 1 - \left(1-\frac{1}{7\ln 10}\right)^4 \approx 1/4.$$
Таким образом, примерно в каждом четвертом случае значение $X_{(4)}$ будет превышать $10^7$. 
Оказывается, что из-за того, что функция $F(x)$ имеет <<сверхтяжелый>> правый <<хвост>>, распределение случайной величины $X_{(n)}$ чрезвычайно быстро с ростом $n$ уходит на бесконечность и никаким линейным преобразованием не удается <<вернуть>> его в конечную область. Точнее, невозможно подобрать такие константы $a_n$ и $b_n>0$, чтобы последовательность $(X_{(n)}-a_{n})/b_n$ сходилась бы по распределению к невырожденному закону.

См. лекции (http://www.mccme.ru/ium/s09/probability.html) в НМУ  А.Н. Соболевского, а также Лидбеттер М., Линдгрен Г., Ротсен X. Экстремумы случайных последовательностей и процессов М.: Мир, 1989.

Отметим также, что распределения Гумбеля, Фреше, Вейбулла исчерпывают все возможные типы предельных распределений в классе max-устойчивых распределений.
\end{remark}
\end{problem}



\begin{problem}

Пусть $X_{1} ,...,X_{n} $ независимые одинаково распределенные с.в. со стандартным распределением Коши:
$$
F\left(x\right)=\frac{2}{\pi } \int _{-\infty }^{x}\frac{dy}{1+y^{2}}  
$$ 
Воспользовавшись предыдущей задачей, найдите предельное распределение для должным образом нормированных с.в. $X_{\left(n\right)} $.
\end{problem}
\begin{remark}
На эту тему также рекомендуется посмотреть задачу 119 из раздела 2, и задачу 27 из раздела 3.
\end{remark}

%\end{comment}

\begin{problem}
Пусть ${X}_{n} \in {\mathbb R}^{m} $, $n=1,2,...$ ~--- независимые одинаково распределенные случайные векторы, причем  
$$\Exp{X}_{n} =0, \quad \Exp({X}_{n} {X}_{n}^{T}) =R$$ ($R$ ~--- неотрицательно определенная матрица (по определению), однако, мы дополнительно будем считать, что $R$ положительно определенная). С помощью аппарата характеристических функций докажите, что тогда для любого борелевского множества $B\subseteq {\mathbb R}^{m} $ верно
\[\mathop{\lim }\limits_{N\to \infty } \PR\left(\frac{1}{\sqrt{N} } \sum _{n=1}^{N}{X}_{n}  \in B\right)=\dfrac{1}{(2\pi)^{m/2}|R|^{1/2}} \int _{B}{\rm e}^{-\frac{1}{2} <{x},R^{-1}{x}>} d {x} .\] 

Переформулируйте и решите задачу для случая, когда матрица $R$ положительная полуопределенная.
\end{problem}
\begin{remark}
Аппарат характеристических функций -- мощная конструкция для изучания свойств распределений случайных величин, т.к. известно, что есть взаимнооднозначное соответствие между вероятностой мерой и её Фурье образом. Читателю можно порекомедовать книгу книгу Боровкова А.А. \cite{3} для более подробного изучения свойств характеристических функций. Также большое количество показательных упражнений на характеристические функции можно найти в книге Ширяева [\ref{chiraiev}] т.1.\\
\indent Развивая историческую часть доказательства ЦПТ, стоит также упомянуть, что классически существует два подхода доказательства указанной выше теоремы. Первый способ -- метод Линденберга (см. задачу 5, раздел 5) и второй более распространенный -- метод характеристических функций, с которым можно ознакомится, например, в книге \cite{6}. Возникает вопрос, какой из способов доказательства более общий. Читателю предлагается проверить, что из первого метода следует второй.
\end{remark}




\begin{problem}

Пусть $X_1,\ldots,X_n$~--- независимые одинаково распределенные с.в. Пусть также характеристическая функция с.в. $X_k$ представляется 
в окрестности $t=0$ в виде 
$$
\varphi_{X_k}(t)={\mathbb E}(e^{it X_k})=1+imt+o(t). 
$$
 Верно ли, что при $n\to\infty$ 
$$
\frac{1}{n}\sum\limits_{i=1}^{n} X_i \xrightarrow{p} m?
$$
\end{problem}
\begin{remark}
Стоит отметить A.S. Cherny, The Kolmogorov student's competitions on probability theory, где можно найти набор весьма интересных небольших задач про связь сходимостей в различных вероятностных смыслах.
\end{remark}

\begin{comment}
\begin{problem}
Пусть $x_1, x_2, x_3, \ldots$~--- последовательность независимых одинаково распределенных с.в. Положим 
$S_n=\sum\limits_{k=1}^{n} x_k$. Покажите, что 

\begin{enumerate}
\item(з.б.ч.) если ${\mathbb E}(|x_k|)<\infty$, то $S_n/n\xrightarrow{P} m$ при $n\to\infty$, где $m={\mathbb E}(x_k)$; 

\item(ц.п.т.) если ${\mathbb E}(x_k^2)<\infty$, то $(S_n-m\cdot n)/\sqrt{n\cdot D}\xrightarrow{d} N(0,1)$ при $n\to\infty$, 
где $D=\Var x_k$. 

\item(задача математической статистики) Предположим, что независимо $n$ раз кидается монетка с вероятностью выпадения орла в каждом 
опыте равной $p$ (точного значения $p$ мы не знаем, а знаем лишь то, что $0.1\leqslant p\leqslant 0.9$), т.е. $x_k\in\Be(p)$. 
Сколько раз нужно кинуть монетку (оцените $p$), чтобы оценка $${\bar p}(x)=\frac{\sum\limits_{k=1}^{n}x_k}{n}$$ с вероятностью 
$\gamma\geqslant 0.95$ отличалась от истинного значения $p$ не более, чем на величину $\delta=0.01$? Применить неравенство Чебышева 
и предельную теорему (точность, которую дает ц.п.т., оцените с помощью неравенства Берри – Эссена). Сравнить результаты. 
\end{enumerate}
\end{problem}

\begin{remark}
См. раздел \ref{measure}, задача \ref{sec:BerryEssen}

\end{remark}
\end{comment}

\begin{problem}
Пусть при каждом $n\geqslant 1$ независимые с.в. $X_{1n}, X_{2n},\ldots, X_{nn}$ таковы, что $X_{kn}\in \Be(p_{kn})$, где 
$$\lim_{n\to\infty}\max\limits_{1\leqslant k\leqslant n} p_{kn}=0\quad\text{и ~} \lim_{n\to\infty}\sum\limits_{k=1}^{n}p_{kn}=\lambda.$$ 
Тогда при $n\to\infty$
\begin{equation*}
\label{TPois}
{\mathbb P}(S_n=m)\to  e^{-\lambda}\frac{\lambda^m}{m!}, \quad m=0,1,2,\ldots, 
\end{equation*}
 где $S_n=\sum\limits_{k=1}^{n} X_{kn}$. 
\begin{remark}
Если $p=\lambda/n$, то $$C_{n}^{m}p^{m}(1-p)^{n-m} = \frac{\lambda^m}{n!}e^{-\lambda}\left( 1 + O\left(\frac{m^2+\lambda^2}{n}\right)\right).$$
Кроме того, имеет место следующая оценка (Прохорова--Ле Кама): 
$$\sum_{m=0}^{\infty}{\left|\mathbb P(S_n=m) - e^{-\lambda}\frac{\lambda^{m}}{m!}\right|} \leq 2 \sum_{k=1}^{n}{p_{kn}^2}.$$
В качестве основных ссылок для этой задачи отметим \cite{19},[\ref{chiraiev}] т.1.
\end{remark}
\end{problem}

\begin{problem}
В течение дня игрок в казино участвует в $N=100$ независимых розыгрышах. В каждом розыгрыше он выигрывает с вероятностью 
$p=0.01$. Оцените вероятность события:\\
\indent а) игрок ни разу не выиграет\\
\indent б) выиграет ровно один раз \\
\indent в) выиграет ровно три раза.\\

Предположим, что игра происходит в течении $n=100$ дней. 
Оцените вероятность того, что за эти $100$ дней в общей сложности реализуется не менее $100$ выигрышей, не менее $300$ выигрышей. 


\end{problem}

\begin{problem}(Дельта метод). 
Пусть $T_n$~--- последовательность случайных величин, такая, что  при $n\to\infty$
$$\sqrt{n}(T_n-\theta)\xrightarrow{d}\mathcal{N}(0,\sigma^2(\theta)),\quad \sigma(\theta)>0.$$
Пусть отображение $g:\mathbb{R}\to\mathbb{R}$ дифференцируемо в $\theta$ и $g^{\prime}(\theta)\not=0$. Покажите, что при $n\to\infty$
$$
\sqrt{n}\left(g(T_n)-g(\theta)\right)\xrightarrow{d} \mathcal{N}(0,[g^{\prime}(\theta)]^2\sigma^2(\theta)).
$$
\end{problem}
\begin{remark} 
Пусть  $g^{\prime}(\theta) = 0$, в таком случае распределение $g(T_n)$ определяется третьим членом разложения Тейлора, то есть:
$$
g(T_n) = g(\theta) + \frac{(T_n-\theta)^2}{2}g^{\prime\prime}(\theta)+o\left((T_n-\theta)^2\right),
$$ 
где в данной задаче $o(1)$--величина, сходящаяся с вероятностью $1$ к $0$. Поэтому при $n\to\infty$
$$
\sqrt{n}(g(T_n) -g(\theta)) = n\frac{(T_n-\theta)^2}{2}g^{\prime\prime}(\theta)+o(1)\xrightarrow{d} \frac{g^{\prime\prime}(\theta)\sigma^2(\theta)}{2}\chi^2(1),
$$
где $\chi^2(1)$ --- случайная величина с распределением $\chi^2$ с одной степенью свободы.

Для решения задачи рекомендуется ознакомиться с литературой по дельта-методу, указанной в \cite{Gupta}.
\end{remark}

\begin{problem}
Пусть $\{T_n\}$ последовательность $k$-мерных случайных векторов, таких, что при $n\to\infty$ 
$$\sqrt{n}(T_n-\theta)\xrightarrow{d}\mathcal{N}(0,\Sigma(\theta)).$$ 
Пусть $g:\mathbb{R}^{k}\to \mathbb{R}^m$ дифференцируемо в $\theta$ с $\nabla g(\theta)$. 
\begin{enumerate}
\item

Покажите, что при $n\to\infty$
$$
\sqrt{n}\left(g(T_n)-g(\theta)\right)\xrightarrow{d} \mathcal{N}(0,\nabla g(\theta)^{T}\Sigma(\theta)\nabla g(\theta)),
$$
если $\nabla g(\theta)^{T}\Sigma(\theta)\nabla g(\theta)$ положительно определена.
\item
Рассмотрите пример, когда   $X_1,\dots,X_n$ независимые одинаково распределенные случайные величины с математическим ожиданием $\mu$ и дисперсией $\sigma^2$, а $T_n = \frac{1}{n}\sum_{i=1}^n X_n$. Покажите, что тогда при $n\to\infty$ 
$$
\sqrt{n}(T_n^2-\mu^2)\xrightarrow{d}\mathcal{N}(0,4\mu^2\sigma^2).
$$
\end{enumerate}
\end{problem}

\begin{remark}
См. также книгу Боровков А.А., Математическая статистика, СПб.: Лань, 2010.
\end{remark}

\begin{problem}(Центральная предельная теорема для стационарных последовательностей)

Пусть $X_i$, $i=1,2,\dots$ --- стационарная последовательность с $\mathbb{E}(X_i)=\mu$ и $\Var (X_i) = \sigma^2<\infty$, обладающая  следующим свойством: для некоторого фиксированного $m$ выполнено, что $(X_1,\dots,X_i)$ и $(X_j,X_{j+1}\dots,)$ независимы при $j-i>m$.

Покажите, что $n\to\infty$

$$
\frac{\frac{1}{n}\sum_{i=1}^n X_i-\mu}{\sqrt{n}}\overset{d}{\to} \mathcal{N}(0,\tau^2),
$$
где $\tau^2 = \sigma^2+2\sum_{i=2}^{m+1}{\rm cov}(X_1,X_i).$

\end{problem}
\begin{remark}
В главах 9 и 10 книги \cite{Gupta} содержится большое количество вариаций центральной предельной теоремы для случая случайных  последовательностей с зависимыми элементами (например, стационарных последовательностей, марковских последовательностей и др.).
\end{remark}

\begin{problem}
Пусть $X_1$, $X_2$, ... независимые одинаково распределенные случайные величины с конечным четвертым моментом. Пусть $\mathbb{E}(X_1)=\mu$ и $\mathbb{D}(X_1)=
\sigma^2$. Пусть $g$ --- функция с равномерно ограниченной четвертой производной. Покажите, что 
\begin{enumerate}
\item $\mathbb{E}[g(\bar{X})] = g(\mu)+\frac{g^{(2)}(\mu)\sigma^2}{2n}+O(n^{-2})$, 
\item
$\mathbb{D}[g(\bar{X})] = \frac{(g^{\prime}(\mu))^2}{n}+O(n^{-2}).$
\end{enumerate}
\end{problem}
\begin{remark}
Основные ссылки по указанным результатам содержатся в \cite{Gupta}.
\end{remark}

\begin{problem}

Показать, что при бросании симметричной монеты $n$ раз отношение числа выпадений герба к числу выпадений решки почти наверное стремится 
к $1$ при $n\to\infty$, а вероятность того, что число выпадений герба в точности равняется числу выпадений решки, при четном числе 
бросаний стремится к $0$ при $n\to\infty$. 
\end{problem}

\begin{problem}
Пусть при любом $\lambda >0$ с.в. $\xi _{\lambda } $ имеет распределение $\Po(\lambda)$. Докажите, что $(\sqrt{\lambda})^{-1}[\xi _{\lambda } -\lambda]  $ по распределению сходится  к стандартному нормальному распределению при $\lambda \to \infty $.
\end{problem}

\begin{problem} 
Пусть с.в. $x_n\in \Gamma(\lambda,n)$, т.е. неотрицательная случайная величина  с плотностью распределения при $x\geq 0$ равной 
$$
p_{x_n}(x) = \dfrac{x^{n-1}{\rm e}^{-x/\lambda}\lambda^{-n}}{\Gamma(n-1)}
$$
Напомним, что для целых $n$ Гамма-функция принимает значения $\Gamma(n) = (n-1)!$.
Покажите, что из ц.п.т. следует 
$$
\frac{x_n-m(\lambda)\cdot n}{\sigma(\lambda)\cdot\sqrt{n}} \xrightarrow{d} \mathcal{N}(0,1) \text{ при } n\to\infty . 
$$
Найдите $m(\lambda)$, $\sigma(\lambda)$. 
\end{problem}
\begin{remark}
Смотрите замечание к задаче \ref{GammaFunc}.
\end{remark}

%\begin{comment}
\begin{problem}
Пусть $X_n$ --- последовательность независимых с.в., сходящаяся по вероятности к с.в. $X$:  $X_n\xrightarrow{p}X$ при $n\to\infty$. Докажите, 
что с.в. $X$ вырождена, т.е. $X\equiv x$, где $x$ --- некоторое число. 
\end{problem}

\begin{remark}

Справедливы следующие утверждения:

\begin{enumerate}
\item
Из любой сходящейся по мере (в частности, по вероятностной) последовательности 
измеримых функций (в частности, с.в.) можно выделить подпоследовательность, сходящуюся почти всюду (п.н.). 

\item
Из закона нуля и единицы Колмогорова следует, что для всякого разбиения прямой ${\mathbb R}$ на борелевские множества $\{ B_m\}_{m\geqslant 1}$ 
ровно для одного $m=m_0:$ $\quad {\mathbb P}(A_{B_{m_0}})=1$, для остальных $m:\quad {\mathbb P}(A_{B_m})=0$, где 
$$
A_{B_m}=\{ \omega: \, X=\lim\limits_{k\to\infty} X_{n_k}\in B_m \} . 
$$
\end{enumerate}

См. A.S. Cherny, The Kolmogorov student's competitions  on probability theory, а также \cite{22,220}.
\end{remark}

%\end{comment}


\begin{problem}
В некотором городе прошел второй тур выборов. Выбор был между двумя кандидатами $A$ и $B$ (графы <<против всех>> на этих выборах не было). 
Сколько человек надо опросить на выходе с избирательных участков, чтобы исходя из ответов можно было определить долю проголосовавших 
за кандидата $A$ с точностью $3\%$ и с вероятностью не меньшей $0.99$. 
\end{problem}

%\begin{comment}
\begin{problem} [Асимптотическое распределение числа инверсий в случайной перестановке]

На множестве $n!$ перестановок $n$ различных элементов задано равномерное распределение. Обозначим через $\xi_k$ случайную величину, 
равную числу инверсий, образованных элементом с номером $k$, т.е. равную числу элементов с номерами меньшими чем $k$, 
которые стоят в перестановке правее элемента с номером $k$. Покажите, что 
$$
\frac{\sum\limits_{k=1}^{n}\xi_k -\left.n^2\right/4}{\left.n^{3/2}\right/6}\xrightarrow{d} N(0,1) \quad \text{ при } n\to\infty . 
$$
\end{problem}


\begin{remark} Для решения этой и следующей задачи полезно ознакомиться с книгой Сачков В.Н. Комбинаторные методы дискретной математики. М.: МЦНМО, 2004.
$ $

\begin{enumerate}

\item
 Введем с.в. 
$$
\xi_{k,i} = I(\text{<<$k$ находится левее числа $i$>>}) . 
$$

Тогда 
$$
\xi_k=\xi_{k,1}+\xi_{k,2}+\ldots +\xi_{k,k-1}, 
$$

Покажите, что
$${\mathbb E}\xi_k=\left.(k-1)\right/2,$$

$$
{\mathbb E}\xi_k^2={\mathbb E}\xi_{k,1}^2+\ldots+{\mathbb E}\xi_{k,k-1}^2+2\sum\limits_{i<j<k}{\mathbb E}(\xi_{k,i}\xi_{k,j})=
$$
$$
=\frac{k-1}{2}+2\cdot\frac{(k-1)(k-2)}{2}\cdot\frac{1}{3}=\frac{2k^2-3k+1}{6}, 
$$
с.в. $\xi_k$ и $\xi_m$ некоррелированы, $k\ne m$. 


\item 
Для всякой сл.в. $X_k=\xi_k-{\mathbb E}\xi_k$ характеристическая функция имеет вид 
\[
\varphi_{X_k}(t)=1-\frac{t^2 \Var\xi_k}{2}+ o(t^2).
\]

\end{enumerate}
\end{remark}
%\end{comment}

\begin{problem}(Асимптотическое распределение числа циклов в случайной перестановке)
\label{permutation}
Перестановка $\pi $ на $n$ элементах задается пошагово следующим образом: $\pi (k)$ выбирается случайно равновероятно из $\left\{1,\ldots ,n\right\}\backslash \left\{\pi (1),\ldots ,\pi (k-1)\right\}$. Ясно, что вероятность получения фиксированной перестановки будет $(n!)^{-1} $. Пусть случайная величина $\xi _{kn} $, $1\le k\le n$ равна 1, если в качестве $\pi (k)$ выбран элемент, замыкающий цикл перестановки, и равна 0 в противном случае. Воспользовавшись центральной предельной теоремой в форме Ляпунова, покажите, что  число циклов в случайной перестановке $X_{n} =\sum _{k=1}^{n}\xi _{kn} $  имеет асимптотически нормальное распределение.
\end{problem}

\begin{remark}
Интересные результаты об асимптотических свойствах группы перестановок содержатся в работах А.М. Вершика. 
В качестве модели перестановки $g$ длины $n$ можно использовать последовательность чисел $i_1=1,i_2,\dots,i_n$, сгруппированных в циклы, причем каждый цикл начинается с наименьшего в нем числа, и концы циклов указаны. Упорядочим циклы по возрастанию наименьших чисел. 
%Можно показать, что вероятность множества $A_k$ перестановок, у которых на $k$-м шаге оканчивается цикл в перестановке длины $n$, равна $(n-k+1)^{-1}$.
Пример множеств $A_k$: для перестановки $g = ((1,5),(2,3,8,6),(4,7))$, $g\in A_2 \bigcap A_6 \bigcap A_8$. Определим отображение $g\to (x_1(g),\dots,x_n(g))$, где $x_i(g)$ --- длина $i$-го цикла, нормированная на $n$, которая переводит меру, заданную на перестановках (по условию вероятность получения фиксированной перестановки будет $(n!)^{-1}$)  в дискретную меру на единичном  симплексе. Оказывается, у такой последовательности мер при $n$ стремящемся к бесконечности существует слабый  предел. Предельная мера порождается случайным рядом $\eta_i$, $i=1,2,\dots,$ с неотрицательными значениями и суммой $1$, причем значения $\eta_k/(\eta_k+\eta_{k+1}+\dots)$ независимы и равномерно распределены  на отрезке $[0,1]$. Иллюстрацией к приведенному результату является задача о "ломании палочки". Отрезок делится с равномерной вероятностью (точка деления распределеная равномерно на $[0,1]$). Левый отрезок фиксируется и затем ломается правый отрезок. Точки разлома выбираются последовательно с помощью последовательности независимых равномерно распределенных на $[0,1]$ случайных величин (или же можно сказать, что распределение длины левого отломанного куска палочки есть $\mathrm{Beta}(1, 1)$).  В работах А.М. Вершика показано, что такое случайное разбиение отрезка задает распределение длин отрезков, сопадающее с распределением длин циклов в рассмотренной модели перестановок. 

Доказательство этого факта технически нетривиально, например, используется принцип инвариантности 
Донскера--Прохорова.  
%Пусть $r_1,\dots,r_k$--- реализация случайных величин  $\xi_1,\dots,\xi_n$. %Пусть $x_i(g)$ --- нормированная на $n$ длина цикла, содержащего не вошедший в первые $i-1$ циклов элемент с наименьшим номером, если $i$ не превосходит числа циклов подстановки $g$.
%Нетрудно убедиться, что 
%\begin{equation}
%\label{versh}
%\mathbb{P}(g|nx_i(g)=r_i,1\leq i\leq m) = \left[n\prod_{i=1}^{m-1}(n-\sum_{k=1}^i r_k)\right]^{-1}.
%\end{equation}
%Видно, что в \eqref{versh} происходит отображение куба  $Q^n=\prod_{i=1}^{n}[0,1]$ на симплекс. Обозначим это отображение за $T_n$. 
%Тогда обратное отображение 
%$T_n^{-1}$, действующее из симплекса в $Q^n$:  
%$\{T_n^{-1}(x)\}_i = \frac{x_i}{1-x_1-\dots-x_{i-1}}$, если $x_k=0$, для всех $k\geq i$, то $\{T_n^{-1}(x)\}_i = 1$.
%Формулы применимы и к бесконечным последовательностям, то есть задают отображение бесконечного куба на бесконечномерный симплекс (сходящихся рядов с суммой единица). Отсюда (см. ссылки ниже) следует результат о том, что нормированные длины циклов случайных подстановок сходятся со скоростью геометрической прогрессии с показателем ${\rm e}^{-1}$
Подробные доказательства имеются в статьях Вершика А.\,М., Шмидта А.\,М.  Предельные меры, возникающие в асимптотической теории симметрических групп. I, ТВП, 1977, том 22, выпуск 1, 72–88 и Вершика  А.\,М., Шмидта А.\,А., Предельные меры, возникающие в асимптотической теории симметрических групп. II, ТВП, 1978, том 23, выпуск 1, 42–54. 



\end{remark}

\begin{problem}(Предельные меры А.М. Вершик и др.) 
\label{permutation1}
В качестве множества элементарных исходов рассматривается группа всевозможных 
подстановок (перестановок) $\mathbb{S}_n $ (симметрическая группа), $n\gg 1$. В этой 
группе $n!$ элементов. Припишем каждой подстановке одинаковую вероятность $1 
\mathord{\left/ {\vphantom {1 {n!}}} \right. \kern-\nulldelimiterspace} 
{n!}$.

\begin{enumerate}

\item Покажите, что математическое ожидание числа циклов есть $\approx 
\ln n$ (см. также задачу \ref{permutation}).

\item\Star В каком смысле нормированные длины циклов случайной подстановки 
убывают со скоростью геометрической прогрессии со знаменателем $e^{-1}$ ?

\item\Star Положим $$\rho _n \left( a \right)={\left| {\left\{ {g\in \mathbb{S}_n:\;n_{\max } \left( g \right)\le an} \right\}} \right|} \mathord{\left/ 
{\vphantom {{\left| {\left\{ {g\in S_n :\;n_{\max } \left( g \right)\le an} 
\right\}} \right|} {n!}}} \right. \kern-\nulldelimiterspace} {n!},$$ где 
\mbox{$n_{\max } \left( g \right)$~--- длина} максимального цикла в подстановке $g$. 
Покажите, что $\rho _n \left( a \right)$ удовлетворяет \textit{уравнению Дикмана--Гончарова} (40-е годы XX 
века):
\[
\rho _n \left( a \right)=\int\limits_0^a {\rho _n \left( {\frac{a}{1-t}} 
\right)dt}.
\]
\item\Star\,\,\Star Покажите, что начиная с некоторого большого числа $N$ 99{\%} 
натуральных чисел $n$, больших, чем $N$ обладают свойством
\[
n^{0.99}<p_1 \cdot ...\cdot p_{11} .
\]
Иначе говоря, у основной части (99{\%}) натуральных чисел основная часть 
(99{\%}) числа есть произведение наибольших простых делителей. Число 11 
возникло из-за того, что мы выбрали 99{\%} и 99{\%}.
\end{enumerate}
\end{problem}
\begin{ordre} Решение задач всех пунктов сводится к задаче о ``ломании палочки'' (см. замечание  к предыдущей задаче).
%См. задачи \ref{po_dir},\ref{stick} из раздела \ref{bayes}.
\end{ordre}

\begin{remark} См. Вершик А.М., Шмидт А.А. Предельные меры, возникающие 
в асимптотической теории симметрических групп // ТВП, Т. 22. № 1. 1977.С. 
72--88; Т. 23. № 1. 1978. С. 42--54; Вершик А.М. Асимптотическое 
распределение разложений натуральных чисел на простые делители // ДАН. 1986. 
Т. 289. № 2. С. 269--272; Tenenbaum G. Introduction to analytic and 
probabilistic number theory. Cambridge Univ. Press, 1995; Arratia R., 
Barbour A.D., Tavar\'{e} S. Logarithmic combinatorial structures: A 
probabilistic approach. EMS Monogr. Math., Eur. Math. Soc., Z\"{u}rich, 
2003.

Интересно отметить, появление по ходу решения задачи распределения 
(Пуассона--)Дирихле (см. задачу \ref{sobord}). Контекст, в котором это распределение возникает в 
решении, объясняет важность этого распределения в приложениях.
\end{remark}

\begin{problem}[Закон повторного логарифма]
Пусть $X_n$~--- независимые одинаково распределённые случайные величины с нулевым математическим ожиданием и единичной дисперсией. Пусть $S_n = X_1+\ldots+ X_n$. Докажите, что почти наверное
\[\underset{n \rightarrow \infty}{\overline{\lim} } \frac{S_n}{\sqrt{n \log_2 (\log_2 n)}} = \sqrt{2},\]
\[\underset{n \rightarrow \infty}{\underline{\lim} } \frac{S_n}{\sqrt{n \log_2 (\log_2 n)}} = -\sqrt{2}.\]
\end{problem}

\begin{remark} 

Хотя величина $\frac{S_n}{\sqrt{n \log_2 \log_2 n}}$ будет меньше, чем любое заданное  $\varepsilon$  с вероятностью, стремящейся к единице (это следует из ц.п.т), она будет бесконечное число раз приближаться сколь угодно близко к любой точке отрезка [$-\sqrt{2}, \sqrt{2}$] почти наверное [\ref{chiraiev}] т.2.

Приведем также некоторые смежные результаты из \cite{Gupta}. Пусть $X_1$, $X_2$, \dots, независимые однаково распределенные случайные величины с $F$ и $\gamma_n$ расходящаяся последовательность (к $+\infty$). Пусть $Z_{n,\gamma}  = \frac{S_n}{\gamma_n}$, $n\geq 1$, и $B(F,\gamma)$ --- множество всех предельных точек $Z_{n,\gamma}$. Тогда существует неслучайное множество $A(F,\gamma)$, такое, что с вероятностью $1$ $B(F,\gamma)$ совпадает c $A(F,\gamma)$.  В частности:
\begin{itemize}
\item Если $\gamma_n = n^{\alpha}$, $0<\alpha<1/2$, тогда для всех $F$ не вырожденных в $0$, $A(F,\gamma)$ равно всей расширеной вещественной оси, если оно содержит хотя бы одно конечное вещественное число.
\item Если $\gamma_n = n$ и если $A(F,\gamma)$ содержит хотя бы два конечных вещественных числа, то тогда оно должно содержать $\pm\infty$.
\item Если $\gamma_n=1$ тогдa $A(F,\gamma) \in \{\pm\infty\}$ тогда и только тогда, если для некоторого $a>0$, $\int_{a}^a \text{Re}\{1/(1-\psi(t))\}\,dt\leq \infty$, где $\psi(t)$ --- характеристическая функция $F$.
\item Пусть $\mathbb{E}(X_1)=0$, $\mathbb{D}X_1 = \sigma^2\leq{\infty}$, тогда при $\gamma_n=\sqrt{2n\log(\log n)}$, $A(F,\gamma) = [-\sigma,\sigma]$, а если $\mathbb{E}(X_1)=0$, $\mathbb{D}X_1 = \infty$, тогда при $\gamma_n=\sqrt{2n\log(\log n)}$, то $A(F,\gamma)$ содержит по крайней мере одну из $\pm\infty$
\end{itemize}
Пусть $m_n$ таковы, что $P(S_n\leq m_n)\geq 1/2$ и $P(S_n\geq m_n)\geq 1/2$. Тогда существует такая положительная последовательность $\gamma_n$, удовлетворяющая $$-\infty< \underset{n \rightarrow \infty}{\underline{\lim} } \frac{S_n-m_n}{\gamma_n}<
\underset{n \rightarrow \infty}{\overline{\lim} }\frac{S_n-m_n}{\gamma_n}<\infty$$ тогда и только тогда, когда для всех $c\geq 1$
$$\underset{x\to\infty}{\lim\inf}\frac{P(|X_1|>cx)}{P(|X_1|>x)}\leq c^{-2}.$$
\end{remark}



\begin{problem}[Логнормальное распределение]
\label{lognorm}
Рассмотрим частицу, которая при перемещении из одного места
в другое может разделиться на несколько меньших частиц вследствие соударения или другого воздействия. Зафиксируем произвольную точку данной частицы.  Обозначим за $K(t) \sim \Po(\lambda t)$ число отделившихся частей от изначально одной частицы относительно зафиксированной точки к моменту времени $t$. Первоначальный размер частицы равен $s_0$. Пусть $D_i$  -- доля частицы, отделившаяся при $i$-м соударении. Тогда размер
частицы в момент времени $t$ имеет вид:
\imgh{70mm}{log_norm_dist.pdf}{График функции плотности логнормального распределения при значении среднего $\mu = 0$ и различных значениях среднеквадратического отклонения (standard deviation).}
\[
Z(t) = s_0 \prod \limits_{i=1}^{K(t)} (1 - D_i),
\]
\[
S(t) = \ln Z(t) =  \mu +  \sum \limits_{i=1}^{K(t)} X_i,
\]
\noindent где $\mu = \ln s_0$, $X_i = \ln (1 - D_i)$ -- независимые с.в. (по предположению), причем $\Exp X = a$, $\Var X = \sigma^2$. 

Покажите, что $Z(t)$ стремится к логнормальному распределению при $n\to\infty$ (см. замечание):
\[
Z(t) \xrightarrow{d} \mathrm{LogN} \left(\mu + \lambda t a, \lambda t (a^2 + \sigma^2) \right).
\] 
\end{problem}

\begin{remark}
Cм. Mitzenmacher M., A Brief History of Generative Models for Power Law and Lognormal Distributions. Internet Math. 1 no. 2, 2003, 226--251.

Говорят, что $X$  имеет логнормальное распределение $\mathrm{LogN}(\mu,  \sigma^2)$ с параметрами $\mu$ и $\sigma$, если распределение задаётся плотностью вероятности, имеющей вид (см. Рис. \ref{Fig:log_norm_dist.pdf}):
\[
f_X(x) = \frac{1}{x \sigma \sqrt{2 \pi}} e^{- \frac{(\ln x - \mu)^2}{2 \sigma^2}}, \; x > 0.
\]
\end{remark}

Если после логарифмирования каждого элемента некоторого набора данных трансформированный набор данных нормально распределен, то исходные данные логарифмически нормально распределены.

Данное распределение хорошо моделирует процессы в случае, когда значение наблюдаемой переменной является случайной долей от значения предыдущего наблюдения.

%Примерами использования логнормального  распределения могут быть: 
%а) размеры и вес частиц, образуемых при дроблении;
%б) доход семьи;
%в) зарплата работников;
%г) долговечность изделия, работающего в режиме износа;
%д) размер банковского вклада;
%е) длины слов в языке;
%ж) длины передаваемых сообщений, размеры файлов или длины запросов к %базе данных.


%\subsection{ Безгранично делимые распределения }



\begin{problem}[Пуассоновский процесс \cite{1}]
\label{sec:poisson}
Пусть необходимо оценить, сколько билетов на метро одного вида $K(T)$ 
продается за одну рабочую смену длительностью $T$. Имеет место формула 
$$
K(T)=\max\Bigl\{ n:\; \sum\limits_{k=1}^{n} X_k<T \Bigr\} , 
$$
где $X_1, X_2, X_3,\ldots$~--- независимые одинаково распределенные по закону ${\rm Exp}(\lambda)$ с.в. ($X_k$ интерпретируется как время между 
$k-1$ и $k$ сделкой (продажей)). Покажите, что 
\begin{enumerate}
\item вероятность ${\mathbb P}(K(T+t)-K(T)=k)$, где $t\geqslant 0$ и $k=0,1,2,\ldots$, не зависит от $T\geqslant 0$; 

\item $\forall n\geqslant 1$, $0\leqslant t_1\leqslant t_2\leqslant \ldots\leqslant t_n$ 
 с.в. $\bigl\{ K(t_k)-K(t_{k-1})\bigr\}_{k=1}^{n}$ независимы; 

\item ${\mathbb P}(K(t)>1)=o(t),\quad t>0$; 

\item $\forall n\geqslant 1$, $0\leqslant t_1\leqslant t_2\leqslant \ldots\leqslant t_n$, 
$0\leqslant k_1\leqslant k_2\leqslant \ldots\leqslant k_n$, $k_1,\ldots, k_n\in {\mathbb N}\cup \{ 0\}$ 
\begin{multline*}
{\mathbb P}(K(t_1) = k_1,\ldots, K(t_n)=k_n)= \\
e^{-\lambda t_1} \frac{(\lambda t_1)^{k_1}}{k_1!}\cdot 
e^{-\lambda(t_2-t_1)} \frac{(\lambda(t_2-t_1))^{k_2-k_1}}{(k_2-k_1)!}\cdot \ldots \\
e^{-\lambda(t_n-t_{n-1})}\frac{(\lambda(t_n-t_{n-1}))^{k_n-k_{n-1}}}{(k_n-k_{n-1})!} , 
\end{multline*}
в частности, $K(T)\in \Po(\lambda T)$. 

\end{enumerate}
\end{problem}
\begin{remark}
Обратите внимание на задачу \ref{exp_eps}.
\end{remark}



\begin{problem}[Сложный пуассоновский процесс \cite{1}]
\label{sec:cpoisson}
В течение рабочего дня фирма осуществляет $K(T)\in \Po(\lambda T)$ сделок ($K(T)$ --- с.в., имеющая распределение Пуассона с параметром 
$\lambda T$, где $\lambda = 100 [\text{сделок/час}]$). Каждая сделка приносит доход $V_n\in R[a,b]$ ($V_n$ --- с.в., имеющая 
равномерное распределение на отрезке $[a,b]=[10, 100]$, $n$ --- номер сделки). Считая, что $K$, $V_1$, $V_2$, $\ldots$ --- 
независимые в совокупности с.в., найдите математическое ожидание и дисперсию выручки за день $Q(T)=\sum\limits_{k=1}^{K(T)} V_k$. Докажите соотношение для характеристической функции $Q(T)$
$$
\varphi_{Q(T)}(t)=\exp\{ \lambda T(\varphi_{V_k}(t)-1)\}. 
$$
Найдите (см. \cite{5}) такие $m(T)$ и $\sigma(T)$, что при $T\to\infty$
$$\frac{Q(T)-m(T)}{\sigma(T)}\xrightarrow{d} \mathcal{N}(0,1).$$
\end{problem}

\begin{ordre}

Примените формулу для условного математического ожидания  
$$
{\mathbb E}Y= \Exp{( {\mathbb E}(Y|X) )} 
$$
Установите справедливость следующего соотношения.
$$
\Var Y=\Var({\mathbb E}(Y|X))+{\mathbb E}(\Var(Y|X)) . 
$$

\end{ordre}


\begin{problem}
В течение трех лет фирма из предыдущей задачи работала $N=1000$ дней (длина рабочего дня и параметры спроса не менялись). 
Оцените (см. \cite{5}) распределение с.в. 
$$Q^N=\sum\limits_{k=1}^{N} Q_k(T), 
$$
где $Q_k(T)$ --- выручка за $k$–й день. Верно ли, что с.в. $Q^N$ и $Q_k(NT)$ одинаково распределены? 
\end{problem}

\begin{comment}
\begin{problem}
В течение года фирма осуществляет $K\in \Po(\lambda)$ сделок ($K$~--- с.в., имеющая распределение Пуассона с параметром  $\lambda=100000$ 
[сделок]). Каждая сделка приносит фирме прибыль $V_n\in R[a,b]$ ($V_n$ -- с.в., имеющая равномерное распределение на отрезке 
$[a,b]=[-50\$,100\$]$, $n$ -- номер сделки). Считая, что $K$, $V_1$, $V_2$, $\ldots$ --- независимые в совокупности с.в., оцените 
\begin{equation}
\label{ProbRatio}
\left. {\mathbb P}\Bigl(\sum\limits_{n=1}^{K} V_n\leqslant 0\Bigr)\right/{\mathbb P}\Bigl(\sum\limits_{n=1}^{K} V_n>0\Bigr). 
\end{equation}
\end{problem}
\end{comment}

\begin{problem}[Пуассоновский поток событий \cite{27,202}]
Рассмотрим интервал 
$\left[ {-N,N} \right]$ и бросим на него независимо и случайно (равномерно) \mbox{$M=\left[ {\rho N} \right]$} точек, где $\rho >0$ -- некоторая 
константа, называемая плотностью. Легко вычислить биномиальную вероятность 
$\PR_{N,M} \left( {k,I} \right)$ того, что в конечный интервал $I\subset 
\left[ {-N,N} \right]$ попадет ровно $k$ точек. Покажите, что для $\PR_{N,M} 
\left( {k,I} \right)$  при $N\to \infty $ верно 
\[
\PR\left( {k,I} \right)\mathop =\limits^{def} \mathop {\lim }\limits_{N\to 
\infty } \PR_{N,M\left( N \right)} \left( {k,I} \right)=\frac{\left( {\rho 
\left| I \right|} \right)^k}{k!}e^{-\rho \left| I \right|},
\quad
k=0,1,...
\]
Покажите также, что если $I_1 ,I_2 \subset \left[ {-N,N} \right]$ и $I_1 
\cap I_2 =\emptyset $, то
\[
\PR \left( {k_1 ,I_1 ;k_2 ,I_2 } \right)\mathop =\limits^{def} \mathop {\lim 
}\limits_{N\to \infty } \PR_{N,M\left( N \right)} \left( {k_1 ,I_1 ;k_2 ,I_2 } 
\right)=\PR\left( {k,I_1 } \right)\PR\left( {k,I_2 } \right).
\]
\end{problem}




\begin{problem}[Безгранично делимые случайные величины]
\label{sec:infdiv}
Случайная величина $X$ называется \textit{безгранично делимой}, если для любого натурального $n$ найдутся $n$ независимых одинаково распределенных случайных величин $X_{kn}$ таких, что $X = \sum_{k=1}^{n}{X_{kn}}$, где равенство понимается по распределению. Теорема Колмогорова--Леви--Хинчина утверждает, что если $X$ ~-- безгранично делимая с.в., то существует тройка
$$
 (b,c,\nu(dx)): \; 
c\geqslant 0, \; \int_{-\infty}^{\infty} \max(1,x^2)\, \nu(dx)<\infty, \; \nu(dx)\geqslant 0 \text{, такая что} 
$$
\[
\varphi_{X}(\mu)
=\exp \Bigl\{  i\mu b-c\mu^2\left.\right/2+
\int_{-\infty}^{\infty} \bigl( e^{i\mu x}-1-i\mu x I(|x|\leqslant 1) \bigr)\, \nu(dx) 
\Bigr\}, 
\]
где $\varphi_{X}(\mu)$~--- характеристическая функция $X$.

Верно ли, что любая безгранично делимая с.в. может быть представлена (имеет такое же распределение), как $N(m, \sigma^2) + Q$, где $Q$ -- с.в., имеющая сложное распределение Пуассона?
Определите $(b,c,\nu(dx))$ для $Q(T)$ из задачи 
\ref{sec:cpoisson}.

\end{problem}
\begin{remark} См. также задачу \ref{sobord}. 
В литературе можно часто встретить другую запись теоремы Колмогорова--Леви--Хинчина, например 
\[
\varphi_{X}(\mu)
=\exp \Bigl\{  i\mu b_0-c\mu^2\left.\right/2+
\int_{-\infty}^{\infty} \bigl( e^{i\mu x}-1-i\frac{\mu|x|}{1+x^2} \bigr)\, \theta(dx) 
\Bigr\}.
\]
Эквивалентность приведенной записи и записи в условии задачи следует из простого соотношения:
\[
\frac{y^2}{1+y^2}\leq \max\{1,y^2\}\leq \frac{2y^2}{1+y^2}.
\]

Из нетривиальных примеров безгранично делимых распределений упомянем распределения Стьюдента, Коши, логнормальное, показательное. Последнее является примером безгранично де\-ли\-мо\-го распределения, которое может быть представлено в виде бесконечной свертки (независимых, одинаково распределенных с.в.) не безгранично-делимых с.в. Для нормального  распределения и распределения Пуассона такое представление не возможно. Теоремы Крамера и Райкова говорят соответственно, что если свертка двух распределений нормальна (распределена по закону Пуассона), то компоненты должны также иметь нормальное распределение (распределение Пуассона) с другими параметрами. Подробнее об этом см. \cite{stoianov}. 

Приведем для справки несколько фактов из \cite{Gupta}.
\begin{enumerate}
\item
Пусть $F$ ---  распределение безгранично делимой случайной величины со средним $b$ и конечной дисперсией $D_F$ и  $\phi(t)$ --- характеристическая функция. Тогда 
$$
\log\phi(t)=ibt+\int_{-\infty}^{\infty}({\rm e}^{itx}-1-itx)\frac{d\mu(x)}{x^2},
$$
где $\mu$ конечная мера на действительной оси, более того $\mu(\mathbb{R}) = D_F$.
\item Теорема Goldie--Steutel. Пусть положительная случайная величина имеет плотность распределения $f(x)$, причем $f$ бесконечно дифференцируема и $(-1)^k f^{(k)}(x)\geq 0$, $\forall x>0$. Тогда $X$ имеет безгранично делимое распределение. 
\end{enumerate}

Также следует обратиться к Гнеденко Б.\,В., Колмогоров А.\,Н. Предельные распределения для сумм независимых случайных величин. — М.-Л.: ГТТИ, 1949.

\end{remark}

\begin{problem}[Центральная предельная теорема без требования существования дисперсий]
Пусть $X_1,X_2, 
\dots$ независимые одинаково распределенные случайные величины из распределения $F$ на действительнной прямой.  Верна следующая теорема 
\cite{Gupta}: cуществуют такие последовательности $\{a_n\},\{b_n\}$ что при $n\to \infty$ $$\frac{\sum_{i=1}^n(X_n-a_n)}{b_n}\xrightarrow{d} \mathcal{N}(0,1)$$ тогда и только тогда, когда функция $v(x) = \int_{[-x,x]}y^2\,dF(y)$ является медленно меняющейся на бесконечности, то есть для нее верно при любом $t>0$, что $$\lim_{x\to\infty}\frac{v(tx)}{v(x)}=1.$$ 
Покажите, что для  распределения Стьюдента с двумя степенями свободы выполняется приведенная теорема. Найдите нормировочные последовательности $\{a_n\}$ и $\{b_n\}$.
\end{problem}
 \begin{remark}
 Напомним, что по определению распределение Стьюдента c $k$ степенями свободы есть распределение случайной величины $$t = \frac{\xi_0}{\sqrt{\frac{1}{k}(\xi^2_1+\dots+\xi^2_k)}},$$
 где $\xi_j$ независимы, $\xi_j\in \mathcal{N}(0,1)$, $j=1,\dots,k$. Плотность распределения Стьюдента 
 $$
 p(x) = \frac{\Gamma((k+1)/2)}{\sqrt{\pi k}\Gamma(k/2)} \left(1+\frac{x^2}{k}\right)^{-(k+1)/2}.
 $$
 Распределение Стьюдента крайне важное распределение в математической статистике, которое, в частности, возникает в задачах, касаяющихся оценивания неизвестного математического ожидания выборки из нормального распределения с неизвестной дисперсией.
 \end{remark}

\begin{problem}\label{bluzd_ust}
Рассмотрим простую и классическую схему блуждания точки по прямой, соответствующую правилам игры в орлянку:
\[\eta (0)=0,\] 
\[\eta (t+1)=\left\{\begin{array}{cc} {\eta (t)+1,} & {p={1\mathord{\left/ {\vphantom {1 2}} \right. \kern-\nulldelimiterspace} 2} }, \\ {\eta (t)-1,} & {p={1\mathord{\left/ {\vphantom {1 2}} \right. \kern-\nulldelimiterspace} 2} }. \end{array}\right. \] 
Занумеруем в порядке возрастания все моменты времени, когда $\eta (t)=0$. Получим бесконечную последовательность $0=\tau _{0} <\tau _{1} <\tau _{2} <...$ Рассмотрим разности $\xi _{i} =\tau _{i} -\tau _{i-1} $, $i=1,2,...$ -- последовательность независимых одинаково распределенных с.в.
\begin{enumerate}
\item Найдите распределение $\xi _{i} =\tau _{i} -\tau _{i-1} $, т.е. $\PR\left\{\xi _{i} =2m\right\}$.
\item Покажите, что математическое ожидание с.в. $\xi _{i} =\tau _{i} -\tau _{i-1} $ равно бесконечности.
\end{enumerate}
\end{problem}
\begin{remark}
Этот результат можно проинтерпретировать так: среднее время до первого возвращения блуждания в 0 бесконечно. 
Тем не менее суммы $\tau _{n} =\sum _{i=1}^{n}\xi _{i}  $ при надлежащей нормировке подчинены предельному распределению: 
\[\mathop{\lim }\limits_{n\to \infty } \PR\left\{\frac{2\tau _{n} }{\pi n^{2} } <z\right\}=\left\{\begin{array}{cc} {\frac{1}{\sqrt{2\pi } } \int _{0}^{z}e^{-\frac{1}{2x} } x^{-\frac{3}{2} }  dx,} & {z>0} \\ {0,} & {z<0} \end{array}\right. .\]  
Случайная величина $\zeta$ имеет устойчивый закон распределения, если для любого $N>1$ найдутся  независимые случайные величины $\{Z_n\}_{n=1}^N$, распределение которых совпадает с  распределением $\zeta$,  и постоянные $a_N$ и $b_N$, такие что $\zeta = \frac{1}{a_N}(\sum_{i=1}^N z_n - b_N)$. Причем верно следующее утверждение (см. \cite{21}): $\zeta$ может быть пределом по распределению с.в. $\frac{1}{a_N}(\sum_{i=1}^N z_n - b_N)$ тогда и только тогда, когда $\zeta$ имеет устойчивый закон распределения.  Также верна теорема о каноническом представлении устойчивых законов  Леви--Хинчина \cite{1}: для того чтобы функция распределения была устойчивой, необходимо и достаточно, чтобы логарифм ее характеристической функции представлялся формулой:
\[\ln \varphi (t)=i\gamma t-c|t|^{\alpha } \left(1+i\beta \frac{t}{|t|} \omega (t,\alpha )\right),\] 
где $\gamma\in\mathbb{R}$, $-1\le \beta \le 1$, $0<\alpha < 2$, $c\ge 0$ и
\[\omega (t,\alpha )=\left\{\begin{array}{cc} {\tg\left(\frac{\pi }{2} \alpha \right),} & {\alpha \ne 1,} \\ {\frac{2}{\pi } \ln |t|,} & {\alpha =1}. \end{array}\right. \] 
Из данной теоремы следует, что такое распределение соответствует каноническому представлению с $\alpha ={1\mathord{\left/ {\vphantom {1 2}} \right. \kern-\nulldelimiterspace} 2} $, $\beta =1$, $\gamma =0$, $c=1$, и принадлежит семейству кривых Пирсона (показано Н.В. Смирновым).



Для решения этой и следующей задачи полезно познакомиться с \cite{28}. Для более глубокого погружения рекомендуется книга Петров В.В. Предельные теоремы для сумм независимых случайных величин. М.: Наука, 1987. 
\end{remark}

\begin{problem}
Верно, что устойчивые законы  распределения являются также и  безгранично делимыми? 
 Покажите, что Пуассоновский закон распределения безгранично делимый, но не устойчивый. 
\end{problem}
\begin{remark}
См. предыдущую задачу и \cite{stoianov}.
\end{remark}

\begin{problem}[Вывод распределения Хольцмарка \cite{202,28}] Рассмотрим шар \mbox{радиуса $r$} с центром в начале координат и $n$ звезд (точек), расположенных в нем случайно и независимо друг от друга. Пусть каждая звезда имеет единичную массу и звезды распределены по Пуассону так, что ожидаемое число их в объеме $V$ равно $\lambda V$. Обозначим $X_{i}$ (вектор) гравитационное поле, соответствующее $i$-й звезде, и положим $S_{n} =X_{1} +...+X_{n} $. Устремим $r$ и $n$ к бесконечности так, чтобы $$\frac{4}{3} \pi r^{3} n^{-1} \to \lambda.$$ Показать, что плотность распределения вектора $S_{n}$ зависит только от модуля поля и стремится к плотности симметричного устойчивого распределения Хольцмарка с $\alpha = 3/2$ (см. \cite{28}). Можно показать, что задача по существу не изменится, если массу каждой звезды считать с.в. с единичным математическим ожиданием и массы различных звезд предполагать взаимно независимыми с.в., не зависящими также от их расположения.
\end{problem}
\begin{remark}
Смотрите также Кендалл М., Моран П. Геометрические вероятности. М.: Наука, 1972.


\end{remark}


\begin{problem} Пусть $n$ единичных масс равномерно распределены в точках $X_1, \ldots, X_n$ на отрезке $[-n, n]$. На единичную массу в начале координат действует гравитационная сила $$f_n = \sum_{k =1}^n \frac{{\rm sign}(X_k)}{X_k^2}.$$ 


Покажите, что $$\mathbb{E}\left[\exp{\left(it f_n\right)}\right] \rightarrow \exp{\left(-c\sqrt{|t|}\right)}, \quad n\to\infty.$$
\end{problem}
\begin{ordre}
Так как $X_k$ распределено равномерно, то (см. \cite{2013})
\begin{equation*}
\mathbb{E}\left[\exp{\left(it \frac{{\rm sign}(X_k)}{X_k^2}\right) }\right] = \int_{-n}^n \exp{\left(it\frac{{\rm sign}(x)}{x^2}\right) }\frac{dx}{2n} = \frac{1}{n}\int_{0}^n \cos{\left(\frac{t}{x^2}\right)}dx.
\end{equation*}
\end{ordre}

\begin{problem} \label{sobord} Покажите, что если $X\left( t \right)$ -- 
процесс Леви, то:
\[
\exists \;\;\left( {b,\;c,\;\nu \left( {dx} \right)} \right),\,
c\ge 0,\,\nu \left( {dx} \right)\ge 0:
\quad
\int_{-\infty }^\infty {\max \left( {1,\;x^2} \right)\;}
\nu \left( {dx} 
\right)<\infty ,\]
\[
\forall \;\;t\ge 0 \quad 
\phi _{X\left( t \right)} \left( \mu \right)=\mathbb{E}{\rm e}^{i\mu X\left( t 
\right)}=
\]
\[\exp \left\{ {t\left[ {i\mu b-{c\mu ^2} \mathord{\left/ {\vphantom 
{{c\mu ^2} 2}} \right. \kern-\nulldelimiterspace} 2+\int_{-\infty }^\infty 
{\left( {e^{i\mu x}-1-i\mu xI\left( {\left| x \right|<1} \right)} \right)\nu 
\left( {dx} \right)} } \right]} \right\}.
\]
\end{problem}
\begin{remark}
 Процессом Леви  $\left\{ {X\left( t 
\right)} \right\}_{t\ge 0} $ называется стохастически непрерывный случайный 
процесс, удовлетворяющий следующим условиям (с небольшими оговорками (cadlag--процесс) -- см.  Applebaum D. B. Levy processes and stochastic calculus, Cambridge University Press 2nd ed., 2009):

\begin{enumerate}
\item $X\left( 0 \right)\mathop =\limits^{\text{п.н.}} 0;$
\item для любых $t>s\ge 0$ распределение $X\left( t \right)-X\left( s \right)$ зависит только от $t-s$ (также говорят, что $\left\{ {X\left( t \right)} \right\}_{t\ge 0} $ имеет стационарные приращения или $\left\{ {X\left( t \right)} \right\}_{t\ge 0} $ -- однородный);
\item для любых $n\in {\rm N}$, $0\le t_0 \le t_1 \le ...\le t_n $ выполняется: $X\left( {t_1 } \right)-X\left( {t_0 } \right)$, {\ldots}, $X\left( {t_n } \right)-X\left( {t_{n-1} } \right)$ -- независимые в совокупности с.в. (также говорят, что $\left\{ {X\left( t \right)} \right\}_{t\ge 0} $ имеет независимые приращения).
\end{enumerate}
См. А.Н. Ширяев, Основы финансовой 
стохастической математики, М.: ФАЗИС, 2004, т. 1, глава 3, п. 
1.
\end{remark}

\begin{remark}
Отметим, что мера $\nu$ может быть сигнулярна  в нуле, и такие процессы могут иметь бесконечно много  скачков на любом непустом временном интервале.

Процесс с положительными независимыми приращениями, обладающий свойствами б), в), а также тем свойством, что приращения его положительны, называется субординатором. Если $X(t)$-- субординатор, то cуществуют такие $a\in\mathbb{R}$ и мера $\nu(dz)$, что $\int_{0}^{\infty}x\nu(dz)<\infty$ и для всех $\lambda\geq 0$ и $t\geq0$ верно
 $$\mathbb{E}\exp{(-\lambda X(t))} = \exp\left(-t\Phi(\lambda)\right),$$ 
 где $\Phi(\lambda) = b\lambda+\int_{0}^{\infty}(1-{\rm e}^{-\lambda z}) \nu(dz)$ и 
 $b = a-\int_{0}^{1} x\nu(dx)$.
 Такой процесс совершает бесконечное счетное  число скачков на любом конченом интервале времени.  
 %Субординатор не может иметь неподвижных точек разрыва, поскольку множество точек разрыва счетно и, следовательно не может быть инвариантным относительно сдвигов (что отражено в его представлении, приведенном выше).
 
 Рассмотрим пример субординатора Морана (также его называют гамма-субординатором). Субординатором Морана называется возрастающий процесс $G(\alpha),\, \alpha>0$ с независимыми стационарными приращениями $G(\alpha_2)-G(\alpha_1), \alpha_2>\alpha_1$ с плотностью распределения  $g_{\alpha_2-\alpha_1}(x)$, где $g_{\alpha}(x)= x^{\alpha-1} {\rm e}^{-x}\Gamma(\alpha)$.
Для субординатора Морана $a=0$, $\nu(dx) = x^{-1}{\rm e}^{-x}dx$. 
Таким образом, субординатор Морана --- это процесс Леви, отвечающий гамма-распределению.

Этот процесс может быть использован для генерирования последовальностей случайных величин с распределением Пуассона--Дирихле, а именно распределение Пуассона--Дирихле есть распределение упорядоченных по убыванию нормированных скачков субординатора Морана. Распределение Пуассона--Дирихле чрезвычайно часто возникает в разнообразных приложениях, особенно в популяционной генетике и экономике (оно является равновесным распределением для ряда эволюционных моделей). С другой стороны распределение Пуассона--Дирихле появляется как предел распределения Дирихле на конечномерных симплексах при размерности симплекса стремящейся к бесконечности и однаковыми параметрами, нормированными на размерность симплекса. См. также \cite{202}.
\end{remark}

\begin{problem}(модель Кокса--Росса--Рубинштейна) 
\label{sec:levilong}
Пусть на ``идеализированном'' фондовом рынке имеется всего две ценные 
бумаги, и торговля осуществляется всего в два момента времени. Пусть цена 
первой бумаги $S$ (будем называть её акцией (stock)) известна в первый 
момент. Цена второй бумаги $C$ (будем называть ее call -- опционом 
европейского типа) не известна в первый момент. Пусть с ненулевой 
вероятностью $p>0$ (мы это $p$ не знаем, но от него ничего зависеть в итоге 
не будет) к моменту времени 2 цена акции вырастет в $u>1$ (up) раз и с 
вероятностью $1-p$ цена акции ``вырастет'' в $d<1$ (down) раз, т.е. упадет. 
Пусть также известны возможные цены опциона во второй момент: $C_u $, если 
акция выросла в цене, и $C_d $,  если акция упала в цене. Для простоты будем 
считать, что банк работает с нулевым процентом, т.е. класть деньги в банк, в 
расчете на проценты, бессмысленно. Говорят, что рынок 
безарбитражный, если не существует таких $k_S $, $k_C $, что
\[
X\left( 1 \right)=k_S S+k_C C=0,
\]
\[
\PR\left( {X\left( 2 \right)\ge 0} \right)=\PR\left( {k_S S\left( 2 \right)+k_C 
C\left( 2 \right)\ge 0} \right)=1,\] причем $\PR\left( {X\left( 2 \right)>0} 
\right)>0.
$

Докажите, что рассматриваемый рынок безарбитражный тогда и только тогда, 
когда

$$C=\tilde {p}C_u +\left( {1-\tilde {p}} \right)C_d ,$$ где $\tilde 
{p}=\frac{1-d}{u-d}$.
\end{problem}
\begin{remark}
 Опцион характеризуется датой исполнения (в нашем 
случае -- момент времени 2) и платежами в момент исполнения ($C_u $ и $C_d 
)$. Причем эти платежи -- заранее известные функции от цены акции в этот 
момент (введение опционов было мотивировано желанием ``хеджироваться'', 
страховаться от нежелательных изменений цен акций). Основная задача 
заключается в установлении ``справедливой'' цены опциона $C$ в момент 
времени 1 (см. \cite{21} глава 7, {\S} 
11; А.Н. Ширяев, Основы финансовой стохастической математики, М.: ФАЗИС, 
2004, т. 1, т. 2).

Не 
имея в начальный момент 1 капитала $X\left( 1 \right)=0$, но, проделав 
некоторую махинацию (продав одних ценных бумаг (в зависимости от специфики 
рынка, иногда разрешается ``вставать в короткую позицию'' -- продавать 
ценные бумаги, не имея их в наличии; приобретая при этом в долг) и купив на 
вырученные деньги других бумаг), можно в момент времени 2 гарантированно 
ничего не проиграть, и при этом с ненулевой вероятностью выиграть (не 
уточняя сколько -- поскольку, ``прокручивая'' по имеющемуся арбитражу 
(пропорционально увеличивая коэффициенты $k_S $, $k_C )$ сколь угодно 
большую сумму, можно получить с ненулевой вероятностью сколь угодно большой 
выигрыш).

Величина $\tilde {p}$ называется мартингальной вероятностью (смысл 
такого определения будет раскрыт в следующих задачах) и задает мартингальную 
меру. Если существует единственная мартингальная мера, то рынок называется 
полным. На полном рынке неизвестная цена опциона $C$ в начальный 
момент определяется однозначно и может интерпретироваться как ``справедливая 
цена''.
\end{remark}
\begin{problem}(биномиальная $n$-периодная 
модель Кокса--Росса--Рубинштейна)\label{n-period} Предложите обобщение рынка и 
соответствующих понятий из задачи \label{sec:levilong} на $n$-периодный рынок. С возможностью класть 
деньги в банк под процент $r-1$ ($d<r<u)$ -- за один период (под такой же 
процент брать деньги из банка). Опцион исполняется в заключительный $(n+1)$-ый 
момент. Платежи по опциону в этот момент известны и описываются известной 
функцией $\bar {C}\left( S \right)$ (например, для указанного в предыдущей задаче 
опциона $\bar {C}\left( S \right)=\max \;\left\{ {0,\;S-X} \right\})$, 
т.е. $C_k \left( {n+1} \right)=\bar {C}\left( {S_k \left( {n+1} \right)} 
\right)$, где $k$ -- состояние в котором находится рынок в момент времени 
$n+1$. Считайте, что $S_k \left( {n+1} \right)=Su^kd^{n-k}$, т.е. $k$  
характеризует то, сколько раз акция поднималась в цене. Также как и в предыдущей задаче, 
требуется определить ``справедливую'' цену опциона.
Обоснуйте формулу Кокса--Росса--Рубинштейна:
\begin{equation}
\label{koks}
C=\frac{1}{r^n}\sum\limits_{k=0}^n {\left( {\begin{array}{l}
 n \\ 
 k \\ 
 \end{array}} \right)} \tilde {p}^k\left( {1-\tilde {p}} \right)^{n-k}\bar 
{C}\left( {Su^kd^{n-k}} \right),
\end{equation}
где $\tilde {p}=\frac{r-d}{u-d}.$
\end{problem}

\begin{remark}
Отметим, что в условиях задачи $X$ называется ценой исполнения опциона и считается 
известной. Собственно, вид функции $\bar {C}\left( S \right)=\max \left\{ 
{0,\;S-X} \right\}$ проясняет смысл опциона. Опцион дает право купить (у 
того, кто продал нам опцион) в момент исполнения опциона акцию по цене $X$. 
Если акция стоит дороже в этот момент, то, конечно, мы этим правом 
воспользуемся и получим прибыль (продавец опциона обязан продать нам акцию). 
Если же цена акции меньше цены исполнения опциона, то нам уже не выгодно 
покупать акцию по более дорогой цене, чем рыночная, и мы не исполняем 
опцион, т.е. ничего не делаем (ведь опцион дает нам право, ни к чему не 
обязывая). См. также Белопольская Я., Теория арбитража в непрерывном времени, C. Петербург: СПбГАСУ, 2006, Булинский А.В., Случайные процессы. Примеры, задачи и упражнения. М: МФТИ, 2010.
\end{remark}
\begin{problem}(континуальная биномиальная модель Блэка--Шоулса) 
\label{bw-cont}
Уместим на отрезке времени $\left[ {0,\;t} \right] \quad n+1$  моментов 
(промежутки между которыми одинаковы), в которые осуществляется торговля 
согласно задаче \ref{n-period}. Введем два параметра: $a$ -- снос, $\sigma ^2\ge 0$ -- 
волатильность (дисперсия). Положим,
\[
\mu =a+\frac{\sigma ^2}{2},
\quad
r=\exp \left( {\mu \frac{t}{n}} \right),
\quad
u=\exp \left( {\sigma \sqrt {\frac{t}{n}} } \right),
\]
\begin{equation}
\label{bw}
d=\exp \left( {-\sigma \sqrt {\frac{t}{n}} } \right),
\quad
\tilde {p}=\frac{r-d}{u-d}\approx \frac{1}{2}\left( {1+\frac{a}{\sigma 
}\sqrt {\frac{t}{n}} } \right).
\end{equation}

Переходя к пределу при $n\to \infty $ в формуле \eqref{koks} (с $\bar {C}\left( S 
\right)=\max\left\{ {0,\;S-X} \right\})$, согласно \eqref{bw}, получите формулу 
Блэка--Шоулса для справедливой цены опциона в ``континуальной биномиальной 
модели''. Почему вводится именно два параметра (а не один, три и т.д.)? 
Почему
\[
r-1\sim \frac{\mu t}{n},
\quad
u-1\approx \sigma \sqrt {\frac{t}{n}} ,
\quad
d-1\approx -\sigma \sqrt {\frac{t}{n}} ?
\]
Возможны ли какие-нибудь другие осмысленные варианты соотношений типа \eqref{bw}, 
при которых будет существовать предел при $n\to \infty $ в формуле \eqref{koks} (для 
простоты вычислений считайте, что $\bar {C}\left( S \right):=S)$?
\end{problem}
\begin{remark} (Донскер--Прохоров--Скороход--леКам--Варадарайн). \\
Имеет место слабая сходимость при $n\to \infty $ описанного случайного 
блуждания в дискретном времени к случайному процессу (в непрерывном 
времени), называемому геометрическим броуновским движением. Детали см., 
например, в книге Биллингсли П. Сходимость вероятностных мер. -- М.: Наука, 
1977. Также см. главу 12 в \cite{Gupta}. Также для решения задачи полезно использовать \cite{101}.
\end{remark}
\begin{problem} (броуновское движение (процесс Башелье) и винеровский процесс). \label{bachelie}
Исходя из формулы \eqref{koks}, имеем, что ``рынок'' при определении 
``справедливой'' цены опциона считает, что случайный процесс $S\left( m 
\right)$ (цена акции в момент времени $m)$ эволюционирует согласно 
биномиальной модели с неизменными параметрами $d$, $u$, $\tilde 
{p}$. Построим случайный процесс $S\left( t \right)$ (в непрерывном 
времени), исходя из процесса $S\left( m \right)$, заданного в дискретном 
времени предельным переходом, аналогичным задаче \ref{bw-cont}. Как уже отмечалось,
полученный процесс $S\left( t \right)$ называют геометрическим 
броуновским движением (или случайным процессом Башелье--Самуэльсона) с 
параметрами $a$, $\sigma ^2\ge 0$, а случайный 
процесс $B\left( t \right)=\ln \left(\frac{S\left( t \right)}{S\left( 0 \right)}\right)$ 
-- броуновским движением с параметрами $a$, $\sigma ^2\ge 0$. Если 
$a=0$, $\sigma ^2=1$, то такое броуновское движение имеет специальное 
название -- винеровский процесс $W\left( t \right)$. Покажите, что 
броуновское движение является процессом Леви (см. задачу \ref{sobord}). Найдите триплет $\left( 
{b,\;c,\;\nu \left( {dx} \right)} \right)$.
\end{problem}
\begin{remark}  Один 
из альтернативных способов введения мартингальных вероятностей $\tilde {p}$ 
основывается на, так называемых, ``риск нейтральных'' или ``мартингальных'' 
соображениях. Заключающихся в том, что $\tilde {p}$ выбирается исходя из 
равенства $$\mathbb{E}_{\tilde {p}} \left[ {Su^{\sum\limits_{k=1}^n {x_k } 
}d^{n-\sum\limits_{k=1}^n {x_k } }} \right]=Sr^n,$$ где независимые одинаково распределенные с.в.  $x_k$ имеют распределение $Be\left( {\tilde {p}} \right)$, или исходя из того, что процесс приведенной 
(продисконтированной) стоимости акции ${\tilde {S}\left( m \right)=S\left( m 
\right)} \mathord{\left/ {\vphantom {{\tilde {S}\left( m \right)=S\left( m 
\right)} {r^m}}} \right. \kern-\nulldelimiterspace} {r^m}$ должен быть 
мартингалом относительно мартингальной меры $\tilde {p}$ (отсюда и 
название), т.е.$$\mathbb{E}_{\tilde {p}} \left( {\left. {\tilde {S}\left( {m+1} 
\right)} \right|\left( {\tilde {S}\left( 1 \right),...,\tilde {S}\left( m 
\right)} \right)} \right)=\tilde {S}\left( m \right).$$

Параметры геометрического броуновского движения имеют следующий смысл: 
$$a=\frac{1}{t}\mathbb{E}\left[ {\ln \frac{S\left( t \right)}{S\left( 0 \right)}} 
\right],$$ 
$$\sigma ^2=\frac{1}{t}\Var\left[ {\ln \frac{S\left( t 
\right)}{S\left( 0 \right)}} \right].$$
%Также для решения задачи полезно использовать \cite{101}.
\end{remark}
\begin{problem} \label{geomB}(геометрическое броуновское движение или процесс 
Башелье--Самуэльсона) В условиях задачи \ref{bachelie} покажите, что геометрическое 
броуновское движение удовлетворяет следующему стохастическому 
дифференциальному уравнению:
\[
dS\left( t \right)=\left( {a+\frac{\sigma ^2}{2}} \right)S\left( t 
\right)dt+\sigma S\left( t \right)dW\left( t \right),
\]
которое определяет $S\left( t \right)$, как случайный процесс, 
удовлетворяющий соотношению:
\[
S\left( t \right)=S\left( 0 \right)+\left( {a+\frac{\sigma ^2}{2}} 
\right)\int\limits_0^t {S\left( \tau \right)dt} +\sigma \int\limits_0^t 
{S\left( \tau \right)dW\left( \tau \right)},
\]
где второй интеграл понимается в смысле Ито (см.  \cite{101}).
\end{problem}
\begin{remark} Случайные процессы, которые задаются 
стохастическими дифференциальными уравнениями наподобие рассмотренного, 
задают по определению диффузионный процесс Ито (см. Оксендаль Б., 
Стохастические дифференциальные уравнения, М.: Мир, 2003, главы 7, 8).
\end{remark}
\begin{problem} (формула Ито) Через конечные разности покажите, что дифференциал процесса $S(t)$ (см.предыдущую задачу) выглядит как: 
\[
\Delta S\left( t \right)=\Delta \left( {S\left( 0 \right)\exp \left( 
{B\left( t \right)} \right)} \right)=
\]
\[
=\left( {a+\frac{\sigma ^2}{2}} \right)S\left( t \right)\Delta t+\sigma 
S\left( t \right)\Delta W\left( t \right),
\]
где
\[
\Delta S\left( t \right)=S\left( {t+h} \right)-S\left( t \right),
\quad
\Delta W\left( t \right)=W\left( {t+h} \right)-W\left( t \right),
\]
\[
\Delta t=t+h-t=h,
\quad
h>0.
\]
Предложите общий вид формулы для $\Delta g\left( {t,\;W\left( t \right)} 
\right)$ ($dg\left( {t,\;W\left( t \right)} \right))$.
\end{problem}

\begin{remark}
Воспользуйтесь приближением:
\[
\Delta \left( {S\left( 0 \right)\exp 
\left( {at+\sigma W\left( t \right)} \right)} \right)\simeq 
\]
\[
\simeq aS\left( 0 \right)\exp \left( {at+\sigma W\left( t \right)} 
\right)\Delta t+\sigma S\left( 0 \right)\exp \left( {at+\sigma W\left( t 
\right)} \right)\Delta W\left( t \right)+
\]
\[
+\frac{\sigma ^2}{2!}S\left( 0 \right)\exp \left( {at+\sigma W\left( t 
\right)} \right)\left( {\Delta W\left( t \right)} \right)^2\simeq 
\]
\[
\simeq \left( {a+\frac{\sigma ^2}{2}} \right)S\left( 0 \right)\exp \left( 
{at+\sigma W\left( t \right)} \right)\Delta t + 
\]
\[+\sigma S\left( 0 \right)\exp 
\left( {at+\sigma W\left( t \right)} \right)\Delta W\left( t \right)
\]

См. Б. Оксендаль, Стохастические 
дифференциальные уравнения, М.: Мир, 2003, глава 4, а также 
Я. Белопольская, Теория арбитража в непрерывном времени, C. Петербург: СПбГАСУ, 2006.
\end{remark}
%\begin{remark}См. Paul Glasserman
%Monte Carlo Methods in Financial Engineering. 
%Applications of mathematics: stochastic modelling and applied probability, Springer, 2004; Carl Graham, Denis Talay Stochastic Simulation and Monte Carlo Methods: Mathematical Foundations of Stochastic Simulation, Springer, 2013; M.B. Giles. Multi-level Monte Carlo path simulation, Operations Research, 56(3):607-617, 2008.
%\end{remark}


\begin{problem}(Пуассоновский процесс). Покажите, что 
пуассоновский процесс и сложный пуассоновский процесс (см. задачи \ref{sec:poisson} и \ref{sec:cpoisson}) 
являются процессами Леви (см. задачу \ref{sobord}). Найдите триплеты $\left( {b,\;c,\;\nu \left( {dx} 
\right)} \right)$.
\end{problem}

\begin{remark} Исходя из задач \ref{sec:infdiv}, \ref{sec:levilong}
можно выдвинуть гипотезу, что любой процесс Леви может быть получен как 
сумма броуновского движения и сложного пуассоновского процесса. Такого рода 
утверждение действительно имеет место и называется представлением 
Леви--Ито. Однако вместо сложного пуассоновского процесса в этом 
представлении в общем случае следует брать процесс, который может быть 
получен как ``предел'' сложных пуассоновских процессов (См. K.-I. 
Sato, Levy processes and infinitely divisible distributions. Cambridge, 
1999). Сделанное замечание отчасти поясняет важность трех ключевых 
распределений теории вероятностей и трех типов процессов Леви: вырожденного (когда дисперсия равняется нулю)  
нормального (вырожденное 
броуновское движение), нормального (броуновское движение), распределения 
Пуассона (пределы сложных пуассоновских процессов). Помимо того, что при 
наиболее естественных предположениях для приложений имеет место сходимость к 
одному из этих трех безгранично делимых законов (например, аналогом теоремы 
Пуассона будет теорема Григелиониса \cite{4}), они в некотором смысле являются 
базисом: любое распределение, которое может возникать в пределе при 
суммировании независимых одинаково распределенных с.в., ``может быть 
получено'' исходя из этих трех базовых распределений. Приведенная задача частично 
проясняет, в каком смысле любое распределение ``может быть так получено''.
\end{remark}

\begin{problem} (Сложный процесс восстановления). Если в 
определении сложного пуассоновского процесса заменить пуассоновский процесс  общим процессом восстановления $$\tilde {K}\left( t \right)=\max \left\{ 
{k:\;\;\sum\limits_{i=1}^k {T_i } <t} \right\},$$ где $\left\{ {T_i } 
\right\}$ независимые одинаково распределенные с.в., но не обязательно, что $T_i \in \mbox{Exp}\left( \lambda 
\right))$, то получится сложный процесс восстановления $\tilde 
{Q}\left( t \right)$ (также играющий важную роль в разнообразных 
приложениях). 

Считая известными $\mathbb{E}\tilde {K}\left( t \right)$, $\Var\tilde 
{K}\left( t \right)$ и $\mathbb{E}V_i $, $\Var V_i $, определите $\mathbb{E}\tilde {Q}\left( t 
\right)$, $\Var\tilde {Q}\left( t \right)$. 

Используя ц.п.т. (в форме А.А. 
Натана \cite{5}, а также см. ориганальные работы  Anscombe, F. Large sample theory of sequential estimation, Proc. Cambridge Philos. Soc., 48, 1952 и  Renyi, A. On the asymptotic distribution of the sum of a random number of
independent random variables, Acta Math. Hung., 8, pp. 193–199,1957) найдите (приближенно) распределение сечения процесса $\tilde 
{Q}\left( t \right)$ при $t\gg 1$.
\end{problem}



\begin{problem}
В первобытной общине для более сбалансированного поддержания брачного отношения один ко многим введено следующее правило: каждая пара (мужчина-женщина) может размножаться до появления ребенка мужского пола. Изначально в общине было отношение количества человек мужского пола к женскому близкое к единице. Каким будет математическое ожидание отношения численности полов через 100 лет? Оцените также отклонение от среднего значения. Считать, что семейная пара размножается в точности один раз в году и производит одного ребенка (с равной вероятностью мальчика или девочку), пара начинает размножение по достижении 20 лет, мужчина с равной вероятностью от одной до трех женщин берет в жены.      
\end{problem}

\begin{problem}(поток Эрланга $m$-го порядка).\label{GammaFunc} Будет ли процессом
Леви (см. задачу \ref{sobord}) $m$ раз ``просеянный'' пуассоновский процесс $E_m \left( t \right)$ с 
параметром $\lambda >0$

$$E_m \left( t \right)=\max \left\{ {k:\;\;\sum\limits_{i=1}^k {T_i } <t} 
\right\},$$ 
где i.i.d. с.в. $T_i \in % \underbrace {\mbox{Exp}\left( \lambda 
%\right)+...+\mbox{Exp}\left( \lambda \right)}_m\mathop %=\limits^d 
\Gamma 
\left( {\lambda ,m} \right)$, $\lambda >0$?
\end{problem}

\begin{remark}
По определению $\Gamma \left( {\lambda ,m} 
\right) \mathop =\limits^d \underbrace {\mbox{Exp}\left( \lambda \right)+...+\mbox{Exp}\left( 
\lambda \right)}_m$ -- гамма распределение (сумма независимых показательных 
с.в.).
\end{remark}

\begin{problem} В модели Блэка--Шоулса--Мертона эволюция цены акции 
описывается геометрическим броуновским движением $$S\left( t \right)=S\left( 
0 \right)\exp \left( {at+\sigma W\left( t \right)} \right),$$ где $W\left( t 
\right)$ -- винеровский процесс ($\sigma >0)$. С помощью эргодической теоремы 
для случайных процессов оцените неизвестный параметр $a$, если известна 
реализация процесса $S\left( t \right)$ на достаточно длинном временном 
отрезке $\left[ {0,\;T} \right]$. Предложите способ оценки неизвестного 
параметра $\sigma $. Имеет ли смысл пытаться строить по $S\left( t \right)$ 
процесс $Y\left( t \right)=f\left( {S\left( t \right)} \right)$, то есть подбирать 
функцию $f(\cdot)$ так, чтобы $Y\left( t \right)$ был 
эргодичен по математическому ожиданию и $\mathbb{E}f\left( {S\left( t \right)} 
\right)=\sigma ^2$?
\end{problem}
\begin{remark}
%Говорят, что случайный скалярный процесс второго порядка $X(t)$ с постоянным математическим ожиданием $m_X$ обладает свойством \textit{эргодичности в среднем квадратичном по математическому ожиданию},
%если при $T \rightarrow \infty$:
%$$
%	\langle X\rangle_T \xrightarrow{\text{c.к.}} m_X, 
%$$
%где $\langle X\rangle_T = \dfrac{1}{T}\int\limits_{0}^{T}X(t)dt$.

%\noindent Эргодическая теорема: из эргодичности процесса в среднем квадратичном следует его эргодичность и по вероятности (при  $T \rightarrow  \infty$):
%$$
%	\langle X\rangle_T \xrightarrow{p} %m_X
%$$
%Выполнение условия эргодичности случайного процесса позволяет оценить $m_X$ величиной $\langle X\rangle_T$ имея лишь одну реализацию на достаточно длинном промежутке времени $T$.

%\noindent Прямая проверка условия эргодичности по матожиданию может оказаться весьма сложной задачей для произвольного процесса. Зачастую можно проверить условие выше, при помощи следующей теоремы:
%Необходимым и достаточным условием эргодичности в среднем квадратичном по математическому ожиданию случайного процесса второго порядка $X(t)$ с постоянным матожиданием $m_X$ является сущесвование предела:
%$$
%	\lim\limits_{T \rightarrow +\infty}
%	\dfrac{1}{T^2}\int\limits_{0}^T\int\limits_{0}^T %\mathrm{Cov}_X(t_1,t_2)
%	dt_1 dt_2 = 0
%$$
%\noindent В случае стационарных процессов (в широком смысле слова) достаточным условием для выполнения условия теоремы, а следовательно и выполнения условия эргодичности является существование предела:
%$$
%	\lim\limits_{|t_1 - t_2| \rightarrow +\infty} %\mathrm{Cov}_X(t_1,t_2) = 0
%$$

Модель Блэка--Шоулса--Мертона широко использовалась на практике (см. задачи \ref{bachelie}, \ref{geomB}). В 1990 г. М. 
Шоулс и Р. Мертон за свою работу были награждены нобелевской премией по 
экономике. Сейчас популярным классом моделей является $S\left( t 
\right)=S\left( 0 \right)\exp \left( {L\left( {\tau \left( t \right)} 
\right)} \right)$, где $L(\cdot)$ -- процесс Леви (см. задачу 39), $\tau 
\left( t \right)$ -- случайный процесс, не зависящий от $L(\cdot)$, с возрастающими почти наверное траекториями.
\end{remark}

\begin{problem}[статистическая эргодическая теорема фон Неймана для динамических систем \cite{21}, т.2] Пусть $(X,\sigma(X))$ -- измеримое пространство, $$T:X\to X:\forall B\in \sigma(X)\to \mu(T^{-1}(B)) = \mu(B).$$ Тогда $$\forall f\in L_2(X)\to\frac{1}{n}\sum_{k=1}^n f(T^k(x))\overset{L_2}{\underset{n\to\infty}{\longrightarrow}} \mathbb{E}(f|\Theta),$$ где $\Theta$ порождено функциями (случайными величиными) $$\left\{f:f(x)\overset{L_2}{=}f(T(x))\right\}.$$
Если $$B\in\sigma(X): T^{-1}(B)\overset{L_2}{=}B\Rightarrow B\overset{L_2}{=}\{\emptyset\}\lor X,$$ то $$\mathbb{E}(f|\Theta) = \mathbb{E} f=\int_{X}f(x)d\mu(x).$$

Не ограничивая общности, будем считать, что $\mu(X)=1$. Построим стационарный в широком смысле случайный процесс $Y(k)=f(T^kx)$, где $x$ --- случайная величина, распределенная согласно мере $\mu$. Верно ли, что 
$$\mathbb{E}(f|\Theta) =\mathbb{E}f\Leftrightarrow$$
$$ \frac{1}{n^2}\sum_{i,j=1}^{n}R_{\gamma}(i,j) {\longrightarrow}0\quad \text{при}\quad n\to 0,$$
где 
$\frac{1}{n^2}\sum_{i,j=1}^{n}R_{\gamma}(i,j) = \frac{1}{n^2}\sum_{i,j=1}^{n} \mathbb{E}[({Y}(i)-\mathbb{E}{Y}(i))({Y}(j)-\mathbb{E}{Y}(i))]$?
\end{problem}

\begin{problem}
Пусть задана последовательность  $x_k = \{\alpha k\}$, где  $\alpha$ -- какое-либо иррациональное число, $\{a\}$ --- дробная часть числа $a$. 

а) С помощью предыдущей задачи объясните, почему для любой $f\in C[0,1]$ выполняется при ${n\to\infty}$:
$$
\frac{1}{n}\sum_{k=1}^n f(x_k){\longrightarrow}\int_{0}^1 f(x)\,dx.
$$

б) Пусть функция, определенная в единичном квадрате $[0,1]^2$ на плоскости имеет вид:
$$
f(x,y) = \begin{cases}
 1, & y\leq x \text{ и } x+y\leq 1\\
 0, & \textit{иначе} 
\end{cases}.
$$
Верно ли, что при ${n\to\infty}$
$$
\frac{1}{n}\sum_{k=1}^n f(x_{2k-1},x_{2k}){\longrightarrow}\int_0^1\int_0^1 f(x,y)\,dxdy ?
$$

\end{problem}


\begin{problem} (Г. Вейль). Рассмотрим последовательность 
$\left\{ {a_k } \right\}_{k\in {\mathbb{N}}} $, где $a_k $ -- первая цифра в 
десятичной записи числа $2^k$. Положим $$I_m \left( {a_k } \right)=\left\{ 
{\begin{array}{l}
 1,\quad a_k =m \\ 
 0,\quad a_k \ne m \\ 
 \end{array}} \right.,\quad m=1,\;2,\;...,9.$$ Существует ли $\mathop {\lim 
}\limits_{n\to \infty } \;\frac{1}{n}\sum\limits_{k=1}^n {I_m \left( {a_k } 
\right)}$? Если существует, то найдите его.
\end{problem}

\begin{ordre} Рассмотрим вероятностное пространство $X=\left( \Omega ,\Xi ,P\right)$, где $\Omega =\left[ {0,\;1} \right)$, $\Xi $ 
-- $\sigma $-алгебра борелевских множеств (т.е. $\Xi $ -- минимальная 
$\sigma $-алгебра, содержащая всевозможные открытые множества $\Omega 
=\left[ {0,\;1} \right)$) на $\left[ {0,\;1} \right)$, а $P$ --- равномерная 
мера на $\Xi $, т.е. $P\left( {\left[ {a,\;b} \right)} \right)=b-a$. 
Рассмотрим с.в.
\[
x\left( \omega \right)=\left\{ {\begin{array}{l}
 1,\quad \omega \in \left[ {\log _{10} m,\;\log _{10} \left( {m+1} \right)} 
\right) \\ 
 0,\quad \omega \in \left[ {0,\;\log _{10} m} \right)\cup \left[ {\log _{10} 
\left( {m+1} \right),\;1} \right) \\ 
 \end{array}} \right..
\]
Рассмотрим случайный процесс (в дискретном времени)
$$X_k \left( \omega \right)=x\left( {T^k\omega } \right),$$ где $T:\left[ 
{0,\;1} \right)\to \left[ {0,\;1} \right)$ определяется по формуле 
 $$T\omega =\left( {\omega +\log _{10} 2} 
\right) \bmod 1,$$ 
где $a mod 1$ --- дробная часть числа $a$.
Важно заметить, что преобразование $T$ сохраняет меру, т.е. $$\forall 
\;\;A\in \Xi \to P\left( {T^{-1}A} \right)=P\left( A \right).$$ Собственно, и 
в более общей ситуации, известная из курса случайных процессов эргодическая 
теорема схожим образом переносится на динамические системы, которые задаются 
фазовым пространством $\Omega $ и динамикой $T:\;\Omega \to \Omega $. 
Согласно теореме Крылова--Боголюбова (см. Синай Я.Г., Введение в 
эргодическую теорию, М.: ФАЗИС, 1996, лекция 2), если $\Omega $ --  компакт, 
то всегда найдется как минимум одна инвариантная относительно $T$ мера на 
$\Xi $. Если построенной по такой динамической системе случайный процесс 
окажется эргодическим, то доля времени пребывания динамической системы в 
заданной области просто равняется мере (той самой инвариантной и 
эргодической) этой области. Ввиду вышесказанного интересно заметить, что 
установление эргодичности является трудной задачей. Например, до сих пор строго не 
обоснована ``эргодическая гипотеза Лоренца'' для идеального газа в сосуде 
(см. Козлов В.В., Тепловое равновесие по Гиббсу и Пуанкаре, Москва -- 
Ижевск, РХД, 2002, Минлос Р., Введение в математическую статистическую 
физику, М.: МЦНМО, 2002), см. также задачи 1, 22 раздела 6, задачу 15 раздела 7.

Покажите, что случайный процесс $X_k $ -- стационарный в узком смысле. В 
предположении, что этот процесс эргодичен по математическому 
ожиданию (см. \cite{21} т. 2, глава 5) 
 найдите искомый предел.
 
Стоит обратить внимание, что в 
эргодической теореме фигурирует сходимость либо в $L_2$, либо в $L_1 $, 
либо п.н. А в данной задаче требуется (для доказательства существования 
предела и его вычисления), сходимость поточечная. Оказывается, для данной 
задачи из сходимости в $L_2 $ легко следует сходимость п.н., откуда (в свою 
очередь) следует поточечная (подробности см. Синай  Я.Г., Введение в 
эргодическую теорию, М.: ФАЗИС, 1996, лекция 3 и  Корнфельд И.П.,  Синай Я.Г., 
 Фомин С.В., Эргодическая теория, М.: Наука, 1980).
\end{ordre}
\begin{problem} (Гаусса--Гильдена--Вимана--Кузьмина) Каждое число из промежутка $\Omega =\left[ {0,\;1} 
\right)$ может быть разложено в цепную дробь (вообще говоря, бесконечную). 

Покажите (см. предыдущую задачу), что для почти всех (в 
равномерной мере) точек $\omega \in \left[ {0,\;1} \right)$
\[
\mathop {\lim }\limits_{n\to \infty } \;\frac{1}{n}\sum\limits_{k=1}^n {I_m 
\left( {a_k \left( \omega \right)} \right)} =\frac{1}{\ln \;2}\ln \left( 
{1+\frac{1}{m\left( {m+2} \right)}} \right).
\]
\end{problem}
\begin{ordre}
Цепные дроби играют важную роль, например, в различных вычислениях 
(поскольку позволяют строить в определенном смысле наилучшие приближения 
иррациональных чисел рациональными), в теории динамических систем (КАМ 
теории). Для рациональных чисел такие дроби конечны, для квадратичных 
иррациональностей -- периодические (см. пример ниже, в котором период равен 
1):
\[
\frac{\sqrt 5 -1}{2}=\frac{1}{a_1 +\frac{1}{a_2 +\frac{1}{a_3 
+...}}}=\frac{1}{1+\frac{1}{1+\frac{1}{1+...}}}.
\]
Чтобы проверить выписанное соотношение достаточно заметить, 
что $\frac{\sqrt 5 -1}{2}$ -- является корнем уравнения $x=\frac{1}{1+x}$ 
(причем, из принципа сжимающих отображений следует, что последовательность 
$$x_0 =1,\quad x_{n+1} =\frac{1}{1+x_n }$$ сходится именно к этому корню),
см.  Арнольд В.И. Цепные дроби, М.: МЦНМО, 
2001, Хинчин А.Я. Цепные дроби, Ленинград: Физматгиз, 1961. Покажите, что преобразование $T:\;\;\left[ 
{0,\;1} \right)\to \left[ {0,\;1} \right)$:
\[
T\omega =\left\{ {\begin{array}{l}
 \left\{ {\frac{1}{\omega }} \right\},\quad \omega \in \left( {0,\;1} 
\right) \\ 
 0,\quad \omega =0 \\ 
 \end{array}} \right.,
\]
где $\left\{ {5.8} \right\}=0.8$ -- дробная часть числа, сохраняет меру Гаусса
\[
\forall \;\;A\in \Xi \to P\left( A \right)=\frac{1}{\ln \;2}\int\limits_A 
{\frac{dx}{1+x}} ,
\]
где $\Xi$ -- сигма-алгебра на $\Omega$.

Далее рассуждайте аналогично предыдущей задаче (эргодичность возникшего 
случайного процесса также можно не доказывать).
\end{ordre}


%\begin{problem}(Ветвящийся процесс). В колонию зайцев внесли зайца с 
%необычным геном. Обозначим через $p_k $ - вероятность того, что в потомстве 
%этого зайца ровно $k$ зайчат унаследуют этот ген ($k=0,1,2,...)$. Это же 
%распределение вероятностей характеризует всех последующих потомков, 
%унаследовавших необычный ген. Будем считать, что каждый заяц дает потомство 
%один раз в жизни в возрасте одного года (как раз в этом возрасте находился 
%самый первый заяц с необычным геном в момент попадания в колонию).

%Обозначим через $G\left( z \right)$ - производящую функцию распределения 
%$p_k $, $k=0,1,2,...$, т.е. $G\left( z %\right)=\sum\limits_{k=0}^\infty {p_k 
%z^k} $. Пусть $X_n $ - количество зайцев в возрасте одного года с необычным 
%геном спустя n лет после попадания в колонию первого такого зайца. 
%Производящую функцию с.в. $X_n $ обозначим $\Pi _n \left( z \right)=\mathbb{E}\left( 
%{z^{X_n }} \right)$.

%\begin{enumerate}
%\item Получите уравнение, связывающее $\Pi _{n+1} \left( z \right)$ с $\Pi _n \left( z \right)$ посредством $G\left( z \right)$.
%\end{enumerate}
%\textbf{Указание. }Покажите, что $M\left( {\left. {z^{X_{n+1} }} \right|X_n 
%} \right)=\left[ {G\left( z \right)} \right]^{X_n }$. Затем возьмите 
%математическое ожидание от обеих частей равенства.

%\begin{enumerate}
%\item Покажите, что вероятность вырождения гена $$q_n =P\left( {X_k =0;\;k\ge n} \right)=\Pi _n \left( 0 \right).$$ Существует ли предел $q=\mathop {\lim }\limits_{n\to \infty } \;q_n $? Если существует, то найдите его.
%\end{enumerate}
%\end{problem}
%\begin{remark}Легко видеть, что функция $G\left( z \right)$ - выпуклая. 
%Уравнение $z=G\left( z \right)$ имеет два корня: один в любом случае равен 
%1, другой $q\le 1$. Если $\nu ={G}'\left( 1 \right)>1$, то $q<1$. Если $\nu 
%\le 1$, то $q=1$.

%См. Севастьянов Б.А. Ветвящиеся процессы (серия "Теория вероятности и математическая статистика"), Наука, 1971; А. В. Калинкин Марковские ветвящиеся процессы с взаимодействием, Успехи математических наук, 2002.
%\end{remark}




%%\begin{problem}(парадокс Эренфестов). На двух камнях сидят кузнечики. Всего 
%%кузнечиков $M\gg 1$. Каждый кузнечик независимо ни от чего в промежутке 
%%времени $\left[ {t,t+\Delta t} \right)$, где $\Delta t$ -- мало, а $t\ge 0$ 
%%-- произвольно, перепрыгивает на другой камень с вероятностью $\lambda 
%%\Delta t+o\left( {\Delta t} \right)$, $\lambda >0$. Введем вектор $\vec 
%%{n}\left( t \right)=\left( {n_1 \left( t \right),n_2 \left( t \right)} 
%%\right)^T$, где $n_k \left( t \right)$ -- число кузнечиков на $k$-м камне в 
%%момент времени $t\ge 0$. Описанная стохастическая динамика имеет 
%%единственный закон сохранения $n_1 \left( t \right)+n_2 \left( t 
%%\right)\equiv M$ (числа кузнечиков). Стационарная (инвариантная) мера имеет 
%вид:
%%\[
%%\nu \left( {n_1 ,n_2 } \right)=\nu \left( {c_1 M,c_2 M} 
%%\right)=M!\frac{\left( {1 \mathord{\left/ {\vphantom {1 2}} \right. 
%\kern-\nulldelimiterspace} 2} \right)^{n_1 }}{n_1 !}\frac{\left( {1 
%\mathord{\left/ {\vphantom {1 2}} \right. \kern-\nulldelimiterspace} 2} 
%\right)^{n_2 }}{n_2 !}=C_M^{n_1 } 2^{-M}\simeq 
%\]
%\begin{equation}
%\label{eq2}
%\simeq \frac{2^{-M}}{\sqrt {2\pi c_1 c_2 } }\exp %\left( {-M\cdot H\left( 
%{c_1 ,c_2 } \right)} \right),
%\end{equation}
%где $H\left( {c_1 ,c_2 } %\right)=\sum\limits_{i=1}^2 {c_i \ln } \;c_i $. 
%Предположим теперь, что существует два предела
%\[
%c_i \left( 0 \right)=\mathop {\lim }\limits_{M\to %\infty } {n_i \left( 0 
%\right)} \mathord{\left/ {\vphantom {{n_i \left( 0 %\right)} M}} \right. 
%\kern-\nulldelimiterspace} M.
%\]
%Тогда в произвольный момент времени $t>0$ и для %любого $i=1,2$ с 
%вероятностью 1 существует предел (заметим, что %$n_i \left( t \right)$ -- 
%случайные величины, тем не менее $c_i \left( t %\right)$ -- уже не случайные 
%величины) $c_i \left( t \right)\mathop %=\limits^{\mbox{п.н.}} \mathop {\lim 
%}\limits_{M\to \infty } {n_i \left( t \right)} %\mathord{\left/ {\vphantom 
%{{n_i \left( t \right)} M}} \right. %\kern-\nulldelimiterspace} M$. Описанный 
%выше приём называется каноническим скейлингом. В %результате такого скейлинга 
%приходим к ``динамике квазисредних'' (как правило, %это динамика описывается 
%нелинейной системой обыкновенных дифференциальных %уравнений):
%\[
%\frac{dc_1 }{dt}=\lambda \left( {c_2 -c_1 } %\right),
%\]
%\[
%\frac{dc_2 }{dt}=\lambda \left( {c_1 -c_2 } %\right).
%\]
%Докажите формулу (\ref{eq1}). Покажите, что %функция (минус энтропия) $H\left( {c_1 
%,c_2 } \right)$ будет функцией Ляпунова выписанной %СОДУ. Покажите, что если 
%в начальный момент все кузнечики находились на %одном камне, то 
%математическое ожидание времени первого %возвращения макросистемы в такое 
%состояние будет порядка $2^M$. 
%\end{problem}
%\begin{remark}На примере этой модели можно %говорить о том, что в 
%макросистеме возврат к неравновесным %макросостояниям вполне допустим, но 
%происходить это может только через очень большое %время (циклы Пуанкаре), так 
%что нам может не хватить отведенного времени, %чтобы это заметить (парадокс 
%Цермело). Напомним, что описанный выше случайный %процесс обратим во времени. 
%Однако наблюдается необратимая динамика %относительной разности числа 
%кузнечиков на камнях (парадокс Лошмидта). Данная %задача является простейшим 
%представителем большого пласта задач на поиск %равновесия макросистемы, 
%которое определяется (в нашем случае, аналогично и %в общем случае), как
%\[
%\vec {c}^\ast =\left( {\begin{array}{l}
% 1 \mathord{\left/ {\vphantom {1 2}} \right. %\kern-\nulldelimiterspace} 2 \\ 
% 1 \mathord{\left/ {\vphantom {1 2}} \right. %\kern-\nulldelimiterspace} 2 \\ 
% \end{array}} \right)=\arg \mathop {\min %}\limits_{\begin{array}{c}
% c_1 +c_2 =M \\ 
% \vec {c}\ge \vec {0} \\ 
% \end{array}} H\left( {\vec {c}} \right).
%\]
%Детали и ссылки см. в работах%
%
%\textit{Sandholm W.} Population games and %Evolutionary dynamics. Economic Learning and %Social 
%Evolution. MIT Press; Cambridge, 2010.
%
%\textit{Гасников А.В}., \textit{Гасникова Е.В.} Об %энтропийно-подобных функционалах, возникающих в %стохастической 
%химической кинетике при концентрации инвариантной %меры и в качестве функций 
%Ляпунова динамики квазисредних // Математические %заметки. 2013. Т. 94. № 6. 
%С. 816--824.
%
%Немного более сложный пример приведен ниже.
%\end{remark}
%
%\begin{problem}{(кинетика социального %неравенства).} В некотором городе 
%живет $N\gg 1$ (например, 10~000) пронумерованных %жителей. У каждого $i$-го 
%жителя есть в начальный (нулевой) момент времени %целое (неотрицательное) 
%количество рублей $s_i \left( 0 \right)$ %(монетками, достоинством в один 
%рубль). Со временем пронумерованные жители %(количество которых не 
%изменяется, также как и суммарное количество %рублей) случайно разыгрывают 
%свое имущество. В каждый момент времени %$t=1,2,3,...$ случайно и независимо 
%от предыстории выбираются два жителя: с %вероятностью $1 \mathord{\left/ 
%{\vphantom {1 2}} \right. %\kern-\nulldelimiterspace} 2$ житель с большим 
%номером отдаёт 1 рубль (если, конечно, он не %банкрот, если банкрот, то не 
%отдает) жителю с меньшим номером, и с вероятностью %$1 \mathord{\left/ 
%{\vphantom {1 2}} \right. %\kern-\nulldelimiterspace} 2$ наоборот. 
%Приблизительно такую постановку задачи в конце %18-го века предложил 
%известный итальянский экономист Вильфредо Парето, %чтобы объяснить социальное 
%неравенство.%
%
%Пусть $c_s \left( t \right)$ - доля жителей %города, имеющих ровно $s$ рублей 
%в момент времени $t$ (заметим, что $c_s \left( t %\right)$ - случайная 
%величина). Пусть
%\[
%S=\sum\limits_{i=1}^N {s_i \left( 0 \right)} ,
%\quad
%\bar {s}=\frac{S}{N}.
%\]
%Покажите, что тогда по эргодической теореме для %конечных однородных 
%марковских цепей и центральной предельной теореме:
%\[
%\exists \;\;\lambda _{q,0.99} >0,\;\;T={\rm %O}\left( {N^2} 
%\right):\;\;\;\forall \;t\ge T,\;s=0,...,S
%%\]
%\[
%P\left( {\left. {\left| {c_s \left( t %\right)-Ce^{-s \mathord{\left/ 
%{\vphantom {s {\bar {s}}}} \right. %\kern-\nulldelimiterspace} {\bar {s}}}} 
%\right|\le \frac{\lambda _{0.99} }{\sqrt N }} %\right)\ge 0.99} \right.,
%\]
%где $C$ определяется из условия нормировки:
%
%$\sum\limits_{s=0}^S {Ce^{-s \mathord{\left/ %{\vphantom {s {\bar {s}}}} 
%\right. \kern-\nulldelimiterspace} {\bar {s}}}} %=1,$ т.е. $C\approx 1 
%\mathord{\left/ {\vphantom {1 {\bar {s}}}} \right. %
%\kern-\nulldelimiterspace} {\bar {s}}$.
%\end{problem}
%
%\begin{problem}{(Замкнутая сеть, теорема %Гордона--Ньюэлла).} Рассматривается 
%транспортная сеть, в которой между $N$ станциями %курсируют $M$ такси. 
%Клиенты пребывают в $i$-й узел в соответствии с %пуассоновским потоком с 
%параметром $\lambda _i >0$ ($i=1,\;...,\;N)$. Если %в момент прибытия в $i$-й 
%узел там есть такси, клиент забирает его и с %вероятностью $p_{ij} \ge 0$ 
%направляется в $j$-й узел, по прибытии в который %покидает сеть. Такси остается 
%ждать в узле прибытия нового клиента. Времена %перемещений из узла в узел -- 
%независимые случайные величины, имеющие %показательное распределение с 
%параметром $\nu _{ij} >0$ для пары узлов $\left( %{i,j} \right)$. Если в 
%момент прихода клиента в узел там нет такси, %клиент сразу покидает узел. 
%Считая $p_{ij} =N^{-1}$, $\lambda _i =\lambda $, %$\nu _{ij} =\nu $, 
%покажите, что вероятность того, что клиент, %поступившей в узел (в 
%установившемся (стационарном) режиме работы сети), %получит отказ, равна
%\[
%p_{\mbox{o}} \left( {N,M} %\right)={\sum\limits_{k=0}^M 
%{\frac{C_{N-2+k}^k \rho ^{M-k}}{\left( {M-k} %\right)!}} } \mathord{\left/ 
%{\vphantom {{\sum\limits_{k=0}^M %{\frac{C_{N-2+k}^k \rho ^{M-k}}{\left( 
%{M-k} \right)!}} } {\sum\limits_{k=0}^M %{\frac{C_{N-1+k}^k \rho 
%^{M-k}}{\left( {M-k} \right)!}} }}} \right. %\kern-\nulldelimiterspace} 
%{\sum\limits_{k=0}^M {\frac{C_{N-1+k}^k \rho %^{M-k}}{\left( {M-k} \right)!}} 
%},
%\quad
%\rho ={N\lambda } \mathord{\left/ {\vphantom %{{N\lambda } \nu }} \right. 
%\kern-\nulldelimiterspace} \nu .
%\]
%Методом перевала покажите справедливость следующей %асимптотики при $N\to 
%\infty $:
%\[
%p_{\mbox{o}} \left( {N,rN} %\right)=1-\frac{2r}{\lambda \mathord{\left/ 
%{\vphantom {\lambda \nu }} \right. %\kern-\nulldelimiterspace} \nu +r+1+\sqrt 
%{\left( {\lambda \mathord{\left/ {\vphantom %{\lambda \nu }} \right. 
%\kern-\nulldelimiterspace} \nu +r+1} %\right)^2-4{\lambda r} \mathord{\left/ 
%{\vphantom {{\lambda r} \nu }} \right. %\kern-\nulldelimiterspace} \nu } 
%}+{\rm O}\left( {\frac{1}{N}} \right).
%\]
%\end{problem}
%
\begin{problem}(Максимальный показатель Ляпунова \cite{27}). Пусть имеется последовательность независимых одинаково распределенных случайных матриц $g_k$ (распределение $g_k$ имеет плотность). Покажите, что существует такое $\lambda\in\mathbb{R}$, что (от выбора нормы число $\lambda$ не зависит) 
\begin{equation*}
\lim_{n\to\infty} \frac{1}{n}\|g_{n}\cdot\ldots\cdot g_{1}\| = \lambda.
\end{equation*}
\end{problem}
\begin{remark}
См. Гренандер У. Вероятности на алгебраических структурах, М.: Мир, 1965.
\end{remark}

\begin{problem}(Теорема Санова)
\label{sanov}
Пусть $A$ --- конечный алфавит. Вероятности появления в слове символа алфавита обозначим $p_a$, $a\in A$. Предполагается, что число символов в алфавите $|A|>1$. 
Обозначим за $v_a(s)$ случайную величину, отвечающую за относительную частоту буквы $a$ в строке $s$, т. е. число вхождений буквы $a$ в строку $s$, деленное на $n$. Назовем типом строки $s$ набор ${v}(s) = (v_1(s), v_2(s), . . . , v_{|A|}(s))$.

Пусть $\Pi$ --- непустое замкнутое подмножество множества распределений вероятности $\big\{{p}=(\dots,p_a,\dots):\, p_{a}\ge 0 \text{ для всех } a\in A,\, \sum_{a\in A}p_{a}=1\bigr\}$,  совпадающее с замыканием своей внутренности, и $$\mathcal{KL}(\Pi|{p}) = \min_{\mu\in\Pi}\mathcal{KL}(\mu|{p}),$$
где расстояние Кульбака-Лейблера между вектором $\mu$ и истинным распределением ${p}$ имеет вид
$$
\mathcal{KL}(\mu|{p}) = \sum_{a\in A} \mu_a\log\frac{\mu_a}{p_a}.
$$
Покажите, что при $n\to\infty$
$$
-\frac{1}{n}\log\mathbb{P}({v}(s)\in \Pi)\to\mathcal{KL}(\Pi|{p}).
$$

\end{problem}
\begin{remark}
Доказательство базируется на использовании формулы Стирлинга $n! = n^n{\rm e}^{-n}\sqrt{2\pi n}[1+O(1/n)]$, откуда вероятность реализации слов $s$ длины $n$ с частотой появления букв $v_a(s)=\mu_a$, $a \in A$ равна
\begin{equation*}
\begin{split}
\mathbb{P}(v(s)=\mu) = \frac{n!}{(n\mu_a)!\dots (n\mu_z)!}p_a^{n\mu_a}\cdot\dots\cdot p_z^{n\mu_a}\\
\approx C \exp(-n\mathcal{KL}(\mu|\vec{p}))\\
= \exp(-n\mathcal{KL}(\mu|\vec{p}) + R),
\end{split}
\end{equation*}
где $|R|<(n+1)(\frac 12 \log{n}+\frac 12 \log(\pi)+\frac{1}{12n}).$

См.  Санов И.Н. О вероятности больших отклонений случайных величин, Матем. сб., 42(84):1 (1957), С. 11–44 лекции А.Н. Соболевского в НМУ   (http://www.mccme.ru/ium/s09/probability.html), \cite{information}.

Также о вероятностях больших отклонений для эмпирических мер стохастических процессов смотрите следующую литературу: Feng, J., Kurtz T.G. Large deviations for stochastic processes, V. 131 of Mathematical surveys and monographs. American Mathematical Society, Providence, RI, USA, 2006, Leonard C. A large deviation approach to optimal transport. arXiv:0710.1461v1, 2007.


\end{remark}

\begin{problem}(Крамеровская зона)
Рассмотрим $S_n = X_1+\cdots+X_n$, где $X_i$, $i=1,\dots,n$ --- независимые одинаково распределенные величины, $\mathbb{E}X_i=0$, $\mathbb{E}X_i^2= d<\infty$.
В силу центральной предельной теоремы при $n\to\infty$ 
\[
\mathbb{P}(S-n\geq x)\approx 1- \Phi\left(\frac{x}{\sqrt{nd}}\right)
\]
равномерно по $x$ из интервала $(0,N_{n}n)$, где $N_n\to \infty$ достаточно медленно. Предположим, что $X_i$ имеют субэкспоненциальное распределение, характеристическим свойством которого является асимптотическая аддитивность хвостов сверток исходных распределений, то есть 
\[
\mathbb{P}(S_n\geq x)\approx n\mathbb{P}(X_1\geq x),
\]
в частности, это свойство выполнено, если $\mathbb{P}(X_1\geq x)$ является правильно меняющейся функцией, то есть $\mathbb{P}(X_1\geq t) = t^{-\alpha}L(t)$, где в свою очередь $L(t)$ --- медленно меняющаяся функция, то есть для любого $u>0$ выполнено $\frac{L(ut)}{L(t)}\to 1$ при $t\to\infty$.
Эти две асимптотики смыкаются следующим образом, если выполнено $\mathbb{E}(X^2;|X|>t)={o}(1/\ln t)$, $x>\sqrt{n}$, $t\to\infty$:
\[
\mathbb{P}(S_n\geq x)\approx 1- \Phi\left(\frac{x}{\sqrt{nd}}\right)+nV(x),\quad n\to\infty.
\]
Значение, характеризующее зону уклонений $S_n$, где происходит смена асимптотик $\mathbb{P}(S_n\geq x)$ от <<нормальной>> $1-\Phi(x/\sqrt{nd})$ на асимптотику $nV(x)$, описывающую $\mathbb{P}(S_n\geq x)$ при достаточно больших $x$ следующее:
\[
\sigma(n) = V^{-1}(1/n) = \sqrt{(\alpha-2)nd\ln n},
\]
при $n=1$ полагаем $\sigma(1)=1$.

Найдите асимптотики уклонений $S_n$ для последовательностей случайных величин из \begin{enumerate}
\item равномерного распределения,
\item пуассоновского распределения,
\item экспоненциального распределения.
\end{enumerate}

\end{problem}
\begin{remark}
Используем обозначения $B_j = \{X_j<y\}$, $B = \bigcup_{i=1}^n B_j$.

Рассмотрим случай $d=1$, $x>\sqrt{n}$, $\mathbb{P}(S_n\geq x) \leq V(x)$, $\alpha>2$, $\mathbb{E}X_i = 0$, $\mathbb{E} X_i^2 = 1$. 

Приведем результат о верхних границах на $\mathbb{P}(\max_{k\leq n} S_n \geq x; B)$.

\noindent 1)  При любых фиксированных $h>1$, $s_0>0$ для $x = \sigma(n)$, $s\geq s_0$ и всех достаточно малых $\Pi = nV(x)$ выполняется 
\[
P = \mathbb{P}(\max_{k\leq n} S_n \geq x; B)\leq {\rm e}^{r} \left(\frac{\Pi(y)}{r}\right)^{r-\theta},
\]
где $\Pi(y)= nV(y)$, $\theta = \frac{hr^2}{4s^2}\left(1+b\frac{\ln s}{\ln n}\right)$, $b = \frac{2\alpha}{\alpha-2}$.

\noindent 2) При любых фиксированых $h>1$, $\tau>0$ для $x = s\sigma(n)$, $s^2<(h-\tau)/2$ и всех достаточно больших $n$ выполняется
\[
P\leq{\rm e}^{-x^2/2nh}.
\]

Покажите, что из приведенной теоремы следуют утверждения:

\noindent a) Если $x=s\sigma(n)$, $s\to\infty$, то при любом $\epsilon>0$ и всех достаточно малых $\Pi=nV(x)$
\[
P\leq \Pi^{r-\epsilon}.
\]

\noindent b) Ecли $s^2>c\ln n$, то 
\[
P\leq c_1\Pi^{r}.
\]

\noindent c) Если $s$ фиксировано, $r = 2s^2/h\geq 1$, то при $n \to\infty$
\[
P\leq c\Pi^{s^2/h+{o}(1)}.
\]

\noindent d) В частности, при всех $s^2>2h$ и всех достаточно больших $n$
\[
P\leq c\Pi^2.
\]

\noindent e) Если $s\to\infty$, то при любом $\delta>0$ и всех достаточно малых $nV(x)$
\[
\mathbb{P}(\sup_{k\leq n}S_k\geq x)\leq nV(x)(1+\delta).
\]

\noindent f)
Если $s^2\geq h+\tau$ при каком-нибудь фиксированном $\tau>0$, то при достаточно малых $nV(x)$
\[
\mathbb{P}(\sup_{k\leq n}S_k\geq x)\leq cnV(x).
\]

\noindent g) При любых фиксированных $h>1$, $\tau>0$ для $s^2<(h-\tau)/2$ и всех достаточно больших $n$
\[
\mathbb{P}(\sup_{k\leq n}S_k\geq x)\leq {\rm e}^{-x^2/2nh}.
\]

См. Боровков А.А., Боровков К.А., Асимптотический анализ случайных блужданий.  Т.1. Медленно убывающие распределения скачков, Москва: Физматлит, 2008, 652 с.
\end{remark}



\begin{problem}(Маловероятные пути блужданий \cite{27}) Будем говорить, что случайное  блуждание $\{S_t\}_{t\geq 0}$, $S_t = \sum_{i=1}^t X_i$, где
$X_i$ принимает значения ${-1,1}$ c вероятностями $q$ и $p$ соответственно, 
удовлетворяет принципу больших уклонений с функционалом действия $L_{\tau}(v)$, если
$$
\ln \mathbb{P}(A_{[\tau N],\delta})\sim L_{\tau}(v)N,
$$
где 
$$
A_{[\tau N],\delta} = \bigg\{\sup_{t=0,1,\dots,[\tau N]}|S_t-vt|\leq \delta N\biggr\}.
$$

Покажите, что для случайного блуждания с  $p=q=1/2$ (соотвествующая мера $\mathbb{P}_{0}$) верен принцип больших уклонений с функционалом действия  $L_{\tau}(v) = \tau (-v\lambda(v)+h(\lambda(v)))$,
где $\lambda=\lambda(v)$ решение уравнения 
$$
\frac{{\rm e}^{\lambda}-{\rm e}^{-\lambda}}{{\rm e}^{\lambda}+{\rm e}^{-\lambda}}=v.
$$
\end{problem}
\begin{remark}
Идея доказательства состоит в использовании замены меры, то есть вероятности скачков $1/2$ заменяют на
$$
p_{\lambda} = \frac{{\rm e}^{\lambda}}{{\rm e}^{\lambda}+{\rm e}^{-\lambda}},\,
q_{\lambda} = \frac{{\rm e}^{-\lambda}}{{\rm e}^{\lambda}+{\rm e}^{-\lambda}}
$$
так, чтобы $S_t-vt$ имело нулевые средние. Тогда вероятности по мере $\mathbb{P}_{0}$ представляются через средние по новой бернуллиевской мере $\mathbb{P}_{\lambda}$
$$
\mathbb{P}(A_{[\tau N],\delta}) = \mathbb{E}_{\lambda}[I(A_{[\tau N,\delta]})\exp(-\lambda S_{[N\tau]}+[N\tau]h(\lambda))],
$$
$$
h(\lambda) = \log\left(\frac{{\rm e}^{\lambda}+{\rm e}^{-\lambda}}{2}\right),
$$
где  $I(A_{[\tau N],\delta})$--- индикатор события $A_{[\tau N],\delta}$.

Доказательство принципа больших уклонений с указанным в условии задачи функционалом действия следует из оценок 
$$
\exp(N L_{\tau}(v))\mathbb{E}_{\lambda}[I(A_{[\tau N],\delta})]\exp(-|\lambda|\delta N)\leq \mathbb{P}(A_{[\tau N],\delta}), 
$$
$$
\mathbb{P}(A_{[\tau N],\delta}) \leq \mathbb{E}_{\lambda}[I(A_{[\tau N],\delta})]\exp(|\lambda|\delta N)\exp(N L_{\tau}(v)).
$$
См. также  Боровков А.А., Боровков К.А. Асимптотический анализ случайных блужданий.  Т.1. Медленно убывающие распределения скачков, Москва: Физматлит, 2008, 652 с.
\end{remark}

\begin{problem}(Случайное блуждание в полуплоскости). Пусть частица находится в начальный момент в одной из точек полуплоскости $\mathbb{Z}\times\mathbb{Z}_{+}$, и совершает на каждом шаге скачок из $(k,l)\in \mathbb{Z}\times\mathbb{Z}_{+}$ в одну из четырех соседних точек решетки $(k+i,l+j)$ с вероятностями $p_{ij}$ (если $l>0$) и в одну из трех соседних точек решетки с вероятностями $q_{ij}$ если $l=0$. Считая, что $\sum_{j}jp_{ij}<0$ опишите движение частицы \cite{27} (движение к границе и последующее движение вдоль границы).
\end{problem}



\section{Неравенства концентрации меры \\ и вероятности больших уклонений}
\label{measure}


\begin{problem}[Концентрация площади сферы и объема шара]
\
\begin{enumerate}
\item Рассмотрим шар $B^{n}(r)$ радиуса $r$ в евклидовом пространстве $\mathbb{R}^n$ большой размерности, пусть в шаре задана равномерная мера. 
%Пусть $V\big[B^{n}(r)\bigl]$ ~--- объем шара.
 Необходимо убедиться в том, что мера сконцентрирована в малой окрестности  границы шара.
\item
Рассмотрим сферу $S^{n-1}(r)$ в евклидовом пространстве $\mathbb{R}^r$ с радиусом в начале координат. Необходимо убедиться в том, что выбранные наугад два единичных вектора в пространстве $\mathbb{R}^n$ большой размерности с большой вероятностью окажутся почти ортогональными, если на сфере задано равномерное распределение.  
%Зафиксируем координатную ось $x$.
%Необходимо убедиться в том, что подавляющая часть площади многомерной сферы $S^{n-1}$ сосредоточена в малой окрестности экватора, перпендикулярного выбранной оси $x$. Каково взаимное расположение двух выбранных наугад единичных векторов в простанстве $\mathbb{R}^n$, если концы векторов распределены на сфере равномерно?
\end{enumerate}
\end{problem}


\begin{remark}
Во втором пункте достаточно доказать, что для всякого сколь угодно малого $\delta>0$ проекция второго вектора на ось $x_1$ с вероятностью, близкой к 
единице, лежит в промежутке $[-\delta, \delta]$ при $n\to\infty$. Это равносильно тому, что доля от площади всей сферы $S^{n-1}(r)$, 
которую занимает сферический слой $S^{n-1}_{\delta}(r)$, проектирующийся в отрезок $[-\delta, \delta]$ оси $x_1$, 
может быть сделана сколь угодно близкой к $1$ при $n\to\infty$. 

Перейдя к $n$-мерным сферическим координатам и обратно, покажите, что мера сферического слоя $S^{n-1}_{\delta}(r)$ равна 
$$
\mu_{n-1} S_{\delta}^{n-1}(r) = Cr^{n-1} 
\int\limits_{-\delta}^{\delta} \Bigl( 1-(x/r)^2\Bigr)^{(n-3)/2} \, dx ,
$$
тогда вероятность попадания в данный слой $S_{\delta}^{n-1}(r)$ равна 
$$
{\mathbb P}[-\delta, \delta]=\frac{\int\limits_{-\delta}^{\delta} \Bigl( 1-(x/r)^2\Bigr)^{(n-3)/2} \, dx}
{\int\limits_{-r}^{r} \Bigl( 1-(x/r)^2\Bigr)^{(n-3)/2} \, dx} . 
$$
Данное отношение на зависит от $r$, поэтому можно считать $r=1$. 

Для нахождения асимптотики имеющихся интегралов ($n\to\infty$) используйте классические результаты относительно асимптотики интеграла 
Лапласа $F(\lambda)=\int_a^b f(x)e^{\lambda S(x)}\, dx$ при $\lambda\to +\infty$ (см. указание к задаче \ref{laplas} из раздела  \ref{genF}). %Если обе функции $f$ и $S$ определены и регулярны 
%на промежутке $I=[a,b]$ и функция $S$ имеет единственный глобальный максимум на $I$, который достигается в точке $x_0\in I$, 
%$f(x_0)\ne 0$, то асимптотика интеграла такая же, как в окрестности точки $x_0$ (принцип локализации). В зависимости от расположения 
%точки $x_0$ и свойств функции $S(x)$ возможны следующие тейлоровские разложения при $\lambda\to +\infty$: 
%$$
%F(\lambda)=\frac{f(x_0)}{-S'(x_0)}e^{\lambda S(x_0)} \lambda^{-1}\bigl( 1+O(\lambda^{-1})\bigr) , 
%$$
%если $x_0=a$ и $S'(x_0)\ne 0$ (т.е. $S'(x_0)<0$); 
%$$
%F(\lambda)=\sqrt{\frac{\pi}{-2S''(x_0)}} f(x_0) e^{\lambda S(x_0)} \lambda^{-1/2}\bigl( 1+O(\lambda^{-1/2})\bigr) , 
%$$
%если $x_0=a$, $S'(x_0)=0$, $S''(x_0)\ne 0$ (т.е. $S''(x_0)<0$); 
%$$
%F(\lambda)=\sqrt{\frac{2\pi}{-S''(x_0)}} f(x_0) e^{\lambda S(x_0)} \lambda^{-1/2}\bigl( 1+O(\lambda^{-1/2})\bigr) , 
%$$
%если $a<x_0<b$, $S'(x_0)=0$, $S''(x_0)\ne 0$ (т.е. $S''(x_0)<0$). 

Для решения этой и следующей задачи рекомендуется ознакомиться с книгой Зорич В.А. Математический анализ задач естествознания,-- М.: МЦНМО, 2008.
\end{remark}



\begin{problem}[Изопериметрическое неравенство и принцип концентрации меры; П. Леви, 1919]
\label{levi}
Число $\mu _f $ называют медианой функции $f$, если
$\mu \left( {\vec {x}\in S_1^n :\;\;f\left( {\vec {x}} \right)\ge \mu _f } 
\right)\ge 1 \mathord{\left/ {\vphantom {1 2}} \right. 
\kern-\nulldelimiterspace} 2$ и $\mu \left( {\vec {x}\in S_1^n :\;\;f\left( 
{\vec {x}} \right)\le \mu _f } \right)\ge 1 \mathord{\left/ {\vphantom {1 
2}} \right. \kern-\nulldelimiterspace} 2$,
где $\mu \left( {d\vec {x}} \right)$ -- равномерное мера на единичной сфере 
$S_1^n $ в ${\mathbb R}^n$. Пусть $A$ -- измеримое (борелевское) множество на 
сфере $S_1^n $. Через $A_\delta $ -- будем обозначать $\delta $-окрестность (расстояние определяется по геодезический)
множества $A$ на сфере $S_1^n $. Предположим теперь, что в некотором 
царстве, расположенном на $S_1^n $, царь предложил царице Дидоне построить 
огород с заданной длиной забора. Царица хочет, чтобы её огород при заданном 
периметре имел наибольшую площадь. Таким образом, царице надо решить 
изопериметрическую задачу (такие задачи обычно рассматриваются в курсах 
вариационного исчисления). Решение этой задачи хорошо известно на плоскости -- ``круглый 
огород'', это можно обобщить на наш случай. Для нас же полезно, рассмотрение двойственной задачи, имеющей 
такое же решение: при заданной площади огорода спроектировать его так, чтобы 
он имел наименьшую длину забора, его ограждающего. Используя решение этой 
задачи, покажите, что если $\mu \left( A \right)\ge 1 \mathord{\left/ 
{\vphantom {1 2}} \right. \kern-\nulldelimiterspace} 2$, то
$$
\mu \left( {A_\delta } \right)\ge 1-\sqrt {\frac{\pi}{8}} \exp \left( \frac{-\delta^2n}{2} \right).
$$
Пусть теперь на $S_1^n $ задана функция с модулем непрерывности
\[
\omega _f \left( \delta \right)=\sup \left\{ {\left| {f\left( {\vec {x}} 
\right)-f\left( {\vec {y}} \right)} \right|:\;\;\rho \left( {\vec {x},\vec 
{y}} \right)\le \delta ,\;\vec {x},\vec {y}\in S_1^n } \right\}.
\]
Покажите, что тогда
\[
\mu \left( {\vec {x}\in S_1^n :\;\;\left| {f\left( {\vec {x}} \right)-\mu _f 
} \right|\ge \omega _f \left( \delta \right)} \right)\le \sqrt {\pi 
\mathord{\left/ {\vphantom {\pi 2}} \right. \kern-\nulldelimiterspace} 2} 
\exp \left( {-{\delta ^2n} \mathord{\left/ {\vphantom {{\delta ^2n} 2}} 
\right. \kern-\nulldelimiterspace} 2} \right).
\]
Можно показать, что при весьма естественных условиях медиана асимптотически 
близка к среднему значению (математическому ожиданию). Аналогичное 
неравенство можно получить (М. Талагран, 1994), например, для модели 
случайных графов (Эрдёша - Реньи) исследовать плотную концентрацию около 
среднего значения различные функций на случайных графах: число 
независимости, хроматическое число и т.п.
\end{problem}


\begin{remark}
Изопериметрические неравенства на сфере были обобщены в начале 80-х М.Л. Громовым на римановы многообразия (см. подробнее в \cite{14}).

Пусть $(X,g)$ ~--- компактное связное гладкое риманово многообразие размерности $n>2$ со строго положительной кривизной Риччи и римановой метрикой $g$, наделенное элементом объема $d\mu=\frac{dv}{V}$, где $V$~--- полный объем $X$. 
Кривизна Риччи~--- способ описания изменения многообразия по отношению к евлидовой мере, то есть степени отличия мнообразия  от евклидова пространства. 
Обозначим за $c(X)$ точную нижнюю грань тензора кривизны Риччи по всем единичным касательным векторам. Пусть $c(X)>0$. Тогда 
$$
\mathcal{P}_{\mu}\geq \mathcal{P}_{\sigma^n_{\mathbb{R}}},
$$
где $\sigma^n_{R}$ -- равномерная инвариантная мера, $\mathcal{P}_{\mu}$ -- изопериметрическая функция, т.е. наибольшая функция на $[0,\mu(X)]$, такая что
$$
\mu^{+}(A) \geq \mathcal{P}_{\mu}(\mu(A)) \quad \text{при~}\mu^{+}(A) = \lim_{t\to\infty}\inf\frac{1}{t}\mu(A_t/A),
$$ 
где $A_t = \{x\in X; g(x,A)<t\}$.

Здесь $R>0$ таково, что 
$$
c({S}^n_{R}) = \frac{n-1}{R^2}=c(X),
$$ 
где ${S}^n_{R}$~--- $n$-сфера с радиусом $R$, снабженная нормированной равномерной инвариантной мерой $\sigma^n_{R}$. 
В частности для $(X,g,\mu)$ верно 
\begin{equation*}
\alpha(X;r)\leq \text{e}^{-cr^2/2},\quad r>0, 
\end{equation*}
где концентрационная функция $\alpha(X;r)$ определяется следующим образом
\begin{equation*}
\alpha(X;r) = 1-\inf\biggl\{\mu(A_{r})|\, A \subset X, \mu(A)\geq \frac{1}{2}\biggr\}.
\end{equation*}
\end{remark}

\begin{problem}(Концентрация на дискретном кубе) Пусть $E = \{-1, 1\}^{n}$ --- дискретный $n$-мерный куб, на множестве вершин которого задана равномерная вероятностная мера $\mu$. Введем на вершинах куба стандартную Хэммингову метрику. Тогда для всякой $1$-липшицевой (по введенной метрике) действительнозначной функции $f$, заданной на $E$, c медианой $M$ для $\varepsilon \ge 0$ имеет место неравенство
\[
\PR\{\left|f(X) - M\right| \ge \varepsilon\} \le 2\exp\left(\frac{-\varepsilon^{2}}{2n}\right).
\]
а) Пусть $\Exp f(X)$ -- математическое ожидание функции. Докажите, что
\[
\left|\mathbb{E} f(X) - M\right| \le \sqrt{2\pi n}.
\]

б) Покажите, что фактор $\frac{1}{n}$ под экспонентой не может быть 'улучшен'. В частности, неравенство концентрации явно зависит от размерности куба (при условии сохранения гауссовского хвоста по $\varepsilon$).
\end{problem}

\begin{remark}
Используйте тот факт, что для случайной величины $Y > 0$ имеет место равенство $\Exp Y = \int\limits_{0}^{\infty}\PR\{Y \ge x\}dx$.

См.\cite{22} и Barvinok A. Math 710: Measure Concentration. Lecture notes, 2005. http://www.math.lsa.umich.edu/~barvinok/total710.pdf.
\end{remark}

\begin{problem}(Неравенство Талаграна для дискретного куба) Пусть $E = \{-1, 1\}^{n}$ --- дискретный $n$-мерный куб, на множестве вершин которого задана равномерная вероятностная мера $\mu$. Введем на вершинах куба стандартную евклидову метрику. Тогда для всякой выпуклой $1$-липшицевой действительнозначной функции $f$, заданной на $E$, c медианой $M$ для $\varepsilon \ge 0$ имеет место неравенство
\[
\Prob\{\left|f(X) - M\right| \ge \varepsilon\} \le 2\exp\left(\frac{-\varepsilon^{2}}{2n}\right).
\]

а) Покажите, что условие выпуклости функции $f$ нельзя опустить в данном неравенстве.


б) Получите неравенство, аналогичное неравенству Талаграна для концентрации около математического ожидания.
\end{problem}
\begin{remark}
См. Talagrand~M. An Isoperimetric Theorem on the Cube and the Kinitchine-Kahane Inequalities,
Proceedings of the American Mathematical Society, 1988. Pp.\,905-909.

В отличие от предыдущей теоремы в данном неравенстве нет зависимости от размерности куба. Одновременно накладываются два дополнительных требования, во-первых, липшицевость по меньшей -- евклидовой метрике, во-вторых, дополнительно на функцию накладывается условие выпуклости.

Существуют аналогичные неравенства концентрации вокруг математического ожидания с гораздо более точными константами. Тем не менее, даже этот простой способ позволяет получить нужный нам гауссовский хвост, не зависящий от размерности.
\end{remark}

\begin{problem} (Концентрация на сечениях куба) Пусть $E = \{-1, 1\}^{n}$ --- дискретный $n$-мерный куб. Рассмотрим его сечение, состоящее из всех его вершин, содержащих ровно $k$ координат, равных $1$. Введем на вершинах сечения нормированную метрику $d$, равную половине Хэмминговой метрики. При этом расстояние между вершинами, координаты которых отличаются лишь перестановкой пары координат равно единице. На множестве вершин сечения зададим равномерную вероятностную меру $\mu$. Введем также понятие длины дискретного градиента $|\nabla f(X)|$. Обозначим
\[
|\nabla f(X)|^{2} = \sum\limits_{Y : d(X, Y) = 1}|f(X) - f(Y)|^{2},
\]
где суммирование ведется по всем $k(n - k)$ вершинам сечения, удаленным от данной вершины на единицу.

Тогда для всякой действительнозначной функции $f$, заданной на описанном сечении, длина дискретного градиента которой в точках сечения ограничена числом $\sigma$, для $\varepsilon \ge 0$ имеет место неравенство
\[
\PR\{f(X) - \Exp f(X) \ge \varepsilon\} \le \exp\left(\frac{-(n + 2)\varepsilon^{2}}{4\sigma^{2}}\right).
\]
Пусть случайна величина $Y$ имеет гипергеометрическое распределение с параметрами $N, K, n$, то есть для неотрицательного целого $k$
\[
\PR\{Y = k\} = \frac{C_{K}^{k}C_{N - K}^{n - k}}{C_{N}^{n}}.
\]

Получите экспоненциальную оценку на величину $\PR\{Y - \Exp Y \ge \varepsilon\}$.
\end{problem}

\begin{remark}
 Вспомните интерпретацию гипергеометрического распределения и свяжите ее с равномерным распределением на вершинах сечения дискретного куба.
См. Bobkov S.\;G. Concentration of normalized sums and a central limit theorem for noncorrelated random variables, Annals of probability, No.\,4, Pp.\,2884-2907, 2004 и Bobkov S.\;G., Tetali,\;P.
 Modified logarithmic Sobolev inequalities in discrete settings, 
Journal of Theoretical Probability,  No.\,2, Pp.\,289--336, 2006.
\end{remark}

%КАТЯ, ПОСТАРАЙТЕСЬ СДЕЛАТЬ ЭТО ЗАМЕЧАНИЕ ПО ДОСТУПНЕЕ И С КАКОЙ-НИБУДЬ ССЫЛКОЙ

\begin{problem}*
\label{talagran}
В сельском районе, имеющем форму квадрата со стороной $1$, находится $n$ домов $(n\gg 1)$, размерами которых можно пренебречь по сравнению с линейным размером района. Будем считать, что при строительстве домов застройщик случайно (согласно равномерному распределению $R[0,1]^2$)  и независимо выбирал их местоположения. Почтальону необходимо обойти все $n$ домов ровно по одному разу (от любого дома почтальон может направиться к любому другому по прямой). Обозначим через $\ell$ длину наикратчайшего из таких путей (кратчайший гамильтонов путь).
Покажите, что найдется такая константа $c>0$, не зависящая от $n$, что 

\begin{equation*}
\mathbb{P}(\ell - \mathbb{E}\ell \geq t)\leq \exp \biggl(-\frac{t^2}{4c}\biggr). 
\end{equation*}
Можно показать, что $\mathbb{E}\ell\sim\beta\sqrt{n}$, где $\beta$ также не зависит от $n$.

\end{problem}

\begin{remark}
См. книгу Dubhashi D. P., Panconesi A. <<Concentration of measure for the analysis of randomized algorithms>>, Cambridge University Press, 2009.

Приведем несколько неравенств Талаграна. Пусть заданы множества $\Omega_i$, $i=1,\dots,n$, элементарных исходов. На этих множествах заданы вероятностные меры $\PR_i$, $i=1,\dots,n$. Положим 
\begin{equation*}
\Omega= \prod_{i=1}^n\Omega_i,\quad \PR = \prod_{i=1}^n \PR_i.
\end{equation*}
Введем взвешенную метрику Хэмминга:
\begin{equation*}
d_{\alpha}(x,y) = \sum_{x_i\not = y_i} \alpha_i \biggr/ \sqrt{\sum_{i=1}^n \alpha^2_i}
\end{equation*}
и определим $d_{\alpha}(x,A) = \min_{y\in A} d_{\alpha}(x,y)$, $\rho(x,A) = \sup_{\alpha\in \mathbb{R}^n} d_{\alpha}(x,A)$. Пусть $A \in \sigma(\Omega)$. Определим $t$-окрестность ($t\geq$ 0) множества $A$ по формуле 
\begin{equation*}
A_t = \{x\in\Omega: \rho (x,A)\leq t\}.
\end{equation*}
Тогда справедливо неравенство Талаграна:
\begin{equation*}
\mathbb{P}(A)(1-\mathbb{P}(A_t))\leq \exp\biggl(-\frac{t^2}{4}\biggr).
\end{equation*}
\begin{suite}
 %Пусть $X = (X_1,\dots,X_n) $ случайный вектор, значения компонент которого лежат в $[0,1]$. Пусть $F: \mathbb{R}^n\to\mathbb{R}$  выпуклая, липшицева функция с константой $1$ по отношению к метрике Хэмминга ($d(u,v) = \sum_{i=1}^n I(u_i\not=v_i)$,\,\, $u,v \in [0,1]^n$). Пусть $M F(X)$ ~--- медиана $F(X)$, тогда для всех $t>0$ верно
%\begin{equation*}
%\PR (|F(X)-MF(X)|\geq t)\leq 4\mathrm{e}^{-t^2/4}.
%\end{equation*}
%Рассмотрим выпуклую Липшицеву функцию $f$ на ($\mathbb{R}^n$, $\|\cdot\|_2$) с константой Липшица $\sigma$. 

В качестве приложения неравенства Талаграна рассмотрим функцию $h:\; \Omega \to {\rm R}$, $\Omega = \mathbb{R}^n$, удовлетворяющую условию Липшица с константой $\sigma$. Функцию $h$ будем называть проверяемой со сложностью $f$, если при $h(x)\ge s$ существует такое множество $I\subseteq \{1,\ldots ,n\}$ с $\vert I\vert \le f(s)$, что для всех $y\in \Omega $, совпадающих с $x$ в координатах из $I$, выполняется $h(y)\ge s$. Грубо говоря, из выполнения неравенства $h(x)\ge s$ следует, что существует сравнительно небольшое количество 
координат, обеспечивающих выполнение данного неравенства. Тогда из 
неравенства Талаграна можно получить следующее соотношение о плотной 
концентрации случайной величины $h$: для всех $b$ и $t$ 
справедливо 
\[
\PR\left[ {h\le b-t\sqrt {f(b)} } \right]\PR\left[ {h\ge b} \right]\le 
e^{-\frac{t^2}{4}}.
\]
Выбирая либо в качестве $b$, либо в качестве $b-t\sqrt {f(b)} $ медиану случайной величины $h$, получаем результат о плотной концентрации $h$ вокруг своей медианы

\begin{equation*}
\mathbb{P} (|h(X)-M|> t)\leq 4\mathrm{e}^{-t^2/8\sigma^8}.
\end{equation*}
\end{suite}


Неравенство Талаграна для независимых случайных величин. Пусть $X_1,\dots,X_n$ независимые случайные величины в $S$. Для любого класса функций ${\mathcal{F}}$
на $S$ равномерно ограниченного  на $S$ константой $U$ для всех $t>0$ выполнено 
\[
\mathbb{P}\left\{\left\|\sum_{i=1}^n f(X_i)\right\|_{\mathcal{F}} - \mathbb{E}\left\|\sum_{i=1}^n f(X_i)\right\|_{\mathcal{F}}
 \geq t \right\}\leq K\exp\left\{-\frac{t}{UK}\log\left(1+\frac{tU}{V}\right)\right\},
\]
где $K$ - универсальная константа и $V$ удовлетворяет условию 
\[
V\geq \mathbb{E}\sup_{f\in\mathcal{F}}\sum_{i=1}^n f^2(X_i) 
\]
и использованы обозначения  $\|Y\|_{\mathcal{F}} = \sup_{\mathcal{F}}|Y(f)|
$, где $Y: \mathcal{F}\to \mathbb{R}$. 

\end{remark}

%КАТЯ, А ВЫ УВЕРЕНЫ, ЧТО ИМЕННО ТАК? Все как-то странно:)
%КАТЯ ВСЕ ХОРОШО, ТОЛЬКО ЧТО ТАКОЕ ЛИПШИЦЕВА ФУНКЦИЯ ОТНОСИТЕЛЬНО ТОЙ МЕТРИКИ, КОТОРУЮ МЫ ВВЫЕЛИ НЕ ДО КОНЦА ЯСНО, ПОСКОЛЬКУ МЫ ЛИШЬ ОПРЕДЕЛИЛИ РАССТОЯНИЕ ОТ ТОЧКИ ДО МНОЖЕСТВА, НО МНЕ МЕЖДУ ТОЧКАМИ, ЧТО НУЖНО В ОПРЕДЕЛЕНИИ КОНСТАНТЫ ЛИПШИЦА

\begin{problem}[Семейства Леви] 
%Определим для компактного множества $X$ с мерой $\mu$  концентрационную функцию 
%\begin{equation*}
%\alpha(X;\varepsilon) = 1-\inf\biggl\{\mu(A_{\varepsilon})|\, A \subset X, \mu(A)\geq \frac{1}{%2}\biggr\},
%\end{equation*}
%где $A_{\varepsilon} = \bigl\{x\in X | \,\, \rho(s,A) \leq \varepsilon \bigr\}$.
Пусть  $(X_n,\rho_n,\mu_n)$  -- метрическое вероятностное пространство, $X_n$ -- компактное множество с метрикой $\rho_n$, $ \diam X_n\geq 1$ и заданной вероятностной мерой $\mu_n$. Семейство  $(X_n,\rho_n,\mu_n)$ метрических вероятностных пространств называется семейством Леви, если для любого $\varepsilon>0$, $\alpha(X_n,\varepsilon \cdot \diam X_n) \to 0$ для $n\to \infty$ (см. определение концентрационной функции в замечании к задаче \ref{levi}).  Семейство называется нормальным семейством Леви с константами $(c_1,c_2)$, если
\begin{equation*}
\alpha(X_n;\varepsilon) \leq c_1 \exp(-c_2\varepsilon^2 n),
\end{equation*}
Докажите, что следующие семейства будут нормальными семействами Леви.
\begin{enumerate} 
\item \label{first} Пусть на $E^n = \{-1,1\}^n$  задана Хэммингова метрика 
\begin{equation*}
d(s,t) = \frac{1}{n} |\{i:s_i\not=t_i\}|
\end{equation*}
и нормированная считающая мера $\mu$, т.е. $\mu(A) = |A|/2^n$. Докажите неравенство 
\begin{equation*}
\alpha(F_2^n;\varepsilon) \leq \frac{1}{2}\exp(-2\varepsilon^2 n).
\end{equation*}
\item 
Задана группа $\Pi_n$ перестановок $\{1,\dots,n\}$ с заданной нормированной метрикой Хэмминга 
\begin{equation*}
d(\pi_1,\pi_2) = \frac{1}{n} |\{i:\pi_1(i)\not=\pi_2(i)\}|
\end{equation*}
 и нормированной считающей мерой (см. пункт \ref{first}). Докажите неравенство 
\begin{equation*}
\alpha(\Pi_n;\varepsilon) \leq \exp\Bigr(-\varepsilon^2n/64\Bigl).
\end{equation*}
\end{enumerate}
\end{problem}
\begin{remark}
Примеры приведены в статье  Milman V.D. The heritage of P.~Levy in geometrical functional analysis // Asterisques. 1988. V. 157-158. P. 273-302, см. также книгу M. Ledoux <<The Concentration of Measure Phenomenon>>, American Mathematical Soc., 2005, а также теорему Талаграна в замечании к предыдущей задаче.

\end{remark}

\begin{problem}(Кац, Секей)
Рассмотрим полином с вещественными коэффициентами
\begin{equation*}
a_0+a_1t+\ldots+a_{n-1}t^{n-1},
\end{equation*}
где $(a_0,a_1,\dots,a_{n-1})$ точка на единичной сфере $S_n(1)$. 
 Среднее число вещественных корней полинома определим следующим образом:
$$
\mathbb{E} N = \frac{1}{|S_n(1)|}\int_{S_n(1)}N(a)d\sigma,
$$
где  $N(a)$ число вещественных корней полинома, $d\sigma$~--- элемент поверхности единичной сферы площадью
$$|S_n(1)| = \frac{(2\pi)^{n/2} }{ \Gamma\left(\frac{n}{2}\right) }.$$


\begin{enumerate}
\item 
Покажите, что среднее число вещественных корней полинома с коэффициентами на единичной сфере равно среднему числу вещественных корней полинома, коэффициенты которого независимы и распределены стандартно нормально, то есть 
\begin{equation*}
\mathbb{E} N = (2\pi)^{-n/2}\int_{-\infty}^{\infty}\dots\int_{-\infty}^{\infty}N(a)\exp \left(-\frac{1}{2}\|a\|^2\right)\,da_0,\dots,\,da_{n-1}.
\end{equation*}
\item 
Пусть $\mathbb{E}N^{(1)}$, $\mathbb{E}N^{(2)}$, $\mathbb{E}N^{(3)}$, $\mathbb{E}N^{(4)}$~--- средние значения числа вещественных корней, заключенных в интервалах $(-\infty,-1)$, $(-1,0)$, $(0,1)$, $(1,\infty)$ соответственно.
Покажите, что 
\begin{equation*}
\mathbb{E}N^{(1)}=\mathbb{E}N^{(2)}=\mathbb{E}N^{(3)}=\mathbb{E}N^{(4)} 
\end{equation*}
и
\begin{equation*}
\mathbb{E}N^{(i)}\sim (2\pi)^{-1}\ln n, \quad n\to\infty.
\end{equation*}
%\item
\end{enumerate}
\end{problem}
\begin{remark}
Н.Б. Маслова в статье О распределении числа вещественных корней случайных полиномов // ТВП, 1974, Т. 19, В. 3, с. 488–-500 доказала следующую теорему: если коэффициенты $X_i$ случайного алгебраического уравнения 
\[
\sum_{j=1}^n X_jz^j=0
\]
являются независимыми одинаково распределенными случайными величиными с нулевыми математическими ожиданиями и $$\mathbb{E}\left(|X_j|^{2+\epsilon}\right)<\infty$$ для некоторого положительного $\epsilon$, то число действительных корней этого уравнения распределено нормально с математическим ожиданием $\frac{2}{\pi}\ln n$ и стандартным отклонением $2\sqrt{\pi^{-1}(1-2\pi^{-1})\ln n}$.
\end{remark}


\begin{problem}$^{**}$
Дан случайный граф (модель Эрдеша--Реньи, см. задачу \ref{sec:erdRenyi} раздела \ref{hard}) $G\left( {n,\;p} \right)$ с $n$ вершинами и 
вероятностью появления каждого ребра $p$. Пусть $p\ge \sqrt {\frac{2 \ln 
n}{n}} $, причем длины ребер $r_{ij} $ являются независимыми случайными величинами, имеющими равномерное распределение на отрезке 
$\left[ {0,\;2r} \right]$. Покажите, что тогда почти наверное граф $G\left( 
{n,\;p} \right)$ имеет гамильтонов цикл, причём длина почти всех 
гамильтоновых циклов стабилизируется около $nr$.
\end{problem}

\begin{problem}$^{**}$
Дан случайный граф (модель Эрдеша--Реньи, см. задачу \ref{hard}.\ref{sec:erdRenyi}) $G\left( {n,\;p} \right)$ с $n$ вершинами и 
вероятностью появления каждого ребра $p\in \left[ {\varepsilon ,1} \right]$, 
$\varepsilon >0$. Вес каждого появившегося случайного ребра разыгрывается  независимо 
согласно равномерному распределению на отрезке $\left[ {0,2} 
\right]$.  Источник и сток выбираются случайно. Обозначим через $S_n $ значение максимального потока для 
полученного случайного взвешенного графа. Покажите, что 
${S_n } 
\mathord{\left/ {\vphantom {{S_n } {pn}}} \right. \kern-\nulldelimiterspace} 
{pn}\buildrel p \over \longrightarrow 1.$
\end{problem}

\begin{problem}[Cтохастическое агрегирование; В.И. Опойцев]
\begin{enumerate}
\item На рынке имеется $n$ 
продавцов, каждый из которых может продать свой товар в объеме $x_k $ для 
$k$-го товара. Спрос на товары обеспечивают покупатели. Пусть $y_k$ -- спрос на 
товар $k$. Общий объем сделок $L_n=\sum\limits_{k=1}^n {\min 
\left\{ {x_k ,y_k } \right\}} $. Предложите такой естественный способ 
определения вероятностной меры на двух симплексах
\[
S_X =\left\{ {\vec {x}\ge \vec {0}:\;\;\sum\limits_{k=1}^n {x_k } =X} 
\right\},
\quad
S_Y =\left\{ {\vec {y}\ge \vec {0}:\;\;\sum\limits_{k=1}^n {y_k } =Y} 
\right\},
\]
чтобы нашлась функция $f\left( {X,Y} \right)$, удовлетворяющая  $$L_n 
\mathord{\left/ {\vphantom {L n}} \right. \kern-\nulldelimiterspace} 
n\buildrel p \over \longrightarrow f\left( {X,Y} \right).$$

\item Пусть $\vec {y}=A\vec {x}$, $A=\left\| {a_{ij} } 
\right\|_{i,j=1,1}^{l,n} $, $X=\left\langle {\vec {p},\vec {x}} 
\right\rangle $, $Y=\left\langle {\vec {q},\vec {y}} \right\rangle $. 
Матрицы и векторы предполагаются положительными. Легко проверить, что
\[
Y=\lambda \left( {\vec {x}} \right)X,
\quad
\lambda \left( {\vec {x}} \right)=\sum\limits_{i,j} {\frac{q_i a_{ij} }{p_j 
}} \frac{p_j x_j }{X}\mathop =\limits^{def} \sum\limits_{i,j} {b_{ij} } z_j 
.
\]
Считая $z_j >0$ независимыми одинаково распределенными с.в.: $\Exp z_j =m_j $ 
($\sum_{j} m_j  =1)$, $\Var z_j =\sigma _j^2 $ (например, $z_j \in 
\left[ {0,2n^{-1}} \right])$, покажите, что если выражение
\[
\frac{\mathop {\max }\limits_j \sum\limits_i {b_{ij} } \cdot \mathop {\max 
}\limits_j \sigma _j }{\mathop {\min }\limits_j \sum\limits_i {b_{ij} } 
\cdot \mathop {\min }\limits_j m_j }
\]
равномерно ограничено с ростом $n$, то существует такое число $\bar {\lambda 
}$, что с вероятностью стремящейся к единице $Y\simeq \bar {\lambda }X$.

\end{enumerate}
\end{problem}


\begin{problem}(Устойчивые системы большой размерности; В.И. Опойцев)
Из курсов функционального анализа и вычислительной математики хорошо известно, что если спектральный радиус матрицы 
$A=\| a_{ij}\|_{i,j=1}^{n}$ меньше единицы, $\rho(A)<1$, то итерационный процесс $x^{k+1}=A x^k + b$ 
(СОДУ $\dot{x}=-x+A x+ b$), вне зависимости от точки старта $x^0$, 
сходится к единственному решению уравнения $x^*=Ax^*+ b$. 
Скажем, если $\| A\|=\max\limits_{i} \sum\limits_j |a_{ij}|<1$, то и $\rho(A)<1$ (обратное, конечно, не верно). Предположим, что 
существует такое $\varepsilon>0$, что 
$$
\frac{1}{n}\sum\limits_{i,j=1}^n |a_{ij}|<1-\varepsilon . 
\quad (S)
$$
Очевидно, что отсюда не следует: $\rho(A)<1$. 
Тем не менее, введя на множестве матриц, удовлетворяющих условию $(S)$, равномерную меру, покажите, что относительная мера тех матриц 
(удовлетворяющих условию $(S)$), для которых спектральный радиус не меньше единицы, стремится к нулю 
с ростом $n$ ($\varepsilon$ --- фиксировано и от $n$ не зависит). 
\end{problem}
\begin{ordre}

1. Покажите, что  достаточно рассматривать матрицы с неотрицательными элементами. 

2.  Покажите, что достаточно доказать утверждение задачи на множестве матриц, удовлетворяющих условию 
$$
\frac{1}{n}\sum\limits_{i,j=1}^n a_{ij}=1-\varepsilon . 
\quad (SE)
$$

3. Далее положим $a_{ij}\in \mathrm{Exp} \bigl( n/(1-\varepsilon)\bigr)$ --- независимые одинаково распределенные случайные величины. Покажите, что при $n\to\infty$ распределение элементов случайной матрицы $A=\| a_{ij}\|_{i,j=1}^n$ 
будет сходиться к равномерному распределению на множестве матриц, удовлетворяющих ($SE$). 

4. Введя обозначения
$P_n={\mathbb P}(\| A\|\ge 1)\ge {\mathbb P}(\rho(A)\ge 1)$, воспользуйтесь неравенством Чебышёва

$$
P_n\le n {\mathbb P}\Bigl( \sum\limits_{j=1}^n a_{1j}\ge 1 \Bigr)=n {\mathbb P}\Bigl( X\ge 1 \Bigr)\le 
n {\mathbb P}\Bigl( |X-(1-\varepsilon)|\ge \varepsilon \Bigr)=
$$
$$
=n {\mathbb P}\Bigl( |X-{\mathbb E}X|\ge \varepsilon \Bigr)\le \frac{n}{\varepsilon^4} {\mathbb E}(X-{\mathbb E}X)^4=
O\Bigl(\frac{1}{n}\Bigr) \xrightarrow{n\to\infty} 0 . 
$$
\end{ordre}



\begin{problem}[Красносельский--Крейн]
Рассмотрим систему линейных алгебраических уравнений 
\begin{equation*}
x=Ax+b,
\end{equation*}
где $A$ положительно определенная неособенная матрица, собственные числа $\lambda_1\leq\dots\leq\lambda_n$ которой меньше единицы, $b$~--- заданный и  $x$ ~--- искомый векторы $n$-мерного подпространства. Пусть эта система решается при помощи итерационного процесса 
\begin{equation*}
x_{m+1}= Ax_{m}+b,\quad m=1,2,\dots,
\end{equation*}
 который заканчивается на $p$-м шаге, если вектор--невязка $\delta_p = x_{p+1}-x_{p}$ попадает в шар радиуса $\alpha$ с центром в нуле. 
Ошибка итерационного процесса $\varepsilon_m = x^{*} - x_m$, где  $x^{*}$ истинное решение системы уравнений, связана с невязкой  (проверить)
 \begin{equation*}
\varepsilon_m = (I-A)^{-1}\delta_m,
\end{equation*}
поэтому  исходя из значений вектора невязки можно определить вероятностное распределение  ошибок. 

Оказывается, что наиболее вероятными ошибками являются максимальные.
А именно, пусть начальная ошибка равномерно распределена в шаре $T$ радиуса $R$, тогда для любого $\eta<1$ вероятность выполнения следующего неравенства стремится к единице при $R\to\infty$
\begin{equation*}
\eta \frac{\alpha \lambda_n}{1-\lambda_n}\leq \|\varepsilon\| \leq \frac{\alpha}{1-\lambda_n}.
\end{equation*} 

Проверьте это для случая $n=2$.

\end{problem}

\begin{remark}
Пусть итерационный процесс заканчивается на шаге $p$, если вектор-невязка $\delta_p=x_{p+1}-x_p$ попадает в окрестность $G$ нуля. Пусть $G$~--- шар радиуса $\alpha$. Тогда ошибка $\varepsilon_p$ попадет в множество $G_0 = (I-A)^{-1}G$, которое является эллипсоидом с полуосями длины $\frac{\alpha}{1-\lambda_1}$ и $\frac{\alpha}{1-\lambda_2}$.
Обозначим $G_{-1}=AG_0$ ~--- эллипсоид с полуосями $\frac{\alpha\lambda_i}{1-\lambda_i}$, $i=1,2$. 
 
\imgh{100mm}{krein.pdf}{Множества ошибок в случае матрицы $2\times 2$. Закрашенная темно-серым цветом часть соответствует области из утверждения задачи при $\eta$ близком к $1$.}

Обозначим $G_m = A^{-m}G_0$. Процесс заканчивается на $p$-м шаге, если $\varepsilon_0\in G_p$ и $\varepsilon_0\not\in G_p$, те когда $\varepsilon_0$ находится в слое $G_p-G_{p-1}$.
При этом окончательная ошибка будет находиться в слое $A^{p}(G_p-G_{p-1}) = G_0-G_{-1}$.Таким образом окончательная ошибка принадлежит $G_{-1}$ только в случае, если $\varepsilon_0$ принадлежит $G_0$. 
Вероятность нахождения окончательной ошибки в элементе слоя $G_0-G_{-1}$ равна сумме вероятностей нахождения начальной ошибки в элементах $\Delta=A^{-m}\Delta_0$ ($m=0,1,\dots$).

Предположим, что начальная ошибка $\varepsilon_0$ равномерно распределена в шаре $T$ достаточно большого радиуса. В этом случае вероятность $P(\Delta_m)$ нахождения начальной ошибки в элементе $\Delta_m\in T$ пропорциональна объему этого элемента. Подсчитайте вероятность $P(\Delta_0)$ попадания окончательной ошибки в элемент $\Delta_0$ и убедитесь, что вероятность попадания окончательной ошибки в точку слоя $G_0-G_{-1}$ для данной точки тем больше, чем позже эта точка выйдет из $T$ при последовательном применении к ней операции $A^{-1}$. 

Пусть $e_i$~--- собственные векторы матрицы $A$, соответствующие собственным числам $\lambda_i$, $i=1,2$.
Для доказательства утверждения задачи оцените вероятность того, что окончательная ошибка $\varepsilon = \sum_{i=1}^n\xi_ie_i$ не удовлетворяет неравенству (утверждению задачи). Перейдя к пределу по $R\to\infty$ получите, что вероятность невыполнения утверждения задачи стремится к нулю.


\end{remark}

%КАТЯ, НЕОБХОДИМО ТАКЖЕ ДОБАВИТЬ ЗАМЕЧАНИЕ С РИСУНКОМ :)

\begin{problem}[Физическая интерпретация концентрации меры на сфере]
Имеется $n$ частиц массы $m$  со скоростями $v_i$, $i=1,\dots,n$. Известно, что вектор скоростей молекул идеального газа равномерно
распределен по поверхности постоянной энергии. Суммарная кинетическая энергия $E_n$ растет пропорционально $n$, то есть 
\begin{equation*}
\frac{1}{2}mv_1^2+\cdots+\frac{1}{2}m v_n^2 = E_n;\quad \sum_{i=1}^n v^2_i=\frac{2E_n}{m}\asymp n.
\end{equation*}
Получите закон распределения Максвелла скоростей частиц одномерного идеального газа. 
\end{problem}

\begin{ordre}
В решении задачи 1 этого раздела перейдите к термодинамическому пределу, когда $n\to\infty$, $r = \sigma n^{1/2}$, чтобы получить закон распределения Максвелла скоростей частиц.
\end{ordre}

\begin{remark}
Равномерное распределение на поверхности постоянной энергии возникло из-за того, что инвариантной (и предельной, то есть возникающей на больших временах, по эргодической гипотезе) мерой для гамильтоновой системы будет как раз равномерная мера  по теореме Лиувилля (фазовый объем сохраняется). Поскольку выполняется закон сохранения энергии, то система  ``живет'' \mbox{~на} поверхности постоянной энергии. Следовательно, носитель инвариантной меры сосредоточен именно на этой поверхности. 

Приведем для справки результат, полученный  для выпуклых тел с заданной на них равномерной мерой (теорема Б. Клартага в статье Milman V.D. Geometrization of Probability // Progress in Mathematics. 2008. V. 265, http://www.math.tau.ac.il/~milman/, также см. статью Klartag B. A central limit theorem for convex sets // Inventiones mathematicae
April 2007, Volume 168, Issue 1, pp. 91--131). Существует последовательность $\epsilon_n\to 0$, ${n\to\infty}$ для которой выполнено: пусть $K\in \mathbb{R}^n$ выпуклое компактное множество с непустой внутренностью, случайный вектор $X$ распределен равномерно в $K$, тогда существуют вектор $\theta\in\mathbb{R}^n$, $t_0\in\mathbb{R}$ и $\sigma>0$, что выполнено
\[
\sup_{A\in\mathbb{R}}\left|\mathbb{P}\left\{\sum_{i=1}^n X_i\theta_i\in A\right\} - \frac{1}{\sqrt{2\pi}\sigma}\int_{A}{\rm e}^{-\frac{(t-t_0)^2}{2\sigma^2}}\,dt\right|\leq \epsilon_n,
\]
где супремум берется по всем измеримым множествами $A\in\mathbb{R}$. Eсли $\mathbb{E}X_i=1$, $\mathbb{E}X_iX_j = \delta_{ij}$, то $t_0=0$, $\sigma=1$, то можно найти такой единичный вектор  $\theta$, что выполняется приведенное неравенство.

\end{remark}

\begin{problem}[Лемма Пуанкаре]
Пусть $X_n$ -- случайный вектор с равномерным распределением на единичной сфере в ${\mathbb R}^n$. Равномерное распределение 
характеризуется тем, что оно инвариантно относительно группы ортогональных преобразований. Пусть $Y_n$ обозначает первую координату $X_n$. 
Докажите, что $\sqrt{n}\, Y_n \xrightarrow{d}N(0,1)$ при $n\to\infty$. Заметим, что в статистической физике с помощью утверждения 
этой задачи получался закон распределения Максвелла скоростей частиц одномерного идеального газа. 
\end{problem}

\begin{ordre}

См. предыдущую задачу. Решение задачи содержит в себе способ генерирования равномерного распределения. 
Пусть $\xi_1,\ldots, \xi_n$ --- независимые в совокупности с.в., имеющие одинаковое распределение $N(0,1)$. Рассмотрим случайный вектор 
$Z_n=(\xi_1,\xi_2,\ldots,\xi_n)$. Тогда $Z_n\in N(0,I_n)$, $I_n$ --- единичная матрица размера $n$. 

Показажите, что $Z_n$ инвариантно относительно группы ортогональных преобразований. Заметим, что распределения 
$$
X_n \quad\text{ и } \quad \frac{Z_n}{\|Z_n \|_{{\mathbb R}^n}} \quad \text{ совпадают. }
$$
Поэтому имеет место равенство по распределению с.в. 
$$
Y_n=\frac{\xi_1}{\sqrt{\xi_1^2+\ldots+ \xi_n^2}} 
$$
$$
\Rightarrow \quad \sqrt{n}Y_n = \frac{\xi_1}{\sqrt{(\xi_1^2+\ldots+ \xi_n^2)/n}} . 
$$
Применить теорему Колмогорова у.з.б.ч. для $\frac{\xi_1^2+\ldots+ \xi_n^2}{n}$. 

\end{ordre}

\begin{problem}[Геометрическая интерпретация закона больших чисел]
Рассмотрим куб $C^n = [-1,1]^n$ в евклидовом пространстве $\mathbb{R}^n$. Пусть $\xi_i$, $i=1,\dots,n$ независимые случайные величины с равномерным распределением на $[-1,1]$. Приведите геометрическую интерпретацию закона больших чисел.
\end{problem}

\begin{ordre} 
Рассмотреть объем  следующего множества ---   пусть $\mathcal{H}$ часть гиперплоскости, содержащаяся в кубе и перпендикулярная главной диагонали куба, т.е.  $\sum_{i=1}^n x_i = 0$. Необходимо подсчитать объем $\varepsilon\sqrt{n}$-окрестности $\mathcal{H}$. 
\end{ordre}



\begin{problem} [В.И. Опойцев]
\begin{enumerate}
\item[а)] Пусть имеются абсолютно непрерывные (имеющие плотность) независимые случайные величины $X_1,\ldots, X_n$ и пусть 
$Y_n=G_n(X_1,\ldots, X_n)$ --- также случайная величина. Докажите следующее неравенство, описывающее нелинейный закон больших чисел: 
$$
\Var Y_n\leqslant \int_{{\mathbb R}^n} \sum\limits_{i=1}^{n} \Bigl(\frac{\partial G_n(x_1,\ldots, x_n)}{\partial x_i} \Bigr)^2 f_i^*(x_i)
\prod\limits_{j\ne i} f_j(x_j)\, dx_1\ldots dx_n , 
$$
где сопряженные плотности $f_i^*(x_i)$ существуют и определяются следующим образом 
$$
f_i^*(x_i)=\mu_i(\infty)\int\limits_{-\infty}^{x_i} f_i(t)\, dt -\mu_i(x_i), \quad 
\mu_i(x)=\int\limits_{-\infty}^{x} tf_i(t)\, dt . 
$$

\item[б)] Пусть независимые одинаково распределенные случайные величины $X_1,\ldots, X_n$ имеют равномерное распределение на отрезке 
$[0,1]$ (часто пишут $X_i\in R[0,1]$), а 
$$
\max\limits_{x\in [0,1]^n} \|\nabla G_n(x_1,\ldots, x_n) \|\xrightarrow{n\to\infty} 0 . 
$$
Тогда $Y_n\xrightarrow{P}{\mathbb E}Y_n$ при $n\to\infty$. 

\item[в)] Пусть независимые одинаково распределенные случайные величины $X_1,\ldots, X_n$ имеют равномерное распределение на отрезке 
$[0,1]$, а $Y_n=G_n(X_1,\ldots, X_n)=\max\limits_{i=1,\ldots, n} X_i$. Тогда $Y_n\xrightarrow{p}{\mathbb E}Y_n$ при $n\to\infty$. 

\end{enumerate}
\end{problem}

\begin{remark}
См. В. Босс Лекции по математике. Т.4: Вероятность, информация, статистика, 2005.
\end{remark}

%По мотивам лекции Голубева, обзора Лугоши.

%\begin{enumerate}
%\item
\begin{problem}[Неравенство Чернова]
Докажите, что неравенство Чернова для неотрицательной случайной величины $X$
\begin{equation*}
\PR\{ X >t\}\leq \inf_{s>0}\mathbb{E}\exp(sX-st)
\end{equation*}
 дает более завышенную границу по сравнению с моментной границей
\begin{equation*}
\PR\{ X >t\}\leq \min_{q>0}\mathbb{E}[X^q]t^{-q},
\end{equation*}
 то есть 
\begin{equation*}
\min_{q>0}\mathbb{E}[X^q]t^{-q}\leq \inf_{s>0}\mathbb{E}\big[\text{e}^{s(X-t)}\bigl].
\end{equation*}
\end{problem}

\begin{ordre}См. \cite{BLM}. Использовать следствие из неравенства Маркова: для монотонной возрастающей неотрицательной функции $\phi(\cdot)$ и произвольной случайной величины $X$ верно
\begin{equation*}
\PR\{\phi(X)\geq \phi(t)\}\leq \frac{\mathbb{E}\phi(X)}{\phi(t)}.
\end{equation*}
\end{ordre}

%\item 
%\begin{problem}[Неравенство Чернова для суммы случайных величин]

%\end{problem}

%\item


\begin{problem}[Лемма Хёфдинга] Пусть $X$--- случайная величина, такая что $\mathbb{E}X =0$, $a\leq X\leq b$. Покажите, что для $s>0$ верно
\begin{equation*}
\mathbb{E}\exp(s X)\leq \exp\bigg[\frac{s^2(b-a)^2}{8}\biggr].
\end{equation*}
\end{problem}

\begin{ordre}
Используя выпуклость экспоненты, для $a\leq x\leq b$

\begin{equation*}
\text{e}^{sx} \leq \frac{x-a}{b-a} \text{e}^{sb}+\frac{b-x}{b-a}\text{e}^{sa},
\end{equation*}

получите 

\begin{equation*}
\mathbb{E}\text{e}^{sx} \leq \text{e}^{\phi(u)},
\end{equation*}
где $u = s(b-a)$, $\phi(u) = -pu+\log(1-p+p\text{e}^u)$, $p = -a/(b-a)$.
Найдите $\phi^{\prime\prime}(u)$,   $\phi(0)$, $\phi^{\prime}(0)$.
Покажите, что 
\begin{equation*}
\phi^{\prime\prime}(u)\leq \frac{1}{4}.
\end{equation*}

Используя формулу Тейлора, получите 
\begin{equation*}
\phi(u) \leq \frac{u^2}{8}\leq \frac{s^2(b-a)^2}{8}.
\end{equation*}
\end{ordre}

\begin{problem}[Теорема Хёфдинга] Пусть $\xi_t$, $t\in T$ -- независимые случайные величины, такие что $\xi_t\in[a,b]$. Докажите, что 
\begin{equation*}
\PR\bigg\{\bigg|\frac{1}{n}\sum_{t\in T}\big(\xi_t-\mathbb{E}\xi_t\bigr)\biggr|\geq x\biggr\}\leq 2\exp\bigg\{-\frac{2nx^2}{(b-a)^2}\biggr\}.
\end{equation*}
\end{problem}
\begin{ordre}
Введите случайную величину $\xi = \frac{1}{n}\sum_{i=1}^n(\xi_i-\mathbb{E}\xi_i)$.
Воспользуйтесь неравеством Чернова и леммой Хёфдинга для $\xi$ и получите
\begin{equation*}
\PR\{\xi>x\}\leq \exp\bigg\{\min_{\lambda}\bigg[-\lambda x + \frac{\lambda^2}{8}\frac{(b-a)^2}{n}\biggr]\biggr\}.
\end{equation*}
Затем найдите оптимальное $\lambda$.  Аналогичное неравенство справедливо для $-\xi$.
\end{ordre}

\begin{problem}[Неравенство Беннетта]
Пусть $X_1,\dots, X_n$ независимые центрированные ограниченные случайные величины, такие, что с вероятностью $1$ выполнено $|X_i|\leq c$.
Пусть $\sigma^2 = \sum_{i=1}^n\Var\{X_i\}$.
Покажите, что для любого $t>0$ 
\begin{equation*}
\PR\bigg\{\sum_{i=1}^n X_i>t\biggr\}\leq \exp\bigg(-\frac{n\sigma^2}{c^2}h\bigg(\frac{ct}{n\sigma^2}\biggr)\biggr),
\end{equation*}
где $h(u) = (1+u)\log(1+u)-u$ для $u\geq 0$.
\end{problem}
\begin{ordre}
Введем $\sigma_i^2 = \mathbb{E}[X_i^2]$ и $F_i = \sum_{r=2}^{\infty}\frac{s^{r-2}\mathbb{E}[X_i^r]}{r!\sigma_i^2}$.
Используя разложение для ряда Тейлора $\exp(sX)$, показать, что 
\begin{equation*}
\mathbb{E}[e^{sX_i}]\leq \exp(s^2\sigma^2_iF_i).
\end{equation*}

Из ограниченности  $X_i$ получите оценку
\begin{equation*}
F_i\leq \frac{\exp(sc)-1-sc}{(sc)^2}.
\end{equation*} 
Далее воспользуйтесь неравенством Чернова для $X_i$ и минимизируйте правую часть в неравенстве Чернова по $s$.
\end{ordre}

\begin{problem}[Неравенство Бернштейна \cite{BLM}]
\label{bernstain}
Докажите, что при выполнении условий предыдущей задачи для любого  $\varepsilon>0$ верно следующее неравенство
\begin{equation*}
\PR\bigg\{\frac{1}{n}\sum_{i=1}^n X_i>\varepsilon\biggr\}\leq \exp\bigg(-\frac{n\varepsilon^2}{2\sigma^2+2c\varepsilon/3}\biggr).
\end{equation*}
\end{problem}

\begin{ordre} 
Покажите, что верно элементарное неравенство 
\begin{equation*}
h(u)\geq \frac{u^2}{2+2u/3}
\end{equation*}
и используйте неравенство Беннетта.

\begin{comment}
Отметим, что в случае схемы Пуассона, т.е. если рассмотреть последовательность $\xi_1,\dots,\xi_n$ с распределнием Бернулли с вероятностью успеха $\lambda/n$,   оценка в асимптотике по $n$ принимает вид, аналогичный  неравенству больших уклонений.
\end{comment}

\end{ordre}
%\end{enumerate}


\begin{problem}[Неравенство Азумы--Хёфдинга]
Пусть $\{X_i\}_{i=0}^{\infty}$ последовательность со следующим свойством (см. замечание к задаче 3.8)
\begin{equation*}
\mathbb{E}(X_n|X_1,\dots,X_{n-1}) =X_{n-1},
\end{equation*}
и пусть $Y_i = X_i-X_{i-1}$ соответствующая последовательность приращений (мартингальная разность). Покажите, что если существуют такие $c_i>0$, что $|Y_i|\leq c_i$ для всех $i$, то
\begin{equation*}
\PR\Bigl\{\sum_{i=1}^{m}Y_i \geq t\Bigr\}\leq 2\exp\bigg\{\frac{-t^2}{2\sum_{i=1}^{m}c^2_i}\biggl\}.
\end{equation*}
\end{problem}
\begin{ordre}

%\begin{enumerate}
%\item \textit{Использовать теорему Дуба.} 
%%\textbf{TODO}
%\item \textit{Cпособ для доказательства не равномерного варианта теоремы %использует результат задачи 12.}

Вначале необходимо докажите следующее утверждение.
 Пусть $Y$ случайная величина,  $Y\in [-1,+1]$ и $\mathbb{E}[Y]=0$. Тогда для любого $t\geq 0$ верно 
\begin{equation*}
\mathbb{E}[\exp(tY)]\leq \exp(t^2/2).
\end{equation*}
Для этого необходимо использовать выпуклость $\exp(tx)$, а именно для $x\in[-1,1]$ верно
\begin{equation*}
\text{e}^{tx}\leq \frac{1}{2}(1+x)\text{e}^{t} +\frac{1}{2}(1-x)\text{e}^{-t}.
\end{equation*}
Подсчитайте оценку математического ожидания $\mathbb{E}[\text{e}^{tY}]$ используя разложение экспоненты в ряд Тейлора и элементарный факт $(2n)!>2^nn!$. Затем покажите, что 
\begin{equation*}
\mathbb{E}\exp\biggl(s\sum_{j=1}^m Y_j\biggr) = \mathbb{E}\biggl[\exp\biggl(s \sum_{j=1}^{m-1} Y_j\biggr)\mathbb{E}[\exp(sY_m)|X_{1},\dots,X_{m-1}]\biggl]. 
\end{equation*}
Используйте неравенство Чернова
\begin{equation*}
\PR[Y_1+\dots+Y_m>t]\leq \exp\bigg[-st+\sum_{i=1}^mc^2_i s^2/2\biggr].
\end{equation*}
Остается минимизировать правую часть неравенства по $s$.

\end{ordre}


\begin{problem}
Пусть независимые одинаково распределенные невырожденные ($\not \equiv \text{const}$) случайные величины $\xi_1,\xi_2,\dots$ с математическим ожиданием $m$ удовлетворяют \textit{условию Крамера}, то есть существует такая окрестность нуля, что для любого $\lambda$ из этой окрестности 
\begin{equation*}
\mathbb{E}e^{\lambda \xi}<\infty.
\end{equation*}
Пусть 
$$
S_n = \xi_1+\dots+\xi_n, \quad \psi(\lambda) = \ln \mathbb{E}\exp(\lambda\xi)
$$ 
и 
\begin{equation*}
H(a) = \sup_{\lambda}[a\lambda - \psi(\lambda)],\quad a\in \mathbb{R},
\end{equation*}
Покажите, что верно следующее неравенство Дуба
\begin{equation*}
\PR\biggl\{\sup_{k\geq n}\biggr|\frac{S_k}{k}-m\biggl|>\varepsilon\biggl\}\leq 2\exp\biggl(-\min\Big[H(m-\varepsilon),H(m+\varepsilon)\Bigr]n\biggr).
\end{equation*}
\end{problem}

\begin{ordre}
Зафиксируйте $n\geq 1$ и положите 
\begin{equation*}
\kappa  = \inf \biggl\{k\geq n: \frac{S_k}{k}>a\biggr\}, 
\end{equation*}
считая $\kappa = \infty$, если $\frac{S_k}{k}\leq a$, $k\geq n$.
Пусть $\lambda>0$  и $\lambda a - \psi(\lambda)\geq 0$. 
Показать, что 
\begin{equation*}
\begin{split}
&\mathbb{P}\biggl\{\sup_{k\geq n}\frac{S_k}{k}>a\biggr\} = \mathbb{P}\biggl\{\frac{S_{\kappa}}{\kappa}>a,\,\kappa<\infty\biggr\} \\
&\leq \mathbb{P}\biggl\{\exp\Bigl(\lambda S_{\kappa} - \kappa  \psi (\lambda)\Bigr)>\exp\Bigl(n\lambda -n  \psi(\lambda)\Bigr),\kappa<\infty \biggr\}\\
&\leq 
 \mathbb{P}\biggl\{\sup_{k\geq n}\exp\Bigl(\lambda S_{k} - k  \psi (\lambda)\Bigr)>\exp\Big(n\lambda -n  \psi(\lambda)\Bigr)\biggr\}.
\end{split}
\end{equation*}

Затем, необходимо воспользоваться тем фактом, что последовательность случайных величин $Y_k = \exp (\lambda S_k-k \psi(\lambda))$, $k\geq 1$ является мартингалом (см. задачу \ref{sec:doob}  раздела \ref{hard}).

Для момента остановки
$\tau = \inf \{k\leq n: Y_k\geq \lambda \}$, $\tau = n$ если $\max_{k\leq n}Y_k <\lambda$ верна теорема Дуба (см. задачу \ref{sec:doob}  раздела \ref{hard}).
Тогда  из неравенства Маркова для любого $x>0$
\begin{equation*}
x \cdot \mathbb{P}\biggl\{\sup_{k\geq n} Y_k \geq x\biggr\}\leq \mathbb{E}Y_n.
\end{equation*}
Отсюда
\begin{equation*}
\mathbb{P}\biggl\{\sup_{k\geq n}\frac{S_k}{k}>a\biggr\}\leq \exp\biggl\{-n \Bigl(\lambda a -  \psi(\lambda)\Bigr)\biggr\}.
\end{equation*}
Рассмотрите случаи $a>m$ и $a<m$.

\end{ordre}
%\begin{problem}
%Что можно сказать о том, как соотносятся между собой неравенство Бернштейна и Хевдинга? Рассмотреть неравенство Бернштейна в случаях, когда $\sigma^2>\varepsilon$ и когда $\sigma^2<\varepsilon$. Что можно сказать про достижимость неравенства Бернштейна? 
%\begin{remark}
%Рассмотреть предельную теорему Пуассона.
%\end{remark}
%\end{problem}



\begin{problem}
Докажите, что для последовательности независимых одинаково распределенных с.в. $\xi_1,\xi_2,\dots$ с математическими ожиданиями $m = \mathbb{E} \xi_i$, дисперсиями $\Var \xi_i = d$ и функцией распределения $F(x)$ верна следующая оценка вероятности больших отклонений %$\PR\Bigl\{\Bigl|\sum_{i=1}^n\xi_i -n m \Bigr|\geq %n\varepsilon \Bigr\}$: 

\begin{equation*}
\PR\Bigl\{\Bigl|\sum_{i=1}^n \xi_i -n m \Bigr|\geq n\varepsilon \Bigr\} \leq B_n(\phi(t_0))^n \exp\big(-t_0 n\varepsilon \bigr) ,
\end{equation*}
где 
%\begin{equation*}
$\lim_{n\to\infty}B_n=\frac{1}{2},$
%\end{equation*}
%\begin{equation*}
$\phi(t)=\int_{-\infty}^{\infty}\text{e}^{t x}\,d F(x),$
%\end{equation*}
%\begin{equation*}
$m(t) = \frac{\phi^{\prime}(t)}{\phi(t)},$
%\end{equation*}
%\begin{equation*}
$r(\lambda_0) = \text{e}^{-\lambda_0 c}R(\lambda_0),$
%\end{equation*}
значение $t_0$ удовлетворяет условию  $m(t_0) = \varepsilon$.

\end{problem}
\begin{remark}


Если использовать для оценки  $\PR\Bigl\{\Bigl|\sum_{i=1}^n\xi_i - n m\Bigr|\geq s\Bigr\}$ неравенство Чебышева, то при $s = \varepsilon \sqrt{n}$ и $s=\varepsilon n$, где $\varepsilon$~--- некоторая постоянная, получается разный порядок сходимости вероятности. В отличие от первого случая, оценка при $s=\varepsilon n$ является очень грубой. 

Доказательство существования, единственности, а также положительности $t_0$  может быть найдено в учебнике Коралова--Синая.

Воспользуемся \textit{методом Крамера} вычисления асимптотики вероятностей. Покажите, что верно 
\begin{equation*}
\PR\Bigl\{\Bigl|\sum_{i=1}^n \xi_i- n m\Bigr|\geq n \varepsilon \Bigr\} \leq (\phi(t_0)\text{e}^{-t_0\varepsilon})^n \idotsint\limits_{\sum_{i=1}^n x_i>\varepsilon n} \,dF_{t_0}(x_1) \dots dF_{t_0}(x_n),
\end{equation*}
где $F_t(x) = \frac{1}{\phi(t)}\int_{-\infty}^{x} \text{e}^{t u}d F(u)$ -- функция распределения (проверьте это).
Для того, чтобы оценить интеграл, рассмотрите случайные величины $\tilde{\xi}_1,\dots,\tilde{\xi}_n$ с распределением $F_{t_0}$, воспользуйтесь центральной предельной теоремой, чтобы показать, что $B_n\to1/2$ при ${n\to\infty}$.
\end{remark}



\begin{problem}
\label{chernov_th}
 Задана последовательность независимых одинаково распределенных случайных величин $\xi_1,\xi_2,\dots$ и $S_n = \xi_1+\dots+\xi_n$.
Показать, что в бернуллиевском случае ($\mathbb{P}\{\xi_1=1\} = p$, $\mathbb{P}\{\xi_1 = 0\}=1-p$) 
\begin{enumerate}
\item  при  $p<x<1$
\begin{equation*}
\lim\frac{1}{n}\ln\mathbb{P}\{S_n\geq nx\} = -H(x),
\end{equation*}
\begin{equation*}
H(x) = x\ln \frac{x}{p}+(1-x)\ln\frac{1-x}{1-p};
\end{equation*}
\item при $x_n = n(x-p)$ и при  $p<x<1$
\begin{equation*}
\mathbb{P}\{S_n\geq np+x_n\}=\exp\biggl\{-nH\biggl(p+\frac{x_n}{n}\biggr)(1+o(1))\biggr\};
\end{equation*}
\item при $x_n = a_n\sqrt{np(1-p)}$ c $a_n\to\infty$, $\frac{a_n}{\sqrt{n}}\to 0$
\begin{equation*}
\mathbb{P}\{S_n\geq np+x_n\}  = \exp\biggl\{-\frac{x_n^2}{2np(1-p)}(1+o(1))\biggr\}.
\end{equation*}

\item Обобщите предыдущие три пункта на случай, когда вместо $\frac{S_n}{n}$ под знаком вероятности будет стоять $\underset{k\geq n}{\sup}\frac{S_k}{k}$.

\end{enumerate}
\end{problem}
\begin{ordre}
Для доказательства первого пункта используйте следующий результат (теорема Чернова). Пусть $S_n = \sum_{i=1}^n\xi_i$, где $\xi_i$, $i=1,\dots,n$ независимые одинаково распределенные простые случайные величины с 
$$
\mathbb{E}\xi_1\leq 0 \quad \text{и} \quad \mathbb{P}\{\xi_1>0\}>0
,
$$
$$
\inf_{\lambda}\phi(\lambda)=\rho,\quad 0<\rho<1, \quad \phi(\lambda) = \mathbb{E}\text{e}^{\lambda\xi_1}.
$$
Тогда
\begin{equation*}
\lim\frac{1}{n}\ln\mathbb{P}\{S_n\geq 0\} \to \ln\rho\quad \text{при~}n\to\infty.
\end{equation*}
\end{ordre}



\begin{problem} %\begin{enumerate} 
%\item 
\label{KL_EF}
а) Сравнить оценки вероятности отклонений выборочного среднего от теоретического среднего для последовательности одинаково распределенных случайных величин $\xi_1,\dots,\xi_n$ с распределением Бернулли с вероятностью успеха $p$ (не предполагая равномерную отделимость $p$ от нуля, в частности, допуская $p = \lambda/n$, где $\lambda$ постоянна), получаемые с помощью неравенства Хёфдинга, Бернштейна, Спокойного, Буске и неравенства больших уклонений из предыдущих задач.
%\item 

б) В некотором городе прошел второй тур выборов. Выбор был между двумя кандидатами $A$ и $B$ (графы <<против всех>> на этих выборах не было). 
Сколько человек надо опросить на выходе с избирательных участков, чтобы исходя из ответов можно было определить долю проголосовавших 
за кандидата $A$ с точностью $5\%$ и с вероятностью не меньшей $0.99$. %Считайте, что исходя из голосования в первом туре, известно, 
%что каждый из кандидатов наберет не меньше $30\%$ голосов избирателей. 
%
Какой способ решения является более точным: с помощью использования центральной предельной теоремы и неравенства Берри-Эссеена или полученный с использованием неравенств концентрации меры, больших уклонений (см. задачи выше в разделе, неравенства Спокойного, Буске, Бернштейна)?
%\end{enumerate}
\end{problem}
\begin{remark}
{\it Неравенство О. Буске.}
Пусть $X_1,\dots,X_n$ -- последовательность независимых случайных величин с распределением $\PR$, принимающих значения из полного сепарабельного метрического пространства $\mathcal{X}$. Пусть $Z = f(X_1,\dots,X_n)$, где $f:\mathcal{X}^n\to \mathbb{R}$ измеримая функция, обозначим $Z_i = f(X_1,\dots,X_{i-1},X_{i+1},\dots,X_n)$. 

Пусть $Z$ удовлетворяет
$$
\sum_{i=1}^k(Z-Z_k)\leq Z
$$
и существуют такие случайные величины $Y_i$, что 
$$
Y_i\leq Z-Z_i\leq 1,\quad Y_i\leq a\,\, \text{п.н.}
$$
для некоторого $a>0$ и $\mathbb{E}_iY_i \geq 0$, где $\mathbb{E}_iX = \mathbb{E}_i[X|X_1,\dots,X_{i-1},X_{i+1},X_n]$.
Также пусть существует $\sigma,$ такое что п.н. $\sigma^2\geq \frac 1n \mathbb{E}_i Y_i^2$.


%Пусть $\mathcal{F}$ -- счетное семейство функций из $\mathcal{X}$ в $\mathbb{R}$. Предположим, что функции $f$ из $\mathcal{F}$ являются $\PR$-измеримыми, интегрируемыми в квадрате и  удовлетворяют $\mathbb{E}[f(X)]=0$. Если $\sup_{f\in\mathcal{F}}\mathrm{ess}\sup f(x)\leq 1$ тогда обозначим
%\begin{equation*}
%Z = \sup_{f\in\mathcal{F}}\sum_{i=1}^n f(X_i),
%\end{equation*}
%если же $\sup_{f\in\mathcal{F}}\|f\|_{\infty}\leq 1$, то
%\begin{equation*}
%Z = \sup_{f\in\mathcal{F}}\biggl|\sum_{i=1}^n f(X_i)\biggr|.
%\end{equation*}
Тогда для всех $x\geq 0 $ верно
\begin{equation*}
\PR[Z\geq \mathbb{E}[Z]+x]\leq \exp\biggl(-vh\biggl(\frac{x}{v}\biggr)\biggr)
\end{equation*}
при    $v = n\sigma^2+ (1+a)\mathbb{E}[Z]$ и $h(x)=(1+x)\ln(1+x)-x$, а также
\begin{equation*}
\PR\biggl[Z\geq \mathbb{E}[Z]+\sqrt{2vx}+\frac{x}{3}\biggr]\leq \exp(-x).
\end{equation*}

\medskip

См. также книгу Stéphane Boucheron, Gábor Lugosi, Pascal Massart <<Concentration Inequalities: A Nonasymptotic Theory of Independence>>, Oxford University Press, 2013.

\medskip

{\it Неравенство В.\,Г. Cпокойного.}  
Пусть $Y_i$ одинаково распределенные случайные величины с распределением $\PR_{u^{*}}$ которое принадлежит экспоненциальному семейству следующего вида
\begin{equation*}
p_{\nu}(y) = p(y)\exp\{y\nu-d(\nu)\},
\end{equation*}
где $d(\nu)$ заданная выпуклая функция на множестве параметров $\Theta\subset \mathbb{R}$, $p(y)$ ~--- неотрицательная функция на множестве значений случайной величины, $d(\nu)$ дважды непрерывно дифференцируема и для всех $u$ верно $d^{\prime\prime}(\nu)>0$.
Тогда для всех $x>0$ верно следующее неравенство
\begin{equation*}
\PR_{v^{*}}(L(\tilde{v},v^{*})>x) = \PR_{v^{*}}\bigl\{n\mathcal{KL}(\tilde{\nu},\nu^{*})> x\bigr\}\leq 2\exp(-x),
\end{equation*}
где $\mathcal{KL}(\nu,\nu^{*})$~--- расстояние Кульбака--Лейблера (см. также задачу \ref{sec:them_modeling} раздела \ref{bayes})
\begin{equation*}
\mathcal{KL}(\nu,\nu^{*}) = \int \ln\frac{d\PR_{\nu}(y)}{d\PR_{\nu^{*}}}d\PR_{\nu}(y),
\end{equation*} 
где $L(\nu,\nu^{*})$~--- $\log$-отношение правдоподобий моделей с параметрами $\nu$ и $\nu^{*}$, которое определяется следующим образом:
\begin{equation*}
L(\nu,\nu^{*}) = L(\nu)-L(\nu^{*}),
\end{equation*}
где $L(\nu)$ ~---  $\log$-правдоподобие модели с параметрами $u$ 
\begin{equation*}
L(\nu) = n^{-1}\sum_{i=1}^n\log f_\nu(Y_i),
\end{equation*}
здесь в случае непрерывного распределения $f_\nu(Y_i)$~--- плотность распределения  случайной величины $Y_i$, в случае дискретного распределения  вероятность получить наблюдаемое $Y_i$; а 
$\tilde{\nu}$ ~--- оценка параметров методом максимального правдоподобия, т.е.
\begin{equation*}
\tilde{\nu} = \mathrm{arg}\max_{\nu\in \Theta}L(\nu).
\end{equation*}
%\begin{equation*}
%\mathcal{K}(\nu,\nu*) = \int
%\end{equation*}

\medskip

\textbf{Pасстояние Кульбака--Лейблер}а $\mathcal{KL}(P \Vert Q)$ характеризует ``вложенность'' распределения $P$ в $Q$.
\imgh{110mm}{voron_kl}{Иллюстрация свойства ``вложенность'' метрики $\mathcal{KL}$.}

\textit{Теорема Берри--Эссеена. (В.В. Сенатов)} 
\label{sec:BerryEssen}
 Пусть $\xi_1, \xi_2\dots$ независимые одинаково распределенные с.в., причем $\mathbb{E}\xi_i = m$
 $\mu^3={\mathbb E}|\xi_i - {\mathbb E}\xi_i|^3<\infty$, $\sigma^2=\mathbb D \xi_i$.
Близость с.в. $\frac{\sum_{i=1}^{n}\xi_i-nm}{\sigma\sqrt{n}}$ к стандартной нормально распределенной с.в. (согласно ц.п.т.) в смысле 
близости их функций распределения определяется неравенством Берри--Эссена 
$$
\sup\limits_x \left| {\mathbb P}\Bigl( \frac{\sum_{i=1}^{n}\xi_i-nm}{\sigma\sqrt{n}}<x \Bigr) - \Phi(x) 
\right| \le \frac{C_0 \mu^3}{\sigma^3 \sqrt{N}} , 
$$
где $0.4<C_0<0.7056$,
$\Phi(x)=\int_{-\infty}^x \frac{e^{-t^2/2}}{\sqrt{2\pi}}\, dt$. 

Неравенство Берри-Эссеена дает неулучшаемый в общем случае результат. 

Теперь пусть $X_1, X_2\dots$ ~--- независимые случайные величины, имеющие одно и то же решетчатое распределение с шагом $h>0$. Очевидно, что распределение их суммы будет решетчатым с тем же шагом $h$  а функция распределения нормированной суммы
\begin{equation*}
\frac{X_1+\dots+X_n- na}{\sigma \sqrt{n}} 
\end{equation*}
будет решетчатым с шагом $h_n = h/(\sigma\sqrt{n})$. Обозначим решетку, на которой сосредоточено распределение нормированной суммы, через $D_n$. Ясно, что на полуинтервале $[-1,1)$ находится не более $2/h_n$ точек из решетки $D_n$. В силу центральной предельной теоремы для эмпирического распределения верно
\begin{equation*} 
F_n(1)-F_n(-1)\to \Phi (1)-\Phi (-1) = 0.6826\dots
\end{equation*}
при $n\to\infty$, поэтому, начиная   с некоторого  $n$, сумма скачков  $F_n$ на полуинтервале $[-1,1)$ будет не меньше $0.5$. Отсюда сразу следует, что при таких $n$ максимальный скачок будет не меньше $0.25h/(\sigma\sqrt{n})$. Так как нормальная функция распределения $\Phi$ непрерывна, а приблизить разрывную функцию $F_n$ непрерывной функцией с точностью, превосходящей половину максимального скачка, невозможно, то 
\begin{equation*}
\sup_x|F_n(x)-\Phi(x)| \geq 0.125 \frac{h}{\sigma \sqrt{n}}.
\end{equation*}
Эти рассуждения показывают, что порядок по $n$ оценки теоремы Берри-Эссеена является правильным.
См. В.В. Сенатов Центральная предельная теорема. Точность аппроксимации и асимптотическое разложение. М. Книжный дом ЛИБРОКОМ, 2009.
\end{remark}


\begin{problem} [В.Г. Спокойный]
\label{sec:spokoiny}
Пусть  $\xi$ -- стандартный нормальный вектор в $\mathbb{R}^p$. Тогда для любого $u>0$ выполнено 
\begin{equation*}
\PR(\|\xi\|^2 >p+u)\leq \exp\bigl\{-(p/2)\psi(u/p)\bigr\},
\end{equation*}
где
\begin{equation*}
\psi(t) = t-\ln(1+t).
\end{equation*}
Пусть $\psi ^{-1}(\cdot)$ обратная функция к $\psi (\cdot)$. 
\begin{enumerate}
\item
Покажите, что для любого $x$ верно 
\begin{equation*}
\PR(\|\xi\|^2>p+\psi^{-1}(2x/p))\leq\exp (-x).
\end{equation*}
И, в частности, при $\kappa = 6.6$
\begin{equation*}
\PR(\|\xi\|^2>p+ \max (\sqrt{\kappa xp}, \kappa x))\leq \exp(-x).
\end{equation*}
Можно ли уменьшить константу $\kappa$?
\item 
Обобщите результаты предыдущего пункта на случай, если компоненты вектора являются независимыми субгауссовскими случайными величинами с параметром $C$, то есть для любого $\lambda>0$, $i=1,\dots,p$ выполнено
\begin{equation*}
\mathbb{E}\exp(\xi_i\lambda)\leq \exp(C\lambda^2/2).
\end{equation*}

\end{enumerate}
\end{problem}

\begin{ordre}
Показать, что 
\begin{equation*}
\ln\mathbb{E}\exp(\mu\|\xi\|^2/2) = -0.5p\ln(1-\mu).
\end{equation*}
Из неравенства Чернова получите
\begin{equation*}
\PR(\|\xi\|^2>p+u)\leq\exp \bigl\{-\mu(p+u)/2-(p/2)\ln(1-\mu)\bigr\}.
\end{equation*}
Минимизируйте правую часть по $\mu$. Затем используйте $x-\ln(1+x)\geq a_0x^2$ при $x\leq 1$ и $x-\ln(1+x)\geq a_0x$ при $x>1$  и $a_0=1-\ln 2\geq 0.3$.
\end{ordre}
\begin{remark}
См. также Spokoiny V. Basics of Modern Parametric Statistics. 2012, http://premolab.ru/sites/default/files/stat.pdf.
\end{remark}

%\begin{problem}


%\end{problem}


%Введем ряд обозначений
%\begin{equation*}
%\PR_{n,c} = \PR\biggl\{\Bigl|\sum_{i=1}^n\xi_i -\sum_{i=1}^m m_i\Bigr|%\geq cn\biggr\},
%\end{equation*}
%\begin{equation*}
%R(\lambda)=\int_{-\infty}^{\infty}\text{e}^{\lambda x}\,d F(x),
%\end{equation*}
%\begin{equation*}
%m(\lambda) = \frac{R^{\prime}(\lambda)}{R(\lambda)}.
%\end{equation*}

%\begin{remark}
%\textbf{TODO}
%\end{remark}
%\begin{problem}[Задача о среднем функции в смысле Леви --- про концентрацию меры на сфере вокруг медианного значения "хорошей" функции]
%\end{problem}
%\subsection{Изопериметрические неравенства Талаграна(?)}
\begin{problem} 

\begin{enumerate}
\label{sec:mirrorDescent}
Пусть $\xi_0,\dots,\xi_k$ ~--- независимые одинаково распределенные случайные величины. Обозначим за $\xi_{[i]}$ совокупность случайных величин $\xi_0,\dots,\xi_{i-1}$
\item
Пусть $\Delta_i = \Delta_i(\xi_{[i]})$ неслучайная измеримая функция от $\xi_{[i]}$, такая, что 
\begin{equation*}
\mathbb{E}\left[\exp\left(\frac{\Delta_i^2}{\sigma^2}\right) \vert \: \xi_{[i-1]}\right]\leq \mathrm{\exp(1)}.
\end{equation*}
Покажите, что для любого $k\geq 0$ и $\Omega>0$ верно 
\begin{equation*}
\PR\left(\sum_{i=0}^k c_i\Delta_i^2\geq (1+\Omega) \sum_{i=0}^kc_i\sigma^2\right)\leq \exp(-\Omega),
\end{equation*}
где $c_0,\dots,c_k$ ~--- последовательность положительных коэффициентов.
\item 
Пусть $\Gamma_k$ и $\eta_k$ неслучайные измеримые функции от $\xi_{[k]}$ такие что 
\begin{itemize}
\item $\mathbb{E}[\Gamma_i|\xi_{[i-1]}]=0,$
\item $|\Gamma_i|\leq c_i\eta_i$, где $c_i$ положительная неслучайная константа,
\item $\mathbb{E}\left[\exp\left(\frac{\eta_i^2}{\sigma^2}\right)|\xi_{[i-1]}\right]\leq \mathrm{\exp(1)}$.
\end{itemize}
Покажите, что для всех $k\geq 0$ и $\Omega\geq 0$ верно
\begin{equation*}
\PR\left(\sum_{i=0}^k \Gamma_i\geq \sqrt{3\Omega}\sigma\sqrt{\sum_{i=0}^kc_i^2}\right)\leq \exp(-\Omega).
\end{equation*}
\end{enumerate}
\begin{remark} 
При доказательстве пункта а) необходимо использовать выпуклость экспоненты и линейность математического ожидания, затем неравенство Маркова.
 Пункт б) является следствием леммы 2 из статьи  Lan G., Nemirovski A. and Shapiro A. Validation analysis of mirror descent stochastic approximation method // Mathem. Programming Serie A. 2012. V.~134(2). P.~425--458.
\end{remark}

\end{problem}

\begin{problem}[Неравенства Эфрона-Стайна и МакДиармида]
\begin{enumerate}
\item 
Пусть $X_i$, $i=1,\dots,n$ произвольные независимые (не обязательно одинаково распределенные)  случайные величины, принимающие значения из $\mathcal{X}$ и пусть  $g: \mathcal{X}^n\to \mathbb{R}$ измеримая функция $n$ переменных. Покажите, что для случайной величины $Z = g(X_1,\dots,X_n)$ верно 
\begin{equation*}
\Var(Z) \leq \sum_{i=1}^n \mathbb{E}\big[ (Z-\mathbb{E}_iZ)^2\bigl],
\end{equation*}
где $\mathbb{E}_iZ = \mathbb{E}[Z|X_1,\dots,X_{i-1},X_{i+1},\dots,X_n]$.
\item Неравенство Эфрона-Стайна. Пусть $X'_1,\dots,X'_n$ ~--- независимые копии $X_1,\dots,X_n$ и 
\begin{equation*}
Z'_i = g(X_1,\dots, X'_i,\dots,X_n).
\end{equation*}
Покажите, что верно неравенство 
\begin{equation*}
\Var(Z)\leq \frac{1}{2}\sum_{i=1}^{n}\mathbb{E}[(Z-Z'_i)^2].
\end{equation*}
\item Неравенство Эфрона-Стайна в случае функций с ограниченными  разностями.
Функция $g: \mathcal{X}^n\to \mathbb{R}$ является функцией с ограниченными разностями, если для некоторых $c_1,\dots,c_n$ выполнено 
\begin{equation*}
\begin{split}
\sup_{x_1,\dots,x_n;\, x'_i\in\mathcal{X}} |g(x_1,\dots,x_n)-g(x_1,\dots,x_{i-1},x'_i,x_{i+1},\dots,x_n)|\leq c_i,\\
\quad 1\leq i\leq n.
\end{split}
\end{equation*}
Выпишите неравенство Эфрона-Стайна для случая функций с ограниченными разностями.
\item Докажите неравенство МакДиармида для функции $g$ с ограниченными разностями  (см. предыдущий пункт), а именно, что для любого $\varepsilon>0$ верно
\begin{equation*}
\mathbb{P}\Big\{\bigl| g(X_1,\dots,X_n)-\mathbb{E}g(X_1,\dots,X_n)\bigr|>\varepsilon\Bigr\}\leq 2\exp\bigg\{-\frac{2\varepsilon^2}{nC^2}\biggr\},
\end{equation*}
 где  $C^2 = \sum_{i=1}^n c_i^2$.
%\item {\textit{\textbf{Неравенство Эфрона-Стайна для отклонения эмпирической функции распределения}}}

\end{enumerate}

\end{problem}

\begin{problem}[Лемма Джонсона-Линденштаусса]
Лемма гласит, что если задан произвольный набор из $n$ точек в многомерном ($D$-мерном) евклидовом пространстве, то существует линейное вложение этих точек в $d$-мерное евклидово пространство, такое что все попарные расстояния сохраняются с точностью до множителя $1\pm\varepsilon$, если $d$ пропорционально $(\log n)/\varepsilon^2$. 

Пусть $A$~--- конечное подмножество $\mathbb{R}^D$ размерности $n$. И для  некоторого $v\geq 0$, случайные величины  $X_{i,j}$, $i=1,\dots,n$, $j=1,\dots,D$~ независимы, одинаково распределены и являются субгауссовскими с параметром $v$ (см. задачу \ref{sec:spokoiny}), причем $\mathbb{E}X_{i,j}=0$, $\mathbb{E}X^2_{i,j}=1$.
При заданном $\varepsilon\in(0,1)$ отображение $f:\mathbb{R}^D\to R^{d}$ называется $\epsilon$-изометрией на $A$ если для каждой пары $a,a'\in A$ выполняется 
$$
(1-\varepsilon)\|a-a'\|^2\leq \|f(a)-f(a')\|^2\leq (1+\varepsilon)\|a-a'\|^2.
$$

Пусть $d\geq 32v\varepsilon^{-2}\log(n/\sqrt{\delta})$, где $\delta\in(0,1)$. Покажите, что тогда с вероятностью не меньшей $1-\delta$, отображение $W: \mathbb{R}^D\to \mathbb{R}^d$, где $W_i(\alpha) =  \frac{1}{\sqrt{d}}\sum_{j=1}^D \alpha_j X_{i,j}$ для всех $\alpha\in \mathbb{R}^D$, $i\in\{1,\dots,d\}$, является $\varepsilon$-изометрией на $A$.
\end{problem}
\begin{remark}
Замечательным фактом является то, что результат не зависит от размерности $D$, которая может быть даже бесконечной!
\end{remark}
\begin{ordre}
Идея доказательства заключается в использовании специальной случайной линейной вектор-функции $W(\alpha)$ и проверке ее на $\varepsilon$-изометрию.
Основные шаги доказательства
\begin{enumerate}
\item Проверьте, что $\mathbb{E}[\|W(\alpha)\|^2]= \|\alpha\|^2,$ где $\alpha\in \mathbb{R}^{D}$.
\item Убедитесь, что доказательство того, что $W$ является $\varepsilon$-изометрией эквивалентно тому, что c вероятностью не меньшей $1-\delta$
$$
\sup_{\alpha\in T}\bigl|\|W(\alpha)\|^2-1|\leq\varepsilon,
$$
где  $T$~---подмножество единичной сферы $S$ в $\mathbb{R}^D$ следующего вида
$$
T = \biggl\{ \frac{a-a'}{\|a-a'\|}:a,a'\in A, a\not = a'\biggr\}.
$$
\item Для того, чтобы это показать, докажите, что $\sqrt{d}W_i(\alpha)$~--- субгауссовская случайная величина (см. задачу \ref{sec:spokoiny}) с параметром $v$. 
\item Воспользуйтесь следующим фактом (см. теорему 2.1 из Boucheron S., Lugosi G., Massart P. Concentration Inequalities: A Nonasymptotic Theory of Independence, Oxford University Press, 2013): если случайная величина $X$ суб-гауссовская с параметром $4C$, тогда $\mathbb{E}[X^{2q}]\leq q!C^q$ для $q>1$.
Получите, что при $q\geq 2$
$$
\mathbb{E}[W_i(\alpha)^{2q}]\leq \frac{q!}{2}(4v)^q.
$$
\item С помощью неравенства Бернштейна (задача \ref{bernstain})  получите 
$$
\mathbf{P}\left\{\sup_{\alpha\in T}|\|W(\alpha)\|^2-1|\geq 4\sqrt{\frac{v\log(n/\sqrt{\delta})}{d}}+\frac{8v\log(n/\sqrt{\delta})}{d} \right\}\leq\delta.
$$
\item Подставьте в правую часть неравенства в выражении вероятности условие на $d$ из формулировки леммы и получите утверждение леммы.

\end{enumerate}
\end{ordre}
\begin{problem}[Теорема Дворецкого]
См. также формулировку задачи \ref{sec:dvor} раздела \ref{geom}. 
Для каждого натурального $k$ и любого $\varepsilon>0$ найдется такое $n$, что всякое $n$-мерное нормированное пространство $X$ имеет $k$-мерное подпространство, расстояние от которого до $l_2^k$ по метрике Банаха-Мазура не превосходит $1+\varepsilon$, то есть можно найти векторы $x_1,\dots,x_k\in X$ такие, что 
$$
\left(\sum_{i=1}{k}|a_i|^2\right)^{1/2}\leq \|\sum_{i=1}^ka_ix_i\|_2\leq (1+\varepsilon)\left(\sum_{i=1}^k|a_i|^2\right)^{1/2} 
$$
для любой последовательности скаляров $a_1,\dots,a_k$.
\end{problem}

\begin{ordre}
Подход состоит в том, чтобы выбрать $k$-мерное подпространство $X$ случайным образом. Преред этим надо выбрать подходящую вероятностную меру. Что может быть сделано с помощью теоремы Фрица Джона. Последняя уверждает, что если существует базис $x_1,\dots,x_n$ пространства $X$, который не слишком далек от ортонормированного в смысле, что 
$$
\left(\sum_{i=1}^n|a_i|^2\right)^{1/2}\leq \left\|\sum_{i=1}^n a_ix_i\right\|_2\leq \sqrt{n}\left(\sum_{i=1}^n|a_i|^2\right)^{1/2}
$$
для всякой последовательности скаляров $a_1,\dots,a_n$. Тогда берется естественная мера грассманиана $G_{n,k}$ относительно этого базиса. Также необходимо следствие неравенства Леви: пусть $f:\,S^n\to\mathbb{R}$~--- функция со средним значением $M$ и пусть $A\in S^n$~--- множество всех точек $x$, для которых $f(x)\leq M$, тогда вероятность того, что случайно выбранная точка $S^n$ удалена от $A$ более, чем на $\varepsilon$ не превосходит $\sqrt{\pi/2}\exp(-\varepsilon^2n/2)$. Заменив $f$ на $-f$ найдем также, что почти каждая точка $y$ близка $x$ с $f(x)\geq M$. Положим теперь $f(a_1,\dots,a_m) = \|\sum_{i=1}^n a_ix_i\|_2$. Поскольку $f$ в достаточной мере непрерывна и почти каждая точка $y$ близка некоторой точке $x\in A$, заключаем, что $f(y)$ не намного больше $M$. Точно так же $f(y)$ не намного меньше $M$ для большинства точек $y$. 

См. В.Д. Мильман, Новое доказательство теоремы А. Дворецкого о сечениях выпуклых тел, Функц. анализ, Т.5, № 4, 1971, C.28--37.
\end{ordre}

\begin{remark}
Еще одна формулировка теоремы: каждое $n$-мерное симметричное выпуклое тело имеет $k$-мерное центральное сечение, которое содержит $k$-мерный эллипсоид $B$ и содержится в $(1+\epsilon)B$, то есть само является почти эллипсоидальным.

\medskip

\textit{Соmpressed sensing}. 
%\begin{ordre}
<<Сжатие измерений>> понимается как метод экономного восстановления неизвестной функции, заданной на конечном множестве мощности $m$, то есть вектора $u\in\mathbb{R}^m$ по информации, полученной измерениями скалярных произведений $(u,\phi_j),$ $\phi_j\in \mathbb{R}^m$,  $j = 1,\dots,n,$ причем $n\ll m$. Пусть $\Phi$~--- матрица со строками $\phi_j\in\mathbb{R}^m$, $j=1,\dots,n$.
Предполагается, что о разреженности вектора известно, что $\|u\|_{0} = |\{i:\,u_i\leq 0 \}|\leq t.$
Целью является 
\begin{enumerate}
\item построение алгоритма аппроксимации функции $u$ по информации $y = ((u,\phi_1),\dots,(u,\phi_n))\in\mathbb{R}^n$, то есть 
\begin{equation*}\label{lo-pr}
\min\|u\|_{0}\text{ при условии, что } \Phi v = y;
\end{equation*}
\item построение измеряющего множества векторов $\phi_j\in\mathbb{R}^m$, $j=1,\dots,n$, то есть описание матриц $\Phi$.
\end{enumerate}

Первую задачу было предложено (D. L. Donoho, M. Elad, V. N. Temlyakov, “Stable recovery of sparse overcomplete repre- sentations in the presence of noise”, IEEE Trans. Inform. Theory, 52:1 (2006), 6–18.) решать с помощью релаксации к выпуклой задаче 
\begin{equation*}\label{l1-pr}
\min\|u\|_{1}\text{ при условии, что } \Phi v = y.
\end{equation*}

В 1977 году Кашиным было доказано, что для любой пары  $(m,n)$, где $m\geq n$, существует такое подпространство $V$ размерности большей или равной $m-n$, такое, что для любого $x\in V$:
$$
\|x\|_{2}\leq C \left(\frac{1+\log(m/n)}{n}\right)^{1/2} \|x\|_1 
$$
(см. Кашин Б.С., Изв. АН СССР, серия матем., Т.41, 1977, 334–351). 
См. также Кашин Б.С., Темляков В.Н. Замечание о задаче сжатого измерения // Математические заметки. 2007. Т. 82, № 6, C. 829–837, E.J. Candes, T. Tao, “Decoding by linear programming”, IEEE Trans. Inform. Theory, Vol.51, № 12, 2005.

\end{remark}




%\end{ordre}




\section{Метод Монте-Карло}
\label{MK}

\begin{problem}
На плоскости дано ограниченное измеримое по Лебегу множество $S$. Требуется найти площадь (меру Лебега) этого множества с заданной точностью 
$\varepsilon$. 

Поскольку по условию множество ограничено, то вокруг него можно описать квадрат со стороной $a$. Выберем декартову систему координат 
в одной из вершин квадрата с осями, параллельными сторонам квадрата. Рассмотрим $n$  независимых с.в. $\{ X_k\}_{k=1}^{n}$,  имеющих 
одинаковое равномерное распределение в этом квадрате, т.е. $X_k\in R([0,a]^2)$. Введем с.в. 
$$
Y_k=I(X_k\in S)=\begin{cases}
1,\quad X_k\in S\\
0, \quad X_k\notin S
\end{cases} 
$$
Тогда $\{ Y_k\}_{k=1}^{n}$ --- независимые одинаково распределенные с.в.. Ясно, что $Y_k\in\Be(p(S))$. Следовательно, по у.з.б.ч. 
$$
\frac{Y_1+\ldots+Y_n}{n} \xrightarrow{\text{ п.н. }} {\mathbb E}(Y_1)=p(S)=\frac{\mu(S)}{a^2} \quad \text{ при  } n\to\infty . 
$$
Оцените сверху следующую вероятность
$$
{\mathbb P}\Bigl( \Bigl| \frac{Y_1+\ldots+Y_n}{n}-\frac{\mu(S)}{a^2}\Bigr|>\delta \Bigr) . 
$$
\end{problem}

\begin{ordre}
См. раздел 5, а также часть 2.
\end{ordre}


\begin{problem}[Вычисление значения интеграла]
\begin{enumerate}
\item Требуется вычислить с заданной точностью $\varepsilon $ и с заданной доверительной вероятностью $\gamma $ абсолютно сходящийся интеграл 
\[
J=\int _{\left[0,\; 1\right]^{m} }f\left({x}\right)d{x}.
\] Считайте, что $\forall \; \; {x}\in \left[0,\; 1\right]^{m} \to \left|f\left({x}\right)\right|\le 1$.

\begin{remark}
Введем случайный $m$-вектор ${X}\in R\left(\left[0,\; 1\right]^{m} \right)$ и с.в. $\xi =f\left({X}\right)$. Тогда $\Exp\xi =\int _{\left[0,\; 1\right]^{m} }f\left({x}\right)d{x} =J$. Поэтому получаем оценку интеграла $\bar{J}_{n} =\frac{1}{n} \sum _{k=1}^{n}f\left({x}^{k} \right) $, где ${x}^{k} $, $k=1,...,n$ -- повторная выборка значений случайного вектора ${X}$ (т.е. все ${x}^{k} $, $k=1,...,n$ -- независимы и одинаково распределены: также как и вектор $\vec{X}$). В задаче требуется оценить сверху число $n$ ($n\gg m$), начиная с которого $\mathbb{P}\left(\left|J-\bar{J}_{n} \right|\le \varepsilon \right)\ge \gamma $.
\end{remark}



\item Решите задачу из п. а) при дополнительном предположении липшецевости функции $f\left({x}\right)$, разбив единичный куб на $n=N^{m} $ одинаковых кубиков со стороной ${1\mathord{\left/ {\vphantom {1 N}} \right. \kern-\nulldelimiterspace} N} $, и используя оценку $\bar{J}_{n} =\frac{1}{n} \sum _{k=1}^{n}f\left({x}^{k} \right) $, где ${x}^{k} $ -- имеет равномерное распределение в \textit{k}-м кубике.

\item (метод включения особенности в плотность). Решите задачу п. а) не предполагая, что $f\left({x}\right)$ -- ограниченная функция на единичном кубе. Предложите способы уменьшения дисперсии полученной оценки интеграла. Как можно использовать информацию об особенностях функции $f\left({x}\right)$?

\end{enumerate}

\end{problem}
\begin{ordre}
См. книгу Соболь И.М. Численный метод Монте-Карло, М.: Наука, 1973, а также \cite{lagutin}.
\end{ordre}

\begin{problem}Стандартный способ моделирования с.в. -- {\it метод обратной функции}. Покажите, что если с.в. $\eta $ равномерно распределена на отрезке $\left[0,1\right]$, то с.в. $\xi =F^{-1} \left(\eta \right)$ имеет функцию распределения $F\left(x\right)$. Предполагается, что $F\left(x\right)$ непрерывна и строго монотонна. Как выглядит формула для моделирования с.в. из показательного распределения с функцией распределения $F\left(x\right)=\left(1-e^{-\lambda x} \right)I\left(x\geq 0\right)$?
\end{problem}

\begin{problem}
Пусть с.в. $\eta _{1} $, $\eta _{2} $ имеют равномерное распределение на отрезке $\left[0,1\right]$. Докажите, что с.в. $X$ и $Y$: $X=\sqrt{-2\ln \eta _{1} } \cos \left(2\pi \eta _{2} \right)$, $Y=\sqrt{-2\ln \eta _{1} } \sin \left(2\pi \eta _{2} \right)$ -- независимые и одинаково распределенные: стандартно нормально ${\rm {\mathcal N}}\left(0,1\right)$.

\begin{ordre}
Покажите, что
\[f_{XY} (x,y)=\frac{1}{\sqrt{2\pi } } e^{-\frac{x^{2} }{2} } \frac{1}{\sqrt{2\pi } } e^{-\frac{y^{2} }{2} } =\frac{1}{2\pi } e^{-\frac{x^{2} +y^{2} }{2} } .\] 
Перейдите к полярным координатам, не забыв о якобиане замены переменных.
\end{ordre}

\end{problem}


\begin{problem}

Если $X$ -- с.в., имеющая стандартное нормальное распределение, то $X^{-2} $ имеет устойчивую плотность (см. замечание к задаче \ref{bluzd_ust}, раздела 5):
\[\frac{1}{\sqrt{2\pi } } e^{-\frac{1}{2x} } x^{-\frac{3}{2} }, \quad x>0.\] 
Используя это, покажите, что если $X$ и $Y$ -- независимые нормально распределенные с.в. с нулевым математическим ожиданием и дисперсиями $\sigma _{1}^{2} $ и $\sigma _{2}^{2} $, то величина $Z=\frac{XY}{\sqrt{X^{2} +Y^{2} } } $ нормально распределена с дисперсией $\sigma _{3}^{2} $, такой, что $\frac{1}{\sigma _{3}^{2} } =\frac{1}{\sigma _{1}^{2} } +\frac{1}{\sigma _{2}^{2} } $.

\end{problem}


\begin{problem}[Теорема Бернштейна \cite{28}, \cite{21} т.2] 
\begin{enumerate}

\item С помощью неравенства Чебышёва установите следующий результат из анализа: 

\[
\forall \; \; f\in C\left[0,1\right]\to \left\| f_{n} -f\right\| _{C\left[0,1\right]} \xrightarrow[{n\to \infty }]{} 0,
\] 

\[
f_{n} \left(x\right)=\sum_{k=0}^{n}f\left(\frac{k}{n} \right) C_{n}^{k} x^{k} \left(1-x\right)^{n-k}. 
\]

\item Исходя из предыдущей задачи и п. а) предложите способ генерирования распределения с.в. $\xi $, имеющей плотность $f_{\xi } \left(x\right)$ с финитным носителем, для определенности, пусть носителем будет отрезок $\left[0,1\right]$.

\end{enumerate}

\end{problem}

\begin{problem}[Метод фон Неймана] 

Пусть с.в. $\xi $ распределена на отрезке $\left[a,b\right]$, причем ее плотность распределения ограничена: $\mathop{\max }\limits_{x\in \left[a,b\right]} f_{\xi } (x) < C$. Пусть с.в. $\eta _{1} $, $\eta _{2} $, \dots  -- независимы и равномерно распределены на $\left[0,1\right]$, $X_{i} =a+\left(b-a\right)\eta _{2i-1} $, $Y_{i} =C\eta _{2i} $, $i=1,2,...$, т.е. пары $\left(X_{i} ,Y_{i} \right)$ независимы и равномерно распределены в прямоугольнике $\left[a,b\right]\times \left[0,C\right]$. Обозначим через $\nu $ номер первой точки с координатами $\left(X_{i} ,Y_{i} \right)$, попавшей под график плотности $f_{\xi } (x)$, т.е. $\nu =\min \left\{i:\: Y_{i} \le f_{\xi } (X_{i} )\right\}$. Положим $X_{\nu } =\sum _{n=1}^{\infty }X_{n} I\left(\nu =n\right) $.

\begin{enumerate}
\item Покажите, что с.в. $X_{\nu } $ распределена так же как $\xi $.

\item Сколько в среднем точек $\left(X_{i} ,Y_{i} \right)$ потребуется «вбросить» в прямоугольник $\left[a,b\right]\times \left[0,C\right]$ для получения одного значения $\xi $?

\item Предложите модификацию рассмотренного метода для генерации дискретной случайной величины, принимающей значения $\lbrace 1, 2, ... , k \rbrace$ с одинаковой вероятностью, имея в распоряжении монету (генератор бинарной случайной величины).   
\end{enumerate}
\end{problem}

\begin{problem}
Как с помощью с.в. $\xi $, равномерно распределенной на отрезке $\left[0,1\right]$ ($\xi \in R\left[0,1\right]$), и симметричной монетки построить с.в. $X$, имеющую плотность распределения $f_{X} (x)=\frac{1}{4} \left(\frac{1}{\sqrt{x} } +\frac{1}{\sqrt{1-x} } \right)$, $x\in \left[0,1\right]$?
\end{problem}

\begin{problem}

Пусть $\xi $ распределена на $\left[0,1\right]$ с плотностью $f_{\xi } (x)$, представимой в виде степенного ряда $\sum _{k=0}^{\infty }a_{k} x^{k}  $ с $a_{k} \ge 0$. Положим $p_{k} ={a_{k} \mathord{\left/ {\vphantom {a_{k}  (k+1)}} \right. \kern-\nulldelimiterspace} (k+1)} $. Тогда $f_{\xi } (x)=\sum _{k=0}^{\infty }p_{k} \cdot (k+1)x^{k}  $. Примените \textit{метод суперпозиции} для моделирования с.в. $\xi $.

\begin{ordre}
\textit{Метод суперпозиции}:

\begin{enumerate}
\item Разыгрывается значение дискретной с.в., принимающей значения $k=0,1,2,...$ с вероятностями $p_{k} $;

\item Моделируется с.в. с функцией распределения $F_{k} (x) = x^{k+1}$, $x\in [0,1]$ (например, методом обратной функции).
\end{enumerate}

\end{ordre}

\end{problem}


\begin{problem}
Покажите, что в случае общего положения невозможно ``приготовить'' распределение дискретной с.в., принимающей $n$ значений менее чем за $O(n)$ операций. Однако единожды приготовив его, можно далее генерировать распределение этой с.в. за $O(\log n)$ операций. Более того, если потребуется перегенерировать распределение немного отличной с.в. (например, в которой поменялось несколько вероятностей исходов и, как следствие, нормировочный множитель всего распределения), то это также можно сделать за $O(\log n)$. Покажите, что если случайная величина принимает, например, разные значения, но с одинаковыми вероятностями, то ``приготовить распределение''  такой с.в. можно и за $O(\log n)$ операций. 
\end{problem}

\begin{problem}[Алгоритм Кнута–Яо]
С помощью бросаний симметричной монетки требуется сгенерировать распределение заданной дискретной с.в., принимающей конечное число значений. Обобщите описанную ниже схему на общий случай. Предположим, что нам нужно сгенерировать распределение с.в., принимающей три значения 1, 2, 3 с равными вероятностями 1/3. Действуем таким образом. Два раза кидаем монетку: если выпало 00, то считаем, что выпало значение 1, если 01, то 2, если 11, то 3. Если 10, то еще два раза кидаем монетку и повторяем рассуждения. Покажите, что можно сгенерировать распределение дискретной с.в., принимающей, вообще говоря, с разными вероятностями $n$ значений в среднем с помощью не более чем $\log_2 (n - 1) + 2$ подбрасываний симметричной монетки.
\end{problem}
\begin{remark}
См. Ермаков С.М. Метод Монте-Карло в вычислительной математике, М.: Бином, 2009.
\end{remark}

\begin{problem}[Метод Уокера]
Пусть случайная величина $\xi$ принимает значения $1, \ldots, n$ с вероятностями $p_1, \ldots, p_n$. Докажите, что генерация $\xi$ может быть осуществлена при помощи смеси дискретных распределений с двумя исходами и случайной величины $\xi'$, принимающей значения $1, \ldots, n$ с одинаковыми вероятностями. 
\end{problem}

\begin{ordre}
\begin{enumerate}
\item Покажите, что $\exists i, j \neq i: \;  p_i \leq 1/n, \; p_i + p_j > 1/n;$
\item Зафиксируем пару $i, j$ с указанным свойством и определим случайную величину $\eta^{(1)}$ с двумя исходами:
\[
\mathbb{P}(\eta^{(1)}=i)=q_i^{(1)} = n p_i, \; \mathbb{P}(\eta^{(1)}=j)=q_j^{(1)} = 1 - n p_i, \; q_k^{(1)} = 0.   
\]
Покажите, что $\xi$ может быть представлена в виде смеси $\eta^{1}$ и случайной величины $\zeta^{(1)}$, принимающей значения из множества $\{1, \ldots, n\} \backslash i$. 
\item Повторите рассуждения пункта б) для с.в. $\zeta^{(1)}$ и таким образом получите%, выразите $p_1, \ldots, p_n$ через  $ \{q_i^{(j)}\}$: 
\[
p_l = \frac{1}{n} \sum \limits_{j=1}^n q_l^{(j)}.
\]
\end{enumerate}

См. также  Ермаков С.М. Метод Монте-Карло в вычислительной математике, М.: Бином, 2009.
\end{ordre}

\begin{problem}[Теорема Пайка] \label{paika} 
Пусть ${x}_{k} ,\; k=1,...,n$ -- независимые равномерно распределенные на отрезке $\left[0,1\right]$ с.в. Упорядочим эти с.в., введя обозначения
\[\mathop{\min }\limits_{k=1,...,n} {x}_{\left(k\right)} ={x}_{\left(1\right)} \le ...\le {x}_{\left(n\right)} =\mathop{\max }\limits_{k=1,...,n} {x}_{\left(k\right)} .\] 
Пусть ${\rm e}_{k} ,\; k=1,...,n+1$ -- независимые показательно распределенные с.в. 
\[\mathbb{P}\left({\rm e}_{k} >t\right)=e^{-t} ,\; t\ge 0.\] 
Покажите, что
\[\left({x}_{\left(1\right)} {\rm ,}\; {x}_{\left(2\right)} {\rm ,...,}\; {x}_{\left(n\right)} \right) \mathop{=}\limits^{d} \left(\frac{{\rm e}_{1} }{\sum _{k=1}^{n+1}{\rm e}_{k}  } {\rm ,}\; \frac{{\rm e}_{1} +{\rm e}_{2} }{\sum _{k=1}^{n+1}{\rm e}_{k}  } {\rm ...,}\; \frac{\sum _{k=1}^{n}{\rm e}_{k}  }{\sum _{k=1}^{n+1}{\rm e}_{k}  } \right) \] 
\end{problem}

\begin{remark}
См. Лагутин М.Б. Наглядная математическая статистика. - М.: Бином, 2009 и Кендалл М., Моран П. Геометрические вероятности. – М.: Наука, 1972.
\end{remark}

\begin{problem}[Генерация равномерного распределения]

а) Пусть ${X}$ ---  $n$-мерный вектор с независимыми одинаково распределенными компонентами с распределением $N(0,1)$. Покажите, что ${\xi} = \frac{{X}}{\|{X}\|_2}$ имеет  равномерное распределение на единичной сфере в $\mathbb{R}^n$. 

б) Пусть ${X}$ вектор с независимыми одинаково распределенными компонентами с распределением Лапласа (т.е. с плотностью $p(x)=\frac{1}{2}\exp(-|x|)$). Какое распределение имеет вектор  ${\xi} = \frac{{X}}{\|{X}\|_1}$? 
\end{problem}




\begin{problem}\Star [Сдвиг Бернулли]
Рассмотрим динамическую систему (ДС): $x\to\{2x\}$ ($\{\cdot \}$ -- дробная часть числа), преобразования отрезка $X = [0,1]$ в себя. Покажите, что для почти всех (по равномерной мере Лебега на отрезке $X$) точек старта, полученная c помощью ДС последовательность точек, будет ``квазислучайной'' (схожей с последовательностью независимых одинаково распределенных на отрезке $X$ с.в.), то есть для неё, например, справедливы з.б.ч. и ц.п.т.
\begin{ordre}
См. Niederreiter H. Random number generation and Quasi-Monte-Carlo methods, 1992.
\end{ordre}
\begin{remark}
В контексте этой задачи рекомендуется ознакомиться также с понятиями непредсказуемой последовательности, типичной последовательности, случайной (сложной) по Колмогорову последовательности, например, по книге Верещагин Н.К., Успенский В.А., Шень А. Колмогоровская сложность и алгоритмическая случайность. -- М.: МЦНМО, 2013. Задача отражает то обстоятельство, что случайность ``переносится'' из начальных данных (даже простым в смысле Колмогорова алгоритмом). Рассматриваемая динамическая система последовательно считывает числа после запятой в двоичной записи начальной точки. Почти все числа из отрезка $[0,1]$ сложны (несжимаемы) в смысле Колмогорова, правда для подавлющего большинства конкретных чисел это утверждение не может быть доказано. Студенты ФУПМ имеют возможность познакомиться с Колмогоровской сложностью и алгоритмичексими вопросами теории вероятностей по курсу (книги) Вьюгин В.В. Колмогоровская сложность и алгоритмическая случайность, М.: МФТИ, 2012. В частности, рекомендуется ознакомиться с сложностным доказательством закона повторного логарифма и эргодической теоремы Биркгофа(-Хинчина), а также с  вопросами эффективности эргодичексой теоремы. 

Другая причина появления случайности -- это поведение динамических систем (например, рассмотренной) в условиях небольших внешних возмущений. Яркое обыгрывание этого направления имеется в конце книги Опойцев В.И. Нелинейная системостатика, М.: Наука, 1986 и в цикле недавних работ В.А. Малышева и А.А. Лыкова. Вообще между динамическими системами и случайными (марковскими) процессами имеется глубокая взаимообогащающая связь. См., например, конструкцию Улама в книге Бланк М.Л. Устойчивость и локализация в хаотической динамике, М.: МЦНМО, 2001 и Синай Я.Г. Как математики изучают хаос // Математическое просвещение. Вып. 5. 2001. С.32-46. Интересные взгляды на эту науку также имеются в книгах Вентцель А.Д., Фрейдлин М.И. Флуктуации в динамических системах под воздействием малых случайных возмущений. -- М.: Наука, 1979, \cite{333}, \cite{101}.

Ну а самое главное, что в приложениях нам нужна не ``первозданная случайность''. Нам нужно лишь выполнение для полученной квазислучайной последовательности неких тестов типа з.б.ч., ц.п.т. с такими же оценками скорости сходимости (или не сильно худшими). Однако оказывается, что псевдослучайная последовательность может даже увеличить скорость сходимости, по сравнению с первозданно случайным случаем. Обнаружено это было более 40 лет назад (сюда можно отнести метод выбора узлов Холтона--Соболя, обобщающий идею задачи 2 п. б) настоящего раздела), но по-прежнему активно используется на практике (см. Соболь И.М. Численные методы Монте-Карло. -- М.: Наука, 1973), и позволяет вместо $~1/\sqrt{n}$ получать точность $~n^{\varepsilon - 1}$ cо сколь угодно малым $\varepsilon > 0$.


В заключение хочется отметить, что в последние десятилетия очень бурно развивается область Theoretical Computer Sсience связанная с изучением генераторов псевдослучайных чисел (в том числе в связи с проблемой $P \ne NP$), см.  
Разборов А.А. Theoretical Computer Sсience: взгляд математика, 2013

\url{people.cs.uchicago.edu/~razborov/files/computerra.pdf}

\end{remark}
\end{problem}

\begin{comment}
\begin{problem}[MCMC]
\label{mcmc}
Для оценки статистик распределения с неизвестной нормировочной константой (см. задачу \ref{varinf} из раздела \ref{CS}) 
\[
p(X) = \frac{1}{S_p} \tilde{p}(X)
\]   
методы Монте Карло требуют возможность генерации выборки из этого распределения
\[
X_1,\ldots,X_N \sim p(X).
\]
Впоследствии, выборка может быть использована для оценки статистик
\[
\Exp_p T(X) \approx \frac{1}{N} \sum_{i = 1}^N T(X_i).
\]
Принцип работы метода MCMC  состоит в использовании некоторой марковской цепи с априорным распределением $p_0(X)$ и вероятностями перехода в момент времени $n:$ $q_n(X_{n+1}|X_n)$. Генерация выборки происходит таким образом:
\[
X_1 \sim p_0(X),  X_2 \sim q_1(X|X_1), \ldots, X_N \sim q_{N-1}(X|X_{N-1}). 
\]
Заметим, что при таком подходе генерируемая выборка не является набором независимых случайных величин, который может быть получен за счет прореживания $X_1,\ldots,X_N $, взяв каждый $m$-й $X$.

Рассмотрим вопрос выбора однородной вероятности перехода $q_n(X_{n+1}|X_n) = q(X_{n+1}|X_n)$, требуя принадлежность генерируемой выборки интересующему нас распределению $p(T)$. Необходимым требованием в таком случае является инвариантность $p(T)$ относительно переходов марковской цепи, т.е.
\[
p(X)=\int q(X|Y) p(Y)dY \Leftarrow q(X|Y) p(Y) = q(Y|X) p(X).
\]
Марковская цепь может иметь более одного инвариантного распределения. 
Пусть $\pi(X)$ –- ее инвариантное распределение. Тогда марковская цепь называется \textit{эргодичной}, если
\[
\forall p_0(X): \; p_n(X) = \int q(X|Y) p_{n-1}(Y)dY \to \pi(X). 
\]
Очевидно, что эргодичная марковская цепь имеет только одно инвариантное распределение. Докажите, что достаточным условием эргодичности однородной марковской цепи является  свойство:
\[
\forall X, \forall Y: \pi(Y) > 0 \Rightarrow q(Y|X) > 0. 
\]

\end{problem}
\end{comment}

\begin{problem}[PageRank] 
Ориентированный граф $G=\left\langle {V,E} \right\rangle$  сети Интернет представляется в виде набора web-страниц $V$ и ссылок между ними: запись $\left( {i,j} \right)\in E$ означает, что на $i$-й странице имеется ссылка на $j$-ю страницу. 

\begin{enumerate}

\item По web-графу случайно блуждает пользователь. За один такт 
времени пользователь, находящийся на web-странице с номером $i$, с вероятностью $p_{ij} $ переходит по ссылке на web-страницу $j$. 
Пусть из любой web-страницы можно по ссылкам перейти на любую другую 
web-страницу (условие неразложимости) графа $G$. Проверьте, что при бесконечно долгом блуждании доля времени, которую пользователь проведет на web-странице с номером $k$ есть $p_k $, где $ 
{p} =\left( {p_1 ,...,p_n } \right)^T $, ${p}^T = {p}^T P$, 
$P=\left\| {p_{ij} } \right\|_{i,j=1}^{n,n} $, $\sum_k p_k = 1$ (решение единственно,  ввиду неразложимости $P$). Обратим внимание, что ответ не зависит от того, с какой вершины стартует пользователь.

\item В условиях предыдущего пункта, пустим независимо блуждать по web-графу $N$ 
пользователей ($N\gg \left| V \right| \gg 1)$. Пусть $n_i \left( t \right)$ -- число посетителей web-страницы $i$ в момент времени $t$. Считая стохастическую матрицу $P$ неразложимой (см. п. а)) и апериодической (см. замечание), покажите, 
что
\[
\exists \;\;\lambda _{0.99} >0,\;T_G>0:\;\;\;\forall \;\;t\ge 
T_G
\]
\[
\PR\left( {\left. {\left| {\frac{n_k \left( t \right)}{N}-p_k } \right|\le 
\frac{\lambda _{0.99} }{\sqrt N }} \right)\ge 0.99} \right.,
\]
где $ {p}^T= {p}^T P$ (решение единственно). Обратим внимание, что ответ не зависит от того, с каких вершин стартуют пользователи.

\end{enumerate}

\end{problem}

\begin{remark}
\begin{comment}
Матрица $P^n = \Vert p_{ij}^{(n)} \Vert$ является матрицей переходных вероятностей за $n$ шагов. Число
\[
    d(j) = \text{НОД} \left\{n : p_{jj}^{(n)} > 0 \right\},
\]
где $\gcd$ обозначает наибольший общий делитель, называется \textit{периодом} вершины $j$. Если $d(j) = 1$, то  вершина называется апериодической. Для неразложимой $P$ периоды у всех вершин совпадают. Соответственно,  стохастическая матрица $P$ называется апериодической, если $\forall j: \; d(j) = 1$.
\end{comment}
Если организовать случайные блуждая, следуя п. б) 
(отметим, что эти блуждания хорошо распараллеливаются по числу 
блуждающих пользователей), то при определенных условиях можно получить решение задачи  ${p}^T = {p}^TP$ значительно быстрее, чем, скажем, ${\rm O}\left( {n^2} \right)$. Такой способ численного поиска вектора $ {p}$ основан на методе Markov 
chain Monte Carlo. %(см. задачу \ref{mcmc})%. 
Детали и ссылки см., например, в работе Гасников А.В., Дмитриев Д.Ю. Об эффективных рандомизированных алгоритмах поиска вектора PageRank // ЖВМиМФ. Т. 55. № 3. 2015 -- arXiv:1410.3120. Упомянем также недавнюю работу Belloni A., Chernozhukov V. On the Computational Complexity of MCMC-based Estimators in Large Samples. -- arXiv:0704.2167, 2012, содержащую строгие результаты об эффективной вычислимости байесовских оценок. Если 
использовать неравенства концентрации меры и иметь оценки на спектральную 
щель матрицы $P$ (см. \cite{44,240}), то приведенный в п. 
б) результат можно также сделать  более строгим, а именно точнее 
оценивать скорость сходимости и плотность концентрации (см.  задачу 19 раздела \ref{macrosystems}).
\end{remark}

\begin{comment}

\begin{problem}

В руки опытных криптографов попалось закодированное письмо (10 000 символов). Чтобы это письмо прочитать нужно его декодировать. Для этого берется стохастическая матрица переходных вероятностей $P$
(линейный размер которой определяется числом возможных символов (букв, знаков препинания и т.п.) в языке на котором до шифрования было написано письмо – этот язык известен и далее будет называться базовым). $P_{ij}$ отвечает за вероятность появления символа с номером $j$ сразу после символа под номером $i$. Такая матрица может быть идентифицирована с помощью статистического анализа  большого текста, скажем,  Войны и мира Л.Н. Толстого.

Пусть способ (де)шифрования определяется некоторой неизвестной функцией $\overline{f}$ – преобразование (перестановка) множества кодовых букв во множество символов базового языка.
В качестве, “начального приближения” выбирается какая-то функция $f$, например,
полученная исходя из легко осуществимого частотного анализа. Далее рассчитывается вероятность выпадения полученного закодированного текста, сгенерированного при заданной функции $f$  (правдоподобие выборки).

Случайно  выбираются два аргумента у функции $f$ и значения функции при этих аргументах меняются местами. Если в результате правдоподобие возрастает, то замена аргументов фиксируется, иначе бросается монетка с вероятностью выпадения орла равной отношению правдоподобий. 

Объясните, почему предложенный алгоритм ``сходится'' именно к $\overline{f}$ ? Оцените скорость сходимости.


\end{problem}



\subsection{ Markov chain Monte Carlo }

\begin{problem}
Чтобы построить однородный дискретный марковский процесс с конечным числом состояний, имеющий заданную инвариантную (стационарную) меру $\pi$ , переходные вероятности ищутся в следующим виде: $p_{ij} = p_{ij}^0 a_{ij}$ , 
$i \neq j$; $p_{ii} = 1 - \sum p_{ij}$ , где $p^0$ – некоторая затравочная матрица, которую будем далее предполагать симметричной. Покажите, что матрица $p$ имеет инвариантную (стационарную) меру  $\pi$, если
\[
\frac{a_{ij}}{a_{ji}} = \frac{\pi_{j} p^0_{ji}}{\pi_{i} p^0_{ij}} = \frac{\pi_{j}}{\pi_{i}} 
\]

Чтобы найти  $a_{ij}$ достаточно найти функцию F: $\mathbb{R_+} \rightarrow [0,1]$ такую, что

\[
\frac{F(z)}{F(1/z)} = z; a_{ij}  \leftarrow F( \frac{\pi_{j}}{\pi_{i}} )
\]

Пример функции $F(z) = \min(z,1)$ определяет алгоритм Метрополиса. 

\end{problem}

\begin{problem}

В руки опытных криптографов попалось закодированное письмо (10 000 символов). Чтобы это письмо прочитать нужно его декодировать. Для этого берется стохастическая матрица переходных вероятностей $P$
(линейный размер которой определяется числом возможных символов (букв, знаков препинания и т.п.) в языке на котором до шифрования было написано письмо – этот язык известен и далее будет называться базовым). $P_{ij}$ отвечает за вероятность появления символа с номером j сразу после символа под номером i . Такая матрица может быть идентифицирована с помощью статистического анализ  большого текста, скажем,  Войны и мира Л.Н. Толстого.

Пусть способ (де)шифрования определяется некоторой неизвестной функцией $\overline{f}$ – преобразование (перестановка) множества кодовых букв во множество символов базового языка.
В качестве, “начального приближения” выбирается какая-то функция f , например,
полученная исходя из легко осуществимого частотного анализа. Далее рассчитывается вероятность выпадения полученного закодированного текста, сгенерированного при заданной функции f  (правдоподобие выборки).

Случайно  выбираются два аргумента у функции f и значения функции при этих аргументах меняются местами. Если в результате правдоподобие возрастает, то замена аргументов фиксируется, иначе бросается монетка с вероятность выпадения орла равной отношению правдоподобий. 

Объясните, почему предложенный алгоритм “сходится” именно к $\overline{f}$ ? Почему сходимость оказывается такой быстрой (0.01 сек. на современном PC)?


\end{problem}

\subsection{Gibbs Sampler}

\subsection{Variational Inference}

\end{comment}




\begin{problem}\Star\,\,\Star (Markov Chain Monte Carlo Revolution и состоятельность
оценок максимального правдоподобия; P. Diaconis)
\label{cript}
В руки опытных криптографов попалось закодированное письмо (10~000 символов). Чтобы это 
письмо прочитать нужно его декодировать. Для этого берется стохастическая 
матрица переходных вероятностей $P=\left\| {p_{ij} } \right\|$ (линейный 
размер которой определяется числом возможных символов (букв, знаков 
препинания и т.п.) в языке на котором до шифрования было написано письмо -- 
этот язык известен и далее будет называться базовым), в которой $p_{ij} $ -- 
отвечает за вероятность появления символа с номером $j$ сразу после символа 
под номером $i$. Такая матрица может быть идентифицирована с помощью 
статистического анализ какого-нибудь большого текста, скажем, ``Войны и 
мира'' Л.Н.~Толстого.

Пускай способ (де)шифрования (подстановочный шифр) определяется некоторой, 
неизвестной, дешифрующей функцией $\bar {f}$ -- преобразование 
(перестановка) множества кодовых букв во множество символов базового языка.

В качестве, ``начального приближения'' выбирается какая-то функция $f$, 
например, полученная исходя из легко осуществимого частотного анализа. Далее 
рассчитывается вероятность выпадения полученного закодированного текста 
$ {x}$, сгенерированного при заданной функции $f$ (функция 
правдоподобия):

\[%\tag{*}
L\left( {{x};f} \right)=\prod\limits_k {p_{f\left( {x_k } 
\right),f\left( {x_{k+1} } \right)} } . 
\]

Случайно выбираются два аргумента у функции $f$ и значения функции при этих 
аргументах меняются местами. Если в результате получилась такая $f^\ast $, 
что $L\left( { {x};f^\ast } \right)\ge L\left( { {x};f} \right)$, то 
$f:=f^\ast $, иначе независимо бросается монетка с вероятностью выпадения 
орла $p={L\left( { {x};f^\ast } \right)} \mathord{\left/ {\vphantom 
{{L\left( {\vec {x};f^\ast } \right)} {L\left( { {x};f} \right)}}} 
\right. \kern-\nulldelimiterspace} {L\left( { {x};f} \right)}$, и если 
выпадает орёл, то $f:=f^\ast $, иначе $f:=f$. Далее процедура повторяется (в 
качестве $f$ выбирается функция, полученная на предыдущем шаге).

Объясните, почему предложенный алгоритм после некоторого числа итераций с 
большой вероятностью и с хорошей точностью восстанавливает дешифрующую 
функцию $\bar {f}$? Почему сходимость оказывается такой быстрой (0.01 сек. на современном PC)?

\end{problem}

\begin{ordre}
Описанный в задаче пример взят из обзора \textit{Diaconis P.} The Markov 
chain Monte Carlo revolution // Bulletin (New Series) of the AMS. 2009. V. 
49. № 2. P. 179--205. Детали того, что будет написано далее можно почерпнуть 
из работ \textit{Jerrum M}., \textit{Sinclair A}. The Markov chain Monte Carlo method: an approach to 
approximate counting and integration // Approximation Algorithms for NP-hard 
Problems / D.S. Hochbaum ed. Boston: PWS Publishing, 1996. P. 482--520; \textit{Joulin A., Ollivier Y.} Curvature, concentration and 
error estimates for Markov chain Monte Carlo // Ann. Prob. 2010. V. 38. № 6. 
P. 2418\textbf{--}2442; \textit{Paulin D.} Concentration inequalities for Markov chains by 
Marton couplings // e-print, \underline {arXiv:1212.2015v2}, 2013, см. также \cite{240}.

Для того чтобы построить однородный дискретный марковский процесс с конечным 
числом состояний, имеющий наперед заданную инвариантную (стационарную) меру 
$\pi $, переходные вероятности ищутся в следующем виде: $p_{ij} =p_{ij}^0 
b_{ij} $, $i\ne j$; $p_{ii} =1-\sum\limits_{j:\;\;j\ne i} {p_{ij} } $, где 
$p_{ij}^0 $ -- некоторая ``затравочная'' матрица, которую будем далее 
предполагать симметричной. Легко проверить, что матрица $p_{ij} $ имеет 
инвариантную (стационарную) меру $\pi $, если при $p_{ij}^0 >0$
\[
\frac{b_{ij} }{b_{ji} }=\frac{\pi _j p_{ji}^0 }{\pi _i p_{ij}^0 }=\frac{\pi 
_j }{\pi _i }.
\]
Чтобы найти $b_{ij} $, достаточно найти функцию $F:\;{\rm R}_+ \to \left[ 
{0,1} \right]$ такую, что

$\frac{F\left( z \right)}{F\left( {1 \mathord{\left/ {\vphantom {1 z}} 
\right. \kern-\nulldelimiterspace} z} \right)}=z$ и $$b_{ij} =F\left( 
{\frac{\pi _j p_{ji}^0 }{\pi _i p_{ij}^0 }} \right)=F\left( {\frac{\pi _j 
}{\pi _i }} \right).$$

Пожалуй, самый известный пример (именно он и использовался в задаче) такой 
функции $\tilde {F}\left( z \right)=\min \left\{ {z,1} \right\}$ -- алгоритм 
Хастингса--Метрополиса. Заметим, что для любой такой функции $F\left( z 
\right)$ имеем $F\left( z \right)\le \tilde {F}\left( z \right)$. Другой 
пример дает функция $F\left( z \right)=z \mathord{\left/ {\vphantom {z 
{\left( {1+z} \right)}}} \right. \kern-\nulldelimiterspace} {\left( {1+z} 
\right)}$. Заметим также, что $p_{ij}^0 $ обычно выбирается равным $p_{ij}^0 
=1 \mathord{\left/ {\vphantom {1 M}} \right. \kern-\nulldelimiterspace} M_i 
$, где $M_i $ число ``соседних'' состояний у $i$, или
\[
p_{ij}^0 =1 \mathord{\left/ {\vphantom {1 {\left( {2M} \right)}}} \right. 
\kern-\nulldelimiterspace} {\left( {2M} \right)},
\quad
i\ne j;
\quad
p_{ii}^0 =1 \mathord{\left/ {\vphantom {1 2}} \right. 
\kern-\nulldelimiterspace} 2,
\quad
i\ne j.
\]
При больших значениях времени $t$, согласно эргодической теореме, имеем, что 
распределение вероятностей близко к стационарному $\pi $. Действительно, при 
описанных выше условиях имеет место условие детального баланса (марковские 
цепи, для которых это условие выполняется, иногда называют обратимыми):
\[
\pi _i p_{ij} =\pi _j p_{ji} ,\;i,j=1,...,n,
\]
из которого сразу следует инвариантность меры $\pi $, т.е.
\[
\sum\limits_i {\pi _i p_{ij} } =\pi _j \sum\limits_i {p_{ji} } =\pi _j 
,\;j=1,...,n.
\]
Основное применение замеченного факта состоит в наблюдении, что время выхода 
марковского процесса на стационарную меру (mixing time) во многих случаях 
оказывается удивительно малым.\footnote{ Более того, задача поиска такого 
симметричного случайного блуждания на графе (с равномерной инвариантной 
мерой в виду симметричности) заданной структуры, которое имеет 
``наименьшее'' mixing time (другими словами, наибольшую спектральную щель), 
сводится к задаче полуопределенного программирования, которая, как известно, 
полиномиально (от числа вершин этого графа) разрешима.} При том, что 
выполнение одного шага по времени случайного блуждания по графу, отвечающему 
рассматриваемой марковской цепи, как следует из алгоритма Кнута--Яо (см. задачу 11), также 
может быть быстро сделано. Таким образом, довольно часто можно получать 
эффективный способ генерирования распределения дискретной случайной величины 
с распределением вероятностей $\pi $ за время полиномиальное от логарифма 
числа компонент вектора $\pi $.

Для лучшего понимания происходящего в условиях задачи, отметим, что одним из 
самых универсальных способов получения асимптотически наилучших оценок 
неизвестных параметров по выборке является метод наибольшего правдоподобия 
(Ибрагимов--Хасьминский, В.Г. Спокойный). Напомним вкратце в чем он 
заключается. Пусть имеется выборка из распределения, зависящего от 
неизвестного параметра -- в нашем случае выборкой $ {x}$ из 10~000 
элементов будет письмо, а неизвестным ``параметром'' будет функция $f$. 
Далее считается вероятность (или плотность вероятности в случае непрерывных 
распределений) $L\left( { {x};f} \right)$ того что выпадет данный $\vec 
{x}$ при условии, что значение параметра $f$. Если посмотреть на 
распределение $L\left( { {x};f} \right)$, как на распределение в 
пространстве параметров ($ {x}$ -- зафиксирован), то при большом объеме 
выборки (размерности $ {x})$ при естественных условиях это распределение 
концентрируется в малой окрестности наиболее вероятного значения
\[
f\left( {{x}} \right)=\arg \mathop {\max }\limits_f L\left( { 
{x};f} \right),
\]
которое ``асимптотически'' совпадает с неизвестным истинным значением $\bar 
{f}$.

\end{ordre}

\begin{remark}
Для оценки mixing time нужно оценить спектральную щель 
стохастической матрицы переходных вероятностей, задающей исследуемую 
марковскую динамику, то есть нужно оценить расстояние от максимального 
собственного значения этой матрицы равного единицы (теорема 
Фробениуса--Перрона) до следующего по величине модуля. Именно это число 
определяет основание геометрической прогрессии, мажорирующей исследуемую 
последовательность норм разностей расстояний (по вариации) между 
распределением в данный момент времени и стационарным (финальным) 
распределением. Для оценки спектральной щели разработано довольно много 
методов, из которых мы упомянем лишь некоторые: неравенство Пуанкаре 
(канонический путь), изопериметрическое неравенство Чигера (проводимость), с 
помощью техники каплинга (получаются простые, но, как правило, довольно 
грубые оценки), с помощью каплинга Мертона, с помощью дискретной кривизны 
Риччи и теорем о концентрации меры (Мильмана--Громова). Приведем некоторые 
примеры применения MCMC: Тасование $n$ карт, разбиением приблизительно на 
две равные кучи и перемешиванием этих куч (mixing time$\sim \log _2 
n)$;\footnote{ Здесь контраст проявляется, пожалуй, наиболее ярко. Скажем 
для колоды из 52 карт пространство состояний марковской цепи будет иметь 
мощность 52! (если сложить времена жизней в наносекундах каждого человека, 
когда либо жившего на Земле, то это число на много порядков меньше 52!). В 
то время как такое тасование: взять сверху колоды карту, и случайно 
поместить ее во внутрь колоды, отвечающее определенному случайному 
блужданию, с очень хорошей точностью выйдет на равномерную меру, отвечающую 
перемешанной колоде, через каких-то 200--300 шагов. Если брать тасование 
разбиением на кучки, то и того меньше -- за 8--10 шагов.} Hit and Run 
(mixing time $\sim n^3)$; Модель Изинга -- $n$ спинов на отрезке, 
стационарное распределение = распределение Гиббса, Глауберова динамика 
(mixing time $\sim n^{{2\log _2 e} \mathord{\left/ {\vphantom {{2\log _2 e} 
T}} \right. \kern-\nulldelimiterspace} T}$, $0<T\ll 1$ (см. задачу 18)); Проблема поиска 
кратчайших гамильтоновых путей; Имитация отжига (см. задачу 20) для решения 
задач комбинаторной оптимизации, MCMC для решения задач перечислительной 
комбинаторики. Но, пожалуй, самым известным примером (Dyer--Frieze--Kannan) 
является полиномиальный вероятностный алгоритм (работающий быстрее известных 
``экспоненциальных'' детерминированных) приближенного поиска центра тяжести 
выпуклого множества и вычисления его объема. Одна из работ в этом 
направлении была удостоена премии Фалкерсона -- аналога Нобелевской премии в 
области Computer Science. Близкие идеи используются и при применении 
экспандеров в Computer Science. В 2010 году премия Неванлинны 
была вручена Д. Спилману, в частности, за сублинейное (по числу элементов 
матрицы, отличных от нуля) решение системы линейных уравнений с помощью 
экспандеров (см. часть 2). 

%Полезно сравнить эту задачу с задачей 7 и с задачей 19 раздела [ССЫЛКА].
\end{remark}

\begin{problem} (Одномерная модель Изинга \cite{44}). 
Рассмотрим конечный отрезок одномерной целочисленной решетки $\{0,1,\dots,n\}$ в каждой вершине $k$ которой находится спин, принимающий два значения $\sigma(k)=\pm1$. При этом считаем, что $\sigma(0)=\sigma(n)=1$. Опредеим Гамильтониан системы $H(\sigma) = \sum_{k=0}^{n-2}(1-\sigma(k)\sigma(k+1))/2$. Определим расстояние Больцмана--Гиббса по формуле $\pi(\sigma) = Z^{-1}\exp(-\beta H(\sigma))$, $\beta = T^{-1}>0$~--- величина, обратная <<температуре>>, а $Z$~--- нормирующий множитель (статсумма). Одним из стандартных способов построения однородной дискретной марковской цепи с заданным стационарным распределением $\pi(\sigma)$ является распределение Глаубера:
\begin{enumerate}
\item Выбираем $k\in\{1,\dots,n-1\}$ согласно равномерному распределению.
\item С вероятностью $$p=\exp(\beta H(\sigma_{k,+1}))/\left(\exp(\beta H(\sigma_{k,+1}))+\exp(\beta H(\sigma_{k,-1}))\right)$$ новым состоянием будет $\sigma_{k,+1}$, а с вероятностью $1-p$ состояние  $\sigma_{k,-1}$,
где $$\sigma_{k,+1} = \sigma_{k}(i),\,i\not=k,\,\sigma_{k,+1} = 1;$$
$$\sigma_{k,-1} = \sigma_{k}(i),\,i\not=k,\,\sigma_{k,-1} = -1.$$
Покажите, что характерное время выхода на стационарное распределение (mixing time) 
этой марковской цепи оценивается сверху как $n^{2\log_2(\beta\exp(1))}$ в предположении $\beta \gg 1$.
\end{enumerate}
\end{problem}


\begin{problem}\Star (Hit and Run).
Ряд задач, в которых используется метод Монте-Карло, предполагает возможность случайно равномерно набрасывать точки в некоторое наперед заданное множество (не обязательно выпуклое и связное). Например, при вычислении интеграла с помощью метода
Монте-Карло или при численном решении задач оптимизации, в которых нужно уметь
приближенно находить центр тяжести множества. Исходя из MCMC подхода (см. \cite{240}) обоснуйте,
аккуратно оговорив детали, следующий способ генерации точек.

Берем любую точку внутри множества и проводим случайно направление через эту точку, далее с помощью
граничного оракула случайно генерируем (с помощью равномерного распределения) на
этом направлении внутреннюю точку рассматриваемого множества. Через эту точку снова
проводим случайное направление и т.д. Хорошо ли будет работать Hit and Run для вытянутых множеств или для множеств, имеющих достаточно острые углы? Предложите модификацию (например, с помощью эллипсоидов Дикина) алгоритма Hit and Run для таких “плохих” множеств. Предложите другие способы случайно равномерно набрасывать точки в некоторое наперед заданное множество (например, Shake and Bake). Хорошо ли будет работать метода Shake and Bake для множеств в пространствах большой размерности? Как следует действовать, если рассматриваемое множество имеет простую структуру: $n$-мерный куб, $n$-мерный шар, $n$-мерный симплекс, многогранник?
\end{problem}

\begin{remark}
См. L. Lovasz, S. Vempala, Hit-and-run from a corner, Proceedings of STOC, 2004, pp. 310– 314.
\end{remark}

\begin{problem}\DStar(Глобальная оптимизация и монотонный симметричный 
марковский поиск; Некруткин--Тихомиров) 
Рассматривается задача глобальной оптимизации $f\left( x \right)\to \mathop {\min }\limits_{x\in 
{\rm R}^n} $. Считаем, что глобальный минимум достигается в единственной 
точке $x^\ast $ (причем для любых $\varepsilon >0$ выполняется 
условие\footnote{$B_\varepsilon ^c \left( {x } \right)$ -- дополнение 
шара $B_\varepsilon { \left( x \right)}$ радиуса $\varepsilon $ с 
центром в точке $x $} $\inf \left\{ {f\left( x \right):\;\;x\in B_\varepsilon^c \left( {x^* } \right)} \right\}>f\left( {x^* } \right))$, 
$f\left( x \right)$ -- непрерывная функция, дважды гладка в точке 
$x^\ast $, причем матрица Гессе $G$ функции $f\left( x \right)$ в этой точке 
положительно определена. Опишем алгоритм (с точностью до выбора функции 
плотности распределения $g\left( r \right)$, $r\in \left[ {0,\infty } 
\right))$.

\begin{enumerate}
\item (начальный шаг) Выбираем точку старта $x_0 =x$;
\item (шаг $k<N$) Независимо генерируем с.в. $\xi _k $ из центрально симметричного распределения с заданной плотностью $g\left( r \right)$. Если
	\begin{itemize}
	\item $f\left( {x_k +\xi _k } \right)\le f\left( {x_k } \right),$ то $x_{k+1} =x_k +\xi _k $,
	\item иначе $x_{k+1} =x_k $;
	\end{itemize}

\end{enumerate}
Введем обозначения
\[
M_r =\inf \left\{ {x\in B_r{\left( {x^* } \right)}:\;\;f\left( x 
\right)<f\left( y \right)\;\mbox{для всех }y\in B_r^{c} \left( {x^* } 
\right)} \right\},
\]
\[
\tau _\varepsilon =\min \left\{ {n\in {\rm N}:\quad x_n \in M_\varepsilon } 
\right\},
\quad
\delta \left( x \right)=\inf \left\{ {r\ge 0:\;\;x\in M_r } \right\},
\quad
\]
\[
\Gamma =\prod\limits_{i=1}^n {\left( {\frac{\lambda _i }{\lambda _{\min } }} 
\right)^{1 \mathord{\left/ {\vphantom {1 2}} \right. 
\kern-\nulldelimiterspace} 2}} ,
\]
где $\lambda _i $ \textbf{-- }собственные числа $G$. Покажите, что, как бы 
мы не выбирали функцию плотности $g\left( r \right)$, всегда при 
$\varepsilon <\rho \left( {x,x^\ast } \right)$ имеет место следующая оценка 
снизу:
\[
\Exp\left[ {\left. {\tau _\varepsilon } \right|x_0 =x} \right]\ge \ln \left( 
{{\rho \left( {x,x^\ast } \right)} \mathord{\left/ {\vphantom {{\rho \left( 
{x,x^\ast } \right)} \varepsilon }} \right. \kern-\nulldelimiterspace} 
\varepsilon } \right)+2.
\]
Покажите, что метод с плотностью (есть много других вариантов)
\[
g\left( r \right)=\nu \left( r \right)r^{-d},
\quad
\nu \left( r \right)=\frac{c}{\left( {e+n\left| {\ln r} \right|} \right)\ln 
^2\left( {e+n\left| {\ln r} \right|} \right)},
\]
где $c$ -- находится из условия нормировки, дает оценку ($\varepsilon 
<{\delta \left( x \right)} \mathord{\left/ {\vphantom {{\delta \left( x 
\right)} 2}} \right. \kern-\nulldelimiterspace} 2)$

\[\tag{*}
\Exp\left[ {\left. {\tau _\varepsilon } \right|x_0 =x} \right]\le b^n\Gamma \ln 
^2\left( \varepsilon \right)\ln ^2\left( {\ln \left( \varepsilon \right)} 
\right)\left| {\ln \left( {\delta \left( x \right)} \right)} \right|, 
\]
где $b\in \left( {2,3} \right)$ (для простоты восприятия мы привели здесь 
огрубленный вариант). 

\end{problem}

\begin{remark}
Если отказаться от гладкости и(или) положительной 
определенности матрицы $G$, то вместо $\Gamma $, которое в типичных 
ситуациях растет с размерностью пространства экспоненциально быстро, в (*) 
можно использовать $F_{\varepsilon ,x}^{-1} $, где
\[
F_{\varepsilon ,x}  = \mathop {\inf }\limits_{\varepsilon \le r<\delta 
\left( x \right)} \left\{ {{\mbox{vol}\left( {M_r } \right)} \mathord{\left/ 
{\vphantom {{\mbox{vol}\left( {M_r } \right)} {\mbox{vol}\left( {B_r \left( 
{x^\ast } \right)} \right)}}} \right. \kern-\nulldelimiterspace} 
{\mbox{vol}\left( {B_r \left( {x^\ast } \right)} \right)}} \right\}.
\]
Отметим, что все приведенные результаты сохраняются с небольшими поправками 
и для оценок вероятностей больших уклонений, т.е. для
\[
n\left( {x,\varepsilon ,\gamma } \right)=\min \left\{ {n:\;\;\PR\left( {x_n 
\in M_\varepsilon \left| {x_0 =x} \right.} \right)\ge \gamma } \right\}=
\]
\[=\min 
\left\{ {n:\;\;\PR\left( {\tau _\varepsilon \le n\left| {x_0 =x} \right.} 
\right)\ge \gamma } \right\}.
\]
Детали имеются в работах А.С. Тихомирова, опубликованных за последние 20 лет 
в ЖВМ и МФ. 

Изложенные в этой задаче результаты могут вызвать на первых порах удивление. 
И, действительно, как такое возможно, чтобы в задачах глобальной оптимизации 
зависимость числа итераций от точности была логарифмическая, в то время как 
известны нижние оценки, в которых эта точность входит в степени размерности 
пространства (в случае равномерной гладкости высокого порядка, степень можно 
понижать) в знаменателе, см., например, \textit{Zhigljavsky A., Zilinskas A.} Stochastic global optimization. 
Springer Optimization and Its Applications, 2008? Тут стоит отметить, что, 
во-первых, нижние оценки получаются для детерминированных методов, ну и 
самое главное, что ``проклятие размерности'' здесь также никуда не делось. 
Даже при самом благоприятном раскладе, в оценку (*) входит фактор $2^n$, 
экспоненциально растущий с ростом размерности пространства. В отличие от 
глобальной оптимизации, в выпуклой оптимизации такие проблемы можно решать (см. Часть 2).

\end{remark}

 \begin{problem}[Глобальная оптимизация и simulated annealing]
 \label{annealing}
 Пожалуй, самым популярным сейчас методом глобальной оптимизации (правда, с очень 
плохими на данный момент теоретическими оценками скорости сходимости) 
является simulated annealing (имитация затвердевания или отжига), 
представляющий собой дискретное приближение решения стохастического 
дифференциального уравнения\footnote{ Детально изученного в статье \textit{German S., Hwang C.P.} 
Diffusions for global optimization // SIAM J. Control and Optimization. 
1986. V. 24. no. 5. P. 1031--1043.}
\[
dx_t =-\nabla f\left( {x_t } \right)dt+\sqrt {2T\left( t \right)} dw_t ,
\]
где $w_t $ -- винеровский процесс. Покажите, что при неограничительных 
условиях и $T\left( t \right)\equiv T$ траектория $x_t $ имеет при $t\to 
\infty $ стационарное распределение с плотностью Гиббса
\[
\frac{\exp \left( {-{f\left( x \right)} \mathord{\left/ {\vphantom {{f\left( 
x \right)} T}} \right. \kern-\nulldelimiterspace} T} \right)}{\int {\exp 
\left( {-{f\left( z \right)} \mathord{\left/ {\vphantom {{f\left( z \right)} 
T}} \right. \kern-\nulldelimiterspace} T} \right)dz} },
\]
экспоненциально концентрирующееся в окрестности единственной точки 
глобального минимума $x^\ast $ дважды гладкой функции $f\left( x \right)$ 
при $T\to 0+$. Однако при $T\to 0+$ и время выхода на это стационарное 
распределение неограниченно возрастает, что создает проблемы для 
практического применения. Более правильно брать $T\left( t \right)=c 
\mathord{\left/ {\vphantom {c {\ln \left( {2+t} \right)}}} \right. 
\kern-\nulldelimiterspace} {\ln \left( {2+t} \right)}$, где $c$ -- 
достаточно большое число. Покажите, что тогда для любой начальной точки $x_0 
$ траектория процесса $x_t $ имеет в пределе $t\to \infty $ (который, 
фактически, с хорошей точностью проявляется уже на конечных временах) 
распределение, сосредоточенное в точке $x^\ast $. 

\end{problem}

\begin{remark} 
Детали и способы дискретизации можно почерпнуть из 
работы \textit{Kushner H.} Asymptotic global behavior for stochastic approximation and 
diffusion with slowly decreasing noise effects: global minimization via 
Monte Carlo // SIAM J. Appl. Math. 1987. V. 47. no. 1. P. 169--183.
\end{remark}

\begin{problem} [Multilevel Monte Carlo; M. Giles] 
Некоторый диффузионный процесс описывается стохастическим дифференциальным уравнением
\[
dS\left( t \right)=a\left( {S,t} \right)dt+b\left( {S,t} \right)dW\left( t 
\right),
\quad
0\le t\le T,
\quad
S\left( 0 \right)=S_0 ,
\]
где $W\left( t \right)$ -- винеровский процесс. Задана липшицева функция 
$f\left( S \right)$. Требуется предложить численный способ оценивания 
\[
Y=\Exp\left[ {f\left( {S\left( T \right)} \right)} \right].
\]

а)* Дискретизируем задачу по схеме Эйлера
\[
\hat {S}_{n+1} =\hat {S}_n +a\left( {\hat {S}_n ,t_n } \right)h+b\left( 
{\hat {S}_n ,t_n } \right)\Delta W_n ,
\]
возьмем $N$ независимых реализаций $\left\{ {\hat {S}_n^{\left( i \right)} } 
\right\}$, и положим
\[
\overline {Y}=\frac{1}{N}\sum\limits_{i=1}^N {f\left( {\hat {S}_{T 
\mathord{\left/ {\vphantom {T h}} \right. \kern-\nulldelimiterspace} 
h}^{\left( i \right)} } \right)} .
\]
Покажите, что найдутся такие $C_1 ,C_2 >0$, что
\[
\mbox{MSE}=E\left[ {\left( {\overline {Y}-Y} \right)^2} \right]\approx C_1 
N^{-1}+C_2 h^2.
\]
Покажите, что если от оценки требуется точность $\varepsilon $ ($\sqrt 
{\mbox{MSE}} ={\rm O}\left( \varepsilon \right))$, то оптимально (с точки 
зрения $\mbox{Total}\left( \varepsilon \right)$ -- общего числа 
арифметических операций / генерирования нормальных с.в. $\Delta W_n )$ 
выбирать
\[
h={\rm O}\left( \varepsilon \right),
\quad
N={\rm O}\left( {\varepsilon ^{-2}} \right),
\quad
\mbox{Total}\left( \varepsilon \right)={\rm O}\left( {\varepsilon ^{-3}} 
\right).
\]

б)** Предложим другой (более эффективный) способ оценивания. Для 
этого введем константу $M>1$ и положим
\[
h_l =M^{-l}T,
\quad
\overline {Y}_l =\frac{1}{N_l }\sum\limits_{i=1}^{N_l } {\left( {f\left( {\hat 
{S}_{T \mathord{\left/ {\vphantom {T {h_l }}} \right. 
\kern-\nulldelimiterspace} {h_l }}^{\left( i \right)} } \right)-f\left( 
{\hat {S}_{T \mathord{\left/ {\vphantom {T {h_{l-1} }}} \right. 
\kern-\nulldelimiterspace} {h_{l-1} }}^{\left( i \right)} } \right)} 
\right)} ,
\]
\[
\overline {Y}_0 =\frac{1}{N_0 }\sum\limits_{i=1}^{N_0 } {f\left( {\hat {S}_{T 
\mathord{\left/ {\vphantom {T {h_0 }}} \right. \kern-\nulldelimiterspace} 
{h_0 }}^{\left( i \right)} } \right)} ,
\quad
\overline {Y}=\sum\limits_{l=0}^L {\overline {Y}_l } .
\]
Покажите, что
\[
\mbox{Bias}=E\left[ {\overline {Y}-Y} \right]={\rm O}\left( {h_L } \right),
\]
\[
V_l =D\left[ {f\left( {\hat {S}_{T \mathord{\left/ {\vphantom {T {h_l }}} 
\right. \kern-\nulldelimiterspace} {h_l }}^{\left( i \right)} } 
\right)-f\left( {\hat {S}_{T \mathord{\left/ {\vphantom {T {h_{l-1} }}} 
\right. \kern-\nulldelimiterspace} {h_{l-1} }}^{\left( i \right)} } \right)} 
\right]={\rm O}\left( {h_l } \right),
\]
Мы хотим, чтобы
\[
\sqrt {\mbox{MSE}} =\sqrt {\mbox{Bias}^2+\Var\left[ {\overline {Y}} \right]} \le 
\mbox{Bias}+\sqrt {\Var\left[ {\overline {Y}} \right]} \sim \varepsilon ,
\]
что достигается, если положить
\[
L={\log \left( {\varepsilon ^{-1}} \right)} \mathord{\left/ {\vphantom 
{{\log \left( {\varepsilon ^{-1}} \right)} {\log M}}} \right. 
\kern-\nulldelimiterspace} {\log M}+{\rm O}\left( 1 \right),
\quad
\Var\left[ {\overline {Y}} \right]=\sum\limits_{l=0}^L {N_l^{-1} V_l } \sim 
\sum\limits_{l=0}^L {N_l^{-1} h_l } \sim \varepsilon ^2.
\]
Учитывая это, покажите, что решение задачи $$\mbox{Total}\left( \varepsilon 
\right)=\sum\limits_{l=0}^L {N_l h_l } \to \mathop {\min }\limits_{\left\{ 
{N_l } \right\}\ge 0}$$ при ограничении $\sum\limits_{l=0}^L {N_l^{-1} h_l } 
={\rm O}\left( {\varepsilon ^2} \right)$ имеет вид $N_l ={\rm O}\left( 
{\varepsilon ^{-2}Lh_l } \right)$. Таким образом, $\mbox{Total}\left( 
\varepsilon \right)={\rm O}\left( {\varepsilon ^{-2}\left( {\log \varepsilon 
} \right)^2} \right)$.

\end{problem}

\begin{remark}
Описанный в п. б) метод был предложен относительно 
недавно в контексте разработки эффективных численных методов оценки 
финансовых инструментов на рынке, поэтому он попал далеко не во все 
классические монографии на эту тему: \textit{Glasserman P.} Monte Carlo methods in financial 
engineering. Springer, 2005; \textit{Graham C., Talay D. }Mathematical foundation of stochastic 
simulation. Series ``Stochastic modelling and applied probability''. V. 68. 
2013. Тем не менее, мы рекомендуем эти книги для погружения в область 
численных методов финансовой математики.

\end{remark}

\section{Теория информации и кодирование}
\label{information}

\begin{comment}
\subsection{Основные определения}
Пусть $X$ - дискретная случайная величина, принимающая значения из конечного множества (алфавита) $A = \{a_1,..., a_{|A|}\}$. $P = \{\mathbb{P}\{X = a_i\} = p_{a_i}\}$ - вероятностное распределение $X$ на $A$.

\textit{Условной энтропией} двух с.в. $X \in P$ и $Y \in Q$ называется $H(P|Q) = \sum_{a \in A} \mathbb{P}(Y = a)H(P|Y=a)$, где $H(P|Y=a) = \sum_{a' \in A} \frac{\mathbb{P}(X = a', Y = a)}{\mathbb{P}(Y = a)} \log \frac{\mathbb{P}(X = a', Y = a)}{\mathbb{P}(Y = a)}$. Условная энтропия характеризует ту среднюю степень неопределнности, содержащейся в $X$, если имеется некоторая информация об $Y$.

\begin{definition} \textit{Словом} в алфавите $A$ будем называть реализацию последовательности случайных величин $X_1,...,X_n..$: $w = (x_1...x_n..)$, $x_i \in A$.
\end{definition}

\begin{definition} 
\textit{Энтропией} $H(X)$ случайной величины $X$ распределенной по закону $P$ называется:
\begin{center}
$H(X) = - \sum_{a \in A} p_a\log p_a$, где $\log = \log_2$.
\end{center}
Иногда вместо $H(X)$  используется запись $H(P)$.
Энтропия измеряется в битах и интерпретируется как мера неопределенности или информационного содержания случайной величины. Чем она больше, тем больше неопределенность. В качестве иллюстрации, читателю предлагается решить первую задачу.
\end{definition}
В случае нескольких случайных величин можно определить два тесно связанных понятия: \textit{условной энтропии} и \textit{совместной информации}.
\begin{definition}
\textit{Условной энтропией} двух с.в. $X$ и $Y$ называется $H(X|Y) = \sum_{a \in A} \mathbb{P}(Y = a)H(X|Y=a)$, где $H(X|Y=a) = \sum_{a' \in A} \frac{\mathbb{P}(X = a', Y = a)}{\mathbb{P}(Y = a)} \log \frac{\mathbb{P}(X = a', Y = a)}{\mathbb{P}(Y = a)}$. Условная энтропия характеризует ту среднюю степень неопределнности, содержащейся в $X$, если имеется некоторая информация об $Y$.
\end{definition}

\begin{definition}
\textit{Совместная информация} $I(X,Y) = H(X) - H(X|Y)$ определяет то, сколько информации об $X$ содержится в $Y$.
\end{definition}

\begin{definition}
\textit{Относительной энтропией} случайных величин $X \backsim P$ и $Y \backsim Q$ на множестве $A$ (или расстоянием Кульбака-Лейблера между ними) называется
\begin{center}
$KL(P||Q) = \sum_{a \in A} p_a \log \frac{p_a}{q_a}$. 
\end{center}
В статистике эта величина определяет то, насколько "неэффективно" использование распределения $Q$ для аппроксимации распределедния $P$, или как много дополнительных бит мы заплатим за такую аппроксимацию.
\end{definition}

\begin{definition}
\textit{Кодом} слова $w \in A^{len(w)}$ в алфавите $\Sigma$ называется отображение $C(w) : w \rightarrow \sigma$, $\sigma \in B^{len(\sigma)}$. $len(\sigma)$ - длина кодового слова.
\end{definition}

\subsection{Задачи}


ЧТО ТАКОЕ РЫЧАЖНЫЕ ВЕСЫ? ПОЧЕМУ В УСЛОВИИ ЗАДАЧИ СТОИТ СЛОВО ВОЗМОЖНО? В ЧЕМ ЗАКЛЮЧАЕТСЯ ВЗВЕШИВАНИЕ?

25 ПОТОМУ ЧТО ФАЛЬШИВОЙ МОНЕТЫ МОЖЕТ НЕ БЫТЬ?

"потому чтокаждый из 25-и исходов может быть закодирован двоичным словомуказанной длины." - и что?

см. две статьи в КВАНТе и то как об этом написано у Щепина в совместной книге с Верещагиным
\end{comment}

\begin{problem}
Имеется 12 монет, из них ровно одна фальшивая, которая может быть как легче, так и тяжелее настоящих.   Предложите алгоритм взвешиваний на чашечных весах без гирь, выявляющий за 3 взвешивания подделку, а также определяющий является ли фальшивая монетка тяжелее или легче настоящих.
\begin{ordre}
Любая из 12 монет может быть фальшивой, при этом быть легче, либо тяжелее настоящих. Для решения требуется с помощью взвешиваний  получить $\log_24$ бит информации (по Хартли), так как каждый  исходов может быть закодирован двоичным словом длины 24. С другой стороны, каждое взвешивание имеет три возможных исхода: выше левая чаша весов, выше правая или они в равновесии. То есть каждое взвешивание дает не более $\log3$ бит информации. Соответственно, требуемое число взвешеваний не может быть мееньше $\frac{\log24}{\log3} = 3$. 

Для решения этой задачи и последующих полезно ознакомиться с книгой Н.К. Верещагин, Е.В. Щепин Информация, кодирование и предсказание. -- М.:  МЦНМО, 2012, а также см. метод Дайсона в статье Г. Шестопала ``Как обнаружить фальшивую монетку'' в журнале Квант 1979, номер 10.
\end{ordre}
\end{problem}

\begin{problem}
Патриций решил устроить праздник и для этого приготовил 240 бочек вина. Однако к нему пробрался недоброжелатель, который подсыпал яд в одну из бочек. Про яд известно, что человек, его выпивший, умирает в течение (не «через»!) 24 часов. До праздника осталось два дня, то есть 48 часов. У патриция есть пять собак, которыми он готов пожертвовать, чтобы узнать в какой именно бочке яд.
Как патрицию вычислить отравленную бочку?
\end{problem}

\begin{problem}[Цена информации] Имеется неизвестное число от $1$ до $n$, $n \geq 2$. Разрешается задавать любые вопросы с ответами ДА/НЕТ. При этом при ответе ДА игрок платит 1 рубль, а при ответе НЕТ - 2 рубля. Сколько необходимо и достаточно заплатить для отгадывания числа?
\end{problem}


\begin{problem}[Аксиоматическое определение энтропии]
Рассмотрим множество функций, заданных на единичном симплексе. Докажите, что существует единственная (с точностью до множителя) функция, удовлетворяющая нижеперечисленным требованиям. Она имеет вид $H(P) = -\sum_{i=1}^n p_i \log p_i$ и используется в качестве количественной характеристики меры неопределенности.
\begin{enumerate}
\item Значение функции $H(P)$ не меняется при перестановке чисел ${p_{a_1},..., p_{a_n}}$,
\item $H(P)$ непрерывная функция,
\item выполняется равенство
\begin{center}
$H(p_1,...,p_n) = H(p_1 + p_2, p_3,..., p_n) + (p_1 + p_2) H(\frac{p_1}{p_1+p_2},\frac{p_2}{p_1+p_2} )$.
\end{center}
То есть неопределенность в исходе опыта не зависит от того, осуществляется ли выбор среди всех возможных альтернатив одномоментно или в несколько этапов.
\end{enumerate}
\end{problem}

\begin{problem}
\begin{enumerate}
\item Ф.М. Достоевский решил изменить своим привычкам и отправился на скачки. У него есть предварительные (априорные) данные о том, какие шансы на победу имеет каждая из восьми лошадей-участниц: $(\frac{1}{2}, \frac{1}{4}, \frac{1}{8}, \frac{1}{16}, \frac{1}{64}, \frac{1}{64}, \frac{1}{64}, \frac{1}{64})$. Оцените энтропию, которая содержится в таких данных. 
\item Сравните результат со случаем, когда все исходы равновероятны. Какое из двух респределений содержит больше информации?
\item Докажите в общем случае, что из всех дискретных распределений на конечном множестве $A$, наибольшей энтропией обладает равномерное.
\end{enumerate}
\end{problem}

\begin{comment}
\begin{problem}
ТО КАК СЕЙЧАС ПЛОХО, см. 

http://dcam.mipt.ru/students/study/stohanaliz-files/ Задание по теории вероятнсотей 2010

Эта таблица ничуть не помогает в восприятии:(

В таблице приведен прогноз погоды в г. Долгопрудный:
$p$ - вероятность наличия/отсутствия осадков, $q$ - вероятность того, что прогноз окажется верным.
\begin{table}[h]
\caption{Прогноз погоды}
\begin{center}
\begin{tabular}{|c|c|c|c|c|}
\hline
 &$p_{rain}$  &$p_{fine}$ & $q_{rain}$ & $q_{fine}$  \\
\hline
$15$ июня & $0.4$ & $0.6$ & $3/5$ & $4/5$\\
\hline
$15$ октября & $0.8$ & $0.2$ & $9/10$ & $1/2$\\
\hline
\end{tabular}
\end{center}
\end{table}
В какой из указанных двух дней прогноз дает нам больше информации о реальной погоде?
\end{problem}
\end{comment}
\begin{problem}
Пусть для некоторого населенного пункта вероятность того, что 15 июня будет дождь, равна 0.4, а вероятность того, что дождя не будет, равна 0.6. Для этого же пункта вероятность дождя 15 октября равна 0.8, а вероятность отсутствия осадков равна 0.2. Предположим, что определенный метод прогноза погоды 15 июня оказывается верным в $3/5$ всех тех случаев, когда предсказывается дождь и в $4/5$ тех случаев, в которых прогнозируется отсутствие осадков. Применительно к погоде на 15 октября этот метод оказывается правильным в 9 из 10 случаев, когда предсказывается дождь, и в половине случаев, когда предсказывается его отсутствие. В какой из указанных двух дней прогноз дает нам больше информации о реальной погоде?
\end{problem}


\begin{problem}[Задача о шляпах. Тодд Эберт, 1998]

Трех игроков отводят в комнату, где на них надевают (случайно и независимо) белые и черные шляпы. Каждый видит 
цвет других шляп и должен написать на бумажке одно из трех слов: <<белый>>, <<черный>>, <<пас>> 
(не советуясь с другими и не показывая им свою бумажку). Команда выигрывает, если хотя бы один из игроков назвал правильный 
цвет своей шляпы и ни один не назвал неправильного. Как им сговориться, чтобы увеличить шансы? Оптимальна ли предложенная Вами стратегия?
Решите эту же задачу, если игроков $n=2^m -1$ $( m\in {\mathbb N} )$. 
\end{problem}

\begin{ordre}
Воспользуйтесь кодом Хэмминга.
Докажем для случая трех игроков, что вероятность выигрыша не может превышать $3/4$. 

Единственная информация, которой владеет $i$-й игрок --- это цвета шляп двух других. Поэтому стратегия для $i$-го игрока должна зависеть 
только от этих двух цветов. В каждом случае имется три варианта ответа для игрока: $0$, $1$ или <<пас>>, т.е. всего $3^{12}$ различных 
стратегий. Поскольку есть $8$ вариантов расположения шляп на игроках, более выгодная стратегия должна обеспечивать выигрыш в $7$ вариантах. 
Тогда один из игроков должен угадать свой цвет в $3$ ситуациях. Значит, имеются для него ответы $\alpha_{i_1 j_1}$, 
$\alpha_{i_2 j_2}$, не являющиеся пасами. Но тогда в ситуациях $\overline{\alpha_{i_1 j_1}} i_1 j_1$ и 
$\overline{\alpha_{i_2 j_2}} i_2 j_2$ он ошибется, что противоречит предположению о $7$ выигрышных ситуациях. 

Таким образом, максимальная вероятность выигрыша не превышает $3/4$.
Обобщите это рассуждение на случай $n = 2^m - 1$.
\end{ordre}

\begin{problem}[Граница Эдгара Гилберта]
Для обеспечения помехоустойчивости кода при передачи информации, вместо исходного $k$-буквенного сообщения, передается $n$-буквенное ($n>k$). Возникает вопрос, при каких значениях параметров $q = |\Sigma|$ - размер алфавита, $k$, $n$, $l$ существует код $F: \Sigma^k \rightarrow \Sigma^n$, исправляющий  $l$ ошибок, и как его построить? 
Достаточное условие существования кода дает так называемая \textit{граница Гилберта}, которая описана ниже. 

Пространство кодовых слов содержит всего $q^n$ элементов. 
Назовем шаром радиуса $e$ с центром в слове $x\in \Sigma^n$ множество слов, отличающихся от $x$ не более чем в $l$ позициях.
Будем выбирать кодовые слова одно за другим произвольным образом, следя за тем, чтобы расстояния (по Хэммингу) между ними были больше $2l$. В какой-то момент пространство окажется полностью покрытым шарами, каждый из которых состоит из $V_q(2l, n)$ элементов, где $V_q(2l, n)$ - объем шара радиуса $2l$. Тогда при выполнении условия:
\[
(q^k - 1) V_q(2e, n) < q^n
\] 
существует код с параметрами  $q$, $k$, $n$, $l$.
Используя формулу Стирлинга ($k! \approx (k/е)^k$), оцените число элементов в шаре $V_q(2l, n)$. В случае $q=2$ последняя оценка легко получается применением оценок больших уклонений биномиальной случайно величины.

С помощью идеи {\it случайного кодирования} можно посторить код, с точностью до двух битов, реализующий границу Гилберта с большой вероятностью. Для этого случайно и независимо выбираются $N$ кодовых слов $\xi_1, \ldots, \xi_N$ в пространстве $\Sigma^n$. 

Найдите достаточное условие на параметры  $q$, $N$, $n$, $e$, при котором среди сгенерированных кодовых слов менее половины в среднем (усреднение по выбору случайного кода) будут иметь ``ближайшего соседа''  на расстоянии, не превышающем  $2l$.

\begin{remark}
В контексте это задачи полезно познакомиться с брошюрой Ромащенко А., Румянцев А., Шень А. Заметки по теории кодирования. -- М.: МЦНМО, 2011.
\end{remark}

\end{problem} 


\begin{problem}
Пусть $\{X_i\}_{i=1}^n$ -- независимые в совокупности одинаково распределенные случайные величины с распределением $P = \{p_{a}\}$, на конечном множестве $A = \{a\}$. Доказать, что 
\begin{center}
$-\frac{1}{n} \sum_{i = 1}^n \log \biggl ( P(X_i) \biggr ) \xrightarrow[n \to \infty]{\mathbb{P}} H(P)$
\end{center}
Или, другими словами, для любых $\delta, \varepsilon > 0$ найдется $n_0$ такое что для всех $n \geq n_0$:
\begin{center}
$\mathbb{P}(|-\frac{1}{n} \sum_{i = 1}^n \log P(X_i) - H(P)| < \delta) > 1-\varepsilon$
\end{center}

\begin{ordre}
Воспользутесь тем, что $-\mathbb{E} \log P(X) = H(P)$.
\end{ordre}
\end{problem}

\begin{remark} При достаточно больших значениях $n$ можно определить множество \textit{типичных последовательностей} или \textit{слов}, энтропия которых близка к истинной энтропии распределения $P$. Вероятность появления слова $w$
\begin{center}
$p_w = p_{x_1}...p_{x_n} = 2^{-n (-\frac{1}{n} \sum_{i = 1}^n \log P(X_i))}$.
\end{center}
\textit{Множеством $\delta$-типичных $n$-буквенных слов} в модели, где буквы появляются независимо, назовем $T_{\delta}^{(n)}$:
$$
T_{\delta}^{(n)} = \{w: 2^{-n(H(P) + \delta)} < p_w < 2^{-n(H(P) - \delta)} \}
$$
\end{remark}

\begin{problem}[Асимптотическая равнораспределенность]
Докажите, что:
\begin{enumerate}
\item мощность множества типичных слов ограничено: \[|T_{\delta}^{(n)}| \leq 2^{n(H(X) + \delta)};\]
\item $|T_{\delta}^{(n)}| \geq (1-\varepsilon)2^{n(H(X) + \delta)}$ для достаточно больших $n$;
\item вероятность нетипичности $w$: $\mathbb{P}\{w \not\in T_{\delta}^{(n)} \} \to 0, \; n\to\infty$.
\end{enumerate}
\end{problem}

\begin{remark} Идея о типичных последовательностях лежит в основе кодирования. Например, $\delta$-типичные $n$-буквенные слова кодируются при помощи двоичных последовательностей длины $n(H(X) + \delta)$, нетипичные отбрасываются или представляются одним и тем же добавочным символом. Очевидно, что при декодировании (восстановлении) вероятность ошибки не превысит $\varepsilon$.
\end{remark}
\begin{comment}
\begin{problem} Рассмотрите связь между доказательством принципа асимптотической равнораспределенности и эквивалентностью (для больших систем)энтропий Больцмана и Гиббса.
\end{problem}
\end{comment}

\begin{comment}
\begin{problem}
Всего существует $2^{n\log m}$ $n$-буквенных случайных текстов над алфавитом $(x_1,\ldots,x_m)$. Для их кодирования понадобиться $n\log m$ бит. Предполагая, что все буквы появляются независимо друг от друга по закону $(p_1,\ldots,p_m)$, который в общем случае не является равномерным, предложите лучший способ кодирования, основанный 
на законе больших чисел.
\begin{ordre}
Воспользуйтесь результатами, полученными в предыдущих задачах.
\end{ordre}
\end{problem}

\begin{remark}
Необходимым и достаточным условием существования префиксных кодов является неравенство Крафта -- Макмиллана. 
Пусть $n_1,..., n_k$ - длины кодовых слов, тогда $\sum_{i=1}^k 2^{-n_i} \leq 1$.
Тогда задача о построении кода минимальной средней длины может быть сформулирована следующим образом:
задан вектор $(p_1,...,p_k)$, $\sum p_i = 1$. Найти $n_1,..., n_k$, $n_i > 0$, для которых выполнено
неравенство Крафта -- Макмиллана, а $\sum_{i=1}^k p_in_i \rightarrow \min$, где минимум берется по всем возможным наборам
$n_1,..., n_k$.
Решением этой оптимизационной задачи является \textit{код Хаффмена}. 

\end{remark}
\end{comment}

\begin{problem}
Рассмотрим $n$-буквенные слова, порождаемые следующей моделью: появление каждой буквы $x_i \in A$ ($|A| = m$ - алфавит) не зависит от контекста и подчиняется закону распределения $P = \{p_a, a \in A\}$. Всего существует $2^{n\log m}$ таких слов, поэтому каждое из них может быть закодировано при помощи $n\log m$ бит информации. Если же предполагать, что распределение букв $P$ не является равномерным, то найдется лучший способ кодирования. Предложите такой способ.
\begin{comment}
\begin{ordre}
Воспользуйтесь результатами, полученными в предыдущих задачах.
\end{ordre}
\end{comment}
\end{problem}
\begin{comment}
НЕ ОЧЕНЬ УДАЧНОЕ ЗАМЕЧАНИЕ!

НУЖНО ПОЯСНИТЬ, ЧТО ТАКОЕ КОД?
В ЧЕМ РАЗЛИЧИЕ И ЧТО ТАКОЕ КОДЫ ХАФФМАНА И ШЕННОНА-ФАНО?
"Этот подход продемонстрирован в следующих задачах" - ЭТО НЕ ТАК!
\end{comment}
\begin{remark}
Во многих приложениях код должен быть не только однозначно декодируем, но и \textit{оптимален} в смысле минимальности средней длины. Необходимым (а в  случае префиксных кодов и достаточным) условием для выполнения первого требования является \textit{неравенство Крафта -- Макмиллана}: пусть $l(a)$ - длина кодового слова для буквы $a \in A$, тогда $\sum_{a \in A} 2^{-l(a)} \leq 1$.

Среди всех кодов, удовлетворяющих неравенству Крафта -- Макмиллана найдется код минимальной средней длины: $\sum_{a \in A} p_{a}l(a)\rightarrow \min$, при условии  $\sum_{a \in A} 2^{-l(a)} \leq 1$.  Для этого можно, например, воспользоваться методом множителей Лагранжа: $$
\sum_{a \in A} p_{a}l(a) + \lambda (\sum_{a \in A} 2^{-l(a)} - 1)\rightarrow \min.
$$ Решением является набор $\{l_{opt}(a) = -\log{p(a)}, a \in A\}$, а величина $-\log{p(a)}$ также называется \textit{собственной информацией}. Тогда минимальная средняя длина кода $-\sum_{a \in A}p(a)\log{p(a)} = H(P)$ это ни что иное, как энтропия.
\end{remark}

\begin{problem}[Оценка энтропии марковской цепи]
Приведем еще одну модель генерирования слов над алфавитом $A = \{a_1,...,a_m\}$. Текст моделируется стационарной конечной цепью Маркова, порождающей слова вида  $X_1,\ldots,X_n \in \{A\}^{n}$.   
Вероятность появления $j$-й буквы зависит только от того, какая буква стоит перед ней: $\PR(X_k = a_j| X_{k-1} = a_i) = p_{ij} > 0$. Стационарность означает, что $\forall k: \; P(X_k = a_j) = p_j$, причем, пользуясь формулой условной вероятности, $p_j$ можно представить как $p_j = \sum_{i=1}^m p_i p_{ij}$. 

Энтропия $X_k$ при фиксированной $(k-1)$-й букве определяется как $H(X_k|X_{k-1} = a_i) = -\sum_{j = 1}^m p_{ij}\log{p_{ij}}$, а условная энтропия $X_k$ при условии, что $X_{k-1}$ станет известным перед генерацией $X_{k-1}$, определяется  как $H = H(X_k|X_{k-1}) = -\sum_{i=1}^m p_i \sum_{j=1}^m p_{ij}\log{p_{ij}}$. Энтропия всей цепочки случайных величин в силу марковского свойства равна
\[
H(X_1,..., X_n) = H(X_1) + H(X_2|X_{1}) + \ldots + H(X_n|X_{n-1}) \sim n H.
\]

Пусть $W_n = (X_1,..., X_n)$ -- некоторое слово, порождаемое описанной моделью, докажите что:
\begin{enumerate}

\item $\frac{-\log \PR(W_n)}{n} \overset{p}{\longrightarrow} H$, т.е. все слова $W_n$  могут быть разбиты на два множества: для первого множества \textit{типичных} слов $|\frac{-\log P(W_n)}{n} - H| < \delta(n)$, для второго -- сумма вероятностей элементов сходится к 0 при $n\rightarrow \infty$;

\item Обозначим через $M_\alpha(n)$ максимальное количество значений $W_n$ с суммарной вероятностью не более $\alpha$. Докажите, что
\[
\forall \alpha: \;  \frac{M_\alpha(n)}{n} \to H, \quad n \to \infty.
\]

\end{enumerate}
\end{problem}

\begin{remark}
Пусть длина алфавита $m = 2^N$, а длина слова -- $n$. Пункт б) утверждает, что найдется код, с помощью которого с высокой вероятностью исходное сообщение $W_n$ может быть передано в $N/H \geq 1$ раз более коротким сообщением, чем при кодировании при помощи двоичных слов, когда каждому слову $W_n$ ставится в однозначное соответствие двоичная цепочка длины $2^n = 2^{nN}$.
\end{remark}


\begin{comment}
Написать о роли условной энтропии в оценке энтропии марковского источника. Эмпирические оценки распределения МЦ и "штраф" за аппроксимацию, равный расстояюни КЛ.

НАМНОГО ПОДРОБНЕЕ СТОИТ РАСПИСАТЬ ЗАДАЧУ про Оценка энтропии марковской цепи

ЕСЛИ ЧЕСТНО, Я ПЛОХО ПОНИМАЮ СМЫСЛ СЛЕДУЮЩЕЙ ЗАДАЧИ:(

\begin{problem}[Оценка энтропии русского языка]
Рассмотрим алфавит $A$ состоящий из $34$х букв: $33$ буквы русского алфавита и пробел. На каждом шаге игроку необходимо угадать следующую букву текста, которая будет открыта, при условии, что он видит все буквы, открытые ранее. За каждую правильную догадку игрок получает $34$ рубля. Предложите стратегию, позволяющую оценить снизу энтропию русского языка.
\begin{ordre} Необходимо на каждом шаге выбирать ту букву, появление которой наиболее вероятно с учетом предыдущей информации. Тогда на шаге $n$ выигрыш может быть записан как $S_n = (34)^n \hat{p}(X_1...X_n)$, где $\hat{p}(X_1,...,X_n) = \sum_{a \in A} \hat{p}(a|x_{n-1}...x_1)$. Показать, что $\mathbb{E}\frac{1}{n}S_n \leq \log(34) + H(X)$, $H(X)$ - энтропия русского языка.
\end{ordre}   
\end{problem}


\begin{problem}[Основная теорема теории кодирования Шеннона]
Пусть буква $X$ --- дискретная с.в., принимающая значения из алфавита $(x_1,\ldots,x_m)$ с вероятностями $(p_1,\ldots,p_m)$. 
Имеется случайный текст из $n$ букв $X$ (предполагается, что буквы в тексте независимы друг от друга). Общее количество таких 
текстов $2^{n\log m}$. Поэтому можно закодировать все эти слова, используя $n\log m$ бит. Однако, используя то обстоятельство, что 
$(p_1,\ldots,p_m)$ --- в общем случае неравномерное распределение, предложите лучший способ кодирования, основанный 
на усиленном законе больших чисел.
\end{problem}

\begin{ordre}
Пусть $\Omega=\{ \omega:\; \omega=(X_1,X_2,\ldots, X_n),\, X_i\in 1,2,\ldots,m\}$ --- пространство элементарных исходов. 
Вероятность появления слова $\omega=(X_1,X_2,\ldots, X_n)$ равна $p(\omega)=p_{X_1}\cdot\ldots\cdot p_{X_n}$. По теореме Колмогорова об 
у.з.б.ч. 
$$
-\frac{1}{n}\log p(\omega)=-\frac{1}{n}\sum\limits_{i=1}^{n}\log p_{X_i} \xrightarrow{\text{ п.н. }} 
-{\mathbb E}p(\omega)=-\sum\limits_{i=1}^{m}p_i\log p_i=H(p) 
$$
В частности, $\frac{S_n}{n}\xrightarrow{P}H(p)$, где $S_n=-\log p(\omega)=-\sum\limits_{i=1}^{n}\log p_{X_i}$. Это можно записать в виде 

$$
{\mathbb P}\Bigl( \Bigl| \frac{S_n-nH(p)}{n}\Bigr|>\delta \Bigr)  \xrightarrow{n\to\infty} 0 
$$
\end{ordre}
\end{comment}

\begin{comment}
\begin{remark}
О связи совместная информации $I(x,y)$ и пропускной способности канала.  
\end{remark}

\begin{problem} \textit{Случайные коды.}
\end{problem}


\begin{remark} Стоит ли писать что на этом подходе основывается построение кодов оптимальной длины?
\end{remark}

\end{comment}

\begin{problem}[Неравенство Пинскера] 
\label{KL}
\textit{Относительной энтропией} двух распределений $P$ и $Q$ на множестве $A$ (или \textit{расстоянием Кульбака--Лейблера} между ними) называется
$\mathcal{KL}(P||Q) = \sum_{a \in A} p_a \log \frac{p_a}{q_a}$. 
\textit{Расстоянием по вариации} между двумя распределениями называется $||\mathbb{P}_1 - \mathbb{P}_2||_1 = \sum_{a \in A} |\mathbb{P}_1(a) - \mathbb{P}_2(a)|$. Доказать, что между ним и расстоянием Кульбака--Лейблера справедливо следующее соотношение:
$$\mathcal{KL}(\mathbb{P}_1||\mathbb{P}_2) \geq \frac{1}{2\ln2} ||\mathbb{P}_1 - \mathbb{P}_2||_1^2.$$
\end{problem}

\begin{problem}
%[Критическая размерность] 

 В уездном городе \textit{M} хотят опросить население с целью восстановления матрицы трудовых корреспонденций. Город разделен на $l\gg 1$ районов. Таким образом, число различных пар (место жительство)--(место работы) равно $m = l^2$. Именно эти пропорции (какая доля людей в городе $p_k$ соответствует корреспонденции с номером $k$): $p_k$, $k = 1,...,m$ и нужно определить, опрашивая случайно выбранных жителей города о том где они живут и работают. Используя формулу Стирлинга (для мультиномиального распределения) и неравенство Пинскера, определите какое количество людей достаточно опросить (постарайтесь оценить это число как можно точнее -- опросы стоят денег), чтобы имело место неравенство 
 $$\PR\left(\sum _{k=1}^{m}\left|\frac{n_{k} }{n} -p_{k} \right| \ge 0.05\right) = \PR\left(\frac{1}{n}\|\vec {n} - n \vec{p} \|_{1} \ge 0.05\right)\le 0.05,$$
Покажите, что справедливо аналогичное неравенство для $l_2$ нормы:
 \[
 \PR\left(\frac{1}{n}\|\vec {n} - n \vec{p} \|_{2} \ge \sqrt{\frac{8x}{n}}\right)
\le e^{-x}. 
 \]
 \begin{ordre}
 В первом неравенстве воспользуйтесь неравенствами концентрации меры из раздела \ref{measure}. Для второго неравенства докажите неравенство Хефдинга в Гильбертовом пространстве: пусть $X_1,\ldots,X_n$ -- независимые случайные вектора, причем $\Exp X_i = 0$, $\Vert X_i \Vert_2 < c/2$, $v = nc^2/4$, тогда при $t \ge \sqrt{v}$
 \[
 \PR\left(\left\Vert \sum_{i=1}^{n} X_i \right\Vert > t
 \right)  \le e^{-(t - \sqrt{v}) / 2v}.
 \]    
 Воспользуйтесь неравенством МакДиармида  и следующим свойством нормы
 \[
 \Exp \left\Vert \sum_{i=1}^{n} X_i \right\Vert \le  \sqrt{  \sum_{i=1}^{n} \Exp \Vert X_i \Vert^2   }.
 \]
 \end{ordre}

\begin{comment}
б) (Спокойный--Клочков) * Решите эту же задачу используя другие нормы: $l_2$, $l_\infty$, расстояние $KL$, используя другую нормировку $\left|n_{k}/(p_{k}n) - 1 \right|$, используя другие неравенства: Бернштейна, Буске, Спокойного (см. раздел 6); используя многомерную ц.п.т. и оценки скорости сходимости в ней (В.В. Сенатов). Рассматрите разные случаи: когда все $p_k$ приблизительно одного порядка и когда это не так. Сравните ответы. Как зависит минимально возможное значение $n = \sum _{k=1}^{m}n_{k}$ от $m$? В каком случае достаточно, чтобы $n\gg m^{3/2}$, в каком случае этого условия не достаточно? Чем чревато нарушение этого условия и что можно сделать, если мы не можем его обеспечить? Всегда ли достаточно, чтобы $n \gg m^3$? 
\end{comment}
\end{problem}


\begin{problem}
Для двух монеток, симметричной ($p = \frac{1}{2}$) и неправильной ($q = \frac{1 +\varepsilon}{2}$), требуется выяснить какая из них является симметричной. Покажите, что для выявления симметричной монетки потребуется $\Omega (1 / \varepsilon^2)$ бросаний. Рассмотрите также случай $p = \varepsilon$, $q = 2 \varepsilon$. В каком случае потребуется больше бросаний? 

\end{problem}

\begin{ordre}
Пусть $f(X_1,\ldots, X_n) \in [0, M]$ является некоторой статистикой, которую можно подсчитать для каждой из монеток и по разности значений идентифицировать монетки. Покажите, что 
\[
\bigg\vert \Exp_{ \mathbb{Q} }[ f(X_1,\ldots, X_n) ] - \Exp_{ \mathbb{P} }[ f(Y_1,\ldots, Y_n) ]  \bigg\vert 
\leq  M \Vert \mathbb{Q} - \mathbb{P} \Vert_1,
\]   
где $(X_1,\ldots, X_n) \in \mathbb{Q}$, $(Y_1,\ldots, Y_n) \in \mathbb{P}$. Далее, для оценки $ \Vert \mathbb{Q} - \mathbb{P} \Vert_1$ можно воспользоваться неравенством Пинскера (см. задачу \ref{KL}), а также цепным правилом для $\mathcal{KL}$ дивергениции (см. указание к задаче \ref{bandit_lower}).
\end{ordre}



\begin{comment}
\begin{remark} Необходимость построения нижних оценок возникает тогда, когда случайная последовательность генерируется одним из распределений $\{P_i\}$ (неизвестно каким), а наилучшая стратегия игрока зависит от вида истинного $P_k \in \{P_i\}$, о котором нет никакой информации  . 
\end{remark}


\begin{problem}[Многорукие бандиты--1]
\end{problem}

ЗАДАЧА ПРО МНОГОРУКИХ БАНДИТОВ НА ОЦЕНКУ СНИЗУ ПО 5-й ЛЕКЦИИИ Yishay Mansour (в рамках курса Advanced Topics in Machine Learning and Algorithmic Game Theory), в которой в двух местах есть опечатка вместо + там стоит -
\end{comment}
\begin{problem}$^*$
Есть $N$ ручек, с каждой из которых связана вероятность успеха $p_k$ (дергая $k$-ю ручку с вероятностью $p_k$ мы получим 1, а с вероятностью $1-p_k$ ничего).
Таким образом, выигрыш игрока при выборе $k$-й ручки есть $r_k \in \text{Be}(p_k)$.    
Вероятности $p_k$ не известны игроку. Игрок намеревается выполнить $T \gg N$ дерганий ручек. При выборе ручки на новом шаге можно использовать всю предысторию. Предложите стратегию, ``максимально близкую'' в среднем по размеру выигрыша к величине $T \max \limits_k p_k$.

\begin{remark} См. монографию Lugoshi G., Cesa-Bianchi N. Prediction, learning and games. New York: Cambridge University Press, 2006, а также S. Bubeck, N. Cesa-Bianchi. Regret Analysis of Stochastic and Nonstochastic Multi-armed Bandit Problems. In Foundations and Trends in Machine Learning, Vol 5, 2012. 
\end{remark}
\end{problem}


\begin{problem}[Многорукие бандиты]
\label{bandit}
В модели \textit{стохастического многорукого бандита} имеется $K$ распределений $f_1, \ldots,f_K$ на множестве $\Omega = [0, 1]$, каждое из которых является распределением выигрыша в зависимости от выбора действия $a \in \{1,\ldots,K\}$. 
Игра повторяется $T \gg K$ раундов, причем на $t$-м раунде
\begin{enumerate}
\item игрок выбирает действие $A(t)$, исходя из результатов предыдущих раундов;
\item среда генерирует выигрыш $r_t$ из распределения  $f_{A(t)}$, независимо от предыдущих раундов.
\end{enumerate}   

Пусть $m_k = \Exp_{f_k} (r)$, $m^{*} = \max_{k} m_k$, $\triangle_k = m^{*} - m_k$  
Цель игрока состоит в минимизации следующей функции потерь
\[
\overline{R}_T = T m^{*} - \sum_{t=1}^{T} \Exp m_{A(t)} = \sum_{k=1}^{K} \triangle_k \Exp A_k(T),
\]
где $A_k(T)$ -- количество действий с номером $k$ за $T$ раундов.
Опишем используемую  стратегию игры (то есть алгоритм выбора действий~$A$).
В каждом раунде будем вычислять доверительные интервалы для оценок $\widehat{m}_1, \ldots, \widehat{m}_K$ и выбирать в следующем раунде действие, соответствующее максимальному значению верхней границы доверительного интервала.

Используя неравенство Маркова в экспоненциальной форме (метод Чернова), покажите 
что согласно описанной стратегии 
\[
A(t+1) \in \mathop{\arg\max} \limits_{k \in \{1,\ldots,K\}} \left[ 
\widehat{m}_{k, A_k(t)} + g^{-1} \left(\frac{\alpha \ln t}{ A(t)} \right)
\right],
\]  
при уровне значимости для доверительного интервала $1 / t^{\alpha}$, где $g$ определяется из условия 
\[
\Exp e^{\lambda |r - \Exp_{f_k} (r)|} \leq  \psi(\lambda), 
\quad
g(x) = \psi^{*}(x) = \sup_{\lambda} (\lambda x - \psi(\lambda)).  
\] 

\end{problem}


\begin{problem}
Докажите верхнюю оценку для стратегии, предложенной в задаче 
\ref{bandit}:
\[
\overline{R}_T \leq \sum_{k: \triangle_k > 0} \left(
\frac{\alpha \triangle_k}{ g(\triangle_k / 2) } \ln (T	) + \frac{\alpha}{\alpha - 2}
\right).
\]
Также установите для случая $f_k = \Be(p_k)$, что для любой стратегии $A$, при выполнении условия $\forall k: \triangle_k > 0 \to \Exp A_k(T) = o(T^{\gamma})$, $\gamma > 0$, справедлива нижняя оценка
\[
\mathop{\underline{\lim}}\limits_{T \to \infty} \frac{\overline{R}_T}{\ln(T)} \geq 
\sum_{k: \triangle_k > 0} 
\frac{\triangle_k}{\KL(m_k, m^{*}) }.
\]
\end{problem}

\begin{ordre} См. S. Bubeck, N. Cesa-Bianchi. Regret Analysis of Stochastic and Nonstochastic Multi-armed Bandit Problems. In Foundations and Trends in Machine Learning, Vol 5, 2012. 
\end{ordre}


\begin{remark}
Последняя оценка говорит об асимптотической неулучшаемости стратегии из задачи \ref{bandit}. Для сравнения приведенных оценок можно воспользоваться неравенством
\[
\KL(m_k, m^{*}) = \KL(p_k, p^{*})  \leq \frac{(p_k - p^{*})^2}{p^{*} (1- p^{*})} =   \frac{\triangle_k^2}{p^{*} (1- p^{*})}.
\] 
\end{remark}


\begin{problem}
\label{bandit_lower}
Пусть параметры распределений $f_k = \Be(p_k)$:  $p_1,\ldots,p_K$ в модели из задачи \ref{bandit} принадлежат распределению $F$. Докажите, что сушествует такое $F$, что для любого детерминированного алгоритма игрока $A$ выполнено неравенство
\[
T \max \limits_k p_k - \sum_{t = 1}^{T} \Exp(r_t | A) \geq \frac{1}{20} \min(\sqrt{KT}, T) = \Omega (\sqrt{KT}).
\]
\end{problem}

\begin{ordre}
В качестве $F$ возьмите следуюшее распределение: все $p_k = 0.5$, за исключением одного $p_i = 0.5 + \varepsilon$, $i$ выбрано случайно равновероятно из множества $\{1,\ldots,K\}$. Установите тождество
\[
\Exp(r_t | A) = \frac{1}{2} + \frac{\varepsilon}{K} \sum_{j=1}^K \sum_{r} \big[A(t,r) = j \big] \PR(r|A, j=i),
\]
где $r \in \{0,1\}^T$, $[\cdot]$ -- индикаторная функция. Далее, воспользуйтесь 
соотношением
\[
\sum_{r} \big[A(t,r) = j \big] \PR_i(r) \leq \sum_{r} \big[A(t,r) = j \big] \PR_u(r) + \Vert \PR_i - \PR_u \Vert_1,
\]
где $\PR_i(r) = \PR(r|A, j=i)$, $\PR_u(r)$ -- распределение выигрышей при $p_1 = \ldots = p_K = 0.5$. Воспользуйтесь неравенством Пинскера и цепным разложением величины 
\[
\mathcal{KL}(p(x,y),q(x,y)) = \mathcal{KL}(p(x),q(x)) + \sum_{x} p(x) \mathcal{KL}(p(y|x),q(y|x)). 
\]

\end{ordre}


\begin{problem}
Пусть казино делает $n$ бросаний, используя распределение вероятностей на бинарных словах длины $n$ -- $p\left(x\right)$, где $x \in \left\{0,1\right\}^{n} $, известное игроку. При этом казино производит выплаты так, как если бы оно использовало распределение $q\left(x\right)$ (то есть выигранная ставка на 0, после выпадения последовательности исходов $x$ увеличивается в $\frac{q\left(x\right)}{q\left(x :+ 0\right)} $ раз, выигранная ставка на 1 -- увеличивается в $\frac{q\left(x\right)}{q\left(x :+ 1\right)} $ раз). Докажите, что у игрока есть стратегия, логарифм значения капитала которой равен расстоянию Кульбака--Лейблера (см. задачу \ref{KL})  между распределениями $p$ и $q$.
\end{problem}

\begin{comment}
ЗАМЕЧАНИЕ ВЕРЕЩАГИН И ЩЕПИН
\begin{problem} \textit{Неравенство больших уклонений}
Рассмотрим последовательность незовисимых одинаково распределенных случайных величин $X_1, ..., X_n$, $X_i~Be(q)$. Доказать, что $-\frac{1}{n}\log \mathbb{P}(\frac{1}{n}\sum X_i \geq p) \rightarrow p\log \frac{p}{q} + (1-p)\log \frac{1-p}{1-q} = KL((p, 1-p)||(q, 1-q))$.
\begin{ordre}
\end{ordre}
\end{problem}

\begin{remark} 
Этот результат является следствием теоремы Санова, которая позволяет оценивать вероятности больших уклонений.. Приведем ниже её упрощенную формулировку.\\
\textit{Теорема Санова}
\end{remark}

\end{comment}


\begin{problem} [Теорема Шеннона о пропускной способности канала с шумом] 

Канал (поток) связи с шумом описывается матрицей переходных вероятностей $p(Y|X)$, где $X$ и $Y$ -- случайные величины с распределениями $P$ (на множестве $A$ -- входной алфавит) и $Q$ (на множестве $B$ -- выходной алфавит) соответственно. Другими словами, $p(y|x)$ -- вероятность прочитать  символ $y$ из потока при условии, что в него был записан символ $x$.
Определим \textit{Шенноновское количество информации} как $I(P, Q) = H(P) + H(Q) - H(P,Q)$. Пропускной способностью такого канала называется величина $C= \max_{\{p_x\}}I(P;Q)$.
Пусть по каналу передаются слова $w_1,...,w_N$ длины $n$.
Разобьем множество кодовых слов в алфавите $Y$ на непересекающиеся области $V_0,...,V_N$. Если принятое слово $y \in V_j$, $j = \overline{1,N}$, то принимается решение о том, что было послано слово $w_j$. Если $y \in V_0$ то никакое определенное решение не принято. Введем \textit{среднюю вероятность ошибки} $$\overline{P}_{\varepsilon}(W, V) = \frac{1}{N} \sum_{i=1}^n (1 - p(V_i|w_i)).$$
Пусть $p_{\varepsilon}(n, N) = \min_{W, V} \overline{P}_{\varepsilon}(W, V)$, докажите что:
\begin{enumerate}
\item[а)] $p_{\varepsilon}(n, 2^{nR}) \rightarrow 0$, $R < C$;
\item[б)] $p_{\varepsilon}(n, 2^{nR}) \not\rightarrow 0$, $R > C$;
\item[в)] $p_{\varepsilon}(n, 2^{nR}) \rightarrow 1$, $R > C$,
\end{enumerate}
где $R = \frac{\log N}{n}$ -- скорость передачи.
\begin{remark}
Величина $H(P,Q) = -\sum_{a\in A} p(a)\sum_{b \in B}p(a, b)\log p(a, b)$ называется \textit{совместной энтропией} двух случайных величин.
\end{remark}
\end{problem} 
\begin{comment}
В ЗАМЕЧАНИИ ХОТЕЛОСЬ БЫ ВИДЕТЬ И СВЯЗЬ С ЗАДАЧЕЙ ПРО ГРАНИЦУ ГИЛБЕРТА, ГДЕ ИСПОЛЬЗУЕТС ЯСЛУЧАЙНОЕ КОДИРОВАНИЕ
ПОЛЕЗНО ЗДЕСЬ ЖЕ ОБЫГРАТЬ СТР, 133 теорему 10.11 Верещагин Щепин

Вообще стоит подробнее в заадчах осветить главу 11 этой книги!
\end{comment}




\section{Вероятностный метод в комбинаторике}
\label{combinatorics}

\begin{problem}
Поверхность некоторой шарообразной планеты состоит из океана и суши (множество мелких островков). Суша занимает больше половины 
площади планеты. Также известно, что суша есть множество, принадлежащее борелевской  $\sigma$-алгебре на сфере. На планету хочет 
совершить посадку космический корабль, сконструированный так, что концы всех шести его ножек лежат на поверхности планеты. 
Посадка окажется успешной, если не меньше четырех ножек из шести окажутся на суши. Возможна ли успешная посадка корабля на планету?
\end{problem}

\begin{ordre}
Введем индикаторную функцию для одной посадки
\[ \xi_i = 
\begin{cases}
1, & i\text{-я ножка оказалась на суше,}\\
0, & \text{иначе.}
\end{cases}
\]
Тогда число ножек, оказавшихся на суше есть $\xi = \sum \limits_{i=1}^6 \xi_i$.
Покажите, что   $ \Exp \xi > 3$ (усреднение берется по всем возможным посадкам). Значит существует посадка, для которой   $\xi  > 3$, то есть успешная.
\end{ordre}

\begin{problem} 
Пусть $n\ge 2k$ и семейство $F$ является пересекающимся семейством $k$-элементных подмножеств множества $\left\{0,\ldots ,n-1\right\}$, то есть для любых двух множеств $A,B\in F$ выполняется условие $A\cap B\ne \emptyset $. Найдите с помощью вероятностного метода верхнюю оценку на размер семейства $F$ (а именно, покажите, что $|F|\le C_{n-1}^{k-1} $). Покажите, что эта оценка не улучшаемая.
\end{problem}

\begin{ordre} 
Семейство $F$ может содержать не более $k$ множеств вида $A_{s} =\left\{s,s+1,\ldots ,s+k-1\right\}$ (сумма берется по модулю $n$), $0\le s\le n-1$.


Пусть $\sigma $ -- случайная перестановка на множестве $\left\{0,\ldots ,n-1\right\}$ и $i$ -- случайное число из множества $\left\{0,\ldots ,n-1\right\}$. Пусть $A=\left\{\sigma (i),\sigma (i+1),\ldots ,\sigma (i+k-1)\right\}$ (сумма берется по модулю~$n$). Покажите, что с одной стороны (согласно доказанному выше утверждению) $\PR\left[A\in F\right]\le \frac{k}{n} $, с другой стороны (с учетом равновероятности выбора $A$ из всех $k$-множеств) $\PR\left[A\in F\right]=\frac{|F|}{C_{n}^{k} } $.


\end{ordre} 
\begin{remark}
К этой и последующим задачам этого раздела можно рекомендовать книгу \cite{15}.
\end{remark}



\begin{problem}[Задача Рамсея]  Докажите, что для произвольного графа $G=(V,E)$ всегда можно 
раскрасить вершины в два цвета таким образом, чтобы не менее половины рёбер 
были ``разноцветными'', то есть соединяли вершины разного цвета.
\end{problem}
\begin{ordre}
Вычислите математическое ожидание числа ``разноцветных'' ребер для случайной раскраски вершин графа в два цвета.
\end{ordre}
\begin{remark}
О применении вероятностного подхода к комбинаторике и теории графов рекомендуется посмотреть также книги

Айгнер М., Циглер Г. "Доказательства из Книги. Лучшие доказательства со времен Евклида до наших дней". -- М.: Мир. -- 2006. -- 256 с.

Эссе  Gowers W. T. "The Two Cultures of Mathematics" в сборнике статей под ред. Д.В.Аносова и А.Н.Паршина. М.: Фазис, 2005.
\end{remark}
\begin{comment}
ЛЕНА, К ЭТОЙ ЗАДАЧЕ СТОИТ ДАТЬ УКАЗАНИЕ С ИДЕЕЙ РЕШЕНИЯ из Gowers'a
\end{comment}


\begin{problem}
На турнир приехало $n$ игроков. Каждая пара игроков, согласно регламенту турнира, должна провести одну встречу (ничьих быть не может). Пусть 
$$
C_n^k\cdot (1-2^{-k})^{n-k}<1 . 
$$
Докажите, что тогда игроки могли сыграть так, что для каждого множества из $k$ игроков найдется игрок, который побеждает их всех. 

\end{problem}

\begin{ordre}
Введем $A_K$ --- событие, состоящее в том, что не существует игрока, побеждающего всех игроков из множества $K$. 
Докажите, что 
$$
{\mathbb P}\bigl(\bigcup\limits_{K\subset\{1,..,n\},|K|=k} A_K \bigr)\leqslant C_n^k\cdot (1-2^{-k})^{n-k} . 
$$

\end{ordre}




\begin{problem}
Рассмотрим матрицу $n\times n$, составленную из лампочек, каждая из которых либо включена $(a_{ij}=1)$, либо выключена $(a_{ij}=-1)$. 
Предположим, что для каждой строки и каждого столбца имеется переключатель, поворот которого ($x_i=-1$ для строки $i$ и 
$y_j=-1$ для столбца $j$) переключает все лампочки в соответствующей линии: с <<вкл.>> на <<выкл.>> и с <<выкл.>> на <<вкл.>>. 
Тогда для любой начальной конфигурации лампочек можно установить такое положение переключателей, что разность между числом включенных и 
выключенных лампочек будет не меньше $(\sqrt{2/\pi}+o(1))n^{3/2}$. 
\end{problem}

\begin{ordre}
Рассмотрите  переключатель по столбцам как случайную величину, принимающую с равной вероятностью значения $1$, $-1$. Каждому переключателю по столбцам необходимо подобрать переключатель по строкам, максимизирующий разность включенных и 
выключенных лампочек. Распределение данной разности можно оценить при помощи ц.п.т.       
\end{ordre}


\begin{problem}
Назовем \textit{турниром} ориентированный граф $T=(V,E)$ такой, что $(x,x)\notin E$ для любой вершины $x\in V$, а для любых двух различных вершин $x\ne y$, $x,y\in V$ либо $(x,y)\in E$, либо $(y,x)\in E$. Множество вершин назовем игроками, каждая пара игроков ровно один раз встречаются на матче, если игрок $x$ выигрывает у игрока $y$, то $(x,y)\in E$. Гамильтоновым путем графа назовем перестановку вершин $(x_{1} ,x_{2} ,\ldots ,x_{n} )$, что для всех $i$ игрок $x_{i} $ выигрывает у $x_{i+1} $. Несложно показать, что любой турнир содержит гамильтонов путь. Покажите, что найдется такой турнир на $n$ вершинах, для которого число гамильтоновых путей не меньше чем $n!/2^{n-1}$.
\end{problem}

\begin{ordre}

Рассмотрите случайный турнир (направление каждого ребра выбирается независимо от других с вероятностью $1/2$). Пусть $X$ -- число гамильтоновых путей в случайном турнире. Для каждой перестановки $\pi $ обозначим за $X_{\pi } $ индикаторную с.в. события, что гамильтонов путь соответствующей этой перестановке содержится в случайном турнире. Представьте $X$ в виде суммы таких индикаторных с.в. и, воспользовавшись линейностью математического ожидания, получите, что $\mathbb E X=n!/2^{n-1}$.
\end{ordre}


\begin{problem}
Дано $k$ перестановок натуральных чисел от 1 до $n$, $n>100$. Оказалось, что этот набор перестановок -- минимальный (по количеству перестановок), обладающий следующим свойством: для любых десяти чисел от 1 до $n$ любую их перестановку можно получить вычеркиванием оcтальных чисел из одной из данных. Докажите, что $\ln n \leq k \leq 10^{100} \ln n$
\end{problem}

\begin{remark}
Эта задача, а также последующие девять задачи от Федора Петрова (ПОМИ РАН).
\end{remark}

\begin{problem}
Докажите, что числа от 1 до $2^n$ можно покрасить в два цвета так, чтобы не было арифметической прогрессии длины $2n$ одного цвета.
\end{problem}

\begin{problem}
На столе лежат $n$ монет орлами вверх. Каждую минуту Вася равновероятно выбирает одну из монет  и переворачивает ее. Докажите, что вероятность того, что через $k$ минут все монет
будут лежать решками вверх, не превосходит $2^{1-n}$.
\end{problem}


\begin{problem}
\begin{enumerate}
\item В алфавите племени УАУАУА только две буквы,
причем никакое слово их языка не является началом другого слова.
Докажите, что $\sum N_i2^{-i}\leq 1$,
где $N_i$ --- количество слов длины $i$ в этом языке.
\item В алфавите племени ОЕЕ только две буквы. Люди
этого племени записывают
предложения без пробелов и это никогда
не приводит к двусмысленности (то есть для
любой конечной последовательности букв есть не более одного способа
разбить их на слова). Докажите, что
$\sum N_i2^{-i}\leq 1$,
где $N_i$ --- количество слов длины $i$ в этом языке.
\end{enumerate}
\end{problem}

\begin{problem}
В таблице $n\times n$ расставлены различные числа.
Докажите, что можно так переставить ее строки, что
ни в одном столбце не будет возрастающей (сверху вниз) последовательности длины $\geq 100 \sqrt{n}$.
\end{problem}

\begin{problem}
В однокруговом волейбольном турнире участвовала тысяча команд.
Всегда ли можно выбрать 21 команду и пронумеровать их так,
чтобы в любой паре из этих команд победила та, номер которой больше?
\end{problem}

\begin{problem}
В двудольном графе меньше, чем $2^n$ вершин, и в каждой
имеется список из $n$ цветов. Докажите, что можно
покрасить каждую вершину в один из цветов ее списка
так, чтобы смежные вершины были разных цветов.
\end{problem}

\begin{problem}
\begin{enumerate}
\item В компании из $n$ человек некоторые пары
дружат, а некоторые другие враждуют, при этом у каждого не более
пяти врагов. Известно, что в любом множестве людей, среди которых нет пар врагов, имеется не более чем
$k$ пар друзей. Докажите, что общее количество пар друзей не превосходит $2^{2011}k$.
\item То же, если вражда (в отличие от дружбы) -- не обязательно симметричное отношение: каждый человек неприязненно относится не более чем к пятерым, и в любом множестве людей, среди которых никто ни к кому не относится неприязненно, не не более чем $k$ пар друзей.
\end{enumerate}
\end{problem}

\begin{problem}
На множестве положительных
чисел задано некоторое вероятностное распределение $X$.
Из одной и той же точки плоскости начинают прыгать две лягушки, каждая
из которых выбирает длину своего прыжка случайно
согласно распределению $X$, а направление случайно
и равномерно. (Направление и длина каждого прыжка независимы,
так же независимы разные прыжки и поведения лягушек).
Первая лягушка сделала $n$ прыжков, а вторая $m$
(где $m,n>0$ и $m+n>2$).
Докажите, что вероятность того, что первая лягушка дальше
от исходной точки, чем вторая, равна $n/(n+m)$.
\end{problem}


\begin{problem}
Несколько мальчиков ``раскидывают на морского'', кому водить в игре. Для этого каждый из них одновременно с другими
``выбрасывает'' на пальцах число от 0 до 5. Числа складываются и сумма отсчитывается по кругу, начиная с заранее
выбранного мальчика (ему соответствует
ноль). Водить будет тот, на ком остановится счет.
При каком числе мальчиков этот метод является справедливым, то есть
вероятность водить одинакова у всех мальчиков?
\end{problem}




\begin{problem}
\label{triangles}
Рассматривается случайный граф $G(n,p)$ (модель Эрдеша--Реньи). Случайная величина $X$ равна числу треугольников в графе. Покажите, что: 

\begin{enumerate}

\item Если $p(n) \ll  n^{-1} $, то граф $G$ почти всегда свободен от треугольников, то есть $\mathop{\lim }\limits_{n\to \infty } \PR (
X > 0)= 0$;

\item Если $p(n)\gg n^{-1} $, то граф $G$ почти всегда содержит треугольник, то есть $\mathop{\lim }\limits_{n\to \infty } \PR (X = 0)=0$.
\end{enumerate}

Говорят, что пороговая функция свойства ``граф  свободен от треугольников'' графа $G(n,p)$ равна $n^{-1} $.

\end{problem}

\begin{ordre}

Для случайной величины $X \geq 0$ справедливы неравенства 

\begin{enumerate}

\item $\PR( X>0) \le \Exp X$, 

\item $\PR( X=0) \le \PR( |X- \Exp X|\ge \Exp X)$. 

\end{enumerate}

\noindent Введем событие $B_{S}$ -- ``$S$ является треугольником''. Тогда 
\[
X=\sum _{|S|=3} \I [B_{S}] = \sum _{|S|=3} X_S,
\]
где $ X_S = \I [B_{S}]$.
Дисперсию зависимых индикаторов предлагается оценивать неравенством  
\[
\Var X\leq  \Exp X+\sum \cov\left( X_{S_1} , X_{S_2} \right), 
\] 
здесь суммирование ведется по всем упорядоченным зависимым парам различных трехэлементных множеств. 

\[
\sum \cov\left( X_{S_1} , X_{S_2} \right) \leq \sum \PR \left( B_{S_1} , B_{S_2} \right)  =
\]\[
\sum _{S_1} \PR ( B_{S_1} )  \sum  \PR (B_{S_2} | B_{S_1} )  = \Exp X\sum _{} \cdot \PR (B_{S_2} | B_{S_1} ).
\] 


\end{ordre}

\begin{problem} [Парадигма Пуассона] 
Рассматривается случайный граф $G\left(n,\frac{c}{n} \right)$ (модель Эрдеша--Реньи), где $c$ -- некоторая константа. С помощью неравенства Янсона (см. замечание) покажите, что случайная величина $X = X(n)$, равная числу треугольников в графе, имеет почти пуассоновское распределение с параметром $\mu=\lim_{n\to\infty}\Exp X = \frac{c^3}{6}$, в частности 
\[ 
\lim_{n\to\infty}\PR (X=0) \sim e^{ - \mu} .
\]

\end{problem}

\begin{remark}
Согласно предыдущей задаче, пороговая функция свойства "граф свободен от треугольников"  равна $n^{-1}$. С учетом обозначений, введенных в предыдущей задаче, \textit{неравенство Янсона} имеет вид:
\[
\prod _{|S|=3} \PR \left( \overline{B}_{S} \right)  \le \PR \left(\mathop{\wedge }\limits_{|S|=3} \overline{B}_{S} \right) \leq e^{-\mu +\frac{\Delta }{2} },
\] 
где $\mu =\sum _{|S|=3} \PR( B_{S} )  $, $\Delta =\sum _{|S\cap T|=2} \PR( B_{S} B_{T} )$.

Заметим, что левое неравенство переходит в равенство для взаимно независимых событий $B_S$. При этом 
$$\prod _{|S|=3} \PR \left( \overline{B}_{S} \right) = \left (1 - \left( \frac{c}{n} \right)^3  \right )^{C_n^3}\to e^{-\frac{c^3}{6}}.$$

На самом деле события $B_S$, $B_T$ зависимы, если $|S\cap T|=2$. Неравенство Янсона дает поправку для "почти независимых" событий через $\Delta=C_n^4C_4^2 \left( \frac {c}{n}\right)^5 = o(1)$. Таким образом, с.в. $X$ -- сумма большого числа индикаторов "почти независимых" событий, имеет почти пуассоновское распределение. 

Более детальный подход к парадигме Пуассона дает метод "решета Бруна" (см. \cite{15}):
$$
\lim_{n\to\infty}\PR (X=k) = \frac{\mu^k}{k!}e^{ -\mu}.
$$
\end{remark}

\begin{problem} 
Покажите, что пороговая функция события: размер максимальной клики $\omega (G)$ в случайном графе $G(n,p)$ не меньше 4 -- равна $n^{-\frac{2}{3} } $.
\end{problem}

\begin{ordre}
См. задачу \ref{triangles}. 
\end{ordre}



\begin{problem}
Покажите, что для каждого целого числа $n$ найдется раскраска ребер полного графа $K_{n} $ в два цвета (синий, красный), при которой число одноцветных подграфов $K_{4} $ не превосходит $C_{n}^{4} 2^{-5} $. Предложите детерминированный алгоритм построения такой раскраски за полиномиальное от $n$ время.
\end{problem}


\begin{ordre}

Зададим весовую функцию $W(K_{n} )$ частично раскрашенного графа $K_{n} $, как $W(K_{n} )=\sum _{}w(K) $, где суммирование ведется по всем копиям $K$ графа $K_{4} $ в $K_{n} $ и вес~$w(K)$ подграфа~$K$ равен вероятности того, что копия $K$ окажется одноцветной в случае, когда все бесцветные ребра графа $K_{n} $ будут случайно и независимо раскрашены в два цвета. 

Произвольным образом упорядочим все $C_{n}^{2} $ ребер графа $K_{n}$ и создадим цикл их перебора.   Цвет очередного ребра  выбирается так, чтобы минимизировать получающийся вес, то есть при $W_{red} \leq W_{blue} $ ребро раскашивается в красный цвет, в противном случае -- в синий. Покажите, что в такой процедуре вес графа $K_{n} $ с течением времени не возрастает. 

\end{ordre}


\begin{problem}
Покажите, что можно так раскрасить в два цвета ребра полного графа с $n$ вершинами (т.е. графа (без петель), в котором любые две 
различные вершины соединены одним ребром), что любой его полный подграф с $m$ вершинами, где 
$2C_n^m (\left.1\right/2)^{C_m^2}<1$, имеет ребра разного цвета. 
\end{problem}





\begin{problem}[Концентрация хроматического числа] 
\label{azuma}

Для произвольных $n$ и $p\in (0, 1)$ покажите, что распределение случайной 
величины, равной $\chi (G)$ -- хроматическому числу графа $G(n,p)$ (случайный 
граф в модели Эрдеша--Реньи) является плотно сконцентрированным около 
среднего значения:

\[
\forall \lambda >0  \quad \PR\left\{ {\vert \chi (G)-\Exp\chi (G)\vert >\lambda 
\sqrt {n-1} } \right\}\le 2e^{-\frac{\lambda ^2}{2}}.
\]

\end{problem}


\begin{ordre}

Воспользуйтесь неравенством Азумы: для мартингальной последовательности $X_0 =c,\ldots ,X_m $, удовлетворяющей условию $\vert X_{i+1} -X_i \vert \le 1$ для всех $0\le i< m$, справедливо 
\[
\PR\left( {\vert X_m -c\vert >\lambda \sqrt m } \right)\le 
2e^{-\frac{\lambda ^2}{2}}.
\]
В качестве такого мартингала можно взять мартингал проявления 
вершин, заданный следующим образом: $X_i = \Exp\left[ {\chi (G)} \vert  G_{1:i}\right]$, $X_i$ -- условное математическое ожидание значения $\chi (G)$ при зафиксированном подграфе с вершинами $1,\ldots,i$.

\end{ordre}

\begin{problem}[Максимальный размер клики]

Для произвольного $n$ покажите, что распределение случайной величины, равной $\omega (G)$ -- максимальному размеру клики графа $G\left( {n,\frac{1}{2}} 
\right)$ (случайный граф в модели Эрдеша--Реньи) задается неравенством:
\[
\PR\left ( {\omega (G)<k} \right) < e^{-(c+o(1))\frac{n^2}{k^8}},
\]
где $c$ -- некоторая положительная константа.

\end{problem}

\begin{ordre}
 Воспользуйтесь неравенством Азумы (см. указание из задачи \ref{azuma}) для мартингала проявления ребер $X_0 ,\ldots ,X_m $ (здесь $m=C_n^2 )$, заданного следующим образом: $X_0 = \Exp \left[ {Y(G)} \right]$, $X_i $ -- условное математическое ожидание значения $Y(G)$, при 
условии, что первые $i$ ребер/пропусков фиксированы, $Y(G)$ -- максимальный размер семейства непересекающихся по ребрам $k$-клик в графе. 
\[
X_0 = \Exp\left[ {Y(G)} \right]\ge \left( {9+o(1)} 
\right)\frac{n^2}{2k^4},
\]
\[
\left\{ \omega (G)<k \right\} 
\Leftrightarrow 
\left\{ Y(G)=0 \right\}
\Leftrightarrow
\left\{ X_m =0 \right\}.
\]
Далее осталось применить неравенство Азумы для 
\[
\PR\left( {X_m =0} \right)\le \PR\left( {X_m -X_0 \le -X_0 } \right).
\]

\end{ordre}

\begin{problem}
Пусть для модели Эрдеша--Реньи случайного графа $G(n,p),\; p=n^{-\alpha }$, где $\alpha $ -- фиксированное, $\alpha > 5/6$. Тогда существует $u=u(n,p)$ такое, что почти всегда 
\[
u\le \chi (G)\le u+3.
\]
\end{problem}


\begin{ordre}
Покажите справедливость следующей технической леммы.

\begin{lemma}
Пусть $\alpha $, $c$ -- фиксированные числа, $\alpha > 5/6$. Пусть 
$p=n^{-\alpha }$. Тогда почти наверное каждые $c \sqrt n $ вершин графа 
$G(n,p)$ могут быть правильно раскрашены в три цвета.
\end{lemma}

Для доказательства леммы предположим противное. Возьмем случайный граф
$G(n,p)$, пусть $T$ -- подмножество (вершин исходного графа) минимального размера, которое нельзя правильно раскрасить в три цвета. Поскольку для всякого $x\in T$ подграф, порожденный множеством $T\backslash \{x\}$, является 3-раскрашиваемым, а подграф, порожденный $T$, не является таковым, $x$ имеет по меньшей мере трех соседей в подграфе, порожденном $T$. То есть если $\vert T\vert =t$, то подграф, порожденный множеством $T$ имеет по меньшей мере $3t/2$ ребер. Вероятность того, что существует такое $T$ с 
$t\le c\sqrt n $, не превосходит $\sum\limits_{t=4}^{c\sqrt n } {C_n^t
C_{C_t^2 }^{\frac{3t}{2}} p^{\frac{3t}{2}}} $. 
Поскольку $C_n^t \le \left( 
{\frac{ne}{t}} \right)^t$ и $C_{C_t^2 }^{\frac{3t}{2}} \le \left( 
{\frac{te}{3}} \right)^{\frac{3t}{2}}$, то 
\[
C_n^t C_{C_t^2 }^{\frac{3t}{2}} 
p^{\frac{3t}{2}}\le \left[ {\frac{ne}{t}\left( {\frac{te}{3}} 
\right)^{\frac{3}{2}}n^{-\frac{3\alpha }{2}}} \right]^t\le \left[ {c_1
n^{1-\frac{3\alpha }{2}}t^{\frac{1}{2}}} \right]^t\le \left[ {c_2 n^{-\left( 
{\frac{3\alpha }{2}-\frac{5}{4}} \right)}} \right]^t
\]
и вероятность заданного события есть $o(1)$ (что и доказывает справедливость леммы).

Далее для произвольного малого $\varepsilon >0$ выберем $u=u(n,p,\varepsilon)$ -- наименьшее целое число, удовлетворяющее неравенству
\[
\PR\left\{ {\chi (G) \leq u} \right\} > \varepsilon.
\]
Далее покажите, что с вероятностью не меньше $1-\varepsilon$ существует $u$-раскраска всех, кроме не более чем $c\sqrt n $ вершин. Для этого воспользуйтесь неравенством Азумы для мартингала проявления вершин с 
теоретико-графовой функцией $Y(G)$, равной минимальному размеру множества 
вершин $S$, для которого граф, индуцированный исходным графом $G$, но без вершин $S$, может быть правильно раскрашен в $u$ цветов:
\[
\begin{array}{l}
 \PR\left\{ {Y\le \Exp Y-\lambda \sqrt {n-1} } \right\}<e^{-\frac{\lambda ^2}{2}}, 
\\ 
 \PR\left\{ {Y\ge \Exp Y+\lambda \sqrt {n-1} } \right\}<e^{-\frac{\lambda ^2}{2}}, 
\\ 
 \end{array}
\]
где $\lambda $ удовлетворяет соотношению $e^{-\frac{\lambda ^2}{2}} < \varepsilon$.
Из определения $u$ имеем $\PR\{Y=0\} > \varepsilon$. Значит $\Exp Y\le \lambda \sqrt {n-1} $ и $\PR\left\{ {Y\ge 2\lambda \sqrt {n-1} } \right\}<\varepsilon$.

\end{ordre}

\begin{problem}[Балансировка векторов]

Пусть ${\rm B}$ -- произвольное нормированное пространство, $v_1 ,\ldots ,v_n $ -- элементы ${\rm B}$, причем $\left\| {v_i } \right\|_2\le 1$. 
Пусть $\varepsilon _1 ,\ldots ,\varepsilon _n $ -- радемахеровские случайные величины, то есть 
независимые с.в. с распределением $\PR\left\{ {\varepsilon _i =+1} 
\right\}=\PR\left\{ {\varepsilon _i =-1} \right\}=1/2$. Положим 
$Y=\left\| {\varepsilon _1 v_1 +\ldots +\varepsilon _n v_n } \right\|_2$.
Покажите справедливость неравенства для произвольного $\lambda >0$
\[
\PR\left\{ {\vert Y-\Exp Y\vert >\lambda \sqrt n } \right\}\le 2e^{-\frac{\lambda 
^2}{2}}.
\]

\end{problem}

\begin{ordre}  
Воспользуйтесь неравенством Азумы для мартингала, 
полученного последовательным проявлением $\varepsilon_i$.
\end{ordre} 




\begin{problem} 
Пусть $\omega (n)\to \infty $ произвольно медленно. Покажите, что число тех $x\in \left\{1,\ldots ,n\right\}$, для которых

\[\left|\nu (x)-\ln (\ln n)\right|>\omega (n)\sqrt{\ln (\ln n)} ,\] 
есть $o(n)$. Здесь $\nu (x)$ -- количество простых чисел $p$, делящих $x$ (без учета кратности).
\end{problem}

\begin{remark} 
Грубо, это утверждение говорит, что ``почти все'' $n$ имеют число простых делителей (без учета кратности) ``в некотором смысле близкое'' к $\ln (\ln n)$.
(См. также задачу 54 из раздела 3)
\end{remark} 

\begin{ordre} 
Пусть $x$ случайно выбирается из множества $\left\{1,\ldots ,n\right\}$. Для простого $p$ положим: 

\[
X_{p} =\left\{\begin{array}{cc} {1,} & {x \; \mbox{ делится на }   \; p ,} \\ {0,} & { x \;  \mbox{ не делится на }\; p .} \end{array}\right. 
\] 
$X=\sum X_{p}  $, где сумма ведется по всем простым $p\le M\equiv n^{0.1} $. Так как никакое $x\le n$ не может иметь более 10 простых делителей, больших $M$, то $\nu (x)-10\le X(x)\le \nu (x)$ (то есть границы больших уклонений для $X$ переходят в асимптотически равные им границы для $\nu $.

Покажите, что математическое ожидание и дисперсия случайной величины $X$ равны $\ln (\ln n)+O(n^{-0.9})$, учтя соотношение $\sum _{p\le x}1/p  =\ln \ln x$, где сумма берется по всем простым $p\le x$.

\end{ordre} 

\begin{remark} 
Справедливо также соотношение
\[
\mathop {\lim }\limits_{n\to \infty } \frac{1}{n}\left| {\left\{ {k\leq n:\;\nu \left( k \right)\ge \ln (\ln n)+\lambda \sqrt {\ln (\ln n)} } 
\right\}} \right|=\frac{1}{\sqrt {2\pi } }\int\limits_\lambda ^\infty 
{e^{-{t^2} \mathord{\left/ {\vphantom {{t^2} 2}} \right. 
\kern-\nulldelimiterspace} 2}dt} .
\]
\end{remark} 


\begin{comment}


\begin{problem}
$V=\left\{ {1,...,m} \right\}$, ${\rm M}=\left\{ {M_1 
,...,M_n } \right\}$, $M_k \subseteq V$.

$\chi :\quad V\to \left\{ {-1,1} \right\}$ (можно интерпретировать, как 
раскраску множества V в два цвета).

$\chi (M_i )=\sum\limits_{a\in M_i } {\chi (a)} $ ($\left| {\chi (M_i )} 
\right|$ отвечает за ``равномерность'' покраски множества $M_i $ в два 
цвета).

$disc({\rm M},\chi )=\mathop {\max }\limits_{i=1..n} \left| {\chi (M_i )} 
\right|$ (от слова discrepancy - уклонение) - мера того, что хотя бы один 
объект в ${\rm M}$ раскрашен ``неравномерно''.

$disc({\rm M})=\mathop {\min }\limits_\chi disc({\rm M},\chi )$(``поиск'' 
наилучшей раскраски).

Показать, что для $\forall n\;\forall m\;\forall {\rm M} \quad disc({\rm M})\le 
\sqrt {2m\ln (2n)} $. Т. е. $\exists \chi :\;disc({\rm M},\chi )\le \sqrt 
{2m\ln (2n)} $.

\end{problem}


\end{comment}




\begin{problem}
Пусть ${\cal M}=\left\{ {M_1 ,\ldots ,M_s } \right\}$ -- совокупность, 
состоящая из различных $k$-сочетаний элементов множества $\left\{ {1,\ldots 
,n} \right\}$. Назовем $S\subset \left\{ {1,\ldots ,n} \right\}$ \textit{системой общих представителей }(с.о.п.) 
для ${\cal M}$, если $S\cap M_i \ne \emptyset $ для всех $i=1,\ldots ,s$. 
Интерес представляет минимальная (по мощности) с.о.п., т.е. та с.о.п., на 
которой достигается минимум: 
\[
\tau \left( {\cal M} \right)=\min \left\{ {\left| S 
\right|:\;S-\mbox{с.о.п.}} \right\}.
\]
Ясно, что минимальная с.о.п. может быть не единственной (т.е. минимум в 
предыдущем выражении достигается не на единственной $S)$. Зафиксируем 
параметры $n,s,k$ и введем искусственно равномерную дискретную вероятностную 
меру на множестве всех совокупностей ${\cal M}$ (в силу того, что при 
фиксированных $n,s,k$ число таких совокупностей ${\cal M}$ конечно).

а) Выберем согласно введенной вероятностной мере случайную совокупность 
${\cal M}$, найдем все возможные минимальные с.о.п. для нее, пусть их 
количество равно $N({\cal M})$ (случайная величина). Найдите математическое 
ожидание $N({\cal M})$, при условии, что $\tau ({\cal M})=l$.

б) Далее будем интересоваться величиной 
\[
\zeta (n,s,k)=\mathop {\max }\limits_{\cal M} \tau ({\cal M}),
\]
где максимум берется по совокупностям ${\cal M}$ с фиксированными 
параметрами $n,s,k$.

Для получения нижней границы на значения величины $\zeta (n,s,k)$ можно 
воспользоваться вероятностным методом. Согласно введенному выше 
вероятностному пространству на множестве всех совокупностей ${\cal M}$ 
рассмотрим случайное событие $A=\left\{ {{\cal M}:\;\tau ({\cal M})\le l} 
\right\}$. Покажите, что 
\[
P(A)\le G(n,s,k,l) = \frac{C_n^l C_{C_n^k -C_{n-l}^k 
}^s }{C_{C_n^k }^s }.
\]
Если параметры $n,s,k,l$, таковы, что $G(n,s,k,l)<1$, то вероятность 
отрицания события $A$ положительна, т.е. существует такая совокупность 
${\cal M}$, для которой $\tau ({\cal M})>l$, а значит и $\zeta (n,s,k)>l$. 

\textbf{Замечание. }Если $l=l(n,s,k)\approx \frac{n}{k}\ln \frac{sk}{n}$ (в 
предположении, что $sk>n)$, то можно показать, что $G(n,s,k,l)\mathop \to 
\limits_{n\to \infty } 0$, а значит ``почти всякая'' совокупность обладает 
огромной по размеру минимальной с.о.п. (т.е. с вероятностью стремящейся к 
единице случайная совокупность ${\cal M}$ имеет $\tau ({\cal M})>l\approx 
\frac{n}{k}\ln \frac{sk}{n})$. На самом деле полученная выше нижняя оценка 
на значения величины $\zeta (n,s,k)$ асимптотически точна. Иными словами 
можно доказать следующую теорему (см. Райгородский А.М. Системы общих 
представителей в комбинаторике и их приложения в геометрии. М.: МЦНМО, 
2009): для любых $n,s,k$ справедливо неравенство
\[
\zeta (n,s,k)\le \max \left\{ {\frac{n}{k},\frac{n}{k}\ln \frac{sk}{n}} 
\right\}+\frac{n}{k}+1.
\]
\end{problem}

\begin{problem}[Локальная лемма Ловаса (ЛЛЛ)]
Орграф зависимостей $(V,D)$ для набора событий $A_{1} ,\ldots A_{t} $ определяется следующим образом:
$V=\{ 1,\ldots ,t\};$ $D$ определяется согласно правилу $(i_{1} ,k),\ldots ,(i_{s} ,k)\notin D \Leftrightarrow$ $A_{k} $ не зависит от группы событий $A_{i_{1} } ,\ldots, A_{i_{s} } $.

Пусть $D$ -- множество дуг орграфа зависимостей набора событий $A_{1} ,\ldots, A_{n} $. Пусть нашлись такие $x_{1} ,\ldots ,x_{n} \in (0,1)$, что $\forall k\in \left\{1,\ldots ,n\right\}$ выполнено неравенство
\[P[A_{k} ]\le x_{k}\cdot \prod _{i:\; (i,k)\notin D}(1-x_{i} ) .\] 
Тогда
\[P\left[\bar{A}_{1} \bigcap \ldots \bigcap \bar{A}_{n} \right]\ge (1-x_{1} )\cdot\ldots\cdot (1-x_{n} )>0.\] 
Примените ЛЛЛ для оценки диагональных чисел Рамсея (см. задачу 3 этого раздела) $R(s,s)>n$, где $n=\left\lfloor 2^{0.5s} \right\rfloor $, то есть покажите, что существует граф на $n$ вершинах, у которого нет ни клик, ни независимых множеств (н.м.) размера $s$. 
\end{problem}

\begin{ordre}
Рассмотрите случайный граф на $n$ вершинах с вероятностью проведения ребра $1/2 $. Для фиксированного подмножества $s$-вершин $U$ определите событие $A_{U} $ -- множество вершин $U$ образует либо клику, либо н.м. Для выбранной модели случайного графа $P[A_{U} ]=2^{1-C_{s}^{2} } $. Заметьте, что на событие $A_{U} $ влияют только те $A_{U'} $, для которых $\left|U\bigcap U'\right|\ge 2$, то есть на фиксированное событие $A_{U} $ влияют менее $C_{s}^{2} C_{n-2}^{s-2} $ событий.
\end{ordre}






\begin{problem}

С помощью неравенства Талаграна (см. \cite{15}) оцените вероятность того, что случайный граф 
$G\left( {n,\frac{1}{2}} \right)$ не имеет клик размера $k$.

\end{problem}

\begin{ordre} 
Заметим, что задание случайного графа $G\left( {n,\frac{1}{2}} \right)$ эквивалентно подбрасыванию симметричной монеты $C_n^2 $ раза. В качестве случайной величины $X$ возьмите максимальное число непересекающихся по ребрам $k$-клик. Проверьте, что эта теоретико-графовая 
функция удовлетворяет условию Липшица с константой 1 (добавление ребра может добавить не более одной клики в семейство непересекающихся по ребрам клик) и является проверяемой со сложностью $f(s)= s C_k^2 $. Обозначим через $M$ медиану $X$. Неравенство Талаграна дает соотношение
\[
\PR\left[ {X\le M-t\sqrt {M C_k^2 } } \right]\PR\left[ {X\ge M} \right]\le e^{-\frac{t^2}{4}}.
\]
Выбирая $t=\Theta \left( {\frac{\sqrt M }{k}} \right)$, получите оценку для 
\[
\PR\left\{ {\omega (G)<k} \right\}=\PR\left\{ {Y\le 0} \right\}.
\]
\end{ordre}

\begin{problem}[Теорема Шеннона]

Используя вероятностный метод, докажите теорему 
Шеннона. Пусть $\Sigma =\{0,1\}$ {\-} двухбуквенный алфавит, 
$p<1/2$, $\varepsilon >0$ сколь угодно мало. Существует схема 
кодирования со скоростью передачи данных, превосходящей $1-H(p)-\varepsilon 
$, и вероятностью ошибки при передаче меньшей, чем $\varepsilon $.

Под схемой кодирования подразумеваются функции кодирования $f$ и 
декодирования $g$: 
\[f:\quad \{0,1\}^k\to \{0,1\}^n; \quad g:\quad \{0,1\}^n\to 
\{0,1\}^k.\] Скорость передачи данных в такой схеме определяется отношением 
$k/n$. Пусть $e=\left( {e_1 \ldots e_n } \right)$ {\-} случайный шумовой
вектор, компоненты которого независимы и имеют распределение Бернулли с 
параметром $p$. В предположении, что случайное сообщение $x$ имеет 
равномерное на $\{0,1\}^k$ распределение, определим вероятность правильной передачи данных как $\PR\left\{ {g\left( {f(x)+e} \right)=x} \right\}$ (здесь сложение берется по $\mod {2})$.
\end{problem}

\begin{ordre}
Рекомендуется ознакомится с книгами \cite{15}, \cite{444} и А. Ромащенко, А. Румянцев, А. Шень. Заметки по теории кодирования. -- М.МЦНМО, 2011.

Для больших значений $n$, выберем значение $k=\left\lceil {n\left( {1-H(p)-\varepsilon } \right)} \right\rceil $, обеспечивающее 
нужную скорость передачи данных. Зададим $\delta >0$ такое, что $p+\delta <\frac{1}{2}$ и $H(p+\delta )<H(p)+\frac{\varepsilon }{2}$. Тогда утверждение теоремы есть применение теоремы Варшамова--Гилберта: в пространстве $\{0,1\}^n$ с метрикой Хэмминга существует $n(p+\delta )$-сеть мощности не менее, чем $2^k$. Основываясь на вероятностном методе докажите 
последнее. 

Функцию кодирования $f:\quad \{0,1\}^k\to 
\{0,1\}^n$ зададим случайно, выбирая для каждого $x$ значение $f(x)$ 
согласно равномерному распределению на $\{0,1\}^n$. Функцию декодирования определим следующим образом: $g(y)=x$, если $x\in \{0,1\}^k$ {\-} это единственный вектор такой, что $f(x)$ находится на расстоянии Хэмминга не более чем $n(p+\delta )$ от вектора $y\in \{0,1\}^n$. Если указанных векторов $x\in \{0,1\}^k$ нет или напротив больше одного, то декодирование назовем некорректным. Покажите, что вероятность некорректного декодирования 
экспоненциально мала.

Воспользуйтесь неравенством, оценивающим объем шара радиуса $an$, где 
$a<1/2$, в пространстве $\{0,1\}^n$ с метрикой Хэмминга: 
$\sum\limits_{i=0}^{na} {C_n^i } \le 2^{n(H(a)+o(1))}$, что есть следствие 
применения неравенства Чернова к вероятности того, что биномиальная 
случайная величина с параметрами $n$, $1/2$ принимает значения, превосходящие $n-na$.
\end{ordre}

\begin{problem}[Игра лжецa] 
Пусть Пол пытается угадать число $x\in \{1,\ldots ,n\}$ 
у лживой Кэрол, сопротивляющейся этому. Пол может задать $q$ вопросов вида ``Верно ли, что $x\in S?$'', где $S$ {\-} производное подмножество $\{1,\ldots ,n\}$. Вопросы задаются последовательно, $i$-ый вопрос Пола может зависеть от предыдущих ответов. Кэрол может лгать, но она не может солгать больше $k$ раз. Покажите с помощью вероятностного подхода, что при условии
\[
n>\frac{2^q}{\sum\limits_{i=0}^k {C_q^i } }
\]
существует выигрышная стратегия у Кэрол. С помощью метода дерандомизации (см. \cite{15}) опишите явную стратегию Кэрол.
\end{problem}

\begin{ordre}
Заметьте, что в связи с тем, что Кэрол сопротивляется 
правильному отгадыванию числа Полом, здесь можно считать, что ее выигрышная стратегия такова, что она в действительности не загадывает число заранее, а выбирает ответы на каждый вопрос Пола так, что все ее ответы согласуются с более чем одним $x$ (с учетом информации о том, что она может за всю игру солгать не более, чем $k$ раз). Тогда зафиксируем стратегию Пола. Пусть Кэрол играет случайно, то есть с вероятностью $\frac{1}{2}$ отвечает либо ``да'', либо ``нет'' на каждый вопрос Пола. Пусть $I_x $ {\-} индикатор того, что число $x$ согласуется с ответами Кэрол. Покажите, что $E\left[ {I_x } \right]=2^{-q}\sum_{i=0}^k {C_q^i } $. Таким образом, в силу 
линейности математического ожидания, среднее значение числа таких $x$, что согласуются с ответами Кэрол, есть $n2^{-q}\sum_{i=0}^k {C_q^i } $, что по условию больше $1$.

Для построения выигрышной стратегии Кэрол, введем вес игровой ситуации $W$, равный математическому ожиданию числа таких $x$, что согласуются с ответами Кэрол, при условии, что она играет случайно. Тогда стратегия Кэрол: на каждом шаге максимизировать вес игровой ситуации. Так как если для произвольной игровой ситуации с весом и $W$ и некоторым ходом Пола, обозначить за $W^y$ и $W^n$ веса игровой ситуации после ответа Кэрол ``да'', ``нет'' соответственно, то выполняется соотношение $W=\frac{1}{2}(W^y+W^n)$. Итак, стратегия Кэрол не дает весу уменьшиться. Если в начале игры вес был 
больше единицы, значит он будет больше единицы и в конце игры, что 
соответствует выигрышу Кэрол.

\end{ordre}




\begin{problem}
\label{hnmgraph}
Рассмотрим случайную последовательность графов $\{G_n\}$ (модель роста интернета), полученную следующим образом. Зафиксируем параметр $a>0$. Пусть $G_1$ -- граф с одной вершиной $1$ и одной петлей $(1,1)$. Далее, предположим, что граф $G_{n-1}$ уже построен. Граф $G_n$ получается путем добавления к графу $G_{n-1}$ одной вершины $n$ и одного ребра. С вероятностью $\frac{a}{an+n-1}$  это ребро будет направлено из $n$ в $n$, с вероятностью $\frac{deg_{n-1}(i)+a-1}{an+n-1}$ будет добавлено ребро $(n,i)$. Здесь $deg_{n-1}(i)$ -- степень вершины $i$ в графе $G_{n-1}$, $1 \le i \le n-1$. Нетрудно видеть, что выбор того, куда проводить следующее ребро зависит от всех предыдущих ребер. Покажите, что случайный граф $G_n$ можно задать с помощью $n$ независимых случайных величин.
\end{problem}

\begin{ordre}
Воспользуйтесь последовательностью с.в. $\{\xi_i\}$, где равенство $\xi_i = 2j - 1$ означает, что ребро из вершины $i$ идет в $j$, если же   $\xi_i = 2j$, то  ребро из вершины $i$ идет в ту же вершину, что и ребро из вершины $j$.
\end{ordre}

\begin{remark}
Частным случаем приведенной модели графа является модель Bollobas--Riordan при $a = 1$, свойства которой хорошо изучены. К такого типа графам для уменьшения разреженности применяется постобработка: граф $G_{nm}$ преобразуется к графу $G_n^{m}$ путем ``склейки'' (объединения) равных групп из $m$ вершин в одну.

Считается, что модель графа должна удовлетворять следующим эмпирически выявленным характеристикам:
\begin{enumerate}
\item Число ребер пропорционально числу вершин, в то время как число треугольников на порядок больше числа ребер;
\item Одна большая компонента связности небольшого ($\sim 10$) диаметра;
\item Граф устойчив к случайному удалению вершин, в то время как удаление вершин максимальной степени приводит к разбиению графа на компоненты;
\item Степени, вторые степени (размеры второй окрестности) и PageRank вершин подчиняются степенному распределению;
\item Кластерный коэффициент -- вероятность того, что соседи случайной вершины  сами соединены между собой -- имеет неизменное с ростом числа вершин значение;
\item Средняя степень соседей случайной вершины степени $d$ имеет распределение $d^{\delta}$, где $\delta < 0$ характерно для веб-графа,  в то время как $\delta > 0$ характерно для социальной сети;
\item Количество ребер между вершинами заданных степеней имеет специфический вид распределения;
\item  Веб-графу свойственно наличие выраженных двудольных подграфов 
(любители--любимые сайты, покупатели--продавцы ссылок).  
\end{enumerate}
Одним из основных недостатков модели $G_n^{m}$ является низкий кластерный коэффициент (убывает с ростом $n$) и как следствие недостаточное число треугольников. Приведем модель (Ryabchenko--Samosvat--Ostroumova), являющуюся обобщением и лишенную последнего недостатка. В этой модели при добавлении новой вершины ($n+1$) происходит соединение с $m$ вершинами при соблюдении правил на изменение степени $d_i^{n}$ $i$-й вершины:
\[
\PR(d_i^{n+1} = d_i^{n}) = 1 - \frac{A d_i^{n} + B}{n} + O\left( \frac{(d_i^n)^2}{n^2} \right),
\]
\[
\PR(d_i^{n+1} = d_i^{n} + 1) = \frac{A d_i^{n} + B}{n} + O\left( \frac{(d_i^n)^2}{n^2} \right),
\]
\[
\PR(d_i^{n+1} - d_i^{n}  > 1) =  O\left( \frac{(d_i^n)^2}{n^2} \right),
\]
\[
\PR(d_{n+1}^{n+1} > m) =  O\left( \frac{1}{n} \right).
\]
Для $A = 1/(2+a)$, $B = ma/(2+a)$ получим модель $G_n^m$.

\end{remark}


\begin{remark}
Представление $G_n$  с помощью  независимых случайных величин позволяет воспользоваться неравенством Талаграна (см. \cite{15}) для подсчета статистик графа, в частности имеет место следующие неравенство для количества вершин $X_n(d)$ с размером второй окрестности равным $d = O(n^{1/(4+a) - \delta})$:
\[
\PR(|X_n(d) - \Exp X_n(d)| > (\Exp X_n(d))^{1-\varepsilon} ) \to 0, \quad n \to \infty.
\]  
См. также задачу \ref{pref_attach} из раздела \ref{hard} и задачу \ref{soc_ineq} из раздела \ref{macrosystems} и А.М. Райгородский, Модели интернета, Долгопрудный: Интеллект, 2013.
\end{remark}



\section{Вероятностные методы в Computer Science}
\label{CS}

%\subsection{Рандомизированные алгоритмы}

\begin{comment}
\begin{problem}

Пять философов сидят за круглым столом. В центре стола находится чаша со
спагетти. Между каждой парой соседних философов лежит вилка. Философы чередуют размышления с приемами пищи, не отвлекаясь на второстепенные занятия. Однако
для того, чтобы вытащить спагетти из чаши и донести их до рта философу требуются
две вилки. Каждый философ может взять вилку рядом с ним (если она доступна), или положить - если он уже держит её. Если требуемая вилка занята соседом, голодный философ вынужден ждать - он не может вернуться к размышлениям, не поев. После окончания еды философ кладет обе вилки на стол.

Время одного приема пищи одним философом равномерно распределено на отрезке [0, a]. 
Время одного размышления равномерно распределено на отрезке [0, b].

Данный процесс подвержен взаимной блокировке (Dead Lock): например, если каждый возьмет по левой вилке, то начнется вечное голодание. Для избежания блокировки каждый философ кладет первую вилку, если за время t после ее взятия вторая не освободилась.

Требуется определить распределение времени t, минимизирующее среднее время ожидания после размышления и перед приемом пищи.

\end{problem}
\end{comment}

\begin{problem}
\noindent Алгоритм быстрой сортировки основан на парадигме ``разделяй и властвуй''. Выбирается из элементов массива опорный элемент, относительно которого переупорядочиваются все остальные элементы. Желательно выбрать опорный элемент близким к значению медианы, чтобы он разбивал список на две примерно равные части. Переупорядочивание элементов относительно опорного происходит так, что все переставленные элементы, лежащие левее опорного, меньше его, а те, что правее -- больше или равны опорному. Далее процедура быстрой сортировки рекурсивно применяется к левому и правому списку для их упорядочивания по отдельности.

Наихудшие входные данные для описанного алгоритма быстрой сортировки (предполагается, что в качестве опорного элемента выбирается последний элемент обрабатываемого массива) -- элементы уже упорядоченные по возрастанию. 
Откуда следует, что асимптотика времени работы быстрой сортировки в худшем случае $\Theta (n^{2} )$.

Оценить время работы алгоритма быстрой сортировки в среднем. 


\begin{ordre}
Получить рекуррентное соотношение для математического ожидания времени работы, введя индикаторную функцию позиции опорного элемента. 
Воспользоваться соотношением:
\[\begin{array}{l} {\sum _{k=1}^{n-1}k\log k \le \log \frac{n}{2} \sum _{k=1}^{\left\lceil \frac{n}{2} \right\rceil -1}k +\log n\sum _{k=\left\lceil \frac{n}{2} \right\rceil }^{n-1}k =} \\ {=\frac{n(n-1)}{2} \log n-\frac{\left\lceil \frac{n}{2} \right\rceil \left(\left\lceil \frac{n}{2} \right\rceil -1\right)}{2} \le \frac{1}{2} n^{2} \log n-\frac{n^{2} }{8} } .\end{array}\]
 
\end{ordre}

Показать неулучшаемость оценки для произвольного алгоритма сортировки. Привести способ сортировки с асимптотикой $O(n \log n)$ в худшем случае.

\end{problem}


\begin{problem}[Задача поиска $k$-ой порядковой статистики]

Рекурсивное применение процедуры, основанной на методе быстрой сортировки, позволяет быстро (в среднем) находить $k$-ую порядковую статистику. Задача вычисления порядковых статистик состоит в следующем: дан список (массив) из $n$ чисел, необходимо найти значение, которое стоит в $k$-ой позиции в отсортированном в возрастающем порядке списке. 

Модифицируем алгоритм быстрой сортировки:


 Выбираем опорный элемент. Делим список на две группы. В первой --- элементы меньше опорного, во второй --- больше либо равны.
 Если размер (число элементов) первой группы больше либо равен $k$, то к ней снова применяется эта процедура. Иначе --- вызывается процедура для второй группы.
 
Покажите, используя ту же технику, что и при анализе в среднем алгоритма быстрой сортировки, что среднее время работы такого алгоритма линейно.

\begin{ordre}

\noindent Покажите, что выполняется оценка среднего времени работы алгоритма:

\[
\Exp [T(n)]\le \Exp\left\{\sum _{k=1}^{n}T\left(\max (k-1,n-k)\right)+O(n)\right\} 
\]
\[ \le \frac{2}{n} \sum _{k=\left\lfloor \frac{n}{2} \right\rfloor }^{n-1} \Exp[T(k)] + O(n).
\]



\end{ordre}
\end{problem}



\begin{problem}[Задача о рюкзаке] 

Рассмотрим  NP-трудную задачу
\[\sum _{j=1}^{n}x_{j}  \to \max , x_{j} \in \left\{0,\; 1\right\}, j=1,...,n;\] 
\[\sum _{j=1}^{n}a_{ij} x_{j}  \le 1, i=1,\ldots,m,                         (*)\] 
где $a_{ij} \in \left\{0,\; 1\right\}$, $i=1,\ldots,m$, $j=1,\ldots,n$.

Булев вектор $\vec{x}$ длины $n$ будем называть допустимыми, если он удовлетворяет системе (*). Обозначим через $T\left(j\right)$ множество всех допустимых булевых векторов для системы (*) с $(n-j)$ нулевыми последними компонентами и через $\vec{e}_{j} $ - вектор длины $n$ с единичной $j$--ой компонентой и с остальными нулевыми компонентами.

Рассмотрим алгоритм: 1) строим множество допустимых решений $T\left(j\right)$ на основе множества $T\left(j-1\right)$, пытаясь добавить вектор $\vec{e}_{j} $ ко всем булевым векторам $T\left(j-1\right)$; 2) среди $\left|T\left(n\right)\right|$ допустимых булевых векторов ищем ``наилучший''.
\begin{enumerate}
\item Покажите, что сложность описанного алгоритма составляет ${\rm O} \left(\left|T\left(n\right)mn\right|\right)$. При каких $a_{ij} \in \left\{0,\; 1\right\}$ алгоритм будет работать экспоненциально долго?

\item Оцените сложность в среднем (математическое ожидание времени работы алгоритма), т.е. ${\rm O} \left(\Exp\left(\left|T\left(n\right)\right|\right)mn\right)$, если с.в. $\left\{a_{ij} \right\}_{i,j=1}^{m,n} $ -- независимые и одинаково распределенные по закону Бернулли \\ $\mathrm{Be}\left(p\right)$ ($mp^{2} \ge \ln n$).
\end{enumerate}

\begin{ordre}
Пусть $k>0$. Положим: $\vec{x}_{j_{1} ,...,j_{k} } $ - вектор с $k$ единицами (на позициях $\left\{j_{1} ,...,j_{k} \right\}$) и $n-k$ нулями; $p_{ki} $ - вероятность выполнения $i$-го неравенства системы (*) для $\vec{x}_{j_{1} ,...,j_{k} } $; $P_{k} $ -- вероятность того, что $\vec{x}^{k} $ -- допустимое решение (покажите, что $p_{ki} $ и $P_{k} $ не зависят от набора $\left\{j_{1} ,...,j_{k} \right\}$). Докажите, что $p_{ki} \le \left(1-p^{2} \right)^{k-1} \le e^{-p^{2} \left(k-1\right)} $, $P_{k} \le e^{-mp^{2} \left(k-1\right)} $ и $\Exp\left(\left|T\left(n\right)\right|\right)=\sum _{k=0}^{n}C_{n}^{k}  P_{k} <1+n+n\sum _{k=2}^{n}e^{\left(k-1\right)\left(\ln n-mp^{2} \right)}  $.
\end{ordre}

\end{problem}

\begin{problem}

Даны три матрицы $A,B,C$ размера $n\times n$. Требуется проверить равенство $AB=C$.

Простой детерминированный алгоритм перемножает матрицы $A$, $B$ и сравнивает результат с $C$. Время работы такого алгоритма при использовании обычного перемножения матриц составляет $O(n^{3} )$, при использовании быстрого - $O(n^{2,376} )$. Вероятностный алгоритм Фрейвалда с односторонней ошибкой проверяет равенство за время $O(n^{2} )$.

Описание вероятностного алгоритма:

\begin{enumerate}
\item \textbf{ }взять случайный вектор $x\in \left\{0,1\right\}^{n}; $

\item  вычислить $y=Bx;$

\item  вычислить $z=Ay;$

\item  вычислить $t=Cx;$

\item  если $z=t$ вернуть «да», иначе «нет».
\end{enumerate}
Покажите, что для предъявленного алгоритма выполняется 
\[\begin{array}{l} {\PR\left\{z=t \vert AB=C\right\}=1,} \\ {\PR\left\{z \neq t \vert AB\ne C\right\}\ge 1/2.} \end{array}\] 

\begin{remark} (амплификация)
Оцените вероятность ошибочного ответа на одной ненулевой строке матрицы $D = AB - C$. Как можно добиться того, чтобы вероятность ошибочного ответа стала меньше 0.01?
\end{remark}

\end{problem}


\begin{problem}
\label{derandom}
Рассмотрим задачу из класса \textit{NP-трудных} задач -- \textit{максимальная выполнимость (MAX-SAT)}: даны $m$ скобок \textit{конъюнктивной нормальной формы }(КНФ) с $n$ переменными, нужно найти значения переменных, максимизирующее число выполненных скобок.

\begin{enumerate}

\item Для \textit{приближенного} решения задачи MAX-SAT воспользуемся простейшим вероятностным алгоритмом, выбирая значения каждой переменной (0 или 1) независимо и равновероятно. Покажите, что такой алгоритм гарантирует точность ${1\mathord{\left/ {\vphantom {1 2}} \right. \kern-\nulldelimiterspace} 2} $: для всех входов $I$

\[\frac{\Exp m_{A} (I)}{m_{0} (I)} \ge \frac{1}{2} ,\] 
где $m_{0} (I)$ - оптимум, $m_{A} (I)$ - случайное значение, найденное алгоритмом. 

\item Алгоритм с лучшими оценками точности строится на основе \textit{метода вероятностного округления}. Для начала переформулируем задачу MAX-SAT в терминах задачи целочисленного линейного программирования (ЦЛП). Каждой скобке $C_{j} $ поставим в соответствие булеву переменную $z_{j} \in \left\{0,1\right\}$, которая равна 1, если скобка $C_{j} $ выполнена; каждой входной переменной $x_{i} $ сопоставляем переменную $y_{i} $, которая равна 1, если $x_{i} =1$, и равна 0 в противном случае. Обозначим $C_{j}^{+} $ индексы переменных в скобке $C_{j} $, которые входят в нее без отрицания, а через $C_{j}^{-} $ - множество индексов переменных, которые входят в скобку с отрицанием. Тогда задача MAX-SAT эквивалентна следующей задаче ЦЛП:

\[\begin{array}{l} {\sum _{j=1}^{m}z_{j}  \to \mathop{\max }\limits_{z,y} } \\ {\sum _{i\in C_{j}^{+} }y_{i} +\sum _{i\in C_{j}^{+} }(1-y_{i} )  \ge z_{j} ,\quad j=1,...,m} \\ {y_{i} ,z_{j} \in \left\{0,1\right\},\quad i=1,...,n,j=1,...,m} \end{array}\] 

Рассмотрим и решим задачу \textit{линейной релаксации} целочисленной программы:

\[\begin{array}{l} {\sum _{j=1}^{m}\hat{z}_{j}  \to \mathop{\max }\limits_{\hat{y},\hat{z}} } \\ {\sum _{i\in C_{j}^{+} }\hat{y}_{i} +\sum _{i\in C_{j}^{+} }(1-\hat{y}_{i} )  \ge \hat{z}_{j} ,\quad j=1,...,m} \\ {\hat{y}_{i} ,\hat{z}_{j} \in \left\{0,1\right\},\quad i=1,...,n,j=1,...,m} \end{array}\] 

Пусть $\hat{y}_{i} ,\; \hat{z}_{j} $ - решения задачи линейной релаксации. Ясно, что $\sum _{j=1}^{m}\hat{z}_{j}  $ является верхней оценкой числа выполненных скобок для данной КНФ.

Рассмотрим вероятностный алгоритм решения задачи максимальной выполнимости, где каждая переменная $y_{i} $ независимо принимает значения 0 или 1 уже не с равными вероятностями, а с вероятностью $\hat{y}_{i} $ принимает значение 1 (и 0 с вероятностью $1-\hat{y}_{i} $). Такой метод называется \textit{вероятностным округлением}.

Докажите, что если в скобке $C_{j} $ имеется $k$ литералов, то вероятность того, что она выполнена при вероятностном округлении, не менее

\[\left(1-\left(1-\frac{1}{k} \right)^{k} \right)\hat{z}_{j} .\] 

\begin{remark}
Тогда из того, что

\[1-\left(1-\frac{1}{k} \right)^{k} \ge 1-\frac{1}{e} >0.63\] 
для всех положительных целых $k$, получаем, что для произвольной КНФ среднее число скобок, выполненное при вероятностном округлении, не меньше $1-\frac{1}{e} >0.63$ от максимально возможного числа выполненных скобок.
\end{remark}

\item Для получения детерминированного алгоритма приближенного решения задачи MAX-SAT воспользуемся один из способов\textit{ дерандомизации -- методом условных математических ожиданий. }Введем случайную величину\textit{ }$Z(x)$, где в булевом векторе $x=(x_{1} ,...,x_{n} )$ компоненты - суть значения переменных в КНФ, назначенных вероятностным алгоритмом - являются независимыми случайными величинами, причем $\PR\left\{x_{i} =1\right\}=p_{i} $, $\PR\left\{x_{i} =0\right\}=1-p_{i} $; $Z(x)$ - число невыполненных скобок. Требуется найти булев вектор $\hat{x}$, для которого выполнено неравенство $Z(\hat{x})\le \Exp Z$. Обозначим через $Z(x|x_{1} =d_{1} ,...,x_{k} =d_{k} )$ новую случайную величину, которая получена из $Z$ фиксированием значений первых $k$ булевых переменных.

Рассмотрим покомпонентную стратегию определения искомого вектора $\hat{x}$. Для определения его первой компоненты вычисляем значения $f_{0} =\Exp Z(x|x_{1} =0)$ и $f_{1} =\Exp Z(x|x_{1} =1)$. Если $f_{0} <f_{1} $ полагаем $x_{1} =0$, иначе полагаем $x_{1} =1$. При определенной таким образом первой компоненты (обозначим ее $d_{1} $) вычисляем значение функционала $f_{0} =\Exp Z(x|x_{1} =d_{1} ,x_{2} =0)$ и $f_{1} =\Exp Z(x|x_{1} =d_{1} ,x_{2} =1)$. Если $f_{0} <f_{1} $ полагаем $x_{2} =0$, иначе полагаем $x_{2} =1$. Фиксируем вторую координату (обозначая ее $d_{2} $) и продолжаем описанный процесс до тех пор, пока не определится последняя компонента решения.

Покажите, что найденный вектор $(x_{1} =d_{1} ,\ldots ,x_{n} =d_{n} )$ будет удовлетворять требованию минимизации оценки математического ожидания:

\begin{enumerate}
\item  для этого докажите неравенство $\Exp Z\ge \Exp Z(x|x_{1} =d_{1} )$;

\item  рекуррентно получите неравенство $\Exp Z\ge \Exp Z(x|x_{1} =d_{1} ,\ldots ,x_{n} =d_{n} )$;

\item  заметьте, что $\Exp Z(x|x_{1} =d_{1} ,\ldots ,x_{n} =d_{n} )=Z(x|x_{1} =d_{1} ,\ldots ,x_{n} =d_{n} )$.
\end{enumerate}

Покажите справедливость формул:
\begin{enumerate} 

\item $\Exp Z=\sum _{j=1}^{m}\PR_{j}$, где $\PR_{j} =\PR\left\{\sum _{i\in C_{j}^{+} }x_{i} +\sum _{i\in C_{j}^{-} }(1-x_{i} )  =0\right\};$
\item в предположении, что значения первых $k$ переменных уже определены и $I_{0} $ --- множество индексов тех переменных, значения которых равно 0, а $I_{1} $ --- множество индексов тех переменных, значения которых равно 1:
$$\Exp Z(x|x_{1} =d_{1} ,\ldots ,x_{k} =d_{k} )=\sum _{j=1}^{m}\PR_{j} ^{k},$$ 
\[ \text{где}\PR_{j} ^{k} = \PR\left\{\sum _{i\in C_{j}^{+} }x_{i} +\sum _{i\in C_{j}^{-} }(1-x_{i} )  =0|x_{1} =d_{1} ,\ldots,x_{k} =d_{k} \right\}=\] 
\[\left\{\begin{array}{cc} {0,} & { \text{при }  (I_{0} \cap C_{j}^{-} ) \cup ( I_{1} \cap C_{j}^{+} ) \ne \emptyset ;} \\ {\prod_{i\in C_{j}^{+} \backslash I_{0} }(1-p_{i} ) \prod _{i\in C_{j}^{-} \backslash I_{1} }p_{i}  ,} & \text{иначе} \end{array}\right. \] 
Здесь $C_{j}^{+} $ --- множество индексов переменных в скобке $C_{j} $, которые входят в нее без отрицания,  $C_{j}^{-} $ --- множество индексов переменных, которые входят в скобку с отрицанием.

\end{enumerate}

\end{enumerate}

\end{problem}

\begin{remark}
Детали см. в  Кузюрин Н.Н., Фомин~С.А. Эффективные алгоритмы и сложность вычислений: Учебное пособие. -- М.: МФТИ, 2007.
\end{remark}




%\begin{comment}

\begin{problem}[Коммуникационная сложность, хэширование]

Требуется сравнить ("достаточно достоверно") две битовые строки $a,b$,  осуществив как можно меньше по битовых сравнений. Основная идея -- сравнивать не сами строки, а функции от них. Так сравниваются $a\; \mod\; p$ и $b\; \mod\; p$, для некоторого простого числа $p$. Для этого требуется передать $2\log p$ бит информации.

Описание алгоритма сравнения строк:

\begin{enumerate}
\item  Пусть $\left|a\right|=\left|b\right|=n$, $N=n^{2} \log  n^{2}; $ 

\item  Выбираем случайное простое число $p$из интервала $\left[2..N\right]$ ;

\item  Выдать «да», если $a\; \mod\; p=b\; \mod\; p\Leftrightarrow (a-b)\equiv 0\; \mod\; p$, иначе выдать «нет».
\end{enumerate}

\noindent  Обоснуйте выбор именно простого числа на шаге 2 и предложите способ его генерации.  

\noindent Покажите что,

\[\begin{array}{l} {\PR\left\{(a-b)\equiv 0,\mod(p)|a=b\right\}=1,} \\ {\PR\left\{(a-b)\equiv 0,\mod(p)|a\ne b\right\}=O(1/n),} \end{array}\] 
 
При этом необходимое количество переданных бит равно $O\left(\log  n\right)$.

\begin{ordre}

Воспользоваться асимптотическим законом распределения простых чисел:
\[\mathop{\lim }\limits_{n\to \infty } \frac{\pi \left(n\right)}{n/\ln n}  =1,\] 
где $\pi \left(n\right)$ - функция распределения простых чисел, равная количеству простых чисел, не превосходящих $n$.

\end{ordre}

\end{problem}

\begin{remark}
В приведенной задаче требуется проверка простоты числа. Согласно малой теореме Ферма, если $N$ - простое число и целое $a$ не делится на $N$, то  
\[a^{N-1} \equiv 1\; \mod\; N.                         \; \; \;            \left(*\right)\] 

Отсюда следует, что если при каком-то $a$ сравнение $\left(*\right)$ нарушается, то можно утверждать, что $N$ - составное. 
К сожалению,  простой вариант подбора $a$ не всегда позволяет эффективно выявить составное число. Имеются составные числа $N$, обладающие свойством $\left(*\right)$ для любого целого $a$ с условием $\left(a,N\right)=1$ ($a$ и $N$ - взаимно простые). Такие числа называются числами Кармайкла.

В 1976 г. Миллер предложил заменить проверку $\left(*\right)$ проверкой несколько иного условия. Если $N$ - простое число, то $N-1=2^{s} t$, где $t$ нечетно, то согласно малой теоремы Ферма для каждого a с условием $\left(a,N\right)=1$ хотя бы одна из скобок в произведении 
\[\left(a^{t} -1\right)\left(a^{t} +1\right)\left(a^{2t} +1\right) \ldots  \left(a^{2^{s-1} t} +1\right)=a^{N-1} -1\] 
делится на $N$. 

Пусть $N$ - нечетное составное число, $N-1=2^{s} t$, где \textbf{$t$ }нечетно. Назовем целое число $a$, $1<a<N$ «выявляющим» для $N$, если нарушается одно из двух условий:

I) $N$ не делится на $a$

II) $a^{t} \equiv 1\; \mod\; N$ или существует целое $k$, $0\le k<s$ такое, что $a^{2^{k} t} \equiv -1\; \mod\; N$.

Если $N$ составное число, то согласно теореме Рабина  существует не менее $\frac{3}{4} \left(N-1\right)$  выявляющих чисел.

\end{remark}

\begin{comment}

\begin{problem}

Пусть $f(x_{1} ,...,x_{n} )=C_{1} \vee \cdots \vee C_{m} $ - булева формула в дизъюнктивной нормальной форме (ДНФ), где каждая скобка $C_{i} $ - есть конъюнкция $L_{1} \wedge \cdots \wedge L_{k_{i} } $ $k_{i} $ литералов (литерал есть либо переменная, либо ее отрицание). Набор значений переменных $a=(a_{1} ,...,a_{n} )$ называется выполняющим для $f$, если $f(a_{1} ,...,a_{n} )=1$. Требуется найти число выполняющих наборов для данной ДНФ.

\noindent $V$ - множество всех двоичных наборов длины $n$.

\noindent $G$ - множество выполняющих наборов.


\noindent  Проведем $N$ независимых испытаний:

\noindent Выбираем случайно $v_{i} \in V$ ( в соответствии с равномерным распределением).
\noindent $y_{i} =f(v_{i} )$. Заметим, что $P\left\{y_{i} =1\right\}=\frac{\left|G\right|}{\left|V\right|} =p$.
Рассмотрим сумму независимых случайных величин $Y=\sum _{i=1}^{N}y_{i}  $. В качестве аппроксимации $\left|G\right|$ возьмем величину $\frac{Y}{N} \left|V\right|$.

\noindent Оцените необходимое число испытаний $N$ как функцию от $|V|$, $|G|$ и точности аппроксимации $\varepsilon$. 

\begin{ordre}
Докажите следующее утверждение. Пусть $X_{1} ,...,X_{n} $ - независимые случайные величины, принимающие значения 0 или 1, при этом $P\left\{X_{i} =1\right\}=p,\quad P\left\{X_{i} =0\right\}=1-p$. Тогда для $X=\sum _{i=1}^{N}X_{i}  $ и для любого $0<\delta <1$, выполнены неравенства
\[\begin{array}{l} {P\left\{X>(1+\delta )EX\right\}\le e^{-\frac{\delta ^{2} }{3} EX} } \\ {P\left\{X<(1-\delta )EX\right\}\le e^{-\frac{\delta ^{2} }{2} EX} } \end{array}\] 
\end{ordre}

\end{problem}


\begin{problem}

(Задача о покрытии). Дано конечное множество из m элементов и система его подмножеств $S_{1} ,...,S_{n} $. Требуется найти минимальную по числу подмножеств подсистему $S_{1} ,...,S_{n} $, покрывающую все множество объектов. 

\noindent Сформулируем ее в терминах булевых матриц и целочисленного линейного программирования:
\[\left\{\begin{array}{l} {cx\to \min ,} \\ {Ax\ge b,} \\ {\forall j\; x_{j} \in \{ 0,1\} .} \end{array}\right. \] 
Здесь переменные $x_{1} ,...,x_{n} $ соответствуют включению подмножеств $S_{1} ,...,S_{n} $ в решение-покрытие, матрица $A$ - матрица инцидентности, $c=(1...1)^{T} \in {\mathbb R}^{n} ,\quad b=(1...1)^{T} \in {\mathbb R}^{m} $ - векторы стоимости и ограничений.

\noindent Пусть элементы матрицы инцидентности -- независимые случайные величины с бернулевским распределением:$P\{ a_{ij} =1\} =p,$ $P\{ a_{ij} =0\} =1-p$. 

\noindent Для решения задачи применяется жадный алгоритм: на каждом шаге выбирается подмножество, максимально покрывающее еще не покрытые объекты. 

Доказать следующее утверждение. Пусть для случайной матрицы $A$, определенной выше, выполнены соотношения:
\[\begin{array}{l} {\forall \gamma >0:} \\ {\frac{\ln n}{m^{\gamma } } \mathop{\to }\limits_{n\to \infty } 0,} \\ {\frac{\ln m}{n} \mathop{\to }\limits_{n\to \infty } 0.} \end{array}\] 
Тогда для $\forall \varepsilon >0:$ $P\left\{\frac{Z}{M} \le 1+\varepsilon \right\}\mathop{\to }\limits_{n\to \infty } 1$, где Z -- решение жадного алгоритма, M -- величина минимального покрытия. 

\begin{fixme}
Добавить указание.
\end{fixme}

\end{problem}

\end{comment}

\begin{comment}
\begin{problem}
А) Пусть имеется генератор случайных чисел, в 
результате обращения к которому появляется 0 или 1 с одинаковой вероятностью 
равной 1/2 (аналог подбрасывания симметричной монеты). Пусть задано 
вещественное число $0\le p\le 1$. С помощью имеющегося генератора определить 
генератор randp, в результате обращения к которому появляется 0 или 1 с 
вероятностями $p$ и $1-p$ соответственно (незначительные отклонения 
допустимы). Оцените сложность в среднем алгоритма получения одного 
случайного числа с помощью randp (затраты определяются числом обращений $к$ 
изначально имеющемуся генератору). 

Б) Пусть имеется генератор 
случайных чисел randp (описанный выше). Известно, что $p\ne 0,\quad p\ne 1$. 
Как с помощью него сконструировать генератор, в результате обращения 
которому появляется 0 или 1 с одинаковой вероятностью 1/2. 

В) Чему равно математическое ожидание числа обращений к изначально имеющемуся 
генератору случайных чисел при построении последовательности пар до 
появления 0,1 или 1,0? Найти сложность в среднем алгоритма получения k 
``равновероятных'' нулей и единиц с помощью сконструированного генератора 
(затраты определяются количеством обращений к изначально имеющемуся 
генератору). Можно ли указать значения $p$, для которых эта сложность имеет 
минимальное и, соответственно, максимальное значение?
\end{problem}
\end{comment}

\begin{problem}[Интерактивные доказательства] 

а) (изоморфизм графов). Авдотье известен изоморфизм $\phi$ графов $G_0$ и $G_1$. Но она посылает Евлампию граф $H =\psi(G_0)$, либо $H =\psi(G_1)$, где $\psi$ -- некоторый другой изоморфизм, не равный $\phi$. Евлампий бросает симметричную монетку и просит изоморфизм либо $H : G_0$, либо $H : G_1$. В первом случае Авдотья  посылает  $\psi$, во втором -- $\psi \phi^{-1}$. Таких партий разыгрывается $N$ штук. Заметим, что в каждой новой партии Авдотья придумывает новую перестановку $\psi$   вершин графа $G_0$. Если $\phi$  -- действительно изоморфизм $G_0 : G_1$, то все проверки Евлампия будут положительны.
Покажите, что если $\phi$ -- блеф, то с вероятностью  $1 - 2^{-N}$  хотя бы одна проверка обнаружит это (та проверка, в которой Евлампий попросил $H : G_1$).

\begin{remark}
Этот пример поучителен с точки зрения ``криптографического фокуса'' -- Авдотья  убедила Евлампия в $G_0 : G_1$ так и не огласив самого изоморфизма $\phi$. Если $\phi$ –- пароль, то диалог можно вести даже в открытую, что служит примером криптосистемы с нулевым разглашением.
\end{remark}

Обоснуйте справедливость данного замечания (Евлампий не получает никакой информации об изоморфизме $\phi$).

б) (неизоморфизм графов). Теперь наделим Авдотью сверхъестественными вычислительными способностями. Для удобства переименуем игроков: ``Prover'' и ``Verifier''. Verifier выбирает случайно (равновероятно) $ i \in \lbrace 0, 1 \rbrace$ и некоторую перестановку $\pi$ вершин графа $G_i$, затем посылает граф $H =\psi(G_i)$ и требует, чтобы Prover определил $i$. Таких партий разыгрывается $N$ штук. Аналогично предыдущему примеру, $\pi$ в каждой новой партии свое. Если графы неизоморфны, то Prover всегда верно определит индекс $i$. Все тесты будут пройдены. Покажите, что иначе с вероятностью $1-2^{-N}$  Prover ошибется хотя бы один раз (хотя бы в одной партии).
 
 \begin{remark}
Заинтересовавшимся в этой теме, можно также порекомендовать посмотреть про цифровую подпись (протокол аутентификации) и электронную систему голосвания в книге Введение в криптографию под ред. В.В. Ященко. М.: МЦНМО, 2013.
\end{remark}
\end{problem}

\begin{problem}[Хеш функции]
\label{hesh_func}
Для компактного хранения данных, индексированных, как правило, ключами целочисленного или строкового форматов, используются хеш-функции, при помощи которых вычисляются  номера ячеек в таблице в зависимости от значения ключа. Пусть хэш-функция задана на множестве ключей $K$ 
\[
h: K \to \{1,\ldots,m\}.
\]  
Заведем таблицу размера $m$, в которой элемент с ключом $k$ будет размещаться по адресу $h(k)$ (во многих случаях $m \ll |K|$). Ключи с одинаковым значением $h(\cdot)$ образуют список, начинающийся с ячейки таблицы. Если всего имеется $n$ элементов, то желательно чтобы количество ключей с одинаковым хеш-значением не  превышало значительно $\alpha = n/m$ (в таком случае длины списков будут ограничены $O(1+\alpha)$). Формально данное свойство может быть сформулировано в виде двух требований:
\begin{enumerate}
\item $h(k)$ является с.в. с равномерным распределением;
\item $\forall k_1, k_2 \in K$ $h(k_1)$ и  $h(k_2)$ независимы.  
\end{enumerate}    
Покажите, что в случае выполнения требований среднее время поиска отсутствующего ключа в таблице составляет $(1+ \alpha)$, а также среднее время поиска присутствующего ключа в таблице составляет $(1+ \alpha/2)$ (где усреднение ведется как по $h(k)$, так и по множеству других присутствующих в таблице ключей). 
\end{problem}

\begin{remark}
Так как хеш-функция является детерминированной, то случайность, присутствующая в требованиях, может быть реализована посредством выбора функции $h$ из некоторого семейства $H$. $H$ называется \textit{универсальным} для множества $K$, если $\forall k_1, k_2 \in K$ вероятность выбрать функцию из $H$, в которой возникает коллизия, не превышает $1/m$. Для универсального семейства справедливы приведенные в задаче оценки времени поиска ключа в таблице. 
Примером универсального семейства $H$ для целых чисел может служить
\[
\{h_{ab}: h_{ab}(k) = ((ak +b) \; \text{mod} \; p) \; \text{mod} \;   m, \; a,b \in \{1,\ldots, p-1\}\},
\]    
где $p$ -- простое число, большее $m$.

Другой пример универсального семейства, используемого при хешировании IP адресов вида $(x_1,x_2,x_3,x_4)$, $x_i \in \{0,\ldots,255\}$ представляет собой 
множество хеш-функций
\[
\{
h_a(x_1,x_2,x_3,x_4) = (a_1 x_1 + a_2 x_2 + a_3 x_3 + a_4 x_4) \; \text{mod} \; p
\}.
\]
где $p$ -- также простое число, большее $m$.
\end{remark}

\begin{problem}[Хеширование без коллизий]
При известном заранее множестве ключей $K, |K| = n$ хеширование без коллизий может быть реализовано при помощи двухуровневой схемы. На первом уровне используется таблица размера $m = n$, для которой хеш функция выбирается случайно из универсального семейства (см. замечание к задаче \ref{hesh_func}). На втором уровне осуществляется хеширование без коллизий (вместо создания списка ключей будем использeтся вторичная хеш-таблица $T_i$, хранящая все ключи, хешированные функцией $h$ в ячейку $i$, со своей функцией $h_i$).  Покажите, что при выборе  на втором уровне таблицы размером $n_i^2$ ($n_i$ -- количество ключей в $i$-й ячейке первого уровня) вероятность возникновения коллизий менее $1/2$ (поэтому для данной ячейки всегда найдется функция $h_i$ свободная от коллизий). Докажите, что существует хеш функция $h$ для первого уровня, при которой затраты памяти составят суммарно $O(n)$.    
\end{problem}

\begin{ordre}
Получите оценку $\Exp \left(\sum_i n_i^2\right) < 2n$, воспользуйтесь неравенством Маркова.
\end{ordre}


\begin{problem}[Bloom filter]
Вероятностная структура данных Bloom filter используется для проверки принадлежности элемента множеству $S$. Она представляет собой массив $A$ длиной $m$ бит и $k$ различных хэш-функций $h_1, \ldots, h_k$, равновероятно отображающих элементы множества  $D \supseteq S$ в позиции массива $A$ ($h_i: S \to \{1, \ldots, m\}$). 
Интерфейс данной структуры включает следующие операции:

\verb|add_element(s)|: устанавливает биты $A[h_1(s)], \ldots, A[h_k(s)]$ равными 1 (изначально все биты равны 0);

\verb|contains(s)|: если биты $A[h_1(s)], \ldots, A[h_k(s)]$ равны 1, то возвращает \verb|true|, иначе \verb|false|;

Последовательно добавив все элементы множества $S$ при помощи операции \verb|add_element|, получим объект структуры данных, который при вызове метода \verb|contains(s)| всегда возвращает \verb|true| при $s \in S$, но не всегда возвращает \verb|false| при  $s \notin S$, так как соответствующие биты могли быть установлены равными 1 за счет других элементов из $S$.     

\verb|approx_size|: возвращает приближенное количество добавленных элементов
 \[
 |S| \approx  - m \frac{\ln (1 - \mathbb{I}/m)}{k},
 \] 
где $\mathbb{I}$ -- сумма элементов $A$. 

\imgh{80mm}{bf.png}{Пример структуры данных Bloom filter ($k = 3$, $m=18$): x, y, z -- принадлежат множеству, w -- не принадлежит.}


Считая события $A[i] = 1$ и $A[j] = 1$ при добавлении $n$ элементов в пустой объект структуры Bloom filter  независимыми, ввиду независимости значений хэш-функций, покажите, что:
\[
\PR(\text{contains(s)} = \text{true} \; | \; s \text{ not in } S) \approx p =  \left( 1 - e^{k n / m} \right)^k.
\]   
Откуда следует, что оптимальные значения $m$ и $k$ могут быть вычислены по формулам:
\[
k = \frac{m}{n} \ln 2, \quad m = -n \frac{\ln p}{(\ln 2)^2}.
\]
\end{problem}

\begin{remark}
Bloom filter  обладает следующими отличительными характеристиками:
\begin{enumerate}
\item При вероятности ошибки $p = 0.01$ требуется всего лишь 9.6 бит на один элемент множества.
\item  Сложность операций \verb|contains| и \verb|add_element| составляет $O(k)$, вне зависимости от $|S|$. Стоит также учесть, что $k$ операций над элементами массива могут быть выполнены параллельно. 
\item Операции объединения и пересечения двух множеств выполняются побитно (за сравнительно короткое время): 
\[
 |S_1 \cup S_2| \approx  - m \frac{\ln (1 - \mathbb{I}_{12}/m)}{k},
 \]   
 \[
 |S_1 \cap S_2| = |S_1| + |S_2| -   |S_1 \cup S_2|,
 \]
где $\mathbb{I}_{12}$ -- скалярное произведение  $A_1 $ и $A_2$. 

\item Сборщик мусора, ровно как и сериализатор могут осуществлять сравнительно быстрые операции чтения/записи/удаления структуры данных Bloom filter, т. к.  $A$ представляет собой единичный объект, состоящий из элементарных типов данных.
\end{enumerate}

Области применения: снижение количества обращений к базе данных, хранящейся на диске в случае отсутствия запрашиваемых данных (Apache Cassandra);  локальная проверка url-адресов на принадлежность списку, хранящемуся на удаленном сервере (Google Chrome); 
выявление содержимого архива (Venti archival storage system);  
\end{remark}

\begin{problem}[Jaccard similarity]
Для измерения близости (веса связи) двух множеств часто используется коэффициент Жаккара:
\[
J(S_1, S_2) = \frac{|S_1 \cap S_2|}{|S_1 \cup S_2|}.
\]
Типичными представителями таких множеств являются списки смежных вершин в графе, изображения, профили пользователей социальных сетей и прочие web-страницы. Вычисление коэффициента Жаккара позволяет выявлять дубликаты, проводить кластеризацию, восстанавливать фрагменты перечисленных выше объектов. 

Для получения попарной близости $S_1, \ldots, S_N$   множеств больших размеров ($|S_i| = s > 500$, $i \in \overline{1,N}$) возникает потребность в приближенном вычислении коэффициента Жаккара, снижающем нагрузку на сеть вычислительного кластера и временную сложность вычислений. Для этой цели достаточно для каждого множества $|S_i|$ найти минимумы $k$ хэш-функций и далее при попарном сравнении оперировать только с набором подмножеств из минимумов. В итоге, временная сложность окажется равной $O(k N^2 + k s N \log N)$, в то время как в исходной задаче сравнения без аппроксимации меры $O(s \log s  N^2)$.

Рассмотрим два варианта приближенного вычисления, базирующиеся на следующем свойстве: пусть $h$ -- хэш-функция, иньективно и равновероятно отображающая элементы множества $\cup_i S_i$ в $\mathbb{N}$, тогда 
 \[
\PR( \min \limits_{s \in S_1} h(s) = \min \limits_{s \in S_2} h(s)) = J(S_1, S_2). 
 \]         
\begin{enumerate}
\item[\textit{1 вар.}] Задействуем $k$ различных хэш-функций. В этом случае оценкой $J$ будет являться доля функций, у которых совпадают минимальные значения на обоих множествах.    
\item[\textit{2 вар.}] Обозначим за $h_{(k)}(S)$ подмножество $S$  из $k$  элементов с наименьшими значениями $h$. Тогда  $J$ можно оценить как $h_{(k)}(S_1) \cap h_{(k)}(S_2) \cap h_{(k)}(S_1 \cup S_2)$.
\end{enumerate}

Сравните временные сложности двух вариантов. Используя метод Чернова  (см. задачу \ref{chernov_th} из раздела \ref{measure}), покажите, что порядок ошибки аппроксимации в обоих вариантах $O(1/ \sqrt{k})$.

\end{problem}


\begin{problem}[EM-алгоритм]
\label{em}

Рассмотрим задачу поиска неизвестных параметров распределения при помощи метода максимального правдоподобия. В ряде случаев, где функция правдоподобия имеет вид, не допускающий удобных аналитических методов исследования, для ее упрощения удобно ввести дополнительные ``скрытые'' (латентные) переменные и воспользоваться EM-алгоритмом.
Пусть требуется найти максимум правдоподобия $L(X,\theta) = \log p(X | \theta)$ в вероятностной модели со скрытыми переменными $Z$
\[
p(X | \theta) = \int p(X, Z | \theta) dZ.
\]
Покажите, что $L(X,\theta)$ можно представить как
\[
L(X,\theta) = \int \log \frac{p(X, Z | \theta)}{q(Z)} q(Z) dZ - \int \log \frac{p(Z | X, \theta)}{q(Z)} q(Z) dZ = 
\]
\[
 = l(X, \theta, q) + \mathcal{KL}(q \Vert p(Z | X, \theta)),
\]
где $q(Z)$ -- произвольное распределение над скрытыми переменными.

Итерационная схема EM-флгоритма состоит в фиксации на шаге $t$ некоторого значения $\theta^{(t)}$ и аппроксимации в этой точке правдоподобия с помощью его нижней оценки $l(\cdot)$:
\[
q(Z) = p(Z | X, \theta^{(t)}), \quad l(X, \theta, q) \to \max_{\theta}.
\]
Expectation step: фиксируется значение параметров $\theta^{(t)}$. Оценивается распределение на скрытые переменные
\[
q(Z) = p(Z |X, \theta^{(t)}) = \frac{p(X, Z | \theta^{(t)})} {p(X|\theta^{(t)})};
\]
Maximization step: фиксируется распределение $p(Z|X, \theta)$ и выполняется поиск новых параметров
\[
\theta^{(t+1)} = \arg\max_{\theta} \Exp_q \log p(X,Z|\theta).
\]
Покажите, что 
\[
L(X,\theta^{(t+1)}) \geq L(X,\theta^{(t)}),
\]
а также при гладкости $L$
\[
\frac{\partial L}{\partial \theta} \bigg|_{\theta^{(\infty)}} = 0.
\]
\end{problem}

\begin{ordre}
Воспользуйтесь конструкцией
\[
\theta^{(t+1)} = \arg \max \limits_{\theta} \{ Q(\theta)  - b_t \mathcal{R} (\theta, \theta^{(t)})\},
\]
где $\mathcal{R} (\theta, \theta^{(t)}) \geq 0$ -- регуляризатор, в нашем случае равный $\mathcal{KL}(\cdot\Vert\cdot) $.
\end{ordre}


\begin{problem}[Pазделение смеси распределений]


%Рассмотрим задачу поиска неизвестных параметров распределения при помощи метода максимального правдоподобия. В ряде случаев, где функция правдоподобия имеет вид, не допускающий удобных аналитических методов исследования, но допускающий серьезные упрощения, если в задачу ввести дополнительные ``скрытые'' (латентные) переменные, удобно воспользоваться EM-алгоритмом.

%В общем случае ЕМ-алгоритм может быть представлен в следующем виде. Пусть $X$ -- наблюдаемые переменные (выборка), $Z$ -- скрытые переменные с вариационным распределением $q$, $\theta$ -- искомые параметры,  $L(\theta, X, Z) = p(X,Z | \theta)$, $L(\theta, X) = \int p(X,z | \theta) p(z | \theta) dz$,
%\[
%\mathcal{F}(q, \theta) = \Exp_q[\log L(\theta, X, Z)] + H_q(q)  = L
%(\theta, X) - \mathcal{KL}(q \Vert p(Z | X, \theta)) 
%\]

%\begin{enumerate}
%\item[1)] Expectation step:
%\[
%q^{(t)} = \arg \max \limits_q \mathcal{F}(q, \theta^{(t)})
%\]
%\item[2)] Maximization step:
%\[
%\theta^{(t+1)} = \arg \max \limits_{\theta} \mathcal{F}(q^{(t)}, \theta)
%\]

%\end{enumerate}

Целью разделения смеси является как восстановление плотности наблюдаемых данных $X$, так и покомпонентная их категоризация (каждый элемент выборки  $X$ принадлежит одному из распределений, входящих в осотав смеси).
Допустим, что плотность распределения в $z$-й компоненте смеси равна $p (x | z) = p (x | z, \alpha_z)$, т.е. известна с точностью до параметра $\alpha_z$. Тогда плотность $x \in X$ можно аппроксимировать смесью 
\[
p_{\theta}(x) = \sum \limits_{z = 1}^K w_i \; p (x | z),
\]
где $\theta = (w_1, \ldots, w_K, z_1, \ldots, z_K)$, $w_i \geq 0$, $\sum_i w_i = 1$, $K$ -- количество компонент смеси. Правдоподобие $X$ задается формулой 
\[
\log L(\theta, X) = \sum \limits_{j=1}^{n} \left(\log  \sum \limits_{z = 1}^K w_i p (x_j | z) \right). 
\]
Непосредственный поиск точки максимума данной функции весьма затруднителен. Для упрощения вычислений применим EM-алгоритм (см. задачу \ref{em}),  сопоставив каждому $x_j$  скрытую переменную $z_j$ -- номер компоненты смеси, породившей  $x_j$, т. е.  $x_j \sim p(x | z_j)$. 
Проверьте, что при такой модификации данных, функция правдоподобия примет вид

\[
\log L(\theta, X, Z) = \sum \limits_{j=1}^{n} \log w_{z_j} + \sum \limits_{j=1}^{n} \log  p (x_j | z_j). 
\]
Покажите, что шагами  EM-алгоритма в этом случае будут 

\begin{enumerate}
\item[1)] Expectation step:
\[
q_{j}^{(t)}(z) \propto w_{z}^{(t)} \; p (x_j | z, \alpha_z^{(t)});
\]
\item[2)] Maximization step:
\[
w_z^{(t+1)} = \frac{1}{n}   \sum \limits_{j=1}^{n} q_{j}^{(t)}(z),
\]
\[
\alpha_z^{(t+1)} = \arg \max \limits_{\alpha}  \sum \limits_{j = 1}^n q_{j}^{(t)}(z) \; p (x_j | z, \alpha). 
\]
\end{enumerate}

\end{problem}

\begin{comment}
Приведем два важных обобщения EM-алгоритма:
\begin{enumerate}
\item PP-алгоритм (см. Васильев, Методы оптимизации – М.: Факториал Пресс, 2002): 
\[
Q(\theta) \to \max
\]
\[
\theta^{(t+1)} = \arg \max \limits_{\theta} \{ Q(\theta)  - b_t \mathcal{R} (\theta, \theta^{(t)})\},
\]
где $\mathcal{R} (\theta, \theta^{(t)}) \geq 0$ -- регуляризатор, в нашем случае равный $\mathcal{KL}(q \Vert p(Z | X, \theta)) $.

\item Variational EM: 
\[
\mathcal{F}(q, \theta)  = \mathcal{F}(q_z(z) \cdot q_\theta(\theta)) = \log p(X) - \mathcal{KL}(q_z  q_\theta \Vert p(Z, \theta | X)). 
\] 

\begin{enumerate}
\item[1)] Expectation step:
\[
q_z^{(t)}(z)  \propto \mathrm{exp} \left\{ \int \log p(X, z, \theta) q_\theta^{(t)}(\theta) d \theta \right\}
\]

В случае $p(X, Z | \theta) = h(X, Z) g(\theta) e^{\phi(\theta)^T u(X, z)}$
\[
q_z^{(t)}(z)  \propto h(X, Z) e^{\phi^T u(X, z)} , \quad \phi^{(t)} = \int  \phi(\theta) \; q_\theta^{(t)}(\theta) d \theta. 
\]
\item[2)] Maximization step:
\[
q_\theta^{(t+1)}(\theta)  \propto \mathrm{exp} \left\{ \int \log p(X, z, \theta) q_z^{(t)}(z) dz \right\}.
\]

\end{enumerate}

\end{enumerate}
\end{comment}

\begin{remark}

Для более глубокого ознакомления с модификациями EM алгоритма -- медианные модификации, SEM, CEM, MCEM, SAEM, выбор начального приближения, определение числа компонент и типа смеси -- рекомендуем ознакомится с работой  В.Ю. Королева ЕМ-алгоритм, его модификации и их применение к задаче разделения смесей вероятностных распределений.  

\end{remark}


\begin{problem}[Вариационный вывод]
\label{varinf}

Рассмотрим  задачу оценки некоторой статистики $T(X)$ для распределения с плотностью $p(X)$, т.е. величины 
\[
\Exp_p T(X) = \int  T(X)p(X)dX.
\] 
Предполагается, что  $p(X)$  известно с точностью до нормировочной константы ($S_p =  \int \tilde{p}(X)dX$ является недоступным):
\[
p ( X ) = \frac{1}{S_p} \tilde{p}( X ) \propto \tilde{p}( X ). 
\]
Например, для получения оценки скрытых параметров $Z$ распределения $p(X,Z)$ можно подсчитать  $\Exp_{p(Z|X)} Z$. В этом случае в качестве недоступной для вычисления нормировочной константы выступает $p(X)$. В задачах, решаемых при помощи ЕМ-алгоритма (см. задачу \ref{em}), в качестве ненормированного распределения $\tilde{p}$ выступает совместное распределение $p(X, Z |\theta)$, недоступной нормировочной константой является неполное правдоподобие $p(X |\theta)$, а искомой статистикой $T(X,Z)$ -- величина $\log p(X, Z | \theta)$.

Одним из вариантов решения задач такого рода является нахождение приближения $q(x)$ в некотором простом семействе распределений и последующая оценка статистики как $\Exp_p T(X) \approx \Exp_q T(X)$. Для нахождения $q(X)$ решается следующая оптимизационная задача  
\[
\mathcal{KL}(q \Vert p) \to \min_q
\Leftrightarrow
\int q(X) \log \frac{\tilde{p}(X)}{q(X)}dX \to \max_q.
\]
Получите для семейства полностью факторизованных распределений $q(X) = \prod_i q_i(x_i)$, $X = (x_1,\ldots,x_n)$ итерационный метод вычисления оптимальных $q_i$: 
\[
q_i(x_i) \propto \exp \left(\int \log \big[ \tilde{p}(X) \big] \prod_{j \neq i} q_j(x_j) dx_j \right), 
\quad i = \overline{1,n}.
\]
Рассмотрите случай экспоненциального распределения 
\[
p(x_i |X_{-i})=h(x_i) e^{ \langle \theta, f(x_i) \rangle - d(\theta) },\quad
\theta = \theta(X_{-i}).
\]
Докажите, что для этого случая справедливо упрощенное  соотношение для  $q_i$:
\[
q_i(x_i) = h(x_i) e^{ \langle \Exp\theta , f(x_i) \rangle - d(\Exp\theta) } , 
\quad i = \overline{1,n}.
\] 

\end{problem}

\begin{remark}
Сравним между собой два метода приближенной оценки статистик $T(X)$: вариационный вывод и методы MCMC. В методе MCMC оценка $\Exp_p T(X)$ является тем точнее, чем больше выборок $X$ генерируется, а в пределе является точной. В вариационном выводе нет никаких гарантий на близость между $\Exp_p T(X)$ и $\Exp_q T(X)$. В итерациях вариационного вывода происходит максимизация функционала $L(q)$, являющегося нижней оценкой для  $\log S_p$, что, как правило,  обеспечивает достаточно точную оценку на значение нормировочной константы даже при существенно ограниченном семействе распределений~$q$. Время работы одной итерации вариационного вывода и одной итерации схемы MCMC, обычно, очень близки. Однако, для сходимости вариационного вывода часто достаточно несколько десятков итераций, в то время как для надежной оценки статистик MCMC требует несколько тысяч итераций.
\end{remark}





Для того чтобы описать следующий цикл задач на метод \textit{зеркального спуска} 
А.С. Немировского, нам потребуется ввести ряд определений и сформулировать 
некоторые необходимые в дальнейшем результаты (см. раздел Вспомогательные материалы). Отметим, что одну задачу, 
которая может быть отнесена к задачам этого цикла мы уже встречали (см. 
задачу \ref{bandit} раздела \ref{information}).  Далее при изложении мы будем опираться в основном на работы

Lugosi G., Cesa-Bianchi N. Prediction, learning and games. -- New York: Cambridge University Press, 2006.

Вьюгин В.В. Математические основы теории машинного обучения и прогнозирования. -- М.: МЦНМО, 2013.

%\underline {http://www.iitp.ru/upload/publications/6256/vyugin1.pdf}

Гасников А.В., Нестеров Ю.Е., Спокойный В.Г. Об эффективности одного метода рандомизации зеркального спуска в задачах онлайн оптимизации. -- М.: Автоматика и телемеханика. 2014.



\begin{problem}
Рассмотрим задачу взвешивание экспертных решений. Имеется $n$ различных 
Экспертов. Каждый Эксперт играет на рынке. Игра повторяется $N\gg 1$ раз 
(это число может быть заранее неизвестно). Пусть $l_i^k $ -- проигрыш 
Эксперта $i$ на шаге $k$ ($\left| {l_i^k } \right|\le M)$. На каждом шаге 
$k$ мы распределяем один доллар между Экспертами, согласно вектору $x^k\in 
S_n \left( 1 \right)$. Потери, которые мы при этом несем, рассчитываются по 
потерям экспертов $\left\langle {l^k,x^k} \right\rangle $. Целью является 
таким образом организовать процедуру распределения доллара на каждом шаге, 
чтобы наши суммарные потери были бы минимальны. Допускается, что потери 
экспертов $l^k$ могут зависеть еще и от текущего хода $x^k$. Легко 
проверить, что для данной постановки применима теорема 1 в детерминированном 
варианте с функциями
\[
f_k \left( {x;\xi ^k} \right)\equiv f_k \left( x \right)=\left\langle 
{l^k,x} \right\rangle .
\]
Покажите, что оценка, даваемая теоремой 1, имеет вид
\[
{\rm O}\left( {M\sqrt {\frac{\ln n}{N}} } \right).
\]

Докажите, что данную оценку нельзя улучшить. 
\end{problem}

\begin{ordre} См. S. Bubeck, N. Cesa-Bianchi. Regret Analysis of Stochastic and Nonstochastic Multi-armed Bandit Problems. In Foundati- ons and Trends in Machine Learning, Vol 5, 2012. 
\end{ordre}

\begin{problem}
В условиях предыдущей задачи предположим, что на $k$-м шаге $i$-й эксперт 
использует стратегию $\zeta _i^k \in \Delta $ (множество $\Delta $ -- 
выпуклое), дающую потери $\lambda \left( {\omega ^k,\zeta _i^k } \right)$, 
где $\omega ^k$ -- ``ход'', возможно, враждебной ``Природы'', знающей, в том 
числе, и нашу текущую стратегию. Функция $\lambda \left( {\,\cdot \,} 
\right)$ -- выпуклая по второму аргументу и $\left| {\lambda \left( {\,\cdot 
\,} \right)} \right|\le M$. На каждом шаге мы должны выбирать свою стратегию
\[
x\mathop =\limits^{def} \sum\limits_{i=1}^n {x_i \cdot \zeta _i^k } \in 
\Delta ,
\]
дающую потери $\lambda \left( {\omega ^k,x} \right)$ так, чтобы наши 
суммарные потери были минимальны. Для данной постановки также применима 
теорема 1 в детерминированном варианте с
\[
f_k \left( {x;\xi ^k} \right)\equiv f_k \left( x \right)=\sum\limits_{i=1}^n 
{x_i \lambda \left( {\omega ^k,\zeta _i^k } \right)} \ge \lambda \left( 
{\omega ^k,x} \right).
\]
Покажите, что оценка, даваемая теоремой 1, имеет вид
\[
{\rm O}\left( {M\sqrt {\frac{\ln n}{N}} } \right).
\]
Отметим, что эта оценка для данного класса задач.

\end{problem}

\begin{ordre}
Чтобы применить теорему заметим, что функция $\lambda 
\left( {\omega ^k,\zeta } \right)$ -- выпуклая по $\zeta $ для любого 
$\omega ^k$, поэтому
\[
\sum\limits_{k=1}^N {\lambda \left( {\omega ^k,x^k} \right)} -\mathop {\min 
}\limits_{i=1,...,n} \sum\limits_{k=1}^N {\lambda \left( {\omega ^k,\zeta 
_i^k } \right)} \le \sum\limits_{k=1}^N {f_k \left( {x^k} \right)} -\mathop 
{\min }\limits_{x\in S_n \left( 1 \right)} \sum\limits_{k=1}^N {f_k \left( x 
\right)} .
\]
\end{ordre}

\begin{problem}
Предположим, что в условиях предыдущей задачи мы не можем 
гарантировать выпуклость $\lambda \left( {\,\cdot \,} \right)$ -- по второму 
аргументу. Тогда мы выбираем стратегию -- распределение вероятностей на 
множестве стратегий Экспертов, и разыгрываем случайную величину согласно 
этому распределению вероятностей. Другими словами мы просто пользуемся МЗС2 
c $f_k \left( {x;\xi ^k} \right)\equiv f_k \left( x 
\right)=\sum\limits_{i=1}^n {x_i \lambda \left( {\omega ^k,\zeta _i^k } 
\right)} $, применимость которого обосновывается теоремой 2. Получите оценки

${\rm O}\left( {M\sqrt {\frac{\ln n}{N}} } \right) $ -- в среднем;

${\rm 
O}\left( {M\sqrt {\frac{\ln \left( {n \mathord{\left/ {\vphantom {n \sigma 
}} \right. \kern-\nulldelimiterspace} \sigma } \right)}{N}} } \right)$ -- с 
вероятностью $\ge 1-\sigma $.

%Отметим, что эти оценки для данного класса задач. 
\end{problem}

\begin{problem}[Антагонистические матричные игры]

Пусть есть два игрока А и Б. Задана матрица игры $A=\left\| {a_{ij} } \right\|$, где $\left| 
{a_{ij} } \right|\le M$, $a_{ij} $ -- выигрыш игрока А (проигрыш игрока Б) в 
случае когда игрок А выбрал стратегию $i$, а игрок Б стратегию $j$. 
Отождествим себя с игроком Б. И предположим, что игра повторяется $N\gg 1$ 
раз (это число может быть заранее неизвестно). Мы находимся в условиях 
предыдущей задачи с $\lambda \left( {\omega ^k,\zeta _j^k } 
\right)=\sum\limits_{i=1}^n {\omega _i^k a_{ij} } $, то есть
\[
f_k \left( x \right)=\left\langle {\omega ^k,Ax} \right\rangle ,
\quad
x\in S_n \left( 1 \right),
\]
где $\omega ^k$ -- вектор\footnote{ Вообще говоря, зависящий от всей истории 
игры до текущего момента включительно, в частности, как-то зависящий и от 
текущей стратегии (не хода) игрока Б, заданной распределением вероятностей 
(результат текущего разыгрывания (ход Б) игроку А не известен).} со всеми 
компонентами равными 0, кроме одной компоненты, соответствующей ходу А на 
шаге $k$, равной 1. Хотя функция $f_k \left( x \right)$ определена на 
единичном симплексе, по ``правилам игры'' вектор $x^k$ имеет ровно одну 
единичную компоненту, соответствующую ходу Б на шаге $k$, остальные 
компоненты равны нулю. Обозначим цену игры

$C=\mathop {\max }\limits_{\omega \in S_n \left( 1 \right)} \mathop {\min 
}\limits_{x\in S_n \left( 1 \right)} \left\langle {\omega ,Ax} \right\rangle 
=\mathop {\min }\limits_{x\in S_n \left( 1 \right)} \mathop {\max 
}\limits_{\omega \in S_n \left( 1 \right)} \left\langle {\omega ,Ax} 
\right\rangle .$ (теорема фон Неймана о минимаксе)

Пару векторов $\left( {\omega ,x} \right)$, доставляющих решение этой 
минимаксной задачи, назовем равновесием Нэша. По определению (это 
неравенство восходит к Ханнану)
\[
\mathop {\min }\limits_{x\in S_n \left( 1 \right)} 
\frac{1}{N}\sum\limits_{k=1}^N {f_k \left( x \right)} \le C.
\]
Тогда, если мы (игрок Б) будем придерживаться рандомизированной стратегии 
МЗС2, то по теореме 2 с вероятностью $\ge 1-\sigma $ (в случае когда $N$ 
заранее известно оценку можно уточнить)
\[
\frac{1}{N}\sum\limits_{k=1}^N {f_k \left( {x^k} \right)} -\mathop {\min 
}\limits_{x\in S_n \left( 1 \right)} \frac{1}{N}\sum\limits_{k=1}^N {f_k 
\left( x \right)} \le \frac{2M}{\sqrt N }\left( {\sqrt {\ln n} +\sqrt {2\ln 
\left( {\sigma ^{-1}} \right)} } \right),
\]
т.е. с вероятностью $\ge 1-\sigma $ наши потери ограничены
\[
\frac{1}{N}\sum\limits_{k=1}^N {f_k \left( {x^k} \right)} \le 
C+\frac{2M}{\sqrt N }\left( {\sqrt {\ln n} +\sqrt {2\ln \left( {\sigma 
^{-1}} \right)} } \right).
\]
Самый плохой для нас случай (с точки зрения такой оценки) -- это когда игрок 
А тоже ``знает'' теорему 2, и действует согласно МЗС2 (точнее версии МЗС2 
для максимизации вогнутых функций на симплексе). Очевидно, что если и А и Б 
будут придерживаться МЗС2, то они сойдутся к равновесию Нэша, причем 
чрезвычайно быстро:

$\frac{8M\left( {\ln n+2\ln \left( {\sigma ^{-1}} \right)} 
\right)}{\varepsilon ^2}$ -- итераций;

${\rm O}\left( {n+M\frac{s\ln n\left( {\ln n+\ln \left( {\sigma ^{-1}} 
\right)} \right)}{\varepsilon ^2}} \right)\quad $ -- общая сложность вычислений,
где $s\le n$ -- среднее число элементов в строках и столбцах матрицы $A$. 
Докажите эти оценки.

\end{problem}


















\begin{comment}
%refactorise%
\begin{problem} [Метод зеркального спуска Немировского--Юдина] ** Рассмотрим задачу стохастической оптимизации
\[
\frac{1}{N}\sum\limits_{k=1}^N {\Exp_{\xi_k} \left[ {f_k \left( {x,\xi_k } \right)} 
\right]} \to \mathop {\min }\limits_{x\in S_n \left( 1 \right)} ,\quad S_n 
\left( 1 \right)=\left\{ {x\ge 0: \; \sum\limits_{i=1}^n {x_i =1} } 
\right\},
\]
где $\xi _k $ -- независимые случайные величины, $f_k \left( {x,\xi _k } 
\right)$:

\begin{enumerate}
\item выпуклые по $x$; 
\item $\left\| {\nabla _x f_k\left( {x,\xi_k } \right)} \right\|_\infty \le M$; 
\item  $\Exp\left[ {\nabla_x f_k \left( {x; \xi_k} \right)} \right]=\nabla _x \Exp\left[ {f_k \left( {x;\xi_k} \right)} \right].$ 
\end{enumerate}

С целью получить адаптивный алгоритм поиска оптимального $x^t$ при известных $\xi_1 \ldots \xi_t$ и уже вычисленных $x^1 \ldots x^{t-1}$ выполним линейную аппроксимацию функций $f_k \left( {x,\xi _k } 
\right)$:  

\[
\min \limits_{x\in S_n \left( 1 \right)} \sum\limits_{k=1}^t 
f_k(x) \approx \min \limits_{x \in S_n(1)} 
\sum\limits_{k=1}^t \left\{ f_k (x^{k-1})+ \left\langle 
\sum\limits_{k=1}^t \nabla f_k(x^{k-1}) , x-x^{t-1}  
\right\rangle  \right\}.
\]
Полагая при этом, что невязка $i$-й компоненты $\nabla f_k \left( x^{k-1} \right)$ равна случайной величине $\varepsilon_{k,i}$:
\[
\PR(x_t = e_j) \mathop{=}
\PR_\varepsilon \left( 
  j=\arg \max \limits_{i=1,...,n} \sum \limits_{k=1}^t 
  \left( 
    \left[ 
        -\nabla f_k \left( x^{k-1} \right) 
    \right]_i + \varepsilon_{k,i}  
  \right) 
\right),
\]
получим, при довольно общих условиях относительно с.в. $\varepsilon _{k,i} $ 
(типа i.i.d.), что если $t\gg 1$, то
\[
\begin{array}{c}
 P_\varepsilon \left( {j=\arg \mathop {\max }\limits_{i=1,...,n} 
\sum\limits_{k=1}^t {\left\{ {\left[ {-\nabla f_k \left( {x^{k-1} } 
\right)} \right]_i +\varepsilon _{k,i} } \right\}} } \right)\approx \\ 
 \approx P_\varsigma \left( {j=\arg \mathop {\max }\limits_{i=1,...,n} 
\left\{ {\sum\limits_{k=1}^t {\left[ {-\nabla f_k \left( {x^{k-1} } 
\right)} \right]_i } } \right\}+\varsigma _{t,i} } \right), \\ 
 \end{array}
\]
где с.в. $\varsigma _{t,i} $ -- i.i.d. с распределением Гумбеля 
(max-устойчивым), с параметром, зависящим от $t$: $P\left( {\varsigma _{t,i} 
<\tau } \right)=\exp \left\{ {-e^{-\tau \mathord{\left/ {\vphantom {\tau 
{\beta _t }}} \right. \kern-\nulldelimiterspace} {\beta _t }}} \right\}$, $\beta _t =\frac{M}{\sqrt {\ln n} }\sqrt {t+1}$
(см. задачу \ref{gumbel}).

Тогда
\[
E_\varsigma \left[ {x^t} \right]=-W_{\beta _t } \left( {\sum\limits_{k=1}^t 
{\nabla f_k \left( {x^{k-1} } \right)} } \right),
\]
где \[W_\beta \left( y \right)=\beta \ln \left( 
{\frac{1}{n}\sum\limits_{i=1}^n {\exp \left( {-\frac{y_i }{\beta }} \right)} 
} \right).\]

Используя результат задачи \ref{gibbs}, получите конечную формулу для независимого вычисления каждой из компонент $x^t$:

\[
x_i^t \propto \exp \left( {-\frac{1}{\beta_t 
}\sum\limits_{k=1}^{t-1} {\frac{\partial f_k \left( {x^{k-1},\xi _k } 
\right)}{\partial x_i }} } \right).
\]

\begin{remark}
Полученный алгоритм является стохастическим вариантом метода 
зеркального спуска (экспоненциального взвешивания в наших условиях).

Стоит подчеркнуть, что независимость вычислений каждой из компонент $x^t$ позволяет использовать распределенные вычисления $x^t$, предварительно разослав (broadcast) $x^{t-1}$ на каждый из рабочих узлов кластера.  
\end{remark}

Известно (см., например, работу Юдицкий А.Б., Назин А.В., Цыбаков А.Б., 
Ваятис Н. Рекуррентное агрегирование оценок методом зеркального спуска с 
усреднением // Пробл. передачи информ., 2005. Т. 41:4. стр. 78--96), что

\[
\sum\limits_{k=1}^t {\gamma _k \left\{ {E_{\xi ,x} \left[ {f_k \left( 
{x^{k-1};\xi_k} \right)} \right]-E_{\xi} \left[ {f_k \left( {x;\xi_k} 
\right)} \right]} \right\}} \le 
\]
\begin{equation}
\label{terrible}
\begin{array}{c}
 \le \sum\limits_{k=1}^t {\gamma _k \left( {x^{k-1}-x} \right)^T} \nabla _x 
E_{\xi ,x} \left[ {f_k \left( {x^{k-1};\xi_k} \right)} \right]\le \beta _t 
V\left( x \right)- \\ 
 -\sum\limits_{k=1}^t {\gamma _k \left( {x^{k-1}-x} \right)^T} \left( 
{\nabla _x f_k \left( {x^{k-1};\xi_k} \right)-\nabla _x E_{\xi ,x} \left[ 
{f_k \left( {x^{k-1};\xi_k} \right)} \right]} \right)+ \\ 
 +\sum\limits_{k=1}^t {\frac{\gamma _k^2 }{2\beta _{k-1} }\left\| {\nabla _x 
f_k \left( {x^{k-1};\xi_k} \right)} \right\|_\infty ^2,} \\ 
 \end{array}
\end{equation}
где  $V\left( x \right)=\ln n+\sum\limits_{i=1}^n {x_i \ln x_i }$.

Введем компактное обозначение для минимума целевой функции
\[
{\rm F_{min}} = \mathop {\min }\limits_{x\in S_n \left( 1 \right)} 
\frac{1}{N}\sum\limits_{k=1}^N {\Exp_{\xi_k} \left[ {f_k \left( {x,\xi _k } 
\right)} \right]}
\]

Получите следующие оценки для среднего отклонения от минимального значения 
целевой функции:

$$
\Exp_{x} \left( \frac{1}{N}\sum\limits_{k=1}^N {\Exp_{\xi_k} \left[ {f_k \left( {x^{k-1},\xi 
_k } \right)} \right]} \right) -\rm{F_{min}} \le 2M\sqrt {\frac{\ln n}{N}},
$$

a если все $f_k \equiv f$ и $\xi_k$ -- независимы и одинаково распределены, то

\[
\Exp_{x}  {\Exp_{\xi} \left[ {f\left( 
{\frac{1}{N}\sum\limits_{k=1}^N {x^{k-1}} ,\xi } \right)} \right] - {\rm F_{min}} } \le 2M\sqrt {\frac{\ln n}{N}}.  
\]

Также получите вероятностные оценки отклонения от минимального значения 
целевой функции на величину  
$\delta(\Omega) = \frac{2M}{\sqrt N } \left({\sqrt {\ln n} +\sqrt {3\Omega }  }\right)$:

\[
\PR_{x} 
\bigg[
    \frac{1}{N}\sum\limits_{k=1}^N \Exp_{\xi _k} 
    \left[ {
      f_k \left( {x^{k-1},\xi_k } \right)
    } \right] 
    - {\rm F_{min}}  \ge  
   \delta(\Omega)
\bigg] 
\le \exp \left( {-\Omega}  \right),
\]
\[
\PR_{x}
\Bigg[
  \Exp_\xi \left[{
     f\left({
        \frac{1}{N}\sum\limits_{k=1}^N {x^{k-1}} ,\xi 
     } \right)
  } \right]
- {\rm F_{min}} \ge
\delta(\Omega)
\Bigg]
\le \exp \left( {-\Omega } \right).
\]
\end{problem}

\begin{ordre}  
Для получения оценок вероятностей больших уклонений, 
воспользуйтесь п. б)  задачи \ref{sec:mirrorDescent} из раздела \ref{measure}. 
\end{ordre}  

\begin{remark} 
Важно отметить, что описанный в этой задаче адаптивный 
метод помимо приложений в статистике имеет широкие приложения в задачах 
оптимального взвешивания экспертных решений (см., например, Вьюгин В.В. Элементы 
математической теории машинного обучения. М.: МФТИ, 2010, глава 3) и в 
задачах о многоруких бандитах (см., например, \textit{Lugoshi G., Cesa-Bianchi N}. Prediction, learning and 
games. New York: Cambridge University Press, 2006, а также Juditsky A., 
Nazin A.V., Tsybakov A.B., Vayatis N. Gap-free Bounds for Stochastic 
Multi-Armed Bandit // IFAC World congress, 2008). Удивительно, что во всех 
случаях, описанный метод дает (с мультипликативной точностью до константы) 
не улучшаемые оценки (только в случае многоруких бандитов c 
мультипликативной точностью $\sim \sqrt {\ln n} )$.

Для того чтобы применять этот метод к многоруким бандитам: $f_k \left( 
{x,\xi _k } \right)=r_i^k $ с вероятностью $x_i $, $i=1,...,n$; а чтобы 
$\nabla _x \Exp\left[ {f_k \left( {x,\xi _k } \right)} \right]=a=\Exp\left[ {r^k} 
\right]$, выбирают $\nabla _x f_k \left( {x,\xi _k } \right)=(\underbrace 
{0,..,r_i^k}_i{} \mathord{\left/ {\vphantom {{r_i^k } {x_i }}} \right. 
\kern-\nulldelimiterspace} {x_i },...,0)^T$ с вероятностью $x_i $, 
$i=1,...,n$, где $r_i^k \ge 0$ -- потери (regret), которые выдает 
$i\mbox{-я}$ ручка, если её дернуть на шаге $k$. При этом вместо слагаемых

$\frac{\gamma _k^2 }{2\beta _{k-1} }\left\| {f_k \left( {x^{k-1};\xi ^k} 
\right)} \right\|_\infty ^2 $ в выражении (\ref{terrible}) необходимо писать точнее \[\gamma _k^2 
\frac{x_j^{k-1} \left( {1-x_j^{k-1} } \right)}{\beta _{k-1} }\left( {\frac{r_j^k 
}{x_j^{k-1} }} \right)^2,\]
где $j$ -- номер ручки, выбранной алгоритмом на $k$-м шаге.

Для задач оптимального рандомизированного взвешивания экспертных решений 
$\Exp_\xi \left[ {f_k \left( {x,\xi_k } \right)} \right]=\sum\limits_{i=1}^n 
{\lambda \left( {\omega _k ,\zeta _k^i } \right)x_i } $, где $\lambda \left( 
{\omega _k ,\zeta _k^i } \right)$ -- потери Эксперта $i$, выбравшего на шаге 
$k$ стратегию $\zeta _k^i $, при ходе ``сопротивляющейся Природы'' $\omega 
_k $ (на каждом шаге первым выбирают свои ходы Эксперты, потом мы, потом 
природа; но есть важный нюанс: наш ход заключается в выборе распределения 
вероятностей, которое становится известным Природе, но разыгрывание согласно 
этому распределению вероятностей происходит после того, как Природа выбрала 
свой ход.

Тонкая разница в постановке этих двух задача (``стоящая'' $\sqrt n )$, 
заключается в том, что в многоруких бандитах игрок имеет только свою историю 
дергания ручек (ему не известно, какой бы \textit{regret} принесли ему другие ручки, кабы 
он их выбрал), а в постановке взвешивания экспертных решений это все 
известно и называется потерями экспертов.

В случае решения детерминированных разреженных задач выпуклой оптимизации в 
пространствах огромных размеров, как, например, задачи о ранжировании 
web-страниц

\begin{center}
http://goo.gl/HLM6w
\end{center}
важную роль играет искусственное введение случайности (рандомизация), позволяющее найти $x.\mathrm{top}(\cdot)$, проводя вычисления с разреженным вектором $x$:

\[
x_{i\left( t \right)}^t =x_{i\left( t \right)}^{t-1} +1,\;\;x_j^t =x_j^{t-1} 
,\;j\ne i\left( t \right), 
\]
где
\[
p_i^t \propto \exp \left( {-\frac{1}{\beta _t }\sum\limits_{k=1}^{t-1} 
{\gamma _k \frac{\partial f_k \left( {x^{k-1},\xi _k } \right)}{\partial x_i }} 
} \right),\quad i=1,...,n,
\]

Для этого алгоритма $\beta _t $ и $\gamma _t $ разумнее брать постоянными:
\[
\beta _t \equiv 1,
\quad
\gamma _t \equiv M^{-1}\sqrt {{2\ln n} \mathord{\left/ {\vphantom {{2\ln n} 
N}} \right. \kern-\nulldelimiterspace} N} .
\]
При этом сохраняются все приведенные в задаче оценки, но, к сожалению, 
теряется адаптивность, то есть теперь мы должны знать заранее либо требуемую точность, либо число шагов.
\end{remark}

\end{comment}


\section{Геометрические вероятности}
\label{geom}

\begin{problem}

Три бабочки капустницы садятся на круглый кочан капусты радиуса 1 случайным образом (имеется в виду, что место положение каждой бабочки -- с.в., равномерно распределенная на сфере) и независимо друг от друга. Если между двумя бабочками (геодезическое) расстояние оказывается меньше ${\pi \mathord{\left/ {\vphantom {\pi  2}} \right. \kern-\nulldelimiterspace} 2} $, то обе улетают. Найдите вероятность того, что на капусте сидят все три бабочки.

\end{problem}

\begin{problem}

Выбирается случайно и равномерно $n$ точек $P_{1} ,\ldots ,P_{n} $ на единичной окружности. Какова вероятность того, что начало координат (цент окружности) окажется внутри выпуклой оболочки этих точек.

\end{problem}

\begin{ordre} 
Выберите $n$ случайных пар диаметрально противоположных точек $Q_{1} ,Q_{n+1} =-Q_{1} $, $Q_{2} ,Q_{n+2} =-Q_{2} $, \dots , $Q_{n} ,Q_{2n} =-Q_{n} $ в соответствии с равномерным распределением. Ясно, что с вероятность 1 все пары различны. В качестве точки $P_{i} $ равновероятно выбирается либо точка $Q_{i} $, либо диаметрально противоположная ей $Q_{n+i} =-Q_{i} $. Покажите, что такая процедура эквивалентна случайному выбору точек $P_{i} $. Покажите, что вероятность того, что начало координат не окажется внутри выпуклой оболочки точек $P_{1} ,\ldots ,P_{n} $, при заданных различных точках $Q_{1} ,\ldots ,Q_{n} ,Q_{n+1} ,\ldots ,Q_{2n} $ равна $\frac{2n}{2^{n} } $, так как нужные точки $P_{1} ,\ldots ,P_{n} $ могут давать только подмножества вида $\left\{\tilde{Q}_{i} ,\ldots ,\tilde{Q}_{i+n-1} \right\}$(суммирование в индексах берется по модулю $2n$), где $\tilde{Q}_{1} ,\ldots ,\tilde{Q}_{2n} $ перенумерованные, например по часовой стрелке, точки $Q_{1} ,\ldots ,Q_{2n} $.
\end{ordre}

\begin{problem}
Найти среднюю длину секущих трехмерного куба с единичной длиной.
\end{problem}

\begin{problem}[Парадокс Бертрана]
Рассмотрим окружность, описанную около равностороннего треугольника ABC. Какова вероятность того, что случайным образом проведенная хорда будет иметь длину большую, чем длина стороны треугольника ABC? 
\end{problem}

\begin{ordre}
Предложите разные способы генерации хорды (не менее трех). Зависит ли ответ на задачу от способа генерации?
\end{ordre}

\begin{problem}
Пусть в пространстве $\mathbb R^n$ с евклидовой нормой задан $n$-мерный шар единичного радиуса. Внутри него имеются две случайные точки с радиус-векторами ${\bf{r}}_1$ и ${\bf{r}}_2$ соответственно, имеющие равномерное пространственное распределение внутри шара. Найти распределение случайной величины, являющейся расстоянием между этими двумя точками $r = \left|{\bf r}_1 - {\bf r}_2\right|$.
\end{problem}




\begin{problem}
На плоскости проведены параллельные прямые на единичном расстоянии друг от друга, и на плоскость наугад бросается иголка длиной $L<1$. 
Угол между прямыми и иголкой и расстояние от середины иглы до ближайшей прямой являются независимыми с.в., равномерно распределенными 
на $(0,2\pi)$ и $(-1/2,1/2)$ соответственно. С помощью серии таких опытов вычислить число $\pi$ с заданной точностью 
$\delta=1\%$ и с вероятностью ошибки не больше $\varepsilon=5\%$. Решите аналогичную задачу для случая погнутой иглы длиной менее 1.
\end{problem}

\begin{ordre}

Рассмотрим окружность диаметра $1$, т.е. длины $\pi$. Такая окружность с вероятностью $1$ пересекает дважды одну из прямых. 
Тогда, исходя из линейности математического ожидания числа попаданий иглы на прямую относительно длины иглы, для иглы длиной $L<1$ 
имеем ${\mathbb E}\xi_L = 2L/\pi$. 

\end{ordre}



\begin{problem}
Покажите, что средняя площадь ортогональной проекции куба с ребром единица на случайную плоскость равна $3/2$. 
\end{problem}

\begin{ordre}
Покажите, что  средняя площадь  ортогональной проекции всякого измеримого тела 
линейно зависит от площади его границы. 
Рассмотрим вспомогательное (см. предыдущую задачу) тело, у которого легко вычисляется средняя площадь ортогональной проекции. 
\end{ordre}

\begin{remark}
Обозначим через $S_k$ $k$-мерный объем ортогональной 
проекции рассматриваемой области $V$ в ${\mathbb R}^n$ на случайную $k$-мерную 
плоскость. Имеет место следующая формула для объема $h$-окрестности данной области:
\[
V\left( h \right)=V_0 +V_1 h+V_2 h^2+...+V_n h^n,
\]
где $V_0 $ -- объем области; $V_1 $ -- $(n-1)$-мерный объем границы области, 
пропорциональный среднему значению от числа 1; число $V_k $ пропорционально 
$S_k $ и выражается через средние значения (усредненным по поверхности рассматриваемой области)  от произведений $k$ главных 
кривизн. 

В случае $n = 3$, из главных кривизн $k_1 $ и $k_2 $ в каждой точке 
можно составить \textit{среднюю кривизну} $k_1 +k_2 $ и \textit{гауссову кривизну} $K=k_1 k_2 $. В этом случае объем 
$h$-окрестности получается $V\left( h \right)=V_0 +V_1 h+V_2 h^2+V_3 h^3$, где 
$V_2 $ пропорционален интегралу от средней кривизны по всей поверхности, а $V_3 $ -- от гауссовской:
\[
V_3 =\frac{4}{3}\pi \int\!\!\!\int {KdS} .
\]
Например, для сферы радиуса $R$
\[
V\left( h \right)=\frac{4}{3}\pi \cdot \left( {R+h} \right)^3=\frac{4}{3}\pi 
R^3+h\cdot \left( {4\pi R^2} \right)+h^2\left( {4\pi R} 
\right)+\frac{4}{3}\pi h^3.
\]
Здесь $k_1 + k_2 = \frac{2}{R}$, $k_1 k_2 = \frac{1}{R^2}$,
\[
\int  (k_1 +k_2)   dS = 8\pi R,
\]
Формула Гаусса-Бонне:
 \[\int\!\!\!\int {\left( {k_1 k_2 } \right)dS} =4\pi. \] 


\end{remark}


\begin{problem}
Приведем геометрическую интерпретацию пуассоновского процесса (см. задачу \ref{sec:poisson}). Пуассоновским процессом $\text{П}$ в пространстве $S \subset \mathbb{R}^m$ называется  счетное множество точек, случайно разбросанных по $S$, но подчиняющихся следующему правилу: существует мера $\mu: S \to [0, \infty]$, соответствующая процессу $\text{П}$, такая что для любых непересекающихся 
измеримых множеств $A_1,\ldots,A_n \subset S$ случайные величины 
\[
N(A_i) = \# \{ A_i \cap \text{П}\} \sim  \Po(\mu(A_i)), \quad i = \overline{1,n},
\] 
порожденные случайным попаданием точек в множества $A_1,\ldots,A_n$, независимы и имеют распределение $\Po(\mu)$. 

К примеру стандартный пуассоновский процесс из задачи \ref{sec:poisson} раздела \ref{zb4} определен в пространстве $S = \mathbb{R}_{+}$, имеет меру $\mu \big( (t_1, t_2] \big) = \lambda ( t_2 - t_1)$, $N( (t_1, t_2] ) = K(t_2) - K(t_1) \sim \Po \big(\lambda (t_2 - t_1) \big)$.

Отметим, что определение $\text{П}$ требует, чтобы мера $\mu$ была \textit{неатомической} (значение на любом счетном множестве равно 0), а также представимой в следующем виде
\[
\mu = \sum_{k = 1}^{\infty} \mu_{k}, \quad \mu_k(S) < \infty.
\]
Пусть $X_1,\ldots,X_n$ -- независимые случайные величины, распределенные по $A \subset S$, $\mu(A) < \infty$ в соответствии с вероятностной мерой $p(\cdot) = \mu(\cdot) / \mu(A)$, Обозначим за $N(B)$ количество $X_i \in B$. Покажите, что для непересекающихся $A_1,\ldots,A_k  \subset A$, $A = A_1 \cup\ldots\cup A_k $ выполнено  
\[
\PR_n(N(A_1) = n_1,\ldots,N(A_k) = n_k) = \frac{n!}{n_1!,\ldots,n_k!} p(A_1)^{n_1} \ldots p(A_k)^{n_k}.
\]
Докажите, что если $X_1,\ldots,X_n$ -- точки пуассоновского процесса с мерой $\mu$ внутри множества $A$, то справедливо равенство
\[
\PR(\cdot | N(A) = n) = \PR_n(\cdot).
\]
\begin{remark}
Последнее выражение утверждает, что при фиксированном числе точек $N(A)$ внутри множества $A$, сами точки пуассоновского процесса выглядят как $N(A)$ независимых случайных величин с плотностью распределения $p(\cdot) = \mu(\cdot) / \mu(A)$. 
Таким образом, пуассоновский процесс является неизбежным следствием моделирования системы с большим числом независимых точек в пространстве~$\mathbb{R}^m$.
\end{remark} 


\begin{problem}
\label{poi_proj}
Однородный пуассоновский процесс $\text{П}$ определен в пространстве $S = \mathbb{R}^{2}$ и имеет интенсивность $\lambda$, т.е. $\mu (A)  = \int_{A} \lambda dxdy$. Осуществим переход к полярным координатам $(r, \theta)$ при помощи преобразования 
\[
f(x, y) = \left( 
(x^2 + y^2)^{1/2}, \; \arctan (y/x)
\right).
\]
Покажите, что образы точек $\text{П}$  образуют пуассоновский процесс в полосе
\[
\{(r,\theta): \; r > 0, \; 0 \leq \theta <  2 \pi \},
\]
который имеет интенсивность $\lambda^{*}(r) = \lambda r$.
Покажите, что значения $r$, соответствующие $\text{П}$, образуют пуассоновский процесс в $(0, \infty)$ с интенсивностью $2 \pi \lambda r$. 

\end{problem}

\begin{remark}
Сформулируем правило отображения $\text{П}$ при помощи произвольного правила преобразования координат $f: S \to T$: если мера $\mu$ процесса $\text{П}$ является $\sigma$-конечной ($S$ можно представить в виде $\cup_n S_n$, где $\mu(S_n) < \infty$), помимо того мера на множестве образов $T$ 
\[
\mu^{*}(B) = \mu( f^{-1} (B))
\]  
является неатомической, то $f(\text{П})$ -- пуассоновский процесс в пространстве $T$ c  мерой интенсивности $\mu^{*}$.
\end{remark}

\begin{problem}
Используя результат предыдущей задачи, покажите, что плотность распределения упорядоченных расстояний $r_{(1)}, r_{(2)}, \ldots$ имеет следующий вид
\[
f_{r_{(s)}} (r) = \frac{2 (\lambda \pi)^s r_{}^{2s - 1} e^{-\lambda \pi r_{}^2}}{ (s-1)! }. 
\]  
В таком случае, $2 \lambda \pi r_{(s)}^2$ распределено как $\chi_{2s}^{2}$. Данный результат может быть использован для оценки плотности точек на плоскости путем выбора случайных точек и измерения расстояния до 1-го ближайшего соседа. Пусть $X_1, \ldots, X_n$ -- $n$ реализаций $r_{(1)}^{2}$. Покажите, что $2\lambda \pi n \overline{X}$ распределено как $\chi_{2n}^{2}$, предложите оценку для $\lambda$ и вычислите ее дисперсию.
\end{problem}

\begin{remark}
Аналогичный подход с измерением переменной плотности точек $\lambda(\cdot)$ продемонстрирован в задаче $\ref{sort_dir}$ раздела $\ref{bayes}$  c использованием упорядоченного распределения Дирихле.
\end{remark}

\begin{problem}
Сопоставим каждой точке $X \in S$ случайного множества  $\text{П}$ (пуассоновский процесс) случайную величину $m_X$ (метку), принимающую значения из множества $M$.
Распределение $m_X \sim p(X, m)$ может зависеть от  $X$, но не зависит от других точек $\text{П}$ и их меток. Докажите, что случайное множество
\[
\text{П}^{*} = \{(X, m_X): \; X \in \text{П}\}
\]
является пуассоновским процессом на множестве $S \times M$ c мерой интенсивности
\[
\mu^{*}(C) = \mathop{\int\int} \limits_{(x,m) \in C} \mu(dx) p(x, m) dm.
\]
Используя замечание к задаче \ref{poi_proj}, убедитесь что метки  $m_X$ образуют пуассоновский процесс на $M$ с мерой интенсивности 
\[
\mu_m(B) = \int_{S}\int_{B}  \mu(dx) p(x, m) dm. 
\]
\end{problem}

\begin{ordre}
Докажем, что $\text{П}$ однозначно задается набором характеристических функционалов вида
\[
\Exp(e^{\Sigma_f}) = \exp \left( - \int_{S}(1 - e^{-f(x)}) \mu(dx)  \right),
\]
где 
\[
\Sigma_f = \sum_{X \in \text{П}} f(X), \quad f \in \mathcal{F},
\]
$\mathcal{F}$ -- класс функций, содержащий все индикаторные функции для измеримых множеств из $S$. Выбрав набор непересекающихся множеств $A_1,\ldots,A_k$, сопоставим 
каждому множеству свое значение $f_i$. Тогда
\[
\Sigma_f  = \sum_{i=1}^k f_i N(A_i), \quad
\Exp(e^{\Sigma_f}) = \exp \left( - \sum_{i=1}^k (1 - e^{-f_i}) \mu(A_i)  \right).
\]    
Сделав замену $z_i  = e^{-f_i}$, получаем
\[
\Exp \left(z_1^{N(A_i)} \ldots z_k^{N(A_k)} \right) = \prod_{i=1}^k e^{\mu (A_i) (z-1)},
\]  
что свидетельствует о независимости $N(A_i)$ и принадлежности распределению Пуассона.
\end{ordre}

\end{problem}

\begin{problem}
\label{subord}
Случайный процесс $\varphi(t)$ называется \textit{субординатором} (процесс Леви с положительными приращениями), если он удовлетворяет следующим свойствам: приращения  процесса независимы, положительны и зависят только от приращения аргумента $t$. Докажите, что  справедливо следующее представление субординатора 
\[
\varphi(t) = \beta t + \sum_{\tau} \{z: \; (\tau, z) \in  \text{П}, \; 0 < \tau < t  \},
\]
где суммируются значения второй координаты пуассоновского процесса $\text{П}$ с мерой интенсивности
\[
\mu(d\tau, dz) = d\tau \gamma(dz). 
\]
Мера $\gamma(dz)$ соответствует скачкам процесса  $\varphi(t)$, образующем пуассоновский процесс по координате $z$, и может быть найдена из выражения (представление Леви-Хинчина)
\[
\Exp  e^{ -s \varphi(t)} = \exp \left(
-t \left[ s\beta - \int_{0}^{\infty} (1 - e^{-s z}) \gamma(dz)  \right]
 \right). 
\] 
\end{problem}

\begin{ordre}
Используя результат предыдущей задачи, покажите, что для чисто атомической меры на $S$, определенной как  
\[
\Psi(A)  = \sum_{\tau} \{z: \; (\tau, z) \in  \text{П}, \;  \tau \in A\} 
\]  
характеристическая функция имеет вид
\[
\Exp  e^{ -s \Psi(A) } = \exp \left(
- \int_{0}^{\infty} (1 - e^{-s z}) \gamma(A, dz) 
 \right),
\]
где $\gamma(\cdot, \cdot)$ --  мера процесса $\text{П}$.

Покажите, что для случайной неатомической меры $\Phi$ с независимыми значениями на непересекающихся множествах $\Phi(A)$ есть безгранично делимая с.в. и для нее  справедливо представление  Леви-Хинчина: 
\[
\Exp  e^{ -s \Phi(A)} = \exp \left(
 - s\beta(A) + \int_{0}^{\infty} (1 - e^{-s z}) \gamma(A, dz) 
 \right). 
\]  
Далее, $\varphi(t)$ может быть представлен как случайная мера $\Phi(0,t]$ при значениях $t \geq 0$. Убедитесь, что такая мера является неатомической, используя свойство инвариантности $\Phi$ относительно сдвига. 
\end{ordre}

\begin{problem}
Рассмотрим звезды, находящиеся на расстоянии, не превышающем $R$ от наблюдателя. Для простоты будем считать, что все звезды имеют одинаковый диаметр $\delta$ и равномерное пространственное распределение с количеством звезд $\lambda$ на единицу объема. Показать, что при $R \rightarrow \infty$ любой участок неба будет полностью светящимся.   

\begin{remark}
В действительности такое явление не наблюдается. В связи с конечным возрастом вселенной ($14 \cdot 10^9$ лет) ее радиус ограничен величиной $ct$.
\end{remark}
\end{problem}

\begin{problem}[Формула Крофтона]
Пусть  $N$ точек независимо распределены в области $D$ $n$-мерного пространства, $P$ -- вероятность того, что фигура $F$, образованная $N$ точками, обладает определенным свойством,
зависящим только от взаимного расположения точек. Область $D$ является измеримой по Лебегу и ее мара равна $V$. Обозначим как $P_1$ вероятность того, что $F$ обладает требуемым свойством для случайных точек в области $D_1 \supset D$. Докажите следующее соотношение для малых приращений $\delta V$:
 \[
 \delta P = N (P_1 - P) V^{-1} \delta V .
 \]  

\end{problem}

\begin{problem}
Вычислите распределение  расстояния между точками, случайно взятыми внутри круга радиуса $R$, воспользовавшись формулой Крофтона.
\end{problem}
\begin{ordre}
Пусть $f(x,R)dx$ есть вероятность того, что расстояние между точками $A$ и $B$ принадлежит интервалу  $(x, x + dx)$. Для случая, когда $A$ лежит на границе
\[
f(x,R) = \frac{ 2 \theta x}{\pi R^2}, \quad \theta = \arccos (x/2R).
\]
Тогда уравнение Крофтона примет вид
\[
\frac{df}{d\theta} + 4 f \tg \theta = \frac{32 \theta}{\pi x} \sin \theta \cos \theta.
\]
\end{ordre}



\begin{problem}[Теорема Дворецкого]
\label{sec:dvor}
Доказать, что для любого $\epsilon > 0$ и $k \in \mathbb{N}$
существует $N = N(k, \epsilon) < \exp(C\frac{\log \epsilon}{\epsilon^2}k)$ такое, что любое конечномерное банахово пространство ($X, \Vert\cdot\Vert)$, где $\dim X > N$, содержит $k$-мерное подпространство $E$, являющееся $\epsilon$-евклидовым, т.е. в нем можно задать такую норму $\vert \cdot \vert$, что $\Vert x \Vert \leqslant \vert x \vert \leqslant (1 + \epsilon) \Vert x \Vert$ $\forall x \in E$.     

\begin{remark} См. работу В. Д. Мильман, “Новое доказательство теоремы А. Дворецкого о сечениях выпуклых тел”, Функц. анализ и его прил., 5:4 (1971), 28–37. 

\end{remark}

\begin{comment}
НАЗАР, СЮДА СТОИТ ВСТАВИТЬ ЗАМЕЧАНИЕ ИЗ СТАТЬИ GOWERS'а
И ДОБАВИТЬ В ВИДЕ ЗАДАЧИ ТЕОРЕМУ Б.С, Кашина - см. статью того же Gowers'a и ссылки в том моем письме
\end{comment}


\end{problem}

\begin{problem}
Выберем наугад (равновероятно) $k$ вершин $m$-мерного куба $[0,1]^m$. Обозначим как $X$ выпуклую оболочку выбранных вершин. Пусть $p_{km}$ - вероятность того, что все вершины многогранника попарно смежны. Докажите справедливость следующей оценки при $m > 3$:

\[
p_{km} > 1 - \frac{k^4 \cdot 5^m}{4 \cdot 8^m}
\]
    
\end{problem}
\begin{remark}
См. монографию Бондаренко В.А., Максименко А.Н. Геометрические конструкции, сложность в комбинаторной оптимизации. -- М.: УРСС, 2008.
\end{remark}


\begin{problem}
На плоскости нарисована выпуклая фигура, ограниченная кривой длины $L$. Докажите, что ее диаметр, т.е. максимальное расстояние между двумя ее точками, не меньше $\frac{L}{\pi }$.
\end{problem}
\begin{ordre}
Проведите в случайном направлении прямую. Покажите, что 
математическое ожидание длины проекции фигуры на случайное направление равно 
$\frac{L}{\pi }$.
\end{ordre}

\begin{problem}
В московском метро можно провозить коробки, у которых сумма измерений (длины, ширины и высоты) не превосходит некоторой границы. Можно ли перехитрить правила, поместив одну коробку в другую (сумма измерений внутренней коробки больше суммы измерений внешней)?
\end{problem}
\begin{ordre}
Спроектируйте коробку на случайно выбранное (в пространстве) 
направление. Длина проекции коробки складывается из проекций отрезков, 
идущих по ее высоте, длине и ширине. Проекция внутренней коробки не 
превосходит проекции внешней.
\end{ordre}

\begin{problem}
Несамопересекающаяся кривая длины 22 находится внутри круга радиуса 1. Докажите, что найдется прямая, имеющая с этой кривой по крайней мере 8 общих точек.
\end{problem}
\begin{problem}
Известно, что более половины поверхности Земли занимают океаны. Докажите, что можно найти две диаметрально противоположные точки, обе попавшие в океан.
\end{problem}


\begin{problem}
На плоскости расположено $2n$ векторов, выходящих из начала координат и длиной не более 1. Доказать, что существует угол $\alpha$ такой, что при повороте каждого из векторов на угол $\pm \alpha$, их векторная сумма окажется не большей 1.  
\end{problem}

\begin{problem}[случайный поворот куба, мера Хаара] 10{\%} поверхности 
шара (по площади) выкрашено в черный цвет, остальные 90{\%} -- белые. 
Докажите, что можно вписать в шар куб таким образом, чтобы все вершины куба 
попали в белые точки.
\end{problem}

\begin{problem}[Теорема Уилкса]
Для выборки $X \sim f(\theta, X)$, $\dim X = n$ осуществляется проверка гипотезы $H_0: \theta = \theta_0$, для чего используется обобщенный критерий отношения правдоподобия. 
Определим статистику для критерия как 
\[
W(\theta_0) = \log f(\widehat{\theta}, X) - \log f(\theta_0, X),
\quad  \widehat{\theta} = \arg\max_{\theta \in \Theta} f(\theta, X).
\] 
Теорема Уилкса утверждает, что в случае истинности гипотезы  $H_0$ и при асимптотической нормальности оценки $\widehat{\theta}$, статистика  
$2W(\theta_0)$ сходится по распределению к $\chi_{p}^2$, где $p = |\Theta|$. 
Тогда критерием отклонения гипотезы $H_0$ будет $[2W(\theta_0) > x]$, $\PR(\chi_{p}^2 > x) < \alpha$. 

Существует также обобщение теоремы Уилкса для случая, когда $\widehat{\theta}$ не является асимптотически нормальной. При довольно слабых дополнительных ограничениях для выполнения теоремы Уилкса достаточно, чтобы поверхности уровня  $S_w = \{\theta: W(\theta) = w\}$ в пределе при $n \to \infty$ имели форму
\[
S_w  \approx \widehat{\theta} + a_n w^r S
\]  
где $a_n \to 0 $ при $n \to \infty$, $S = \{\theta: h(\theta) = 1 \}$, $h(t\theta) = t^{1/r} h(\theta)$. В частности, при асимптотической нормальности оценки $\widehat{\theta}$ поверхности уровня являются эллиптическими:
\[
S_w  \approx \widehat{\theta} +  \sqrt{\frac{2w}{n}} \theta^T D \theta.
\]
Докажите при дополнительных ограничениях, $W(\widehat{\theta})$ строго локально вогнута и $\inf\{h(\theta): \Vert \theta \Vert = 1\} > 0$, что 
\[
W(\theta_0) \overset{d}{\longrightarrow} \text{Г}(rp, 1),
\]
где плотность гамма $\text{Г}(\alpha, \lambda)$ распределения задается как $\frac{x^{\alpha-1} e^{-x/\lambda}}{\lambda^\alpha\text{Г}(\alpha)}$.

\end{problem}

\begin{ordre}
См. работу Fan et al. Geometric Understanding of Likelihood Ratio Statistics. Department of Statistics, UCLA 1998.
\end{ordre}


\section{Вероятностные основы математической статистики}
\label{stats}

\begin{problem}
В байесовской теории оценивания оптимальная оценка $\widehat{\theta}$ для функции потерь $\lambda$, выборки $X$ и параметрического распределения $\PR(X|\theta)$ определяется по следующему правилу
\[
\widehat{\theta} = \arg\min_{\theta} \Exp_{\PR(\theta^* | X)} \lambda(\theta, \theta^*).
\] 
Докажите соответствие функций потерь и оптимальных оценок, приведенных в таблице
\begin{center}
\begin{tabular}{ c | c }

функция потерь $\lambda(\theta, \theta^*)$ & оптимальная оценка \\
\hline
$\sum_i [\theta_i \neq \theta^*_i]$ & $\widehat{\theta}_i = \arg\max_{\theta_i}\PR(\theta_i | X)$  \\ 
\hline
$\sum_i (\theta_i - \theta^*_i)^2$ & $\widehat{\theta}_i = \Exp_{\PR(\theta_i | X)} \theta_i$   \\ 
\hline
$[\theta \neq \theta^*]$ & $\widehat{\theta} = \arg\max_{\theta}\PR(\theta | X)$  \\ 

\end{tabular}
\end{center}

\end{problem}

\begin{problem}[Эмпирическая функция распределения]
Пусть ${\rm x}_{k} ,\; k=1,...,n$ -- независимые одинаково распределенные с.в. с непрерывной функцией распределения $F\left(x\right)$. Введём
\[F\left(x;\left\{\vec{{\rm x}}\right\}_{n} \right)=\frac{1}{n} \sum _{k=1}^{n}I\left({\rm x}_{k} <x\right) , I\left({\rm x}_{k} <x\right)=\left\{\begin{array}{l} {1,\; {\rm x}_{k} <x} \\ {0,\; {\rm x}_{k} \ge x} \end{array}\right. \; .\] 

\begin{enumerate}
\item Покажите, что $F\left(x;\left\{\vec{{\rm x}}\right\}_{n} \right)\xrightarrow[{n\to \infty }]{\text{п.н.}} F\left(x\right)$ $\forall x\in {\mathbb R}$.

\item (теорема Гливенко) Покажите, что 
\begin{center}
$\mathop{\max }\limits_{x\in {\mathbb R}} \left|F\left(x;\left\{\vec{{\rm x}}\right\}_{n} \right)-F\left(x\right)\right|\xrightarrow[{n\to \infty }]{\text{п.н.}} 0$.
\end{center}

\begin{remark}(Неравенство Дворецкого--Кифера--Вольфовица) 

Имеет место более тонкий результат:
\[P\left(\mathop{\sqrt{n} \max }\limits_{x\in {\mathbb R}} \left|F\left(x;\left\{\vec{{\rm x}}\right\}_{n} \right)-F\left(x\right)\right|\ge \varepsilon \right)\le 2e^{-2\varepsilon ^{2} } .\] 
Этот замечание, равно как и идеи следующих трех задач, взяты из конспекта лекций Г.К. Голубева "Введение в математическую статистику".

\end{remark}

\item Какие из трех приведенных фактов справедливы без условия непрерывности функции $F\left(x\right)$?
\end{enumerate}

\end{problem}



\begin{problem}[Бутстреп, оценка дисперсии]
$T = T(X_1,\ldots,X_n)$ некоторая статистика i.i.d. выборки из распределения $F$.  Для оценки дисперсии $\Var_F (T)$ можно воспользоваться выборочной функцией распределения $\widehat{F}_n$:
\[
\Var_{\widehat{F}} (T) = \int (T - \Exp T)^2 d\widehat{F}_n(X_1) \ldots d\widehat{F}_n(X_n).
\]
Выражение для $\Var_{\widehat{F}} (T)$ в некоторых случаях удается подсчитать в явном виде (например при $T = \overline{X}$, $\Var_{\widehat{F}} (T) = \frac{1}{n} \sum (X - \overline{X})^2$),  но в общем случае используется метод Монте--Карло для приближенного вычисления интеграла:
\begin{enumerate}
\item Выполнить $B$ раз генерацию выборки $X_1^*,\ldots,X_n^* \sim \widehat{F}_n$;
\item Вычислить значения $T_1^*,\ldots,T_B^*$;
\item Найти оценку $\Var_{\widehat{F}} (T)$ по формуле
\[
V_{\text{boot}}(T) = \frac{1}{B} \sum_{b=1}^B \left( T_b^* -  \overline{T}^* \right)^2.
\]  
\end{enumerate}

Существует также альтернативный метод оценки $\Var_F (T)$ -- \textit{метод складного ножа} -- где выборка $X_1^*,\ldots,X_{n-1}^*$ генерируется посредством поочередного выбрасывания одного из элементов выборки. 
Обозначим за $T_{-i}$ значение статистики, подсчитанное на основе подвыборки  $X_1,\ldots,X_{i-1},X_{i+1}\ldots,X_n$, тогда оценка равна  
\[
V_{\text{jack}}(T) = \frac{n-1}{n} \sum_{i=1}^n \left( T_{-i} -  \overline{T} \right)^2.
\]  
Покажите, что оценка квантилей распределения $F$ посредством $V_{\text{boot}}$ является состоятельной ($V_{\text{boot}}(X_{(k)}) \to \Var X_{(k)} $ при $n \to \infty$), в то время как $V_{\text{jack}}(X_{(k)}) $ не является состоятельной, однако
\[
\frac{
V_{\text{jack}}(X_{(k)})  }{ \Var X_{(k)} 
} \overset{p}{\longrightarrow} 1.
\] 

\end{problem}

\begin{problem}[Бутстреп, доверительные интервалы]
Существует несколько методов оценки доверительных интервалов на основе бутстрепа (к примеру, нормальный интервал, центральный интервал, интервал на основе процентилей). Рассмотрим способ нахождения центрального интервала. 

Обозначим через  $T_1^*,\ldots,T_B^*$ повторную выборку значений статистики $T (X_1^*,\ldots,X_n^*)$ на основе бутстрепа, а через $F_\triangle$ распределение случайной величины $\triangle_n = T(X_1,\ldots,X_n) - \Exp T$. Определим доверительный интервал $(a,b)$ по формуле
\[
a = T(X_1,\ldots,X_n) - F_\triangle^{-1} \left(1 - \frac{\alpha}{2} \right),
\quad
b = T(X_1,\ldots,X_n) - F_\triangle^{-1} \left( \frac{\alpha}{2} \right).
\] 
Убедитесь, что $\PR(a \leq  \Exp T \leq b) = 1 - \alpha$, где $\alpha$ -- заданный уровень значимости. Неизвестное распределение $F_\triangle$ можно оценить как 
\[
\widehat{F}_\triangle(x) = \frac{1}{B} \sum_{b=1}^B [T_b^* - T_n < x].
\]
Предложите способ нахождения $\widehat{F}_\triangle^{-1}(x)$, например с использованием квантилей выборки $T_1^* - T_n, \ldots, T_n^* - T_n$.
\end{problem}

\begin{problem}[Оценка максимума правдоподобия]
Докажите приведенные ниже свойства ОМП оценок 
\[
\widehat{\theta} = \arg\max L(X, \theta) = \arg\max  \sum_{i=1}^{n} \log f(X_i, \theta).
\]
\begin{enumerate}
\item ОМП состоятельная, то есть $\widehat{\theta} \to \theta^*$;
\item ОМП не зависит от параметризации, то есть  при замене параметра $\eta = \eta(\theta)$, $\widehat{\eta} = \eta(\widehat{\theta})$;
\item ОМП асимптотически нормальна 
\[
\frac{\widehat{\theta} - \theta^*}{\sqrt{S}} \overset{d}{\longrightarrow} \N(0, 1),
\quad
S  = \widehat{\Var}\big(\widehat{\theta}\big);
\]
\item ОМП асимптотически оптимальна (имеет наименьшую дисперсию). 
\end{enumerate}

\begin{ordre}
Введем функцию 
\[
\KL_n(\theta) = \frac{1}{n} \sum_{i} \log \frac{f(X_i, \theta)}{f(X_i, \theta^*)}.
\]
Потребуем, чтобы имела место сходимость по вероятности равномерно по $\theta$
\[
\Exp_X \KL_n(\theta) \to - \KL(\theta^*, \theta) =  - \int f(x, \theta^*)  \log \frac{f(x, \theta^*)}{f(x, \theta)} dx.
\]
Докажите тождество
\[
\PR(\KL(\theta^*, \widehat{\theta}) > \KL(\theta^*, \theta^*) + \delta ) \to 0,
\]
из которого в случае локальной вогнутости $\KL(\theta^*, \theta)$ в окрестности нуля будет следовать состоятельность ОМП.

Для доказательства асимптотической нормальности потребуем разложимость $L(\theta)$ по формуле Тейлора в точке $\theta^*$ до второго члена, тогда справедливо соотношение
\[
 0 = \nabla L(\widehat{\theta}) \approx \nabla L(\theta^*) + \nabla^2 L(\theta^*) (\widehat{\theta} - \theta^*), 
\]
из которого следует сходимость 
\[
\sqrt{n} (\widehat{\theta} - \theta^*) \to \frac{ \frac{1}{\sqrt{n}} \nabla L(\theta^*) } { \frac{1}{n} \nabla^2 L(\theta^*) }. 
\]
Введем обозначение $Y_i = \nabla \log f(X_i, \theta)$. Покажите, что 
\[
\Exp (Y_i) = 0, \quad
\Var (Y_i)  = I(\theta) = - \int \big(\nabla^2 \log f(x, \theta) \big) f(x, \theta) dx.
\]
Тогда из ЦПТ следует, что $\sqrt{n} \overline{Y} \to \N(0, I(\theta))$.   

\end{ordre}

\end{problem}

\begin{problem}[Точность ОМП]
\label{mle_square}
Ввиду заложенной в наблюдаемые данные $X$ случайности ОМП оценка $\widehat{\theta}(X) = \arg\max L(X,\theta)$ является случайной величиной, сходящейся в пределе к истинному значению параметра $\theta^*$. Отметим, что для выборки $X$, сгенерированной при помощи модели $P$ отличной от $L(X,\theta)$, под ``истинным'' значением параметра $\theta^*$ подразумевается 
\[
\theta^* = \arg\max \Exp L(X,\theta)
\]
Такой выбор $\theta^*$ соответствует наиболее близкому распределнию из параметрического семейства $L(X,\theta)$ к распределению $P$. 

Для оценки отклонения $\widehat{\theta}$ от $\theta^*$ исследуем \textit{квадратичную}  модель $L(Y,\theta)$ на примере регрессии
\[
Y = X^{T} \theta + \varepsilon,
\quad 
\dim \theta = p,
\varepsilon \in \N(0, S).  
\]  
Предположим, что выборка $Y$ была сгенерирована из распределения с параметрами $\Exp Y = M_Y$, $\Var Y = S_0$, где $S_0$ может не совпадать с $S$. Докажите справедливость следующих выражений 
\[
D_0 (\widehat{\theta} - \theta^*) = \xi \sim \N(0, I),
\]
\[
L(Y,\widehat{\theta}) - L(Y, \theta^*) = \frac{\Vert \xi \Vert^2}{2} \sim \frac{\chi^2_{p}}{2}, 
\]
где 
\[
D_0^2 = -\nabla^2 \Exp L(Y, \theta) \bigg |_{\theta^*} = X S^{-1} X^{T},
\]
\[
\xi = D_0^{-1} \nabla L(\theta^*) = D_0^{-1} X S^{-1} \varepsilon,
\]
\[
\theta^{*} = (X S^{-1} X^{T})^{-1} X S^{-1} M_Y, \quad \widehat{\theta} = (X S^{-1} X^{T})^{-1} X S^{-1} Y.
\]

\begin{remark}
Отметим, что вне зависимости от распределения $Y$ для нахождения истинного значения $\theta^{*}$ достаточно знать $\Exp Y$.  Также стоит заметить, что вектор $\xi$ зависит от неизвестного параметра $\theta^{*}$  и может быть оценен с погрешностью в точке $\widehat{\theta}$ или при помощи метода бутстреп (см. задачу \ref{bootstrep}). 
\end{remark}

\end{problem}

\begin{problem}
Случайный вектор $\xi \in \mathbb{R}^p$ имеет следующую суб-гауссовскую верхнюю границу характеристической функции
\[
\Exp \; \text{exp} \left( \lambda \frac{\gamma^{T} \xi}{\Vert V \gamma \Vert}\right) \leq \text{exp} \left(\frac{c^2 \lambda^2}{2} \right), \quad 
V^2 = \Var \xi.
\]  
Докажите следующие неравенства и найдите константы $c_1$, $c_2$.
\[
\forall \mu < 1: \; \text{exp} \left( \frac{\mu  \Vert \xi \Vert^2 }{2} \right) \leq c_1 \int \text{exp} \left( 
\gamma^{T} \xi -  \frac{  \Vert \gamma \Vert^2 }{2 \mu } \right) d \gamma, 
\]
\[
\PR( \Vert \xi \Vert^2 - \Exp \Vert \xi \Vert^2 > x c_2 ) \leq 2 e^{-x}.
\]
\end{problem}


\begin{problem}[Приближение Лапласа]
Чтобы исследовать отклонение ОМП оценки от истинного значения ($\widehat{\theta} - \theta^*$, a также $L(Y,\widehat{\theta}) - L(Y, \theta^*)$) для произвольной модели $L(Y, \theta)$, необходимо наложить ряд условий на функцию правдоподобия, обеспечивающих приближение  $L(Y, \theta)$ квадратичной формой $\mathbb{L}(Y, \theta)$ в эллиптической окрестности точки~$\theta^*$:
\[
\theta \in \Theta(r) = \{\theta: \Vert D_0 (\theta  - \theta^*) \Vert  \leq r\}.
\]
Наложим ограничения на первую и вторую производные $L(Y, \theta)$ в окрестности $\Theta(r)$:
\begin{enumerate}
\item $D^2(\theta) = - \nabla^2 \Exp_Y L(Y, \theta)$ непрерывна и  
\[
\Vert D_0^{-1} D^2(\theta) D_0^{-1} - I \Vert \leq \delta(r),
\quad
D_0(\theta) = D(\theta^*);
\]
\item  $\nabla \overset{o}{L}(Y, \theta)$ является суб-гауссовской с.в., т.е. существуют такие константы $\omega$ и $c$, что $\forall  \gamma_1, \gamma_2, \lambda: \; \Vert \gamma \Vert \leq 1, \; \Vert \gamma_2 \Vert \leq 1$
\[
\Exp_Y \text{exp} \left( \frac{\lambda}{\omega} \gamma_1^{T} D_0^{-1} \nabla^2 \overset{o}{L}(Y, \theta) D_0^{-1} \gamma_2 \right) \leq \text{exp} \left( \frac{c^2 \lambda^2}{2}\right).
\]
\end{enumerate}
 
Покажите, что при наложенных ограничениях справедливы неасимптотические варианты теорем Фишера и Вилкса, c вероятностью не менее $1 - e^{-x}$ в окрестности $\Theta(r)$
\[
\Vert D_0^{-1} \left(  \nabla  L( \theta) -  \nabla  L(\theta^*) \right) - D_0 (\theta - \theta^*) \Vert \leq \diamondsuit (r,x),
\]
\[
\bigg| \bigg(L(\theta) - L(\theta^*) \bigg) - \frac{(D_0(\theta - \theta^{*}))^2}{2} \bigg| \leq r \diamondsuit (r,x),
\]
где
\[
\diamondsuit (r,x) = r \delta(r) + 6 \omega c  r \sqrt{2p + 2x},
\]
для простой выборки характерны следующие значения констант 
\[ 
\quad \omega \sim \frac{1}{\sqrt{n}}, 
\quad \delta(r) \sim \frac{r}{\sqrt{n}},
\quad r^2 \sim (p + x).
\]
 

\end{problem}

\begin{ordre}
Воспользуйтесь теоремой для оценки отклонения центрированного случайного процесса $u(Y,s)$ от нуля в окрестности $\{s: d(s,s_o) < r\}$. Пусть выполнено условие 
\[
\Exp_Y \text{exp} \left( \lambda \frac{u(Y,s) - u(Y,s_o)}{d(s,s_o)}  \right) \leq \text{exp} \left( \frac{c^2 \lambda^2}{2}\right),
\]
тогда c вероятностью не менее $1 - e^{-x}$ в окрестности $\{s: d(s,s_o) < r\}$
\[
\frac{1}{3 c r} |u(Y, s) - u(Y, s_0)| \leq \sqrt{Q + 2x},
\quad Q = 2p \; \text{при} \; s \in \mathbb{R}^p,
\]
в более общем случае $Q$ представляет собой энтропию пространства значений параметра $s$. 
\end{ordre}

\begin{problem}[Критическая размерность]
В регрессионной модели, представленной ниже, требуется найти только первую компоненту параметра $\theta \in \mathbb{R}^p$. 
\[
X_i =  \begin{pmatrix}
  \theta_1 + \Vert \theta \Vert^2 \\
  \theta_2   \\
  \vdots  \\
  \theta_{p(n)}  
 \end{pmatrix} + \varepsilon_i, 
\quad \varepsilon_i  \in N(0, I_p), 
\quad i = \overline{1,p}
\]
Положим значение истинного параметра $\theta^*$ равным нулю.
Покажите, что  оценка максимума правдоподобия $\widehat{\theta}_1$ сходится с ростом $n$ к нулю со скоростью $n^{-1/2}$ только при $p(n) = o(\sqrt{n})$.
\end{problem}




\begin{problem}
\label{ed_local_u}
Исследуем сепарабельный процесс $U(v)$, $v \in \Upsilon$,  характеризуемый суб-экспоненциальным ограничением характеристической функции
\[
\Exp \text{exp} \left( \lambda \frac{U(v) - U(v')}{d(v,v')} \right) \leq e^{\nu_0^2 \lambda^2 / 2},
\quad |\lambda| \leq g, \; d(v, v') \leq r_0, \; \nu_0 \geq 1. 
\]
На множестве $\Upsilon$ задана $\sigma$-конечная мера $\pi$. Введем обозначения 
\[
B_r(v) = \{u \in \Upsilon: d(v,u) \leq r\}, 
\quad
r_k = 2^{-k} r_0,
\quad
\pi_k(v) = \int_{B_k(u)} \pi (du), 
\]
\[
M_k = \max_{v \in \Upsilon} \frac{\pi(\Upsilon)}{\pi_k(u)},
\]
\[
H_1 = \sum_{k=0}^{\infty} c_k \sqrt{2 \log (2M_k)}, 
\quad H_2 = 2 \sum_{k=0}^{\infty} c_k  \log (2M_k),
\quad
c_0 = \frac{1}{3}, \;
c_k = \frac{2^{1-k}}{3}.
\]

Докажите, что с вероятностью не менее $1 - e^{-x}$ имеет место неравенство
\[
\frac{1}{3 \nu_0 r_0} \sup_{v \in B_{r_0}(v_0)}\{ U(v) - U(v_0)\} \leq H(x) = 
H_1 + \sqrt{2x} + \frac{g^{-2} x + 1}{g} H_2.
\]
\end{problem}

\begin{ordre}
Введем замену $U(\cdot) = \nu_0^{-1} U(\cdot)$, что равносильно случаю 
$\nu_0  = 1$, $g_0 = g\nu_0$, а также определим оператор $S_k$ по правилу
\[
S_k f(v_0) = \frac{1}{\pi_k(v_0)} \int_{B_k(v_0)} f(v) \pi (dv),
\quad k \geq 0,
\]
\[
S_{-1} U(v) = U(v_0) = S_k S_{k-1} \ldots S_{-1} U(v). 
\]
\[
|U(v) - U(v_0)| = \lim_{k\to\infty} |S_k U(v) -  S_k S_{k-1} \ldots S_{-1} U(v)| \leq 
\]
\[
\leq \lim_{k\to\infty} \sum_{i = 0}^k |S_k \ldots S_{i} (I - S_{i-1}) U(v)|  \leq \sum_{i = 0}^{\infty} \xi_i^{*},
\]
где
\[
\xi_i^{*} = \sup_{v \in B_{r_0}(v_0)} \xi_i(v),
\quad
 \xi_0(v) = | S_{0} U(v) - U(v_0) |,
 \quad
 \xi_i(v) = | S_{i} (I - S_{i-1}) U(v) |. 
\]
Докажите свойство
\[
\Exp e^{\lambda \xi_i^{*} / r_{i-1}} \leq 2M_i e^{\lambda^2/2},
\]
воспользуйтесь результатом задачи \ref{sub_exp} из раздела \ref{gen_func}.

\end{ordre}

\begin{problem}
В контексте задачи \ref{ed_local_u} рассмотрите частный случай 
\[
\Upsilon  = B(r_0, v_0) = \{ v \in \mathbb{R}^p : \Vert D (v - v_0) \Vert \leq r_0 \},
\]
где 
\[
d(v,v_0) \leq \Vert D (v - v_0) \Vert.
\]
Докажите верхнюю оценку \[H_2 \leq  4 p\] при $p \geq 2$, а также соотношение $H_1 \leq \sqrt{H_2}$. 
 
\end{problem}

\begin{ordre}
Не теряя общности, можно выполнить расчеты для случая
\[
v_0 = 0,
\quad D = I_p,
\quad r_0 = 1.
\]
Чтобы найти верхнюю границу $M_k$, оценим снизу величину \[\pi (B(1, 0) \cap B(r,v))\], значение которой достигает минимума при $\Vert v \Vert = 1$. 
Рассмотрите шар $B(\rho, u)$, где $\rho = r - r^2/2$, $u = (1 - r^2/2) v$.
Покажите, что 
\[
B(r, v) \supseteq B(\rho, u), 
\quad
\pi (B(1, 0) \cap B(\rho,u)) \geq \pi (B(\rho,u)) /2,
\]
откуда будет следовать, что $2M_k \leq 2^{2+kp} (1 - 2^{-k-1})^{-p}$ при $r_k = 2^{-k}$.
\end{ordre}

\begin{remark}
При наличии регуляризатора $\Vert Gv \Vert^2$, при котором появляется дополнительное ограничение на значения параметра $v$:
\[
B(r, v_0) = \{ v \in \mathbb{R}^p : \Vert D (v - v_0) \Vert^2 + \Vert Gv \Vert^2 \leq r_0^2  \} 
\] 
в результате \textit{эффективная} размерность параметра $v$ уменьшается согласно формулам
\[
H_2 \leq 1+\frac{8}{3} \text{tr} [B^{-1}],
\quad
H_1 \leq 1 + 2 \text{tr}^{1/2} [B^{-2} \log^2 (B^2)],
\]  
где $B^2 =  I_p + D^{-1} G^2 D^{-1}$.
\end{remark}

\begin{problem}
\label{ed_local_grad_u}

Докажите, что из ограничения на градиент случайного процесса $U(v)$ вида
\[
\log \Exp \text{exp} \left\{\lambda \frac{\gamma^T \nabla U(v)}{\Vert D(v) \gamma \Vert} \right\} \leq 
\frac{\nu_0^2 \lambda^2}{2},
\]
следует ограничение 
\[
\log \Exp \text{exp} \left\{\lambda \frac{ U(v) - U(v_0)}{\Vert D (v - v_0) \Vert} \right\} \leq 
\frac{\nu_0^2 \lambda^2}{2},
\]
где
\[
\nu_0 \geq 1, \; |\lambda| \leq g, \; \gamma \in \mathbb{R}^p, D \succcurlyeq D(v).  
\]
\end{problem}

\begin{ordre}
Представьте приращение процесса $U$ в виде
\[
U(v, v_0) = \delta \gamma^T \int_{0}^{1} \nabla U (v_0 - t \delta \gamma) dt, 
\quad \delta = \Vert v - v_0 \Vert, \; \gamma  = (v - v_0) / \delta.
\]
\end{ordre}

\begin{problem}[Локализация ОМП]
Чтобы определить границы локальной области параметра $v$ -- $B_{r}(v_0)$, вне которой с малой вероятностью выполнено неравенство
\[
\mathcal{U}(v) - \mathcal{U}(v_0) \geq 0, 
\]
разделим разности сепарабельного процесса $\mathcal{U}(v) - \mathcal{U}(v_0)$ на стохастическую $U(v, v_0)$ и детерминированную $ (- f(v, v_0, \rho))$ части. Предположим, что для $U(v, v_0)$ выполнены условия из задачи \ref{ed_local_u}, а также 
\[
f(v, v_0, \rho) \geq 3 \nu_0 r H \left(x + \log \left( \frac{\rho^{-1} d(v, v_0)}{r_0} \right) \right),
\quad
r_0 \leq  d(v, v_0) \leq r^*.
\]
Докажите, что с вероятностью менее $\frac{\rho}{1-\rho} e^{-x}$ имеет место неравенство
\[
\sup_{v \in B_{r^*}(v_0) \setminus B_{r_0}(v_0)} \{ U(v, v_0) - f(v, v_0, \rho) \} \geq 0,
\quad
0 < \rho \leq 1.
\]
\end{problem}


\begin{ordre}
Разбейте область $B_{r^*}(v_0) \setminus B_{r_0}(v_0)$ на слои с радиусами $r_k = r_0 \rho^{-k}$, примените для каждого слоя результат задачи \ref{ed_local_u}.
\end{ordre}
\begin{remark}
Найдем область, в которой $L(\theta) - L(\theta^*) < 0$ с большой вероятностью. Данная область является дополнением к области локализации ОМП.  Пусть для 
$\zeta(\theta) = L(\theta, Y) - \Exp_Y L(\theta, Y)$ выполнено условие
\[
\Exp_Y \text{exp} \left( \frac{\lambda}{\omega} \gamma_1^{T} D_0^{-1} \nabla^2 \zeta(\theta) D_0^{-1} \gamma_2 \right) \leq \text{exp} \left( \frac{\nu_0^2 \lambda^2}{2}\right), 
\quad \Vert \gamma_1 \Vert = \Vert \gamma_2 \Vert = 1.
\]
Тогда из результатов задач \ref{ed_local_u} и \ref{ed_local_grad_u} следует, что в локальной области $\Vert D_0(\theta - \theta^*)\Vert \leq r$ выполнено
\[
 | \zeta(\theta, \theta^*) - (\theta - \theta^*)^T \nabla \zeta(\theta^*)  | \leq  6 \nu_0 H(x) w r.
\]
Из результата данной задачи получаем неравенство
\[
|L(\theta, \theta^*) - \Exp L(\theta, \theta^*) - (\theta - \theta^*)^T \nabla L(\theta^*) |\leq
\]
\[
\leq 6 \nu_0  w r H \left(x + \log \left( \frac{2 r}{r_0} \right) \right) = 
\rho(r,x) r.
\]
Так как
\[
|(\theta - \theta^*)^T \nabla L(\theta^*) | \leq r \Vert \xi \Vert,
\]
то из неравенств
\[
-2 \Exp L(\theta, \theta^*) \geq \Vert D_0(\theta, \theta^*) \Vert^2 b,
\quad
r^*  b(r^*) \geq 2 (\Vert \xi \Vert + \rho(r^*,x))
\]
получаем возможность найти радиус локальной области ОМП $r^*$.
\end{remark}

\begin{problem}
Пусть $y(v): \mathbb{R}^p \to \mathbb{R}^q$ -- гладкий случайный процесс, причем $\Exp y(v) = 0$, $y(v_0) = 0$. Без ограничения общности можно положить $v_0 = 0$. Предположим, что для любых $\Vert \gamma \Vert = \Vert \alpha \Vert = 1$
\[
\log \Exp \text{exp}\left( \lambda \gamma^T \nabla y (v) \alpha \right) \leq \frac{\nu_0^2 \lambda^2}{2},
\quad |\lambda| \leq g.
\] 
Докажите неравенство
\[
\PR \left(\sup_{\Vert v - v_0 \Vert \leq r} \Vert y(v) \Vert  > 6 \nu_0 r H(x) \right) \leq e^{-x},
\]
где
\[
H(x) = H_1 + \sqrt{2x} + \frac{g^{-2} x + 1}{g} H_2, 
\quad H_1 = \sqrt{4(p + q)}, \; H_2 = 4(p + q). 
\]
\end{problem}

\begin{ordre}
Используя утверждение задачи \ref{ed_local_grad_u}, получите неравенство
\[
\log \Exp \text{exp}\left( \frac{\lambda}{r} \gamma^T y (v)  \right) \leq \frac{\nu_0^2 \lambda^2 \Vert v - v_0 \Vert^2}{2r^2}.
\]
Представьте норму вектора в виде
\[
\Vert y(v) \Vert = \sup_{\Vert u \Vert \leq r } \frac{1}{r} u^T y(v). 
\]
Получите неравенство
\[
\log \Exp \text{exp}\left( \frac{\lambda}{2r} (\gamma,\alpha)^T \nabla [u^T y (v)] \right) \leq \frac{\nu_0^2 \lambda^2}{2}.
\] 
\end{ordre}

\begin{problem}[Теорема Уилкса]
Функция правдоподобия $L(\theta, Y)$ достигает максимального  значения  при параметре $\widehat{\theta}$, $\theta^*$ -- истинные значения параметра. Введем также набор параметров $\widehat{\theta}_m$, при котором достигается минимум функции $L$ при условии, что $\forall i \in [1,m]: \; \widehat{\theta}_m(i) = \theta^*(i)$. Определим статистику 
\[
W(\theta, Y) = L(\widehat{\theta}, Y) -  L(\widehat{\theta}_m, Y) 
\]
Предполагая возможность квадратичной аппроксимации Лапласа (см. задачу \ref{laplas_approx}), докажите слабую сходимость $2W(\theta, Y)$ к распределению хи-квадрат со степенями свободы $m$ при неограниченном увеличении выборки. 
 
\end{problem}

\begin{problem}
Для следующих примеров выборок, проверьте асимптотическую ненормальность ОМП оценок. Исходя из результата задачи \ref{wilks_geom} из раздела \ref{geom_prob}, докажите, что, тем не менее, имеет место сходимость статистики $W(\theta, Y)$   к распределению хи-квадрат.
\begin{enumerate}
\item $Y_1,\ldots,Y_n \sim N(\theta^3, I_p)$, $\widehat{\theta} = \overline{Y}^{1/3}$;
\item $Y_i \sim \theta + \varepsilon_i$, $\PR(\varepsilon_{ik} > x) = e^{-x} $, $\widehat{\theta}_k = \min \{Y_{1k},\ldots,Y_{nk} \}$;
\item $Y_1,\ldots,Y_n$ выборка с заданной совместной плотностью распределения 
\[
p(Y_1,\ldots,Y_n, y, \theta) = c e^{\left(-n \Vert y - \theta \Vert^{\gamma} - \sum_{i=1}^{n} (Y_i - y)^2 \right)}.
\]
\end{enumerate}
 
\end{problem}

\begin{ordre}
В приведенных примерах поверхности уровня определяются как
\begin{enumerate}
\item $S_w = \{\theta: n \Vert \overline{Y} - \theta^3 \Vert^2 = 2w\} \approx  \overline{Y}^{1/3} - (3\overline{Y}^{2/3})^{-1} \sqrt{2w/n} S^{1/3}$, $S$~--~ единичная сфера;
\item $S_w = \{\theta: n \sum_i (\widehat{\theta}_i - \theta_i ) = w, \; \widehat{\theta} > \theta\} =  \widehat{\theta} + (w/n) S$, $S$~--~единичный симплекс; 
\item $S_w =  \overline{Y}  + (w/n)^{1/\gamma} S$, $S$ -- единичная сфера.
\end{enumerate}
\end{ordre}








\begin{problem}[Неравенство Ван Трисса] 
\label{van_tris}
Пусть ${\rm x}_{k} ,\; k=1,...,n$ -- независимые одинаково распределенные с.в. с плотностью распределения $p_{\vec{{\rm x}}} \left(\vec{x},\vec{\theta }\right)$, зависящей от случайного вектора $\vec{\theta }$ (носитель распределения $\vec{{\rm x}}$ предполагается не зависящим от $\vec{\theta }$, т.е. $p_{\vec{{\rm x}}} \left(\vec{x},\vec{\theta }\right)$ - регулярное семейство), который имеет плотность распределения $\pi (\vec{\theta })$: 
\[\mathop{\lim }\limits_{\left\| \vec{\theta }\right\| \to \infty } \left(\left\| \vec{\theta }\right\| \pi \left(\vec{\theta }\right)\right)=0.\] 

Покажите (для скалярного случая когда $\vec{\theta}  \in \mathbb{R}$), 
что для любой измеримой вектор-функции 
$\vec{\tilde{\theta }}\left(\vec{{\rm x}}\right)$: 
\[D_{\vec{{\rm x}},\vec{\theta }} \left[\vec{\tilde{\theta }}\left(\vec{{\rm x}}\right)\right]<\infty \] 
имеет место неравенство:

\[\Exp_{\vec{{\rm x}},\vec{\theta }} \left[\left(\vec{\tilde{\theta }}\left(\vec{{\rm x}}\right)-\vec{\theta }\right)\left(\vec{\tilde{\theta }}\left(\vec{{\rm x}}\right)-\vec{\theta }\right)^{T} \right]\succ \left[I_{p,n} +I_{\pi } \right]^{-1},\] т.е.

\begin{center}
$\Exp_{\vec{{\rm x}},\vec{\theta }} \left[\left(\vec{\tilde{\theta }}\left(\vec{{\rm x}}\right)-\vec{\theta }\right)\left(\vec{\tilde{\theta }}\left(\vec{{\rm x}}\right)-\vec{\theta }\right)^{T} \right]-\left[I_{p} +I_{\pi } \right]^{-1} $ 
\end{center}
-- неотрицательно определенная матрица.

Здесь информационные матрицы (Фишера) рассчитываются по формулам: 
\[I_{p,n} \mathop{=}\limits^{def} E_{\vec{{\rm x}},\vec{\theta }} \left[\frac{\partial \ln p_{\vec{{\rm x}}} \left(\vec{{\rm x}},\vec{\theta }\right)}{\partial \vec{\theta }} \left(\frac{\partial \ln p_{\vec{{\rm x}}} \left(\vec{{\rm x}},\vec{\theta }\right)}{\partial \vec{\theta }} \right)^{T} \right]=nI_{p,1} <\infty , 
\]
\[I_{\pi } \mathop{=}\limits^{def} E_{\vec{\theta }} \left[\frac{\partial \ln \pi \left(\vec{\theta }\right)}{\partial \vec{\theta }} \left(\frac{\partial \ln \pi \left(\vec{\theta }\right)}{\partial \vec{\theta }} \right)^{T} \right]<\infty .\] 

\end{problem}

\begin{remark}
В случае неинформативного $\pi (\vec{\theta})$ (несобственное равномерное распределение на всем пространстве, которое получается предельным переходом из равномерного на шаре, при стремлении радиуса шара к бесконечности) и $E_{\vec{{\rm x}}} \left[\vec{\tilde{\theta }}\left(\vec{{\rm x}}\right)\right]\equiv \vec{\theta }$, неравенство Ван Трисса переходит в намного более известное по классическим стохастическим курсам \textit{неравенство Рао--Крамера}, которое мы приводим (для наглядности) в скалярном случае: \[D_{\vec{{\rm x}}} \left[\tilde{\theta }\left(\vec{{\rm x}}\right)\right]\ge n^{-1} E_{\vec{{\rm x}}} \left[\left({\partial \ln p_{{\rm x}} \left(x,\theta \right)\mathord{\left/ {\vphantom {\partial \ln p_{{\rm x}} \left(x,\theta \right) \partial \theta }} \right. \kern-\nulldelimiterspace} \partial \theta } \right)^{2} \right]^{-1} .\]

В классе оценок, для которых смещение $b(\vec{\theta}) = E[\tilde{\theta} - \vec{\theta}] \not\equiv 0$ ненеравенство Рао-Крамера имеет вид \[D_{\vec{{\rm x}}} \left[\tilde{\theta }\left(\vec{{\rm x}}\right)\right]\ge (1 + b'(\vec{\theta}))^2 n^{-1} E_{\vec{{\rm x}}} \left[\left({\partial \ln p_{{\rm x}} \left(x,\theta \right)\mathord{\left/ {\vphantom {\partial \ln p_{{\rm x}} \left(x,\theta \right) \partial \theta }} \right. \kern-\nulldelimiterspace} \partial \theta } \right)^{2} \right]^{-1} .\] 
\end{remark}

\begin{remark}
Hесобственная плотность распределения отличается от классического определения плотности распределение тем, что интеграл по области определения расходится. Примером несобственной плотности распределения может служить константная функция, заданная на действительной оси (обобщение равномерной плотности с бесконечным интервалом определения). 
\end{remark}


\begin{ordre}
Для скалярного случая $\vec{\theta }=\theta $ воспользуйтесь неравенством Коши--(Шварца)--Буняковского: $\left\langle a,b\right\rangle ^{2} \le \left\langle a,a\right\rangle \left\langle b,b\right\rangle $, рассмотрев случай, когда
 $\left\langle a,b\right\rangle \mathop{=}\limits^{def} E\left(ab\right)$, где $a=\tilde{\theta }\left(\vec{{\rm x}}\right)-\theta $, $b=\frac{\partial \left(\ln p_{\vec{{\rm x}}} \left(\vec{{\rm x}},\theta \right)\pi \left(\theta \right)\right)}{\partial \theta } $.

\end{ordre}

\begin{problem}
$F(x)$ -- произвольная функция распределения с нулевым средним и конечным стандартным отклонением, $Y_i \in F_\theta$, где $F_\theta(x) = F(x -\theta)$, $\theta \in \mathbb{R}$. Приведите пример $F(x)$ для которого оценка $(\min Y_i + \max Y_i) / 2$ имеет меньшую дисперсию (равную $O(1/n^2)$) нежели $\overline{Y}$. Не противоречит ли приведенный пример неравенству Рао-Крамера? 
\end{problem}

\begin{remark}
Оценка максимального правдоподобия параметра сдвига $\theta$ зачастую принимает следующий вид 
\[
\widehat{\theta} = \sum \limits_{i = 1}^n a_i Y_{(i)},  
\]
где $\sum a_i = 1$, $Y_{(i)}$ -- $i$-я порядковая статистика.
В этом случае если $\widehat{\theta}$ отлично от $\overline{Y}$, то не более двух коэффициентов среди $a_i$ отличны от нуля: это либо $a_1$ и  $a_n$ и в этом случае элементы выборки распределены равномерно, либо $a_i$ и  $a_{i+1}$, либо только один коэффициент не нулевой.
\end{remark}


\begin{problem}[Байесовская оценка с квадратичным штрафом] 
В условия предыдущей задачи введем штраф: $I(\vec{\theta'}\left(\, \cdot \, \right),\vec{\theta })$. Оценка $\vec{\tilde{\theta }}\left(\vec{{\rm x}}\right)$ вектора неизвестных параметров $\vec{\theta }$ называется \textit{байесовской}, если для любого $\vec{x}$:
\[\vec{\tilde{\theta }}\left(\vec{x}\right)=\arg \mathop{\min }\limits_{\vec{\theta }'} \int I\left(\vec{\theta }',\vec{\theta }\right) p_{\vec{{\rm x}}} \left(\vec{x},\vec{\theta }\right)\pi \left(\vec{\theta }\right)d\vec{\theta }.\] 

\begin{enumerate}
\item(Уравнение Винера--Хопфа) Покажите, что если 
$I\left(\vec{\theta'}\left(\, \cdot \, \right),\vec{\theta }\right)=\left\| \vec{\theta'}\left(\vec{{\rm x}}\right)-\vec{\theta }\right\| _{2}^{2} $ и $\left(\vec{{\rm x}}^T{\rm ,}\; \vec{\theta }^T\right)^T$ -- нормальный случайный вектор, то $\vec{\tilde{\theta }}\left(\vec{x}\right)=A\vec{x}+\vec{b}$ (линейная регрессия).

\item Рассмотрим следующую схему эксперимента ${\rm x}_{i} =\theta +\varepsilon _{i} $, где $\varepsilon _{i} \in N\left(0,\sigma ^{2} \right)$, а $\theta \in N\left(0,\delta ^{2} \right)$, причем с.в. $\varepsilon _{i} $, $i=1,...,n$ и $\theta $ независимы в совокупности. Постройте с помощью п. a) байесовскую оценку неизвестного параметра $\theta $ (функция штрафа квадратичная). С помощью неравенства Ван Трисса исследуйте качество этой оценки. Сравните эту байесовскую оценку с байесовской оценкой в случае неинформативного априорного распределения ($\theta \in N\left(0,\delta ^{2} \right)$, $\delta ^{2} \to \infty $) -- это будет оценка метода наименьших квадратов, хорошо известная из курса  лабораторных работ по физике.

\item (Байесовская регуляризация) Пусть нужно решить систему уравнений $\vec{X}=A\vec{\theta }$ относительно $\vec{\theta }$ ($m$ -- размер вектора $\vec{\theta }$ может быть много меньше $n$ -- размера вектора $\vec{X}$). Однако из-за ошибок округления, всевозможных шумов и неточностей измерений в действительности приходится решать систему $\vec{X}=A\vec{\theta }+\vec{\varepsilon }$, где $\vec{\varepsilon }\in N\left(\vec{0},\sigma ^{2} I_{n} \right)$. Считая, что $\vec{\theta }\in N\left(0,\delta ^{2} I_{m} \right)$, постройте байесовскую оценку (с квадратичным штрафом) неизвестного вектора $\vec{\theta }$. Рассмотрите два случая: $\delta ^{2} <\infty $ -- информация о локализации искомого вектора есть; $\delta ^{2} =\infty $ -- информации нет.

\end{enumerate}

\end{problem}

\begin{remark}

В лабораторных работах по физике: $y_{i} =kx_{i} +b+\varepsilon _{i} $, т.е.
\[\vec{X}=\vec{y}, \vec{\theta }=\left(k,b\right)^{T} , A=\left(\begin{array}{cc} {x_{1} } & {...\quad x_{n} } \\ {1} & {...\quad 1} \end{array}\right)^{T} , \delta ^{2} =\infty .\] 
Отметим,  что задача поиска байесовской оценки может быть проинтерпретирована, как поиск минимума регуляризованного (по Тихонову) функционала метода наименьших квадратов \[\bar{\theta }\left(\vec{x}\right)=\arg \mathop{\min }\limits_{\vec{\theta }} \left\{\left\| \vec{x}-A\vec{\theta }\right\| _{2}^{2} +\left({\sigma \mathord{\left/ {\vphantom {\sigma  \delta }} \right. \kern-\nulldelimiterspace} \delta } \right)^{2} \left\| \vec{\theta }\right\| _{2}^{2} \right\}\] Регуляризация крайне важна, например, в случае, когда положительно определенная матрица $A^{T} A$ плохо обусловлена (в частности, это означает, что матрица $\left(A^{T} A\right)^{-1} $ может содержать очень большие элементы). Отметим, также, что эта оценка может быть получена при помощи метода наибольшего правдоподобия Фишера: 

\[\vec{\theta }_{} \left(\vec{x}\right)=\arg \mathop{\max }\limits_{\vec{\theta }} \ln \left( p_{\vec{{\rm x}}} \left(\vec{x} \mid \vec{\theta }\right) \pi(\vec{\theta})\right) =\arg \mathop{\max }\limits_{\vec{\theta }} p_{\vec{{\rm x}}} \left(\vec{x} \mid \vec{\theta }\right) \pi(\vec{\theta}).\] 

\end{remark}


\begin{problem}[Робастное оценивание или $l_{1}$--оптимизация] 
Рассмотрим следующую схему эксперимента ${\rm x}_{i} =\theta +\varepsilon _{i} $, $i=1,...,n$, где $\theta $ -- неизвестный параметр, $\varepsilon _{i} $ -- независимые одинаково распределенные с.в. с нулевым математическим ожиданием (если математичексое ожидние не существует, то считаем, что с.в. имеют симметричное распределение относительно 0).

\begin{enumerate}
\item (Робастное оценивание) 
Положим 

\noindent $\breve{\theta }\left(\vec{x}\right)=\arg \mathop{\min }\limits_{\theta } \left\| \vec{x}-\left(\theta ,...,\theta \right)^{T} \right\| _{1} =\arg \mathop{\min }\limits_{\theta } \sum _{k=1}^{n}\left|x_{k} -\theta \right| \approx x_{\left({n\mathord{\left/ {\vphantom {n 2}} \right. \kern-\nulldelimiterspace} 2} \right)} $ -- медиана.
\begin{comment}
САША, ВСПОМНИТЕ ЧТО ГВОРИЛ СПОКОЙНЫЙ
\end{comment}
\noindent Покажите, что
\[\sqrt{n} \cdot \left(\breve{\theta }\left(\vec{{\rm x}}\right)-\theta \right)\xrightarrow[{n\to \infty }]{d} N\left(0,\left(4\left(p_{\varepsilon } \left(0\right)\right)^{2} \right)^{-1} \right).\] 

\begin{remark} 
Отметим, что в этом  пункте с.в. $\varepsilon _{i} $ могут иметь, например, распределение Коши, для которого не существует математического ожидания. Тем не менее, среднеквадратичное отклонение $\breve{\theta }\left(\vec{{\rm x}}\right)$ от математического ожидания $\theta $ имеет порядок $\sim n^{-{1\mathord{\left/ {\vphantom {1 2}} \right. \kern-\nulldelimiterspace} 2} } $. Используя, например, неравенство Чебышёва отсюда можно заключить, что истинное значение $\theta $ лежит в интервале порядка $\sim n^{-{1\mathord{\left/ {\vphantom {1 2}} \right. \kern-\nulldelimiterspace} 2} } $ с центром в $\breve{\theta }\left(\vec{{\rm x}}\right)$. Таким образом, качество оценки неизвестного параметра вполне естественно характеризовать его дисперсией. Чем дисперсия меньше, тем оценка лучше. Это обстоятельство отчасти объясняет выбор квадратичной функции штрафа (например, в байесовском оценивании). Из неравенства Рао---Крамера следует, что для регулярных случаев такой  порядок убывания дисперсии $\sim n^{-1} $ с ростом объема выборки $n$ является типичным, и ``борьба идет'', как правило, за константу при  $n^{-1} $.
\end{remark}

\item Будем считать, что с.в. $\varepsilon _{i} \in N\left(0,\sigma ^{2} \right)$. Покажите, что оценка метода наименьших квадратов:
\[\bar{\theta }\left(\vec{x}\right)=\arg \mathop{\min }\limits_{\theta } \left\| \vec{x}-\left(\theta ,...,\theta \right)^{T} \right\| _{2}^{2} =\arg \mathop{\min }\limits_{\theta } \sum _{k=1}^{n}\left(x_{k} -\theta \right)^{2} = \frac{1}{n} \sum _{k=1}^{n}x_{k}  \] 
доставляет равенство в неравенстве Рао--Крамера.

\item (Нерегулярное семейство) Нерегулярность семейства означает, что носитель распределения вектора $\vec{x}$ зависит от параметра, это дает возможность существования в нерегулряной модели лучших, но не робастных оценок. Пусть $\varepsilon _{i} \in R\left[-\sqrt{3} \sigma ,\sqrt{3} \sigma \right]$. Положим, $\tilde{\theta }\left(\vec{x}\right)=\frac{1}{2} \left(x_{\left(1\right)} +x_{\left(n\right)} \right)$, где $x_{\left(1\right)} =\mathop{\min }\limits_{k=1,...,n} x_{\left(k\right)} $, $x_{\left(n\right)} =\mathop{\max }\limits_{k=1,...,n} x_{\left(k\right)} $. Покажите, что 
\[n\left(\tilde{\theta }\left(\vec{{\rm x}}\right)-\theta \right)\xrightarrow[{n\to \infty }]{P} \sigma \left(e_{1} -e_{2} \right),\] 
где $e_{1} $, $e_{2} $ -- независимые с.в., имеющие распределение Лапласа (показательное) с параметром равным 1. Покажите, что если мы ошиблись в предположении, что $\varepsilon _{i} \in R\left[-\sqrt{3} \sigma ,\sqrt{3} \sigma \right]$, и на самом деле $\left|\varepsilon _{i} \right|$ имеет, скажем, распределение Лапласа (показательное) с параметром равным 1, то
\[\tilde{\theta }\left(\vec{{\rm x}}\right)-\theta \xrightarrow[{n\to \infty }]{d} \ln e_{1} -\ln e_{2} +{\rm O} \left(\frac{1}{\sqrt{n} } \right).\] 

\begin{remark} 
Все приведенные выше оценки могут быть получены методом наибольшего правдоподобия, когда:  
\begin{enumerate}
\item $\left|\varepsilon _{i} \right|$ имеет распределения Лапласа; 
\item $\varepsilon _{i} \in N\left(0,\sigma ^{2} \right)$; 
\item $\varepsilon _{i} \in R\left[-\sqrt{3} \sigma ,\sqrt{3} \sigma \right]$.
\end{enumerate}
Метод наибольшего правдоподобия также можно понимать, как способ поиска такого значения параметра(-ов) распределения, при котором это распределение наиболее близко (в смысле расстояния Кульбака--Лейблера) к эмпирическому распределению, построенному по имеющейся выборке (данным) $\vec{{\rm x}}$ (реализации $x_{i} $ с.в. ${\rm x}_{i} $ считаются известными экспериментатору).

\end{remark} 

\end{enumerate}
\end{problem}

\begin{problem}[Суперэффективные оценки]
Оценка $T_n$ некоторого параметра $\theta$ называется \textit{суперэффективной} если 
\[
\lim \limits_{n \to \infty}(  \sqrt{n} (T_n - \theta) )^2 \leq I^{-1}(\theta)
\] 
и хотя бы в одной точке $\theta$  имеет место строгое неравенство. 

Для простой выборки $X_1 \ldots X_n \in N(\theta, 1)$ определим две  оценки: $\widehat{\theta}_n = \overline{X}$ (оценка максимума правдоподобия) и 
\[
T_n = \begin{cases}
\widehat{\theta}_n, \quad |\widehat{\theta}_n| > n^{-1/4} \\
\alpha \widehat{\theta}_n, \quad |\widehat{\theta}_n| \leq n^{-1/4}. \\
\end{cases}
\]  
где $|\alpha| < 1$. Является ли $T_n$ суперэффективной с точкой суперэффективности $\theta = 0$?
Для исследования качества оценки  $T_n$ в окрестности точки  $\theta = 0$  рассмотрим последовательность $\{\theta_n = c/ \sqrt{n} \}$, сходящуюся к 0. Покажите, что 
\[
\lim \limits_{n \to \infty}(  \sqrt{n} (T_n - \theta_n) )^2 > 1,
\] 
в то время как 
\[
\lim \limits_{n \to \infty}(  \sqrt{n} (\widehat{\theta}_n - \theta_n) )^2 \leq 1.
\] 

\end{problem}

\begin{remark}
Для проверки эффективности оценки в некотором множестве $W$ можно воспользоваться следующим правилом: для набора экспериментов $\langle \Omega_\varepsilon, \mathcal{F}_\varepsilon, \PR_\varepsilon \rangle$ оценка $\widehat{\theta}_\varepsilon$ будет \textit{асимптотически эффективной} по отношению к функциям потерь $\lambda_\varepsilon$, если равномерно по $u \in W$ существует предел 
\[
\lim\limits_{\varepsilon \to 0} \Exp \lambda_\varepsilon (\widehat{\theta}_\varepsilon - u) = L(u),
\]  
а также для любой оценки $T_\varepsilon$ и непустого открытого множества $U \subseteq W$ выполнено
\[
\mathop{\underline{\lim}} \limits_{\varepsilon \to 0} \sup \limits_{u \in U} \Exp \lambda_\varepsilon (T_\varepsilon - u) \geq \sup \limits_{u \in U} L(u).
\]
В частном случае, когда эксперимент \textit{регулярный} (достаточным условием регулярности есть существование $I(\theta)$) для любой $T_\varepsilon$ и не слишком быстро возрастающей функции  $\lambda$ справедливо неравенство
\[
\lim\limits_{\delta \to 0} \mathop{\underline{\lim}} \limits_{\varepsilon \to 0}  
\sup \limits_{|u - \theta| < \delta} \Exp \lambda (c_{\theta, \varepsilon} |T_\varepsilon - u|) \geq \frac{1}{\sqrt{2 \pi}} \int  \lambda(|y|)  e^{-y^2/2} dy.
\]
С другой стороны, оценки максимального правдоподобия и байесовские оценки достигают равенства в последнем выражении.
\end{remark}

\begin{problem}
Пусть $\widehat{t}_{\varepsilon, q}$ -- байесовская оценка параметра $\theta \in \mathbb{R}^k$ относительно априорной плотности $q$ и функции потерь $\lambda_\varepsilon$ для набора экспериментов  $\langle \Omega_\varepsilon, \mathcal{F}_\varepsilon, \PR_\varepsilon \rangle$. Допустим, что $\forall u \in W,  \forall q: q(u) > 0$, 
выполнено соотношение
\[
\lim \limits_{\varepsilon \to 0} \sup \limits_{u \in U} \Exp \lambda_\varepsilon (\widehat{t}_{\varepsilon, q} - u) = L(u).
\] 
Докажите $\forall U \subseteq W,  \forall T_\varepsilon$ справедливость неравенства
\[
\mathop{\underline{\lim}} \limits_{\varepsilon \to 0}  
\sup \limits_{u \in U} \Exp \lambda_\varepsilon (T_\varepsilon - u) \geq \int \limits_{U} L(u) q(u)du.
\]
\end{problem}

\begin{problem}[Оракульное неравенство]
\begin{enumerate}
\item Пусть $\mbox{x}_k ,\;k=1,...,n$ -- 
независимые одинаково распределенные с.в. $\mbox{x}_k \in N\left( {0,\sigma^2} \right)$. Покажите, что $E\left[ {\mathop {\max }\limits_{k=1,...,n}\mbox{x}_k } \right]\le \sqrt {2\sigma ^2\ln n} $.

\item Покажите, что
\[
P\left[ {\mathop {\max }\limits_{k=1,...,n} \left| {\mbox{x}_k } \right|\ge 
\sigma (\sqrt {2\ln n} +u) } \right]\le \frac{1}{\sqrt {\pi \ln n} }e^{-{u^2} 
\mathord{\left/ {\vphantom {{u^2} 2}} \right. \kern-\nulldelimiterspace} 2}\quad\]

\item (Sparsity, экспрессия генов) $y_k =\theta _k +\varepsilon _k $, 
$\varepsilon _k \in N\left( {0,\sigma ^2} \right)$, $k=1,...,n$ -- 
независимые с.в. Результаты измерений $y_k $ -- известны, параметры $\theta 
_k $ -- неизвестны. Однако известно, что большинство компонент (правда, не 
известно какие именно) вектора $\vec {\theta }$ -- нулевые. Предложите выбор 
такого порога $\tau >0$, чтобы оценка неизвестных параметров
 $\tilde {\theta }_k =y_k I\left( {\left| {y_k } \right|>\tau } \right),$ где 
$I(\cdot)$ - индикаторная функция была бы ``наиболее разумной''.

\end{enumerate}
 \end{problem}
 
\begin{ordre}
a) 1. $\mathop {\max }\limits_{k=1,...,n}\mbox{x}_k =\lambda ^{-1}\ln \left[ {\mathop {\max }\limits_{k=1,...,n} e^{\mbox{x}_k }} \right]\le \lambda ^{-1}\ln \left[ {\sum\limits_{k=1}^n {e^{\mbox{x}_k }} } \right]$. 2.(Неравенство Йенсена) Для вогнутой 
функции $Ef\left( \xi \right)\le f\left( {E\xi } \right)$. 3. 
Воспользовавшись 1 и 2, оптимально подберите $\lambda $. 
В действительности, для доказательства этой оценки не тербуется независимость, а нормальность можно заменить на субгауссовость.
б) Воспользуйтесь 
\textit{неравенством Буля}: $P\left( {\bigcup {U_k } } \right)\le \sum {P\left( {U_k } \right)} $. 

\end{ordre}
 
\begin{remark} (Оракульное неравенство) Можно показать, что 
существует такая константа $C>0$, что при $\tau =\sigma \sqrt {2\ln n} $ и 
числе ненулевых компонент вектора $\vec {\theta }$ равным $m\ll n$ имеет 
место следующее (с точностью до константы не улучшаемое) неравенство:
\[
E\left\| {\tilde {\vec {\theta }}-\vec {\theta }} \right\|_2^2 \le C\sigma 
^2\frac{m\ln n}{n}.
\]
Кроме того, если $ \underset{ k=1,...,n \; \theta _k \ne 0}{\min}  \theta _k >2\tau $, то
с вероятностью не меньшей $ 1-1 \mathord{\left/ {\vphantom {1 {\sqrt {\pi \ln n} }}} \right. \kern-\nulldelimiterspace} {\sqrt {\pi \ln n} }$ выполняется $\tilde {\theta }_k >0\Leftrightarrow \theta _k >0$.

 
Можно немного ``поднять'' порог $\tau $, тогда существенно улучшится 
скорость сходимости.
\end{remark}


\begin{problem}[Теорема Бернштейна--фон-Мизеса]
Пусть $\mathbb{Y} = (Y_1, ..., Y_n)$ -- независимые в совокупности, одинаково распределенные случаные величины, подчиняющиеся закону $\mathrm{Be}(p)$, причем параметр $p$ так же является случайно величиной: $p \in \mathrm{Beta}(1,1)$. Докажите, что апостериорное распределение параметра $p$ ассимптотически нормальное:

$$
p|\mathbb{Y} \rightarrow N \left(\overline{p}, \frac{1}{n}I_{p^*}^{-1} \right),
$$ 
где $p^* \in (0, 1)$ -- истиное значение параметра, $I_{p^*}$ -- информационная матрица Фишера, $\overline{p}$ -- средневыборочная оценка.

Рассмотрим следующую последовательность экспериментов: 
\begin{center}
$Y_1^1$ \\
$Y_2^1, Y_2^2$ \\ 
\ldots \\
$Y_n^1,\ldots, Y_n^n$,
\end{center}
где  $Y_k^j \in \Po(\frac{p}{k})$ независимые в совокупности одинаково распределенные случайные величины. Докажите, что:
\[
p|\mathbb{Y} \not\rightarrow N \left(\overline{p},\frac{1}{n}I_{\frac{p^*}{n}}^{-1} \right),
\]
\[
[p]|\mathbb{Y} \rightarrow \Po(p^*).
\]


\end{problem}



\begin{remark}
Приведем более общую формулировку теоремы БфМ.  Ведем вспомогательные обозначения: 
\[
p^* = \arg \max \limits_{p \in \Theta} \Exp L(p), \quad 
\Tilde{p} = \arg \max \limits_{p \in \Theta}  L(p) , \quad
\PR(p | \mathbb{Y}) \propto e^{L(p)} \pi(p).
\] 
\[
D_0^2 = - \bigtriangledown^2 \Exp L(p^*), \quad
V_0^2 = \Var[\bigtriangledown L(p^*)].
\] 
\begin{theorem}
Пусть $\PR_p$ -- некоторое семейство распределений с фиксированным носителем. Ковариационная функция $k_p(y, y')$ трижды непрерывно дифференцируема по $p$, а соответствующие ковариационные матрицы $K$, $K_p = \{k_p(y_i, y_j)\}$ удовлетворяют следующим условиям:
 собственные числа $ 0 < \lambda_{min}< \lambda < \lambda_{max} < \infty$;
$\big \Vert \frac{\partial  K_p} {\partial p_i}  \big\Vert_2 < \lambda_1 < \infty $, 
 $\big\Vert \frac{\partial^2 K_p} { \partial p_i \partial p_j } \big\Vert_2 < \lambda_2 < \infty $, $\big\Vert \frac{ \partial^3 K_p }{ \partial p_i \partial p_j \partial p_k } \big\Vert_2 < \lambda_3 < \infty $.
Минимальное собственное число матрицы $\frac{1}{n} D_0^2 > d_0 >0$ . Вектор $p^*$ существует.  $\exists r: \forall p \in \{\Vert V_0(p - p^*) \Vert_2 > r \}$ выполнено $\Exp L(p) - \Exp L(p^*) \neq 0$. Тогда $\exists \tau, x$, такие что с вероятностью $1 - c e^{-x}$ выполнено: 
\[
\big\Vert D_0 (\overline{p} - \Tilde{p}) \big\Vert_2 \leq c \tau (\dim p + x),
\]
\[
\big\Vert E_{\dim p} - D_0 \sigma^2  D_0 \big\Vert_{\infty} \leq c \tau (\dim p + x),
\]
где  значение $\tau (\dim p + x)$ мало и уменьшается с ростом выборки,
$E_{\dim p}$ -- единичная матрица, $c \gg \tau$ -- некоторая константа,
\[
\overline{p} = \Exp(p|\mathbb{Y}), \quad  \sigma^2 =  \Exp((p - \overline{p}) (p - \overline{p})^T |\mathbb{Y}).
\] 
Кроме того, $\forall \lambda:  \big\Vert \lambda \big\Vert_2 \leq (\dim p + x)$ выполнено: 
\[
\left|
\ln \Exp \left[
       exp\{\lambda^T \sigma^{-1}(p - \overline{p}) \} | \mathbb{Y} 
     \right] -  \frac{ \big\Vert \lambda \big\Vert_2}{2}
\right| \leq c \tau (\dim p + x),
\]
что описывает близость апостериорного распределения $p$ к нормальному распределению. 

Стоит отметить, что истинное распределение может не лежать в семействе:  $\mathbb{P} \not\in (\mathbb{P}_{p})$ -- модель данных является ошибочной. В таком случае 
$\mathbb{P}_{p^*}$ -- ближайшее распределение по метрике $\mathcal{KL}$.
\end{theorem}
\end{remark}


\begin{problem}[Парадокс Байеса]
Пусть $X_1, X_2, \ldots$ -- независимые с.в. из распределения с неизвестным параметром $p  \in [a, b]$. Можно ли ожидать, что последовательность апостериорных распределений (при равномерном априорном распределении) все более и более концентрируются около истинного значения $p$? Оказывается, что это не всегда верно.
Приведите пример такого распределения.   
\end{problem}

\begin{ordre}
В качестве примера можно рассмотреть следующую с.в. $X$:
\[
\PR(X = k) = c(1-p)p^k, \quad k = \overline{0, f(p)},
\]
где $f(p) \in \mathbb{N}$, $f(1/4) = f(3/4) = \infty$, $p \in [1/8, 7/8]$, $1/4$ -- истинное значение $p$, $3/4$ -- точка концентрации  апостериорных вероятностей.
\end{ordre}


\begin{problem}[Парадокс Стейна]
Пусть $\mathbb{Y} = (Y_1, ..., Y_n)$, где $Y_i \in N(\theta, \sigma^2 E_k)$, $\theta \geq 0$, $E_k$ -- единичная матрица. $\hat{\theta}$ -- произвольная оценка вектора параметров $\theta$.  Для квадратичной функции потерь $\| \theta - \hat{\theta}\|^2 = \sum \limits_{i=1}^k (\theta_i - \hat{\theta}_i)^2$ обозначим через $r_{\hat{\theta}}(\theta)$ функцию риска оценки $\hat{\theta}$, то есть $r_{\hat{\theta}}(\theta) = \Exp_{\theta} \| \theta - \hat{\theta}\|^2$.  Покажите, что $\hat{\theta} = \overline{Y}$ является допустимой оценкой (см. замечание) только при $k \leq 2$.

\begin{ordre}
Сравните значения функции потерь для оценок $\overline{Y}$ и 
\[
\max \left(0, 1 - \frac{(k-2)^2}{\Vert \overline{Y} \Vert^2} \right)\overline{Y},
\]
\[
\left(1 - \frac{(k-2)^2}{\Vert \overline{Y} \Vert^2} \right)\overline{Y}.
\]
\end{ordre}

Пусть $Y_i \in 0.91 N(\theta, 1) + 0.09 N(\theta, 9)$. Покажите, что в этом случае медиана выборки оценивает параметр лучше, чем $\overline{Y}$.
\end{problem}


\begin{remark}
Оценку параметра $\hat{\theta}_1$ называют допустимой (при заданной функции потерь), если не существует такой оценки $\hat{\theta}_2$, что при любом значении параметра $\theta$  $r_{\hat{\theta}_1}(\theta) \leq r_{\hat{\theta}_2}(\theta)$ (и хотя бы при одном значении параметра неравенство строгое).
\end{remark}

\begin{problem}
Найдите оценку максимального правдоподобия параметра $\theta$ для выборки $\mathbb{Y} = (Y_1, ..., Y_n)$, где $Y_i \in N(\theta, c \theta^2)$, где $c>0$. 
\end{problem}

\begin{remark}
Данный пример демонстрирует не единственность локального максимума функции правдоподобия. 
\end{remark}

\begin{problem}
Выборка $X_1,\ldots, X_n$  состоит из независимых с.в. из распределения  $[\theta, 2\theta]$. Покажите, что оценкой максимального правдоподобия $\theta$ является $\max(X_i) / 2$.  
\[
\widehat{\theta} = \frac{2n+2}{2n+1} \max(X_i) / 2
\] 
-- есть несмещенная оценка $\theta$ с дисперсией $1/(4n^2)$.
Найдите дисперсию более эффективной оценки оценки:
 \[
 \frac{n+1}{5n+4}(\min (X_i) + 2 \max(X_i)).
 \] 
\end{problem}

\begin{remark}
Пара $\min (X_i)$,  $\max(X_i)$ в совокупности образуют \emph{достаточную статистику} в отличие от $\max(X_i)$: т.е. при заданных значениях  $\min (X_i)$ и $\max(X_i)$ совместное распределение $X_1,\ldots, X_n$ не зависит от $\theta$. 
\end{remark}

\begin{problem}
Проведем сравнение оценок параметра $\theta$ согласно критерию:
\[
\PR(| \widehat{\theta}_1 - \theta | < | \widehat{\theta}_2 - \theta |) \mathop{>} \limits^? \frac{1}{2}.
\] 
\begin{enumerate}
\item Покажите, что  $X_1$ является более эффективной оценкой, нежели  $(X_1 + X_2)/2$, где $X_1$, $X_2$ имеют симметричную плотность распределения $f(x-\theta)$, 
\[
f(x) = \frac{3}{2} \cdot \frac{1}{(1+|x|)^4}.
\] 
\item Для простой выборки $X_1,\ldots, X_n$ из $N(\theta, 1)$ покажите, что оценка математического ожидания $\overline{X}$ менее эффективна нежели:
\[
\widehat{\theta} = 
\begin{cases}
\overline{X} - \frac{1}{2\sqrt{n}} \min\{\sqrt{n}\overline{X}, \text{Ф}(-\sqrt{n}\overline{X})\},  & \overline{X} \geq 0;\\
\overline{X} + \frac{1}{2\sqrt{n}} \min\{\sqrt{n}\overline{X}, \text{Ф}(-\sqrt{n}\overline{X})\}, & \overline{X} \leq 0,
\end{cases}
\] 
где $\text{Ф}$ обозначает функцию стандартного нормального распределения.
\end{enumerate}
\end{problem}


\begin{problem}[Критическая размерность]
Согласно теореме Фишера, ОМП $\tilde{\theta}$ близка к значению $\theta^*$ при условии $\dim \theta  = p_n = o(n^{1/3})$. Покажем на примере, что данное условие нельзя ослабить. Пусть искомое распределение $\PR = \Po(\mathrm{exp}{\theta^*})$, $\mathbb{Y} = (Y_1, ..., Y_n)$, где $Y_i \in \PR$. Определим параметрическое семейство $\PR_\theta$, в котором будем искать элемент наиболее близкий к  $\PR$ по метрике 
$\mathcal{KL}$: $Y_i \in \Po(v_j)$ при $j \in \mathcal{I}_j = \{i: \lceil i p_n / n \rceil = j \}$ (будем считать, что $n / p_n \in \mathbb{N}$). Оцениваемый параметр определим как 
\[
\theta = \frac{1}{p_n} \sum \limits_{k = 1}^{p_n} \ln (v_j).
\]
Покажите, что ОМП: 
\[
\tilde{\theta} = \frac{1}{p_n} \sum \limits_{k = 1}^{p_n} \ln \left(\frac{S_k} {n / p_n} \right), \quad S_k = \sum \limits_{i \in \mathcal{I}_k} Y_i.
\]
Используя свойства экспоненциального семейства распределений (см. замечание к задаче  \ref{KL_EF} из раздела \ref{measure}), проверьте следующее неравенство: 
\[
\PR (\tilde{\vec{v}} \in \Theta_0(r)) \geq 1 - 4 e^{-x},
\]
где $\Theta_0(r) = \{\vec{v}: \mathcal{KL}(v_j, v_j^*) n /p_n \leq r, j \in \overline{1, p_n} \}$, $r = x + \ln(p_n)$.

Данное неравенство позволяет ограничится областью $\Theta_0(r)$ при исследовании разности значений $\sup L(\theta)$ и $L(\theta^*)$.

Покажите, что:

\begin{enumerate}
\item при $\beta_n \to 0$, $p_n \to \infty$: 
$
D_0(\tilde{\theta} - \theta^*) \to N(0, 1);
$

\item при $\beta_n = \beta > 0$: 
$
D_0(\tilde{\theta} - \theta^*) \to N(\beta/2, 1);
$

\item при $\beta_n \to \infty$, $\beta_n^2 / \sqrt{p_n} \to 0$: 
$
D_0(\tilde{\theta} - \theta^*) \to -\infty,
$
\end{enumerate}
где $\beta_n = \sqrt{p_n^3 / n}$, $D_0^2 = -\bigtriangledown^{2} \Exp L(\theta^{*}) = p_n^{2} \beta_n^{-2}$.

\end{problem}

\begin{ordre}
В пунктах а) б) в)  воспользуйтесь разложением по формуле Тейлора: 
\[
p_n \beta_n^{-1}(\tilde{\theta} - \theta^*) = \beta_n^{-1} \sum \limits_{j = 1}^{p_n}\ln \left(\frac{S_j} {v^* n / p_n} \right) = \beta_n^{-1} \sum \limits_{j = 1}^{p_n} \ln \left(1 + \frac{\beta_n}{\sqrt{p_n}}\gamma_j \right) = \]\[
\frac{1}{\sqrt{p_n}} \sum \limits_{j = 1}^{p_n}\gamma_j - \frac{\beta_n}{2 p_n} \sum \limits_{j = 1}^{p_n}\gamma_j^2 + o(1),
\]   
применимость которой следует из неравенства, доказываемого в задаче: 
\[
\ln \left(\frac{S_j} {v^* n / p_n} \right) \leq \sqrt{r/p_n}. 
\]

\end{ordre}

\begin{comment}
Введем следующие обозначения, используемые в задачах на тему классификация (см. http://www.machinelearning.ru/):

${\mathbb X }$ - генеральная выборка (матрица объект-признак) размера $L$ (число объектов).

$X$ - наблюдаемоя (обучающая) выборка размера $l$. У объектов из данной выборки известны значения классов.

$\overline{X}$ - тестовая выборка размера $L - l$. У объектов из данной выборки значения классов не известны и их необходимо предсказать, проанализировав $X$.

$a:{\mathbb X } \rightarrow \{0,1\}$ - алгоритм классификации, где множество значений - есть множество индексов классов.

$I:{\mathbb X } \rightarrow \{0,1\} $ - индикатор ошибки, для заданного алгоритма.

\begin{problem}
Докажите, что при $n(a, {\mathbb X }) = m$ число ошибок в наблюдаемой подвыборке $n(a,X)$
подчиняется гипергеометрическому распределению:
\[
P(n(a,X) = s) = h^{l,m}_L (s) = \frac{C^s_m C^{l-s}_{L-m}}{C^l_L},
\]

\noindent где $s$ принимает значения от $s_0 = \max\{0,m - k\}$ до $s_1 = \min\{l,m\}$.
\end{problem}

\begin{problem}
Пусть алгоритм $a$ допускает $m$ ошибок на генеральной выборке:
$n(a, {\mathbb X }) = m$. Покажите, что в этом случае для любого $\epsilon \in [0, 1]$ справедливы равенства:
\[
Q\epsilon(a,{\mathbb X }) = P(\nu(a, \overline{X} ) - \nu(a,X) \geq \epsilon) =                     H^{l,m}_L(\frac{l}{L}(m - \epsilon k))
\]

\[
R\epsilon(a,{\mathbb X }) = P(\nu(a, \overline{X} ) \geq \epsilon) =                     H^{l,m}_L((m - \epsilon k))
\]
\[
C(a,{\mathbb X }) = \Exp\nu(a, \overline{X} ) = \frac{m}{L},
\]

\noindent где $\nu(\cdot)$ - частота ошибок алгоритма,
\[
H^{l,m}_L(z) = \overset{\lfloor z \rfloor}{\underset{s=s_0}{\sum}}h^{l,m}_L (s)
\]
\end{problem}

\begin{problem}
Для любых $\mu$ - метод обучения (выбора алгоритма по обучающей выборке), ${\mathbb X }$, $I$, $A_e$ - множество векторов ошибок, порождаемых множеством алгоритмов $A$ и $\epsilon \in [0, 1]$ требуется доказать следующие оценки:

\[
Q\epsilon(\mu, {\mathbb X }) \leq |A_e| \underset{m = 1, \ldots, L}{\max} H^{l,m}_L(\frac{l}{L}(m - \epsilon k)),
\]

\[
Q\epsilon(\mu, {\mathbb X }) \leq |A_e|\; \frac{3}{2} e^{-\epsilon^2 l}, \; l = k
\]

\end{problem}

\end{problem}



\end{comment}

\begin{remark}
Классическая постановка задачи \textit{обучения распознаванию образов} с двумя классами объектов. Изучается некоторое множество объектов $\omega \in \Omega $, каждый из которых обладает $n$ измеряемыми свойствами, выраженными действительными числами $x_{i} (\omega )\in {\rm {\mathbb R}}$, $i=1,...,n$. Совокупность результатов этих измерений будем называть вектором действительных признаков объекта $x(\omega )=\left(x_{1} (\omega )\cdots x_{n} (\omega )\right)^{T} \in {\rm {\mathbb R}}^{n} $. \underbar{}

Допустим, что все множество объектов $\omega \in \Omega $ разбито на два класса индикаторной функцией $y(\omega ):\; \Omega \to \{ -1,\; 1\} $, вообще говоря, неизвестной наблюдателю. Целью наблюдателя является определение класса предъявленного объекта $y(\omega )\in \{ -1,\; 1\} $, зная лишь доступный для непосредственного наблюдения вектор признаков $x(\omega )\in {\rm {\mathbb R}}^{n} $. Иными словами, желание наблюдателя сводится к построению дискриминантной функции $\hat{y}(x):\; {\rm {\mathbb R}}^{n} \to \{ -1,\; 1\} $. В качестве исходной информации для выбора дискриминантной функции будем рассматривать обучающую совокупность объектов, представленных и векторам их признаков $x_{j} =x(\omega _{j} )\in {\rm {\mathbb R}}^{n} $, и фактическими значениями индикаторной функции класса $y_{j} =y(\omega _{j} )\in \{ -1,\; 1\} $. Таким образом, обучающая совокупность представлена конечным множеством пар $(X,Y)=\left\{(x_{j} ,y_{j} ),\; j=1,...,N\right\}$. 
\end{remark}


\begin{problem}[Байесовская интерпретация Метода опорных векторов~SVM]

Рассмотрим следующую модель наблюдения. Пусть в ${\rm {\mathbb R}}^{n} $ определена некоторая гиперплоскость $a^{T} x+b$ с направляющим вектором $a\in {\rm {\mathbb R}}^{n} $ и параметром сдвига $b\in {\rm {\mathbb R}}$, а также пара несобственных плотностей распределения вероятностей $\phi (x|y;\; a,b,c)$, $x,a\in {\rm {\mathbb R}}^{n} $, $b,c\in {\rm {\mathbb R}}$, $y=\pm 1$ (см. замечание к задаче \ref{van_tris}), сконцентрированных преимущественно по разные стороны от этой гиперплоскости (параметр $c$ считается известным): 

\[
\phi (x|y;\; a,b,c)=\exp \left[-c\left(1-y(a^{T} x+b)\right)_{+} \right],
\]
где $(x)_{+} = \max(0, x)$. 


Направляющий вектор $a\in {\rm {\mathbb R}}^{n} $ будем рассматривать как случайный, распределенный с некоторой известной плотностью $\Psi (a)$. Никаких априорных предположений о значении случайного сдвига гиперплоскости $b\in {\rm {\mathbb R}}$ приниматься будет, так что совместное распределение $\Psi (a,b)$ будет рассматриваться как несобственное: 

\[
\Psi (a,b)\propto \Psi (a).  
\] 
Далее, пусть обучающая совокупность \[(X,Y) = \left\{(x_{j} ,y_{j} ), \; j=1,...,N\right\}\] есть результат многократных случайных независимых реализаций распределений $\phi (x|y=1;\; a,b,c)$ и $\phi (x|y=-1;\; a,b,c)$, всякий раз с известным индексом $y=\pm 1$ принадлежности очередного объекта к одному их классов. 

а) Запишите апостериорное распределение параметров разделяющей гиперплоскости после наблюдения обучающей совокупности (согласно формуле Байеса). Покажите, что с точностью до множителя, не зависящего от параметров гиперплоскости

\[\PR(a,b\, |\, X,Y)\propto \Psi (a)\Phi (X\, |\, Y;a,b),\] 
\[\Phi (X|Y;\; a,b,c)=\prod _{j=1}^{N}\phi (x_{j} |y_{j} ;\; a,b,c).\]

Обучение естественно понимать как вычисление байесовской оценки параметров разделяющей гиперплоскости: 

\[(\hat{a},\hat{b}\, |\, X,Y;c)=\mathop{\arg \max }\limits_{a\in {\rm {\mathbb R}}^{n} ,\; b\in {\rm {\mathbb R}}} \PR(a,b\, |\, X,Y).\] 

Покажите, что критерий обучения можно записать через следующую задачу оптимизации:

\[
-\ln \Psi (a)+ c \sum_{j} \left(1-y_i(a^{T} x_i+b)\right)_{+} \to \min_{a,b},
\] 
или как задачу квадратичного программирования

\[\left\{\begin{array}{l} {-\ln \Psi (a)+c\sum _{j=1}^{N}\delta _{j}  \to \min (a,b,\delta _{1} ,,...,\delta _{N} ),} \\ {y_{j} (a^{T} x_{j} +b)\ge 1-\delta _{j} ,\; \delta _{j} \ge 0,\; j=1,...,N.} \end{array}\right. \] 

б) Приняв дополнительное предположение об априорных вероятностях принадлежности случайно появляющегося объекта одному либо другому классу: 

\[q(1)=P\left(y(\omega )=1\right), q(-1)=P\left(y(\omega )=-1\right), q(1)+q(-1)=1, \] 
запишите апостериорную вероятность принадлежности объекта $\omega \in \Omega $ с вектором признаков $x\in {\rm {\mathbb R}}$ классу $y=1$ и $y=-1$: $p(y=1\, |\, x;\; a,b,c)$ и $p(y=-1\, |\, x;\; a,b,c)$.

в) Приняв дополнительное предположение, что априорной плотности распределения направляющего вектора разделяющей гиперплоскости является нормальным с нулевым математическим ожиданием и независимыми компонентами $a=(a_{1} ,...,a_{n} )$, характеризующимися одинаковыми дисперсиями $r_{1} =...=r_{n} =r$: 

\[\Psi (a)=\prod _{i=1}^{n}\frac{1}{r^{{1\mathord{\left/ {\vphantom {1 2}} \right. \kern-\nulldelimiterspace} 2} } (2\pi )^{{1\mathord{\left/ {\vphantom {1 2}} \right. \kern-\nulldelimiterspace} 2} } } \exp \left(-\frac{1}{2r} a_{i}^{2} \right),\] 
запишите критерий обучения.

г) Классическая детерминированная постановка задачи SVM имеет наглядное объяснение для выбора параметров гиперплоскости. Пусть поступила обучающая совокупность $\left\{(x_{j} ,y_{j} ),\; j=1,...,N\right\}$. Представляется естественной эвристическая идея выбрать такую разделяющую гиперплоскость $(a,b)$, которая правильно разделяет объекты двух классов: $y_{j} (a^{T} x_{j} +b)>0$ для всех $j=1,...,N$. 

Допустим, что для предъявленной обучающей совокупности разделяющая гиперплоскость существует. Но в этом случае существует континуум разделяющих гиперплоскостей. Идея В.Н. Вапника заключается в выборе той из них, которая обеспечивает наибольший «зазор» между гиперплоскостью и ближайшими точками обучающей совокупности как одного, таки другого класса $y_{j} (a^{T} x_{j} +b)\ge \varepsilon >0$. Правда, величина зазора $\varepsilon $ условна, и определяется еще и нормой направляющего вектора, поэтому задача формулируется в виде задачи условной оптимизации: 

\[y_{j} (a^{T} x_{j} +b)\ge \varepsilon \to \max_{a,b},\; j=1,\ldots,N, \; a^{T} a=1.  \] 

Такая концепция обучения названа концепцией оптимальной разделяющей гиперплоскости. 

Или эквивалентная формулировка задачи поиска оптимальной разделяющей гиперплоскости: 

\[\left\{\begin{array}{l} {a^{T} a\to \min ,} \\ {y_{j} (a^{T} x_{j} +b)\ge 1,\; j=1,...,N.} \end{array}\right. \] 

Однако предъявленная обучающая совокупность может оказаться линейно неразделимой, и поставленная оптимизационная задача не будет иметь решения. В качестве еще одной эвристики В.Н. Вапник предложил в качестве компромисса «разрешить» некоторым точкам обучающей совокупности располагаться с «неправильной» стороны разделяющей гиперплоскости, потребовав, чтобы такой дефект был минимальным: 

\[\left\{\begin{array}{l} {J(a,b,\delta _{1} ,...,\delta _{N} )=a^{T} a+C\sum _{j=1}^{N}\delta _{j}  \to \min ,} \\ {y_{j} (a^{T} x_{j} +b)\ge 1-\delta _{j} ,\; \delta _{j} \ge 0,\; j=1,...,N,} \end{array}\right. 
\] 
где $C>0$ -- некоторый коэффициент, согласующий два, вообще говоря, взаимно противоречивых требования -- обеспечить как можно меньшее значение нормы направляющего вектора и как можно меньшую ошибку классификации в пределах обучающей совокупности. Сравните классическую постановку задачи SVM с вероятностной (из п. в), чему соответствует в вероятностной интерпретации задачи параметр $C$.

Записав двойственную задачу оптимизации и найдя множители Лагранжа, выпишите значение направляющего вектора оптимальной гиперплоскости. Обратите внимание, что направляющий вектор оптимальной гиперплоскости выражается как линейная комбинация векторов признаков только части объектов обучения, для которых множитель Лагранжа ненулевой, т.е. лежащих за границей поверхности разделяющей полосы, образованной разделяющей гиперплоскостью и обладающей шириной обратно пропорциональной $\Vert a\Vert$ (это подмножество объектов обучения и называется «опорным» откуда и происходит название метода обучения -- «метод опорных векторов» или «support vector machine»).
\end{problem}


\begin{problem}
Выбирая алгоритм классификации $a: \mathbb{X} \rightarrow \{0,1\}, a \in A$ ($A$ - некоторое семейство алгоритмов) при помощи обучающей (заранее известной) выборки $X$ ($|X| = l$), исследователей инетересует оценка частоты ошибок $a$ на будущих данных $\overline{X}$: $\nu(a, \overline{X})$, причем $\mathbb{X} =  X \sqcup \overline{X}$ Для того, чтобы было возможно получить эту оценку, необходимо сделать предположение о том, что элементы генеральной выборки $\mathbb{X}$ ($X \in \mathbb{X}$, $\overline{X} \in \mathbb{X}$) появляются в случайном порядке. Причем все $L!$ перестановок ($|\mathbb{X}| = L$) равновозможны. Иногда условие ослабляют и равновозможными считаются все $C_L^l$ разбиений. Пусть алгоритм $a$ допускает на генеральной совокупности $m$ ошибок: $n(a, \mathbb{X}) = m$, докажите, что в этом случае число ошибок в наблюдаемой подвыборке $n(a, X)$ подчиняется гипергеометрическому распределению:
\begin{center}
$P(n(a, X) = s) = \frac{C_m^s C_{L-m}^{l-s}}{C_L^l}$,
\end{center}
где $s\in \left\{s_0 = \max(0, m-k),..., \min(l,m)\right\}$.
\end{problem}

\begin{problem}
В условиях предыдущей задачи определим \textit{функционалы обобщающей способности} алогритма:
\begin{enumerate}
\item вероятность переобучения $Q_{\epsilon}(a, \mathbb{X}) = P(\nu(a, \overline{X}) - \nu(a,X) \geq \epsilon)$;
\item вероятность высокой частоты ошибок на контрольной выборке $R_{\epsilon}(a, \mathbb{X}) = P(\nu(a. \overline{X})\geq \epsilon)$;
\item средняя частота ошибок на скользящем контроле $C(a, \mathbb{X}) = \mathbb{E}\nu(a, \overline{X})$.
\end{enumerate}
Докажите, что если $n(a, \mathbb{X}) = m$, то $\forall \epsilon \in [0, 1]$: $C(a,\mathbb{X}) = \frac{m}{L}$, $Q_{\epsilon}(a, \mathbb{X}) = H_L^{l,m}$, $R_{\epsilon}(a, \mathbb{X}) = H_L^{l,m}$, где $H_L^{l,m}(z) = \sum_{s=s_0}^{\lfloor z \rfloor} h_L^{l, m}(s)$, а $h_L^{l, m}(s) =  \frac{C_m^s C_{L-m}^{l-s}}{C_L^l}$.
\end{problem}

\begin{problem} Пусть $\mu$ - метод выбора алгоритма по обучающей выборке $X$ (метод обучения), $I: \mathbb{X} \rightarrow \{0, 1\}$ - индикатор ошибки для заданного алгоритма, $A_{\epsilon}$ - множество векторов ошибок, порождаемых множеством алгоритмов $A$. Доказать, что $\forall \epsilon \in [0, 1]$:
\begin{center}
$Q(\mu, \mathbb{X}) \leq |A_{\epsilon}| \max_{m = 1,...,L} H_L^{l,m}(s_m(\epsilon))$, где $s_m(\epsilon) = \frac{l}{L}(m - k\epsilon)$.
\end{center}

\begin{ordre}
Для получения верхних оценок вероятности переобучения, не зависящих от метода, часто используется \textit{принцип равномерной сходимости}
\begin{center}
$Q(\mu, \mathbb{X}) \leq P( \max_{a \in A} ( \nu (a, \overline{X}) - \nu (a,X) )\geq \epsilon)$.
\end{center}
\end{ordre}
\end{problem}


\begin{problem}[Оценка вероятности переобучения]
В статистической теории переобучения центральным объектом анализа является задача минимизации математического ожидания функции штрафа:
\[\tag{1}
M(\alpha) = \Exp \lambda(x, \alpha) = \int \lambda(x, \alpha) dF(x) \to \min,
\]
где $\alpha \in \Omega$ -- набор параметров метода обучения, $F(x)$ -- функция распределения выборки, 
$0\leq \lambda(x, \alpha) \leq \Lambda$ -- некоторая функция, измеримая $\forall \alpha \in \Omega$ относительно меры $F(x)$.  
  
Ввиду того, что в большинстве практических задач $F(x)$ неизвестна, $M(\alpha)$  приближается эмпирическим риском: 
\[\tag{2}
M_{l}(\alpha) = \frac{1}{l}\sum \limits_{i=1}^l \lambda(X_i, \alpha) \to \min,
\] 
где $\{X_i\}_{i=1}^{l}$ -- выборка из распределения 
$F(x)$.

В задаче классификации обычно в качестве $M(\alpha)$  берется вероятность неправильной классификации с помощью алгоритма $a(f, \alpha): \mathbb{F} \to \mathbb{Y}$, где $x = (f,y)$, $f$ -- множество признаков, $y$ -- индекс класса, при этом 
\[
\lambda(x, \alpha) = I[a(f, \alpha) \neq y],
\] 
\[
M(\alpha) = \PR(A_\alpha) = \PR(a(f, \alpha) \neq y).
\]
В свою очередь $M_{l}(\alpha)$ равна частоте события  $A_\alpha$ при заданной выборке.

В качестве меры близости между оптимальными значениями параметров $\alpha_{min}$, $\alpha^*$  в задачах (1) и (2) естественно взять 
\[
M(\alpha^*) - M(\alpha_{min}) \leq 2 \sup\limits_{\alpha} |M(\alpha) - M_l(\alpha)|.  
\]    
Таким образом, возникает вопрос: имеет ли место равномерная сходимость  $M_{l}(\alpha)$ к $M(\alpha)$ 
по заданной системе событий $S$ (или же по параметру $\alpha$  задающему систему событий). В случае задачи классификации $S = \{A_\alpha\}$ и близость оптимальных параметров означает близость частот к вероятностям системы $S$. 

Применив ц.п.т., покажите, что для конечной системы $S$  $M_{l}(\alpha)$ равномерно сходится к $M(\alpha)$.

Основная идея, на которой строятся условия равномерной сходимости для бесконечной системы $S$, состоит в разбиении $S$ на конечное число классов эквивалентности так, что в каждом классе события неотличимы относительно выборки. Для понимания применимости данной замены, проверьте, что близость $M_{l}(\alpha)$ к $M(\alpha)$
равносильна сходимости  $M_{l}(\alpha)$ на обучающей и тестовой выборках, а именно: 
\[
\PR \bigg\{\sup \limits_{\alpha \in \Omega} |M(\alpha) - M_l(\alpha)| > \varepsilon \bigg\} \leq 2 \PR \bigg\{\sup \limits_{\alpha \in \Omega} |M_l(\alpha) - M_{l,2l}(\alpha)| > \frac{\varepsilon}{2}  \bigg\} 
\] 
при $l > 2/ \varepsilon$.

Рассмотрим систему событий $S$ более общего вида $A(\alpha, c) = \{x: \lambda(x, \alpha) \geq c\}$ для всевозможных значений $\alpha \in \Omega$ и $c$. Обозначим за $\Delta^S(x_1,...., x_l)$ -- число классов  эквивалентности системы $S$. Введем функцию роста $m^S(l) = \max \Delta^S(x_1,..., x_l)$, где максимум берется по всем последовательностям $(x_1,...,x_l)$ длины $l$. Покажите, что 
\[\tag{3}
\PR \bigg\{\sup \limits_{\alpha \in \Omega} |M(\alpha) - M_l(\alpha)| > \varepsilon \bigg\} \leq 6m^S(2l)\exp\bigg\{-\frac{\varepsilon^2 (l -1)}{4\Lambda^2}\bigg\}.
\]
При помощи данной оценки докажите теорему Гливенко (см. задачу \ref{Fn}), взяв $S = \{x: x \leq \alpha\}$, а также ее обобщение на n-мерный случай, где  $S = \{x:  \langle x, \alpha \rangle \geq 0 \}, \; \alpha \neq 0$.
\end{problem}
\begin{remark} Для любой системы событий $S$ имеет место
\begin{center}
$m^S(l) = 2^l$ или 
\end{center}
\begin{center}
$m^S(l) \leq \sum \limits_{i=0}^{n-1}C_l^i$ 
т.е. $\exists n_0 \in \mathbb{N}: m^S(l) = O(l^{n_0})$.
\end{center}
Минимально возможное значение $n_0$ принято называть \textit{размерностью Вапника-Червоненкиса} (VC-размерность). Однако А.Я.Червоненкис предлагает называть её \textit{комбинаторной размерностью} $S$. Так, например, для множества всевозможных решающих правил в пространстве размерности $n$ комбинаторная размерность $n_0 = n + 1$. Если $m^S(l) = 2^l$, то говорят, что комбинаторная размерность бесконечна. Для рассматриваемого в задаче случая достаточным условием конечности комбинаторной размерности, как следствие равномерной сходимости с ростом объема выборки $M_{l}(\alpha)$ к  $M(\alpha)$, является то, что $\Omega$ -- компакт, $\lambda(x, \alpha)$ непрерывна по $\alpha$, $|\lambda(x, \alpha)| < K(x)$, где $\int K(x)dx < \infty$.

Наряду с достаточным условием (3) критерием равномерной сходимости является условие
\[
\lim \limits_{l \to \infty} \frac{H^S(l)}{l} = 0,
\] 
где $H^S(l) = \Exp \log_2 \Delta^S(x_1,...., x_l)$ -- энтропия системы $S$ относительно выборок длины $l$.

Отметим, что оценка (3) в большинстве случаев является чрезмерно завышенной (на несколько порядков) и поэтому не может быть использована для подсчета достаточного размера обучающей выборки на практике.     
\end{remark}


\begin{comment}
НАПОМИНАЮ ПРО ЭНТРОПИЙНЫЕ ОЦЕНКИ СНИЗУ (Концовки Лекций 11 и 12 Червоненкиса) - это полезно в виде замечания добавить!

НАПОМИНАЮ ПРО И КОНЦЕНТРАЦИЮ МЕРЫ ДЛЯ ПЕРЕСТАНОВОК И ЕЁ ИСПОЛЬЗОВАНИЕ ДЛЯ АСИМПТОТИЧЕСКОГО ОЦЕНИВАНИЯ ПЕРЕОБУЧЕНИЯ


CЛЕДУЮЩИЕ ЗАДАЧИ НУЖДАЮТСЯ В ПЕРЕФОРМУЛИРОВКИ, И ЛОГИЧНО ВСЕ "ТРИ КИТА СПОКОЙНОГО", И ВСЕ ЧТО ИЗ СЕКЕЯ, ВКЛЮЧАЯ ТО ЧТО НИЖЕ, ВСТАВИТЬ ПЕРЕД КОМБИНАТОРНОЙ ТЕОРИЕ ОБУЧЕНИЯ

\begin{problem}[Оптимальные оценки]
Назовем оценку $T_n^*$ оптимальной в классе всех оценок $\{\theta_n^*\}$, если она удовлетворяет следующему неравенству: $\mathbb{E}_{\theta}w(T_n^*, \theta) \leq \mathbb{E}_{\theta}w(\theta_n^*, \theta)$, для любого $\theta \in \Theta$, где $w(\cdot, \cdot)$ - функция потерь. Несложно видеть, что такая оценка не может быть построена. Р.Фишер предложил рассматривать только те асимптотически нормальные оценки, для которых равномерно в $\Theta$ минимальна дисперсия предельного нормального закона. Эффективность таких оценок принято считать равной единице. Оказывается, что можно привести оценку такую оценку, что найдется $\theta \in \Theta$, для которой эффективность превысит еденицу. Приведетие пример.
\begin{ordre}
Одним из возможных является \textit{пример Ходжеса-ЛеКама}, в котором строится \textit{суперэффективная оценка} для квадратичной функции потерь $\{T_n\}$:\\ 
$\lim_{n \to \infty} \mathbb{E}_{\theta}[\sqrt{n}(T_n - \theta)]^2 \leq I^{-1}(\theta)$, $\theta \in \Theta$ и существует $\theta$, при которой выполнено строгое неравенство.
Пусть $X_1,...,X_n$ - независимые в совокупности нормально распределенные случайные величины, $X_i \in N(0, \theta)$, $\theta \in \mathbb{R}$. Из оценки параметра $\theta$ методом максимального правдоподобия $\tilde{\theta} = \frac{1}{n}\sum_{i = 1}^n X_i$ может быть получена суперэффективная оценка. 
\end{ordre}
\end{problem}

\begin{problem}[Парадокс Стейна]
Пусть $\vec{X} \in N(\vec{\theta},I)$ нормально распределенный случайный вектор, $\vec{\theta} \in \mathbb{R}^n$, $n\geq 3$. Оказывается, что полученная при помощи минимизации квадратичного риска $L(\tilde{\theta}, \vec{\theta}) = \sum_{i=1}^n|\theta_i - \tilde{\theta}_i|^2$ оценка $\tilde{\theta} = \vec{X}$ может быть улучшена. Приведите пример такой оценки.
\begin{ordre}
На оценке, предложенной Стейном и Джеймсом $\theta^* = (1 - \frac{n-2}{||X||^2})\vec{X}$ действительно достигается меньшее значение квадратичного риска. Докажите это.
Парадокс является следствием того, что в неравенстве Рао-Крамера мы отказываемся от условия $b(\theta) \equiv 0$ и, если $b'(\theta) < 0$, то $\mathbb{E}(\tilde{\theta} - \theta)$ может убывать значительно быстрее, чем дисперсия несмещенной оценки с минимальной диспресией.
\end{ordre}
\end{problem}


НЕ ПОНЯЛ ПРИЧЕМ ЗДЕСЬ Джеймс - ЕСЛИ СТАТЬЯ 1956 года Стейна?
\end{comment}


\begin{problem}
\label{emp_err}
Дана {\it обучающая выборка} $X =\{(X_i,Y_i)\}_{i=1}^l$, состоящая из $l$ независимых пар $(X_i,Y_i)$ из распределения $\mathbb{P}$ и некоторая {\it функция потерь} $\lambda \colon \mathbb{Y}^2\to [0,1]$, которая характеризует величину потерь при отнесении объекта класса $y\in\mathbb{Y}$ к классу $y'\in\mathbb{Y}$.
Рассматривается задача поиска алгоритма  $a(x)$, минимизирующего {\it средний риск}:
\begin{equation}
\label{eq:rm}
M(a)\equiv\mathrm{\mathbb{E}_{\mathbb{P}}}\bigl[ \lambda\bigl(Y, a(X)\bigr) \bigr] \to \min_{a \in A}.
\end{equation}
Поскольку распределение $\mathbb{P}$ неизвестно, задачу \eqref{eq:rm} часто заменяют задачей минимизации {\it эмпирического риска}:
\begin{equation}
\label{eq:erm}
M_l(a)\equiv \frac{1}{l}\sum_{i=1}^l \lambda \bigl(Y_i, h(X_i)\bigr) \to \min_{a \in A}.
\end{equation}
Таким образом встает вопрос о соотношении величин $M(a^*)$ и $M_l(a^*)$, где $a^*$ -- решение задачи \eqref{eq:erm}.

Пусть $\mathbb{Y}=\{-1,+1\}$,  $\lambda(y,y')=\mathrm{Int}(y\neq y')$ и $A = \{a_1,\dots, a_N\}$.
Докажите, что $\forall \delta > 0$ выполнено:
\[
\mathbb{P} \left( M(a^*) \leqslant M_l(a^*) + \sqrt{\frac{\log_2{\frac{N}{\delta}}}{2l}} \right)  \geq 1-\delta.
\]
\end{problem}

\begin{ordre}
Получите оценку для $\sup_{a\in A}\bigl( M(a) - M_l(a)\bigr)$ с помощью неравенства Хефдинга из раздела \ref{measure} и неравенства Буля: $\mathbb{P}\{A\cup B\}\leqslant \mathbb{P}\{A\} + \mathbb{P}\{B\}$.
\end{ordre}

\begin{remark}
Эта оценка без изменений обобщается на произвольное множество ответов $\mathbb{Y}$ и любую функцию потерь $\lambda \colon \mathbb{Y}^2\to[0,1]$.
В том числе она может использоваться в задачах регрессии ($\mathbb{Y} = \mathbb{R}$) с квадратичной функцией потерь $\lambda(y,y') = (y - y')^2$.
\end{remark}

\begin{problem}
В постановке задачи \ref{emp_err} предположим дополнительно, что существует  алгоритм $\hat{a}\colon \mathbb{X}\to\mathbb{Y} \in A$, такой что $\mathbb{P}\{Y = \hat{a}(X)\} = 1$. Такую упрощенную постановку принято называть реализуемым случаем без шума (noise-free realizable setting).

Докажите, что $\forall \delta > 0$ выполнено:
\[
\mathbb{P} \left(
M(a^*) \leqslant M_l(a^*)+ C\cdot\frac{\log\frac{N}{\delta}}{l}
\right)
\geq 1-\delta,
\]
где $C$ "--- некоторая универсальная константа.

\end{problem}


\begin{ordre}
Покажите, что $\forall \delta > 0$ и $\forall a \in A$ выполнено:
\[
\mathbb{P} \left(
M(a) \leqslant M_l(a) + \sqrt{\frac{2M(a)\log\frac{N}{\delta}}{l}} + \frac{2\log\frac{N}{\delta}}{3l}
\right)
\geq 1-\delta.
\]
Для этого воспользуйтесь неравенством Бернштейна из раздела \ref{measure} вместе с неравенством Буля.

\end{ordre}


\begin{problem}[Радемахеровская сложность]
{\it Радемахеровской сложностью} и {\it условной радемахеровской сложностью} множества $A$ при фиксированной функции потерь $\lambda$ назовем соответственно
\[
\mathcal{R}(\lambda,  A) = 
\mathbb{E}\left[ \sup_{a\in A}
\frac{1}{l}\sum_{i=1}^l \sigma_i \lambda\bigl(Y_i,a(X_i)\bigr)
\right],
\]
\[
\mathcal{R}_l(\lambda,  A) = 
\mathbb{E}\left[ \sup_{a \in A}
\frac{1}{l}\sum_{i=1}^l \sigma_i \lambda\bigl(Y_i,a(X_i)\bigr)\Big| X
\right],
\]
где $\sigma_1,\dots,\sigma_l$ "--- последовательность независимых радемахеровских случайных величин, принимающих значения $+1$ и $-1$ с вероятностями $1/2$. Математические ожидания берутся по всем случайным величинам.

Докажите, что $\forall \delta > 0$ выполнено:
\begin{equation}
\label{eq:radem}
\mathbb{P} \left(
M(a^*) \leqslant M_l(a^*) + 2\mathcal{R}(\lambda,  A) + \sqrt{\frac{\log \frac{1}{\delta}}{2l}}
\right)
\geq 1-\delta,
\end{equation}

\begin{equation}
\label{eq:condradem}
\mathbb{P} \left(
M(a^*) \leqslant M_l(a^*) + 2\mathcal{R}_n(\lambda,  A) + 3\sqrt{\frac{\log \frac{2}{\delta}}{2n}}
\right)
\geq 1-\delta.
\end{equation}

\end{problem}

\begin{ordre}

\begin{enumerate}

\item Докажите неравенство {\it симметризации}:
\[
\mathbb{E}\left[\sup_{a\in A}\bigl(M(a)  - M_l(a) \bigr)\right]
\leqslant
2\,\mathcal{R}(\lambda,  A).
\]
Для этого введите независимую копию обучающей выборки \\ $\{(X'_i,Y'_i)\}_{i=1}^l$;

\item Воспользуйтесь два раза неравенством ограниченных разностей для случайных величин $\sup_{a \in A}\bigl(M(a) - M_l(a)\bigr)$ и
$\mathcal{R}_l(\lambda,  A)$;

\item Объедините все результаты с помощью неравенства Буля.

\end{enumerate}

\end{ordre}

\fixme{
неравенством ограниченных разностей - не па эвидан
}

\begin{remark}
Обратим внимание, что оценки \eqref{eq:radem} и \eqref{eq:condradem} справедливы для любого класса $ A$, в том числе несчетного.
Случай задачи классификации и бинарной функции потерь был изучен ранее, и для него справедлива оценка  Вапника--Червоненкиса, которая использует другую комбинаторную меру сложности семейства $ A$, известную как {\it  размерность Вапника--Червоненкиса}.
Помимо того, оценка \eqref{eq:condradem}, в отличие от оценки \eqref{eq:radem}, полностью вычислима по обучающей выборке.
\end{remark}

\begin{problem}[Неравенство Талаграна]
На декартовом произведении $\mathbb{X}\times\mathbb{Y}$  задана вероятностная мера
$\mathbb{P}$,  $\{(X_i,Y_i)\}_{i=1}^n$ "--- i.i.d обучающая выборка из $\mathbb{P}$, {\it минимизатор эмпирического риска} $a^* = \arg\min_{a\in A} M_l(a)$. Наша цель -- оценить отличие среднего риска алгоритма  $a^*$ от  минимального среднего риска. 
Для этого введем понятие {\it избыточного риска}: 
\[\mathcal{E}(a^*) = M(a^*) - \min_{a\in A} M(a).\]

Получите следующую верхнюю границу, с которой работать удобнее нежели с самим избыточным риском:   
\begin{gather*}
\mathcal{E}(a^*) 
\leqslant{}\sup_{a_1, a_2 \in A} \bigl( (M - M_l)(a_1 - a_2)\bigr)
\leqslant{}\\
\tag{1}
{}\leqslant 2\sup_{a\in A} \bigl( (M(a) - M_l(a)\bigr).
\end{gather*}

Этим путем были получены ключевые в Теории Статистического Обучения (ТСО) оценки Вапника--Червоненкиса, позже "--- оценки, основанные на Радемахеровских сложностях.

Проверьте, что если множество алгоритмов $A$, в некотором смысле, не слишком ``сложно'' (например, имеет конечную VC"=размерность), то выражение в (1) имеет порядок $O(1/\sqrt{n})$.

Для конкретного распределения $\mathbb{P}$, скорее всего, только очень маленькая часть $A$ подходит для решения задачи минимизации $M(a)$. По этой причине оптимальнее брать $\sup$ по множеству:
\[
A(\delta) = \{a\in A \colon \mathcal{E}(a) \leqslant \delta\}.
\]
Приходим к следующему результату:
\[
\delta^* = \mathcal{E}(a^*) \leqslant 
\sup_{a_1, a_2 \in A(\delta^*)} \bigl( (M - M_l)(a_1-a_2)\bigr).
\]

Используя неравенство Буске (одна из версий неравенства Талаграна), докажите, что с вероятностью не меньше $1 - e^{-t}$ справедливо следующее:
\[
\tag{2}
\sup_{a_1, a_2 \in A(\delta)} \bigl( (M - M_l)(a_1- a_2)\bigr)
\leqslant
\phi_l(\delta) + \sqrt{
2\frac{t}{l}\bigl( D^2(\delta) + 2\phi_l(\delta)\bigr)
}
+\frac{t}{2l},
\]
где
\[
D(\delta) = \sup_{a_1, a_2 \in A(\delta)} \sqrt{M\bigl( (a_1 - a_2)^2\bigr)}, \] \[
\phi_l(\delta) = \mathbb{E} \left[\sup_{a_1, a_2 \in A(\delta)} \left| (M - M_l)(a_1 - a_2) \right|\right].
\]

\begin{remark}
Неравенство (2) дает оценку порядка $o(1/\sqrt{n})$.
Установление точного порядка зависит от конкретного выбора функции потерь и условий, накладываемых на распределение $\mathbb{P}$. В частности, часто в качестве такого условия берут Tsybakov's low noise condition. 
\end{remark}
\end{problem}









\newpage


\renewcommand\refname{Литература}
% В самом списке 1. вместо [1]
\makeatletter
\renewcommand{\@biblabel}[1]{#1.}
\makeatother

\begin{thebibliography} {20}

%К обеим частям

\bibitem{3}
Колмогоров А.Н. Основные понятия теории вероятностей. – М.: Наука, 1974. 

\bibitem{1} 
Феллер В. Введение в теорию вероятностей и ее приложения. Т.~1, 2. – М.: Мир, 1984.

\bibitem{1b} 
Боровков А.А. Теория вероятностей. – М.: Наука, 1986.

\bibitem{2} 
Гнеденко Б.В. Курс теории вероятностей. – М: Наука, 1988.

\bibitem{5} 
Натан А.А., Горбачев О.Г., Гуз С.А. Теория вероятностей: Учеб. пособие. – М.: МЗ Пресс – МФТИ, 2007. 
\bibitem{51} 
Натан А.А., Горбачев О.Г., Гуз С.А.  Основы теории случайных процессов: Учеб. пособие. – М.: МЗ Пресс – МФТИ, 2003.
\bibitem{52} 
Натан А.А., Горбачев О.Г., Гуз С.А. Математическая статистика: Учеб. пособие. – М.: МЗ Пресс – МФТИ, 2005.

\bibitem{6} 
Розанов Ю.А. Лекции по теории вероятностей. – М.: Долгопрудный: Издательский дом “Интеллект”, 2008. 



\bibitem{28} 
Чеботарев А.М.  Введение в теорию вероятностей и математическую статистику для физиков. – М.: МФТИ, 2009.

\bibitem{27} 
Малышев В.А. Кратчайшее введение в современные вероятностные модели. – М.: Изд-во мехмата МГУ, 2009. 
{\small http://mech.math.msu.su/~malyshev/Malyshev/Lectures/course.pdf}

\bibitem{19} \label{durrett}
Durrett R. Probability: Theory and Examples. – М.: Cambridge Univ. Press, 2010.

\bibitem{21} \label{chiraiev}
Ширяев А.Н. Вероятность 1, 2. – М.: МЦНМО, 2011.

\bibitem{book2012}
Шень А. Вероятность: примеры и задачи. – М.: МЦНМО, 2012.

\bibitem{2013}
Босс В. Лекции по математике: Вероятность, информация, статистика. Т.~4 (см. также Т.~10, 12) – М.: УРСС, 2013.

\bibitem{7} 
Коралов Л.Б., Синай Я.Г. Теория вероятностей. Случайные процессы. – М.: МЦНМО, 2013.

\bibitem{8} 
Гмурман В.Е. Руководство к решению задач по теории вероятностей и математической статистике. – М.: Высшая школа, 1979. 

\bibitem{10} 
Прохоров А.В., Ушаков В.Г., Ушаков Н.Г. Задачи по теории вероятностей. Основные понятия. Предельные теоремы. Случайные процессы. – М.: Наука, 1986. 

\bibitem{9} 
Зубков А.М., Севастьянов Б.А., Чистяков В.П. Сборник задач по теории вероятностей. – М.: Наука, 1989. 

\bibitem{4} 
Кельберт М.Я., Сухов Ю.М. Вероятность и статистика в примерах и задачах. 1 Основные понятия теории вероятностей и математической статистики. – М.: МЦНМО, 2007.

\bibitem{44} 
Кельберт М.Я., Сухов Ю.М. Вероятность и статистика в примерах и задачах. 2 Марковские цепи как отправная точка теории случайных процессов. – М.: МЦНМО, 2010.

\bibitem{444} 
Кельберт М.Я., Сухов Ю.М. Вероятность и статистика в примерах и задачах. 3 Теория информации и кодирования. – \mbox{М.: МЦНМО}, 2014.



\bibitem{22}
Ширяев А.Н. Задачи по теории вероятностей. – М.: МЦНМО, 2011. 

\bibitem{220}
Ширяев А.Н., Эрлих И.Г., Яськов П.А. Вероятность в теоремах и задачах. – М.: МЦНМО, 2013. 

\bibitem{20} 
Кац М. Вероятность и смежные вопросы в физике. – М.: Мир, 1965.

\bibitem{book12}\label{sekei}  
Секей Г. Парадоксы в теории вероятностей и математической статистике. – М.: РХД, 2003.

\bibitem{stoianov}\label{stoianov} 
Стоянов Й. Контрпримеры в теории вероятностей. – \mbox{М.: МЦНМО}, 2012.

%К части 1

\bibitem{29}
Кнут Д., Грэхем Р., Паташник О. Конкретная математика. Основание информатики.  — М.: Мир; Бином. Лаборатория знаний, 2009.

\bibitem{lando}
Ландо С.К. Лекции о производящих функциях. - М.: МЦНМО, 2007.

\bibitem{202}
Кингман Дж. Пуассоновские процессы. -- М.: МЦНМО, 2007.

\bibitem{Gupta}\label{Gupta}
DasGupta A. Asymptotic theory of statistic and probability. - Springer, 2008.

\bibitem{13} 
Flajolet P., Sedgewick R. Analytic combinatorics. – М.: Cambr. Univ. Press, 2009.
{\small http://algo.inria.fr/flajolet/Publications/book.pdf}

\bibitem{333}
Гардинер К.В. Стохастические модели в естественных науках. -- М.: Мир, 1986.

\bibitem{101}
Ethier N.S., Kurtz T.G. Markov processes. -- Wiley Series in Probability and Mathematical Statistics. New York, 2005.

\bibitem{222}
Sandholm W. Population games and Evolutionary dynamics. Economic Learning and Social Evolution. -- MIT Press. Cambridge, 2010.

\bibitem{24}
Михайлов Г.А., Войтишек А.В Численное статистическое моделирование. Методы Монте-Карло. – М.: Академия, 2006.

\bibitem{240}
Levin D.A., Peres Y., Wilmer E.L. Markov chain and mixing times. -- AMS, 2009.

\bibitem{15} 
Алон Н., Спенсер Дж. Вероятностный метод. – М.: Бином, 2006.




\bibitem{17} 
Motwani R., Raghavan P. Randomized algorithms. – М.: Cambridge Univ. Press, 1995.

\bibitem{14} 
Ledoux M. Concentration of measure phenomenon. – М.: Amer. Math. Soc.,  Math. Surv. Mon. V. 89, 2005. 

\bibitem{1401} 
Hopcroft J., Kannan R. Computer Science Theory for the Information Age. - 2012.
{\small http://www.cs.cmu.edu/~venkatg/teaching/CStheory-infoage/}

\bibitem{BLM}\label{BLM}
Boucheron S., Lugosi G., Massart P. Concentration inequalities: A nonasymptotic theory of independence. - Oxford University Press, 2013.

\bibitem{200}
Janes E.T. Probability theory. The logic of science. - Cambridge University Press, 2003.

\bibitem{18} 
Cover T.M., Thomas J.A. Elements of Information theory. – М.:  Wiley-Interscience, 2006.

\bibitem{information}
Верещагин Н.К., Щепин Е.В. Информация, кодирование и предсказание. - М.: МЦНМО, 2012.

\bibitem{1711} 
Motwani R., Raghavan P. Randomized algorithms. – М.: Cambridge Univ. Press, 1995.

\bibitem{177}
Dubhashi D.P., Panconesi A. Concentration of measure for the analysis of randomized algorithms. - Cambridge University Press, 2009.

\bibitem{16} 
Кендалл М., Моран П. Геометрические вероятности. – М.: Наука, 1972.

\bibitem{160}
Сантало Л. Интегральная геометрия и геометрическая вероятности. - М.: Наука, 1983


\bibitem{LC}\label{LC}
Lugosi G., Cesa-Bianchi N. Prediction, learning and games. - New York: Cambridge University Press, 2006.

\bibitem{2031}
Rakhlin A., Sridharan K. Statistical Learning Theory and Sequential Prediction. - STAT928, 2014.
{\small http://www-stat.wharton.upenn.edu/~rakhlin/}

\bibitem{pmr}
Bishop C.M. Pattern Recognition and Machine Learning. – М.: Springer,  Information Science and Statistics,  2006.


\bibitem{lagutin}
Лагутин М.Б. Наглядная математическая статистика. - М.: Бином, 2009.

\bibitem{spok}\label{spok}
Голубев Г.К., А.Н. Соболевский, Спокойный В.Г.;  Пособия по теории вероятностей и математической статистике. 
Электронные версии доступны здесь.
http://premolab.ru/content/books



\end{thebibliography}




\section{Вспомогательные материалы}

\subsection*{Метод зеркального спуска}

Рассмотрим задачу стохастической онлайн оптимизации\footnote{ Запись $\Exp_{\xi 
^k} \left[ {f_k \left( {x;\xi ^k} \right)} \right]$ означает, что 
математическое ожидание берется по $\xi^k$, то есть $x$ и $f_k $ понимаются 
в такой записи не случайными. Отметим, что $\xi ^k$ может зависеть от $\xi 
^1$, {\ldots}, $\xi ^{k-1}$, а распределение $\xi^k$ может зависеть от $x$.}
\begin{equation}
\label{eq1}
\frac{1}{N}\sum\limits_{k=1}^N {\Exp_{\xi ^k} \left[ {f_k \left( {x;\xi ^k} 
\right)} \right]} \to \mathop {\min }\limits_{x\in S_n \left( 1 \right)} 
,
\end{equation}
\[
S_n \left( 1 \right)=\left\{ {x\ge 0: \; \sum\limits_{i=1}^n {x_i =1} } 
\right\}
\]
при следующих условиях:

\begin{enumerate}
\item $\Exp_{\xi^k} \left[ {f_k \left( {x;\xi ^k} \right)} \right]\quad -$ выпуклые функции (по $x)$, для этого достаточно выпуклости по $x$ функций $f_k \left( {x;\xi ^k} \right)$;
\item Существует такой вектор $\nabla _x f_k \left( {x;\xi ^k} \right)$, который для компактности будем называть субградиентом, хотя последнее верно не всегда, что
\[
\Exp_{\xi ^k} \left( {\left. {\nabla _x f_k \left( {x;\xi ^k} \right)-\nabla _x 
\Exp_{\xi ^k} \left[ {f_k \left( {x;\xi ^k} \right)} \right]} \right|\Xi 
^{k-1}} \right)\equiv 0,
\]
где $\Xi ^{k-1}$ -- $\sigma $-алгебра, порожденная с.в. 
$\xi ^1, \ldots ,\xi ^{k-1}$. Далее мы будем использовать 
обозначения обычного градиента для векторов, которые мы назвали здесь 
субградиентами. 
%В частности, если мы имеем дело с обычным субградиентом, то 
%запись $\nabla _x f_k \left( {x;\xi ^k} \right)$ в вычислительном контексте 
%(например, в итерационной процедуре МЗС, описанной ниже) означает какой-то 
%его элемент (не важно какой именно), а если в контексте проверки условий 
%(например, в условии 3 ниже), то $\nabla _x f_k \left( {x;\xi ^k} \right)$ 
%пробегает все элементы субградиента (говорят также, субдифференциала);

\item $\left\| {\nabla _x f_k \left( {x;\xi ^k} \right)} \right\|_\infty \le M\quad -$ (равномерно, с вероятностью 1) ограниченный субградиент. Для справедливости части утверждений достаточно требовать одно из следующих (более слабых) условий:
\end{enumerate}
\begin{center}
 $1) \Exp_{\xi ^k} \left[ {\left\| {\nabla _x f_k \left( {x;\xi ^k} \right)} 
\right\|_\infty ^2 } \right]\le M^2$; $2) \Exp_{\xi ^k} \left[ {\left. {\exp 
\left( {\frac{\left\| {\nabla _x f_k \left( {x;\xi ^k} \right)} 
\right\|_\infty ^2 }{M^2}} \right)} \right|\Xi ^{k-1}} \right]\le \exp 
\left( 1 \right)$.
\end{center}

Онлайновость постановки задачи допускает, что на каждом шаге $k$ функция 
$f_k $ может подбираться из рассматриваемого класса функций враждебно по 
отношению к используемому нами методу генерации последовательности $\left\{ 
{x^k} \right\}$. В частности, $f_k $ может зависеть от $\left\{ {x^1,\xi 
^1;...;x^{k-1},\xi ^{k-1};x^k} \right\}$, если выбор $x^k$ осуществляется 
исходя только из информации $\left\{ {x^1,\xi ^1;...;x^{k-1},\xi ^{k-1}} 
\right\}$, т.е. без дополнительной рандомизации. Ситуация с дополнительной 
рандомизацией при выборе $x^k$ рассматривается в следующем пункте.

Для решения задачи (\ref{eq1}) воспользуемся адаптивным методом зеркального спуска (точнее двойственных усреднений). Положим $x_i^1 =1 \mathord{\left/ 
{\vphantom {1 n}} \right. \kern-\nulldelimiterspace} n$, $i=1,...,n$. Пусть 
$t=1,...,N-1$.

\underline {Алгоритм МЗС1-адаптивный}

\[
x_i^{t+1} \propto \exp \left( {-\frac{1}{\beta _{t+1} }\sum\limits_{k=1}^t 
{\frac{\partial f_k \left( {x^k;\xi ^k} \right)}{\partial x_i }} } 
\right),\quad i=\overline{1,n},\; \beta _t 
=\frac{M\sqrt t }{\sqrt {\ln n} }.
\]

\begin{theorem}
Пусть справедливы условия а, б, в.1, тогда
\[
\mbox{ }\frac{1}{N}\sum\limits_{k=1}^N {\Exp\left[ {f_k \left( {x^k;\xi ^k} 
\right)} \right]} -\mathop {\min }\limits_{x\in S_n \left( 1 \right)} 
\frac{1}{N}\sum\limits_{k=1}^N {\Exp \left[ {f_k \left( {x;\xi ^k} 
\right)} \right]} \le 2M\sqrt {\frac{\ln n}{N}} .
\]
Если $f_k = f$, а $\left\{ {\xi ^k} \right\}$ -- независимы и одинаково распределены, как $\xi$, то
\[
\Exp\left[ {f\left( {\frac{1}{N}\sum\limits_{k=1}^N {x^k} ;\xi } \right)} 
\right]-\mathop {\min }\limits_{x\in S_n \left( 1 \right)} \Exp \left[ 
{f\left( {x;\xi } \right)} \right]\le 2M\sqrt {\frac{\ln n}{N}} .
\]
Пусть справедливы условия а, б, в, тогда\footnote{ Запись\par ``$\PR_{x^1,...,x^N} \left\{ 
{\frac{1}{N}\sum\limits_{k=1}^N {\Exp_{\xi ^k} \left[ {f_k \left( {x^k;\xi ^k} 
\right)} \right]} -...} \right.$''\par означает, что внутри фигурных скобок мы 
считаем математическое ожидание по $\xi ^k$ при фиксированных $x^k$.} при $\Omega \ge 0$
\[
\PR_{x^1,...,x^N} \left\{ {\frac{1}{N}\sum\limits_{k=1}^N {\Exp_{\xi ^k} \left[ 
{f_k \left( {x^k;\xi ^k} \right)} \right]} -\mathop {\min }\limits_{x\in S_n 
\left( 1 \right)} \frac{1}{N}\sum\limits_{k=1}^N {\Exp \left[ {f_k 
\left( {x;\xi ^k} \right)} \right]} \ge } \right.
\]
\[
\left. {\frac{2M}{\sqrt N }\left( {\sqrt {\ln n} +\sqrt {8\Omega } } 
\right)} \right\} \le \exp \left( {-\Omega } \right).
\]
Если $f_k = f$, а $\left\{ {\xi ^k} \right\}$ -- независимы и одинаково распределены, как $\xi$, то
\[
\PR_{x^1,...,x^N} \left\{ {\Exp_\xi \left[ {f\left( 
{\frac{1}{N}\sum\limits_{k=1}^N {x^k} ;\xi } \right)} \right]-\mathop {\min 
}\limits_{x\in S_n \left( 1 \right)} \Exp \left[ {f\left( {x;\xi } \right)} 
\right]} \right.
\]
\[
\left. {\ge \frac{2M}{\sqrt N }\left( {\sqrt {\ln n} +\sqrt {8\Omega } } 
\right)} \right\} \le \exp \left( {-\Omega } \right).
\]
\end{theorem}

\textbf{Замечание.} Для задач стохастической выпуклой оптимизации (см., 
например, Shapiro A., Dentcheva D., Ruszczynski A. Lecture on stochastic programming. Modeling and theory. MPS-SIAM series on Optimization, 2009), в которых $\left\{ {\xi ^k} \right\}$ -- 
независимые случайные величины, можно привести оценку вероятностей больших 
уклонений с точностью до констант аналогичную оценке, приведенной в теореме 
и для случая, когда вместо условия в) предполагается условие в.2 при этом в 
правой части неравенства под вероятностью вместо $\sqrt \Omega $ необходимо 
писать $\Omega $ (с точностью до констант).

Снова рассмотрим постановку задачи стохастической онлайн оптимизации (\ref{eq1}). Но на этот раз будем допускать, что метод генерирования последовательности 
$\left\{ {x^k} \right\}$ может допускать (внешнюю, дополнительную) 
рандомизацию. Т.е.  вместо   Также как и раньше онлайновость постановки задачи допускает, 
что на каждом шаге $k$ функция $f_k $ может подбираться из рассматриваемого 
класса функций враждебно по отношению к используемому нами методу генерации 
последовательности $\left\{ {x^k} \right\}$. В частности, $f_k $ и $\xi ^k$ 
могут зависеть от $\left\{ {x^1,\xi ^1;...;x^{k-1},\xi ^{k-1}} \right\}$, и 
даже от распределения вероятностей $p^k$ (многорукие бандиты), согласно 
которому осуществляется выбор $x^k$. Чтобы можно было работать с таким 
классом задач, нам придется наложить дополнительное условие:


г) На каждом шаге генерирование случайной величины $x^k$ согласно распределению вероятностей $p^k$ осуществляется независимо ни от чего.

Положим $p_i^1 =x_i^1 =1 \mathord{\left/ {\vphantom {1 n}} \right. 
\kern-\nulldelimiterspace} n$, $i=1,...,n$. Пусть $t=1,...,N-1$.

$ $

\underline {Алгоритм МЗС2-адаптивный}

\noindent Согласно распределению вероятностей

\[
p_i^{t+1} \propto \exp \left( {-\frac{1}{\beta _{t+1} }\sum\limits_{k=1}^t 
{\frac{\partial f_k \left( {x^k;\xi ^k} \right)}{\partial x_i }} } 
\right), \quad i=\overline{1,n},\;
\beta _t =\frac{M\sqrt t }{\sqrt {\ln n} },
\]
получаем случайную величину 
$i\left( {t+1} \right)$, т.е. \\ $x_{i\left( {t+1} \right)}^{t+1} =1,\;\;x_j^{t+1} 
=0,\;j\ne i\left( {t+1} \right).$



\underline {Алгоритм МЗС2-неадаптивный (заранее известно $N$)}

\noindent Согласно распределению вероятностей

\[
p_i^{t+1} \propto \exp \left( {-\frac{1}{\beta _{t+1} }\sum\limits_{k=1}^t 
{\gamma _k \frac{\partial f_k \left( {x^k;\xi ^k} \right)}{\partial x_i }} } 
\right),\quad i=\overline{1,n},
\]
\[
\gamma _k \equiv M^{-1}\sqrt {{2\ln n} \mathord{\left/ {\vphantom {{2\ln n} 
N}} \right. \kern-\nulldelimiterspace} N} ,
\quad
\beta _t \equiv 1,
\]
получаем случайную величину 
$i\left( {t+1} \right)$, т.е. \\ $x_{i\left( {t+1} \right)}^{t+1} =1,\;\;x_j^{t+1} 
=0,\;j\ne i\left( {t+1} \right).$

Аппроксимируя
\[
\mathop {\min }
\limits_{x\in S_n \left( 1 \right)} 
\frac{1}{N}
\sum_{k=1}^t 
f_k \left( x \right) \approx \mathop {\min} \limits_{x\in S_n \left( 1 \right)}  
\frac{1}{N}\sum\limits_{k=1}^t 
\left\{ 
  f_k \left( x^k  \right)+
  \left\langle \nabla f_k \left( {x^k } 
   \right),x-x^k  \right\rangle  
\right\} ,
\]
получим
\[
\PR\left( {x_j^{t+1} =1;\;x_i^{t+1} =0,\;i\ne j} \right)\mathop =
\]
\[ 
= \PR_\varsigma \left( {j=\arg \mathop {\max }\limits_{i=1,...,n} \left\{ 
{\sum\limits_{k=1}^t {\left[ {-\nabla f_k \left( {x^k } \right)} \right]_i 
+\varsigma _{t,i} } } \right\}} \right),
\]
где $\varsigma _{t,i} $ -- независимые одинаково распределенные случайные 
величины по закону Гумбеля с параметром $\beta _{t+1} $, характеризующим 
среднеквадратичное отклонение $\varsigma _{t,i} $ (см. задачу \ref{gumbel} раздела 
\ref{hard}):\footnote{ Поскольку случайные величины $\varsigma _{t,i} $ получаются в 
результате суммирования $t$ слагаемых (невязок в аппроксимации выпуклой 
функции линейными минорантами), то можно ожидать, что среднеквадратичное 
отклонение $\varsigma _{t,i} $ имеет порядок $\sqrt t $. Условие одинаковой 
распределенности $\varsigma _{t,i} $, которое не понятно за счет чего может 
иметь место, в действительности, не очень здесь и нужно. Достаточно, чтобы 
$E\left[ {\varsigma _{t,i} -\varsigma _{t,j} } \right]=o\left( {\sqrt t } 
\right)$ и $\mbox{Var}\left[ {\varsigma _{t,i} } \right]\sim \sqrt t $. 
Более того, если, с некоторой натяжкой, наряду с независимостью $\left\{ 
{\varsigma _{t,i} } \right\}_{i=1}^n $ также считать, что при формировании 
$\varsigma _{t,i} $ суммируются независимые слагаемые, то можно ожидать, что 
$\varsigma _{t,i} $ нормальные случайные величины (центральная предельная 
теорема). Следовательно, в контексте последующих операций при $n\gg 1$, 
$\varsigma _{t,i} $ могут быть заменены на случайные величины, 
распределенные по закону Гумбеля с соответствующими среднеквадратичными 
отклонениями. }
\[
P\left( {\varsigma _{t,i} <\tau } \right)=\exp \left\{ {-e^{-\tau 
\mathord{\left/ {\vphantom {\tau {\beta _{t+1} }}} \right. 
\kern-\nulldelimiterspace} {\beta _{t+1} }}} \right\}.
\]
Тогда (см. задачу \ref{gibbs} раздела \ref{hard})
\[
E_\varsigma \left[ {x^{t+1}} \right]=\nabla W_{\beta _{t+1} } \left( 
{-\sum\limits_{k=1}^t {\nabla f_k \left( {x^k } \right)} } \right).
\]
Распределение Гумбеля возникает именно в таком контексте совсем не случайно, 
и связано это с тем, что оно max-устойчиво. 
Другими словами, при некоторых довольно общих предположениях (типа Крамера) 
$\varsigma _{t,i} $ могут быть распределенными произвольно, тем не менее, 
при $n\gg 1$ с хорошей точностью можно считать, что мы в итоге имеем дело с 
соответствующим распределением Гумбеля. Более того, если задаться вопросом: 
а какое распределение ``наиболее подходит'' для $\varsigma _{t,i} $, чтобы в 
случае ``враждебной Природы'' (то есть в минимаксном смысле) иметь наилучшие 
оценки, то ответом будет: симметричное показательное распределение 
(Лапласа), которое ведет себя в интересном для анализа диапазоне подобно 
распределению Гумбеля, но в отличие от распределения Лапласа, в случае 
Гумбеля мы явно можем посчитать интересующие нас вероятности. Как правило, 
такого рода задачи явно не решаются, и распределение Гумбеля является 
приятным исключением, для которого есть явные формулы. 

К сожалению, не делая относительно функций $f_k \left( {x;\xi ^k} \right)$ 
дополнительно никаких предположений, не удается доказать для МЗС2 аналог 
теоремы 1. Чтобы можно было сформулировать такой аналог, мы вынуждены будем 
предполагать, что $f_k \left( {x;\xi ^k} \right)$ -- линейные функции по $x$ 
(можно обобщить и на сублинейные). С одной стороны это существенно сужает 
класс задач, к которым применим МЗС2. С другой стороны, как будет 
продемонстрировано в следующем пункте, даже такой узкий класс функций за 
счет ``онлайновости'' позволяет применять МЗС2 к довольно широкому кругу 
задач. Для того чтобы лучше чувствовалась преемственность методов и 
доказательств их сходимости, далее мы по-прежнему будем использовать общие 
обозначения $f_k \left( {x;\xi ^k} \right)$, не подчеркивая в формулах 
линейность.

\begin{theorem}

Пусть справедливы условия а, б, в.1, г и $f_k \left( {x;\xi ^k} \right)$ -- линейные функции по $x$, тогда
\[
\frac{1}{N}\sum\limits_{k=1}^N {\Exp\left[ {f_k \left( {x^k;\xi ^k} \right)} 
\right]} -\mathop {\min }\limits_{x\in S_n \left( 1 \right)} 
\frac{1}{N}\sum\limits_{k=1}^N {\Exp\left[ {f_k \left( {x;\xi ^k} \right)} 
\right]} \le 2M\sqrt {\frac{\ln n}{N}} .
\]
Для неадаптивного метода ``2''-у перед $M$ можно занести под знак корня.
Кроме того если справедливы условия а, б, в, г, при  $\Omega \ge 0$
\[
\PR_{x^1,...,x^N} \left\{ {\frac{1}{N}\sum\limits_{k=1}^N {\Exp_{\xi ^k} \left[ 
{f_k \left( {x^k;\xi ^k} \right)} \right]} -\mathop {\min }\limits_{x\in S_n 
\left( 1 \right)} \frac{1}{N}\sum\limits_{k=1}^N {\Exp_{\xi ^k} \left[ {f_k 
\left( {x;\xi ^k} \right)} \right]} \ge } 
\right.
\]
\[
\left. {\frac{2M}{\sqrt N 
}\left( {\sqrt {\ln n} +\sqrt {18\Omega } } \right)} \right\} \le \exp 
\left( {-\Omega } \right).
\]
Если $f_k \left( {x;\xi ^k} \right) = f_k \left( x \right)$ , то это неравенство можно уточнить
\[
\frac{2M}{\sqrt N }\left( {\sqrt {\ln n} +\sqrt {18\Omega } } \right)\to 
\frac{2M}{\sqrt N }\left( {\sqrt {\ln n} +\sqrt {2\Omega } } \right),
\]
при этом же условии для неадаптивного метода можно еще больше уточнить:
\[
\frac{2M}{\sqrt N }\left( {\sqrt {\ln n} +\sqrt {2\Omega } } \right)\to 
\frac{\sqrt 2 M}{\sqrt N }\left( {\sqrt {\ln n} +2\sqrt \Omega } \right).
\]

\end{theorem}

\end{document}