\begin{center}
\textbf{
Список обозначений
}
\end{center}

\noindent $\langle \Omega, \mathcal{F}, \PR \rangle$ -- вероятностное пространство ($\Omega$ -- множество исходов, $\mathcal{F}$ -- $\sigma$-алгебра, $\PR$ -- вероятностная мера); \\
$p(x), \; f(x), \; f_X(x)$ -- плотность распределения случайной величины $X$;
$\Exp X$ -- математическое ожидание случайной величины $X$; \\
$\Exp_p X$ -- математическое ожидание $X$ с плотностью распределения $p$; \\
$\Exp_\PR X$ -- математическое ожидание $X$ по мере $\PR$; \\
$\Var X$ -- дисперсия случайной величины $X$; \\
$\N(m,\sigma^2)$ -- нормальное распределение; \\
$\Po(\lambda)$ -- распределение Пуассона; \\
$\Dir(\alpha_1,...,\alpha_n)$ -- распределение Дирихле;\\
$\Beta(\alpha,\beta)$ -- бета-распределение;\\
$\Be(p)$ -- распределение Бернулли; \\
$R[a,b], [a,b]$ -- равномерное распределение; \\
$\Phi(x)$ -- функция стандартного нормального распределения $\N(0,1)$; \\
$\mathrm{Exp}(x)$ -- показательное распределение; \\
$\overset{d}{\longrightarrow}$ $\left(\mathop{\longrightarrow}\limits_{n\to\infty}^{d}\right)$ -- сходимость по распределению, в ряде случаев $n\to\infty$ опущено во избежание громоздких обозначений; \\
$\overset{p}{\longrightarrow}$ -- сходимость по вероятности; \\
$\overset{\text{п.н.}}{\longrightarrow}$ -- сходимость с вероятностью 1;\\
$[x^{n}]\varphi(x)$ -- коэффициент при $x^n$ в разложении в степенной ряд функции $\varphi(x)$;\\
с.в. -- случайная величина; \\
з.б.ч. -- закон больших чисел; \\
х.ф. -- характеристическая функция; \\
ц.п.т. -- центральная предельная теорема; \\
ПФ -- производящая функция; \\
ЭПФ -- экспоненциальная производящая функция;\\
$\langle \cdot, \cdot\rangle$ -- скалярное произведение; \\
Индикаторная функция:
\[
\I(\text{true}) = [\text{true}] = 1, \quad \I(\text{false}) = [\text{false}] = 0;
\]
Простая выборка с плотностью распределения $p$:
\[
X_1,\ldots,X_n \sim p(X);
\]
В схожих ситуациях используется символ $\in$ вместо $\sim$ (например, $X \in \N(m, \sigma^2)$), если подразумевается принадлежность случайной величины  семейству распределений, $\sim$ также обозначает пропорциональность; \\ 
Ненормированная плотность распределения:
\[
p(x) \propto g(x), \quad p(x) = \frac{g(x)}{\int g(x) dx};
\]
Математическое ожидание по заданной переменной:
\[
\Exp_X h(X,Y) = \int_{-\infty}^{+\infty} h(x,Y) dF(x); 
\]
Дивергенция Кульбака--Лейблера для распределений $\PR_1$ и $\PR_2$ с общим носителем~$\Omega$ (соответствующие плотности распределений обозначены как $p_1$ и $p_2$):
\[
\KL(\PR_1 \Vert \PR_2) = \int_{\Omega} \log \left( \frac{p_1(x)}{p_2(x)} \right) p_1(x) dx;
\]
Дивергенция Кульбака--Лейблера для распределения $\PR_\theta$, зависящего от параметра (соответствующая плотность распределения обозначена как $p(x| \theta)$):
\[
\KL(\theta_1 \Vert \theta_2) = \KL(\PR_{\theta_1} \Vert \PR_{\theta_2});
\]
$A \triangle B = A \setminus B + B \setminus A$ -- симметрическая разность; \\
$\log = \ln$; \\
$\mathcal{B}$ -- борелевская сигма алгебра;




