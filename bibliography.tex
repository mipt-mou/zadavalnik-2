\newpage


\renewcommand\refname{Литература}
% В самом списке 1. вместо [1]
\makeatletter
\renewcommand{\@biblabel}[1]{#1.}
\makeatother

\begin{thebibliography} {20}

%К обеим частям

\bibitem{3}
Колмогоров А.Н. Основные понятия теории вероятностей. – М.: Наука, 1974. 

\bibitem{1} 
Феллер В. Введение в теорию вероятностей и ее приложения. Т.~1, 2. – М.: Мир, 1984.

\bibitem{1b} 
Боровков А.А. Теория вероятностей. – М.: Наука, 1986.

\bibitem{2} 
Гнеденко Б.В. Курс теории вероятностей. – М: Наука, 1988.

\bibitem{5} 
Натан А.А., Горбачев О.Г., Гуз С.А. Теория вероятностей: Учеб. пособие. – М.: МЗ Пресс – МФТИ, 2007. 
\bibitem{51} 
Натан А.А., Горбачев О.Г., Гуз С.А.  Основы теории случайных процессов: Учеб. пособие. – М.: МЗ Пресс – МФТИ, 2003.
\bibitem{52} 
Натан А.А., Горбачев О.Г., Гуз С.А. Математическая статистика: Учеб. пособие. – М.: МЗ Пресс – МФТИ, 2005.

\bibitem{6} 
Розанов Ю.А. Лекции по теории вероятностей. – М.: Долгопрудный: Издательский дом “Интеллект”, 2008. 



\bibitem{28} 
Чеботарев А.М.  Введение в теорию вероятностей и математическую статистику для физиков. – М.: МФТИ, 2009.

\bibitem{27} 
Малышев В.А. Кратчайшее введение в современные вероятностные модели. – М.: Изд-во мехмата МГУ, 2009. 
{\small http://mech.math.msu.su/~malyshev/Malyshev/Lectures/course.pdf}

\bibitem{19} \label{durrett}
Durrett R. Probability: Theory and Examples. – М.: Cambridge Univ. Press, 2010.

\bibitem{21} \label{chiraiev}
Ширяев А.Н. Вероятность 1, 2. – М.: МЦНМО, 2011.

\bibitem{book2012}
Шень А. Вероятность: примеры и задачи. – М.: МЦНМО, 2012.

\bibitem{2013}
Босс В. Лекции по математике: Вероятность, информация, статистика. Т.~4 (см. также Т.~10, 12) – М.: УРСС, 2013.

\bibitem{7} 
Коралов Л.Б., Синай Я.Г. Теория вероятностей. Случайные процессы. – М.: МЦНМО, 2013.

\bibitem{8} 
Гмурман В.Е. Руководство к решению задач по теории вероятностей и математической статистике. – М.: Высшая школа, 1979. 

\bibitem{10} 
Прохоров А.В., Ушаков В.Г., Ушаков Н.Г. Задачи по теории вероятностей. Основные понятия. Предельные теоремы. Случайные процессы. – М.: Наука, 1986. 

\bibitem{9} 
Зубков А.М., Севастьянов Б.А., Чистяков В.П. Сборник задач по теории вероятностей. – М.: Наука, 1989. 

\bibitem{4} 
Кельберт М.Я., Сухов Ю.М. Вероятность и статистика в примерах и задачах. 1 Основные понятия теории вероятностей и математической статистики. – М.: МЦНМО, 2007.

\bibitem{44} 
Кельберт М.Я., Сухов Ю.М. Вероятность и статистика в примерах и задачах. 2 Марковские цепи как отправная точка теории случайных процессов. – М.: МЦНМО, 2010.

\bibitem{444} 
Кельберт М.Я., Сухов Ю.М. Вероятность и статистика в примерах и задачах. 3 Теория информации и кодирования. – \mbox{М.: МЦНМО}, 2014.



\bibitem{22}
Ширяев А.Н. Задачи по теории вероятностей. – М.: МЦНМО, 2011. 

\bibitem{220}
Ширяев А.Н., Эрлих И.Г., Яськов П.А. Вероятность в теоремах и задачах. – М.: МЦНМО, 2013. 

\bibitem{20} 
Кац М. Вероятность и смежные вопросы в физике. – М.: Мир, 1965.

\bibitem{book12}\label{sekei}  
Секей Г. Парадоксы в теории вероятностей и математической статистике. – М.: РХД, 2003.

\bibitem{stoianov}\label{stoianov} 
Стоянов Й. Контрпримеры в теории вероятностей. – \mbox{М.: МЦНМО}, 2012.

%К части 1

\bibitem{29}
Кнут Д., Грэхем Р., Паташник О. Конкретная математика. Основание информатики.  — М.: Мир; Бином. Лаборатория знаний, 2009.

\bibitem{lando}
Ландо С.К. Лекции о производящих функциях. - М.: МЦНМО, 2007.

\bibitem{202}
Кингман Дж. Пуассоновские процессы. -- М.: МЦНМО, 2007.

\bibitem{Gupta}\label{Gupta}
DasGupta A. Asymptotic theory of statistic and probability. - Springer, 2008.

\bibitem{13} 
Flajolet P., Sedgewick R. Analytic combinatorics. – М.: Cambr. Univ. Press, 2009.
{\small http://algo.inria.fr/flajolet/Publications/book.pdf}

\bibitem{333}
Гардинер К.В. Стохастические модели в естественных науках. -- М.: Мир, 1986.

\bibitem{101}
Ethier N.S., Kurtz T.G. Markov processes. -- Wiley Series in Probability and Mathematical Statistics. New York, 2005.

\bibitem{222}
Sandholm W. Population games and Evolutionary dynamics. Economic Learning and Social Evolution. -- MIT Press. Cambridge, 2010.

\bibitem{24}
Михайлов Г.А., Войтишек А.В Численное статистическое моделирование. Методы Монте-Карло. – М.: Академия, 2006.

\bibitem{240}
Levin D.A., Peres Y., Wilmer E.L. Markov chain and mixing times. -- AMS, 2009.

\bibitem{15} 
Алон Н., Спенсер Дж. Вероятностный метод. – М.: Бином, 2006.




\bibitem{17} 
Motwani R., Raghavan P. Randomized algorithms. – М.: Cambridge Univ. Press, 1995.

\bibitem{14} 
Ledoux M. Concentration of measure phenomenon. – М.: Amer. Math. Soc.,  Math. Surv. Mon. V. 89, 2005. 

\bibitem{1401} 
Hopcroft J., Kannan R. Computer Science Theory for the Information Age. - 2012.
{\small http://www.cs.cmu.edu/~venkatg/teaching/CStheory-infoage/}

\bibitem{BLM}\label{BLM}
Boucheron S., Lugosi G., Massart P. Concentration inequalities: A nonasymptotic theory of independence. - Oxford University Press, 2013.

\bibitem{200}
Janes E.T. Probability theory. The logic of science. - Cambridge University Press, 2003.

\bibitem{18} 
Cover T.M., Thomas J.A. Elements of Information theory. – М.:  Wiley-Interscience, 2006.

\bibitem{information}
Верещагин Н.К., Щепин Е.В. Информация, кодирование и предсказание. - М.: МЦНМО, 2012.

\bibitem{1711} 
Motwani R., Raghavan P. Randomized algorithms. – М.: Cambridge Univ. Press, 1995.

\bibitem{177}
Dubhashi D.P., Panconesi A. Concentration of measure for the analysis of randomized algorithms. - Cambridge University Press, 2009.

\bibitem{16} 
Кендалл М., Моран П. Геометрические вероятности. – М.: Наука, 1972.

\bibitem{160}
Сантало Л. Интегральная геометрия и геометрическая вероятности. - М.: Наука, 1983


\bibitem{LC}\label{LC}
Lugosi G., Cesa-Bianchi N. Prediction, learning and games. - New York: Cambridge University Press, 2006.

\bibitem{2031}
Rakhlin A., Sridharan K. Statistical Learning Theory and Sequential Prediction. - STAT928, 2014.
{\small http://www-stat.wharton.upenn.edu/~rakhlin/}

\bibitem{pmr}
Bishop C.M. Pattern Recognition and Machine Learning. – М.: Springer,  Information Science and Statistics,  2006.


\bibitem{lagutin}
Лагутин М.Б. Наглядная математическая статистика. - М.: Бином, 2009.

\bibitem{spok}\label{spok}
Голубев Г.К., А.Н. Соболевский, Спокойный В.Г.;  Пособия по теории вероятностей и математической статистике. 
Электронные версии доступны здесь.
http://premolab.ru/content/books



\end{thebibliography}


