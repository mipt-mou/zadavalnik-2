\section{Геометрические вероятности}
\label{geom}

\begin{problem}

Три бабочки капустницы садятся на круглый кочан капусты радиуса 1 случайным образом (имеется в виду, что место положение каждой бабочки -- с.в., равномерно распределенная на сфере) и независимо друг от друга. Если между двумя бабочками (геодезическое) расстояние оказывается меньше ${\pi \mathord{\left/ {\vphantom {\pi  2}} \right. \kern-\nulldelimiterspace} 2} $, то обе улетают. Найдите вероятность того, что на капусте сидят все три бабочки.

\end{problem}

\begin{problem}

Выбирается случайно и равномерно $n$ точек $P_{1} ,\ldots ,P_{n} $ на единичной окружности. Какова вероятность того, что начало координат (цент окружности) окажется внутри выпуклой оболочки этих точек.

\end{problem}

\begin{ordre} 
Выберите $n$ случайных пар диаметрально противоположных точек $Q_{1} ,Q_{n+1} =-Q_{1} $, $Q_{2} ,Q_{n+2} =-Q_{2} $, \dots , $Q_{n} ,Q_{2n} =-Q_{n} $ в соответствии с равномерным распределением. Ясно, что с вероятность 1 все пары различны. В качестве точки $P_{i} $ равновероятно выбирается либо точка $Q_{i} $, либо диаметрально противоположная ей $Q_{n+i} =-Q_{i} $. Покажите, что такая процедура эквивалентна случайному выбору точек $P_{i} $. Покажите, что вероятность того, что начало координат не окажется внутри выпуклой оболочки точек $P_{1} ,\ldots ,P_{n} $, при заданных различных точках $Q_{1} ,\ldots ,Q_{n} ,Q_{n+1} ,\ldots ,Q_{2n} $ равна $\frac{2n}{2^{n} } $, так как нужные точки $P_{1} ,\ldots ,P_{n} $ могут давать только подмножества вида $\left\{\tilde{Q}_{i} ,\ldots ,\tilde{Q}_{i+n-1} \right\}$(суммирование в индексах берется по модулю $2n$), где $\tilde{Q}_{1} ,\ldots ,\tilde{Q}_{2n} $ перенумерованные, например по часовой стрелке, точки $Q_{1} ,\ldots ,Q_{2n} $.
\end{ordre}

\begin{problem}
Найти среднюю длину секущих трехмерного куба с единичной длиной.
\end{problem}

\begin{problem}[Парадокс Бертрана]
Рассмотрим окружность, описанную около равностороннего треугольника ABC. Какова вероятность того, что случайным образом проведенная хорда будет иметь длину большую, чем длина стороны треугольника ABC? 
\end{problem}

\begin{ordre}
Предложите разные способы генерации хорды (не менее трех). Зависит ли ответ на задачу от способа генерации?
\end{ordre}

\begin{problem}
Пусть в пространстве $\mathbb R^n$ с евклидовой нормой задан $n$-мерный шар единичного радиуса. Внутри него имеются две случайные точки с радиус-векторами ${\bf{r}}_1$ и ${\bf{r}}_2$ соответственно, имеющие равномерное пространственное распределение внутри шара. Найти распределение случайной величины, являющейся расстоянием между этими двумя точками $r = \left|{\bf r}_1 - {\bf r}_2\right|$.
\end{problem}




\begin{problem}
На плоскости проведены параллельные прямые на единичном расстоянии друг от друга, и на плоскость наугад бросается иголка длиной $L<1$. 
Угол между прямыми и иголкой и расстояние от середины иглы до ближайшей прямой являются независимыми с.в., равномерно распределенными 
на $(0,2\pi)$ и $(-1/2,1/2)$ соответственно. С помощью серии таких опытов вычислить число $\pi$ с заданной точностью 
$\delta=1\%$ и с вероятностью ошибки не больше $\varepsilon=5\%$. Решите аналогичную задачу для случая погнутой иглы длиной менее 1.
\end{problem}

\begin{ordre}

Рассмотрим окружность диаметра $1$, т.е. длины $\pi$. Такая окружность с вероятностью $1$ пересекает дважды одну из прямых. 
Тогда, исходя из линейности математического ожидания числа попаданий иглы на прямую относительно длины иглы, для иглы длиной $L<1$ 
имеем ${\mathbb E}\xi_L = 2L/\pi$. 

\end{ordre}



\begin{problem}
Покажите, что средняя площадь ортогональной проекции куба с ребром единица на случайную плоскость равна $3/2$. 
\end{problem}

\begin{ordre}
Покажите, что  средняя площадь  ортогональной проекции всякого измеримого тела 
линейно зависит от площади его границы. 
Рассмотрим вспомогательное (см. предыдущую задачу) тело, у которого легко вычисляется средняя площадь ортогональной проекции. 
\end{ordre}

\begin{remark}
Обозначим через $S_k$ $k$-мерный объем ортогональной 
проекции рассматриваемой области $V$ в ${\mathbb R}^n$ на случайную $k$-мерную 
плоскость. Имеет место следующая формула для объема $h$-окрестности данной области:
\[
V\left( h \right)=V_0 +V_1 h+V_2 h^2+...+V_n h^n,
\]
где $V_0 $ -- объем области; $V_1 $ -- $(n-1)$-мерный объем границы области, 
пропорциональный среднему значению от числа 1; число $V_k $ пропорционально 
$S_k $ и выражается через средние значения (усредненным по поверхности рассматриваемой области)  от произведений $k$ главных 
кривизн. 

В случае $n = 3$, из главных кривизн $k_1 $ и $k_2 $ в каждой точке 
можно составить \textit{среднюю кривизну} $k_1 +k_2 $ и \textit{гауссову кривизну} $K=k_1 k_2 $. В этом случае объем 
$h$-окрестности получается $V\left( h \right)=V_0 +V_1 h+V_2 h^2+V_3 h^3$, где 
$V_2 $ пропорционален интегралу от средней кривизны по всей поверхности, а $V_3 $ -- от гауссовской:
\[
V_3 =\frac{4}{3}\pi \int\!\!\!\int {KdS} .
\]
Например, для сферы радиуса $R$
\[
V\left( h \right)=\frac{4}{3}\pi \cdot \left( {R+h} \right)^3=\frac{4}{3}\pi 
R^3+h\cdot \left( {4\pi R^2} \right)+h^2\left( {4\pi R} 
\right)+\frac{4}{3}\pi h^3.
\]
Здесь $k_1 + k_2 = \frac{2}{R}$, $k_1 k_2 = \frac{1}{R^2}$,
\[
\int  (k_1 +k_2)   dS = 8\pi R,
\]
Формула Гаусса-Бонне:
 \[\int\!\!\!\int {\left( {k_1 k_2 } \right)dS} =4\pi. \] 


\end{remark}


\begin{problem}
Приведем геометрическую интерпретацию пуассоновского процесса (см. задачу \ref{sec:poisson}). Пуассоновским процессом $\text{П}$ в пространстве $S \subset \mathbb{R}^m$ называется  счетное множество точек, случайно разбросанных по $S$, но подчиняющихся следующему правилу: существует мера $\mu: S \to [0, \infty]$, соответствующая процессу $\text{П}$, такая что для любых непересекающихся 
измеримых множеств $A_1,\ldots,A_n \subset S$ случайные величины 
\[
N(A_i) = \# \{ A_i \cap \text{П}\} \sim  \Po(\mu(A_i)), \quad i = \overline{1,n},
\] 
порожденные случайным попаданием точек в множества $A_1,\ldots,A_n$, независимы и имеют распределение $\Po(\mu)$. 

К примеру стандартный пуассоновский процесс из задачи \ref{sec:poisson} раздела \ref{zb4} определен в пространстве $S = \mathbb{R}_{+}$, имеет меру $\mu \big( (t_1, t_2] \big) = \lambda ( t_2 - t_1)$, $N( (t_1, t_2] ) = K(t_2) - K(t_1) \sim \Po \big(\lambda (t_2 - t_1) \big)$.

Отметим, что определение $\text{П}$ требует, чтобы мера $\mu$ была \textit{неатомической} (значение на любом счетном множестве равно 0), а также представимой в следующем виде
\[
\mu = \sum_{k = 1}^{\infty} \mu_{k}, \quad \mu_k(S) < \infty.
\]
Пусть $X_1,\ldots,X_n$ -- независимые случайные величины, распределенные по $A \subset S$, $\mu(A) < \infty$ в соответствии с вероятностной мерой $p(\cdot) = \mu(\cdot) / \mu(A)$, Обозначим за $N(B)$ количество $X_i \in B$. Покажите, что для непересекающихся $A_1,\ldots,A_k  \subset A$, $A = A_1 \cup\ldots\cup A_k $ выполнено  
\[
\PR_n(N(A_1) = n_1,\ldots,N(A_k) = n_k) = \frac{n!}{n_1!,\ldots,n_k!} p(A_1)^{n_1} \ldots p(A_k)^{n_k}.
\]
Докажите, что если $X_1,\ldots,X_n$ -- точки пуассоновского процесса с мерой $\mu$ внутри множества $A$, то справедливо равенство
\[
\PR(\cdot | N(A) = n) = \PR_n(\cdot).
\]
\begin{remark}
Последнее выражение утверждает, что при фиксированном числе точек $N(A)$ внутри множества $A$, сами точки пуассоновского процесса выглядят как $N(A)$ независимых случайных величин с плотностью распределения $p(\cdot) = \mu(\cdot) / \mu(A)$. 
Таким образом, пуассоновский процесс является неизбежным следствием моделирования системы с большим числом независимых точек в пространстве~$\mathbb{R}^m$.
\end{remark} 


\begin{problem}
\label{poi_proj}
Однородный пуассоновский процесс $\text{П}$ определен в пространстве $S = \mathbb{R}^{2}$ и имеет интенсивность $\lambda$, т.е. $\mu (A)  = \int_{A} \lambda dxdy$. Осуществим переход к полярным координатам $(r, \theta)$ при помощи преобразования 
\[
f(x, y) = \left( 
(x^2 + y^2)^{1/2}, \; \arctan (y/x)
\right).
\]
Покажите, что образы точек $\text{П}$  образуют пуассоновский процесс в полосе
\[
\{(r,\theta): \; r > 0, \; 0 \leq \theta <  2 \pi \},
\]
который имеет интенсивность $\lambda^{*}(r) = \lambda r$.
Покажите, что значения $r$, соответствующие $\text{П}$, образуют пуассоновский процесс в $(0, \infty)$ с интенсивностью $2 \pi \lambda r$. 

\end{problem}

\begin{remark}
Сформулируем правило отображения $\text{П}$ при помощи произвольного правила преобразования координат $f: S \to T$: если мера $\mu$ процесса $\text{П}$ является $\sigma$-конечной ($S$ можно представить в виде $\cup_n S_n$, где $\mu(S_n) < \infty$), помимо того мера на множестве образов $T$ 
\[
\mu^{*}(B) = \mu( f^{-1} (B))
\]  
является неатомической, то $f(\text{П})$ -- пуассоновский процесс в пространстве $T$ c  мерой интенсивности $\mu^{*}$.
\end{remark}

\begin{problem}
Используя результат предыдущей задачи, покажите, что плотность распределения упорядоченных расстояний $r_{(1)}, r_{(2)}, \ldots$ имеет следующий вид
\[
f_{r_{(s)}} (r) = \frac{2 (\lambda \pi)^s r_{}^{2s - 1} e^{-\lambda \pi r_{}^2}}{ (s-1)! }. 
\]  
В таком случае, $2 \lambda \pi r_{(s)}^2$ распределено как $\chi_{2s}^{2}$. Данный результат может быть использован для оценки плотности точек на плоскости путем выбора случайных точек и измерения расстояния до 1-го ближайшего соседа. Пусть $X_1, \ldots, X_n$ -- $n$ реализаций $r_{(1)}^{2}$. Покажите, что $2\lambda \pi n \overline{X}$ распределено как $\chi_{2n}^{2}$, предложите оценку для $\lambda$ и вычислите ее дисперсию.
\end{problem}

\begin{remark}
Аналогичный подход с измерением переменной плотности точек $\lambda(\cdot)$ продемонстрирован в задаче $\ref{sort_dir}$ раздела $\ref{bayes}$  c использованием упорядоченного распределения Дирихле.
\end{remark}

\begin{problem}
Сопоставим каждой точке $X \in S$ случайного множества  $\text{П}$ (пуассоновский процесс) случайную величину $m_X$ (метку), принимающую значения из множества $M$.
Распределение $m_X \sim p(X, m)$ может зависеть от  $X$, но не зависит от других точек $\text{П}$ и их меток. Докажите, что случайное множество
\[
\text{П}^{*} = \{(X, m_X): \; X \in \text{П}\}
\]
является пуассоновским процессом на множестве $S \times M$ c мерой интенсивности
\[
\mu^{*}(C) = \mathop{\int\int} \limits_{(x,m) \in C} \mu(dx) p(x, m) dm.
\]
Используя замечание к задаче \ref{poi_proj}, убедитесь что метки  $m_X$ образуют пуассоновский процесс на $M$ с мерой интенсивности 
\[
\mu_m(B) = \int_{S}\int_{B}  \mu(dx) p(x, m) dm. 
\]
\end{problem}

\begin{ordre}
Докажем, что $\text{П}$ однозначно задается набором характеристических функционалов вида
\[
\Exp(e^{\Sigma_f}) = \exp \left( - \int_{S}(1 - e^{-f(x)}) \mu(dx)  \right),
\]
где 
\[
\Sigma_f = \sum_{X \in \text{П}} f(X), \quad f \in \mathcal{F},
\]
$\mathcal{F}$ -- класс функций, содержащий все индикаторные функции для измеримых множеств из $S$. Выбрав набор непересекающихся множеств $A_1,\ldots,A_k$, сопоставим 
каждому множеству свое значение $f_i$. Тогда
\[
\Sigma_f  = \sum_{i=1}^k f_i N(A_i), \quad
\Exp(e^{\Sigma_f}) = \exp \left( - \sum_{i=1}^k (1 - e^{-f_i}) \mu(A_i)  \right).
\]    
Сделав замену $z_i  = e^{-f_i}$, получаем
\[
\Exp \left(z_1^{N(A_i)} \ldots z_k^{N(A_k)} \right) = \prod_{i=1}^k e^{\mu (A_i) (z-1)},
\]  
что свидетельствует о независимости $N(A_i)$ и принадлежности распределению Пуассона.
\end{ordre}

\end{problem}

\begin{problem}
\label{subord}
Случайный процесс $\varphi(t)$ называется \textit{субординатором} (процесс Леви с положительными приращениями), если он удовлетворяет следующим свойствам: приращения  процесса независимы, положительны и зависят только от приращения аргумента $t$. Докажите, что  справедливо следующее представление субординатора 
\[
\varphi(t) = \beta t + \sum_{\tau} \{z: \; (\tau, z) \in  \text{П}, \; 0 < \tau < t  \},
\]
где суммируются значения второй координаты пуассоновского процесса $\text{П}$ с мерой интенсивности
\[
\mu(d\tau, dz) = d\tau \gamma(dz). 
\]
Мера $\gamma(dz)$ соответствует скачкам процесса  $\varphi(t)$, образующем пуассоновский процесс по координате $z$, и может быть найдена из выражения (представление Леви-Хинчина)
\[
\Exp  e^{ -s \varphi(t)} = \exp \left(
-t \left[ s\beta - \int_{0}^{\infty} (1 - e^{-s z}) \gamma(dz)  \right]
 \right). 
\] 
\end{problem}

\begin{ordre}
Используя результат предыдущей задачи, покажите, что для чисто атомической меры на $S$, определенной как  
\[
\Psi(A)  = \sum_{\tau} \{z: \; (\tau, z) \in  \text{П}, \;  \tau \in A\} 
\]  
характеристическая функция имеет вид
\[
\Exp  e^{ -s \Psi(A) } = \exp \left(
- \int_{0}^{\infty} (1 - e^{-s z}) \gamma(A, dz) 
 \right),
\]
где $\gamma(\cdot, \cdot)$ --  мера процесса $\text{П}$.

Покажите, что для случайной неатомической меры $\Phi$ с независимыми значениями на непересекающихся множествах $\Phi(A)$ есть безгранично делимая с.в. и для нее  справедливо представление  Леви-Хинчина: 
\[
\Exp  e^{ -s \Phi(A)} = \exp \left(
 - s\beta(A) + \int_{0}^{\infty} (1 - e^{-s z}) \gamma(A, dz) 
 \right). 
\]  
Далее, $\varphi(t)$ может быть представлен как случайная мера $\Phi(0,t]$ при значениях $t \geq 0$. Убедитесь, что такая мера является неатомической, используя свойство инвариантности $\Phi$ относительно сдвига. 
\end{ordre}

\begin{problem}
Рассмотрим звезды, находящиеся на расстоянии, не превышающем $R$ от наблюдателя. Для простоты будем считать, что все звезды имеют одинаковый диаметр $\delta$ и равномерное пространственное распределение с количеством звезд $\lambda$ на единицу объема. Показать, что при $R \rightarrow \infty$ любой участок неба будет полностью светящимся.   

\begin{remark}
В действительности такое явление не наблюдается. В связи с конечным возрастом вселенной ($14 \cdot 10^9$ лет) ее радиус ограничен величиной $ct$.
\end{remark}
\end{problem}

\begin{problem}[Формула Крофтона]
Пусть  $N$ точек независимо распределены в области $D$ $n$-мерного пространства, $P$ -- вероятность того, что фигура $F$, образованная $N$ точками, обладает определенным свойством,
зависящим только от взаимного расположения точек. Область $D$ является измеримой по Лебегу и ее мара равна $V$. Обозначим как $P_1$ вероятность того, что $F$ обладает требуемым свойством для случайных точек в области $D_1 \supset D$. Докажите следующее соотношение для малых приращений $\delta V$:
 \[
 \delta P = N (P_1 - P) V^{-1} \delta V .
 \]  

\end{problem}

\begin{problem}
Вычислите распределение  расстояния между точками, случайно взятыми внутри круга радиуса $R$, воспользовавшись формулой Крофтона.
\end{problem}
\begin{ordre}
Пусть $f(x,R)dx$ есть вероятность того, что расстояние между точками $A$ и $B$ принадлежит интервалу  $(x, x + dx)$. Для случая, когда $A$ лежит на границе
\[
f(x,R) = \frac{ 2 \theta x}{\pi R^2}, \quad \theta = \arccos (x/2R).
\]
Тогда уравнение Крофтона примет вид
\[
\frac{df}{d\theta} + 4 f \tg \theta = \frac{32 \theta}{\pi x} \sin \theta \cos \theta.
\]
\end{ordre}



\begin{problem}[Теорема Дворецкого]
\label{sec:dvor}
Доказать, что для любого $\epsilon > 0$ и $k \in \mathbb{N}$
существует $N = N(k, \epsilon) < \exp(C\frac{\log \epsilon}{\epsilon^2}k)$ такое, что любое конечномерное банахово пространство ($X, \Vert\cdot\Vert)$, где $\dim X > N$, содержит $k$-мерное подпространство $E$, являющееся $\epsilon$-евклидовым, т.е. в нем можно задать такую норму $\vert \cdot \vert$, что $\Vert x \Vert \leqslant \vert x \vert \leqslant (1 + \epsilon) \Vert x \Vert$ $\forall x \in E$.     

\begin{remark} См. работу В. Д. Мильман, “Новое доказательство теоремы А. Дворецкого о сечениях выпуклых тел”, Функц. анализ и его прил., 5:4 (1971), 28–37. 

\end{remark}

\begin{comment}
НАЗАР, СЮДА СТОИТ ВСТАВИТЬ ЗАМЕЧАНИЕ ИЗ СТАТЬИ GOWERS'а
И ДОБАВИТЬ В ВИДЕ ЗАДАЧИ ТЕОРЕМУ Б.С, Кашина - см. статью того же Gowers'a и ссылки в том моем письме
\end{comment}


\end{problem}

\begin{problem}
Выберем наугад (равновероятно) $k$ вершин $m$-мерного куба $[0,1]^m$. Обозначим как $X$ выпуклую оболочку выбранных вершин. Пусть $p_{km}$ - вероятность того, что все вершины многогранника попарно смежны. Докажите справедливость следующей оценки при $m > 3$:

\[
p_{km} > 1 - \frac{k^4 \cdot 5^m}{4 \cdot 8^m}
\]
    
\end{problem}
\begin{remark}
См. монографию Бондаренко В.А., Максименко А.Н. Геометрические конструкции, сложность в комбинаторной оптимизации. -- М.: УРСС, 2008.
\end{remark}


\begin{problem}
На плоскости нарисована выпуклая фигура, ограниченная кривой длины $L$. Докажите, что ее диаметр, т.е. максимальное расстояние между двумя ее точками, не меньше $\frac{L}{\pi }$.
\end{problem}
\begin{ordre}
Проведите в случайном направлении прямую. Покажите, что 
математическое ожидание длины проекции фигуры на случайное направление равно 
$\frac{L}{\pi }$.
\end{ordre}

\begin{problem}
В московском метро можно провозить коробки, у которых сумма измерений (длины, ширины и высоты) не превосходит некоторой границы. Можно ли перехитрить правила, поместив одну коробку в другую (сумма измерений внутренней коробки больше суммы измерений внешней)?
\end{problem}
\begin{ordre}
Спроектируйте коробку на случайно выбранное (в пространстве) 
направление. Длина проекции коробки складывается из проекций отрезков, 
идущих по ее высоте, длине и ширине. Проекция внутренней коробки не 
превосходит проекции внешней.
\end{ordre}

\begin{problem}
Несамопересекающаяся кривая длины 22 находится внутри круга радиуса 1. Докажите, что найдется прямая, имеющая с этой кривой по крайней мере 8 общих точек.
\end{problem}
\begin{problem}
Известно, что более половины поверхности Земли занимают океаны. Докажите, что можно найти две диаметрально противоположные точки, обе попавшие в океан.
\end{problem}


\begin{problem}
На плоскости расположено $2n$ векторов, выходящих из начала координат и длиной не более 1. Доказать, что существует угол $\alpha$ такой, что при повороте каждого из векторов на угол $\pm \alpha$, их векторная сумма окажется не большей 1.  
\end{problem}

\begin{problem}[случайный поворот куба, мера Хаара] 10{\%} поверхности 
шара (по площади) выкрашено в черный цвет, остальные 90{\%} -- белые. 
Докажите, что можно вписать в шар куб таким образом, чтобы все вершины куба 
попали в белые точки.
\end{problem}

\begin{problem}[Теорема Уилкса]
Для выборки $X \sim f(\theta, X)$, $\dim X = n$ осуществляется проверка гипотезы $H_0: \theta = \theta_0$, для чего используется обобщенный критерий отношения правдоподобия. 
Определим статистику для критерия как 
\[
W(\theta_0) = \log f(\widehat{\theta}, X) - \log f(\theta_0, X),
\quad  \widehat{\theta} = \arg\max_{\theta \in \Theta} f(\theta, X).
\] 
Теорема Уилкса утверждает, что в случае истинности гипотезы  $H_0$ и при асимптотической нормальности оценки $\widehat{\theta}$, статистика  
$2W(\theta_0)$ сходится по распределению к $\chi_{p}^2$, где $p = |\Theta|$. 
Тогда критерием отклонения гипотезы $H_0$ будет $[2W(\theta_0) > x]$, $\PR(\chi_{p}^2 > x) < \alpha$. 

Существует также обобщение теоремы Уилкса для случая, когда $\widehat{\theta}$ не является асимптотически нормальной. При довольно слабых дополнительных ограничениях для выполнения теоремы Уилкса достаточно, чтобы поверхности уровня  $S_w = \{\theta: W(\theta) = w\}$ в пределе при $n \to \infty$ имели форму
\[
S_w  \approx \widehat{\theta} + a_n w^r S
\]  
где $a_n \to 0 $ при $n \to \infty$, $S = \{\theta: h(\theta) = 1 \}$, $h(t\theta) = t^{1/r} h(\theta)$. В частности, при асимптотической нормальности оценки $\widehat{\theta}$ поверхности уровня являются эллиптическими:
\[
S_w  \approx \widehat{\theta} +  \sqrt{\frac{2w}{n}} \theta^T D \theta.
\]
Докажите при дополнительных ограничениях, $W(\widehat{\theta})$ строго локально вогнута и $\inf\{h(\theta): \Vert \theta \Vert = 1\} > 0$, что 
\[
W(\theta_0) \overset{d}{\longrightarrow} \text{Г}(rp, 1),
\]
где плотность гамма $\text{Г}(\alpha, \lambda)$ распределения задается как $\frac{x^{\alpha-1} e^{-x/\lambda}}{\lambda^\alpha\text{Г}(\alpha)}$.

\end{problem}

\begin{ordre}
См. работу Fan et al. Geometric Understanding of Likelihood Ratio Statistics. Department of Statistics, UCLA 1998.
\end{ordre}
