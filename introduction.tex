\thispagestyle{empty} 
\begin{center}

{ОБЯЗАТЕЛЬНАЯ ЧАСТЬ ПРОГРАММЫ УЧЕБНОГО КУРСА  \\ 
«Теория вероятностей»  }
 
\end{center}
 
 {\small
 
Интуитивные предпосылки теории вероятностей. Множество элементарных исходов опыта, событие. Классическое и статистическое определение вероятности. Математическое определение вероятности. Алгебра и сигма-алгебра событий, минимальная сигма алгебра. Аксиомы теории вероятностей и следствия из них. Вероятностное пространство.

Теорема непрерывности вероятности. Теорема сложения вероятностей. Зависимые и независимые события. Условная вероятность события. Формула полной вероятности. Формула Байеса. Леммы Бореля--Кантелли. Закон ``0-1'' Колмогорова.

Случайная величина как измеримая функция. Функция распределения случайной величины. Дискретные и непрерывные случайные величины. Плотность распределения вероятностей. Формула включений-исключений.

Конкретные распределения случайных величин. Схема Бернулли, геометрическое и биномиальное распределение. Простейший поток событий и распределение Пуассона. Показательное, равномерное, нормальное, log-нормальное и отрицательно-биномиальное распределения. Бета-распределение и гамма-распределение.

Случайный вектор. Функция распределения случайного вектора. Зависимые и независимые случайные величины, условные законы распределения. Функции случайных величин. Невырожденное функциональное преобразование случайного вектора.
Интеграл Стилтьеса. Математическое ожидание и дисперсия случайной величины. Моменты случайной величины. Условное математическое ожидание. Корреляционная матрица случайного вектора. Коэффициент корреляции двух случайных величин.

Характеристическая функция и ее свойства. Связь моментов случайной величины с ее характеристической функцией. Разложение характеристической функции в ряд.
Сходимость последовательностей случайных величин с вероятностью единица (почти наверное), порядка $p$ (в среднем квадратичном), по вероятности, по распределению. Соотношение между различными типами сходимости. 

Неравенство Чебышева. Закон больших чисел. Критерий Колмогорова. Теоремы Хинчина и Чебышева.  Усиленный закон больших чисел. Теорема Колмогорова и Бореля. Оценивание скорости сходимости частоты к вероятности в схеме Бернулли. Неравенство Бернштейна. 

Интегральная и локальная теоремы Myавра--Лапласа. Дискретная поправка. Теорема Линдберга. Центральная предельная теорема для одинаково распределенных случайных величин. Центральная предельная теорема в форме Натана. Условие Ляпунова. Теорема Гливенко.
}


\newpage

\section{Введение}

В основу предлагаемого сборника задач по теории вероятностей положены задачи (в том числе повышенной сложности), предлагавшиеся в разные годы студентам факультета управления и прикладной математики (ФУПМ) МФТИ и Независимого московского университета на семинарах, сдачах заданий и экзаменах. Главными отличительными особенностями пособия являются: а) широкий спектр представленного материала, б) отражение ряда современных направлений развития теории вероятностей и в) нацеленность на приложения.

По замыслу авторов, предлагаемый сборник задач отчасти демонстрирует роль курса как “вероятностного фундамента” для ряда других дисциплин: прикладной статистики, стохастических дифференциальных уравнений, эффективных алгоритмов, экспериментальной экономики (финансовой математики) и др. Несмотря на широкий спектр представленных тем, основной  акцент делается на формирование у читателей геометрической интуиции, восходящей к Пуанкаре, которая позволяет с единых позиций понять многообразие асимптотических результатов стохастической теории как проявление одного общего принципа концентрации меры. 

Большое внимание в сборнике уделяется различным приемам доказательства предельных теорем – асимптотических результатов теории вероятностей. Для этого прежде всего используется аппарат производящих функций и теории функций комплексного переменного.

Более половины представленных в сборнике задач не являются стандартными. Для таких задач даны указания, комментарии (замечания), ссылки на публикации. 

Желающих более глубоко изучить представленные темы отсылаем к списку литературы, приведенному в конце сборника задач. 

Следует отметить, что за последние десять лет заметно возросло значение для выпускников Физтеха глубоких знаний вероятностных дисциплин. Это обусловлено множеством причин.

Прежде всего, это вызвано широким распространением задач анализа больших массивов данных (machine learning, data mining). Подтверждением служит взаимная востребованность студентов ФУПМ и Школы анализа данных компании Яндекс, большая популярность среди  студентов ФУПМ кафедры интеллектуальных систем (заведующий кафедрой член-корреспондент РАН К.В. Рудаков, Вычислительный центр РАН) и кафедры предсказательного моделирования (заведующий кафедрой академик РАН А.П. Кулешов, Институт проблем передачи информации РАН). 
%Во многом именно для таких студентов и написаны разделы \ref{bayes}, \ref{zb4}, \ref{MK}, \ref{stats}, ССЫЛКА[Оптимизация и стохастика]. %3 – 6, 9, 10.

Другая не менее важная причина -- разработка эффективных (приближенных, рандомизированных) алгоритмов решения сложных задач. 
%Этому посвящены разделы \ref{MK}, \ref{information}, \ref{CS}, ССЫЛКА[Оптимизация и стохастика].
Сложно, например, представить себе современного специалиста по моделированию, который бы не использовал методы Монте-Карло. Также сложно представить себе специалиста в области computer science, которому не приходилось бы применять рандомизированные алгоритмы и подвергать алгоритмы вероятностному анализу (например, для оценки сложности в среднем). 

Еще одна причина  связана с тем, что приложение вероятностных методов к анализу и разработке экономических моделей для части студентов ФУПМ является фундаментом в работе на базовых кафедрах. Например, заведующие базовыми кафедрами ФУПМ член-коррерспондент РАН И.Г. Поспелов (Вычислительный центр РАН) и  член-коррерспондент РАН Ю.С. Попков (Институт системного анализа РАН) активно работают в направлении разработки вероятностных моделей экономических агентов и вероятностного анализа агломерационных моделей. %Этому посвящен раздел ССЫЛКА[Марковские модели макросистем].

Наконец, можно заметить, что задачи анализа больших компьютерных, социальных, транспортных сетей в последнее время выходят на передний план во многих приложениях. Огромную роль в изучении таких сетей играют вероятностные модели, некоторые из которых будут приведены в предлагаемом сборнике задач. 
%Разделы \ref{hard}, \ref{genF}, \ref{combinatorics}, ССЫЛКА[Марковские модели макросистем] сборника посвящены изучению подобных сетей, в частности, различным предельным переходам.

Важную роль в подготовке настоящего сборника задач сыграла Лаборатория структурных методов анализа данных в предсказательном моделировании (ПреМоЛаб), открытая на базе ФУПМ МФТИ в 2011 году. В частности, благодаря этой лаборатории, у студентов есть возможность посмотреть на сайте www.mathnet.ru, www.premolab.ru (семинары: Стохастический анализ в задачах, Математический кружок и др.) видеозаписи выступлений ведущих ученых, посвященные ряду нестандартных задач из этого сборника. 

Мы благодарны нашим коллегам Д.В. Беломестному, Я.И.~Белопольской, М.Л.~Бланку, Н.Д.~Введенской, В.В.~Веденяпину, Д.П.~Ветрову, А.М.~Вершику, К.В.~Воронцову, В.В.~Вьюгину, В.В.~Высоцкому, А.В.~Калинкину, М.Н.~Вялому, М.С.~Гельфанду, Э.Х.~Гимади, Г.К.~Голубеву, А.Б.~Дайняку, П.Е.~Двуреченскому, Н.Х.~Ибрагимову, М.И.~Исаеву, Г.А.~Кабатянскому, А.В.~Колесникову, А.В.~Леонизову, Г.~Лугоши, Ю.В.~Максимову, В.А.~Малышеву, В.Д.~Мильману, В.В.~Моттлю, Т.А.~Нагапетяну, А.В.~Назину, Ю.Е.~Нестерову, А.С.~Немировскому, В.И.~Опойцеву, Ф.В.~Петрову, С.А.~Пирогову, Б.Т.~Поляку, И.Г.~Поспелову, А.М.~Райгородскому, В.Н.~Разжевайкину, М.А.~Раскину, В.Г.~Редько, A.Е.~Ромащенко, А.В. Савватееву, А.Н.~Соболевскому, А.~Содину, В.Г.~Спокойному, Й.~Стоянову, У.~Сэндхольму, С.П.~Тарасову, И.О.~Толстихину, М.Ю.~Хачаю, О.С.~Федько, Ю.А.~Флёрову, А.X.~Шеню, оказавшим заметное влияние на формирование различных разделов этого учебного пособия, и А.А. Шананину, во многом способствовавшему развитию на ФУПМ базового цикла вероятностных дисциплин. Также отметим большую помощь старшекурсников, аспирантов ФУПМ и участников нашего стохастичексого семинара в НМУ и Физтехе в вычитке этого сборника задач (в особенности, Д.~Бабичева, А.~Балицкого, Ф.~Гончарова, Ю.~Дорна, Е.~Клочкова, А.~Макарова, Е.~Молчанова, Н.~Животовского, М.~Панова, Л.~Прохоренкову(Остроумову), А.~Суворикову, Д.~Петрашко, М.~Широбокова).