\section{Неравенства концентрации меры \\ и вероятности больших уклонений}
\label{measure}


\begin{problem}[Концентрация площади сферы и объема шара]
\
\begin{enumerate}
\item Рассмотрим шар $B^{n}(r)$ радиуса $r$ в евклидовом пространстве $\mathbb{R}^n$ большой размерности, пусть в шаре задана равномерная мера. 
%Пусть $V\big[B^{n}(r)\bigl]$ ~--- объем шара.
 Необходимо убедиться в том, что мера сконцентрирована в малой окрестности  границы шара.
\item
Рассмотрим сферу $S^{n-1}(r)$ в евклидовом пространстве $\mathbb{R}^r$ с радиусом в начале координат. Необходимо убедиться в том, что выбранные наугад два единичных вектора в пространстве $\mathbb{R}^n$ большой размерности с большой вероятностью окажутся почти ортогональными, если на сфере задано равномерное распределение.  
%Зафиксируем координатную ось $x$.
%Необходимо убедиться в том, что подавляющая часть площади многомерной сферы $S^{n-1}$ сосредоточена в малой окрестности экватора, перпендикулярного выбранной оси $x$. Каково взаимное расположение двух выбранных наугад единичных векторов в простанстве $\mathbb{R}^n$, если концы векторов распределены на сфере равномерно?
\end{enumerate}
\end{problem}


\begin{remark}
Во втором пункте достаточно доказать, что для всякого сколь угодно малого $\delta>0$ проекция второго вектора на ось $x_1$ с вероятностью, близкой к 
единице, лежит в промежутке $[-\delta, \delta]$ при $n\to\infty$. Это равносильно тому, что доля от площади всей сферы $S^{n-1}(r)$, 
которую занимает сферический слой $S^{n-1}_{\delta}(r)$, проектирующийся в отрезок $[-\delta, \delta]$ оси $x_1$, 
может быть сделана сколь угодно близкой к $1$ при $n\to\infty$. 

Перейдя к $n$-мерным сферическим координатам и обратно, покажите, что мера сферического слоя $S^{n-1}_{\delta}(r)$ равна 
$$
\mu_{n-1} S_{\delta}^{n-1}(r) = Cr^{n-1} 
\int\limits_{-\delta}^{\delta} \Bigl( 1-(x/r)^2\Bigr)^{(n-3)/2} \, dx ,
$$
тогда вероятность попадания в данный слой $S_{\delta}^{n-1}(r)$ равна 
$$
{\mathbb P}[-\delta, \delta]=\frac{\int\limits_{-\delta}^{\delta} \Bigl( 1-(x/r)^2\Bigr)^{(n-3)/2} \, dx}
{\int\limits_{-r}^{r} \Bigl( 1-(x/r)^2\Bigr)^{(n-3)/2} \, dx} . 
$$
Данное отношение на зависит от $r$, поэтому можно считать $r=1$. 

Для нахождения асимптотики имеющихся интегралов ($n\to\infty$) используйте классические результаты относительно асимптотики интеграла 
Лапласа $F(\lambda)=\int_a^b f(x)e^{\lambda S(x)}\, dx$ при $\lambda\to +\infty$ (см. указание к задаче \ref{laplas} из раздела  \ref{genF}). %Если обе функции $f$ и $S$ определены и регулярны 
%на промежутке $I=[a,b]$ и функция $S$ имеет единственный глобальный максимум на $I$, который достигается в точке $x_0\in I$, 
%$f(x_0)\ne 0$, то асимптотика интеграла такая же, как в окрестности точки $x_0$ (принцип локализации). В зависимости от расположения 
%точки $x_0$ и свойств функции $S(x)$ возможны следующие тейлоровские разложения при $\lambda\to +\infty$: 
%$$
%F(\lambda)=\frac{f(x_0)}{-S'(x_0)}e^{\lambda S(x_0)} \lambda^{-1}\bigl( 1+O(\lambda^{-1})\bigr) , 
%$$
%если $x_0=a$ и $S'(x_0)\ne 0$ (т.е. $S'(x_0)<0$); 
%$$
%F(\lambda)=\sqrt{\frac{\pi}{-2S''(x_0)}} f(x_0) e^{\lambda S(x_0)} \lambda^{-1/2}\bigl( 1+O(\lambda^{-1/2})\bigr) , 
%$$
%если $x_0=a$, $S'(x_0)=0$, $S''(x_0)\ne 0$ (т.е. $S''(x_0)<0$); 
%$$
%F(\lambda)=\sqrt{\frac{2\pi}{-S''(x_0)}} f(x_0) e^{\lambda S(x_0)} \lambda^{-1/2}\bigl( 1+O(\lambda^{-1/2})\bigr) , 
%$$
%если $a<x_0<b$, $S'(x_0)=0$, $S''(x_0)\ne 0$ (т.е. $S''(x_0)<0$). 

Для решения этой и следующей задачи рекомендуется ознакомиться с книгой Зорич В.А. Математический анализ задач естествознания,-- М.: МЦНМО, 2008.
\end{remark}



\begin{problem}[Изопериметрическое неравенство и принцип концентрации меры; П. Леви, 1919]
\label{levi}
Число $\mu _f $ называют медианой функции $f$, если
$\mu \left( {\vec {x}\in S_1^n :\;\;f\left( {\vec {x}} \right)\ge \mu _f } 
\right)\ge 1 \mathord{\left/ {\vphantom {1 2}} \right. 
\kern-\nulldelimiterspace} 2$ и $\mu \left( {\vec {x}\in S_1^n :\;\;f\left( 
{\vec {x}} \right)\le \mu _f } \right)\ge 1 \mathord{\left/ {\vphantom {1 
2}} \right. \kern-\nulldelimiterspace} 2$,
где $\mu \left( {d\vec {x}} \right)$ -- равномерное мера на единичной сфере 
$S_1^n $ в ${\mathbb R}^n$. Пусть $A$ -- измеримое (борелевское) множество на 
сфере $S_1^n $. Через $A_\delta $ -- будем обозначать $\delta $-окрестность (расстояние определяется по геодезический)
множества $A$ на сфере $S_1^n $. Предположим теперь, что в некотором 
царстве, расположенном на $S_1^n $, царь предложил царице Дидоне построить 
огород с заданной длиной забора. Царица хочет, чтобы её огород при заданном 
периметре имел наибольшую площадь. Таким образом, царице надо решить 
изопериметрическую задачу (такие задачи обычно рассматриваются в курсах 
вариационного исчисления). Решение этой задачи хорошо известно на плоскости -- ``круглый 
огород'', это можно обобщить на наш случай. Для нас же полезно, рассмотрение двойственной задачи, имеющей 
такое же решение: при заданной площади огорода спроектировать его так, чтобы 
он имел наименьшую длину забора, его ограждающего. Используя решение этой 
задачи, покажите, что если $\mu \left( A \right)\ge 1 \mathord{\left/ 
{\vphantom {1 2}} \right. \kern-\nulldelimiterspace} 2$, то
$$
\mu \left( {A_\delta } \right)\ge 1-\sqrt {\frac{\pi}{8}} \exp \left( \frac{-\delta^2n}{2} \right).
$$
Пусть теперь на $S_1^n $ задана функция с модулем непрерывности
\[
\omega _f \left( \delta \right)=\sup \left\{ {\left| {f\left( {\vec {x}} 
\right)-f\left( {\vec {y}} \right)} \right|:\;\;\rho \left( {\vec {x},\vec 
{y}} \right)\le \delta ,\;\vec {x},\vec {y}\in S_1^n } \right\}.
\]
Покажите, что тогда
\[
\mu \left( {\vec {x}\in S_1^n :\;\;\left| {f\left( {\vec {x}} \right)-\mu _f 
} \right|\ge \omega _f \left( \delta \right)} \right)\le \sqrt {\pi 
\mathord{\left/ {\vphantom {\pi 2}} \right. \kern-\nulldelimiterspace} 2} 
\exp \left( {-{\delta ^2n} \mathord{\left/ {\vphantom {{\delta ^2n} 2}} 
\right. \kern-\nulldelimiterspace} 2} \right).
\]
Можно показать, что при весьма естественных условиях медиана асимптотически 
близка к среднему значению (математическому ожиданию). Аналогичное 
неравенство можно получить (М. Талагран, 1994), например, для модели 
случайных графов (Эрдёша - Реньи) исследовать плотную концентрацию около 
среднего значения различные функций на случайных графах: число 
независимости, хроматическое число и т.п.
\end{problem}


\begin{remark}
Изопериметрические неравенства на сфере были обобщены в начале 80-х М.Л. Громовым на римановы многообразия (см. подробнее в \cite{14}).

Пусть $(X,g)$ ~--- компактное связное гладкое риманово многообразие размерности $n>2$ со строго положительной кривизной Риччи и римановой метрикой $g$, наделенное элементом объема $d\mu=\frac{dv}{V}$, где $V$~--- полный объем $X$. 
Кривизна Риччи~--- способ описания изменения многообразия по отношению к евлидовой мере, то есть степени отличия мнообразия  от евклидова пространства. 
Обозначим за $c(X)$ точную нижнюю грань тензора кривизны Риччи по всем единичным касательным векторам. Пусть $c(X)>0$. Тогда 
$$
\mathcal{P}_{\mu}\geq \mathcal{P}_{\sigma^n_{\mathbb{R}}},
$$
где $\sigma^n_{R}$ -- равномерная инвариантная мера, $\mathcal{P}_{\mu}$ -- изопериметрическая функция, т.е. наибольшая функция на $[0,\mu(X)]$, такая что
$$
\mu^{+}(A) \geq \mathcal{P}_{\mu}(\mu(A)) \quad \text{при~}\mu^{+}(A) = \lim_{t\to\infty}\inf\frac{1}{t}\mu(A_t/A),
$$ 
где $A_t = \{x\in X; g(x,A)<t\}$.

Здесь $R>0$ таково, что 
$$
c({S}^n_{R}) = \frac{n-1}{R^2}=c(X),
$$ 
где ${S}^n_{R}$~--- $n$-сфера с радиусом $R$, снабженная нормированной равномерной инвариантной мерой $\sigma^n_{R}$. 
В частности для $(X,g,\mu)$ верно 
\begin{equation*}
\alpha(X;r)\leq \text{e}^{-cr^2/2},\quad r>0, 
\end{equation*}
где концентрационная функция $\alpha(X;r)$ определяется следующим образом
\begin{equation*}
\alpha(X;r) = 1-\inf\biggl\{\mu(A_{r})|\, A \subset X, \mu(A)\geq \frac{1}{2}\biggr\}.
\end{equation*}
\end{remark}

\begin{problem}(Концентрация на дискретном кубе) Пусть $E = \{-1, 1\}^{n}$ --- дискретный $n$-мерный куб, на множестве вершин которого задана равномерная вероятностная мера $\mu$. Введем на вершинах куба стандартную Хэммингову метрику. Тогда для всякой $1$-липшицевой (по введенной метрике) действительнозначной функции $f$, заданной на $E$, c медианой $M$ для $\varepsilon \ge 0$ имеет место неравенство
\[
\PR\{\left|f(X) - M\right| \ge \varepsilon\} \le 2\exp\left(\frac{-\varepsilon^{2}}{2n}\right).
\]
а) Пусть $\Exp f(X)$ -- математическое ожидание функции. Докажите, что
\[
\left|\mathbb{E} f(X) - M\right| \le \sqrt{2\pi n}.
\]

б) Покажите, что фактор $\frac{1}{n}$ под экспонентой не может быть 'улучшен'. В частности, неравенство концентрации явно зависит от размерности куба (при условии сохранения гауссовского хвоста по $\varepsilon$).
\end{problem}

\begin{remark}
Используйте тот факт, что для случайной величины $Y > 0$ имеет место равенство $\Exp Y = \int\limits_{0}^{\infty}\PR\{Y \ge x\}dx$.

См.\cite{22} и Barvinok A. Math 710: Measure Concentration. Lecture notes, 2005. http://www.math.lsa.umich.edu/~barvinok/total710.pdf.
\end{remark}

\begin{problem}(Неравенство Талаграна для дискретного куба) Пусть $E = \{-1, 1\}^{n}$ --- дискретный $n$-мерный куб, на множестве вершин которого задана равномерная вероятностная мера $\mu$. Введем на вершинах куба стандартную евклидову метрику. Тогда для всякой выпуклой $1$-липшицевой действительнозначной функции $f$, заданной на $E$, c медианой $M$ для $\varepsilon \ge 0$ имеет место неравенство
\[
\Prob\{\left|f(X) - M\right| \ge \varepsilon\} \le 2\exp\left(\frac{-\varepsilon^{2}}{2n}\right).
\]

а) Покажите, что условие выпуклости функции $f$ нельзя опустить в данном неравенстве.


б) Получите неравенство, аналогичное неравенству Талаграна для концентрации около математического ожидания.
\end{problem}
\begin{remark}
См. Talagrand~M. An Isoperimetric Theorem on the Cube and the Kinitchine-Kahane Inequalities,
Proceedings of the American Mathematical Society, 1988. Pp.\,905-909.

В отличие от предыдущей теоремы в данном неравенстве нет зависимости от размерности куба. Одновременно накладываются два дополнительных требования, во-первых, липшицевость по меньшей -- евклидовой метрике, во-вторых, дополнительно на функцию накладывается условие выпуклости.

Существуют аналогичные неравенства концентрации вокруг математического ожидания с гораздо более точными константами. Тем не менее, даже этот простой способ позволяет получить нужный нам гауссовский хвост, не зависящий от размерности.
\end{remark}

\begin{problem} (Концентрация на сечениях куба) Пусть $E = \{-1, 1\}^{n}$ --- дискретный $n$-мерный куб. Рассмотрим его сечение, состоящее из всех его вершин, содержащих ровно $k$ координат, равных $1$. Введем на вершинах сечения нормированную метрику $d$, равную половине Хэмминговой метрики. При этом расстояние между вершинами, координаты которых отличаются лишь перестановкой пары координат равно единице. На множестве вершин сечения зададим равномерную вероятностную меру $\mu$. Введем также понятие длины дискретного градиента $|\nabla f(X)|$. Обозначим
\[
|\nabla f(X)|^{2} = \sum\limits_{Y : d(X, Y) = 1}|f(X) - f(Y)|^{2},
\]
где суммирование ведется по всем $k(n - k)$ вершинам сечения, удаленным от данной вершины на единицу.

Тогда для всякой действительнозначной функции $f$, заданной на описанном сечении, длина дискретного градиента которой в точках сечения ограничена числом $\sigma$, для $\varepsilon \ge 0$ имеет место неравенство
\[
\PR\{f(X) - \Exp f(X) \ge \varepsilon\} \le \exp\left(\frac{-(n + 2)\varepsilon^{2}}{4\sigma^{2}}\right).
\]
Пусть случайна величина $Y$ имеет гипергеометрическое распределение с параметрами $N, K, n$, то есть для неотрицательного целого $k$
\[
\PR\{Y = k\} = \frac{C_{K}^{k}C_{N - K}^{n - k}}{C_{N}^{n}}.
\]

Получите экспоненциальную оценку на величину $\PR\{Y - \Exp Y \ge \varepsilon\}$.
\end{problem}

\begin{remark}
 Вспомните интерпретацию гипергеометрического распределения и свяжите ее с равномерным распределением на вершинах сечения дискретного куба.
См. Bobkov S.\;G. Concentration of normalized sums and a central limit theorem for noncorrelated random variables, Annals of probability, No.\,4, Pp.\,2884-2907, 2004 и Bobkov S.\;G., Tetali,\;P.
 Modified logarithmic Sobolev inequalities in discrete settings, 
Journal of Theoretical Probability,  No.\,2, Pp.\,289--336, 2006.
\end{remark}

%КАТЯ, ПОСТАРАЙТЕСЬ СДЕЛАТЬ ЭТО ЗАМЕЧАНИЕ ПО ДОСТУПНЕЕ И С КАКОЙ-НИБУДЬ ССЫЛКОЙ

\begin{problem}*
\label{talagran}
В сельском районе, имеющем форму квадрата со стороной $1$, находится $n$ домов $(n\gg 1)$, размерами которых можно пренебречь по сравнению с линейным размером района. Будем считать, что при строительстве домов застройщик случайно (согласно равномерному распределению $R[0,1]^2$)  и независимо выбирал их местоположения. Почтальону необходимо обойти все $n$ домов ровно по одному разу (от любого дома почтальон может направиться к любому другому по прямой). Обозначим через $\ell$ длину наикратчайшего из таких путей (кратчайший гамильтонов путь).
Покажите, что найдется такая константа $c>0$, не зависящая от $n$, что 

\begin{equation*}
\mathbb{P}(\ell - \mathbb{E}\ell \geq t)\leq \exp \biggl(-\frac{t^2}{4c}\biggr). 
\end{equation*}
Можно показать, что $\mathbb{E}\ell\sim\beta\sqrt{n}$, где $\beta$ также не зависит от $n$.

\end{problem}

\begin{remark}
См. книгу Dubhashi D. P., Panconesi A. <<Concentration of measure for the analysis of randomized algorithms>>, Cambridge University Press, 2009.

Приведем несколько неравенств Талаграна. Пусть заданы множества $\Omega_i$, $i=1,\dots,n$, элементарных исходов. На этих множествах заданы вероятностные меры $\PR_i$, $i=1,\dots,n$. Положим 
\begin{equation*}
\Omega= \prod_{i=1}^n\Omega_i,\quad \PR = \prod_{i=1}^n \PR_i.
\end{equation*}
Введем взвешенную метрику Хэмминга:
\begin{equation*}
d_{\alpha}(x,y) = \sum_{x_i\not = y_i} \alpha_i \biggr/ \sqrt{\sum_{i=1}^n \alpha^2_i}
\end{equation*}
и определим $d_{\alpha}(x,A) = \min_{y\in A} d_{\alpha}(x,y)$, $\rho(x,A) = \sup_{\alpha\in \mathbb{R}^n} d_{\alpha}(x,A)$. Пусть $A \in \sigma(\Omega)$. Определим $t$-окрестность ($t\geq$ 0) множества $A$ по формуле 
\begin{equation*}
A_t = \{x\in\Omega: \rho (x,A)\leq t\}.
\end{equation*}
Тогда справедливо неравенство Талаграна:
\begin{equation*}
\mathbb{P}(A)(1-\mathbb{P}(A_t))\leq \exp\biggl(-\frac{t^2}{4}\biggr).
\end{equation*}
\begin{suite}
 %Пусть $X = (X_1,\dots,X_n) $ случайный вектор, значения компонент которого лежат в $[0,1]$. Пусть $F: \mathbb{R}^n\to\mathbb{R}$  выпуклая, липшицева функция с константой $1$ по отношению к метрике Хэмминга ($d(u,v) = \sum_{i=1}^n I(u_i\not=v_i)$,\,\, $u,v \in [0,1]^n$). Пусть $M F(X)$ ~--- медиана $F(X)$, тогда для всех $t>0$ верно
%\begin{equation*}
%\PR (|F(X)-MF(X)|\geq t)\leq 4\mathrm{e}^{-t^2/4}.
%\end{equation*}
%Рассмотрим выпуклую Липшицеву функцию $f$ на ($\mathbb{R}^n$, $\|\cdot\|_2$) с константой Липшица $\sigma$. 

В качестве приложения неравенства Талаграна рассмотрим функцию $h:\; \Omega \to {\rm R}$, $\Omega = \mathbb{R}^n$, удовлетворяющую условию Липшица с константой $\sigma$. Функцию $h$ будем называть проверяемой со сложностью $f$, если при $h(x)\ge s$ существует такое множество $I\subseteq \{1,\ldots ,n\}$ с $\vert I\vert \le f(s)$, что для всех $y\in \Omega $, совпадающих с $x$ в координатах из $I$, выполняется $h(y)\ge s$. Грубо говоря, из выполнения неравенства $h(x)\ge s$ следует, что существует сравнительно небольшое количество 
координат, обеспечивающих выполнение данного неравенства. Тогда из 
неравенства Талаграна можно получить следующее соотношение о плотной 
концентрации случайной величины $h$: для всех $b$ и $t$ 
справедливо 
\[
\PR\left[ {h\le b-t\sqrt {f(b)} } \right]\PR\left[ {h\ge b} \right]\le 
e^{-\frac{t^2}{4}}.
\]
Выбирая либо в качестве $b$, либо в качестве $b-t\sqrt {f(b)} $ медиану случайной величины $h$, получаем результат о плотной концентрации $h$ вокруг своей медианы

\begin{equation*}
\mathbb{P} (|h(X)-M|> t)\leq 4\mathrm{e}^{-t^2/8\sigma^8}.
\end{equation*}
\end{suite}


Неравенство Талаграна для независимых случайных величин. Пусть $X_1,\dots,X_n$ независимые случайные величины в $S$. Для любого класса функций ${\mathcal{F}}$
на $S$ равномерно ограниченного  на $S$ константой $U$ для всех $t>0$ выполнено 
\[
\mathbb{P}\left\{\left\|\sum_{i=1}^n f(X_i)\right\|_{\mathcal{F}} - \mathbb{E}\left\|\sum_{i=1}^n f(X_i)\right\|_{\mathcal{F}}
 \geq t \right\}\leq K\exp\left\{-\frac{t}{UK}\log\left(1+\frac{tU}{V}\right)\right\},
\]
где $K$ - универсальная константа и $V$ удовлетворяет условию 
\[
V\geq \mathbb{E}\sup_{f\in\mathcal{F}}\sum_{i=1}^n f^2(X_i) 
\]
и использованы обозначения  $\|Y\|_{\mathcal{F}} = \sup_{\mathcal{F}}|Y(f)|
$, где $Y: \mathcal{F}\to \mathbb{R}$. 

\end{remark}

%КАТЯ, А ВЫ УВЕРЕНЫ, ЧТО ИМЕННО ТАК? Все как-то странно:)
%КАТЯ ВСЕ ХОРОШО, ТОЛЬКО ЧТО ТАКОЕ ЛИПШИЦЕВА ФУНКЦИЯ ОТНОСИТЕЛЬНО ТОЙ МЕТРИКИ, КОТОРУЮ МЫ ВВЫЕЛИ НЕ ДО КОНЦА ЯСНО, ПОСКОЛЬКУ МЫ ЛИШЬ ОПРЕДЕЛИЛИ РАССТОЯНИЕ ОТ ТОЧКИ ДО МНОЖЕСТВА, НО МНЕ МЕЖДУ ТОЧКАМИ, ЧТО НУЖНО В ОПРЕДЕЛЕНИИ КОНСТАНТЫ ЛИПШИЦА

\begin{problem}[Семейства Леви] 
%Определим для компактного множества $X$ с мерой $\mu$  концентрационную функцию 
%\begin{equation*}
%\alpha(X;\varepsilon) = 1-\inf\biggl\{\mu(A_{\varepsilon})|\, A \subset X, \mu(A)\geq \frac{1}{%2}\biggr\},
%\end{equation*}
%где $A_{\varepsilon} = \bigl\{x\in X | \,\, \rho(s,A) \leq \varepsilon \bigr\}$.
Пусть  $(X_n,\rho_n,\mu_n)$  -- метрическое вероятностное пространство, $X_n$ -- компактное множество с метрикой $\rho_n$, $ \diam X_n\geq 1$ и заданной вероятностной мерой $\mu_n$. Семейство  $(X_n,\rho_n,\mu_n)$ метрических вероятностных пространств называется семейством Леви, если для любого $\varepsilon>0$, $\alpha(X_n,\varepsilon \cdot \diam X_n) \to 0$ для $n\to \infty$ (см. определение концентрационной функции в замечании к задаче \ref{levi}).  Семейство называется нормальным семейством Леви с константами $(c_1,c_2)$, если
\begin{equation*}
\alpha(X_n;\varepsilon) \leq c_1 \exp(-c_2\varepsilon^2 n),
\end{equation*}
Докажите, что следующие семейства будут нормальными семействами Леви.
\begin{enumerate} 
\item \label{first} Пусть на $E^n = \{-1,1\}^n$  задана Хэммингова метрика 
\begin{equation*}
d(s,t) = \frac{1}{n} |\{i:s_i\not=t_i\}|
\end{equation*}
и нормированная считающая мера $\mu$, т.е. $\mu(A) = |A|/2^n$. Докажите неравенство 
\begin{equation*}
\alpha(F_2^n;\varepsilon) \leq \frac{1}{2}\exp(-2\varepsilon^2 n).
\end{equation*}
\item 
Задана группа $\Pi_n$ перестановок $\{1,\dots,n\}$ с заданной нормированной метрикой Хэмминга 
\begin{equation*}
d(\pi_1,\pi_2) = \frac{1}{n} |\{i:\pi_1(i)\not=\pi_2(i)\}|
\end{equation*}
 и нормированной считающей мерой (см. пункт \ref{first}). Докажите неравенство 
\begin{equation*}
\alpha(\Pi_n;\varepsilon) \leq \exp\Bigr(-\varepsilon^2n/64\Bigl).
\end{equation*}
\end{enumerate}
\end{problem}
\begin{remark}
Примеры приведены в статье  Milman V.D. The heritage of P.~Levy in geometrical functional analysis // Asterisques. 1988. V. 157-158. P. 273-302, см. также книгу M. Ledoux <<The Concentration of Measure Phenomenon>>, American Mathematical Soc., 2005, а также теорему Талаграна в замечании к предыдущей задаче.

\end{remark}

\begin{problem}(Кац, Секей)
Рассмотрим полином с вещественными коэффициентами
\begin{equation*}
a_0+a_1t+\ldots+a_{n-1}t^{n-1},
\end{equation*}
где $(a_0,a_1,\dots,a_{n-1})$ точка на единичной сфере $S_n(1)$. 
 Среднее число вещественных корней полинома определим следующим образом:
$$
\mathbb{E} N = \frac{1}{|S_n(1)|}\int_{S_n(1)}N(a)d\sigma,
$$
где  $N(a)$ число вещественных корней полинома, $d\sigma$~--- элемент поверхности единичной сферы площадью
$$|S_n(1)| = \frac{(2\pi)^{n/2} }{ \Gamma\left(\frac{n}{2}\right) }.$$


\begin{enumerate}
\item 
Покажите, что среднее число вещественных корней полинома с коэффициентами на единичной сфере равно среднему числу вещественных корней полинома, коэффициенты которого независимы и распределены стандартно нормально, то есть 
\begin{equation*}
\mathbb{E} N = (2\pi)^{-n/2}\int_{-\infty}^{\infty}\dots\int_{-\infty}^{\infty}N(a)\exp \left(-\frac{1}{2}\|a\|^2\right)\,da_0,\dots,\,da_{n-1}.
\end{equation*}
\item 
Пусть $\mathbb{E}N^{(1)}$, $\mathbb{E}N^{(2)}$, $\mathbb{E}N^{(3)}$, $\mathbb{E}N^{(4)}$~--- средние значения числа вещественных корней, заключенных в интервалах $(-\infty,-1)$, $(-1,0)$, $(0,1)$, $(1,\infty)$ соответственно.
Покажите, что 
\begin{equation*}
\mathbb{E}N^{(1)}=\mathbb{E}N^{(2)}=\mathbb{E}N^{(3)}=\mathbb{E}N^{(4)} 
\end{equation*}
и
\begin{equation*}
\mathbb{E}N^{(i)}\sim (2\pi)^{-1}\ln n, \quad n\to\infty.
\end{equation*}
%\item
\end{enumerate}
\end{problem}
\begin{remark}
Н.Б. Маслова в статье О распределении числа вещественных корней случайных полиномов // ТВП, 1974, Т. 19, В. 3, с. 488–-500 доказала следующую теорему: если коэффициенты $X_i$ случайного алгебраического уравнения 
\[
\sum_{j=1}^n X_jz^j=0
\]
являются независимыми одинаково распределенными случайными величиными с нулевыми математическими ожиданиями и $$\mathbb{E}\left(|X_j|^{2+\epsilon}\right)<\infty$$ для некоторого положительного $\epsilon$, то число действительных корней этого уравнения распределено нормально с математическим ожиданием $\frac{2}{\pi}\ln n$ и стандартным отклонением $2\sqrt{\pi^{-1}(1-2\pi^{-1})\ln n}$.
\end{remark}


\begin{problem}$^{**}$
Дан случайный граф (модель Эрдеша--Реньи, см. задачу \ref{sec:erdRenyi} раздела \ref{hard}) $G\left( {n,\;p} \right)$ с $n$ вершинами и 
вероятностью появления каждого ребра $p$. Пусть $p\ge \sqrt {\frac{2 \ln 
n}{n}} $, причем длины ребер $r_{ij} $ являются независимыми случайными величинами, имеющими равномерное распределение на отрезке 
$\left[ {0,\;2r} \right]$. Покажите, что тогда почти наверное граф $G\left( 
{n,\;p} \right)$ имеет гамильтонов цикл, причём длина почти всех 
гамильтоновых циклов стабилизируется около $nr$.
\end{problem}

\begin{problem}$^{**}$
Дан случайный граф (модель Эрдеша--Реньи, см. задачу \ref{hard}.\ref{sec:erdRenyi}) $G\left( {n,\;p} \right)$ с $n$ вершинами и 
вероятностью появления каждого ребра $p\in \left[ {\varepsilon ,1} \right]$, 
$\varepsilon >0$. Вес каждого появившегося случайного ребра разыгрывается  независимо 
согласно равномерному распределению на отрезке $\left[ {0,2} 
\right]$.  Источник и сток выбираются случайно. Обозначим через $S_n $ значение максимального потока для 
полученного случайного взвешенного графа. Покажите, что 
${S_n } 
\mathord{\left/ {\vphantom {{S_n } {pn}}} \right. \kern-\nulldelimiterspace} 
{pn}\buildrel p \over \longrightarrow 1.$
\end{problem}

\begin{problem}[Cтохастическое агрегирование; В.И. Опойцев]
\begin{enumerate}
\item На рынке имеется $n$ 
продавцов, каждый из которых может продать свой товар в объеме $x_k $ для 
$k$-го товара. Спрос на товары обеспечивают покупатели. Пусть $y_k$ -- спрос на 
товар $k$. Общий объем сделок $L_n=\sum\limits_{k=1}^n {\min 
\left\{ {x_k ,y_k } \right\}} $. Предложите такой естественный способ 
определения вероятностной меры на двух симплексах
\[
S_X =\left\{ {\vec {x}\ge \vec {0}:\;\;\sum\limits_{k=1}^n {x_k } =X} 
\right\},
\quad
S_Y =\left\{ {\vec {y}\ge \vec {0}:\;\;\sum\limits_{k=1}^n {y_k } =Y} 
\right\},
\]
чтобы нашлась функция $f\left( {X,Y} \right)$, удовлетворяющая  $$L_n 
\mathord{\left/ {\vphantom {L n}} \right. \kern-\nulldelimiterspace} 
n\buildrel p \over \longrightarrow f\left( {X,Y} \right).$$

\item Пусть $\vec {y}=A\vec {x}$, $A=\left\| {a_{ij} } 
\right\|_{i,j=1,1}^{l,n} $, $X=\left\langle {\vec {p},\vec {x}} 
\right\rangle $, $Y=\left\langle {\vec {q},\vec {y}} \right\rangle $. 
Матрицы и векторы предполагаются положительными. Легко проверить, что
\[
Y=\lambda \left( {\vec {x}} \right)X,
\quad
\lambda \left( {\vec {x}} \right)=\sum\limits_{i,j} {\frac{q_i a_{ij} }{p_j 
}} \frac{p_j x_j }{X}\mathop =\limits^{def} \sum\limits_{i,j} {b_{ij} } z_j 
.
\]
Считая $z_j >0$ независимыми одинаково распределенными с.в.: $\Exp z_j =m_j $ 
($\sum_{j} m_j  =1)$, $\Var z_j =\sigma _j^2 $ (например, $z_j \in 
\left[ {0,2n^{-1}} \right])$, покажите, что если выражение
\[
\frac{\mathop {\max }\limits_j \sum\limits_i {b_{ij} } \cdot \mathop {\max 
}\limits_j \sigma _j }{\mathop {\min }\limits_j \sum\limits_i {b_{ij} } 
\cdot \mathop {\min }\limits_j m_j }
\]
равномерно ограничено с ростом $n$, то существует такое число $\bar {\lambda 
}$, что с вероятностью стремящейся к единице $Y\simeq \bar {\lambda }X$.

\end{enumerate}
\end{problem}


\begin{problem}(Устойчивые системы большой размерности; В.И. Опойцев)
Из курсов функционального анализа и вычислительной математики хорошо известно, что если спектральный радиус матрицы 
$A=\| a_{ij}\|_{i,j=1}^{n}$ меньше единицы, $\rho(A)<1$, то итерационный процесс $x^{k+1}=A x^k + b$ 
(СОДУ $\dot{x}=-x+A x+ b$), вне зависимости от точки старта $x^0$, 
сходится к единственному решению уравнения $x^*=Ax^*+ b$. 
Скажем, если $\| A\|=\max\limits_{i} \sum\limits_j |a_{ij}|<1$, то и $\rho(A)<1$ (обратное, конечно, не верно). Предположим, что 
существует такое $\varepsilon>0$, что 
$$
\frac{1}{n}\sum\limits_{i,j=1}^n |a_{ij}|<1-\varepsilon . 
\quad (S)
$$
Очевидно, что отсюда не следует: $\rho(A)<1$. 
Тем не менее, введя на множестве матриц, удовлетворяющих условию $(S)$, равномерную меру, покажите, что относительная мера тех матриц 
(удовлетворяющих условию $(S)$), для которых спектральный радиус не меньше единицы, стремится к нулю 
с ростом $n$ ($\varepsilon$ --- фиксировано и от $n$ не зависит). 
\end{problem}
\begin{ordre}

1. Покажите, что  достаточно рассматривать матрицы с неотрицательными элементами. 

2.  Покажите, что достаточно доказать утверждение задачи на множестве матриц, удовлетворяющих условию 
$$
\frac{1}{n}\sum\limits_{i,j=1}^n a_{ij}=1-\varepsilon . 
\quad (SE)
$$

3. Далее положим $a_{ij}\in \mathrm{Exp} \bigl( n/(1-\varepsilon)\bigr)$ --- независимые одинаково распределенные случайные величины. Покажите, что при $n\to\infty$ распределение элементов случайной матрицы $A=\| a_{ij}\|_{i,j=1}^n$ 
будет сходиться к равномерному распределению на множестве матриц, удовлетворяющих ($SE$). 

4. Введя обозначения
$P_n={\mathbb P}(\| A\|\ge 1)\ge {\mathbb P}(\rho(A)\ge 1)$, воспользуйтесь неравенством Чебышёва

$$
P_n\le n {\mathbb P}\Bigl( \sum\limits_{j=1}^n a_{1j}\ge 1 \Bigr)=n {\mathbb P}\Bigl( X\ge 1 \Bigr)\le 
n {\mathbb P}\Bigl( |X-(1-\varepsilon)|\ge \varepsilon \Bigr)=
$$
$$
=n {\mathbb P}\Bigl( |X-{\mathbb E}X|\ge \varepsilon \Bigr)\le \frac{n}{\varepsilon^4} {\mathbb E}(X-{\mathbb E}X)^4=
O\Bigl(\frac{1}{n}\Bigr) \xrightarrow{n\to\infty} 0 . 
$$
\end{ordre}



\begin{problem}[Красносельский--Крейн]
Рассмотрим систему линейных алгебраических уравнений 
\begin{equation*}
x=Ax+b,
\end{equation*}
где $A$ положительно определенная неособенная матрица, собственные числа $\lambda_1\leq\dots\leq\lambda_n$ которой меньше единицы, $b$~--- заданный и  $x$ ~--- искомый векторы $n$-мерного подпространства. Пусть эта система решается при помощи итерационного процесса 
\begin{equation*}
x_{m+1}= Ax_{m}+b,\quad m=1,2,\dots,
\end{equation*}
 который заканчивается на $p$-м шаге, если вектор--невязка $\delta_p = x_{p+1}-x_{p}$ попадает в шар радиуса $\alpha$ с центром в нуле. 
Ошибка итерационного процесса $\varepsilon_m = x^{*} - x_m$, где  $x^{*}$ истинное решение системы уравнений, связана с невязкой  (проверить)
 \begin{equation*}
\varepsilon_m = (I-A)^{-1}\delta_m,
\end{equation*}
поэтому  исходя из значений вектора невязки можно определить вероятностное распределение  ошибок. 

Оказывается, что наиболее вероятными ошибками являются максимальные.
А именно, пусть начальная ошибка равномерно распределена в шаре $T$ радиуса $R$, тогда для любого $\eta<1$ вероятность выполнения следующего неравенства стремится к единице при $R\to\infty$
\begin{equation*}
\eta \frac{\alpha \lambda_n}{1-\lambda_n}\leq \|\varepsilon\| \leq \frac{\alpha}{1-\lambda_n}.
\end{equation*} 

Проверьте это для случая $n=2$.

\end{problem}

\begin{remark}
Пусть итерационный процесс заканчивается на шаге $p$, если вектор-невязка $\delta_p=x_{p+1}-x_p$ попадает в окрестность $G$ нуля. Пусть $G$~--- шар радиуса $\alpha$. Тогда ошибка $\varepsilon_p$ попадет в множество $G_0 = (I-A)^{-1}G$, которое является эллипсоидом с полуосями длины $\frac{\alpha}{1-\lambda_1}$ и $\frac{\alpha}{1-\lambda_2}$.
Обозначим $G_{-1}=AG_0$ ~--- эллипсоид с полуосями $\frac{\alpha\lambda_i}{1-\lambda_i}$, $i=1,2$. 
 
\imgh{100mm}{krein.pdf}{Множества ошибок в случае матрицы $2\times 2$. Закрашенная темно-серым цветом часть соответствует области из утверждения задачи при $\eta$ близком к $1$.}

Обозначим $G_m = A^{-m}G_0$. Процесс заканчивается на $p$-м шаге, если $\varepsilon_0\in G_p$ и $\varepsilon_0\not\in G_p$, те когда $\varepsilon_0$ находится в слое $G_p-G_{p-1}$.
При этом окончательная ошибка будет находиться в слое $A^{p}(G_p-G_{p-1}) = G_0-G_{-1}$.Таким образом окончательная ошибка принадлежит $G_{-1}$ только в случае, если $\varepsilon_0$ принадлежит $G_0$. 
Вероятность нахождения окончательной ошибки в элементе слоя $G_0-G_{-1}$ равна сумме вероятностей нахождения начальной ошибки в элементах $\Delta=A^{-m}\Delta_0$ ($m=0,1,\dots$).

Предположим, что начальная ошибка $\varepsilon_0$ равномерно распределена в шаре $T$ достаточно большого радиуса. В этом случае вероятность $P(\Delta_m)$ нахождения начальной ошибки в элементе $\Delta_m\in T$ пропорциональна объему этого элемента. Подсчитайте вероятность $P(\Delta_0)$ попадания окончательной ошибки в элемент $\Delta_0$ и убедитесь, что вероятность попадания окончательной ошибки в точку слоя $G_0-G_{-1}$ для данной точки тем больше, чем позже эта точка выйдет из $T$ при последовательном применении к ней операции $A^{-1}$. 

Пусть $e_i$~--- собственные векторы матрицы $A$, соответствующие собственным числам $\lambda_i$, $i=1,2$.
Для доказательства утверждения задачи оцените вероятность того, что окончательная ошибка $\varepsilon = \sum_{i=1}^n\xi_ie_i$ не удовлетворяет неравенству (утверждению задачи). Перейдя к пределу по $R\to\infty$ получите, что вероятность невыполнения утверждения задачи стремится к нулю.


\end{remark}

%КАТЯ, НЕОБХОДИМО ТАКЖЕ ДОБАВИТЬ ЗАМЕЧАНИЕ С РИСУНКОМ :)

\begin{problem}[Физическая интерпретация концентрации меры на сфере]
Имеется $n$ частиц массы $m$  со скоростями $v_i$, $i=1,\dots,n$. Известно, что вектор скоростей молекул идеального газа равномерно
распределен по поверхности постоянной энергии. Суммарная кинетическая энергия $E_n$ растет пропорционально $n$, то есть 
\begin{equation*}
\frac{1}{2}mv_1^2+\cdots+\frac{1}{2}m v_n^2 = E_n;\quad \sum_{i=1}^n v^2_i=\frac{2E_n}{m}\asymp n.
\end{equation*}
Получите закон распределения Максвелла скоростей частиц одномерного идеального газа. 
\end{problem}

\begin{ordre}
В решении задачи 1 этого раздела перейдите к термодинамическому пределу, когда $n\to\infty$, $r = \sigma n^{1/2}$, чтобы получить закон распределения Максвелла скоростей частиц.
\end{ordre}

\begin{remark}
Равномерное распределение на поверхности постоянной энергии возникло из-за того, что инвариантной (и предельной, то есть возникающей на больших временах, по эргодической гипотезе) мерой для гамильтоновой системы будет как раз равномерная мера  по теореме Лиувилля (фазовый объем сохраняется). Поскольку выполняется закон сохранения энергии, то система  ``живет'' \mbox{~на} поверхности постоянной энергии. Следовательно, носитель инвариантной меры сосредоточен именно на этой поверхности. 

Приведем для справки результат, полученный  для выпуклых тел с заданной на них равномерной мерой (теорема Б. Клартага в статье Milman V.D. Geometrization of Probability // Progress in Mathematics. 2008. V. 265, http://www.math.tau.ac.il/~milman/, также см. статью Klartag B. A central limit theorem for convex sets // Inventiones mathematicae
April 2007, Volume 168, Issue 1, pp. 91--131). Существует последовательность $\epsilon_n\to 0$, ${n\to\infty}$ для которой выполнено: пусть $K\in \mathbb{R}^n$ выпуклое компактное множество с непустой внутренностью, случайный вектор $X$ распределен равномерно в $K$, тогда существуют вектор $\theta\in\mathbb{R}^n$, $t_0\in\mathbb{R}$ и $\sigma>0$, что выполнено
\[
\sup_{A\in\mathbb{R}}\left|\mathbb{P}\left\{\sum_{i=1}^n X_i\theta_i\in A\right\} - \frac{1}{\sqrt{2\pi}\sigma}\int_{A}{\rm e}^{-\frac{(t-t_0)^2}{2\sigma^2}}\,dt\right|\leq \epsilon_n,
\]
где супремум берется по всем измеримым множествами $A\in\mathbb{R}$. Eсли $\mathbb{E}X_i=1$, $\mathbb{E}X_iX_j = \delta_{ij}$, то $t_0=0$, $\sigma=1$, то можно найти такой единичный вектор  $\theta$, что выполняется приведенное неравенство.

\end{remark}

\begin{problem}[Лемма Пуанкаре]
Пусть $X_n$ -- случайный вектор с равномерным распределением на единичной сфере в ${\mathbb R}^n$. Равномерное распределение 
характеризуется тем, что оно инвариантно относительно группы ортогональных преобразований. Пусть $Y_n$ обозначает первую координату $X_n$. 
Докажите, что $\sqrt{n}\, Y_n \xrightarrow{d}N(0,1)$ при $n\to\infty$. Заметим, что в статистической физике с помощью утверждения 
этой задачи получался закон распределения Максвелла скоростей частиц одномерного идеального газа. 
\end{problem}

\begin{ordre}

См. предыдущую задачу. Решение задачи содержит в себе способ генерирования равномерного распределения. 
Пусть $\xi_1,\ldots, \xi_n$ --- независимые в совокупности с.в., имеющие одинаковое распределение $N(0,1)$. Рассмотрим случайный вектор 
$Z_n=(\xi_1,\xi_2,\ldots,\xi_n)$. Тогда $Z_n\in N(0,I_n)$, $I_n$ --- единичная матрица размера $n$. 

Показажите, что $Z_n$ инвариантно относительно группы ортогональных преобразований. Заметим, что распределения 
$$
X_n \quad\text{ и } \quad \frac{Z_n}{\|Z_n \|_{{\mathbb R}^n}} \quad \text{ совпадают. }
$$
Поэтому имеет место равенство по распределению с.в. 
$$
Y_n=\frac{\xi_1}{\sqrt{\xi_1^2+\ldots+ \xi_n^2}} 
$$
$$
\Rightarrow \quad \sqrt{n}Y_n = \frac{\xi_1}{\sqrt{(\xi_1^2+\ldots+ \xi_n^2)/n}} . 
$$
Применить теорему Колмогорова у.з.б.ч. для $\frac{\xi_1^2+\ldots+ \xi_n^2}{n}$. 

\end{ordre}

\begin{problem}[Геометрическая интерпретация закона больших чисел]
Рассмотрим куб $C^n = [-1,1]^n$ в евклидовом пространстве $\mathbb{R}^n$. Пусть $\xi_i$, $i=1,\dots,n$ независимые случайные величины с равномерным распределением на $[-1,1]$. Приведите геометрическую интерпретацию закона больших чисел.
\end{problem}

\begin{ordre} 
Рассмотреть объем  следующего множества ---   пусть $\mathcal{H}$ часть гиперплоскости, содержащаяся в кубе и перпендикулярная главной диагонали куба, т.е.  $\sum_{i=1}^n x_i = 0$. Необходимо подсчитать объем $\varepsilon\sqrt{n}$-окрестности $\mathcal{H}$. 
\end{ordre}



\begin{problem} [В.И. Опойцев]
\begin{enumerate}
\item[а)] Пусть имеются абсолютно непрерывные (имеющие плотность) независимые случайные величины $X_1,\ldots, X_n$ и пусть 
$Y_n=G_n(X_1,\ldots, X_n)$ --- также случайная величина. Докажите следующее неравенство, описывающее нелинейный закон больших чисел: 
$$
\Var Y_n\leqslant \int_{{\mathbb R}^n} \sum\limits_{i=1}^{n} \Bigl(\frac{\partial G_n(x_1,\ldots, x_n)}{\partial x_i} \Bigr)^2 f_i^*(x_i)
\prod\limits_{j\ne i} f_j(x_j)\, dx_1\ldots dx_n , 
$$
где сопряженные плотности $f_i^*(x_i)$ существуют и определяются следующим образом 
$$
f_i^*(x_i)=\mu_i(\infty)\int\limits_{-\infty}^{x_i} f_i(t)\, dt -\mu_i(x_i), \quad 
\mu_i(x)=\int\limits_{-\infty}^{x} tf_i(t)\, dt . 
$$

\item[б)] Пусть независимые одинаково распределенные случайные величины $X_1,\ldots, X_n$ имеют равномерное распределение на отрезке 
$[0,1]$ (часто пишут $X_i\in R[0,1]$), а 
$$
\max\limits_{x\in [0,1]^n} \|\nabla G_n(x_1,\ldots, x_n) \|\xrightarrow{n\to\infty} 0 . 
$$
Тогда $Y_n\xrightarrow{P}{\mathbb E}Y_n$ при $n\to\infty$. 

\item[в)] Пусть независимые одинаково распределенные случайные величины $X_1,\ldots, X_n$ имеют равномерное распределение на отрезке 
$[0,1]$, а $Y_n=G_n(X_1,\ldots, X_n)=\max\limits_{i=1,\ldots, n} X_i$. Тогда $Y_n\xrightarrow{p}{\mathbb E}Y_n$ при $n\to\infty$. 

\end{enumerate}
\end{problem}

\begin{remark}
См. В. Босс Лекции по математике. Т.4: Вероятность, информация, статистика, 2005.
\end{remark}

%По мотивам лекции Голубева, обзора Лугоши.

%\begin{enumerate}
%\item
\begin{problem}[Неравенство Чернова]
Докажите, что неравенство Чернова для неотрицательной случайной величины $X$
\begin{equation*}
\PR\{ X >t\}\leq \inf_{s>0}\mathbb{E}\exp(sX-st)
\end{equation*}
 дает более завышенную границу по сравнению с моментной границей
\begin{equation*}
\PR\{ X >t\}\leq \min_{q>0}\mathbb{E}[X^q]t^{-q},
\end{equation*}
 то есть 
\begin{equation*}
\min_{q>0}\mathbb{E}[X^q]t^{-q}\leq \inf_{s>0}\mathbb{E}\big[\text{e}^{s(X-t)}\bigl].
\end{equation*}
\end{problem}

\begin{ordre}См. \cite{BLM}. Использовать следствие из неравенства Маркова: для монотонной возрастающей неотрицательной функции $\phi(\cdot)$ и произвольной случайной величины $X$ верно
\begin{equation*}
\PR\{\phi(X)\geq \phi(t)\}\leq \frac{\mathbb{E}\phi(X)}{\phi(t)}.
\end{equation*}
\end{ordre}

%\item 
%\begin{problem}[Неравенство Чернова для суммы случайных величин]

%\end{problem}

%\item


\begin{problem}[Лемма Хёфдинга] Пусть $X$--- случайная величина, такая что $\mathbb{E}X =0$, $a\leq X\leq b$. Покажите, что для $s>0$ верно
\begin{equation*}
\mathbb{E}\exp(s X)\leq \exp\bigg[\frac{s^2(b-a)^2}{8}\biggr].
\end{equation*}
\end{problem}

\begin{ordre}
Используя выпуклость экспоненты, для $a\leq x\leq b$

\begin{equation*}
\text{e}^{sx} \leq \frac{x-a}{b-a} \text{e}^{sb}+\frac{b-x}{b-a}\text{e}^{sa},
\end{equation*}

получите 

\begin{equation*}
\mathbb{E}\text{e}^{sx} \leq \text{e}^{\phi(u)},
\end{equation*}
где $u = s(b-a)$, $\phi(u) = -pu+\log(1-p+p\text{e}^u)$, $p = -a/(b-a)$.
Найдите $\phi^{\prime\prime}(u)$,   $\phi(0)$, $\phi^{\prime}(0)$.
Покажите, что 
\begin{equation*}
\phi^{\prime\prime}(u)\leq \frac{1}{4}.
\end{equation*}

Используя формулу Тейлора, получите 
\begin{equation*}
\phi(u) \leq \frac{u^2}{8}\leq \frac{s^2(b-a)^2}{8}.
\end{equation*}
\end{ordre}

\begin{problem}[Теорема Хёфдинга] Пусть $\xi_t$, $t\in T$ -- независимые случайные величины, такие что $\xi_t\in[a,b]$. Докажите, что 
\begin{equation*}
\PR\bigg\{\bigg|\frac{1}{n}\sum_{t\in T}\big(\xi_t-\mathbb{E}\xi_t\bigr)\biggr|\geq x\biggr\}\leq 2\exp\bigg\{-\frac{2nx^2}{(b-a)^2}\biggr\}.
\end{equation*}
\end{problem}
\begin{ordre}
Введите случайную величину $\xi = \frac{1}{n}\sum_{i=1}^n(\xi_i-\mathbb{E}\xi_i)$.
Воспользуйтесь неравеством Чернова и леммой Хёфдинга для $\xi$ и получите
\begin{equation*}
\PR\{\xi>x\}\leq \exp\bigg\{\min_{\lambda}\bigg[-\lambda x + \frac{\lambda^2}{8}\frac{(b-a)^2}{n}\biggr]\biggr\}.
\end{equation*}
Затем найдите оптимальное $\lambda$.  Аналогичное неравенство справедливо для $-\xi$.
\end{ordre}

\begin{problem}[Неравенство Беннетта]
Пусть $X_1,\dots, X_n$ независимые центрированные ограниченные случайные величины, такие, что с вероятностью $1$ выполнено $|X_i|\leq c$.
Пусть $\sigma^2 = \sum_{i=1}^n\Var\{X_i\}$.
Покажите, что для любого $t>0$ 
\begin{equation*}
\PR\bigg\{\sum_{i=1}^n X_i>t\biggr\}\leq \exp\bigg(-\frac{n\sigma^2}{c^2}h\bigg(\frac{ct}{n\sigma^2}\biggr)\biggr),
\end{equation*}
где $h(u) = (1+u)\log(1+u)-u$ для $u\geq 0$.
\end{problem}
\begin{ordre}
Введем $\sigma_i^2 = \mathbb{E}[X_i^2]$ и $F_i = \sum_{r=2}^{\infty}\frac{s^{r-2}\mathbb{E}[X_i^r]}{r!\sigma_i^2}$.
Используя разложение для ряда Тейлора $\exp(sX)$, показать, что 
\begin{equation*}
\mathbb{E}[e^{sX_i}]\leq \exp(s^2\sigma^2_iF_i).
\end{equation*}

Из ограниченности  $X_i$ получите оценку
\begin{equation*}
F_i\leq \frac{\exp(sc)-1-sc}{(sc)^2}.
\end{equation*} 
Далее воспользуйтесь неравенством Чернова для $X_i$ и минимизируйте правую часть в неравенстве Чернова по $s$.
\end{ordre}

\begin{problem}[Неравенство Бернштейна \cite{BLM}]
\label{bernstain}
Докажите, что при выполнении условий предыдущей задачи для любого  $\varepsilon>0$ верно следующее неравенство
\begin{equation*}
\PR\bigg\{\frac{1}{n}\sum_{i=1}^n X_i>\varepsilon\biggr\}\leq \exp\bigg(-\frac{n\varepsilon^2}{2\sigma^2+2c\varepsilon/3}\biggr).
\end{equation*}
\end{problem}

\begin{ordre} 
Покажите, что верно элементарное неравенство 
\begin{equation*}
h(u)\geq \frac{u^2}{2+2u/3}
\end{equation*}
и используйте неравенство Беннетта.

\begin{comment}
Отметим, что в случае схемы Пуассона, т.е. если рассмотреть последовательность $\xi_1,\dots,\xi_n$ с распределнием Бернулли с вероятностью успеха $\lambda/n$,   оценка в асимптотике по $n$ принимает вид, аналогичный  неравенству больших уклонений.
\end{comment}

\end{ordre}
%\end{enumerate}


\begin{problem}[Неравенство Азумы--Хёфдинга]
Пусть $\{X_i\}_{i=0}^{\infty}$ последовательность со следующим свойством (см. замечание к задаче 3.8)
\begin{equation*}
\mathbb{E}(X_n|X_1,\dots,X_{n-1}) =X_{n-1},
\end{equation*}
и пусть $Y_i = X_i-X_{i-1}$ соответствующая последовательность приращений (мартингальная разность). Покажите, что если существуют такие $c_i>0$, что $|Y_i|\leq c_i$ для всех $i$, то
\begin{equation*}
\PR\Bigl\{\sum_{i=1}^{m}Y_i \geq t\Bigr\}\leq 2\exp\bigg\{\frac{-t^2}{2\sum_{i=1}^{m}c^2_i}\biggl\}.
\end{equation*}
\end{problem}
\begin{ordre}

%\begin{enumerate}
%\item \textit{Использовать теорему Дуба.} 
%%\textbf{TODO}
%\item \textit{Cпособ для доказательства не равномерного варианта теоремы %использует результат задачи 12.}

Вначале необходимо докажите следующее утверждение.
 Пусть $Y$ случайная величина,  $Y\in [-1,+1]$ и $\mathbb{E}[Y]=0$. Тогда для любого $t\geq 0$ верно 
\begin{equation*}
\mathbb{E}[\exp(tY)]\leq \exp(t^2/2).
\end{equation*}
Для этого необходимо использовать выпуклость $\exp(tx)$, а именно для $x\in[-1,1]$ верно
\begin{equation*}
\text{e}^{tx}\leq \frac{1}{2}(1+x)\text{e}^{t} +\frac{1}{2}(1-x)\text{e}^{-t}.
\end{equation*}
Подсчитайте оценку математического ожидания $\mathbb{E}[\text{e}^{tY}]$ используя разложение экспоненты в ряд Тейлора и элементарный факт $(2n)!>2^nn!$. Затем покажите, что 
\begin{equation*}
\mathbb{E}\exp\biggl(s\sum_{j=1}^m Y_j\biggr) = \mathbb{E}\biggl[\exp\biggl(s \sum_{j=1}^{m-1} Y_j\biggr)\mathbb{E}[\exp(sY_m)|X_{1},\dots,X_{m-1}]\biggl]. 
\end{equation*}
Используйте неравенство Чернова
\begin{equation*}
\PR[Y_1+\dots+Y_m>t]\leq \exp\bigg[-st+\sum_{i=1}^mc^2_i s^2/2\biggr].
\end{equation*}
Остается минимизировать правую часть неравенства по $s$.

\end{ordre}


\begin{problem}
Пусть независимые одинаково распределенные невырожденные ($\not \equiv \text{const}$) случайные величины $\xi_1,\xi_2,\dots$ с математическим ожиданием $m$ удовлетворяют \textit{условию Крамера}, то есть существует такая окрестность нуля, что для любого $\lambda$ из этой окрестности 
\begin{equation*}
\mathbb{E}e^{\lambda \xi}<\infty.
\end{equation*}
Пусть 
$$
S_n = \xi_1+\dots+\xi_n, \quad \psi(\lambda) = \ln \mathbb{E}\exp(\lambda\xi)
$$ 
и 
\begin{equation*}
H(a) = \sup_{\lambda}[a\lambda - \psi(\lambda)],\quad a\in \mathbb{R},
\end{equation*}
Покажите, что верно следующее неравенство Дуба
\begin{equation*}
\PR\biggl\{\sup_{k\geq n}\biggr|\frac{S_k}{k}-m\biggl|>\varepsilon\biggl\}\leq 2\exp\biggl(-\min\Big[H(m-\varepsilon),H(m+\varepsilon)\Bigr]n\biggr).
\end{equation*}
\end{problem}

\begin{ordre}
Зафиксируйте $n\geq 1$ и положите 
\begin{equation*}
\kappa  = \inf \biggl\{k\geq n: \frac{S_k}{k}>a\biggr\}, 
\end{equation*}
считая $\kappa = \infty$, если $\frac{S_k}{k}\leq a$, $k\geq n$.
Пусть $\lambda>0$  и $\lambda a - \psi(\lambda)\geq 0$. 
Показать, что 
\begin{equation*}
\begin{split}
&\mathbb{P}\biggl\{\sup_{k\geq n}\frac{S_k}{k}>a\biggr\} = \mathbb{P}\biggl\{\frac{S_{\kappa}}{\kappa}>a,\,\kappa<\infty\biggr\} \\
&\leq \mathbb{P}\biggl\{\exp\Bigl(\lambda S_{\kappa} - \kappa  \psi (\lambda)\Bigr)>\exp\Bigl(n\lambda -n  \psi(\lambda)\Bigr),\kappa<\infty \biggr\}\\
&\leq 
 \mathbb{P}\biggl\{\sup_{k\geq n}\exp\Bigl(\lambda S_{k} - k  \psi (\lambda)\Bigr)>\exp\Big(n\lambda -n  \psi(\lambda)\Bigr)\biggr\}.
\end{split}
\end{equation*}

Затем, необходимо воспользоваться тем фактом, что последовательность случайных величин $Y_k = \exp (\lambda S_k-k \psi(\lambda))$, $k\geq 1$ является мартингалом (см. задачу \ref{sec:doob}  раздела \ref{hard}).

Для момента остановки
$\tau = \inf \{k\leq n: Y_k\geq \lambda \}$, $\tau = n$ если $\max_{k\leq n}Y_k <\lambda$ верна теорема Дуба (см. задачу \ref{sec:doob}  раздела \ref{hard}).
Тогда  из неравенства Маркова для любого $x>0$
\begin{equation*}
x \cdot \mathbb{P}\biggl\{\sup_{k\geq n} Y_k \geq x\biggr\}\leq \mathbb{E}Y_n.
\end{equation*}
Отсюда
\begin{equation*}
\mathbb{P}\biggl\{\sup_{k\geq n}\frac{S_k}{k}>a\biggr\}\leq \exp\biggl\{-n \Bigl(\lambda a -  \psi(\lambda)\Bigr)\biggr\}.
\end{equation*}
Рассмотрите случаи $a>m$ и $a<m$.

\end{ordre}
%\begin{problem}
%Что можно сказать о том, как соотносятся между собой неравенство Бернштейна и Хевдинга? Рассмотреть неравенство Бернштейна в случаях, когда $\sigma^2>\varepsilon$ и когда $\sigma^2<\varepsilon$. Что можно сказать про достижимость неравенства Бернштейна? 
%\begin{remark}
%Рассмотреть предельную теорему Пуассона.
%\end{remark}
%\end{problem}



\begin{problem}
Докажите, что для последовательности независимых одинаково распределенных с.в. $\xi_1,\xi_2,\dots$ с математическими ожиданиями $m = \mathbb{E} \xi_i$, дисперсиями $\Var \xi_i = d$ и функцией распределения $F(x)$ верна следующая оценка вероятности больших отклонений %$\PR\Bigl\{\Bigl|\sum_{i=1}^n\xi_i -n m \Bigr|\geq %n\varepsilon \Bigr\}$: 

\begin{equation*}
\PR\Bigl\{\Bigl|\sum_{i=1}^n \xi_i -n m \Bigr|\geq n\varepsilon \Bigr\} \leq B_n(\phi(t_0))^n \exp\big(-t_0 n\varepsilon \bigr) ,
\end{equation*}
где 
%\begin{equation*}
$\lim_{n\to\infty}B_n=\frac{1}{2},$
%\end{equation*}
%\begin{equation*}
$\phi(t)=\int_{-\infty}^{\infty}\text{e}^{t x}\,d F(x),$
%\end{equation*}
%\begin{equation*}
$m(t) = \frac{\phi^{\prime}(t)}{\phi(t)},$
%\end{equation*}
%\begin{equation*}
$r(\lambda_0) = \text{e}^{-\lambda_0 c}R(\lambda_0),$
%\end{equation*}
значение $t_0$ удовлетворяет условию  $m(t_0) = \varepsilon$.

\end{problem}
\begin{remark}


Если использовать для оценки  $\PR\Bigl\{\Bigl|\sum_{i=1}^n\xi_i - n m\Bigr|\geq s\Bigr\}$ неравенство Чебышева, то при $s = \varepsilon \sqrt{n}$ и $s=\varepsilon n$, где $\varepsilon$~--- некоторая постоянная, получается разный порядок сходимости вероятности. В отличие от первого случая, оценка при $s=\varepsilon n$ является очень грубой. 

Доказательство существования, единственности, а также положительности $t_0$  может быть найдено в учебнике Коралова--Синая.

Воспользуемся \textit{методом Крамера} вычисления асимптотики вероятностей. Покажите, что верно 
\begin{equation*}
\PR\Bigl\{\Bigl|\sum_{i=1}^n \xi_i- n m\Bigr|\geq n \varepsilon \Bigr\} \leq (\phi(t_0)\text{e}^{-t_0\varepsilon})^n \idotsint\limits_{\sum_{i=1}^n x_i>\varepsilon n} \,dF_{t_0}(x_1) \dots dF_{t_0}(x_n),
\end{equation*}
где $F_t(x) = \frac{1}{\phi(t)}\int_{-\infty}^{x} \text{e}^{t u}d F(u)$ -- функция распределения (проверьте это).
Для того, чтобы оценить интеграл, рассмотрите случайные величины $\tilde{\xi}_1,\dots,\tilde{\xi}_n$ с распределением $F_{t_0}$, воспользуйтесь центральной предельной теоремой, чтобы показать, что $B_n\to1/2$ при ${n\to\infty}$.
\end{remark}



\begin{problem}
\label{chernov_th}
 Задана последовательность независимых одинаково распределенных случайных величин $\xi_1,\xi_2,\dots$ и $S_n = \xi_1+\dots+\xi_n$.
Показать, что в бернуллиевском случае ($\mathbb{P}\{\xi_1=1\} = p$, $\mathbb{P}\{\xi_1 = 0\}=1-p$) 
\begin{enumerate}
\item  при  $p<x<1$
\begin{equation*}
\lim\frac{1}{n}\ln\mathbb{P}\{S_n\geq nx\} = -H(x),
\end{equation*}
\begin{equation*}
H(x) = x\ln \frac{x}{p}+(1-x)\ln\frac{1-x}{1-p};
\end{equation*}
\item при $x_n = n(x-p)$ и при  $p<x<1$
\begin{equation*}
\mathbb{P}\{S_n\geq np+x_n\}=\exp\biggl\{-nH\biggl(p+\frac{x_n}{n}\biggr)(1+o(1))\biggr\};
\end{equation*}
\item при $x_n = a_n\sqrt{np(1-p)}$ c $a_n\to\infty$, $\frac{a_n}{\sqrt{n}}\to 0$
\begin{equation*}
\mathbb{P}\{S_n\geq np+x_n\}  = \exp\biggl\{-\frac{x_n^2}{2np(1-p)}(1+o(1))\biggr\}.
\end{equation*}

\item Обобщите предыдущие три пункта на случай, когда вместо $\frac{S_n}{n}$ под знаком вероятности будет стоять $\underset{k\geq n}{\sup}\frac{S_k}{k}$.

\end{enumerate}
\end{problem}
\begin{ordre}
Для доказательства первого пункта используйте следующий результат (теорема Чернова). Пусть $S_n = \sum_{i=1}^n\xi_i$, где $\xi_i$, $i=1,\dots,n$ независимые одинаково распределенные простые случайные величины с 
$$
\mathbb{E}\xi_1\leq 0 \quad \text{и} \quad \mathbb{P}\{\xi_1>0\}>0
,
$$
$$
\inf_{\lambda}\phi(\lambda)=\rho,\quad 0<\rho<1, \quad \phi(\lambda) = \mathbb{E}\text{e}^{\lambda\xi_1}.
$$
Тогда
\begin{equation*}
\lim\frac{1}{n}\ln\mathbb{P}\{S_n\geq 0\} \to \ln\rho\quad \text{при~}n\to\infty.
\end{equation*}
\end{ordre}



\begin{problem} %\begin{enumerate} 
%\item 
\label{KL_EF}
а) Сравнить оценки вероятности отклонений выборочного среднего от теоретического среднего для последовательности одинаково распределенных случайных величин $\xi_1,\dots,\xi_n$ с распределением Бернулли с вероятностью успеха $p$ (не предполагая равномерную отделимость $p$ от нуля, в частности, допуская $p = \lambda/n$, где $\lambda$ постоянна), получаемые с помощью неравенства Хёфдинга, Бернштейна, Спокойного, Буске и неравенства больших уклонений из предыдущих задач.
%\item 

б) В некотором городе прошел второй тур выборов. Выбор был между двумя кандидатами $A$ и $B$ (графы <<против всех>> на этих выборах не было). 
Сколько человек надо опросить на выходе с избирательных участков, чтобы исходя из ответов можно было определить долю проголосовавших 
за кандидата $A$ с точностью $5\%$ и с вероятностью не меньшей $0.99$. %Считайте, что исходя из голосования в первом туре, известно, 
%что каждый из кандидатов наберет не меньше $30\%$ голосов избирателей. 
%
Какой способ решения является более точным: с помощью использования центральной предельной теоремы и неравенства Берри-Эссеена или полученный с использованием неравенств концентрации меры, больших уклонений (см. задачи выше в разделе, неравенства Спокойного, Буске, Бернштейна)?
%\end{enumerate}
\end{problem}
\begin{remark}
{\it Неравенство О. Буске.}
Пусть $X_1,\dots,X_n$ -- последовательность независимых случайных величин с распределением $\PR$, принимающих значения из полного сепарабельного метрического пространства $\mathcal{X}$. Пусть $Z = f(X_1,\dots,X_n)$, где $f:\mathcal{X}^n\to \mathbb{R}$ измеримая функция, обозначим $Z_i = f(X_1,\dots,X_{i-1},X_{i+1},\dots,X_n)$. 

Пусть $Z$ удовлетворяет
$$
\sum_{i=1}^k(Z-Z_k)\leq Z
$$
и существуют такие случайные величины $Y_i$, что 
$$
Y_i\leq Z-Z_i\leq 1,\quad Y_i\leq a\,\, \text{п.н.}
$$
для некоторого $a>0$ и $\mathbb{E}_iY_i \geq 0$, где $\mathbb{E}_iX = \mathbb{E}_i[X|X_1,\dots,X_{i-1},X_{i+1},X_n]$.
Также пусть существует $\sigma,$ такое что п.н. $\sigma^2\geq \frac 1n \mathbb{E}_i Y_i^2$.


%Пусть $\mathcal{F}$ -- счетное семейство функций из $\mathcal{X}$ в $\mathbb{R}$. Предположим, что функции $f$ из $\mathcal{F}$ являются $\PR$-измеримыми, интегрируемыми в квадрате и  удовлетворяют $\mathbb{E}[f(X)]=0$. Если $\sup_{f\in\mathcal{F}}\mathrm{ess}\sup f(x)\leq 1$ тогда обозначим
%\begin{equation*}
%Z = \sup_{f\in\mathcal{F}}\sum_{i=1}^n f(X_i),
%\end{equation*}
%если же $\sup_{f\in\mathcal{F}}\|f\|_{\infty}\leq 1$, то
%\begin{equation*}
%Z = \sup_{f\in\mathcal{F}}\biggl|\sum_{i=1}^n f(X_i)\biggr|.
%\end{equation*}
Тогда для всех $x\geq 0 $ верно
\begin{equation*}
\PR[Z\geq \mathbb{E}[Z]+x]\leq \exp\biggl(-vh\biggl(\frac{x}{v}\biggr)\biggr)
\end{equation*}
при    $v = n\sigma^2+ (1+a)\mathbb{E}[Z]$ и $h(x)=(1+x)\ln(1+x)-x$, а также
\begin{equation*}
\PR\biggl[Z\geq \mathbb{E}[Z]+\sqrt{2vx}+\frac{x}{3}\biggr]\leq \exp(-x).
\end{equation*}

\medskip

См. также книгу Stéphane Boucheron, Gábor Lugosi, Pascal Massart <<Concentration Inequalities: A Nonasymptotic Theory of Independence>>, Oxford University Press, 2013.

\medskip

{\it Неравенство В.\,Г. Cпокойного.}  
Пусть $Y_i$ одинаково распределенные случайные величины с распределением $\PR_{u^{*}}$ которое принадлежит экспоненциальному семейству следующего вида
\begin{equation*}
p_{\nu}(y) = p(y)\exp\{y\nu-d(\nu)\},
\end{equation*}
где $d(\nu)$ заданная выпуклая функция на множестве параметров $\Theta\subset \mathbb{R}$, $p(y)$ ~--- неотрицательная функция на множестве значений случайной величины, $d(\nu)$ дважды непрерывно дифференцируема и для всех $u$ верно $d^{\prime\prime}(\nu)>0$.
Тогда для всех $x>0$ верно следующее неравенство
\begin{equation*}
\PR_{v^{*}}(L(\tilde{v},v^{*})>x) = \PR_{v^{*}}\bigl\{n\mathcal{KL}(\tilde{\nu},\nu^{*})> x\bigr\}\leq 2\exp(-x),
\end{equation*}
где $\mathcal{KL}(\nu,\nu^{*})$~--- расстояние Кульбака--Лейблера (см. также задачу \ref{sec:them_modeling} раздела \ref{bayes})
\begin{equation*}
\mathcal{KL}(\nu,\nu^{*}) = \int \ln\frac{d\PR_{\nu}(y)}{d\PR_{\nu^{*}}}d\PR_{\nu}(y),
\end{equation*} 
где $L(\nu,\nu^{*})$~--- $\log$-отношение правдоподобий моделей с параметрами $\nu$ и $\nu^{*}$, которое определяется следующим образом:
\begin{equation*}
L(\nu,\nu^{*}) = L(\nu)-L(\nu^{*}),
\end{equation*}
где $L(\nu)$ ~---  $\log$-правдоподобие модели с параметрами $u$ 
\begin{equation*}
L(\nu) = n^{-1}\sum_{i=1}^n\log f_\nu(Y_i),
\end{equation*}
здесь в случае непрерывного распределения $f_\nu(Y_i)$~--- плотность распределения  случайной величины $Y_i$, в случае дискретного распределения  вероятность получить наблюдаемое $Y_i$; а 
$\tilde{\nu}$ ~--- оценка параметров методом максимального правдоподобия, т.е.
\begin{equation*}
\tilde{\nu} = \mathrm{arg}\max_{\nu\in \Theta}L(\nu).
\end{equation*}
%\begin{equation*}
%\mathcal{K}(\nu,\nu*) = \int
%\end{equation*}

\medskip

\textbf{Pасстояние Кульбака--Лейблер}а $\mathcal{KL}(P \Vert Q)$ характеризует ``вложенность'' распределения $P$ в $Q$.
\imgh{110mm}{voron_kl}{Иллюстрация свойства ``вложенность'' метрики $\mathcal{KL}$.}

\textit{Теорема Берри--Эссеена. (В.В. Сенатов)} 
\label{sec:BerryEssen}
 Пусть $\xi_1, \xi_2\dots$ независимые одинаково распределенные с.в., причем $\mathbb{E}\xi_i = m$
 $\mu^3={\mathbb E}|\xi_i - {\mathbb E}\xi_i|^3<\infty$, $\sigma^2=\mathbb D \xi_i$.
Близость с.в. $\frac{\sum_{i=1}^{n}\xi_i-nm}{\sigma\sqrt{n}}$ к стандартной нормально распределенной с.в. (согласно ц.п.т.) в смысле 
близости их функций распределения определяется неравенством Берри--Эссена 
$$
\sup\limits_x \left| {\mathbb P}\Bigl( \frac{\sum_{i=1}^{n}\xi_i-nm}{\sigma\sqrt{n}}<x \Bigr) - \Phi(x) 
\right| \le \frac{C_0 \mu^3}{\sigma^3 \sqrt{N}} , 
$$
где $0.4<C_0<0.7056$,
$\Phi(x)=\int_{-\infty}^x \frac{e^{-t^2/2}}{\sqrt{2\pi}}\, dt$. 

Неравенство Берри-Эссеена дает неулучшаемый в общем случае результат. 

Теперь пусть $X_1, X_2\dots$ ~--- независимые случайные величины, имеющие одно и то же решетчатое распределение с шагом $h>0$. Очевидно, что распределение их суммы будет решетчатым с тем же шагом $h$  а функция распределения нормированной суммы
\begin{equation*}
\frac{X_1+\dots+X_n- na}{\sigma \sqrt{n}} 
\end{equation*}
будет решетчатым с шагом $h_n = h/(\sigma\sqrt{n})$. Обозначим решетку, на которой сосредоточено распределение нормированной суммы, через $D_n$. Ясно, что на полуинтервале $[-1,1)$ находится не более $2/h_n$ точек из решетки $D_n$. В силу центральной предельной теоремы для эмпирического распределения верно
\begin{equation*} 
F_n(1)-F_n(-1)\to \Phi (1)-\Phi (-1) = 0.6826\dots
\end{equation*}
при $n\to\infty$, поэтому, начиная   с некоторого  $n$, сумма скачков  $F_n$ на полуинтервале $[-1,1)$ будет не меньше $0.5$. Отсюда сразу следует, что при таких $n$ максимальный скачок будет не меньше $0.25h/(\sigma\sqrt{n})$. Так как нормальная функция распределения $\Phi$ непрерывна, а приблизить разрывную функцию $F_n$ непрерывной функцией с точностью, превосходящей половину максимального скачка, невозможно, то 
\begin{equation*}
\sup_x|F_n(x)-\Phi(x)| \geq 0.125 \frac{h}{\sigma \sqrt{n}}.
\end{equation*}
Эти рассуждения показывают, что порядок по $n$ оценки теоремы Берри-Эссеена является правильным.
См. В.В. Сенатов Центральная предельная теорема. Точность аппроксимации и асимптотическое разложение. М. Книжный дом ЛИБРОКОМ, 2009.
\end{remark}


\begin{problem} [В.Г. Спокойный]
\label{sec:spokoiny}
Пусть  $\xi$ -- стандартный нормальный вектор в $\mathbb{R}^p$. Тогда для любого $u>0$ выполнено 
\begin{equation*}
\PR(\|\xi\|^2 >p+u)\leq \exp\bigl\{-(p/2)\psi(u/p)\bigr\},
\end{equation*}
где
\begin{equation*}
\psi(t) = t-\ln(1+t).
\end{equation*}
Пусть $\psi ^{-1}(\cdot)$ обратная функция к $\psi (\cdot)$. 
\begin{enumerate}
\item
Покажите, что для любого $x$ верно 
\begin{equation*}
\PR(\|\xi\|^2>p+\psi^{-1}(2x/p))\leq\exp (-x).
\end{equation*}
И, в частности, при $\kappa = 6.6$
\begin{equation*}
\PR(\|\xi\|^2>p+ \max (\sqrt{\kappa xp}, \kappa x))\leq \exp(-x).
\end{equation*}
Можно ли уменьшить константу $\kappa$?
\item 
Обобщите результаты предыдущего пункта на случай, если компоненты вектора являются независимыми субгауссовскими случайными величинами с параметром $C$, то есть для любого $\lambda>0$, $i=1,\dots,p$ выполнено
\begin{equation*}
\mathbb{E}\exp(\xi_i\lambda)\leq \exp(C\lambda^2/2).
\end{equation*}

\end{enumerate}
\end{problem}

\begin{ordre}
Показать, что 
\begin{equation*}
\ln\mathbb{E}\exp(\mu\|\xi\|^2/2) = -0.5p\ln(1-\mu).
\end{equation*}
Из неравенства Чернова получите
\begin{equation*}
\PR(\|\xi\|^2>p+u)\leq\exp \bigl\{-\mu(p+u)/2-(p/2)\ln(1-\mu)\bigr\}.
\end{equation*}
Минимизируйте правую часть по $\mu$. Затем используйте $x-\ln(1+x)\geq a_0x^2$ при $x\leq 1$ и $x-\ln(1+x)\geq a_0x$ при $x>1$  и $a_0=1-\ln 2\geq 0.3$.
\end{ordre}
\begin{remark}
См. также Spokoiny V. Basics of Modern Parametric Statistics. 2012, http://premolab.ru/sites/default/files/stat.pdf.
\end{remark}

%\begin{problem}


%\end{problem}


%Введем ряд обозначений
%\begin{equation*}
%\PR_{n,c} = \PR\biggl\{\Bigl|\sum_{i=1}^n\xi_i -\sum_{i=1}^m m_i\Bigr|%\geq cn\biggr\},
%\end{equation*}
%\begin{equation*}
%R(\lambda)=\int_{-\infty}^{\infty}\text{e}^{\lambda x}\,d F(x),
%\end{equation*}
%\begin{equation*}
%m(\lambda) = \frac{R^{\prime}(\lambda)}{R(\lambda)}.
%\end{equation*}

%\begin{remark}
%\textbf{TODO}
%\end{remark}
%\begin{problem}[Задача о среднем функции в смысле Леви --- про концентрацию меры на сфере вокруг медианного значения "хорошей" функции]
%\end{problem}
%\subsection{Изопериметрические неравенства Талаграна(?)}
\begin{problem} 

\begin{enumerate}
\label{sec:mirrorDescent}
Пусть $\xi_0,\dots,\xi_k$ ~--- независимые одинаково распределенные случайные величины. Обозначим за $\xi_{[i]}$ совокупность случайных величин $\xi_0,\dots,\xi_{i-1}$
\item
Пусть $\Delta_i = \Delta_i(\xi_{[i]})$ неслучайная измеримая функция от $\xi_{[i]}$, такая, что 
\begin{equation*}
\mathbb{E}\left[\exp\left(\frac{\Delta_i^2}{\sigma^2}\right) \vert \: \xi_{[i-1]}\right]\leq \mathrm{\exp(1)}.
\end{equation*}
Покажите, что для любого $k\geq 0$ и $\Omega>0$ верно 
\begin{equation*}
\PR\left(\sum_{i=0}^k c_i\Delta_i^2\geq (1+\Omega) \sum_{i=0}^kc_i\sigma^2\right)\leq \exp(-\Omega),
\end{equation*}
где $c_0,\dots,c_k$ ~--- последовательность положительных коэффициентов.
\item 
Пусть $\Gamma_k$ и $\eta_k$ неслучайные измеримые функции от $\xi_{[k]}$ такие что 
\begin{itemize}
\item $\mathbb{E}[\Gamma_i|\xi_{[i-1]}]=0,$
\item $|\Gamma_i|\leq c_i\eta_i$, где $c_i$ положительная неслучайная константа,
\item $\mathbb{E}\left[\exp\left(\frac{\eta_i^2}{\sigma^2}\right)|\xi_{[i-1]}\right]\leq \mathrm{\exp(1)}$.
\end{itemize}
Покажите, что для всех $k\geq 0$ и $\Omega\geq 0$ верно
\begin{equation*}
\PR\left(\sum_{i=0}^k \Gamma_i\geq \sqrt{3\Omega}\sigma\sqrt{\sum_{i=0}^kc_i^2}\right)\leq \exp(-\Omega).
\end{equation*}
\end{enumerate}
\begin{remark} 
При доказательстве пункта а) необходимо использовать выпуклость экспоненты и линейность математического ожидания, затем неравенство Маркова.
 Пункт б) является следствием леммы 2 из статьи  Lan G., Nemirovski A. and Shapiro A. Validation analysis of mirror descent stochastic approximation method // Mathem. Programming Serie A. 2012. V.~134(2). P.~425--458.
\end{remark}

\end{problem}

\begin{problem}[Неравенства Эфрона-Стайна и МакДиармида]
\begin{enumerate}
\item 
Пусть $X_i$, $i=1,\dots,n$ произвольные независимые (не обязательно одинаково распределенные)  случайные величины, принимающие значения из $\mathcal{X}$ и пусть  $g: \mathcal{X}^n\to \mathbb{R}$ измеримая функция $n$ переменных. Покажите, что для случайной величины $Z = g(X_1,\dots,X_n)$ верно 
\begin{equation*}
\Var(Z) \leq \sum_{i=1}^n \mathbb{E}\big[ (Z-\mathbb{E}_iZ)^2\bigl],
\end{equation*}
где $\mathbb{E}_iZ = \mathbb{E}[Z|X_1,\dots,X_{i-1},X_{i+1},\dots,X_n]$.
\item Неравенство Эфрона-Стайна. Пусть $X'_1,\dots,X'_n$ ~--- независимые копии $X_1,\dots,X_n$ и 
\begin{equation*}
Z'_i = g(X_1,\dots, X'_i,\dots,X_n).
\end{equation*}
Покажите, что верно неравенство 
\begin{equation*}
\Var(Z)\leq \frac{1}{2}\sum_{i=1}^{n}\mathbb{E}[(Z-Z'_i)^2].
\end{equation*}
\item Неравенство Эфрона-Стайна в случае функций с ограниченными  разностями.
Функция $g: \mathcal{X}^n\to \mathbb{R}$ является функцией с ограниченными разностями, если для некоторых $c_1,\dots,c_n$ выполнено 
\begin{equation*}
\begin{split}
\sup_{x_1,\dots,x_n;\, x'_i\in\mathcal{X}} |g(x_1,\dots,x_n)-g(x_1,\dots,x_{i-1},x'_i,x_{i+1},\dots,x_n)|\leq c_i,\\
\quad 1\leq i\leq n.
\end{split}
\end{equation*}
Выпишите неравенство Эфрона-Стайна для случая функций с ограниченными разностями.
\item Докажите неравенство МакДиармида для функции $g$ с ограниченными разностями  (см. предыдущий пункт), а именно, что для любого $\varepsilon>0$ верно
\begin{equation*}
\mathbb{P}\Big\{\bigl| g(X_1,\dots,X_n)-\mathbb{E}g(X_1,\dots,X_n)\bigr|>\varepsilon\Bigr\}\leq 2\exp\bigg\{-\frac{2\varepsilon^2}{nC^2}\biggr\},
\end{equation*}
 где  $C^2 = \sum_{i=1}^n c_i^2$.
%\item {\textit{\textbf{Неравенство Эфрона-Стайна для отклонения эмпирической функции распределения}}}

\end{enumerate}

\end{problem}

\begin{problem}[Лемма Джонсона-Линденштаусса]
Лемма гласит, что если задан произвольный набор из $n$ точек в многомерном ($D$-мерном) евклидовом пространстве, то существует линейное вложение этих точек в $d$-мерное евклидово пространство, такое что все попарные расстояния сохраняются с точностью до множителя $1\pm\varepsilon$, если $d$ пропорционально $(\log n)/\varepsilon^2$. 

Пусть $A$~--- конечное подмножество $\mathbb{R}^D$ размерности $n$. И для  некоторого $v\geq 0$, случайные величины  $X_{i,j}$, $i=1,\dots,n$, $j=1,\dots,D$~ независимы, одинаково распределены и являются субгауссовскими с параметром $v$ (см. задачу \ref{sec:spokoiny}), причем $\mathbb{E}X_{i,j}=0$, $\mathbb{E}X^2_{i,j}=1$.
При заданном $\varepsilon\in(0,1)$ отображение $f:\mathbb{R}^D\to R^{d}$ называется $\epsilon$-изометрией на $A$ если для каждой пары $a,a'\in A$ выполняется 
$$
(1-\varepsilon)\|a-a'\|^2\leq \|f(a)-f(a')\|^2\leq (1+\varepsilon)\|a-a'\|^2.
$$

Пусть $d\geq 32v\varepsilon^{-2}\log(n/\sqrt{\delta})$, где $\delta\in(0,1)$. Покажите, что тогда с вероятностью не меньшей $1-\delta$, отображение $W: \mathbb{R}^D\to \mathbb{R}^d$, где $W_i(\alpha) =  \frac{1}{\sqrt{d}}\sum_{j=1}^D \alpha_j X_{i,j}$ для всех $\alpha\in \mathbb{R}^D$, $i\in\{1,\dots,d\}$, является $\varepsilon$-изометрией на $A$.
\end{problem}
\begin{remark}
Замечательным фактом является то, что результат не зависит от размерности $D$, которая может быть даже бесконечной!
\end{remark}
\begin{ordre}
Идея доказательства заключается в использовании специальной случайной линейной вектор-функции $W(\alpha)$ и проверке ее на $\varepsilon$-изометрию.
Основные шаги доказательства
\begin{enumerate}
\item Проверьте, что $\mathbb{E}[\|W(\alpha)\|^2]= \|\alpha\|^2,$ где $\alpha\in \mathbb{R}^{D}$.
\item Убедитесь, что доказательство того, что $W$ является $\varepsilon$-изометрией эквивалентно тому, что c вероятностью не меньшей $1-\delta$
$$
\sup_{\alpha\in T}\bigl|\|W(\alpha)\|^2-1|\leq\varepsilon,
$$
где  $T$~---подмножество единичной сферы $S$ в $\mathbb{R}^D$ следующего вида
$$
T = \biggl\{ \frac{a-a'}{\|a-a'\|}:a,a'\in A, a\not = a'\biggr\}.
$$
\item Для того, чтобы это показать, докажите, что $\sqrt{d}W_i(\alpha)$~--- субгауссовская случайная величина (см. задачу \ref{sec:spokoiny}) с параметром $v$. 
\item Воспользуйтесь следующим фактом (см. теорему 2.1 из Boucheron S., Lugosi G., Massart P. Concentration Inequalities: A Nonasymptotic Theory of Independence, Oxford University Press, 2013): если случайная величина $X$ суб-гауссовская с параметром $4C$, тогда $\mathbb{E}[X^{2q}]\leq q!C^q$ для $q>1$.
Получите, что при $q\geq 2$
$$
\mathbb{E}[W_i(\alpha)^{2q}]\leq \frac{q!}{2}(4v)^q.
$$
\item С помощью неравенства Бернштейна (задача \ref{bernstain})  получите 
$$
\mathbf{P}\left\{\sup_{\alpha\in T}|\|W(\alpha)\|^2-1|\geq 4\sqrt{\frac{v\log(n/\sqrt{\delta})}{d}}+\frac{8v\log(n/\sqrt{\delta})}{d} \right\}\leq\delta.
$$
\item Подставьте в правую часть неравенства в выражении вероятности условие на $d$ из формулировки леммы и получите утверждение леммы.

\end{enumerate}
\end{ordre}
\begin{problem}[Теорема Дворецкого]
См. также формулировку задачи \ref{sec:dvor} раздела \ref{geom}. 
Для каждого натурального $k$ и любого $\varepsilon>0$ найдется такое $n$, что всякое $n$-мерное нормированное пространство $X$ имеет $k$-мерное подпространство, расстояние от которого до $l_2^k$ по метрике Банаха-Мазура не превосходит $1+\varepsilon$, то есть можно найти векторы $x_1,\dots,x_k\in X$ такие, что 
$$
\left(\sum_{i=1}{k}|a_i|^2\right)^{1/2}\leq \|\sum_{i=1}^ka_ix_i\|_2\leq (1+\varepsilon)\left(\sum_{i=1}^k|a_i|^2\right)^{1/2} 
$$
для любой последовательности скаляров $a_1,\dots,a_k$.
\end{problem}

\begin{ordre}
Подход состоит в том, чтобы выбрать $k$-мерное подпространство $X$ случайным образом. Преред этим надо выбрать подходящую вероятностную меру. Что может быть сделано с помощью теоремы Фрица Джона. Последняя уверждает, что если существует базис $x_1,\dots,x_n$ пространства $X$, который не слишком далек от ортонормированного в смысле, что 
$$
\left(\sum_{i=1}^n|a_i|^2\right)^{1/2}\leq \left\|\sum_{i=1}^n a_ix_i\right\|_2\leq \sqrt{n}\left(\sum_{i=1}^n|a_i|^2\right)^{1/2}
$$
для всякой последовательности скаляров $a_1,\dots,a_n$. Тогда берется естественная мера грассманиана $G_{n,k}$ относительно этого базиса. Также необходимо следствие неравенства Леви: пусть $f:\,S^n\to\mathbb{R}$~--- функция со средним значением $M$ и пусть $A\in S^n$~--- множество всех точек $x$, для которых $f(x)\leq M$, тогда вероятность того, что случайно выбранная точка $S^n$ удалена от $A$ более, чем на $\varepsilon$ не превосходит $\sqrt{\pi/2}\exp(-\varepsilon^2n/2)$. Заменив $f$ на $-f$ найдем также, что почти каждая точка $y$ близка $x$ с $f(x)\geq M$. Положим теперь $f(a_1,\dots,a_m) = \|\sum_{i=1}^n a_ix_i\|_2$. Поскольку $f$ в достаточной мере непрерывна и почти каждая точка $y$ близка некоторой точке $x\in A$, заключаем, что $f(y)$ не намного больше $M$. Точно так же $f(y)$ не намного меньше $M$ для большинства точек $y$. 

См. В.Д. Мильман, Новое доказательство теоремы А. Дворецкого о сечениях выпуклых тел, Функц. анализ, Т.5, № 4, 1971, C.28--37.
\end{ordre}

\begin{remark}
Еще одна формулировка теоремы: каждое $n$-мерное симметричное выпуклое тело имеет $k$-мерное центральное сечение, которое содержит $k$-мерный эллипсоид $B$ и содержится в $(1+\epsilon)B$, то есть само является почти эллипсоидальным.

\medskip

\textit{Соmpressed sensing}. 
%\begin{ordre}
<<Сжатие измерений>> понимается как метод экономного восстановления неизвестной функции, заданной на конечном множестве мощности $m$, то есть вектора $u\in\mathbb{R}^m$ по информации, полученной измерениями скалярных произведений $(u,\phi_j),$ $\phi_j\in \mathbb{R}^m$,  $j = 1,\dots,n,$ причем $n\ll m$. Пусть $\Phi$~--- матрица со строками $\phi_j\in\mathbb{R}^m$, $j=1,\dots,n$.
Предполагается, что о разреженности вектора известно, что $\|u\|_{0} = |\{i:\,u_i\leq 0 \}|\leq t.$
Целью является 
\begin{enumerate}
\item построение алгоритма аппроксимации функции $u$ по информации $y = ((u,\phi_1),\dots,(u,\phi_n))\in\mathbb{R}^n$, то есть 
\begin{equation*}\label{lo-pr}
\min\|u\|_{0}\text{ при условии, что } \Phi v = y;
\end{equation*}
\item построение измеряющего множества векторов $\phi_j\in\mathbb{R}^m$, $j=1,\dots,n$, то есть описание матриц $\Phi$.
\end{enumerate}

Первую задачу было предложено (D. L. Donoho, M. Elad, V. N. Temlyakov, “Stable recovery of sparse overcomplete repre- sentations in the presence of noise”, IEEE Trans. Inform. Theory, 52:1 (2006), 6–18.) решать с помощью релаксации к выпуклой задаче 
\begin{equation*}\label{l1-pr}
\min\|u\|_{1}\text{ при условии, что } \Phi v = y.
\end{equation*}

В 1977 году Кашиным было доказано, что для любой пары  $(m,n)$, где $m\geq n$, существует такое подпространство $V$ размерности большей или равной $m-n$, такое, что для любого $x\in V$:
$$
\|x\|_{2}\leq C \left(\frac{1+\log(m/n)}{n}\right)^{1/2} \|x\|_1 
$$
(см. Кашин Б.С., Изв. АН СССР, серия матем., Т.41, 1977, 334–351). 
См. также Кашин Б.С., Темляков В.Н. Замечание о задаче сжатого измерения // Математические заметки. 2007. Т. 82, № 6, C. 829–837, E.J. Candes, T. Tao, “Decoding by linear programming”, IEEE Trans. Inform. Theory, Vol.51, № 12, 2005.

\end{remark}




%\end{ordre}


