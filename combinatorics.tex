\section{Вероятностный метод в комбинаторике}
\label{combinatorics}

\begin{problem}
Поверхность некоторой шарообразной планеты состоит из океана и суши (множество мелких островков). Суша занимает больше половины 
площади планеты. Также известно, что суша есть множество, принадлежащее борелевской  $\sigma$-алгебре на сфере. На планету хочет 
совершить посадку космический корабль, сконструированный так, что концы всех шести его ножек лежат на поверхности планеты. 
Посадка окажется успешной, если не меньше четырех ножек из шести окажутся на суши. Возможна ли успешная посадка корабля на планету?
\end{problem}

\begin{ordre}
Введем индикаторную функцию для одной посадки
\[ \xi_i = 
\begin{cases}
1, & i\text{-я ножка оказалась на суше,}\\
0, & \text{иначе.}
\end{cases}
\]
Тогда число ножек, оказавшихся на суше есть $\xi = \sum \limits_{i=1}^6 \xi_i$.
Покажите, что   $ \Exp \xi > 3$ (усреднение берется по всем возможным посадкам). Значит существует посадка, для которой   $\xi  > 3$, то есть успешная.
\end{ordre}

\begin{problem} 
Пусть $n\ge 2k$ и семейство $F$ является пересекающимся семейством $k$-элементных подмножеств множества $\left\{0,\ldots ,n-1\right\}$, то есть для любых двух множеств $A,B\in F$ выполняется условие $A\cap B\ne \emptyset $. Найдите с помощью вероятностного метода верхнюю оценку на размер семейства $F$ (а именно, покажите, что $|F|\le C_{n-1}^{k-1} $). Покажите, что эта оценка не улучшаемая.
\end{problem}

\begin{ordre} 
Семейство $F$ может содержать не более $k$ множеств вида $A_{s} =\left\{s,s+1,\ldots ,s+k-1\right\}$ (сумма берется по модулю $n$), $0\le s\le n-1$.


Пусть $\sigma $ -- случайная перестановка на множестве $\left\{0,\ldots ,n-1\right\}$ и $i$ -- случайное число из множества $\left\{0,\ldots ,n-1\right\}$. Пусть $A=\left\{\sigma (i),\sigma (i+1),\ldots ,\sigma (i+k-1)\right\}$ (сумма берется по модулю~$n$). Покажите, что с одной стороны (согласно доказанному выше утверждению) $\PR\left[A\in F\right]\le \frac{k}{n} $, с другой стороны (с учетом равновероятности выбора $A$ из всех $k$-множеств) $\PR\left[A\in F\right]=\frac{|F|}{C_{n}^{k} } $.


\end{ordre} 
\begin{remark}
К этой и последующим задачам этого раздела можно рекомендовать книгу \cite{15}.
\end{remark}



\begin{problem}[Задача Рамсея]  Докажите, что для произвольного графа $G=(V,E)$ всегда можно 
раскрасить вершины в два цвета таким образом, чтобы не менее половины рёбер 
были ``разноцветными'', то есть соединяли вершины разного цвета.
\end{problem}
\begin{ordre}
Вычислите математическое ожидание числа ``разноцветных'' ребер для случайной раскраски вершин графа в два цвета.
\end{ordre}
\begin{remark}
О применении вероятностного подхода к комбинаторике и теории графов рекомендуется посмотреть также книги

Айгнер М., Циглер Г. "Доказательства из Книги. Лучшие доказательства со времен Евклида до наших дней". -- М.: Мир. -- 2006. -- 256 с.

Эссе  Gowers W. T. "The Two Cultures of Mathematics" в сборнике статей под ред. Д.В.Аносова и А.Н.Паршина. М.: Фазис, 2005.
\end{remark}
\begin{comment}
ЛЕНА, К ЭТОЙ ЗАДАЧЕ СТОИТ ДАТЬ УКАЗАНИЕ С ИДЕЕЙ РЕШЕНИЯ из Gowers'a
\end{comment}


\begin{problem}
На турнир приехало $n$ игроков. Каждая пара игроков, согласно регламенту турнира, должна провести одну встречу (ничьих быть не может). Пусть 
$$
C_n^k\cdot (1-2^{-k})^{n-k}<1 . 
$$
Докажите, что тогда игроки могли сыграть так, что для каждого множества из $k$ игроков найдется игрок, который побеждает их всех. 

\end{problem}

\begin{ordre}
Введем $A_K$ --- событие, состоящее в том, что не существует игрока, побеждающего всех игроков из множества $K$. 
Докажите, что 
$$
{\mathbb P}\bigl(\bigcup\limits_{K\subset\{1,..,n\},|K|=k} A_K \bigr)\leqslant C_n^k\cdot (1-2^{-k})^{n-k} . 
$$

\end{ordre}




\begin{problem}
Рассмотрим матрицу $n\times n$, составленную из лампочек, каждая из которых либо включена $(a_{ij}=1)$, либо выключена $(a_{ij}=-1)$. 
Предположим, что для каждой строки и каждого столбца имеется переключатель, поворот которого ($x_i=-1$ для строки $i$ и 
$y_j=-1$ для столбца $j$) переключает все лампочки в соответствующей линии: с <<вкл.>> на <<выкл.>> и с <<выкл.>> на <<вкл.>>. 
Тогда для любой начальной конфигурации лампочек можно установить такое положение переключателей, что разность между числом включенных и 
выключенных лампочек будет не меньше $(\sqrt{2/\pi}+o(1))n^{3/2}$. 
\end{problem}

\begin{ordre}
Рассмотрите  переключатель по столбцам как случайную величину, принимающую с равной вероятностью значения $1$, $-1$. Каждому переключателю по столбцам необходимо подобрать переключатель по строкам, максимизирующий разность включенных и 
выключенных лампочек. Распределение данной разности можно оценить при помощи ц.п.т.       
\end{ordre}


\begin{problem}
Назовем \textit{турниром} ориентированный граф $T=(V,E)$ такой, что $(x,x)\notin E$ для любой вершины $x\in V$, а для любых двух различных вершин $x\ne y$, $x,y\in V$ либо $(x,y)\in E$, либо $(y,x)\in E$. Множество вершин назовем игроками, каждая пара игроков ровно один раз встречаются на матче, если игрок $x$ выигрывает у игрока $y$, то $(x,y)\in E$. Гамильтоновым путем графа назовем перестановку вершин $(x_{1} ,x_{2} ,\ldots ,x_{n} )$, что для всех $i$ игрок $x_{i} $ выигрывает у $x_{i+1} $. Несложно показать, что любой турнир содержит гамильтонов путь. Покажите, что найдется такой турнир на $n$ вершинах, для которого число гамильтоновых путей не меньше чем $n!/2^{n-1}$.
\end{problem}

\begin{ordre}

Рассмотрите случайный турнир (направление каждого ребра выбирается независимо от других с вероятностью $1/2$). Пусть $X$ -- число гамильтоновых путей в случайном турнире. Для каждой перестановки $\pi $ обозначим за $X_{\pi } $ индикаторную с.в. события, что гамильтонов путь соответствующей этой перестановке содержится в случайном турнире. Представьте $X$ в виде суммы таких индикаторных с.в. и, воспользовавшись линейностью математического ожидания, получите, что $\mathbb E X=n!/2^{n-1}$.
\end{ordre}


\begin{problem}
Дано $k$ перестановок натуральных чисел от 1 до $n$, $n>100$. Оказалось, что этот набор перестановок -- минимальный (по количеству перестановок), обладающий следующим свойством: для любых десяти чисел от 1 до $n$ любую их перестановку можно получить вычеркиванием оcтальных чисел из одной из данных. Докажите, что $\ln n \leq k \leq 10^{100} \ln n$
\end{problem}

\begin{remark}
Эта задача, а также последующие девять задачи от Федора Петрова (ПОМИ РАН).
\end{remark}

\begin{problem}
Докажите, что числа от 1 до $2^n$ можно покрасить в два цвета так, чтобы не было арифметической прогрессии длины $2n$ одного цвета.
\end{problem}

\begin{problem}
На столе лежат $n$ монет орлами вверх. Каждую минуту Вася равновероятно выбирает одну из монет  и переворачивает ее. Докажите, что вероятность того, что через $k$ минут все монет
будут лежать решками вверх, не превосходит $2^{1-n}$.
\end{problem}


\begin{problem}
\begin{enumerate}
\item В алфавите племени УАУАУА только две буквы,
причем никакое слово их языка не является началом другого слова.
Докажите, что $\sum N_i2^{-i}\leq 1$,
где $N_i$ --- количество слов длины $i$ в этом языке.
\item В алфавите племени ОЕЕ только две буквы. Люди
этого племени записывают
предложения без пробелов и это никогда
не приводит к двусмысленности (то есть для
любой конечной последовательности букв есть не более одного способа
разбить их на слова). Докажите, что
$\sum N_i2^{-i}\leq 1$,
где $N_i$ --- количество слов длины $i$ в этом языке.
\end{enumerate}
\end{problem}

\begin{problem}
В таблице $n\times n$ расставлены различные числа.
Докажите, что можно так переставить ее строки, что
ни в одном столбце не будет возрастающей (сверху вниз) последовательности длины $\geq 100 \sqrt{n}$.
\end{problem}

\begin{problem}
В однокруговом волейбольном турнире участвовала тысяча команд.
Всегда ли можно выбрать 21 команду и пронумеровать их так,
чтобы в любой паре из этих команд победила та, номер которой больше?
\end{problem}

\begin{problem}
В двудольном графе меньше, чем $2^n$ вершин, и в каждой
имеется список из $n$ цветов. Докажите, что можно
покрасить каждую вершину в один из цветов ее списка
так, чтобы смежные вершины были разных цветов.
\end{problem}

\begin{problem}
\begin{enumerate}
\item В компании из $n$ человек некоторые пары
дружат, а некоторые другие враждуют, при этом у каждого не более
пяти врагов. Известно, что в любом множестве людей, среди которых нет пар врагов, имеется не более чем
$k$ пар друзей. Докажите, что общее количество пар друзей не превосходит $2^{2011}k$.
\item То же, если вражда (в отличие от дружбы) -- не обязательно симметричное отношение: каждый человек неприязненно относится не более чем к пятерым, и в любом множестве людей, среди которых никто ни к кому не относится неприязненно, не не более чем $k$ пар друзей.
\end{enumerate}
\end{problem}

\begin{problem}
На множестве положительных
чисел задано некоторое вероятностное распределение $X$.
Из одной и той же точки плоскости начинают прыгать две лягушки, каждая
из которых выбирает длину своего прыжка случайно
согласно распределению $X$, а направление случайно
и равномерно. (Направление и длина каждого прыжка независимы,
так же независимы разные прыжки и поведения лягушек).
Первая лягушка сделала $n$ прыжков, а вторая $m$
(где $m,n>0$ и $m+n>2$).
Докажите, что вероятность того, что первая лягушка дальше
от исходной точки, чем вторая, равна $n/(n+m)$.
\end{problem}


\begin{problem}
Несколько мальчиков ``раскидывают на морского'', кому водить в игре. Для этого каждый из них одновременно с другими
``выбрасывает'' на пальцах число от 0 до 5. Числа складываются и сумма отсчитывается по кругу, начиная с заранее
выбранного мальчика (ему соответствует
ноль). Водить будет тот, на ком остановится счет.
При каком числе мальчиков этот метод является справедливым, то есть
вероятность водить одинакова у всех мальчиков?
\end{problem}




\begin{problem}
\label{triangles}
Рассматривается случайный граф $G(n,p)$ (модель Эрдеша--Реньи). Случайная величина $X$ равна числу треугольников в графе. Покажите, что: 

\begin{enumerate}

\item Если $p(n) \ll  n^{-1} $, то граф $G$ почти всегда свободен от треугольников, то есть $\mathop{\lim }\limits_{n\to \infty } \PR (
X > 0)= 0$;

\item Если $p(n)\gg n^{-1} $, то граф $G$ почти всегда содержит треугольник, то есть $\mathop{\lim }\limits_{n\to \infty } \PR (X = 0)=0$.
\end{enumerate}

Говорят, что пороговая функция свойства ``граф  свободен от треугольников'' графа $G(n,p)$ равна $n^{-1} $.

\end{problem}

\begin{ordre}

Для случайной величины $X \geq 0$ справедливы неравенства 

\begin{enumerate}

\item $\PR( X>0) \le \Exp X$, 

\item $\PR( X=0) \le \PR( |X- \Exp X|\ge \Exp X)$. 

\end{enumerate}

\noindent Введем событие $B_{S}$ -- ``$S$ является треугольником''. Тогда 
\[
X=\sum _{|S|=3} \I [B_{S}] = \sum _{|S|=3} X_S,
\]
где $ X_S = \I [B_{S}]$.
Дисперсию зависимых индикаторов предлагается оценивать неравенством  
\[
\Var X\leq  \Exp X+\sum \cov\left( X_{S_1} , X_{S_2} \right), 
\] 
здесь суммирование ведется по всем упорядоченным зависимым парам различных трехэлементных множеств. 

\[
\sum \cov\left( X_{S_1} , X_{S_2} \right) \leq \sum \PR \left( B_{S_1} , B_{S_2} \right)  =
\]\[
\sum _{S_1} \PR ( B_{S_1} )  \sum  \PR (B_{S_2} | B_{S_1} )  = \Exp X\sum _{} \cdot \PR (B_{S_2} | B_{S_1} ).
\] 


\end{ordre}

\begin{problem} [Парадигма Пуассона] 
Рассматривается случайный граф $G\left(n,\frac{c}{n} \right)$ (модель Эрдеша--Реньи), где $c$ -- некоторая константа. С помощью неравенства Янсона (см. замечание) покажите, что случайная величина $X = X(n)$, равная числу треугольников в графе, имеет почти пуассоновское распределение с параметром $\mu=\lim_{n\to\infty}\Exp X = \frac{c^3}{6}$, в частности 
\[ 
\lim_{n\to\infty}\PR (X=0) \sim e^{ - \mu} .
\]

\end{problem}

\begin{remark}
Согласно предыдущей задаче, пороговая функция свойства "граф свободен от треугольников"  равна $n^{-1}$. С учетом обозначений, введенных в предыдущей задаче, \textit{неравенство Янсона} имеет вид:
\[
\prod _{|S|=3} \PR \left( \overline{B}_{S} \right)  \le \PR \left(\mathop{\wedge }\limits_{|S|=3} \overline{B}_{S} \right) \leq e^{-\mu +\frac{\Delta }{2} },
\] 
где $\mu =\sum _{|S|=3} \PR( B_{S} )  $, $\Delta =\sum _{|S\cap T|=2} \PR( B_{S} B_{T} )$.

Заметим, что левое неравенство переходит в равенство для взаимно независимых событий $B_S$. При этом 
$$\prod _{|S|=3} \PR \left( \overline{B}_{S} \right) = \left (1 - \left( \frac{c}{n} \right)^3  \right )^{C_n^3}\to e^{-\frac{c^3}{6}}.$$

На самом деле события $B_S$, $B_T$ зависимы, если $|S\cap T|=2$. Неравенство Янсона дает поправку для "почти независимых" событий через $\Delta=C_n^4C_4^2 \left( \frac {c}{n}\right)^5 = o(1)$. Таким образом, с.в. $X$ -- сумма большого числа индикаторов "почти независимых" событий, имеет почти пуассоновское распределение. 

Более детальный подход к парадигме Пуассона дает метод "решета Бруна" (см. \cite{15}):
$$
\lim_{n\to\infty}\PR (X=k) = \frac{\mu^k}{k!}e^{ -\mu}.
$$
\end{remark}

\begin{problem} 
Покажите, что пороговая функция события: размер максимальной клики $\omega (G)$ в случайном графе $G(n,p)$ не меньше 4 -- равна $n^{-\frac{2}{3} } $.
\end{problem}

\begin{ordre}
См. задачу \ref{triangles}. 
\end{ordre}



\begin{problem}
Покажите, что для каждого целого числа $n$ найдется раскраска ребер полного графа $K_{n} $ в два цвета (синий, красный), при которой число одноцветных подграфов $K_{4} $ не превосходит $C_{n}^{4} 2^{-5} $. Предложите детерминированный алгоритм построения такой раскраски за полиномиальное от $n$ время.
\end{problem}


\begin{ordre}

Зададим весовую функцию $W(K_{n} )$ частично раскрашенного графа $K_{n} $, как $W(K_{n} )=\sum _{}w(K) $, где суммирование ведется по всем копиям $K$ графа $K_{4} $ в $K_{n} $ и вес~$w(K)$ подграфа~$K$ равен вероятности того, что копия $K$ окажется одноцветной в случае, когда все бесцветные ребра графа $K_{n} $ будут случайно и независимо раскрашены в два цвета. 

Произвольным образом упорядочим все $C_{n}^{2} $ ребер графа $K_{n}$ и создадим цикл их перебора.   Цвет очередного ребра  выбирается так, чтобы минимизировать получающийся вес, то есть при $W_{red} \leq W_{blue} $ ребро раскашивается в красный цвет, в противном случае -- в синий. Покажите, что в такой процедуре вес графа $K_{n} $ с течением времени не возрастает. 

\end{ordre}


\begin{problem}
Покажите, что можно так раскрасить в два цвета ребра полного графа с $n$ вершинами (т.е. графа (без петель), в котором любые две 
различные вершины соединены одним ребром), что любой его полный подграф с $m$ вершинами, где 
$2C_n^m (\left.1\right/2)^{C_m^2}<1$, имеет ребра разного цвета. 
\end{problem}





\begin{problem}[Концентрация хроматического числа] 
\label{azuma}

Для произвольных $n$ и $p\in (0, 1)$ покажите, что распределение случайной 
величины, равной $\chi (G)$ -- хроматическому числу графа $G(n,p)$ (случайный 
граф в модели Эрдеша--Реньи) является плотно сконцентрированным около 
среднего значения:

\[
\forall \lambda >0  \quad \PR\left\{ {\vert \chi (G)-\Exp\chi (G)\vert >\lambda 
\sqrt {n-1} } \right\}\le 2e^{-\frac{\lambda ^2}{2}}.
\]

\end{problem}


\begin{ordre}

Воспользуйтесь неравенством Азумы: для мартингальной последовательности $X_0 =c,\ldots ,X_m $, удовлетворяющей условию $\vert X_{i+1} -X_i \vert \le 1$ для всех $0\le i< m$, справедливо 
\[
\PR\left( {\vert X_m -c\vert >\lambda \sqrt m } \right)\le 
2e^{-\frac{\lambda ^2}{2}}.
\]
В качестве такого мартингала можно взять мартингал проявления 
вершин, заданный следующим образом: $X_i = \Exp\left[ {\chi (G)} \vert  G_{1:i}\right]$, $X_i$ -- условное математическое ожидание значения $\chi (G)$ при зафиксированном подграфе с вершинами $1,\ldots,i$.

\end{ordre}

\begin{problem}[Максимальный размер клики]

Для произвольного $n$ покажите, что распределение случайной величины, равной $\omega (G)$ -- максимальному размеру клики графа $G\left( {n,\frac{1}{2}} 
\right)$ (случайный граф в модели Эрдеша--Реньи) задается неравенством:
\[
\PR\left ( {\omega (G)<k} \right) < e^{-(c+o(1))\frac{n^2}{k^8}},
\]
где $c$ -- некоторая положительная константа.

\end{problem}

\begin{ordre}
 Воспользуйтесь неравенством Азумы (см. указание из задачи \ref{azuma}) для мартингала проявления ребер $X_0 ,\ldots ,X_m $ (здесь $m=C_n^2 )$, заданного следующим образом: $X_0 = \Exp \left[ {Y(G)} \right]$, $X_i $ -- условное математическое ожидание значения $Y(G)$, при 
условии, что первые $i$ ребер/пропусков фиксированы, $Y(G)$ -- максимальный размер семейства непересекающихся по ребрам $k$-клик в графе. 
\[
X_0 = \Exp\left[ {Y(G)} \right]\ge \left( {9+o(1)} 
\right)\frac{n^2}{2k^4},
\]
\[
\left\{ \omega (G)<k \right\} 
\Leftrightarrow 
\left\{ Y(G)=0 \right\}
\Leftrightarrow
\left\{ X_m =0 \right\}.
\]
Далее осталось применить неравенство Азумы для 
\[
\PR\left( {X_m =0} \right)\le \PR\left( {X_m -X_0 \le -X_0 } \right).
\]

\end{ordre}

\begin{problem}
Пусть для модели Эрдеша--Реньи случайного графа $G(n,p),\; p=n^{-\alpha }$, где $\alpha $ -- фиксированное, $\alpha > 5/6$. Тогда существует $u=u(n,p)$ такое, что почти всегда 
\[
u\le \chi (G)\le u+3.
\]
\end{problem}


\begin{ordre}
Покажите справедливость следующей технической леммы.

\begin{lemma}
Пусть $\alpha $, $c$ -- фиксированные числа, $\alpha > 5/6$. Пусть 
$p=n^{-\alpha }$. Тогда почти наверное каждые $c \sqrt n $ вершин графа 
$G(n,p)$ могут быть правильно раскрашены в три цвета.
\end{lemma}

Для доказательства леммы предположим противное. Возьмем случайный граф
$G(n,p)$, пусть $T$ -- подмножество (вершин исходного графа) минимального размера, которое нельзя правильно раскрасить в три цвета. Поскольку для всякого $x\in T$ подграф, порожденный множеством $T\backslash \{x\}$, является 3-раскрашиваемым, а подграф, порожденный $T$, не является таковым, $x$ имеет по меньшей мере трех соседей в подграфе, порожденном $T$. То есть если $\vert T\vert =t$, то подграф, порожденный множеством $T$ имеет по меньшей мере $3t/2$ ребер. Вероятность того, что существует такое $T$ с 
$t\le c\sqrt n $, не превосходит $\sum\limits_{t=4}^{c\sqrt n } {C_n^t
C_{C_t^2 }^{\frac{3t}{2}} p^{\frac{3t}{2}}} $. 
Поскольку $C_n^t \le \left( 
{\frac{ne}{t}} \right)^t$ и $C_{C_t^2 }^{\frac{3t}{2}} \le \left( 
{\frac{te}{3}} \right)^{\frac{3t}{2}}$, то 
\[
C_n^t C_{C_t^2 }^{\frac{3t}{2}} 
p^{\frac{3t}{2}}\le \left[ {\frac{ne}{t}\left( {\frac{te}{3}} 
\right)^{\frac{3}{2}}n^{-\frac{3\alpha }{2}}} \right]^t\le \left[ {c_1
n^{1-\frac{3\alpha }{2}}t^{\frac{1}{2}}} \right]^t\le \left[ {c_2 n^{-\left( 
{\frac{3\alpha }{2}-\frac{5}{4}} \right)}} \right]^t
\]
и вероятность заданного события есть $o(1)$ (что и доказывает справедливость леммы).

Далее для произвольного малого $\varepsilon >0$ выберем $u=u(n,p,\varepsilon)$ -- наименьшее целое число, удовлетворяющее неравенству
\[
\PR\left\{ {\chi (G) \leq u} \right\} > \varepsilon.
\]
Далее покажите, что с вероятностью не меньше $1-\varepsilon$ существует $u$-раскраска всех, кроме не более чем $c\sqrt n $ вершин. Для этого воспользуйтесь неравенством Азумы для мартингала проявления вершин с 
теоретико-графовой функцией $Y(G)$, равной минимальному размеру множества 
вершин $S$, для которого граф, индуцированный исходным графом $G$, но без вершин $S$, может быть правильно раскрашен в $u$ цветов:
\[
\begin{array}{l}
 \PR\left\{ {Y\le \Exp Y-\lambda \sqrt {n-1} } \right\}<e^{-\frac{\lambda ^2}{2}}, 
\\ 
 \PR\left\{ {Y\ge \Exp Y+\lambda \sqrt {n-1} } \right\}<e^{-\frac{\lambda ^2}{2}}, 
\\ 
 \end{array}
\]
где $\lambda $ удовлетворяет соотношению $e^{-\frac{\lambda ^2}{2}} < \varepsilon$.
Из определения $u$ имеем $\PR\{Y=0\} > \varepsilon$. Значит $\Exp Y\le \lambda \sqrt {n-1} $ и $\PR\left\{ {Y\ge 2\lambda \sqrt {n-1} } \right\}<\varepsilon$.

\end{ordre}

\begin{problem}[Балансировка векторов]

Пусть ${\rm B}$ -- произвольное нормированное пространство, $v_1 ,\ldots ,v_n $ -- элементы ${\rm B}$, причем $\left\| {v_i } \right\|_2\le 1$. 
Пусть $\varepsilon _1 ,\ldots ,\varepsilon _n $ -- радемахеровские случайные величины, то есть 
независимые с.в. с распределением $\PR\left\{ {\varepsilon _i =+1} 
\right\}=\PR\left\{ {\varepsilon _i =-1} \right\}=1/2$. Положим 
$Y=\left\| {\varepsilon _1 v_1 +\ldots +\varepsilon _n v_n } \right\|_2$.
Покажите справедливость неравенства для произвольного $\lambda >0$
\[
\PR\left\{ {\vert Y-\Exp Y\vert >\lambda \sqrt n } \right\}\le 2e^{-\frac{\lambda 
^2}{2}}.
\]

\end{problem}

\begin{ordre}  
Воспользуйтесь неравенством Азумы для мартингала, 
полученного последовательным проявлением $\varepsilon_i$.
\end{ordre} 




\begin{problem} 
Пусть $\omega (n)\to \infty $ произвольно медленно. Покажите, что число тех $x\in \left\{1,\ldots ,n\right\}$, для которых

\[\left|\nu (x)-\ln (\ln n)\right|>\omega (n)\sqrt{\ln (\ln n)} ,\] 
есть $o(n)$. Здесь $\nu (x)$ -- количество простых чисел $p$, делящих $x$ (без учета кратности).
\end{problem}

\begin{remark} 
Грубо, это утверждение говорит, что ``почти все'' $n$ имеют число простых делителей (без учета кратности) ``в некотором смысле близкое'' к $\ln (\ln n)$.
(См. также задачу 54 из раздела 3)
\end{remark} 

\begin{ordre} 
Пусть $x$ случайно выбирается из множества $\left\{1,\ldots ,n\right\}$. Для простого $p$ положим: 

\[
X_{p} =\left\{\begin{array}{cc} {1,} & {x \; \mbox{ делится на }   \; p ,} \\ {0,} & { x \;  \mbox{ не делится на }\; p .} \end{array}\right. 
\] 
$X=\sum X_{p}  $, где сумма ведется по всем простым $p\le M\equiv n^{0.1} $. Так как никакое $x\le n$ не может иметь более 10 простых делителей, больших $M$, то $\nu (x)-10\le X(x)\le \nu (x)$ (то есть границы больших уклонений для $X$ переходят в асимптотически равные им границы для $\nu $.

Покажите, что математическое ожидание и дисперсия случайной величины $X$ равны $\ln (\ln n)+O(n^{-0.9})$, учтя соотношение $\sum _{p\le x}1/p  =\ln \ln x$, где сумма берется по всем простым $p\le x$.

\end{ordre} 

\begin{remark} 
Справедливо также соотношение
\[
\mathop {\lim }\limits_{n\to \infty } \frac{1}{n}\left| {\left\{ {k\leq n:\;\nu \left( k \right)\ge \ln (\ln n)+\lambda \sqrt {\ln (\ln n)} } 
\right\}} \right|=\frac{1}{\sqrt {2\pi } }\int\limits_\lambda ^\infty 
{e^{-{t^2} \mathord{\left/ {\vphantom {{t^2} 2}} \right. 
\kern-\nulldelimiterspace} 2}dt} .
\]
\end{remark} 


\begin{comment}


\begin{problem}
$V=\left\{ {1,...,m} \right\}$, ${\rm M}=\left\{ {M_1 
,...,M_n } \right\}$, $M_k \subseteq V$.

$\chi :\quad V\to \left\{ {-1,1} \right\}$ (можно интерпретировать, как 
раскраску множества V в два цвета).

$\chi (M_i )=\sum\limits_{a\in M_i } {\chi (a)} $ ($\left| {\chi (M_i )} 
\right|$ отвечает за ``равномерность'' покраски множества $M_i $ в два 
цвета).

$disc({\rm M},\chi )=\mathop {\max }\limits_{i=1..n} \left| {\chi (M_i )} 
\right|$ (от слова discrepancy - уклонение) - мера того, что хотя бы один 
объект в ${\rm M}$ раскрашен ``неравномерно''.

$disc({\rm M})=\mathop {\min }\limits_\chi disc({\rm M},\chi )$(``поиск'' 
наилучшей раскраски).

Показать, что для $\forall n\;\forall m\;\forall {\rm M} \quad disc({\rm M})\le 
\sqrt {2m\ln (2n)} $. Т. е. $\exists \chi :\;disc({\rm M},\chi )\le \sqrt 
{2m\ln (2n)} $.

\end{problem}


\end{comment}




\begin{problem}
Пусть ${\cal M}=\left\{ {M_1 ,\ldots ,M_s } \right\}$ -- совокупность, 
состоящая из различных $k$-сочетаний элементов множества $\left\{ {1,\ldots 
,n} \right\}$. Назовем $S\subset \left\{ {1,\ldots ,n} \right\}$ \textit{системой общих представителей }(с.о.п.) 
для ${\cal M}$, если $S\cap M_i \ne \emptyset $ для всех $i=1,\ldots ,s$. 
Интерес представляет минимальная (по мощности) с.о.п., т.е. та с.о.п., на 
которой достигается минимум: 
\[
\tau \left( {\cal M} \right)=\min \left\{ {\left| S 
\right|:\;S-\mbox{с.о.п.}} \right\}.
\]
Ясно, что минимальная с.о.п. может быть не единственной (т.е. минимум в 
предыдущем выражении достигается не на единственной $S)$. Зафиксируем 
параметры $n,s,k$ и введем искусственно равномерную дискретную вероятностную 
меру на множестве всех совокупностей ${\cal M}$ (в силу того, что при 
фиксированных $n,s,k$ число таких совокупностей ${\cal M}$ конечно).

а) Выберем согласно введенной вероятностной мере случайную совокупность 
${\cal M}$, найдем все возможные минимальные с.о.п. для нее, пусть их 
количество равно $N({\cal M})$ (случайная величина). Найдите математическое 
ожидание $N({\cal M})$, при условии, что $\tau ({\cal M})=l$.

б) Далее будем интересоваться величиной 
\[
\zeta (n,s,k)=\mathop {\max }\limits_{\cal M} \tau ({\cal M}),
\]
где максимум берется по совокупностям ${\cal M}$ с фиксированными 
параметрами $n,s,k$.

Для получения нижней границы на значения величины $\zeta (n,s,k)$ можно 
воспользоваться вероятностным методом. Согласно введенному выше 
вероятностному пространству на множестве всех совокупностей ${\cal M}$ 
рассмотрим случайное событие $A=\left\{ {{\cal M}:\;\tau ({\cal M})\le l} 
\right\}$. Покажите, что 
\[
P(A)\le G(n,s,k,l) = \frac{C_n^l C_{C_n^k -C_{n-l}^k 
}^s }{C_{C_n^k }^s }.
\]
Если параметры $n,s,k,l$, таковы, что $G(n,s,k,l)<1$, то вероятность 
отрицания события $A$ положительна, т.е. существует такая совокупность 
${\cal M}$, для которой $\tau ({\cal M})>l$, а значит и $\zeta (n,s,k)>l$. 

\textbf{Замечание. }Если $l=l(n,s,k)\approx \frac{n}{k}\ln \frac{sk}{n}$ (в 
предположении, что $sk>n)$, то можно показать, что $G(n,s,k,l)\mathop \to 
\limits_{n\to \infty } 0$, а значит ``почти всякая'' совокупность обладает 
огромной по размеру минимальной с.о.п. (т.е. с вероятностью стремящейся к 
единице случайная совокупность ${\cal M}$ имеет $\tau ({\cal M})>l\approx 
\frac{n}{k}\ln \frac{sk}{n})$. На самом деле полученная выше нижняя оценка 
на значения величины $\zeta (n,s,k)$ асимптотически точна. Иными словами 
можно доказать следующую теорему (см. Райгородский А.М. Системы общих 
представителей в комбинаторике и их приложения в геометрии. М.: МЦНМО, 
2009): для любых $n,s,k$ справедливо неравенство
\[
\zeta (n,s,k)\le \max \left\{ {\frac{n}{k},\frac{n}{k}\ln \frac{sk}{n}} 
\right\}+\frac{n}{k}+1.
\]
\end{problem}

\begin{problem}[Локальная лемма Ловаса (ЛЛЛ)]
Орграф зависимостей $(V,D)$ для набора событий $A_{1} ,\ldots A_{t} $ определяется следующим образом:
$V=\{ 1,\ldots ,t\};$ $D$ определяется согласно правилу $(i_{1} ,k),\ldots ,(i_{s} ,k)\notin D \Leftrightarrow$ $A_{k} $ не зависит от группы событий $A_{i_{1} } ,\ldots, A_{i_{s} } $.

Пусть $D$ -- множество дуг орграфа зависимостей набора событий $A_{1} ,\ldots, A_{n} $. Пусть нашлись такие $x_{1} ,\ldots ,x_{n} \in (0,1)$, что $\forall k\in \left\{1,\ldots ,n\right\}$ выполнено неравенство
\[P[A_{k} ]\le x_{k}\cdot \prod _{i:\; (i,k)\notin D}(1-x_{i} ) .\] 
Тогда
\[P\left[\bar{A}_{1} \bigcap \ldots \bigcap \bar{A}_{n} \right]\ge (1-x_{1} )\cdot\ldots\cdot (1-x_{n} )>0.\] 
Примените ЛЛЛ для оценки диагональных чисел Рамсея (см. задачу 3 этого раздела) $R(s,s)>n$, где $n=\left\lfloor 2^{0.5s} \right\rfloor $, то есть покажите, что существует граф на $n$ вершинах, у которого нет ни клик, ни независимых множеств (н.м.) размера $s$. 
\end{problem}

\begin{ordre}
Рассмотрите случайный граф на $n$ вершинах с вероятностью проведения ребра $1/2 $. Для фиксированного подмножества $s$-вершин $U$ определите событие $A_{U} $ -- множество вершин $U$ образует либо клику, либо н.м. Для выбранной модели случайного графа $P[A_{U} ]=2^{1-C_{s}^{2} } $. Заметьте, что на событие $A_{U} $ влияют только те $A_{U'} $, для которых $\left|U\bigcap U'\right|\ge 2$, то есть на фиксированное событие $A_{U} $ влияют менее $C_{s}^{2} C_{n-2}^{s-2} $ событий.
\end{ordre}






\begin{problem}

С помощью неравенства Талаграна (см. \cite{15}) оцените вероятность того, что случайный граф 
$G\left( {n,\frac{1}{2}} \right)$ не имеет клик размера $k$.

\end{problem}

\begin{ordre} 
Заметим, что задание случайного графа $G\left( {n,\frac{1}{2}} \right)$ эквивалентно подбрасыванию симметричной монеты $C_n^2 $ раза. В качестве случайной величины $X$ возьмите максимальное число непересекающихся по ребрам $k$-клик. Проверьте, что эта теоретико-графовая 
функция удовлетворяет условию Липшица с константой 1 (добавление ребра может добавить не более одной клики в семейство непересекающихся по ребрам клик) и является проверяемой со сложностью $f(s)= s C_k^2 $. Обозначим через $M$ медиану $X$. Неравенство Талаграна дает соотношение
\[
\PR\left[ {X\le M-t\sqrt {M C_k^2 } } \right]\PR\left[ {X\ge M} \right]\le e^{-\frac{t^2}{4}}.
\]
Выбирая $t=\Theta \left( {\frac{\sqrt M }{k}} \right)$, получите оценку для 
\[
\PR\left\{ {\omega (G)<k} \right\}=\PR\left\{ {Y\le 0} \right\}.
\]
\end{ordre}

\begin{problem}[Теорема Шеннона]

Используя вероятностный метод, докажите теорему 
Шеннона. Пусть $\Sigma =\{0,1\}$ {\-} двухбуквенный алфавит, 
$p<1/2$, $\varepsilon >0$ сколь угодно мало. Существует схема 
кодирования со скоростью передачи данных, превосходящей $1-H(p)-\varepsilon 
$, и вероятностью ошибки при передаче меньшей, чем $\varepsilon $.

Под схемой кодирования подразумеваются функции кодирования $f$ и 
декодирования $g$: 
\[f:\quad \{0,1\}^k\to \{0,1\}^n; \quad g:\quad \{0,1\}^n\to 
\{0,1\}^k.\] Скорость передачи данных в такой схеме определяется отношением 
$k/n$. Пусть $e=\left( {e_1 \ldots e_n } \right)$ {\-} случайный шумовой
вектор, компоненты которого независимы и имеют распределение Бернулли с 
параметром $p$. В предположении, что случайное сообщение $x$ имеет 
равномерное на $\{0,1\}^k$ распределение, определим вероятность правильной передачи данных как $\PR\left\{ {g\left( {f(x)+e} \right)=x} \right\}$ (здесь сложение берется по $\mod {2})$.
\end{problem}

\begin{ordre}
Рекомендуется ознакомится с книгами \cite{15}, \cite{444} и А. Ромащенко, А. Румянцев, А. Шень. Заметки по теории кодирования. -- М.МЦНМО, 2011.

Для больших значений $n$, выберем значение $k=\left\lceil {n\left( {1-H(p)-\varepsilon } \right)} \right\rceil $, обеспечивающее 
нужную скорость передачи данных. Зададим $\delta >0$ такое, что $p+\delta <\frac{1}{2}$ и $H(p+\delta )<H(p)+\frac{\varepsilon }{2}$. Тогда утверждение теоремы есть применение теоремы Варшамова--Гилберта: в пространстве $\{0,1\}^n$ с метрикой Хэмминга существует $n(p+\delta )$-сеть мощности не менее, чем $2^k$. Основываясь на вероятностном методе докажите 
последнее. 

Функцию кодирования $f:\quad \{0,1\}^k\to 
\{0,1\}^n$ зададим случайно, выбирая для каждого $x$ значение $f(x)$ 
согласно равномерному распределению на $\{0,1\}^n$. Функцию декодирования определим следующим образом: $g(y)=x$, если $x\in \{0,1\}^k$ {\-} это единственный вектор такой, что $f(x)$ находится на расстоянии Хэмминга не более чем $n(p+\delta )$ от вектора $y\in \{0,1\}^n$. Если указанных векторов $x\in \{0,1\}^k$ нет или напротив больше одного, то декодирование назовем некорректным. Покажите, что вероятность некорректного декодирования 
экспоненциально мала.

Воспользуйтесь неравенством, оценивающим объем шара радиуса $an$, где 
$a<1/2$, в пространстве $\{0,1\}^n$ с метрикой Хэмминга: 
$\sum\limits_{i=0}^{na} {C_n^i } \le 2^{n(H(a)+o(1))}$, что есть следствие 
применения неравенства Чернова к вероятности того, что биномиальная 
случайная величина с параметрами $n$, $1/2$ принимает значения, превосходящие $n-na$.
\end{ordre}

\begin{problem}[Игра лжецa] 
Пусть Пол пытается угадать число $x\in \{1,\ldots ,n\}$ 
у лживой Кэрол, сопротивляющейся этому. Пол может задать $q$ вопросов вида ``Верно ли, что $x\in S?$'', где $S$ {\-} производное подмножество $\{1,\ldots ,n\}$. Вопросы задаются последовательно, $i$-ый вопрос Пола может зависеть от предыдущих ответов. Кэрол может лгать, но она не может солгать больше $k$ раз. Покажите с помощью вероятностного подхода, что при условии
\[
n>\frac{2^q}{\sum\limits_{i=0}^k {C_q^i } }
\]
существует выигрышная стратегия у Кэрол. С помощью метода дерандомизации (см. \cite{15}) опишите явную стратегию Кэрол.
\end{problem}

\begin{ordre}
Заметьте, что в связи с тем, что Кэрол сопротивляется 
правильному отгадыванию числа Полом, здесь можно считать, что ее выигрышная стратегия такова, что она в действительности не загадывает число заранее, а выбирает ответы на каждый вопрос Пола так, что все ее ответы согласуются с более чем одним $x$ (с учетом информации о том, что она может за всю игру солгать не более, чем $k$ раз). Тогда зафиксируем стратегию Пола. Пусть Кэрол играет случайно, то есть с вероятностью $\frac{1}{2}$ отвечает либо ``да'', либо ``нет'' на каждый вопрос Пола. Пусть $I_x $ {\-} индикатор того, что число $x$ согласуется с ответами Кэрол. Покажите, что $E\left[ {I_x } \right]=2^{-q}\sum_{i=0}^k {C_q^i } $. Таким образом, в силу 
линейности математического ожидания, среднее значение числа таких $x$, что согласуются с ответами Кэрол, есть $n2^{-q}\sum_{i=0}^k {C_q^i } $, что по условию больше $1$.

Для построения выигрышной стратегии Кэрол, введем вес игровой ситуации $W$, равный математическому ожиданию числа таких $x$, что согласуются с ответами Кэрол, при условии, что она играет случайно. Тогда стратегия Кэрол: на каждом шаге максимизировать вес игровой ситуации. Так как если для произвольной игровой ситуации с весом и $W$ и некоторым ходом Пола, обозначить за $W^y$ и $W^n$ веса игровой ситуации после ответа Кэрол ``да'', ``нет'' соответственно, то выполняется соотношение $W=\frac{1}{2}(W^y+W^n)$. Итак, стратегия Кэрол не дает весу уменьшиться. Если в начале игры вес был 
больше единицы, значит он будет больше единицы и в конце игры, что 
соответствует выигрышу Кэрол.

\end{ordre}




\begin{problem}
\label{hnmgraph}
Рассмотрим случайную последовательность графов $\{G_n\}$ (модель роста интернета), полученную следующим образом. Зафиксируем параметр $a>0$. Пусть $G_1$ -- граф с одной вершиной $1$ и одной петлей $(1,1)$. Далее, предположим, что граф $G_{n-1}$ уже построен. Граф $G_n$ получается путем добавления к графу $G_{n-1}$ одной вершины $n$ и одного ребра. С вероятностью $\frac{a}{an+n-1}$  это ребро будет направлено из $n$ в $n$, с вероятностью $\frac{deg_{n-1}(i)+a-1}{an+n-1}$ будет добавлено ребро $(n,i)$. Здесь $deg_{n-1}(i)$ -- степень вершины $i$ в графе $G_{n-1}$, $1 \le i \le n-1$. Нетрудно видеть, что выбор того, куда проводить следующее ребро зависит от всех предыдущих ребер. Покажите, что случайный граф $G_n$ можно задать с помощью $n$ независимых случайных величин.
\end{problem}

\begin{ordre}
Воспользуйтесь последовательностью с.в. $\{\xi_i\}$, где равенство $\xi_i = 2j - 1$ означает, что ребро из вершины $i$ идет в $j$, если же   $\xi_i = 2j$, то  ребро из вершины $i$ идет в ту же вершину, что и ребро из вершины $j$.
\end{ordre}

\begin{remark}
Частным случаем приведенной модели графа является модель Bollobas--Riordan при $a = 1$, свойства которой хорошо изучены. К такого типа графам для уменьшения разреженности применяется постобработка: граф $G_{nm}$ преобразуется к графу $G_n^{m}$ путем ``склейки'' (объединения) равных групп из $m$ вершин в одну.

Считается, что модель графа должна удовлетворять следующим эмпирически выявленным характеристикам:
\begin{enumerate}
\item Число ребер пропорционально числу вершин, в то время как число треугольников на порядок больше числа ребер;
\item Одна большая компонента связности небольшого ($\sim 10$) диаметра;
\item Граф устойчив к случайному удалению вершин, в то время как удаление вершин максимальной степени приводит к разбиению графа на компоненты;
\item Степени, вторые степени (размеры второй окрестности) и PageRank вершин подчиняются степенному распределению;
\item Кластерный коэффициент -- вероятность того, что соседи случайной вершины  сами соединены между собой -- имеет неизменное с ростом числа вершин значение;
\item Средняя степень соседей случайной вершины степени $d$ имеет распределение $d^{\delta}$, где $\delta < 0$ характерно для веб-графа,  в то время как $\delta > 0$ характерно для социальной сети;
\item Количество ребер между вершинами заданных степеней имеет специфический вид распределения;
\item  Веб-графу свойственно наличие выраженных двудольных подграфов 
(любители--любимые сайты, покупатели--продавцы ссылок).  
\end{enumerate}
Одним из основных недостатков модели $G_n^{m}$ является низкий кластерный коэффициент (убывает с ростом $n$) и как следствие недостаточное число треугольников. Приведем модель (Ryabchenko--Samosvat--Ostroumova), являющуюся обобщением и лишенную последнего недостатка. В этой модели при добавлении новой вершины ($n+1$) происходит соединение с $m$ вершинами при соблюдении правил на изменение степени $d_i^{n}$ $i$-й вершины:
\[
\PR(d_i^{n+1} = d_i^{n}) = 1 - \frac{A d_i^{n} + B}{n} + O\left( \frac{(d_i^n)^2}{n^2} \right),
\]
\[
\PR(d_i^{n+1} = d_i^{n} + 1) = \frac{A d_i^{n} + B}{n} + O\left( \frac{(d_i^n)^2}{n^2} \right),
\]
\[
\PR(d_i^{n+1} - d_i^{n}  > 1) =  O\left( \frac{(d_i^n)^2}{n^2} \right),
\]
\[
\PR(d_{n+1}^{n+1} > m) =  O\left( \frac{1}{n} \right).
\]
Для $A = 1/(2+a)$, $B = ma/(2+a)$ получим модель $G_n^m$.

\end{remark}


\begin{remark}
Представление $G_n$  с помощью  независимых случайных величин позволяет воспользоваться неравенством Талаграна (см. \cite{15}) для подсчета статистик графа, в частности имеет место следующие неравенство для количества вершин $X_n(d)$ с размером второй окрестности равным $d = O(n^{1/(4+a) - \delta})$:
\[
\PR(|X_n(d) - \Exp X_n(d)| > (\Exp X_n(d))^{1-\varepsilon} ) \to 0, \quad n \to \infty.
\]  
См. также задачу \ref{pref_attach} из раздела \ref{hard} и задачу \ref{soc_ineq} из раздела \ref{macrosystems} и А.М. Райгородский, Модели интернета, Долгопрудный: Интеллект, 2013.
\end{remark}

