\documentclass[russian,10pt,a4paper]{article}

\usepackage[intlimits]{amsmath}
\usepackage{amsthm,amsfonts}
\usepackage{amssymb}
\usepackage{mathrsfs}
%\usepackage{graphicx}
\usepackage[final]{graphicx,epsfig} 
\usepackage{longtable}
\usepackage{indentfirst}
\usepackage[utf8]{inputenc}
\usepackage[T2A]{fontenc}
\usepackage[russian,english]{babel}
\usepackage[usenames]{color}





\hyphenpenalty=200
\tolerance=800

% Adjust first page number according to real document position in the book.
\setcounter{page}{1}

% Dot after section number
\makeatletter
% In section title
\def\@seccntformat#1{\csname the#1\endcsname.\quad}
\makeatother

%\tolerance = 2000

% To place author above title
\def\maketitle{
  \begin{center}


\thispagestyle{empty} 
\begin{center}

МИНИСТЕРСТВО ОБРАЗОВАНИЯ И НАУКИ \\
РОССИЙСКОЙ ФЕДЕРАЦИИ \\

 $ $ \\

Московский физико-технический институт \\
(государственный университет) \\
 
  $ $ \\ $ $ \\ $ $ \\
  $ $ \\
  $ $  \\
\smallskip
\smallskip
\premierAuthors, \\
\autresAuthors \\
 
 $ $ \\ $ $ \\
 
\textbf{СТОХАСТИЧЕСКИЙ АНАЛИЗ \\ В ЗАДАЧАХ} \\
 
 $ $ \\
  $ $ \\
  $ $ \\
Учебно-методическое пособие \\
 
  $ $ \\ $ $ \\ $ $ \\ $ $ \\
  \smallskip
   Часть 2
 $ $ \\ $ $ \\ $ $ \\ $ $ \\ $ $ \\ $ $ \\
 
Москва--Долгопрудный 2015


\end{center}

\newpage
  \end{center}
}

\renewcommand{\refname}{Литература}

%\sloppy
%\DeclareGraphicsRule{*}{eps}{*}{}
\DeclareMathOperator{\diam}{diam}

\graphicspath{{images/}}
\newcommand{\imgh}[3]{\begin{figure}[!h]\center{\includegraphics[width=#1]{#2}}\caption{#3}\label{Fig:#2}\end{figure}}

% Perfectly typesetted tilde for url links.
%\def\urltilde{\kern -.15em\lower .7ex\hbox{\~{}}\kern .04em}

\addto{\captionsrussian}{
    \renewcommand{\proofname}{\bf Решение. }
}

\usepackage{wasysym}

\theoremstyle{definition}
%\newtheorem{problem}{\noindent\normalsize\bfЗадача \No\!\!}
\newtheorem{problem}{\noindent\normalsize\bf{}}[section]
\renewcommand{\theproblem}{\arabic{problem}}
\newtheorem{example}{\noindent\normalsize\bf{}Пример}
\newtheorem*{definition}{\noindent\normalsize\bf{}Определение}
\newtheorem*{remark}{\noindent\normalsize\bf{}Замечание}
\newtheorem{theorem}{\noindent\normalsize\bf{}Теорема}
\newtheorem{fixme}{\noindent\normalsize\bf{}\frownie{} fixme}
\renewcommand{\thefixme}{}

\newtheorem*{ordre}{\noindent\normalsize\bf{}Указание}

\newenvironment{solution}{\begin{proof}\vspace{1em}} {\end{proof} \vspace{2em}}

\usepackage{verbatim}



\usepackage{enumitem}

\RequirePackage{enumitem}
\renewcommand{\alph}[1]{\asbuk{#1}} % костыль для кирилической нумерации 

\setenumerate[1]{label=\alph*), fullwidth, itemindent=\parindent, 
  listparindent=\parindent} 
\setenumerate[2]{label=\arabic*), fullwidth, itemindent=\parindent, 
  listparindent=\parindent, leftmargin=\parindent}

\newcommand{\rg}{\ensuremath{\mathrm{rg}}}
\newcommand{\grad}{\ensuremath{\mathrm{grad}}}
\newcommand{\diag}{\ensuremath{\mathrm{diag}}}
\newcommand{\const}{\ensuremath{\mathop{\mathrm{const}}}\nolimits}
\newcommand{\Var}{\ensuremath{\mathop{\mathbb{D}}}\nolimits}
\newcommand{\Exp}{\ensuremath{\mathrm{{\mathbb E}}}}
\newcommand{\PR}{\ensuremath{\mathrm{{\mathbb P}}}}
\newcommand{\Be}{\ensuremath{\mathrm{Be}}}
\newcommand{\Po}{\ensuremath{\mathrm{Po}}}
\newcommand{\Bi}{\ensuremath{\mathrm{Bi}}}
\newcommand{\Ker}{\ensuremath{\mathrm{Ker}}}
\newcommand{\Real}{\ensuremath{\mathrm{Re}}}
\newcommand{\Lin}{\ensuremath{\mathrm{Lin}}}
\newcommand{\Gl}{\ensuremath{\mathrm{Gl}}}
\newcommand{\mes}{\ensuremath{\mathrm{mes}}}
\newcommand{\cov}{\ensuremath{\mathrm{cov}}}


\begin{document}

\selectlanguage{russian}



\section{Предельные теоремы}
\label{zb4}


\begin{problem}
\label{contin}
Пусть для любой $f(\cdot) \in C^{\infty}\left(\mathbb{R}\right)$, обладающей равномерно ограниченными (для  $\forall i > 0$, $|f^{(i)}| <  \infty$) непрерывными производными,  при $n\to\infty$ выполнено соотношение 
 \[
 \int f(x) dF_n(x) \rightarrow \int f(x) dF(x),
 \]
\noindent где $F_n$, $F$ ~--- функции распределения с.в. $X_n$ и $X$ соответственно. Докажите, что в таком случае имеет место сходимость по распределению  $X_n$ к $X$, то есть при $n \to\infty$
\begin{equation*}
F_n(x) \to F(x)\quad\text{ в точках непрерывности }F(x).
\end{equation*}
 
\end{problem}
\begin{remark}
Если бы мы рассмотрели в качестве класса функций $f$ непрерывные ограниченные функции на метрическом пространстве $S$, то получили бы в точности сходимость по распределению (распределения задаются на классе борелевских множеств метрического пространства $S$, в некоторых учебниках такую сходимость называют слабой сходимостью, что, пожалуй, точнее отражает суть дела). 

См. Биллингсли П. Сходимость вероятностных мер. М.: Наука, 1977, 352 с.
\end{remark}

\begin{problem}

Объясните, почему при $n\to\infty$ нет слабой сходимости последовательности с.в. $\xi_n$  следующего вида:
\begin{equation*}
\mathbb{P} (\xi_n = -n) = \mathbb{P} (\xi_n = n) = \frac{1}{2}. 
\end{equation*}
Имеет ли место сходимость соответствующей последовательности функций распределений? Если ответ положительный, то будет ли функцией распределения предельная функция? Будет ли поточечно сходиться соответствующая последовательность характеристических функций? Приведите пример, когда имеет место поточечная сходимость характеристичесих функций, но нет слабой сходимости (сходимости по распределению) соответствующих случайных величин.
\end{problem}


\begin{problem}\Star(Теорема Леви--Крамера)
\label{levi_kramer}
Пусть $\left\{ {\xi _n } 
\right\}_{n\in { \mathbb{N}}} $ последовательность с.в., а $\left\{ {F_n \left( x 
\right)} \right\}_{n\in { \mathbb{N}}} $ и $\left\{ {\phi _n \left( t \right)} 
\right\}_{n\in { \mathbb{N}}} $ соответствующие последовательности функций 
распределений и характеристических функций. Покажите, что верны следующие утверждения.
\begin{enumerate}
\item Если существует с.в. $\xi $ с характеристической функцией $\phi 
\left( t \right)$ и такой функцией распределения $F\left( x \right)$, что при $n\to\infty$  
$F_n \left( x \right)\to F\left( x \right)$ в точках непрерывности $F\left( 
x \right)$, то (при $n\to\infty$) $\phi _n \left( t \right)\to \phi \left( t \right)$ 
равномерно на каждом конечном интервале.
\item Если существует такая непрерывная в нуле функция $\phi \left( t 
\right)$, что $\phi _n \left( t \right)\to \phi \left( t \right)$ при $n\to\infty$, то 
существует такая с.в. $\xi $ с характеристической функцией $\phi \left( t 
\right)$ и функцией распределения $F\left( x \right)$, такой что (при $n\to\infty$)  $F_n \left( x 
\right)\to F\left( x \right)$ в точках непрерывности $F\left( x \right)$, то 
есть $\xi _n \buildrel d \over \longrightarrow \xi $. Более того, при $n\to\infty$  
$F_n \left( x \right)$ сходится к $ F\left( x \right)$ равномерно на любом конечном или 
бесконечном множестве точек непрерывности функции $F\left( x \right)$, а 
также $\phi _n \left( t \right)\to \phi \left( t \right)$ равномерно на 
каждом конечном интервале.
\end{enumerate}

\end{problem}
\begin{remark}
См. книгу Сачков В.Н. Вероятностные методы в комбинаторном анализе. М.:~Наука, 1978, а также \cite{Gupta,21,stoianov}.
\end{remark}

\begin{problem}
Пусть с.в. $X_k$ нормально распределены и существует такая с.в. $X$, что $X_n \to X$ в $L_2$, т.е. $\mathbb{E} \left[|X_n-X |^2\right] \to 0$ при $n\to\infty$. Покажите, что с.в. $X$  нормально распределена.
\end{problem}

\begin{problem}[Теорема Линдберга]
Пусть   
$
\xi_{1}, \ldots , \xi_{n}
$, $n\in\mathbb{N}$ 
 независимые случайные величины, причем  $\Exp(\xi_{i}) = 0$ и $\Var(\xi_{i}) = \sigma^2_{i} < \infty$. Отметим, что вероятностное пространство может меняться в зависимости от $n$. Введем обозначения: 
 \[
 S_n = \xi_{1} + \ldots + \xi_{n},
 \]
 \[
 s^2_n = \sigma^2_{1} + \ldots + \sigma^2_{n}.
 \]
  
Покажите, что если $\forall \epsilon > 0$ 
   \begin{equation} \label{lindberg}
  \frac{1}{s_n^2} \underset{k=1}{\overset{n}{\sum}} \underset{x \geq \epsilon s_n}{\int} x^2 dF_{\xi_{k}}(x) \rightarrow 0, \: n \rightarrow \infty,
  \end{equation} 
  
  \noindent то при $n\to\infty$
  $$
  \frac{S_n}{s_n} \overset{d}{\longrightarrow} N(0, 1).
  $$
  Покажите также, что из условия \eqref{lindberg} следует следующее условие на частичную дисперсию:
  $$
    \max_{j\leq n}\dfrac{\sigma_j^2}{s_n^2} \rightarrow 0, \, n\rightarrow \infty
  $$
\end{problem}

\begin{ordre}[Л. Биллингсли]
Используя результат задачи \ref{contin}, необходимо доказать, что при $n\to\infty$
\[
\Exp\bigg[f\bigg(\frac{S_n}{s_n}\biggr)\biggr] \rightarrow \Exp(f(N)),
\] 
где $N$ --- нормально распределенная случайная величина с математическим ожиданием 0 и дисперсией 1.

Зафиксируем произвольную $f \in C^{\infty}, |f^{(i)}| <  \infty, \forall i > 0$.
Используя формулу Тейлора, покажите что
\begin{equation}\label{tayl_lind}
f(x + h_1) - f(x+h_2)  - \bigg[f'(x)(h_1 - h_2) + \frac{1}{2} f''(x)(h_1^2 - h_2^2)\biggr] \leq g(h_1) + g(h_2),
\end{equation}
\noindent где 
$$g(h) = \underset{x}{\sup}\bigg| f(x+h) - f(x) - f'(x)h - \frac{1}{2} f'(x)h^2\biggr|.$$

Последовательно заменяя $\xi_{i}$ в сумме $S_n$ на случайные величины $\eta_{k} \in N(0, \sigma_{k}^2)$, получим последовательность 

\[
\Exp\bigg[f\bigg(\frac{S_n}{s_n}\biggr)\biggr],\,
\ldots,\,
\Exp\bigg[f\bigg(\frac{\xi_{1} + \ldots + \eta_{n}}{s_n}\biggr)\biggr],\,
\ldots,\,
\Exp[f(N)].
\]

Модуль разности первого и последнего членов данной последовательности можно оценить сверху суммой модулей разности пар последовательно идущих элементов.  Введя вспомогательную переменную $\beta_{k} = \underset{i<k}{\sum}\xi_{k} + \underset{i>k}{\sum}\eta_{k}$, докажите, используя \eqref{tayl_lind}, следующее неравенство
\[
\biggr|\Exp\bigg[f\bigg(\frac{S_n}{s_n}\biggr)\biggr] - \Exp(f(N))\biggr| \leq \underset{k=1}{\overset{n}{\sum}} \Exp \bigg[g\bigg(\frac{\xi_{k}}{s_n}\biggr)\biggr] + \underset{k=1}{\overset{n}{\sum}} \Exp \bigg[g\bigg(\frac{\eta_{nk}}{s_n}\biggr)\biggr].
\] 

Далее, воспользовавшись свойством функции $g$ 
$$\exists K: g(h) < K \min(h^2, |h|^3)$$
и разбив математическое ожидание на интегралы по множеству  $(x \geq \epsilon s_n)$ и его дополнению, установите сходимость правой части неравенства к нулю.  
\end{ordre}

\begin{remark}

Классический вариант центральной предельной теоремы (ц.п.т.) может быть получен как следствие теоремы Линдберга \cite{5}: пусть $\xi_{i}$, $i=1,2,\dots$  независимые и одинаково распределенные случайные величины с нулевым математическим ожиданием и конечной дисперсией, тогда при $n\to\infty$
\[
\frac{1}{\sigma \sqrt{n}} \underset{k=1}{\overset{n}{\sum}} \xi_{k} \overset{d}{\longrightarrow}  N(0, 1).
 \]
 Отметим также, что с помощью описанной схемы можно оценить и скорость сходимости в ц.п.т.
 
Можно показать, что если 
\[
\lim_{n\to\infty}\max_{1\le k \le n} \Exp\left[\xi_{k}^2\right]=0\text{ или }\lim_{n\to\infty}\max_{1\le k \le n} \PR\left(|\xi_{k}|\ge\epsilon\right)=0,
\]
то условие (\ref{lindberg}) является не только достаточным, но и необходимым, для выполнения ц.п.т. Несложно проверить, что условие (\ref{lindberg}) влечет первое из этих условий, из которого, в свою очередь, следует второе.

См. Биллингсли П. Сходимость вероятностных мер. М.: Наука, 1977, 352 с.
\end{remark}




\begin{comment}
\begin{remark} 

\noindent $F_{\xi _{n} } (x)=\left\{\begin{array}{cc} {0,} & {x\le -n} \\ {{\raise0.7ex\hbox{$ 1 $}\!\mathord{\left/ {\vphantom {1 2}} \right. \kern-\nulldelimiterspace}\!\lower0.7ex\hbox{$ 2 $}} ,} & {-n<x\le n} \\ {1,} & {x>n} \end{array}\right. $ сходятся к функции $G(x)\equiv \frac{1}{2} $.

\end{remark} 
\end{comment}
\begin{problem}
Пусть $\xi _{1} ,\xi _{2} ,...$  независимые одинаково распределенные с.в. с конечной ненулевой дисперсией. \mbox{Обозначим $S_{n} =\sum _{i=1}^{n}\xi _{i}$}. Выяснить при каких значениях $c$ имеет или не имеет место сходимость при $n\to\infty$
$$
\PR\left(\frac{S_{n} }{n} \leq c\right)\to I\left(\Exp\xi _{1} \leq c\right).
$$

\begin{remark}
См. A.S. Cherny, The Kolmogorov student's competitions  on probability theory.
\end{remark}

\end{problem}


\begin{problem}
Пусть имеются независимые одинаково распределенные случайные величины $X_i$, $i=1,\dots,n$:

\[X_{i} =\left\{\begin{array}{cc} {1,} & {p;} \\ {-1,} & {q=1-p.} \end{array}\right. \] 

Покажите, что верна {\it локальная предельная теорема} (см. \cite{2}) для суммы независимых случайных величин $X_i$: равномерно по всем $x=O\left(\sqrt{n} \right)$ таким, что $(p-q)n+x$ целое неотрицательное число, выполнено (при $n\to\infty$)
\[\PR\left\{\sum _{i=1}^{n}X_{i} =(p-q)n+x \right\}\approx \frac{1}{\sqrt{2\pi npq} } \exp \left\{-\frac{x^{2} }{2npq} \right\}.\] 
\end{problem}
\begin{ordre} Покажите, используя \textit{теорему Коши} из курса ТФКП, что
$$\PR\left\{S_{n} =\left\lfloor \alpha n\right\rfloor \right\}=\frac{1}{2\pi i}     \ointctrclockwise _{|z|=\rho }\frac{(pz+qz^{-1} )^{n} }{z^{\left\lfloor \alpha n\right\rfloor +1} } dz$$
$$=\frac{1}{2\pi i} \ointctrclockwise  _{|z|=\rho }e^{n\left[\ln (pz+qz^{-1} )-\alpha \ln z\right]} e^{\left[\alpha n-\left\lfloor \alpha n\right\rfloor -1\right]\ln z} dz.$$


Примените \textit{метод перевала} для аппроксимации полученного интеграла. Выберите радиус окружности $\rho $ так, чтобы точка перевала находилась на пересечении этой окружности с положительной вещественной осью: $z=\rho $; для этого найдите максимум функции $\ln (pz+qz^{-} )-\alpha \ln z$. Замените интеграл по всей окружности на интеграл по $\delta $-дуге, содержащей точку перевала \cite{27}.
\end{ordre}


\begin{comment}
\begin{problem}

Доказать локальную предельную теорему:

\noindent Пусть $0<p<1$ и $X_{i} $, $i=1,...,n$ - независимые случайные величины, имеющие распределение:
\[X_{i} =\left\{\begin{array}{cc} {1,} & {p,} \\ {-1,} & {q=1-p;} \end{array}\right. \] 
Тогда равномерно по всем $x=O\left(\sqrt{n} \right)$ таким, что $(p-q)n+x$ целое неотрицательное число
\[\PR\left\{\sum _{i=1}^{n}X_{i} =(p-q)n+x \right\}\sim \frac{1}{\sqrt{2\pi npq} } \exp \left\{-\frac{x^{2} }{2npq} \right\}\] 
при $n\to \infty $. 

\begin{remark}
Воспользоваться формулой Стирлинга
\[
n! \sim \sqrt{2 \pi n} \frac{n^n}{e^n} 
\]
\end{remark}

\begin{remark}
Пусть $n=2k$ и $p=\frac{1}{2} $, тогда вероятность того, что число единиц в точности рано числу минус единиц мало (но не экспоненциально мало):
\[\PR\left\{\sum _{i=1}^{2k}X_{i} =k \right\}\sim \frac{1}{\sqrt{\pi k} } \] 
\end{remark}

\end{problem}

\end{comment}


\begin{problem}\Star(Коралов--Синай \cite{7})
Пусть задано гильбертово пространство
$\rm H = L^2\left( {{\rm R},{\rm B},\mu _G } \right)$, где $\mu _G$ --  
гауссовская мера, со скалярным произведением 
$$
\langle {f,g} 
\rangle=\int\limits_{-\infty }^{+\infty } 
{f(x)g(x)d\mu_G(x)}.
$$ 

Пусть $\xi _i $, $i=1,\dots,n$ независимые с.в. с нулевым математическим ожиданием и 
плотностью распределения 
$$
p_h =\frac{1}{\sqrt {2\pi } }\left( {1+h(x)} 
\right)e^{-\frac{x^2}{2}},
$$ где такой $h$ -- элемент $\rm H$, что
\begin{enumerate}
\item величина $\left\| h \right\|$ достаточно мала, 
где 
$$\left\| h \right\|^2=\frac{1}{\sqrt {2\pi } }\int\limits_{-\infty }^{+\infty } 
{h^2(x)e^{-\frac{x^2}{2}}dx},$$

\item $\langle h(x),\mathbbm{1} \rangle=0$,

\item $\langle {h(x),x} \rangle=0$.

\end{enumerate}
Покажите, что последовательность с.в. 
$$
\zeta _n 
=2^{-\frac{n}{2}}\sum\limits_{i=1}^{2^n} {\xi _i } 
$$ 
сходится 
по распределению к нормальной с.в. с нулевым математическим ожиданием и 
дисперсией 
$$
\sigma ^2(p_h )=\frac{1}{\sqrt {2\pi } }\int\limits_{-\infty 
}^{+\infty } {x^2p_h (x)dx}.$$

\end{problem}
\begin{remark} 
Для решения задачи можно воспользоваться методом ренормгруппы. Заметьте, что
$$\zeta _{n+1} =\frac{1}{\sqrt 2 }\left( {\zeta '_n 
+\zeta ''_n } \right),$$
где 
$$
\zeta '_n 
=2^{-\frac{n}{2}}\sum\limits_{i=1}^{2^n} {\xi _i },\quad 
\zeta ''_n 
=2^{-\frac{n}{2}}\sum\limits_{i=2^n+1}^{2^{n+1}} {\xi _i }
$$
независимые 
одинаково распределенные случайные величины, поэтому 
$$
p_{n+1} (x)=Tp_n (x),
$$ 
где $p_n (x)$~--- плотность распределения $\zeta _n $, а $T$~--- нелинейный 
оператор (действующий в пространстве плотностей), такой что 
$$
Tp(x)=\sqrt 2 
\int\limits_{-\infty }^{+\infty } {p(\sqrt 2 x-u)p(u)du}.
$$ 
В этих терминах 
в задаче нужно показать, что 
\[
T^n p_h (x)\mathop \to \limits_{n\to \infty } 
\frac{1}{\sqrt {2\pi } \sigma (p_h )}{\rm e}^{-\frac{x^2}{2\sigma ^2(p_h )}}.
\] 

Затем рассмотрите нелинейный оператор $\tilde{L}$ на пространстве ${\rm H}$, связанный с 
оператором $T$ следующим образом
$$
T p_h (x)=\frac{1}{\sqrt {2\pi } }\left( 
{1+\tilde{L}\left( {h(x)} \right)} \right){\rm e}^{-\frac{x^2}{2}}.
$$ 
Линеаризуйте 
оператор $\tilde {L}$, показав, что $\tilde {L}h=Lh+O\left( {\left\| h 
\right\|^2} \right)$, где 
\[L(h)(x)=\frac{2}{\sqrt \pi }\int\limits_{-\infty 
}^{+\infty } {e^{-\left( {\frac{x^2}{2}-\sqrt 2 xu+u^2} \right)}h(u)du} .
\]

Далее покажите, что линейный оператор $L$ имеет полное множество собственных 
векторов 
$$h_k (x)=e^{\frac{x^2}{2}}\left( {\frac{d}{dx}} 
\right)^ke^{-\frac{x^2}{2}},\quad k\ge 0,
$$
с собственными значениями $\lambda 
_k =2^{1-\frac{k}{2}}$, $k\ge 0$. 

Пусть ${\rm H}_k $~--- одномерное 
подпространство, натянутое на $h_k (x)$. Тогда $L$~--- сжимающий оператор на 
${\rm H} \setminus \left( {{\rm H}_0 \oplus {\rm H}_1 \oplus {\rm H}_2 } \right)$.


Покажите, что $\tilde {L}$~--- сжимающий оператор на ${\rm H} \setminus \left( {{\rm 
H}_0 \oplus {\rm H}_1 } \right)$, имеющий единственную неподвижную точку 
\[f_h (x)=\frac{1}{\sigma (p_h )}e^{\frac{x^2}{2}-\frac{x^2}{2\sigma ^2(p_h 
)}}-1.
\]
\end{remark}



\begin{problem}
В игре в рулетку колесо разделено на 38 равных секторов: 18 красных, 18 белых и два сектора (0 и 00) зеленого цвета. Пусть ставка игрока на каждом шаге равна 1000 рублей. Обозначим за $X_{i} $ выигрыш в $i$-ой игре. Тогда $X_{1} ,X_{2} ,X_{3} ,...$ ~--- независимые с.в., имеющие распределение: 
\[X_{i} =\left\{\begin{array}{cc} {+1,} & p = 18/38, \\ {-1,} & p = 20/38. \end{array}\right. \] 
Пусть сыграно $n=19^2=361$ партий. С помощью ц.п.т.  и  неравенства Берри--Эссена оцените погрешность приближения выигрыша.
\end{problem}
\begin{remark}
Приведем формулировку теоремы  Берри--Эссена \cite{19}.
\label{sec:BerryEssen}
 Пусть $\xi_1, \xi_2\dots$ независимые одинаково распределенные с.в., причем $\mathbb{E}\xi_i = m$, 
 $\mu^3={\mathbb E}|\xi_i - {\mathbb E}\xi_i|^3<\infty$, $\sigma^2=\mathbb D \xi_i$.
Близость с.в. $\frac{\sum_{i=1}^{n}\xi_i-nm}{\sigma\sqrt{n}}$ к стандартной нормально распределенной с.в. (согласно ц.п.т.) в смысле 
близости их функций распределения определяется неравенством Берри--Эссена 
$$
\sup\limits_x \left| {\mathbb P}\Bigl( \frac{\sum_{i=1}^{n}\xi_i-nm}{\sigma\sqrt{n}}<x \Bigr) - \Phi(x) 
\right| \le \frac{C_0 \mu^3}{\sigma^3 \sqrt{n}} , 
$$
где $0.4<C_0<0.7056$,
$\Phi(x)=\int_{-\infty}^x \frac{e^{-t^2/2}}{\sqrt{2\pi}}\, dt$.
\end{remark}

\begin{problem}[Петербургский парадокс \cite{19, book12}] 
\label{piter}
Рассмотрим игру в орлянку: игрок делает ставку и подкидывает симметричную монету, если выпадает «орел», то игрок забирает двойную ставку, иначе теряет все свои деньги. Броски повторяются независимо в каждой игровой серии, изначально игрок имеет в бане сумму равную 1 и в каждой серии разыгрывает все деньги, что у него есть.\\
\indent Пусть $X_{1} ,X_{2} ,X_{3} ,...$  --- независимые с.в., обозначают состояние банка игрока после 1,2,3.. игр соответственно. Несложно заметить, что распределение $X_i$ выглядит как $\PR\left(X_{i} =2^{k} \right)=2^{-k} $, $k=1,2,3,...$. То есть, если в игре в орлянку $k$ раз выпал «орел», то выигрыш будет $2^{k}$. \\
\indent Справедливой ценой за игру называют математическое ожидание выигрыша. Но здесь $\Exp X_{i} =\infty $, однако, для этого нужно играть бесконечное число раз и иметь бесконечно много денег. Покажите, что 
$$
\frac{S_{n} }{n\log _{2} n} \mathop{\to }\limits^{p} 1\quad \text{при~} n\to \infty ,
$$ 
где $S_{n} =\sum _{k=1}^{n}X_{k}$. Проинтерпретируйте этот результат, введя цену за $n$ игр.

\begin{remark} 

Пусть для каждого $n$ с.в. $X_{nk} $, $1\le k\le n$ независимы. Пусть также $b_{n} >0$ с $b_{n} \to \infty $ и $\bar{X}_{nk} =X_{nk}  I \left\{X_{nk} \le b_{n} \right\}$. Предположим, что выполняются условия:
\begin{enumerate}
\item 
$
\sum _{k=1}^{n}\PR\left\{\left|X_{nk} \right|>b_{n} \right\} \mathop{\to }\limits_{n\to \infty } 0;
$ 
\item 
$
\frac{1}{b_{n} ^{2} }\sum _{k=1}^{n}\Var \bar{X}_{nk}   \mathop{\to }\limits_{n\to \infty } 0.
$ 
\end{enumerate}
Тогда верно
$$\frac{1}{b_{n} }\left(\sum _{k=1}^{n}X_{nk}  -\sum _{k=1}^{n}\Exp \bar{X}_{nk}  \right) \mathop{\to }\limits^{p} 0\quad \text{при~} n\to \infty. 
$$

\noindent Для решения задачи положите $X_{nk} =X_{k} $. В качестве $b_{n} >0$ возьмите $b_{n} =2^{m(n)} $, где $m(n)$ -- целое число, которое можно представить в виде $m(n)=\log _{2} n+K(n)$, $K(n)\to \infty $ при $n\to \infty $. Например, если \mbox{$K(n)\le \log_2 (\log_2 n)$,} то результатом применения приведенного результата будет $\frac{S_{n} }{n\log _{2} n} \mathop{\to }\limits^{p} 1$ при $n\to \infty $.

\end{remark} 

\end{problem}

\begin{problem}
Случайная величина (размер выигрыша) принимает значение $2^{k} -1$ с вероятностью 
$$p_{k} =\frac{1}{2^{k} k(k+1)} \quad \text{для~} k=1,2,3,...$$ 
и значение\textit{ $-1$} с вероятностью 
$$p_{0} =1-\sum _{k=1}^{\infty }p_{k} .$$ 
Проверьте, что математическое ожидание выигрыша равно нулю. Применив теорему из предыдущей задачи, покажите, что при $n\to \infty $  для суммарного размера выигрыша за $n$ партий ($S_{n} $) справедливо
$$\frac{S_{n} }{n \log _{2} n}\mathop{\to }\limits^{p} 1.$$

\begin{remark}  
Положите в замечании к задаче \ref{piter} $b_{n} =2^{m(n)} $, где $$m(n)=\min \left\{m\in \mathbb{N}:\; 2^{-m} \frac{1}{\sqrt{m^{3}}} \le n^{-1} \right\}.$$ Следует также обратиться к книге \cite{19}.
\end{remark} 

\end{problem}


%\begin{comment}
\begin{problem}[MAX-устойчивые распределения: Гумбеля, Фреше, Вейбулла]
Распределение $G\left(x\right)$ называется max-устойчивым, если для любых $n=1,2,...$ существуют $a_{n} >0$ и $b_{n} \in {\mathbb R}$, такие что $G^{n} \left(a_{n} x+b_{n} \right)=G\left(x\right)$.
Пусть есть независимые одинаково распределенные с.в. $X_{1} ,...,X_{n} $ с распределением $F\left(x\right)$. Обозначим $X_{\left(n\right)} =\max \left\{X_{1} ,...,X_{n} \right\}$. Распределение такой с.в. $F_{X_{\left(n\right)} } \left(x\right)=\left[F\left(x\right)\right]^{n} $.
Обозначим также за $x_{(n)}$ решение уравнения $$\mathbb{P}(X>x_{(n)}) = 1-F(x_{(n)}) = 1/n$$ (так называемое, характеристическое наибольшее значение).
\begin{enumerate}
\item Пусть при $x\to\infty$ функция распределения $X_i$ имеет вид $F(x) = 1-Ax^{-\alpha}$, $\alpha>0$. Покажите, что функция распределения случайной величины $Y = X_{(n)}/x_{(n)}$ поточечно стремится к пределу $F_{Y}(y) = {\rm e}^{{-y}^{-\alpha}}$ при $y\geq 0$, $F_{Y}(y) = 0$ при $y<0$ (распределение Фреше).
%\item  Пусть $\mathop{\lim }\limits_{x\to \infty } e^{\alpha x} \left(1-F\left(x\right)\right)=\beta $, где $\alpha ,\beta >0$ и $x\in {\mathbb R}$. Покажите, что $X_{\left(n\right)} -\frac{1}{\alpha } \ln \left(\beta n\right)\mathop{\to }\limits^{d} \chi $, где $\chi $ имеет распределение Гумбеля: $\PR\left(\chi \le x\right)=e^{-e^{-\alpha x} } $, $x\in {\mathbb R}$.
\item Пусть случайные величины $X_i$ ограничены сверху ($X_i\leq 0$) и в окрестности нуля при $x\to\infty$ функция распределения $X_i$ имеет вид $F(x) = 1-A|x|^{\alpha}$, $\alpha>0$ при $x\leq 0$, $F(x)=1$ при $x>0$. Покажите, что функция распределения случайной величины $Y = X_{(n)}/x_{(n)}$ стремится к пределу $F_{Y}(y) = {\rm e}^{{-|y|}^{\alpha}}$ при $y\leq 0$, $F_{Y}(y) = 1$ при $y>0$ (распределение Вейбулла).
%\item  Пусть $\mathop{\lim }\limits_{x\to \infty } x^{\alpha } \left(1-F\left(x\right)\right)=\beta $, где $\alpha ,\beta >0$ и $x\in {\mathbb R}_{+} $. Покажите, что $X_{\left(n\right)} \left(\beta n\right)^{-\frac{1}{\alpha } } \mathop{\to }\limits^{d} \eta $, где $\eta $ имеет распределение Фреше: $\PR\left(\eta \le x\right)=e^{-x^{-\alpha } } $, $x>0$.
\item Пусть при $x\to\infty$ функция распределения $X_i$ имеет вид $F(x) = 1- {\rm e}^{-\lambda x}$. Покажите, что функция распределения случайной величины $Y = X_{(n)}-x_{(n)}$ стремится к пределу $F_{Y}(y) = {\rm e}^{-{\rm e}^{-\lambda y}}$ (распределение Гумбеля).
%\item  Пусть $\mathop{\lim }\limits_{x\to \infty } \left(c-x\right)^{\alpha } \left(1-F\left(x\right)\right)=\beta $, $F\left(c\right)=1$, где $\alpha ,\beta >0$, $c\in {\mathbb R}$ и $x\in {\mathbb R}$. Покажите, что $\left(X_{\left(n\right)} -c\right)\left(\beta n\right)^{\frac{1}{\alpha } } \mathop{\to }\limits^{d} \gamma $, где $\gamma $ имеет распределение Вейбулла: $\PR\left(\gamma \le x\right)=e^{-(-x)^{-\alpha } } $, $x<0$.

\item Покажите, что распределения Гумбеля, Фреше, Вейбулла являются max-устойчивыми.

\item Покажите, что если $X_i$, $i=1,2,\dots$ --- независимые одинаково распределенные стандартные
\\
1)  гауссовские случайные величины, то верно
\[
\lim_{n\to\infty}\PR\left\{2m_n[\max_{i=1,\dots,n}{X_i} - m_n]\leq z \right\}= \exp(-\exp(-z)),
\]
\noindent
где 
\[
  m_n = \left[2\log\frac{n+1}{\sqrt{8\pi}\log(n+1)}\right]^{1/2};
\]
2) лапласовские случайные величины, то верно
\[
\lim_{n\to\infty}\PR\left\{\max_{i=1,\dots,n}{X_i} - \log(n/2)\leq z \right\}= \exp(-\exp(-z)).
\]
%\item случайные величины с распределением Коши, %то верно
%\begin{equation}
%\lim_{n\to\infty}\PR\left\{\frac{\pi}{4}\max_{i=1,\dots,n}{X_i} \leq z \right\}= \exp(-z^{-1}).
%\end{equation}
\end{enumerate}

\begin{remark}
Рассмотрим следующий пример (см. Лагутин М.\,Б. Наглядная математическая статистика Издательство: М.: БИНОМ. Лаборатория знаний, 2007). Пусть $X_1,\dots,X_n$~--- независимые одинаково распределенные случайные величины с распределением 
$$
F(x) = \left(1-\frac{1}{\ln x}\right)I\left(x>\exp(1)\right),  
$$
Обозначим $X_{(n)} = \max\{X_1,\dots,X_n\}$. Оценим $\gamma = P(X_{(n)}>10^7)$ при $n=4$. Из независимости и одинаковой распределенности $X_i$ 
$$P(X_{(n)}\leq x) = [F(x)]^n.$$
Поскольку $\ln 10 \approx 2.3$, а $(1-\epsilon)^n\approx 1-\epsilon n$ при малых $\epsilon$, получаем
$$\gamma = 1 - \left(1-\frac{1}{7\ln 10}\right)^4 \approx 1/4.$$
Таким образом, примерно в каждом четвертом случае значение $X_{(4)}$ будет превышать $10^7$. 
Оказывается, что из-за того, что функция $F(x)$ имеет <<сверхтяжелый>> правый <<хвост>>, распределение случайной величины $X_{(n)}$ чрезвычайно быстро с ростом $n$ уходит на бесконечность и никаким линейным преобразованием не удается <<вернуть>> его в конечную область. Точнее, невозможно подобрать такие константы $a_n$ и $b_n>0$, чтобы последовательность $(X_{(n)}-a_{n})/b_n$ сходилась бы по распределению к невырожденному закону.

См. лекции (http://www.mccme.ru/ium/s09/probability.html) в НМУ  А.Н. Соболевского, а также Лидбеттер М., Линдгрен Г., Ротсен X. Экстремумы случайных последовательностей и процессов М.: Мир, 1989.

Отметим также, что распределения Гумбеля, Фреше, Вейбулла исчерпывают все возможные типы предельных распределений в классе max-устойчивых распределений.
\end{remark}
\end{problem}



\begin{problem}

Пусть $X_{1} ,...,X_{n} $ независимые одинаково распределенные с.в. со стандартным распределением Коши:
$$
F\left(x\right)=\frac{2}{\pi } \int _{-\infty }^{x}\frac{dy}{1+y^{2}}  
$$ 
Воспользовавшись предыдущей задачей, найдите предельное распределение для должным образом нормированных с.в. $X_{\left(n\right)} $.
\end{problem}
\begin{remark}
На эту тему также рекомендуется посмотреть задачу 119 из раздела 2, и задачу 27 из раздела 3.
\end{remark}

%\end{comment}

\begin{problem}
Пусть ${X}_{n} \in {\mathbb R}^{m} $, $n=1,2,...$ ~--- независимые одинаково распределенные случайные векторы, причем  
$$\Exp{X}_{n} =0, \quad \Exp({X}_{n} {X}_{n}^{T}) =R$$ ($R$ ~--- неотрицательно определенная матрица (по определению), однако, мы дополнительно будем считать, что $R$ положительно определенная). С помощью аппарата характеристических функций докажите, что тогда для любого борелевского множества $B\subseteq {\mathbb R}^{m} $ верно
\[\mathop{\lim }\limits_{N\to \infty } \PR\left(\frac{1}{\sqrt{N} } \sum _{n=1}^{N}{X}_{n}  \in B\right)=\dfrac{1}{(2\pi)^{m/2}|R|^{1/2}} \int _{B}{\rm e}^{-\frac{1}{2} <{x},R^{-1}{x}>} d {x} .\] 

Переформулируйте и решите задачу для случая, когда матрица $R$ положительная полуопределенная.
\end{problem}
\begin{remark}
Аппарат характеристических функций -- мощная конструкция для изучания свойств распределений случайных величин, т.к. известно, что есть взаимнооднозначное соответствие между вероятностой мерой и её Фурье образом. Читателю можно порекомедовать книгу книгу Боровкова А.А. \cite{3} для более подробного изучения свойств характеристических функций. Также большое количество показательных упражнений на характеристические функции можно найти в книге Ширяева [\ref{chiraiev}] т.1.\\
\indent Развивая историческую часть доказательства ЦПТ, стоит также упомянуть, что классически существует два подхода доказательства указанной выше теоремы. Первый способ -- метод Линденберга (см. задачу 5, раздел 5) и второй более распространенный -- метод характеристических функций, с которым можно ознакомится, например, в книге \cite{6}. Возникает вопрос, какой из способов доказательства более общий. Читателю предлагается проверить, что из первого метода следует второй.
\end{remark}




\begin{problem}

Пусть $X_1,\ldots,X_n$~--- независимые одинаково распределенные с.в. Пусть также характеристическая функция с.в. $X_k$ представляется 
в окрестности $t=0$ в виде 
$$
\varphi_{X_k}(t)={\mathbb E}(e^{it X_k})=1+imt+o(t). 
$$
 Верно ли, что при $n\to\infty$ 
$$
\frac{1}{n}\sum\limits_{i=1}^{n} X_i \xrightarrow{p} m?
$$
\end{problem}
\begin{remark}
Стоит отметить A.S. Cherny, The Kolmogorov student's competitions on probability theory, где можно найти набор весьма интересных небольших задач про связь сходимостей в различных вероятностных смыслах.
\end{remark}

\begin{comment}
\begin{problem}
Пусть $x_1, x_2, x_3, \ldots$~--- последовательность независимых одинаково распределенных с.в. Положим 
$S_n=\sum\limits_{k=1}^{n} x_k$. Покажите, что 

\begin{enumerate}
\item(з.б.ч.) если ${\mathbb E}(|x_k|)<\infty$, то $S_n/n\xrightarrow{P} m$ при $n\to\infty$, где $m={\mathbb E}(x_k)$; 

\item(ц.п.т.) если ${\mathbb E}(x_k^2)<\infty$, то $(S_n-m\cdot n)/\sqrt{n\cdot D}\xrightarrow{d} N(0,1)$ при $n\to\infty$, 
где $D=\Var x_k$. 

\item(задача математической статистики) Предположим, что независимо $n$ раз кидается монетка с вероятностью выпадения орла в каждом 
опыте равной $p$ (точного значения $p$ мы не знаем, а знаем лишь то, что $0.1\leqslant p\leqslant 0.9$), т.е. $x_k\in\Be(p)$. 
Сколько раз нужно кинуть монетку (оцените $p$), чтобы оценка $${\bar p}(x)=\frac{\sum\limits_{k=1}^{n}x_k}{n}$$ с вероятностью 
$\gamma\geqslant 0.95$ отличалась от истинного значения $p$ не более, чем на величину $\delta=0.01$? Применить неравенство Чебышева 
и предельную теорему (точность, которую дает ц.п.т., оцените с помощью неравенства Берри – Эссена). Сравнить результаты. 
\end{enumerate}
\end{problem}

\begin{remark}
См. раздел \ref{measure}, задача \ref{sec:BerryEssen}

\end{remark}
\end{comment}

\begin{problem}
Пусть при каждом $n\geqslant 1$ независимые с.в. $X_{1n}, X_{2n},\ldots, X_{nn}$ таковы, что $X_{kn}\in \Be(p_{kn})$, где 
$$\lim_{n\to\infty}\max\limits_{1\leqslant k\leqslant n} p_{kn}=0\quad\text{и ~} \lim_{n\to\infty}\sum\limits_{k=1}^{n}p_{kn}=\lambda.$$ 
Тогда при $n\to\infty$
\begin{equation*}
\label{TPois}
{\mathbb P}(S_n=m)\to  e^{-\lambda}\frac{\lambda^m}{m!}, \quad m=0,1,2,\ldots, 
\end{equation*}
 где $S_n=\sum\limits_{k=1}^{n} X_{kn}$. 
\begin{remark}
Если $p=\lambda/n$, то $$C_{n}^{m}p^{m}(1-p)^{n-m} = \frac{\lambda^m}{n!}e^{-\lambda}\left( 1 + O\left(\frac{m^2+\lambda^2}{n}\right)\right).$$
Кроме того, имеет место следующая оценка (Прохорова--Ле Кама): 
$$\sum_{m=0}^{\infty}{\left|\mathbb P(S_n=m) - e^{-\lambda}\frac{\lambda^{m}}{m!}\right|} \leq 2 \sum_{k=1}^{n}{p_{kn}^2}.$$
В качестве основных ссылок для этой задачи отметим \cite{19},[\ref{chiraiev}] т.1.
\end{remark}
\end{problem}

\begin{problem}
В течение дня игрок в казино участвует в $N=100$ независимых розыгрышах. В каждом розыгрыше он выигрывает с вероятностью 
$p=0.01$. Оцените вероятность события:\\
\indent а) игрок ни разу не выиграет\\
\indent б) выиграет ровно один раз \\
\indent в) выиграет ровно три раза.\\

Предположим, что игра происходит в течении $n=100$ дней. 
Оцените вероятность того, что за эти $100$ дней в общей сложности реализуется не менее $100$ выигрышей, не менее $300$ выигрышей. 


\end{problem}

\begin{problem}(Дельта метод). 
Пусть $T_n$~--- последовательность случайных величин, такая, что  при $n\to\infty$
$$\sqrt{n}(T_n-\theta)\xrightarrow{d}\mathcal{N}(0,\sigma^2(\theta)),\quad \sigma(\theta)>0.$$
Пусть отображение $g:\mathbb{R}\to\mathbb{R}$ дифференцируемо в $\theta$ и $g^{\prime}(\theta)\not=0$. Покажите, что при $n\to\infty$
$$
\sqrt{n}\left(g(T_n)-g(\theta)\right)\xrightarrow{d} \mathcal{N}(0,[g^{\prime}(\theta)]^2\sigma^2(\theta)).
$$
\end{problem}
\begin{remark} 
Пусть  $g^{\prime}(\theta) = 0$, в таком случае распределение $g(T_n)$ определяется третьим членом разложения Тейлора, то есть:
$$
g(T_n) = g(\theta) + \frac{(T_n-\theta)^2}{2}g^{\prime\prime}(\theta)+o\left((T_n-\theta)^2\right),
$$ 
где в данной задаче $o(1)$--величина, сходящаяся с вероятностью $1$ к $0$. Поэтому при $n\to\infty$
$$
\sqrt{n}(g(T_n) -g(\theta)) = n\frac{(T_n-\theta)^2}{2}g^{\prime\prime}(\theta)+o(1)\xrightarrow{d} \frac{g^{\prime\prime}(\theta)\sigma^2(\theta)}{2}\chi^2(1),
$$
где $\chi^2(1)$ --- случайная величина с распределением $\chi^2$ с одной степенью свободы.

Для решения задачи рекомендуется ознакомиться с литературой по дельта-методу, указанной в \cite{Gupta}.
\end{remark}

\begin{problem}
Пусть $\{T_n\}$ последовательность $k$-мерных случайных векторов, таких, что при $n\to\infty$ 
$$\sqrt{n}(T_n-\theta)\xrightarrow{d}\mathcal{N}(0,\Sigma(\theta)).$$ 
Пусть $g:\mathbb{R}^{k}\to \mathbb{R}^m$ дифференцируемо в $\theta$ с $\nabla g(\theta)$. 
\begin{enumerate}
\item

Покажите, что при $n\to\infty$
$$
\sqrt{n}\left(g(T_n)-g(\theta)\right)\xrightarrow{d} \mathcal{N}(0,\nabla g(\theta)^{T}\Sigma(\theta)\nabla g(\theta)),
$$
если $\nabla g(\theta)^{T}\Sigma(\theta)\nabla g(\theta)$ положительно определена.
\item
Рассмотрите пример, когда   $X_1,\dots,X_n$ независимые одинаково распределенные случайные величины с математическим ожиданием $\mu$ и дисперсией $\sigma^2$, а $T_n = \frac{1}{n}\sum_{i=1}^n X_n$. Покажите, что тогда при $n\to\infty$ 
$$
\sqrt{n}(T_n^2-\mu^2)\xrightarrow{d}\mathcal{N}(0,4\mu^2\sigma^2).
$$
\end{enumerate}
\end{problem}

\begin{remark}
См. также книгу Боровков А.А., Математическая статистика, СПб.: Лань, 2010.
\end{remark}

\begin{problem}(Центральная предельная теорема для стационарных последовательностей)

Пусть $X_i$, $i=1,2,\dots$ --- стационарная последовательность с $\mathbb{E}(X_i)=\mu$ и $\Var (X_i) = \sigma^2<\infty$, обладающая  следующим свойством: для некоторого фиксированного $m$ выполнено, что $(X_1,\dots,X_i)$ и $(X_j,X_{j+1}\dots,)$ независимы при $j-i>m$.

Покажите, что $n\to\infty$

$$
\frac{\frac{1}{n}\sum_{i=1}^n X_i-\mu}{\sqrt{n}}\overset{d}{\to} \mathcal{N}(0,\tau^2),
$$
где $\tau^2 = \sigma^2+2\sum_{i=2}^{m+1}{\rm cov}(X_1,X_i).$

\end{problem}
\begin{remark}
В главах 9 и 10 книги \cite{Gupta} содержится большое количество вариаций центральной предельной теоремы для случая случайных  последовательностей с зависимыми элементами (например, стационарных последовательностей, марковских последовательностей и др.).
\end{remark}

\begin{problem}
Пусть $X_1$, $X_2$, ... независимые одинаково распределенные случайные величины с конечным четвертым моментом. Пусть $\mathbb{E}(X_1)=\mu$ и $\mathbb{D}(X_1)=
\sigma^2$. Пусть $g$ --- функция с равномерно ограниченной четвертой производной. Покажите, что 
\begin{enumerate}
\item $\mathbb{E}[g(\bar{X})] = g(\mu)+\frac{g^{(2)}(\mu)\sigma^2}{2n}+O(n^{-2})$, 
\item
$\mathbb{D}[g(\bar{X})] = \frac{(g^{\prime}(\mu))^2}{n}+O(n^{-2}).$
\end{enumerate}
\end{problem}
\begin{remark}
Основные ссылки по указанным результатам содержатся в \cite{Gupta}.
\end{remark}

\begin{problem}

Показать, что при бросании симметричной монеты $n$ раз отношение числа выпадений герба к числу выпадений решки почти наверное стремится 
к $1$ при $n\to\infty$, а вероятность того, что число выпадений герба в точности равняется числу выпадений решки, при четном числе 
бросаний стремится к $0$ при $n\to\infty$. 
\end{problem}

\begin{problem}
Пусть при любом $\lambda >0$ с.в. $\xi _{\lambda } $ имеет распределение $\Po(\lambda)$. Докажите, что $(\sqrt{\lambda})^{-1}[\xi _{\lambda } -\lambda]  $ по распределению сходится  к стандартному нормальному распределению при $\lambda \to \infty $.
\end{problem}

\begin{problem} 
Пусть с.в. $x_n\in \Gamma(\lambda,n)$, т.е. неотрицательная случайная величина  с плотностью распределения при $x\geq 0$ равной 
$$
p_{x_n}(x) = \dfrac{x^{n-1}{\rm e}^{-x/\lambda}\lambda^{-n}}{\Gamma(n-1)}
$$
Напомним, что для целых $n$ Гамма-функция принимает значения $\Gamma(n) = (n-1)!$.
Покажите, что из ц.п.т. следует 
$$
\frac{x_n-m(\lambda)\cdot n}{\sigma(\lambda)\cdot\sqrt{n}} \xrightarrow{d} \mathcal{N}(0,1) \text{ при } n\to\infty . 
$$
Найдите $m(\lambda)$, $\sigma(\lambda)$. 
\end{problem}
\begin{remark}
Смотрите замечание к задаче \ref{GammaFunc}.
\end{remark}

%\begin{comment}
\begin{problem}
Пусть $X_n$ --- последовательность независимых с.в., сходящаяся по вероятности к с.в. $X$:  $X_n\xrightarrow{p}X$ при $n\to\infty$. Докажите, 
что с.в. $X$ вырождена, т.е. $X\equiv x$, где $x$ --- некоторое число. 
\end{problem}

\begin{remark}

Справедливы следующие утверждения:

\begin{enumerate}
\item
Из любой сходящейся по мере (в частности, по вероятностной) последовательности 
измеримых функций (в частности, с.в.) можно выделить подпоследовательность, сходящуюся почти всюду (п.н.). 

\item
Из закона нуля и единицы Колмогорова следует, что для всякого разбиения прямой ${\mathbb R}$ на борелевские множества $\{ B_m\}_{m\geqslant 1}$ 
ровно для одного $m=m_0:$ $\quad {\mathbb P}(A_{B_{m_0}})=1$, для остальных $m:\quad {\mathbb P}(A_{B_m})=0$, где 
$$
A_{B_m}=\{ \omega: \, X=\lim\limits_{k\to\infty} X_{n_k}\in B_m \} . 
$$
\end{enumerate}

См. A.S. Cherny, The Kolmogorov student's competitions  on probability theory, а также \cite{22,220}.
\end{remark}

%\end{comment}


\begin{problem}
В некотором городе прошел второй тур выборов. Выбор был между двумя кандидатами $A$ и $B$ (графы <<против всех>> на этих выборах не было). 
Сколько человек надо опросить на выходе с избирательных участков, чтобы исходя из ответов можно было определить долю проголосовавших 
за кандидата $A$ с точностью $3\%$ и с вероятностью не меньшей $0.99$. 
\end{problem}

%\begin{comment}
\begin{problem} [Асимптотическое распределение числа инверсий в случайной перестановке]

На множестве $n!$ перестановок $n$ различных элементов задано равномерное распределение. Обозначим через $\xi_k$ случайную величину, 
равную числу инверсий, образованных элементом с номером $k$, т.е. равную числу элементов с номерами меньшими чем $k$, 
которые стоят в перестановке правее элемента с номером $k$. Покажите, что 
$$
\frac{\sum\limits_{k=1}^{n}\xi_k -\left.n^2\right/4}{\left.n^{3/2}\right/6}\xrightarrow{d} N(0,1) \quad \text{ при } n\to\infty . 
$$
\end{problem}


\begin{remark} Для решения этой и следующей задачи полезно ознакомиться с книгой Сачков В.Н. Комбинаторные методы дискретной математики. М.: МЦНМО, 2004.
$ $

\begin{enumerate}

\item
 Введем с.в. 
$$
\xi_{k,i} = I(\text{<<$k$ находится левее числа $i$>>}) . 
$$

Тогда 
$$
\xi_k=\xi_{k,1}+\xi_{k,2}+\ldots +\xi_{k,k-1}, 
$$

Покажите, что
$${\mathbb E}\xi_k=\left.(k-1)\right/2,$$

$$
{\mathbb E}\xi_k^2={\mathbb E}\xi_{k,1}^2+\ldots+{\mathbb E}\xi_{k,k-1}^2+2\sum\limits_{i<j<k}{\mathbb E}(\xi_{k,i}\xi_{k,j})=
$$
$$
=\frac{k-1}{2}+2\cdot\frac{(k-1)(k-2)}{2}\cdot\frac{1}{3}=\frac{2k^2-3k+1}{6}, 
$$
с.в. $\xi_k$ и $\xi_m$ некоррелированы, $k\ne m$. 


\item 
Для всякой сл.в. $X_k=\xi_k-{\mathbb E}\xi_k$ характеристическая функция имеет вид 
\[
\varphi_{X_k}(t)=1-\frac{t^2 \Var\xi_k}{2}+ o(t^2).
\]

\end{enumerate}
\end{remark}
%\end{comment}

\begin{problem}(Асимптотическое распределение числа циклов в случайной перестановке)
\label{permutation}
Перестановка $\pi $ на $n$ элементах задается пошагово следующим образом: $\pi (k)$ выбирается случайно равновероятно из $\left\{1,\ldots ,n\right\}\backslash \left\{\pi (1),\ldots ,\pi (k-1)\right\}$. Ясно, что вероятность получения фиксированной перестановки будет $(n!)^{-1} $. Пусть случайная величина $\xi _{kn} $, $1\le k\le n$ равна 1, если в качестве $\pi (k)$ выбран элемент, замыкающий цикл перестановки, и равна 0 в противном случае. Воспользовавшись центральной предельной теоремой в форме Ляпунова, покажите, что  число циклов в случайной перестановке $X_{n} =\sum _{k=1}^{n}\xi _{kn} $  имеет асимптотически нормальное распределение.
\end{problem}

\begin{remark}
Интересные результаты об асимптотических свойствах группы перестановок содержатся в работах А.М. Вершика. 
В качестве модели перестановки $g$ длины $n$ можно использовать последовательность чисел $i_1=1,i_2,\dots,i_n$, сгруппированных в циклы, причем каждый цикл начинается с наименьшего в нем числа, и концы циклов указаны. Упорядочим циклы по возрастанию наименьших чисел. 
%Можно показать, что вероятность множества $A_k$ перестановок, у которых на $k$-м шаге оканчивается цикл в перестановке длины $n$, равна $(n-k+1)^{-1}$.
Пример множеств $A_k$: для перестановки $g = ((1,5),(2,3,8,6),(4,7))$, $g\in A_2 \bigcap A_6 \bigcap A_8$. Определим отображение $g\to (x_1(g),\dots,x_n(g))$, где $x_i(g)$ --- длина $i$-го цикла, нормированная на $n$, которая переводит меру, заданную на перестановках (по условию вероятность получения фиксированной перестановки будет $(n!)^{-1}$)  в дискретную меру на единичном  симплексе. Оказывается, у такой последовательности мер при $n$ стремящемся к бесконечности существует слабый  предел. Предельная мера порождается случайным рядом $\eta_i$, $i=1,2,\dots,$ с неотрицательными значениями и суммой $1$, причем значения $\eta_k/(\eta_k+\eta_{k+1}+\dots)$ независимы и равномерно распределены  на отрезке $[0,1]$. Иллюстрацией к приведенному результату является задача о "ломании палочки". Отрезок делится с равномерной вероятностью (точка деления распределеная равномерно на $[0,1]$). Левый отрезок фиксируется и затем ломается правый отрезок. Точки разлома выбираются последовательно с помощью последовательности независимых равномерно распределенных на $[0,1]$ случайных величин (или же можно сказать, что распределение длины левого отломанного куска палочки есть $\mathrm{Beta}(1, 1)$).  В работах А.М. Вершика показано, что такое случайное разбиение отрезка задает распределение длин отрезков, сопадающее с распределением длин циклов в рассмотренной модели перестановок. 

Доказательство этого факта технически нетривиально, например, используется принцип инвариантности 
Донскера--Прохорова.  
%Пусть $r_1,\dots,r_k$--- реализация случайных величин  $\xi_1,\dots,\xi_n$. %Пусть $x_i(g)$ --- нормированная на $n$ длина цикла, содержащего не вошедший в первые $i-1$ циклов элемент с наименьшим номером, если $i$ не превосходит числа циклов подстановки $g$.
%Нетрудно убедиться, что 
%\begin{equation}
%\label{versh}
%\mathbb{P}(g|nx_i(g)=r_i,1\leq i\leq m) = \left[n\prod_{i=1}^{m-1}(n-\sum_{k=1}^i r_k)\right]^{-1}.
%\end{equation}
%Видно, что в \eqref{versh} происходит отображение куба  $Q^n=\prod_{i=1}^{n}[0,1]$ на симплекс. Обозначим это отображение за $T_n$. 
%Тогда обратное отображение 
%$T_n^{-1}$, действующее из симплекса в $Q^n$:  
%$\{T_n^{-1}(x)\}_i = \frac{x_i}{1-x_1-\dots-x_{i-1}}$, если $x_k=0$, для всех $k\geq i$, то $\{T_n^{-1}(x)\}_i = 1$.
%Формулы применимы и к бесконечным последовательностям, то есть задают отображение бесконечного куба на бесконечномерный симплекс (сходящихся рядов с суммой единица). Отсюда (см. ссылки ниже) следует результат о том, что нормированные длины циклов случайных подстановок сходятся со скоростью геометрической прогрессии с показателем ${\rm e}^{-1}$
Подробные доказательства имеются в статьях Вершика А.\,М., Шмидта А.\,М.  Предельные меры, возникающие в асимптотической теории симметрических групп. I, ТВП, 1977, том 22, выпуск 1, 72–88 и Вершика  А.\,М., Шмидта А.\,А., Предельные меры, возникающие в асимптотической теории симметрических групп. II, ТВП, 1978, том 23, выпуск 1, 42–54. 



\end{remark}

\begin{problem}(Предельные меры А.М. Вершик и др.) 
\label{permutation1}
В качестве множества элементарных исходов рассматривается группа всевозможных 
подстановок (перестановок) $\mathbb{S}_n $ (симметрическая группа), $n\gg 1$. В этой 
группе $n!$ элементов. Припишем каждой подстановке одинаковую вероятность $1 
\mathord{\left/ {\vphantom {1 {n!}}} \right. \kern-\nulldelimiterspace} 
{n!}$.

\begin{enumerate}

\item Покажите, что математическое ожидание числа циклов есть $\approx 
\ln n$ (см. также задачу \ref{permutation}).

\item\Star В каком смысле нормированные длины циклов случайной подстановки 
убывают со скоростью геометрической прогрессии со знаменателем $e^{-1}$ ?

\item\Star Положим $$\rho _n \left( a \right)={\left| {\left\{ {g\in \mathbb{S}_n:\;n_{\max } \left( g \right)\le an} \right\}} \right|} \mathord{\left/ 
{\vphantom {{\left| {\left\{ {g\in S_n :\;n_{\max } \left( g \right)\le an} 
\right\}} \right|} {n!}}} \right. \kern-\nulldelimiterspace} {n!},$$ где 
\mbox{$n_{\max } \left( g \right)$~--- длина} максимального цикла в подстановке $g$. 
Покажите, что $\rho _n \left( a \right)$ удовлетворяет \textit{уравнению Дикмана--Гончарова} (40-е годы XX 
века):
\[
\rho _n \left( a \right)=\int\limits_0^a {\rho _n \left( {\frac{a}{1-t}} 
\right)dt}.
\]
\item\Star\,\,\Star Покажите, что начиная с некоторого большого числа $N$ 99{\%} 
натуральных чисел $n$, больших, чем $N$ обладают свойством
\[
n^{0.99}<p_1 \cdot ...\cdot p_{11} .
\]
Иначе говоря, у основной части (99{\%}) натуральных чисел основная часть 
(99{\%}) числа есть произведение наибольших простых делителей. Число 11 
возникло из-за того, что мы выбрали 99{\%} и 99{\%}.
\end{enumerate}
\end{problem}
\begin{ordre} Решение задач всех пунктов сводится к задаче о ``ломании палочки'' (см. замечание  к предыдущей задаче).
%См. задачи \ref{po_dir},\ref{stick} из раздела \ref{bayes}.
\end{ordre}

\begin{remark} См. Вершик А.М., Шмидт А.А. Предельные меры, возникающие 
в асимптотической теории симметрических групп // ТВП, Т. 22. № 1. 1977.С. 
72--88; Т. 23. № 1. 1978. С. 42--54; Вершик А.М. Асимптотическое 
распределение разложений натуральных чисел на простые делители // ДАН. 1986. 
Т. 289. № 2. С. 269--272; Tenenbaum G. Introduction to analytic and 
probabilistic number theory. Cambridge Univ. Press, 1995; Arratia R., 
Barbour A.D., Tavar\'{e} S. Logarithmic combinatorial structures: A 
probabilistic approach. EMS Monogr. Math., Eur. Math. Soc., Z\"{u}rich, 
2003.

Интересно отметить, появление по ходу решения задачи распределения 
(Пуассона--)Дирихле (см. задачу \ref{sobord}). Контекст, в котором это распределение возникает в 
решении, объясняет важность этого распределения в приложениях.
\end{remark}

\begin{problem}[Закон повторного логарифма]
Пусть $X_n$~--- независимые одинаково распределённые случайные величины с нулевым математическим ожиданием и единичной дисперсией. Пусть $S_n = X_1+\ldots+ X_n$. Докажите, что почти наверное
\[\underset{n \rightarrow \infty}{\overline{\lim} } \frac{S_n}{\sqrt{n \log_2 (\log_2 n)}} = \sqrt{2},\]
\[\underset{n \rightarrow \infty}{\underline{\lim} } \frac{S_n}{\sqrt{n \log_2 (\log_2 n)}} = -\sqrt{2}.\]
\end{problem}

\begin{remark} 

Хотя величина $\frac{S_n}{\sqrt{n \log_2 \log_2 n}}$ будет меньше, чем любое заданное  $\varepsilon$  с вероятностью, стремящейся к единице (это следует из ц.п.т), она будет бесконечное число раз приближаться сколь угодно близко к любой точке отрезка [$-\sqrt{2}, \sqrt{2}$] почти наверное [\ref{chiraiev}] т.2.

Приведем также некоторые смежные результаты из \cite{Gupta}. Пусть $X_1$, $X_2$, \dots, независимые однаково распределенные случайные величины с $F$ и $\gamma_n$ расходящаяся последовательность (к $+\infty$). Пусть $Z_{n,\gamma}  = \frac{S_n}{\gamma_n}$, $n\geq 1$, и $B(F,\gamma)$ --- множество всех предельных точек $Z_{n,\gamma}$. Тогда существует неслучайное множество $A(F,\gamma)$, такое, что с вероятностью $1$ $B(F,\gamma)$ совпадает c $A(F,\gamma)$.  В частности:
\begin{itemize}
\item Если $\gamma_n = n^{\alpha}$, $0<\alpha<1/2$, тогда для всех $F$ не вырожденных в $0$, $A(F,\gamma)$ равно всей расширеной вещественной оси, если оно содержит хотя бы одно конечное вещественное число.
\item Если $\gamma_n = n$ и если $A(F,\gamma)$ содержит хотя бы два конечных вещественных числа, то тогда оно должно содержать $\pm\infty$.
\item Если $\gamma_n=1$ тогдa $A(F,\gamma) \in \{\pm\infty\}$ тогда и только тогда, если для некоторого $a>0$, $\int_{a}^a \text{Re}\{1/(1-\psi(t))\}\,dt\leq \infty$, где $\psi(t)$ --- характеристическая функция $F$.
\item Пусть $\mathbb{E}(X_1)=0$, $\mathbb{D}X_1 = \sigma^2\leq{\infty}$, тогда при $\gamma_n=\sqrt{2n\log(\log n)}$, $A(F,\gamma) = [-\sigma,\sigma]$, а если $\mathbb{E}(X_1)=0$, $\mathbb{D}X_1 = \infty$, тогда при $\gamma_n=\sqrt{2n\log(\log n)}$, то $A(F,\gamma)$ содержит по крайней мере одну из $\pm\infty$
\end{itemize}
Пусть $m_n$ таковы, что $P(S_n\leq m_n)\geq 1/2$ и $P(S_n\geq m_n)\geq 1/2$. Тогда существует такая положительная последовательность $\gamma_n$, удовлетворяющая $$-\infty< \underset{n \rightarrow \infty}{\underline{\lim} } \frac{S_n-m_n}{\gamma_n}<
\underset{n \rightarrow \infty}{\overline{\lim} }\frac{S_n-m_n}{\gamma_n}<\infty$$ тогда и только тогда, когда для всех $c\geq 1$
$$\underset{x\to\infty}{\lim\inf}\frac{P(|X_1|>cx)}{P(|X_1|>x)}\leq c^{-2}.$$
\end{remark}



\begin{problem}[Логнормальное распределение]
\label{lognorm}
Рассмотрим частицу, которая при перемещении из одного места
в другое может разделиться на несколько меньших частиц вследствие соударения или другого воздействия. Зафиксируем произвольную точку данной частицы.  Обозначим за $K(t) \sim \Po(\lambda t)$ число отделившихся частей от изначально одной частицы относительно зафиксированной точки к моменту времени $t$. Первоначальный размер частицы равен $s_0$. Пусть $D_i$  -- доля частицы, отделившаяся при $i$-м соударении. Тогда размер
частицы в момент времени $t$ имеет вид:
\imgh{70mm}{log_norm_dist.pdf}{График функции плотности логнормального распределения при значении среднего $\mu = 0$ и различных значениях среднеквадратического отклонения (standard deviation).}
\[
Z(t) = s_0 \prod \limits_{i=1}^{K(t)} (1 - D_i),
\]
\[
S(t) = \ln Z(t) =  \mu +  \sum \limits_{i=1}^{K(t)} X_i,
\]
\noindent где $\mu = \ln s_0$, $X_i = \ln (1 - D_i)$ -- независимые с.в. (по предположению), причем $\Exp X = a$, $\Var X = \sigma^2$. 

Покажите, что $Z(t)$ стремится к логнормальному распределению при $n\to\infty$ (см. замечание):
\[
Z(t) \xrightarrow{d} \mathrm{LogN} \left(\mu + \lambda t a, \lambda t (a^2 + \sigma^2) \right).
\] 
\end{problem}

\begin{remark}
Cм. Mitzenmacher M., A Brief History of Generative Models for Power Law and Lognormal Distributions. Internet Math. 1 no. 2, 2003, 226--251.

Говорят, что $X$  имеет логнормальное распределение $\mathrm{LogN}(\mu,  \sigma^2)$ с параметрами $\mu$ и $\sigma$, если распределение задаётся плотностью вероятности, имеющей вид (см. Рис. \ref{Fig:log_norm_dist.pdf}):
\[
f_X(x) = \frac{1}{x \sigma \sqrt{2 \pi}} e^{- \frac{(\ln x - \mu)^2}{2 \sigma^2}}, \; x > 0.
\]
\end{remark}

Если после логарифмирования каждого элемента некоторого набора данных трансформированный набор данных нормально распределен, то исходные данные логарифмически нормально распределены.

Данное распределение хорошо моделирует процессы в случае, когда значение наблюдаемой переменной является случайной долей от значения предыдущего наблюдения.

%Примерами использования логнормального  распределения могут быть: 
%а) размеры и вес частиц, образуемых при дроблении;
%б) доход семьи;
%в) зарплата работников;
%г) долговечность изделия, работающего в режиме износа;
%д) размер банковского вклада;
%е) длины слов в языке;
%ж) длины передаваемых сообщений, размеры файлов или длины запросов к %базе данных.


%\subsection{ Безгранично делимые распределения }



\begin{problem}[Пуассоновский процесс \cite{1}]
\label{sec:poisson}
Пусть необходимо оценить, сколько билетов на метро одного вида $K(T)$ 
продается за одну рабочую смену длительностью $T$. Имеет место формула 
$$
K(T)=\max\Bigl\{ n:\; \sum\limits_{k=1}^{n} X_k<T \Bigr\} , 
$$
где $X_1, X_2, X_3,\ldots$~--- независимые одинаково распределенные по закону ${\rm Exp}(\lambda)$ с.в. ($X_k$ интерпретируется как время между 
$k-1$ и $k$ сделкой (продажей)). Покажите, что 
\begin{enumerate}
\item вероятность ${\mathbb P}(K(T+t)-K(T)=k)$, где $t\geqslant 0$ и $k=0,1,2,\ldots$, не зависит от $T\geqslant 0$; 

\item $\forall n\geqslant 1$, $0\leqslant t_1\leqslant t_2\leqslant \ldots\leqslant t_n$ 
 с.в. $\bigl\{ K(t_k)-K(t_{k-1})\bigr\}_{k=1}^{n}$ независимы; 

\item ${\mathbb P}(K(t)>1)=o(t),\quad t>0$; 

\item $\forall n\geqslant 1$, $0\leqslant t_1\leqslant t_2\leqslant \ldots\leqslant t_n$, 
$0\leqslant k_1\leqslant k_2\leqslant \ldots\leqslant k_n$, $k_1,\ldots, k_n\in {\mathbb N}\cup \{ 0\}$ 
\begin{multline*}
{\mathbb P}(K(t_1) = k_1,\ldots, K(t_n)=k_n)= \\
e^{-\lambda t_1} \frac{(\lambda t_1)^{k_1}}{k_1!}\cdot 
e^{-\lambda(t_2-t_1)} \frac{(\lambda(t_2-t_1))^{k_2-k_1}}{(k_2-k_1)!}\cdot \ldots \\
e^{-\lambda(t_n-t_{n-1})}\frac{(\lambda(t_n-t_{n-1}))^{k_n-k_{n-1}}}{(k_n-k_{n-1})!} , 
\end{multline*}
в частности, $K(T)\in \Po(\lambda T)$. 

\end{enumerate}
\end{problem}
\begin{remark}
Обратите внимание на задачу \ref{exp_eps}.
\end{remark}



\begin{problem}[Сложный пуассоновский процесс \cite{1}]
\label{sec:cpoisson}
В течение рабочего дня фирма осуществляет $K(T)\in \Po(\lambda T)$ сделок ($K(T)$ --- с.в., имеющая распределение Пуассона с параметром 
$\lambda T$, где $\lambda = 100 [\text{сделок/час}]$). Каждая сделка приносит доход $V_n\in R[a,b]$ ($V_n$ --- с.в., имеющая 
равномерное распределение на отрезке $[a,b]=[10, 100]$, $n$ --- номер сделки). Считая, что $K$, $V_1$, $V_2$, $\ldots$ --- 
независимые в совокупности с.в., найдите математическое ожидание и дисперсию выручки за день $Q(T)=\sum\limits_{k=1}^{K(T)} V_k$. Докажите соотношение для характеристической функции $Q(T)$
$$
\varphi_{Q(T)}(t)=\exp\{ \lambda T(\varphi_{V_k}(t)-1)\}. 
$$
Найдите (см. \cite{5}) такие $m(T)$ и $\sigma(T)$, что при $T\to\infty$
$$\frac{Q(T)-m(T)}{\sigma(T)}\xrightarrow{d} \mathcal{N}(0,1).$$
\end{problem}

\begin{ordre}

Примените формулу для условного математического ожидания  
$$
{\mathbb E}Y= \Exp{( {\mathbb E}(Y|X) )} 
$$
Установите справедливость следующего соотношения.
$$
\Var Y=\Var({\mathbb E}(Y|X))+{\mathbb E}(\Var(Y|X)) . 
$$

\end{ordre}


\begin{problem}
В течение трех лет фирма из предыдущей задачи работала $N=1000$ дней (длина рабочего дня и параметры спроса не менялись). 
Оцените (см. \cite{5}) распределение с.в. 
$$Q^N=\sum\limits_{k=1}^{N} Q_k(T), 
$$
где $Q_k(T)$ --- выручка за $k$–й день. Верно ли, что с.в. $Q^N$ и $Q_k(NT)$ одинаково распределены? 
\end{problem}

\begin{comment}
\begin{problem}
В течение года фирма осуществляет $K\in \Po(\lambda)$ сделок ($K$~--- с.в., имеющая распределение Пуассона с параметром  $\lambda=100000$ 
[сделок]). Каждая сделка приносит фирме прибыль $V_n\in R[a,b]$ ($V_n$ -- с.в., имеющая равномерное распределение на отрезке 
$[a,b]=[-50\$,100\$]$, $n$ -- номер сделки). Считая, что $K$, $V_1$, $V_2$, $\ldots$ --- независимые в совокупности с.в., оцените 
\begin{equation}
\label{ProbRatio}
\left. {\mathbb P}\Bigl(\sum\limits_{n=1}^{K} V_n\leqslant 0\Bigr)\right/{\mathbb P}\Bigl(\sum\limits_{n=1}^{K} V_n>0\Bigr). 
\end{equation}
\end{problem}
\end{comment}

\begin{problem}[Пуассоновский поток событий \cite{27,202}]
Рассмотрим интервал 
$\left[ {-N,N} \right]$ и бросим на него независимо и случайно (равномерно) \mbox{$M=\left[ {\rho N} \right]$} точек, где $\rho >0$ -- некоторая 
константа, называемая плотностью. Легко вычислить биномиальную вероятность 
$\PR_{N,M} \left( {k,I} \right)$ того, что в конечный интервал $I\subset 
\left[ {-N,N} \right]$ попадет ровно $k$ точек. Покажите, что для $\PR_{N,M} 
\left( {k,I} \right)$  при $N\to \infty $ верно 
\[
\PR\left( {k,I} \right)\mathop =\limits^{def} \mathop {\lim }\limits_{N\to 
\infty } \PR_{N,M\left( N \right)} \left( {k,I} \right)=\frac{\left( {\rho 
\left| I \right|} \right)^k}{k!}e^{-\rho \left| I \right|},
\quad
k=0,1,...
\]
Покажите также, что если $I_1 ,I_2 \subset \left[ {-N,N} \right]$ и $I_1 
\cap I_2 =\emptyset $, то
\[
\PR \left( {k_1 ,I_1 ;k_2 ,I_2 } \right)\mathop =\limits^{def} \mathop {\lim 
}\limits_{N\to \infty } \PR_{N,M\left( N \right)} \left( {k_1 ,I_1 ;k_2 ,I_2 } 
\right)=\PR\left( {k,I_1 } \right)\PR\left( {k,I_2 } \right).
\]
\end{problem}




\begin{problem}[Безгранично делимые случайные величины]
\label{sec:infdiv}
Случайная величина $X$ называется \textit{безгранично делимой}, если для любого натурального $n$ найдутся $n$ независимых одинаково распределенных случайных величин $X_{kn}$ таких, что $X = \sum_{k=1}^{n}{X_{kn}}$, где равенство понимается по распределению. Теорема Колмогорова--Леви--Хинчина утверждает, что если $X$ ~-- безгранично делимая с.в., то существует тройка
$$
 (b,c,\nu(dx)): \; 
c\geqslant 0, \; \int_{-\infty}^{\infty} \max(1,x^2)\, \nu(dx)<\infty, \; \nu(dx)\geqslant 0 \text{, такая что} 
$$
\[
\varphi_{X}(\mu)
=\exp \Bigl\{  i\mu b-c\mu^2\left.\right/2+
\int_{-\infty}^{\infty} \bigl( e^{i\mu x}-1-i\mu x I(|x|\leqslant 1) \bigr)\, \nu(dx) 
\Bigr\}, 
\]
где $\varphi_{X}(\mu)$~--- характеристическая функция $X$.

Верно ли, что любая безгранично делимая с.в. может быть представлена (имеет такое же распределение), как $N(m, \sigma^2) + Q$, где $Q$ -- с.в., имеющая сложное распределение Пуассона?
Определите $(b,c,\nu(dx))$ для $Q(T)$ из задачи 
\ref{sec:cpoisson}.

\end{problem}
\begin{remark} См. также задачу \ref{sobord}. 
В литературе можно часто встретить другую запись теоремы Колмогорова--Леви--Хинчина, например 
\[
\varphi_{X}(\mu)
=\exp \Bigl\{  i\mu b_0-c\mu^2\left.\right/2+
\int_{-\infty}^{\infty} \bigl( e^{i\mu x}-1-i\frac{\mu|x|}{1+x^2} \bigr)\, \theta(dx) 
\Bigr\}.
\]
Эквивалентность приведенной записи и записи в условии задачи следует из простого соотношения:
\[
\frac{y^2}{1+y^2}\leq \max\{1,y^2\}\leq \frac{2y^2}{1+y^2}.
\]

Из нетривиальных примеров безгранично делимых распределений упомянем распределения Стьюдента, Коши, логнормальное, показательное. Последнее является примером безгранично де\-ли\-мо\-го распределения, которое может быть представлено в виде бесконечной свертки (независимых, одинаково распределенных с.в.) не безгранично-делимых с.в. Для нормального  распределения и распределения Пуассона такое представление не возможно. Теоремы Крамера и Райкова говорят соответственно, что если свертка двух распределений нормальна (распределена по закону Пуассона), то компоненты должны также иметь нормальное распределение (распределение Пуассона) с другими параметрами. Подробнее об этом см. \cite{stoianov}. 

Приведем для справки несколько фактов из \cite{Gupta}.
\begin{enumerate}
\item
Пусть $F$ ---  распределение безгранично делимой случайной величины со средним $b$ и конечной дисперсией $D_F$ и  $\phi(t)$ --- характеристическая функция. Тогда 
$$
\log\phi(t)=ibt+\int_{-\infty}^{\infty}({\rm e}^{itx}-1-itx)\frac{d\mu(x)}{x^2},
$$
где $\mu$ конечная мера на действительной оси, более того $\mu(\mathbb{R}) = D_F$.
\item Теорема Goldie--Steutel. Пусть положительная случайная величина имеет плотность распределения $f(x)$, причем $f$ бесконечно дифференцируема и $(-1)^k f^{(k)}(x)\geq 0$, $\forall x>0$. Тогда $X$ имеет безгранично делимое распределение. 
\end{enumerate}

Также следует обратиться к Гнеденко Б.\,В., Колмогоров А.\,Н. Предельные распределения для сумм независимых случайных величин. — М.-Л.: ГТТИ, 1949.

\end{remark}

\begin{problem}[Центральная предельная теорема без требования существования дисперсий]
Пусть $X_1,X_2, 
\dots$ независимые одинаково распределенные случайные величины из распределения $F$ на действительнной прямой.  Верна следующая теорема 
\cite{Gupta}: cуществуют такие последовательности $\{a_n\},\{b_n\}$ что при $n\to \infty$ $$\frac{\sum_{i=1}^n(X_n-a_n)}{b_n}\xrightarrow{d} \mathcal{N}(0,1)$$ тогда и только тогда, когда функция $v(x) = \int_{[-x,x]}y^2\,dF(y)$ является медленно меняющейся на бесконечности, то есть для нее верно при любом $t>0$, что $$\lim_{x\to\infty}\frac{v(tx)}{v(x)}=1.$$ 
Покажите, что для  распределения Стьюдента с двумя степенями свободы выполняется приведенная теорема. Найдите нормировочные последовательности $\{a_n\}$ и $\{b_n\}$.
\end{problem}
 \begin{remark}
 Напомним, что по определению распределение Стьюдента c $k$ степенями свободы есть распределение случайной величины $$t = \frac{\xi_0}{\sqrt{\frac{1}{k}(\xi^2_1+\dots+\xi^2_k)}},$$
 где $\xi_j$ независимы, $\xi_j\in \mathcal{N}(0,1)$, $j=1,\dots,k$. Плотность распределения Стьюдента 
 $$
 p(x) = \frac{\Gamma((k+1)/2)}{\sqrt{\pi k}\Gamma(k/2)} \left(1+\frac{x^2}{k}\right)^{-(k+1)/2}.
 $$
 Распределение Стьюдента крайне важное распределение в математической статистике, которое, в частности, возникает в задачах, касаяющихся оценивания неизвестного математического ожидания выборки из нормального распределения с неизвестной дисперсией.
 \end{remark}

\begin{problem}\label{bluzd_ust}
Рассмотрим простую и классическую схему блуждания точки по прямой, соответствующую правилам игры в орлянку:
\[\eta (0)=0,\] 
\[\eta (t+1)=\left\{\begin{array}{cc} {\eta (t)+1,} & {p={1\mathord{\left/ {\vphantom {1 2}} \right. \kern-\nulldelimiterspace} 2} }, \\ {\eta (t)-1,} & {p={1\mathord{\left/ {\vphantom {1 2}} \right. \kern-\nulldelimiterspace} 2} }. \end{array}\right. \] 
Занумеруем в порядке возрастания все моменты времени, когда $\eta (t)=0$. Получим бесконечную последовательность $0=\tau _{0} <\tau _{1} <\tau _{2} <...$ Рассмотрим разности $\xi _{i} =\tau _{i} -\tau _{i-1} $, $i=1,2,...$ -- последовательность независимых одинаково распределенных с.в.
\begin{enumerate}
\item Найдите распределение $\xi _{i} =\tau _{i} -\tau _{i-1} $, т.е. $\PR\left\{\xi _{i} =2m\right\}$.
\item Покажите, что математическое ожидание с.в. $\xi _{i} =\tau _{i} -\tau _{i-1} $ равно бесконечности.
\end{enumerate}
\end{problem}
\begin{remark}
Этот результат можно проинтерпретировать так: среднее время до первого возвращения блуждания в 0 бесконечно. 
Тем не менее суммы $\tau _{n} =\sum _{i=1}^{n}\xi _{i}  $ при надлежащей нормировке подчинены предельному распределению: 
\[\mathop{\lim }\limits_{n\to \infty } \PR\left\{\frac{2\tau _{n} }{\pi n^{2} } <z\right\}=\left\{\begin{array}{cc} {\frac{1}{\sqrt{2\pi } } \int _{0}^{z}e^{-\frac{1}{2x} } x^{-\frac{3}{2} }  dx,} & {z>0} \\ {0,} & {z<0} \end{array}\right. .\]  
Случайная величина $\zeta$ имеет устойчивый закон распределения, если для любого $N>1$ найдутся  независимые случайные величины $\{Z_n\}_{n=1}^N$, распределение которых совпадает с  распределением $\zeta$,  и постоянные $a_N$ и $b_N$, такие что $\zeta = \frac{1}{a_N}(\sum_{i=1}^N z_n - b_N)$. Причем верно следующее утверждение (см. \cite{21}): $\zeta$ может быть пределом по распределению с.в. $\frac{1}{a_N}(\sum_{i=1}^N z_n - b_N)$ тогда и только тогда, когда $\zeta$ имеет устойчивый закон распределения.  Также верна теорема о каноническом представлении устойчивых законов  Леви--Хинчина \cite{1}: для того чтобы функция распределения была устойчивой, необходимо и достаточно, чтобы логарифм ее характеристической функции представлялся формулой:
\[\ln \varphi (t)=i\gamma t-c|t|^{\alpha } \left(1+i\beta \frac{t}{|t|} \omega (t,\alpha )\right),\] 
где $\gamma\in\mathbb{R}$, $-1\le \beta \le 1$, $0<\alpha < 2$, $c\ge 0$ и
\[\omega (t,\alpha )=\left\{\begin{array}{cc} {\tg\left(\frac{\pi }{2} \alpha \right),} & {\alpha \ne 1,} \\ {\frac{2}{\pi } \ln |t|,} & {\alpha =1}. \end{array}\right. \] 
Из данной теоремы следует, что такое распределение соответствует каноническому представлению с $\alpha ={1\mathord{\left/ {\vphantom {1 2}} \right. \kern-\nulldelimiterspace} 2} $, $\beta =1$, $\gamma =0$, $c=1$, и принадлежит семейству кривых Пирсона (показано Н.В. Смирновым).



Для решения этой и следующей задачи полезно познакомиться с \cite{28}. Для более глубокого погружения рекомендуется книга Петров В.В. Предельные теоремы для сумм независимых случайных величин. М.: Наука, 1987. 
\end{remark}

\begin{problem}
Верно, что устойчивые законы  распределения являются также и  безгранично делимыми? 
 Покажите, что Пуассоновский закон распределения безгранично делимый, но не устойчивый. 
\end{problem}
\begin{remark}
См. предыдущую задачу и \cite{stoianov}.
\end{remark}

\begin{problem}[Вывод распределения Хольцмарка \cite{202,28}] Рассмотрим шар \mbox{радиуса $r$} с центром в начале координат и $n$ звезд (точек), расположенных в нем случайно и независимо друг от друга. Пусть каждая звезда имеет единичную массу и звезды распределены по Пуассону так, что ожидаемое число их в объеме $V$ равно $\lambda V$. Обозначим $X_{i}$ (вектор) гравитационное поле, соответствующее $i$-й звезде, и положим $S_{n} =X_{1} +...+X_{n} $. Устремим $r$ и $n$ к бесконечности так, чтобы $$\frac{4}{3} \pi r^{3} n^{-1} \to \lambda.$$ Показать, что плотность распределения вектора $S_{n}$ зависит только от модуля поля и стремится к плотности симметричного устойчивого распределения Хольцмарка с $\alpha = 3/2$ (см. \cite{28}). Можно показать, что задача по существу не изменится, если массу каждой звезды считать с.в. с единичным математическим ожиданием и массы различных звезд предполагать взаимно независимыми с.в., не зависящими также от их расположения.
\end{problem}
\begin{remark}
Смотрите также Кендалл М., Моран П. Геометрические вероятности. М.: Наука, 1972.


\end{remark}


\begin{problem} Пусть $n$ единичных масс равномерно распределены в точках $X_1, \ldots, X_n$ на отрезке $[-n, n]$. На единичную массу в начале координат действует гравитационная сила $$f_n = \sum_{k =1}^n \frac{{\rm sign}(X_k)}{X_k^2}.$$ 


Покажите, что $$\mathbb{E}\left[\exp{\left(it f_n\right)}\right] \rightarrow \exp{\left(-c\sqrt{|t|}\right)}, \quad n\to\infty.$$
\end{problem}
\begin{ordre}
Так как $X_k$ распределено равномерно, то (см. \cite{2013})
\begin{equation*}
\mathbb{E}\left[\exp{\left(it \frac{{\rm sign}(X_k)}{X_k^2}\right) }\right] = \int_{-n}^n \exp{\left(it\frac{{\rm sign}(x)}{x^2}\right) }\frac{dx}{2n} = \frac{1}{n}\int_{0}^n \cos{\left(\frac{t}{x^2}\right)}dx.
\end{equation*}
\end{ordre}

\begin{problem} \label{sobord} Покажите, что если $X\left( t \right)$ -- 
процесс Леви, то:
\[
\exists \;\;\left( {b,\;c,\;\nu \left( {dx} \right)} \right),\,
c\ge 0,\,\nu \left( {dx} \right)\ge 0:
\quad
\int_{-\infty }^\infty {\max \left( {1,\;x^2} \right)\;}
\nu \left( {dx} 
\right)<\infty ,\]
\[
\forall \;\;t\ge 0 \quad 
\phi _{X\left( t \right)} \left( \mu \right)=\mathbb{E}{\rm e}^{i\mu X\left( t 
\right)}=
\]
\[\exp \left\{ {t\left[ {i\mu b-{c\mu ^2} \mathord{\left/ {\vphantom 
{{c\mu ^2} 2}} \right. \kern-\nulldelimiterspace} 2+\int_{-\infty }^\infty 
{\left( {e^{i\mu x}-1-i\mu xI\left( {\left| x \right|<1} \right)} \right)\nu 
\left( {dx} \right)} } \right]} \right\}.
\]
\end{problem}
\begin{remark}
 Процессом Леви  $\left\{ {X\left( t 
\right)} \right\}_{t\ge 0} $ называется стохастически непрерывный случайный 
процесс, удовлетворяющий следующим условиям (с небольшими оговорками (cadlag--процесс) -- см.  Applebaum D. B. Levy processes and stochastic calculus, Cambridge University Press 2nd ed., 2009):

\begin{enumerate}
\item $X\left( 0 \right)\mathop =\limits^{\text{п.н.}} 0;$
\item для любых $t>s\ge 0$ распределение $X\left( t \right)-X\left( s \right)$ зависит только от $t-s$ (также говорят, что $\left\{ {X\left( t \right)} \right\}_{t\ge 0} $ имеет стационарные приращения или $\left\{ {X\left( t \right)} \right\}_{t\ge 0} $ -- однородный);
\item для любых $n\in {\rm N}$, $0\le t_0 \le t_1 \le ...\le t_n $ выполняется: $X\left( {t_1 } \right)-X\left( {t_0 } \right)$, {\ldots}, $X\left( {t_n } \right)-X\left( {t_{n-1} } \right)$ -- независимые в совокупности с.в. (также говорят, что $\left\{ {X\left( t \right)} \right\}_{t\ge 0} $ имеет независимые приращения).
\end{enumerate}
См. А.Н. Ширяев, Основы финансовой 
стохастической математики, М.: ФАЗИС, 2004, т. 1, глава 3, п. 
1.
\end{remark}

\begin{remark}
Отметим, что мера $\nu$ может быть сигнулярна  в нуле, и такие процессы могут иметь бесконечно много  скачков на любом непустом временном интервале.

Процесс с положительными независимыми приращениями, обладающий свойствами б), в), а также тем свойством, что приращения его положительны, называется субординатором. Если $X(t)$-- субординатор, то cуществуют такие $a\in\mathbb{R}$ и мера $\nu(dz)$, что $\int_{0}^{\infty}x\nu(dz)<\infty$ и для всех $\lambda\geq 0$ и $t\geq0$ верно
 $$\mathbb{E}\exp{(-\lambda X(t))} = \exp\left(-t\Phi(\lambda)\right),$$ 
 где $\Phi(\lambda) = b\lambda+\int_{0}^{\infty}(1-{\rm e}^{-\lambda z}) \nu(dz)$ и 
 $b = a-\int_{0}^{1} x\nu(dx)$.
 Такой процесс совершает бесконечное счетное  число скачков на любом конченом интервале времени.  
 %Субординатор не может иметь неподвижных точек разрыва, поскольку множество точек разрыва счетно и, следовательно не может быть инвариантным относительно сдвигов (что отражено в его представлении, приведенном выше).
 
 Рассмотрим пример субординатора Морана (также его называют гамма-субординатором). Субординатором Морана называется возрастающий процесс $G(\alpha),\, \alpha>0$ с независимыми стационарными приращениями $G(\alpha_2)-G(\alpha_1), \alpha_2>\alpha_1$ с плотностью распределения  $g_{\alpha_2-\alpha_1}(x)$, где $g_{\alpha}(x)= x^{\alpha-1} {\rm e}^{-x}\Gamma(\alpha)$.
Для субординатора Морана $a=0$, $\nu(dx) = x^{-1}{\rm e}^{-x}dx$. 
Таким образом, субординатор Морана --- это процесс Леви, отвечающий гамма-распределению.

Этот процесс может быть использован для генерирования последовальностей случайных величин с распределением Пуассона--Дирихле, а именно распределение Пуассона--Дирихле есть распределение упорядоченных по убыванию нормированных скачков субординатора Морана. Распределение Пуассона--Дирихле чрезвычайно часто возникает в разнообразных приложениях, особенно в популяционной генетике и экономике (оно является равновесным распределением для ряда эволюционных моделей). С другой стороны распределение Пуассона--Дирихле появляется как предел распределения Дирихле на конечномерных симплексах при размерности симплекса стремящейся к бесконечности и однаковыми параметрами, нормированными на размерность симплекса. См. также \cite{202}.
\end{remark}

\begin{problem}(модель Кокса--Росса--Рубинштейна) 
\label{sec:levilong}
Пусть на ``идеализированном'' фондовом рынке имеется всего две ценные 
бумаги, и торговля осуществляется всего в два момента времени. Пусть цена 
первой бумаги $S$ (будем называть её акцией (stock)) известна в первый 
момент. Цена второй бумаги $C$ (будем называть ее call -- опционом 
европейского типа) не известна в первый момент. Пусть с ненулевой 
вероятностью $p>0$ (мы это $p$ не знаем, но от него ничего зависеть в итоге 
не будет) к моменту времени 2 цена акции вырастет в $u>1$ (up) раз и с 
вероятностью $1-p$ цена акции ``вырастет'' в $d<1$ (down) раз, т.е. упадет. 
Пусть также известны возможные цены опциона во второй момент: $C_u $, если 
акция выросла в цене, и $C_d $,  если акция упала в цене. Для простоты будем 
считать, что банк работает с нулевым процентом, т.е. класть деньги в банк, в 
расчете на проценты, бессмысленно. Говорят, что рынок 
безарбитражный, если не существует таких $k_S $, $k_C $, что
\[
X\left( 1 \right)=k_S S+k_C C=0,
\]
\[
\PR\left( {X\left( 2 \right)\ge 0} \right)=\PR\left( {k_S S\left( 2 \right)+k_C 
C\left( 2 \right)\ge 0} \right)=1,\] причем $\PR\left( {X\left( 2 \right)>0} 
\right)>0.
$

Докажите, что рассматриваемый рынок безарбитражный тогда и только тогда, 
когда

$$C=\tilde {p}C_u +\left( {1-\tilde {p}} \right)C_d ,$$ где $\tilde 
{p}=\frac{1-d}{u-d}$.
\end{problem}
\begin{remark}
 Опцион характеризуется датой исполнения (в нашем 
случае -- момент времени 2) и платежами в момент исполнения ($C_u $ и $C_d 
)$. Причем эти платежи -- заранее известные функции от цены акции в этот 
момент (введение опционов было мотивировано желанием ``хеджироваться'', 
страховаться от нежелательных изменений цен акций). Основная задача 
заключается в установлении ``справедливой'' цены опциона $C$ в момент 
времени 1 (см. \cite{21} глава 7, {\S} 
11; А.Н. Ширяев, Основы финансовой стохастической математики, М.: ФАЗИС, 
2004, т. 1, т. 2).

Не 
имея в начальный момент 1 капитала $X\left( 1 \right)=0$, но, проделав 
некоторую махинацию (продав одних ценных бумаг (в зависимости от специфики 
рынка, иногда разрешается ``вставать в короткую позицию'' -- продавать 
ценные бумаги, не имея их в наличии; приобретая при этом в долг) и купив на 
вырученные деньги других бумаг), можно в момент времени 2 гарантированно 
ничего не проиграть, и при этом с ненулевой вероятностью выиграть (не 
уточняя сколько -- поскольку, ``прокручивая'' по имеющемуся арбитражу 
(пропорционально увеличивая коэффициенты $k_S $, $k_C )$ сколь угодно 
большую сумму, можно получить с ненулевой вероятностью сколь угодно большой 
выигрыш).

Величина $\tilde {p}$ называется мартингальной вероятностью (смысл 
такого определения будет раскрыт в следующих задачах) и задает мартингальную 
меру. Если существует единственная мартингальная мера, то рынок называется 
полным. На полном рынке неизвестная цена опциона $C$ в начальный 
момент определяется однозначно и может интерпретироваться как ``справедливая 
цена''.
\end{remark}
\begin{problem}(биномиальная $n$-периодная 
модель Кокса--Росса--Рубинштейна)\label{n-period} Предложите обобщение рынка и 
соответствующих понятий из задачи \label{sec:levilong} на $n$-периодный рынок. С возможностью класть 
деньги в банк под процент $r-1$ ($d<r<u)$ -- за один период (под такой же 
процент брать деньги из банка). Опцион исполняется в заключительный $(n+1)$-ый 
момент. Платежи по опциону в этот момент известны и описываются известной 
функцией $\bar {C}\left( S \right)$ (например, для указанного в предыдущей задаче 
опциона $\bar {C}\left( S \right)=\max \;\left\{ {0,\;S-X} \right\})$, 
т.е. $C_k \left( {n+1} \right)=\bar {C}\left( {S_k \left( {n+1} \right)} 
\right)$, где $k$ -- состояние в котором находится рынок в момент времени 
$n+1$. Считайте, что $S_k \left( {n+1} \right)=Su^kd^{n-k}$, т.е. $k$  
характеризует то, сколько раз акция поднималась в цене. Также как и в предыдущей задаче, 
требуется определить ``справедливую'' цену опциона.
Обоснуйте формулу Кокса--Росса--Рубинштейна:
\begin{equation}
\label{koks}
C=\frac{1}{r^n}\sum\limits_{k=0}^n {\left( {\begin{array}{l}
 n \\ 
 k \\ 
 \end{array}} \right)} \tilde {p}^k\left( {1-\tilde {p}} \right)^{n-k}\bar 
{C}\left( {Su^kd^{n-k}} \right),
\end{equation}
где $\tilde {p}=\frac{r-d}{u-d}.$
\end{problem}

\begin{remark}
Отметим, что в условиях задачи $X$ называется ценой исполнения опциона и считается 
известной. Собственно, вид функции $\bar {C}\left( S \right)=\max \left\{ 
{0,\;S-X} \right\}$ проясняет смысл опциона. Опцион дает право купить (у 
того, кто продал нам опцион) в момент исполнения опциона акцию по цене $X$. 
Если акция стоит дороже в этот момент, то, конечно, мы этим правом 
воспользуемся и получим прибыль (продавец опциона обязан продать нам акцию). 
Если же цена акции меньше цены исполнения опциона, то нам уже не выгодно 
покупать акцию по более дорогой цене, чем рыночная, и мы не исполняем 
опцион, т.е. ничего не делаем (ведь опцион дает нам право, ни к чему не 
обязывая). См. также Белопольская Я., Теория арбитража в непрерывном времени, C. Петербург: СПбГАСУ, 2006, Булинский А.В., Случайные процессы. Примеры, задачи и упражнения. М: МФТИ, 2010.
\end{remark}
\begin{problem}(континуальная биномиальная модель Блэка--Шоулса) 
\label{bw-cont}
Уместим на отрезке времени $\left[ {0,\;t} \right] \quad n+1$  моментов 
(промежутки между которыми одинаковы), в которые осуществляется торговля 
согласно задаче \ref{n-period}. Введем два параметра: $a$ -- снос, $\sigma ^2\ge 0$ -- 
волатильность (дисперсия). Положим,
\[
\mu =a+\frac{\sigma ^2}{2},
\quad
r=\exp \left( {\mu \frac{t}{n}} \right),
\quad
u=\exp \left( {\sigma \sqrt {\frac{t}{n}} } \right),
\]
\begin{equation}
\label{bw}
d=\exp \left( {-\sigma \sqrt {\frac{t}{n}} } \right),
\quad
\tilde {p}=\frac{r-d}{u-d}\approx \frac{1}{2}\left( {1+\frac{a}{\sigma 
}\sqrt {\frac{t}{n}} } \right).
\end{equation}

Переходя к пределу при $n\to \infty $ в формуле \eqref{koks} (с $\bar {C}\left( S 
\right)=\max\left\{ {0,\;S-X} \right\})$, согласно \eqref{bw}, получите формулу 
Блэка--Шоулса для справедливой цены опциона в ``континуальной биномиальной 
модели''. Почему вводится именно два параметра (а не один, три и т.д.)? 
Почему
\[
r-1\sim \frac{\mu t}{n},
\quad
u-1\approx \sigma \sqrt {\frac{t}{n}} ,
\quad
d-1\approx -\sigma \sqrt {\frac{t}{n}} ?
\]
Возможны ли какие-нибудь другие осмысленные варианты соотношений типа \eqref{bw}, 
при которых будет существовать предел при $n\to \infty $ в формуле \eqref{koks} (для 
простоты вычислений считайте, что $\bar {C}\left( S \right):=S)$?
\end{problem}
\begin{remark} (Донскер--Прохоров--Скороход--леКам--Варадарайн). \\
Имеет место слабая сходимость при $n\to \infty $ описанного случайного 
блуждания в дискретном времени к случайному процессу (в непрерывном 
времени), называемому геометрическим броуновским движением. Детали см., 
например, в книге Биллингсли П. Сходимость вероятностных мер. -- М.: Наука, 
1977. Также см. главу 12 в \cite{Gupta}. Также для решения задачи полезно использовать \cite{101}.
\end{remark}
\begin{problem} (броуновское движение (процесс Башелье) и винеровский процесс). \label{bachelie}
Исходя из формулы \eqref{koks}, имеем, что ``рынок'' при определении 
``справедливой'' цены опциона считает, что случайный процесс $S\left( m 
\right)$ (цена акции в момент времени $m)$ эволюционирует согласно 
биномиальной модели с неизменными параметрами $d$, $u$, $\tilde 
{p}$. Построим случайный процесс $S\left( t \right)$ (в непрерывном 
времени), исходя из процесса $S\left( m \right)$, заданного в дискретном 
времени предельным переходом, аналогичным задаче \ref{bw-cont}. Как уже отмечалось,
полученный процесс $S\left( t \right)$ называют геометрическим 
броуновским движением (или случайным процессом Башелье--Самуэльсона) с 
параметрами $a$, $\sigma ^2\ge 0$, а случайный 
процесс $B\left( t \right)=\ln \left(\frac{S\left( t \right)}{S\left( 0 \right)}\right)$ 
-- броуновским движением с параметрами $a$, $\sigma ^2\ge 0$. Если 
$a=0$, $\sigma ^2=1$, то такое броуновское движение имеет специальное 
название -- винеровский процесс $W\left( t \right)$. Покажите, что 
броуновское движение является процессом Леви (см. задачу \ref{sobord}). Найдите триплет $\left( 
{b,\;c,\;\nu \left( {dx} \right)} \right)$.
\end{problem}
\begin{remark}  Один 
из альтернативных способов введения мартингальных вероятностей $\tilde {p}$ 
основывается на, так называемых, ``риск нейтральных'' или ``мартингальных'' 
соображениях. Заключающихся в том, что $\tilde {p}$ выбирается исходя из 
равенства $$\mathbb{E}_{\tilde {p}} \left[ {Su^{\sum\limits_{k=1}^n {x_k } 
}d^{n-\sum\limits_{k=1}^n {x_k } }} \right]=Sr^n,$$ где независимые одинаково распределенные с.в.  $x_k$ имеют распределение $Be\left( {\tilde {p}} \right)$, или исходя из того, что процесс приведенной 
(продисконтированной) стоимости акции ${\tilde {S}\left( m \right)=S\left( m 
\right)} \mathord{\left/ {\vphantom {{\tilde {S}\left( m \right)=S\left( m 
\right)} {r^m}}} \right. \kern-\nulldelimiterspace} {r^m}$ должен быть 
мартингалом относительно мартингальной меры $\tilde {p}$ (отсюда и 
название), т.е.$$\mathbb{E}_{\tilde {p}} \left( {\left. {\tilde {S}\left( {m+1} 
\right)} \right|\left( {\tilde {S}\left( 1 \right),...,\tilde {S}\left( m 
\right)} \right)} \right)=\tilde {S}\left( m \right).$$

Параметры геометрического броуновского движения имеют следующий смысл: 
$$a=\frac{1}{t}\mathbb{E}\left[ {\ln \frac{S\left( t \right)}{S\left( 0 \right)}} 
\right],$$ 
$$\sigma ^2=\frac{1}{t}\Var\left[ {\ln \frac{S\left( t 
\right)}{S\left( 0 \right)}} \right].$$
%Также для решения задачи полезно использовать \cite{101}.
\end{remark}
\begin{problem} \label{geomB}(геометрическое броуновское движение или процесс 
Башелье--Самуэльсона) В условиях задачи \ref{bachelie} покажите, что геометрическое 
броуновское движение удовлетворяет следующему стохастическому 
дифференциальному уравнению:
\[
dS\left( t \right)=\left( {a+\frac{\sigma ^2}{2}} \right)S\left( t 
\right)dt+\sigma S\left( t \right)dW\left( t \right),
\]
которое определяет $S\left( t \right)$, как случайный процесс, 
удовлетворяющий соотношению:
\[
S\left( t \right)=S\left( 0 \right)+\left( {a+\frac{\sigma ^2}{2}} 
\right)\int\limits_0^t {S\left( \tau \right)dt} +\sigma \int\limits_0^t 
{S\left( \tau \right)dW\left( \tau \right)},
\]
где второй интеграл понимается в смысле Ито (см.  \cite{101}).
\end{problem}
\begin{remark} Случайные процессы, которые задаются 
стохастическими дифференциальными уравнениями наподобие рассмотренного, 
задают по определению диффузионный процесс Ито (см. Оксендаль Б., 
Стохастические дифференциальные уравнения, М.: Мир, 2003, главы 7, 8).
\end{remark}
\begin{problem} (формула Ито) Через конечные разности покажите, что дифференциал процесса $S(t)$ (см.предыдущую задачу) выглядит как: 
\[
\Delta S\left( t \right)=\Delta \left( {S\left( 0 \right)\exp \left( 
{B\left( t \right)} \right)} \right)=
\]
\[
=\left( {a+\frac{\sigma ^2}{2}} \right)S\left( t \right)\Delta t+\sigma 
S\left( t \right)\Delta W\left( t \right),
\]
где
\[
\Delta S\left( t \right)=S\left( {t+h} \right)-S\left( t \right),
\quad
\Delta W\left( t \right)=W\left( {t+h} \right)-W\left( t \right),
\]
\[
\Delta t=t+h-t=h,
\quad
h>0.
\]
Предложите общий вид формулы для $\Delta g\left( {t,\;W\left( t \right)} 
\right)$ ($dg\left( {t,\;W\left( t \right)} \right))$.
\end{problem}

\begin{remark}
Воспользуйтесь приближением:
\[
\Delta \left( {S\left( 0 \right)\exp 
\left( {at+\sigma W\left( t \right)} \right)} \right)\simeq 
\]
\[
\simeq aS\left( 0 \right)\exp \left( {at+\sigma W\left( t \right)} 
\right)\Delta t+\sigma S\left( 0 \right)\exp \left( {at+\sigma W\left( t 
\right)} \right)\Delta W\left( t \right)+
\]
\[
+\frac{\sigma ^2}{2!}S\left( 0 \right)\exp \left( {at+\sigma W\left( t 
\right)} \right)\left( {\Delta W\left( t \right)} \right)^2\simeq 
\]
\[
\simeq \left( {a+\frac{\sigma ^2}{2}} \right)S\left( 0 \right)\exp \left( 
{at+\sigma W\left( t \right)} \right)\Delta t + 
\]
\[+\sigma S\left( 0 \right)\exp 
\left( {at+\sigma W\left( t \right)} \right)\Delta W\left( t \right)
\]

См. Б. Оксендаль, Стохастические 
дифференциальные уравнения, М.: Мир, 2003, глава 4, а также 
Я. Белопольская, Теория арбитража в непрерывном времени, C. Петербург: СПбГАСУ, 2006.
\end{remark}
%\begin{remark}См. Paul Glasserman
%Monte Carlo Methods in Financial Engineering. 
%Applications of mathematics: stochastic modelling and applied probability, Springer, 2004; Carl Graham, Denis Talay Stochastic Simulation and Monte Carlo Methods: Mathematical Foundations of Stochastic Simulation, Springer, 2013; M.B. Giles. Multi-level Monte Carlo path simulation, Operations Research, 56(3):607-617, 2008.
%\end{remark}


\begin{problem}(Пуассоновский процесс). Покажите, что 
пуассоновский процесс и сложный пуассоновский процесс (см. задачи \ref{sec:poisson} и \ref{sec:cpoisson}) 
являются процессами Леви (см. задачу \ref{sobord}). Найдите триплеты $\left( {b,\;c,\;\nu \left( {dx} 
\right)} \right)$.
\end{problem}

\begin{remark} Исходя из задач \ref{sec:infdiv}, \ref{sec:levilong}
можно выдвинуть гипотезу, что любой процесс Леви может быть получен как 
сумма броуновского движения и сложного пуассоновского процесса. Такого рода 
утверждение действительно имеет место и называется представлением 
Леви--Ито. Однако вместо сложного пуассоновского процесса в этом 
представлении в общем случае следует брать процесс, который может быть 
получен как ``предел'' сложных пуассоновских процессов (См. K.-I. 
Sato, Levy processes and infinitely divisible distributions. Cambridge, 
1999). Сделанное замечание отчасти поясняет важность трех ключевых 
распределений теории вероятностей и трех типов процессов Леви: вырожденного (когда дисперсия равняется нулю)  
нормального (вырожденное 
броуновское движение), нормального (броуновское движение), распределения 
Пуассона (пределы сложных пуассоновских процессов). Помимо того, что при 
наиболее естественных предположениях для приложений имеет место сходимость к 
одному из этих трех безгранично делимых законов (например, аналогом теоремы 
Пуассона будет теорема Григелиониса \cite{4}), они в некотором смысле являются 
базисом: любое распределение, которое может возникать в пределе при 
суммировании независимых одинаково распределенных с.в., ``может быть 
получено'' исходя из этих трех базовых распределений. Приведенная задача частично 
проясняет, в каком смысле любое распределение ``может быть так получено''.
\end{remark}

\begin{problem} (Сложный процесс восстановления). Если в 
определении сложного пуассоновского процесса заменить пуассоновский процесс  общим процессом восстановления $$\tilde {K}\left( t \right)=\max \left\{ 
{k:\;\;\sum\limits_{i=1}^k {T_i } <t} \right\},$$ где $\left\{ {T_i } 
\right\}$ независимые одинаково распределенные с.в., но не обязательно, что $T_i \in \mbox{Exp}\left( \lambda 
\right))$, то получится сложный процесс восстановления $\tilde 
{Q}\left( t \right)$ (также играющий важную роль в разнообразных 
приложениях). 

Считая известными $\mathbb{E}\tilde {K}\left( t \right)$, $\Var\tilde 
{K}\left( t \right)$ и $\mathbb{E}V_i $, $\Var V_i $, определите $\mathbb{E}\tilde {Q}\left( t 
\right)$, $\Var\tilde {Q}\left( t \right)$. 

Используя ц.п.т. (в форме А.А. 
Натана \cite{5}, а также см. ориганальные работы  Anscombe, F. Large sample theory of sequential estimation, Proc. Cambridge Philos. Soc., 48, 1952 и  Renyi, A. On the asymptotic distribution of the sum of a random number of
independent random variables, Acta Math. Hung., 8, pp. 193–199,1957) найдите (приближенно) распределение сечения процесса $\tilde 
{Q}\left( t \right)$ при $t\gg 1$.
\end{problem}



\begin{problem}
В первобытной общине для более сбалансированного поддержания брачного отношения один ко многим введено следующее правило: каждая пара (мужчина-женщина) может размножаться до появления ребенка мужского пола. Изначально в общине было отношение количества человек мужского пола к женскому близкое к единице. Каким будет математическое ожидание отношения численности полов через 100 лет? Оцените также отклонение от среднего значения. Считать, что семейная пара размножается в точности один раз в году и производит одного ребенка (с равной вероятностью мальчика или девочку), пара начинает размножение по достижении 20 лет, мужчина с равной вероятностью от одной до трех женщин берет в жены.      
\end{problem}

\begin{problem}(поток Эрланга $m$-го порядка).\label{GammaFunc} Будет ли процессом
Леви (см. задачу \ref{sobord}) $m$ раз ``просеянный'' пуассоновский процесс $E_m \left( t \right)$ с 
параметром $\lambda >0$

$$E_m \left( t \right)=\max \left\{ {k:\;\;\sum\limits_{i=1}^k {T_i } <t} 
\right\},$$ 
где i.i.d. с.в. $T_i \in % \underbrace {\mbox{Exp}\left( \lambda 
%\right)+...+\mbox{Exp}\left( \lambda \right)}_m\mathop %=\limits^d 
\Gamma 
\left( {\lambda ,m} \right)$, $\lambda >0$?
\end{problem}

\begin{remark}
По определению $\Gamma \left( {\lambda ,m} 
\right) \mathop =\limits^d \underbrace {\mbox{Exp}\left( \lambda \right)+...+\mbox{Exp}\left( 
\lambda \right)}_m$ -- гамма распределение (сумма независимых показательных 
с.в.).
\end{remark}

\begin{problem} В модели Блэка--Шоулса--Мертона эволюция цены акции 
описывается геометрическим броуновским движением $$S\left( t \right)=S\left( 
0 \right)\exp \left( {at+\sigma W\left( t \right)} \right),$$ где $W\left( t 
\right)$ -- винеровский процесс ($\sigma >0)$. С помощью эргодической теоремы 
для случайных процессов оцените неизвестный параметр $a$, если известна 
реализация процесса $S\left( t \right)$ на достаточно длинном временном 
отрезке $\left[ {0,\;T} \right]$. Предложите способ оценки неизвестного 
параметра $\sigma $. Имеет ли смысл пытаться строить по $S\left( t \right)$ 
процесс $Y\left( t \right)=f\left( {S\left( t \right)} \right)$, то есть подбирать 
функцию $f(\cdot)$ так, чтобы $Y\left( t \right)$ был 
эргодичен по математическому ожиданию и $\mathbb{E}f\left( {S\left( t \right)} 
\right)=\sigma ^2$?
\end{problem}
\begin{remark}
%Говорят, что случайный скалярный процесс второго порядка $X(t)$ с постоянным математическим ожиданием $m_X$ обладает свойством \textit{эргодичности в среднем квадратичном по математическому ожиданию},
%если при $T \rightarrow \infty$:
%$$
%	\langle X\rangle_T \xrightarrow{\text{c.к.}} m_X, 
%$$
%где $\langle X\rangle_T = \dfrac{1}{T}\int\limits_{0}^{T}X(t)dt$.

%\noindent Эргодическая теорема: из эргодичности процесса в среднем квадратичном следует его эргодичность и по вероятности (при  $T \rightarrow  \infty$):
%$$
%	\langle X\rangle_T \xrightarrow{p} %m_X
%$$
%Выполнение условия эргодичности случайного процесса позволяет оценить $m_X$ величиной $\langle X\rangle_T$ имея лишь одну реализацию на достаточно длинном промежутке времени $T$.

%\noindent Прямая проверка условия эргодичности по матожиданию может оказаться весьма сложной задачей для произвольного процесса. Зачастую можно проверить условие выше, при помощи следующей теоремы:
%Необходимым и достаточным условием эргодичности в среднем квадратичном по математическому ожиданию случайного процесса второго порядка $X(t)$ с постоянным матожиданием $m_X$ является сущесвование предела:
%$$
%	\lim\limits_{T \rightarrow +\infty}
%	\dfrac{1}{T^2}\int\limits_{0}^T\int\limits_{0}^T %\mathrm{Cov}_X(t_1,t_2)
%	dt_1 dt_2 = 0
%$$
%\noindent В случае стационарных процессов (в широком смысле слова) достаточным условием для выполнения условия теоремы, а следовательно и выполнения условия эргодичности является существование предела:
%$$
%	\lim\limits_{|t_1 - t_2| \rightarrow +\infty} %\mathrm{Cov}_X(t_1,t_2) = 0
%$$

Модель Блэка--Шоулса--Мертона широко использовалась на практике (см. задачи \ref{bachelie}, \ref{geomB}). В 1990 г. М. 
Шоулс и Р. Мертон за свою работу были награждены нобелевской премией по 
экономике. Сейчас популярным классом моделей является $S\left( t 
\right)=S\left( 0 \right)\exp \left( {L\left( {\tau \left( t \right)} 
\right)} \right)$, где $L(\cdot)$ -- процесс Леви (см. задачу 39), $\tau 
\left( t \right)$ -- случайный процесс, не зависящий от $L(\cdot)$, с возрастающими почти наверное траекториями.
\end{remark}

\begin{problem}[статистическая эргодическая теорема фон Неймана для динамических систем \cite{21}, т.2] Пусть $(X,\sigma(X))$ -- измеримое пространство, $$T:X\to X:\forall B\in \sigma(X)\to \mu(T^{-1}(B)) = \mu(B).$$ Тогда $$\forall f\in L_2(X)\to\frac{1}{n}\sum_{k=1}^n f(T^k(x))\overset{L_2}{\underset{n\to\infty}{\longrightarrow}} \mathbb{E}(f|\Theta),$$ где $\Theta$ порождено функциями (случайными величиными) $$\left\{f:f(x)\overset{L_2}{=}f(T(x))\right\}.$$
Если $$B\in\sigma(X): T^{-1}(B)\overset{L_2}{=}B\Rightarrow B\overset{L_2}{=}\{\emptyset\}\lor X,$$ то $$\mathbb{E}(f|\Theta) = \mathbb{E} f=\int_{X}f(x)d\mu(x).$$

Не ограничивая общности, будем считать, что $\mu(X)=1$. Построим стационарный в широком смысле случайный процесс $Y(k)=f(T^kx)$, где $x$ --- случайная величина, распределенная согласно мере $\mu$. Верно ли, что 
$$\mathbb{E}(f|\Theta) =\mathbb{E}f\Leftrightarrow$$
$$ \frac{1}{n^2}\sum_{i,j=1}^{n}R_{\gamma}(i,j) {\longrightarrow}0\quad \text{при}\quad n\to 0,$$
где 
$\frac{1}{n^2}\sum_{i,j=1}^{n}R_{\gamma}(i,j) = \frac{1}{n^2}\sum_{i,j=1}^{n} \mathbb{E}[({Y}(i)-\mathbb{E}{Y}(i))({Y}(j)-\mathbb{E}{Y}(i))]$?
\end{problem}

\begin{problem}
Пусть задана последовательность  $x_k = \{\alpha k\}$, где  $\alpha$ -- какое-либо иррациональное число, $\{a\}$ --- дробная часть числа $a$. 

а) С помощью предыдущей задачи объясните, почему для любой $f\in C[0,1]$ выполняется при ${n\to\infty}$:
$$
\frac{1}{n}\sum_{k=1}^n f(x_k){\longrightarrow}\int_{0}^1 f(x)\,dx.
$$

б) Пусть функция, определенная в единичном квадрате $[0,1]^2$ на плоскости имеет вид:
$$
f(x,y) = \begin{cases}
 1, & y\leq x \text{ и } x+y\leq 1\\
 0, & \textit{иначе} 
\end{cases}.
$$
Верно ли, что при ${n\to\infty}$
$$
\frac{1}{n}\sum_{k=1}^n f(x_{2k-1},x_{2k}){\longrightarrow}\int_0^1\int_0^1 f(x,y)\,dxdy ?
$$

\end{problem}


\begin{problem} (Г. Вейль). Рассмотрим последовательность 
$\left\{ {a_k } \right\}_{k\in {\mathbb{N}}} $, где $a_k $ -- первая цифра в 
десятичной записи числа $2^k$. Положим $$I_m \left( {a_k } \right)=\left\{ 
{\begin{array}{l}
 1,\quad a_k =m \\ 
 0,\quad a_k \ne m \\ 
 \end{array}} \right.,\quad m=1,\;2,\;...,9.$$ Существует ли $\mathop {\lim 
}\limits_{n\to \infty } \;\frac{1}{n}\sum\limits_{k=1}^n {I_m \left( {a_k } 
\right)}$? Если существует, то найдите его.
\end{problem}

\begin{ordre} Рассмотрим вероятностное пространство $X=\left( \Omega ,\Xi ,P\right)$, где $\Omega =\left[ {0,\;1} \right)$, $\Xi $ 
-- $\sigma $-алгебра борелевских множеств (т.е. $\Xi $ -- минимальная 
$\sigma $-алгебра, содержащая всевозможные открытые множества $\Omega 
=\left[ {0,\;1} \right)$) на $\left[ {0,\;1} \right)$, а $P$ --- равномерная 
мера на $\Xi $, т.е. $P\left( {\left[ {a,\;b} \right)} \right)=b-a$. 
Рассмотрим с.в.
\[
x\left( \omega \right)=\left\{ {\begin{array}{l}
 1,\quad \omega \in \left[ {\log _{10} m,\;\log _{10} \left( {m+1} \right)} 
\right) \\ 
 0,\quad \omega \in \left[ {0,\;\log _{10} m} \right)\cup \left[ {\log _{10} 
\left( {m+1} \right),\;1} \right) \\ 
 \end{array}} \right..
\]
Рассмотрим случайный процесс (в дискретном времени)
$$X_k \left( \omega \right)=x\left( {T^k\omega } \right),$$ где $T:\left[ 
{0,\;1} \right)\to \left[ {0,\;1} \right)$ определяется по формуле 
 $$T\omega =\left( {\omega +\log _{10} 2} 
\right) \bmod 1,$$ 
где $a mod 1$ --- дробная часть числа $a$.
Важно заметить, что преобразование $T$ сохраняет меру, т.е. $$\forall 
\;\;A\in \Xi \to P\left( {T^{-1}A} \right)=P\left( A \right).$$ Собственно, и 
в более общей ситуации, известная из курса случайных процессов эргодическая 
теорема схожим образом переносится на динамические системы, которые задаются 
фазовым пространством $\Omega $ и динамикой $T:\;\Omega \to \Omega $. 
Согласно теореме Крылова--Боголюбова (см. Синай Я.Г., Введение в 
эргодическую теорию, М.: ФАЗИС, 1996, лекция 2), если $\Omega $ --  компакт, 
то всегда найдется как минимум одна инвариантная относительно $T$ мера на 
$\Xi $. Если построенной по такой динамической системе случайный процесс 
окажется эргодическим, то доля времени пребывания динамической системы в 
заданной области просто равняется мере (той самой инвариантной и 
эргодической) этой области. Ввиду вышесказанного интересно заметить, что 
установление эргодичности является трудной задачей. Например, до сих пор строго не 
обоснована ``эргодическая гипотеза Лоренца'' для идеального газа в сосуде 
(см. Козлов В.В., Тепловое равновесие по Гиббсу и Пуанкаре, Москва -- 
Ижевск, РХД, 2002, Минлос Р., Введение в математическую статистическую 
физику, М.: МЦНМО, 2002), см. также задачи 1, 22 раздела 6, задачу 15 раздела 7.

Покажите, что случайный процесс $X_k $ -- стационарный в узком смысле. В 
предположении, что этот процесс эргодичен по математическому 
ожиданию (см. \cite{21} т. 2, глава 5) 
 найдите искомый предел.
 
Стоит обратить внимание, что в 
эргодической теореме фигурирует сходимость либо в $L_2$, либо в $L_1 $, 
либо п.н. А в данной задаче требуется (для доказательства существования 
предела и его вычисления), сходимость поточечная. Оказывается, для данной 
задачи из сходимости в $L_2 $ легко следует сходимость п.н., откуда (в свою 
очередь) следует поточечная (подробности см. Синай  Я.Г., Введение в 
эргодическую теорию, М.: ФАЗИС, 1996, лекция 3 и  Корнфельд И.П.,  Синай Я.Г., 
 Фомин С.В., Эргодическая теория, М.: Наука, 1980).
\end{ordre}
\begin{problem} (Гаусса--Гильдена--Вимана--Кузьмина) Каждое число из промежутка $\Omega =\left[ {0,\;1} 
\right)$ может быть разложено в цепную дробь (вообще говоря, бесконечную). 

Покажите (см. предыдущую задачу), что для почти всех (в 
равномерной мере) точек $\omega \in \left[ {0,\;1} \right)$
\[
\mathop {\lim }\limits_{n\to \infty } \;\frac{1}{n}\sum\limits_{k=1}^n {I_m 
\left( {a_k \left( \omega \right)} \right)} =\frac{1}{\ln \;2}\ln \left( 
{1+\frac{1}{m\left( {m+2} \right)}} \right).
\]
\end{problem}
\begin{ordre}
Цепные дроби играют важную роль, например, в различных вычислениях 
(поскольку позволяют строить в определенном смысле наилучшие приближения 
иррациональных чисел рациональными), в теории динамических систем (КАМ 
теории). Для рациональных чисел такие дроби конечны, для квадратичных 
иррациональностей -- периодические (см. пример ниже, в котором период равен 
1):
\[
\frac{\sqrt 5 -1}{2}=\frac{1}{a_1 +\frac{1}{a_2 +\frac{1}{a_3 
+...}}}=\frac{1}{1+\frac{1}{1+\frac{1}{1+...}}}.
\]
Чтобы проверить выписанное соотношение достаточно заметить, 
что $\frac{\sqrt 5 -1}{2}$ -- является корнем уравнения $x=\frac{1}{1+x}$ 
(причем, из принципа сжимающих отображений следует, что последовательность 
$$x_0 =1,\quad x_{n+1} =\frac{1}{1+x_n }$$ сходится именно к этому корню),
см.  Арнольд В.И. Цепные дроби, М.: МЦНМО, 
2001, Хинчин А.Я. Цепные дроби, Ленинград: Физматгиз, 1961. Покажите, что преобразование $T:\;\;\left[ 
{0,\;1} \right)\to \left[ {0,\;1} \right)$:
\[
T\omega =\left\{ {\begin{array}{l}
 \left\{ {\frac{1}{\omega }} \right\},\quad \omega \in \left( {0,\;1} 
\right) \\ 
 0,\quad \omega =0 \\ 
 \end{array}} \right.,
\]
где $\left\{ {5.8} \right\}=0.8$ -- дробная часть числа, сохраняет меру Гаусса
\[
\forall \;\;A\in \Xi \to P\left( A \right)=\frac{1}{\ln \;2}\int\limits_A 
{\frac{dx}{1+x}} ,
\]
где $\Xi$ -- сигма-алгебра на $\Omega$.

Далее рассуждайте аналогично предыдущей задаче (эргодичность возникшего 
случайного процесса также можно не доказывать).
\end{ordre}


%\begin{problem}(Ветвящийся процесс). В колонию зайцев внесли зайца с 
%необычным геном. Обозначим через $p_k $ - вероятность того, что в потомстве 
%этого зайца ровно $k$ зайчат унаследуют этот ген ($k=0,1,2,...)$. Это же 
%распределение вероятностей характеризует всех последующих потомков, 
%унаследовавших необычный ген. Будем считать, что каждый заяц дает потомство 
%один раз в жизни в возрасте одного года (как раз в этом возрасте находился 
%самый первый заяц с необычным геном в момент попадания в колонию).

%Обозначим через $G\left( z \right)$ - производящую функцию распределения 
%$p_k $, $k=0,1,2,...$, т.е. $G\left( z %\right)=\sum\limits_{k=0}^\infty {p_k 
%z^k} $. Пусть $X_n $ - количество зайцев в возрасте одного года с необычным 
%геном спустя n лет после попадания в колонию первого такого зайца. 
%Производящую функцию с.в. $X_n $ обозначим $\Pi _n \left( z \right)=\mathbb{E}\left( 
%{z^{X_n }} \right)$.

%\begin{enumerate}
%\item Получите уравнение, связывающее $\Pi _{n+1} \left( z \right)$ с $\Pi _n \left( z \right)$ посредством $G\left( z \right)$.
%\end{enumerate}
%\textbf{Указание. }Покажите, что $M\left( {\left. {z^{X_{n+1} }} \right|X_n 
%} \right)=\left[ {G\left( z \right)} \right]^{X_n }$. Затем возьмите 
%математическое ожидание от обеих частей равенства.

%\begin{enumerate}
%\item Покажите, что вероятность вырождения гена $$q_n =P\left( {X_k =0;\;k\ge n} \right)=\Pi _n \left( 0 \right).$$ Существует ли предел $q=\mathop {\lim }\limits_{n\to \infty } \;q_n $? Если существует, то найдите его.
%\end{enumerate}
%\end{problem}
%\begin{remark}Легко видеть, что функция $G\left( z \right)$ - выпуклая. 
%Уравнение $z=G\left( z \right)$ имеет два корня: один в любом случае равен 
%1, другой $q\le 1$. Если $\nu ={G}'\left( 1 \right)>1$, то $q<1$. Если $\nu 
%\le 1$, то $q=1$.

%См. Севастьянов Б.А. Ветвящиеся процессы (серия "Теория вероятности и математическая статистика"), Наука, 1971; А. В. Калинкин Марковские ветвящиеся процессы с взаимодействием, Успехи математических наук, 2002.
%\end{remark}




%%\begin{problem}(парадокс Эренфестов). На двух камнях сидят кузнечики. Всего 
%%кузнечиков $M\gg 1$. Каждый кузнечик независимо ни от чего в промежутке 
%%времени $\left[ {t,t+\Delta t} \right)$, где $\Delta t$ -- мало, а $t\ge 0$ 
%%-- произвольно, перепрыгивает на другой камень с вероятностью $\lambda 
%%\Delta t+o\left( {\Delta t} \right)$, $\lambda >0$. Введем вектор $\vec 
%%{n}\left( t \right)=\left( {n_1 \left( t \right),n_2 \left( t \right)} 
%%\right)^T$, где $n_k \left( t \right)$ -- число кузнечиков на $k$-м камне в 
%%момент времени $t\ge 0$. Описанная стохастическая динамика имеет 
%%единственный закон сохранения $n_1 \left( t \right)+n_2 \left( t 
%%\right)\equiv M$ (числа кузнечиков). Стационарная (инвариантная) мера имеет 
%вид:
%%\[
%%\nu \left( {n_1 ,n_2 } \right)=\nu \left( {c_1 M,c_2 M} 
%%\right)=M!\frac{\left( {1 \mathord{\left/ {\vphantom {1 2}} \right. 
%\kern-\nulldelimiterspace} 2} \right)^{n_1 }}{n_1 !}\frac{\left( {1 
%\mathord{\left/ {\vphantom {1 2}} \right. \kern-\nulldelimiterspace} 2} 
%\right)^{n_2 }}{n_2 !}=C_M^{n_1 } 2^{-M}\simeq 
%\]
%\begin{equation}
%\label{eq2}
%\simeq \frac{2^{-M}}{\sqrt {2\pi c_1 c_2 } }\exp %\left( {-M\cdot H\left( 
%{c_1 ,c_2 } \right)} \right),
%\end{equation}
%где $H\left( {c_1 ,c_2 } %\right)=\sum\limits_{i=1}^2 {c_i \ln } \;c_i $. 
%Предположим теперь, что существует два предела
%\[
%c_i \left( 0 \right)=\mathop {\lim }\limits_{M\to %\infty } {n_i \left( 0 
%\right)} \mathord{\left/ {\vphantom {{n_i \left( 0 %\right)} M}} \right. 
%\kern-\nulldelimiterspace} M.
%\]
%Тогда в произвольный момент времени $t>0$ и для %любого $i=1,2$ с 
%вероятностью 1 существует предел (заметим, что %$n_i \left( t \right)$ -- 
%случайные величины, тем не менее $c_i \left( t %\right)$ -- уже не случайные 
%величины) $c_i \left( t \right)\mathop %=\limits^{\mbox{п.н.}} \mathop {\lim 
%}\limits_{M\to \infty } {n_i \left( t \right)} %\mathord{\left/ {\vphantom 
%{{n_i \left( t \right)} M}} \right. %\kern-\nulldelimiterspace} M$. Описанный 
%выше приём называется каноническим скейлингом. В %результате такого скейлинга 
%приходим к ``динамике квазисредних'' (как правило, %это динамика описывается 
%нелинейной системой обыкновенных дифференциальных %уравнений):
%\[
%\frac{dc_1 }{dt}=\lambda \left( {c_2 -c_1 } %\right),
%\]
%\[
%\frac{dc_2 }{dt}=\lambda \left( {c_1 -c_2 } %\right).
%\]
%Докажите формулу (\ref{eq1}). Покажите, что %функция (минус энтропия) $H\left( {c_1 
%,c_2 } \right)$ будет функцией Ляпунова выписанной %СОДУ. Покажите, что если 
%в начальный момент все кузнечики находились на %одном камне, то 
%математическое ожидание времени первого %возвращения макросистемы в такое 
%состояние будет порядка $2^M$. 
%\end{problem}
%\begin{remark}На примере этой модели можно %говорить о том, что в 
%макросистеме возврат к неравновесным %макросостояниям вполне допустим, но 
%происходить это может только через очень большое %время (циклы Пуанкаре), так 
%что нам может не хватить отведенного времени, %чтобы это заметить (парадокс 
%Цермело). Напомним, что описанный выше случайный %процесс обратим во времени. 
%Однако наблюдается необратимая динамика %относительной разности числа 
%кузнечиков на камнях (парадокс Лошмидта). Данная %задача является простейшим 
%представителем большого пласта задач на поиск %равновесия макросистемы, 
%которое определяется (в нашем случае, аналогично и %в общем случае), как
%\[
%\vec {c}^\ast =\left( {\begin{array}{l}
% 1 \mathord{\left/ {\vphantom {1 2}} \right. %\kern-\nulldelimiterspace} 2 \\ 
% 1 \mathord{\left/ {\vphantom {1 2}} \right. %\kern-\nulldelimiterspace} 2 \\ 
% \end{array}} \right)=\arg \mathop {\min %}\limits_{\begin{array}{c}
% c_1 +c_2 =M \\ 
% \vec {c}\ge \vec {0} \\ 
% \end{array}} H\left( {\vec {c}} \right).
%\]
%Детали и ссылки см. в работах%
%
%\textit{Sandholm W.} Population games and %Evolutionary dynamics. Economic Learning and %Social 
%Evolution. MIT Press; Cambridge, 2010.
%
%\textit{Гасников А.В}., \textit{Гасникова Е.В.} Об %энтропийно-подобных функционалах, возникающих в %стохастической 
%химической кинетике при концентрации инвариантной %меры и в качестве функций 
%Ляпунова динамики квазисредних // Математические %заметки. 2013. Т. 94. № 6. 
%С. 816--824.
%
%Немного более сложный пример приведен ниже.
%\end{remark}
%
%\begin{problem}{(кинетика социального %неравенства).} В некотором городе 
%живет $N\gg 1$ (например, 10~000) пронумерованных %жителей. У каждого $i$-го 
%жителя есть в начальный (нулевой) момент времени %целое (неотрицательное) 
%количество рублей $s_i \left( 0 \right)$ %(монетками, достоинством в один 
%рубль). Со временем пронумерованные жители %(количество которых не 
%изменяется, также как и суммарное количество %рублей) случайно разыгрывают 
%свое имущество. В каждый момент времени %$t=1,2,3,...$ случайно и независимо 
%от предыстории выбираются два жителя: с %вероятностью $1 \mathord{\left/ 
%{\vphantom {1 2}} \right. %\kern-\nulldelimiterspace} 2$ житель с большим 
%номером отдаёт 1 рубль (если, конечно, он не %банкрот, если банкрот, то не 
%отдает) жителю с меньшим номером, и с вероятностью %$1 \mathord{\left/ 
%{\vphantom {1 2}} \right. %\kern-\nulldelimiterspace} 2$ наоборот. 
%Приблизительно такую постановку задачи в конце %18-го века предложил 
%известный итальянский экономист Вильфредо Парето, %чтобы объяснить социальное 
%неравенство.%
%
%Пусть $c_s \left( t \right)$ - доля жителей %города, имеющих ровно $s$ рублей 
%в момент времени $t$ (заметим, что $c_s \left( t %\right)$ - случайная 
%величина). Пусть
%\[
%S=\sum\limits_{i=1}^N {s_i \left( 0 \right)} ,
%\quad
%\bar {s}=\frac{S}{N}.
%\]
%Покажите, что тогда по эргодической теореме для %конечных однородных 
%марковских цепей и центральной предельной теореме:
%\[
%\exists \;\;\lambda _{q,0.99} >0,\;\;T={\rm %O}\left( {N^2} 
%\right):\;\;\;\forall \;t\ge T,\;s=0,...,S
%%\]
%\[
%P\left( {\left. {\left| {c_s \left( t %\right)-Ce^{-s \mathord{\left/ 
%{\vphantom {s {\bar {s}}}} \right. %\kern-\nulldelimiterspace} {\bar {s}}}} 
%\right|\le \frac{\lambda _{0.99} }{\sqrt N }} %\right)\ge 0.99} \right.,
%\]
%где $C$ определяется из условия нормировки:
%
%$\sum\limits_{s=0}^S {Ce^{-s \mathord{\left/ %{\vphantom {s {\bar {s}}}} 
%\right. \kern-\nulldelimiterspace} {\bar {s}}}} %=1,$ т.е. $C\approx 1 
%\mathord{\left/ {\vphantom {1 {\bar {s}}}} \right. %
%\kern-\nulldelimiterspace} {\bar {s}}$.
%\end{problem}
%
%\begin{problem}{(Замкнутая сеть, теорема %Гордона--Ньюэлла).} Рассматривается 
%транспортная сеть, в которой между $N$ станциями %курсируют $M$ такси. 
%Клиенты пребывают в $i$-й узел в соответствии с %пуассоновским потоком с 
%параметром $\lambda _i >0$ ($i=1,\;...,\;N)$. Если %в момент прибытия в $i$-й 
%узел там есть такси, клиент забирает его и с %вероятностью $p_{ij} \ge 0$ 
%направляется в $j$-й узел, по прибытии в который %покидает сеть. Такси остается 
%ждать в узле прибытия нового клиента. Времена %перемещений из узла в узел -- 
%независимые случайные величины, имеющие %показательное распределение с 
%параметром $\nu _{ij} >0$ для пары узлов $\left( %{i,j} \right)$. Если в 
%момент прихода клиента в узел там нет такси, %клиент сразу покидает узел. 
%Считая $p_{ij} =N^{-1}$, $\lambda _i =\lambda $, %$\nu _{ij} =\nu $, 
%покажите, что вероятность того, что клиент, %поступившей в узел (в 
%установившемся (стационарном) режиме работы сети), %получит отказ, равна
%\[
%p_{\mbox{o}} \left( {N,M} %\right)={\sum\limits_{k=0}^M 
%{\frac{C_{N-2+k}^k \rho ^{M-k}}{\left( {M-k} %\right)!}} } \mathord{\left/ 
%{\vphantom {{\sum\limits_{k=0}^M %{\frac{C_{N-2+k}^k \rho ^{M-k}}{\left( 
%{M-k} \right)!}} } {\sum\limits_{k=0}^M %{\frac{C_{N-1+k}^k \rho 
%^{M-k}}{\left( {M-k} \right)!}} }}} \right. %\kern-\nulldelimiterspace} 
%{\sum\limits_{k=0}^M {\frac{C_{N-1+k}^k \rho %^{M-k}}{\left( {M-k} \right)!}} 
%},
%\quad
%\rho ={N\lambda } \mathord{\left/ {\vphantom %{{N\lambda } \nu }} \right. 
%\kern-\nulldelimiterspace} \nu .
%\]
%Методом перевала покажите справедливость следующей %асимптотики при $N\to 
%\infty $:
%\[
%p_{\mbox{o}} \left( {N,rN} %\right)=1-\frac{2r}{\lambda \mathord{\left/ 
%{\vphantom {\lambda \nu }} \right. %\kern-\nulldelimiterspace} \nu +r+1+\sqrt 
%{\left( {\lambda \mathord{\left/ {\vphantom %{\lambda \nu }} \right. 
%\kern-\nulldelimiterspace} \nu +r+1} %\right)^2-4{\lambda r} \mathord{\left/ 
%{\vphantom {{\lambda r} \nu }} \right. %\kern-\nulldelimiterspace} \nu } 
%}+{\rm O}\left( {\frac{1}{N}} \right).
%\]
%\end{problem}
%
\begin{problem}(Максимальный показатель Ляпунова \cite{27}). Пусть имеется последовательность независимых одинаково распределенных случайных матриц $g_k$ (распределение $g_k$ имеет плотность). Покажите, что существует такое $\lambda\in\mathbb{R}$, что (от выбора нормы число $\lambda$ не зависит) 
\begin{equation*}
\lim_{n\to\infty} \frac{1}{n}\|g_{n}\cdot\ldots\cdot g_{1}\| = \lambda.
\end{equation*}
\end{problem}
\begin{remark}
См. Гренандер У. Вероятности на алгебраических структурах, М.: Мир, 1965.
\end{remark}

\begin{problem}(Теорема Санова)
\label{sanov}
Пусть $A$ --- конечный алфавит. Вероятности появления в слове символа алфавита обозначим $p_a$, $a\in A$. Предполагается, что число символов в алфавите $|A|>1$. 
Обозначим за $v_a(s)$ случайную величину, отвечающую за относительную частоту буквы $a$ в строке $s$, т. е. число вхождений буквы $a$ в строку $s$, деленное на $n$. Назовем типом строки $s$ набор ${v}(s) = (v_1(s), v_2(s), . . . , v_{|A|}(s))$.

Пусть $\Pi$ --- непустое замкнутое подмножество множества распределений вероятности $\big\{{p}=(\dots,p_a,\dots):\, p_{a}\ge 0 \text{ для всех } a\in A,\, \sum_{a\in A}p_{a}=1\bigr\}$,  совпадающее с замыканием своей внутренности, и $$\mathcal{KL}(\Pi|{p}) = \min_{\mu\in\Pi}\mathcal{KL}(\mu|{p}),$$
где расстояние Кульбака-Лейблера между вектором $\mu$ и истинным распределением ${p}$ имеет вид
$$
\mathcal{KL}(\mu|{p}) = \sum_{a\in A} \mu_a\log\frac{\mu_a}{p_a}.
$$
Покажите, что при $n\to\infty$
$$
-\frac{1}{n}\log\mathbb{P}({v}(s)\in \Pi)\to\mathcal{KL}(\Pi|{p}).
$$

\end{problem}
\begin{remark}
Доказательство базируется на использовании формулы Стирлинга $n! = n^n{\rm e}^{-n}\sqrt{2\pi n}[1+O(1/n)]$, откуда вероятность реализации слов $s$ длины $n$ с частотой появления букв $v_a(s)=\mu_a$, $a \in A$ равна
\begin{equation*}
\begin{split}
\mathbb{P}(v(s)=\mu) = \frac{n!}{(n\mu_a)!\dots (n\mu_z)!}p_a^{n\mu_a}\cdot\dots\cdot p_z^{n\mu_a}\\
\approx C \exp(-n\mathcal{KL}(\mu|\vec{p}))\\
= \exp(-n\mathcal{KL}(\mu|\vec{p}) + R),
\end{split}
\end{equation*}
где $|R|<(n+1)(\frac 12 \log{n}+\frac 12 \log(\pi)+\frac{1}{12n}).$

См.  Санов И.Н. О вероятности больших отклонений случайных величин, Матем. сб., 42(84):1 (1957), С. 11–44 лекции А.Н. Соболевского в НМУ   (http://www.mccme.ru/ium/s09/probability.html), \cite{information}.

Также о вероятностях больших отклонений для эмпирических мер стохастических процессов смотрите следующую литературу: Feng, J., Kurtz T.G. Large deviations for stochastic processes, V. 131 of Mathematical surveys and monographs. American Mathematical Society, Providence, RI, USA, 2006, Leonard C. A large deviation approach to optimal transport. arXiv:0710.1461v1, 2007.


\end{remark}

\begin{problem}(Крамеровская зона)
Рассмотрим $S_n = X_1+\cdots+X_n$, где $X_i$, $i=1,\dots,n$ --- независимые одинаково распределенные величины, $\mathbb{E}X_i=0$, $\mathbb{E}X_i^2= d<\infty$.
В силу центральной предельной теоремы при $n\to\infty$ 
\[
\mathbb{P}(S-n\geq x)\approx 1- \Phi\left(\frac{x}{\sqrt{nd}}\right)
\]
равномерно по $x$ из интервала $(0,N_{n}n)$, где $N_n\to \infty$ достаточно медленно. Предположим, что $X_i$ имеют субэкспоненциальное распределение, характеристическим свойством которого является асимптотическая аддитивность хвостов сверток исходных распределений, то есть 
\[
\mathbb{P}(S_n\geq x)\approx n\mathbb{P}(X_1\geq x),
\]
в частности, это свойство выполнено, если $\mathbb{P}(X_1\geq x)$ является правильно меняющейся функцией, то есть $\mathbb{P}(X_1\geq t) = t^{-\alpha}L(t)$, где в свою очередь $L(t)$ --- медленно меняющаяся функция, то есть для любого $u>0$ выполнено $\frac{L(ut)}{L(t)}\to 1$ при $t\to\infty$.
Эти две асимптотики смыкаются следующим образом, если выполнено $\mathbb{E}(X^2;|X|>t)={o}(1/\ln t)$, $x>\sqrt{n}$, $t\to\infty$:
\[
\mathbb{P}(S_n\geq x)\approx 1- \Phi\left(\frac{x}{\sqrt{nd}}\right)+nV(x),\quad n\to\infty.
\]
Значение, характеризующее зону уклонений $S_n$, где происходит смена асимптотик $\mathbb{P}(S_n\geq x)$ от <<нормальной>> $1-\Phi(x/\sqrt{nd})$ на асимптотику $nV(x)$, описывающую $\mathbb{P}(S_n\geq x)$ при достаточно больших $x$ следующее:
\[
\sigma(n) = V^{-1}(1/n) = \sqrt{(\alpha-2)nd\ln n},
\]
при $n=1$ полагаем $\sigma(1)=1$.

Найдите асимптотики уклонений $S_n$ для последовательностей случайных величин из \begin{enumerate}
\item равномерного распределения,
\item пуассоновского распределения,
\item экспоненциального распределения.
\end{enumerate}

\end{problem}
\begin{remark}
Используем обозначения $B_j = \{X_j<y\}$, $B = \bigcup_{i=1}^n B_j$.

Рассмотрим случай $d=1$, $x>\sqrt{n}$, $\mathbb{P}(S_n\geq x) \leq V(x)$, $\alpha>2$, $\mathbb{E}X_i = 0$, $\mathbb{E} X_i^2 = 1$. 

Приведем результат о верхних границах на $\mathbb{P}(\max_{k\leq n} S_n \geq x; B)$.

\noindent 1)  При любых фиксированных $h>1$, $s_0>0$ для $x = \sigma(n)$, $s\geq s_0$ и всех достаточно малых $\Pi = nV(x)$ выполняется 
\[
P = \mathbb{P}(\max_{k\leq n} S_n \geq x; B)\leq {\rm e}^{r} \left(\frac{\Pi(y)}{r}\right)^{r-\theta},
\]
где $\Pi(y)= nV(y)$, $\theta = \frac{hr^2}{4s^2}\left(1+b\frac{\ln s}{\ln n}\right)$, $b = \frac{2\alpha}{\alpha-2}$.

\noindent 2) При любых фиксированых $h>1$, $\tau>0$ для $x = s\sigma(n)$, $s^2<(h-\tau)/2$ и всех достаточно больших $n$ выполняется
\[
P\leq{\rm e}^{-x^2/2nh}.
\]

Покажите, что из приведенной теоремы следуют утверждения:

\noindent a) Если $x=s\sigma(n)$, $s\to\infty$, то при любом $\epsilon>0$ и всех достаточно малых $\Pi=nV(x)$
\[
P\leq \Pi^{r-\epsilon}.
\]

\noindent b) Ecли $s^2>c\ln n$, то 
\[
P\leq c_1\Pi^{r}.
\]

\noindent c) Если $s$ фиксировано, $r = 2s^2/h\geq 1$, то при $n \to\infty$
\[
P\leq c\Pi^{s^2/h+{o}(1)}.
\]

\noindent d) В частности, при всех $s^2>2h$ и всех достаточно больших $n$
\[
P\leq c\Pi^2.
\]

\noindent e) Если $s\to\infty$, то при любом $\delta>0$ и всех достаточно малых $nV(x)$
\[
\mathbb{P}(\sup_{k\leq n}S_k\geq x)\leq nV(x)(1+\delta).
\]

\noindent f)
Если $s^2\geq h+\tau$ при каком-нибудь фиксированном $\tau>0$, то при достаточно малых $nV(x)$
\[
\mathbb{P}(\sup_{k\leq n}S_k\geq x)\leq cnV(x).
\]

\noindent g) При любых фиксированных $h>1$, $\tau>0$ для $s^2<(h-\tau)/2$ и всех достаточно больших $n$
\[
\mathbb{P}(\sup_{k\leq n}S_k\geq x)\leq {\rm e}^{-x^2/2nh}.
\]

См. Боровков А.А., Боровков К.А., Асимптотический анализ случайных блужданий.  Т.1. Медленно убывающие распределения скачков, Москва: Физматлит, 2008, 652 с.
\end{remark}



\begin{problem}(Маловероятные пути блужданий \cite{27}) Будем говорить, что случайное  блуждание $\{S_t\}_{t\geq 0}$, $S_t = \sum_{i=1}^t X_i$, где
$X_i$ принимает значения ${-1,1}$ c вероятностями $q$ и $p$ соответственно, 
удовлетворяет принципу больших уклонений с функционалом действия $L_{\tau}(v)$, если
$$
\ln \mathbb{P}(A_{[\tau N],\delta})\sim L_{\tau}(v)N,
$$
где 
$$
A_{[\tau N],\delta} = \bigg\{\sup_{t=0,1,\dots,[\tau N]}|S_t-vt|\leq \delta N\biggr\}.
$$

Покажите, что для случайного блуждания с  $p=q=1/2$ (соотвествующая мера $\mathbb{P}_{0}$) верен принцип больших уклонений с функционалом действия  $L_{\tau}(v) = \tau (-v\lambda(v)+h(\lambda(v)))$,
где $\lambda=\lambda(v)$ решение уравнения 
$$
\frac{{\rm e}^{\lambda}-{\rm e}^{-\lambda}}{{\rm e}^{\lambda}+{\rm e}^{-\lambda}}=v.
$$
\end{problem}
\begin{remark}
Идея доказательства состоит в использовании замены меры, то есть вероятности скачков $1/2$ заменяют на
$$
p_{\lambda} = \frac{{\rm e}^{\lambda}}{{\rm e}^{\lambda}+{\rm e}^{-\lambda}},\,
q_{\lambda} = \frac{{\rm e}^{-\lambda}}{{\rm e}^{\lambda}+{\rm e}^{-\lambda}}
$$
так, чтобы $S_t-vt$ имело нулевые средние. Тогда вероятности по мере $\mathbb{P}_{0}$ представляются через средние по новой бернуллиевской мере $\mathbb{P}_{\lambda}$
$$
\mathbb{P}(A_{[\tau N],\delta}) = \mathbb{E}_{\lambda}[I(A_{[\tau N,\delta]})\exp(-\lambda S_{[N\tau]}+[N\tau]h(\lambda))],
$$
$$
h(\lambda) = \log\left(\frac{{\rm e}^{\lambda}+{\rm e}^{-\lambda}}{2}\right),
$$
где  $I(A_{[\tau N],\delta})$--- индикатор события $A_{[\tau N],\delta}$.

Доказательство принципа больших уклонений с указанным в условии задачи функционалом действия следует из оценок 
$$
\exp(N L_{\tau}(v))\mathbb{E}_{\lambda}[I(A_{[\tau N],\delta})]\exp(-|\lambda|\delta N)\leq \mathbb{P}(A_{[\tau N],\delta}), 
$$
$$
\mathbb{P}(A_{[\tau N],\delta}) \leq \mathbb{E}_{\lambda}[I(A_{[\tau N],\delta})]\exp(|\lambda|\delta N)\exp(N L_{\tau}(v)).
$$
См. также  Боровков А.А., Боровков К.А. Асимптотический анализ случайных блужданий.  Т.1. Медленно убывающие распределения скачков, Москва: Физматлит, 2008, 652 с.
\end{remark}

\begin{problem}(Случайное блуждание в полуплоскости). Пусть частица находится в начальный момент в одной из точек полуплоскости $\mathbb{Z}\times\mathbb{Z}_{+}$, и совершает на каждом шаге скачок из $(k,l)\in \mathbb{Z}\times\mathbb{Z}_{+}$ в одну из четырех соседних точек решетки $(k+i,l+j)$ с вероятностями $p_{ij}$ (если $l>0$) и в одну из трех соседних точек решетки с вероятностями $q_{ij}$ если $l=0$. Считая, что $\sum_{j}jp_{ij}<0$ опишите движение частицы \cite{27} (движение к границе и последующее движение вдоль границы).
\end{problem}





\end{document}