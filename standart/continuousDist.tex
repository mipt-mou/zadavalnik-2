
%\subsection{Примеры непрерывных распределений}

\begin{problem}
Допустим, что вероятность столкновения молекулы с другими молекулами в промежутке времени $[t,t + \Delta t)$ 
равна $p = \lambda\Delta t + o(\Delta t)$ и не зависит от времени, прошедшего после предыдущего столкновения $(\lambda = \const)$. 
Найти распределение времени свободного пробега молекулы и вероятность того, что это время превысит заданную величину $t^*$. 
\end{problem}

\begin{ordre}
Разобьем интервал $\Delta=[0,t)$ на $n$ отрезков равной длины.
Пусть $A_i$ --- событие, означающее, что на отрезке $\Delta_i$  молекула претерпит столкновение с другими молекулами. Можно представить вероятность отсутствия столкновения в виде произведения вероятностей событий $\overline{A_i}$.
\end{ordre}

\begin{problem}(Задача об оптимальном моменте замены оборудования)
Пусть некоторая система состоит из $n$ элементов, выходящих из строя независимо друг от друга через случайное время $\tau _{i} $, имеющее показательное распределение с параметром $\lambda _{i} $ ($i$ -- номер элемента, $1\le i\le n$). Вышедший из строя элемент немедленно заменяется новым. Пусть $c_{i} $ -- убытки, связанные с выходом из строя и заменой элемента $i$-го типа. Через промежуток времени $T$ разрешается провести профилактический ремонт, при котором все $n$ элементов заменяются новыми. Пусть $D$ -- стоимость профилактического ремонта. Найдите оптимальное время $T$, минимизирующее средние убытки в единицу времени.
\end{problem}

\begin{remark} 
Согласуется ли Ваш результат со здравым смыслом: не стоит проводить профилактический ремонт, если выход оборудования из строя не связан с его старением? Детали см. в Афанасьева Л.Г. Очерк исследования операций -- М.: Мехмат, 2007. 
\end{remark}



\begin{problem}
Приведите пример таких с.в., что $X=-Y$, но $F_X=F_Y$  (имеют одинаковые  функции распределения).
\end{problem}

\begin{problem}
Случайная величина $X$ имеет функцию распределения $F_X(u)$ и функцию плотности распределения $f_X(u)$. Найти функции распределения и плотности распределения (если последние существуют) для случайных величин:\\
\indent а) $Y = aX+b$\\
\indentб) $Z =e^X$\\
\indent в) $V = X^2$ ($a$ и $b$ – неслучайные величины)\\
Конкретизировать решения при $X \in \N(0,1)$.
\end{problem}

\begin{problem}
Пусть случайные величины $X$ и $Y$ связаны соотношением $Y = \phi(X)$ ($\phi^{-1}(X)$ – непрерывная вещественнозначная функция). Найти распределение случайной величины $Y$, если случайная величина $X$ равномерно распределена на отрезке $[0,1]$.
\end{problem}

\begin{problem}
С.в. $X$, $Y$, $Z$ независимы и равномерно распределены на $(0,1)$. Докажите равенство распределений  $(XY)^{Z}$  и $X$.
\end{problem}

\begin{ordre}
Докажите, что величина $-Z(\ln X + \ln Y)$ показательно распределена на $[0,\infty)$. Для этого удобно воспользоваться заменой $U = Z W$, $V = W/Z$, где $W = - \ln X - \ln Y$, $f_W(w) = w e^{-w} [w>0]$. 
\end{ordre}

\begin{problem}
Случайные величины $X_1,\ldots,X_n$ независимы и имеют показательное распределение с параметром $\lambda$.
Докажите равенство распределений $(X_1+  \frac{1}{2} X_2 + \ldots+ \frac{1}{n} X_n)$ и  $\max(X_1,\ldots,X_n)$.
\end{problem}

\begin{problem}
\label{exp_eps}
Пусть с.в. $X$ имеет показательное распределение $\mathrm{Exp}(\lambda)$. Покажите, что имеет место ``отсутствие последействия'': 
$$
{\mathbb P}(X>x+y\,|\, X>x)={\mathbb P}(X>y) . 
$$
Верно ли обратное утверждение?
\end{problem}
\begin{remark}
Следует также обратить внимание на задачу \ref{sec:poisson} раздела \ref{zb4}.
\end{remark}

\begin{problem}(А.Н. Соболевский)
Пусть плотность $p(x,y)$ задает равномерное распределение на единичном квадрате $(0, 1)^2$. Проверьте, что $p(x | X = Y )$, понимаемая как предел при $\delta \to 0$ условной плотности $p(x | -\delta < X - Y <  \delta)$, постоянна и равна 1, в то время как аналогичный предел для $p (x | 1 - \delta < X/Y < 1 + \delta)$ равен $2x$. 
\end{problem}



\begin{comment}
\begin{problem}[распределение Коши]
Радиоактивный источник испускает 
частицы в случайном направлении (при этом все направления равновероятны). 
Источник находится на расстоянии $d$ от фотопластины, которая представляет 
собой бесконечную вертикальную плоскость.

\begin{enumerate}
\item При условии, что частица попадает в плоскость, покажите, что 
горизонтальная координата точки попадания (если начало координат выбирается 
в точке, ближайшей к источнику) имеет плотность распределения:
\[
p\left( x \right)=\frac{d}{\pi \left( {d^2+x^2} \right)}.
\]
Это распределение известно как \textit{распределение Коши}.

\item Можно ли вычислить среднее (математическое ожидание) этого 
распределения?
\end{enumerate}
\end{problem}
\end{comment}
