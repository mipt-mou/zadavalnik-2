%\subsection{Дискретное равномерное распределение}


\begin{problem}
Некто имеет $N$ ключей, из которых только один от его двери. Какова вероятность того, что, используя ключи в случайном порядке, 
он откроет дверь 
\begin{enumerate}
\item[а)] первым ключом; 
\item[б)] последним ключом? 
\end{enumerate}
Требуется найти вероятность того, что потребуется не менее $k$ попыток, чтобы открыть дверь, если ключи, которые не подошли 
\begin{enumerate}
\item[в)] откладываются; 
\item[г)] не откладываются. 
\end{enumerate}

\begin{ordre}
В пункте в) воспользоваться заменой вероятности бинарной величины на математическое ожидание.
\end{ordre}

\end{problem}


\begin{problem}
Ребенок играет с десятью буквами разрезной азбуки: А, А, А, Е, И, К, М, М, Т, Т. 
Какова вероятность того, что при случайном расположении букв в ряд он получит слово <<МАТЕМАТИКА>>? 
\end{problem}


\begin{problem}
Казалось бы, при бросании двух игральных костей как 9, так и 10 можно получить двумя разными способами: $9 = 3+6 = 4+5$, $10 = 4+6 = 5+5$. В случае бросания трех костей 9 и 10 получаются 6 различными способами. Почему тогда 9 появляется чаще, когда бросают две кости, а 10, когда бросают три? 
\end{problem}



\begin{problem}(Парадокс раздела ставки \cite{2013}, \cite{book12})
Два игрока играют в безобидную игру (с вероятностью победы в каждой из партий 1/2). Они договорились, что тот, кто первым выиграет 6 партий, получит  весь приз. Предположим, что игроки не успели доиграть: первый выиграл 5 партий а, второй -- 3. Как следует разделить приз?   
\end{problem}

\begin{problem}(Вероятностный парадокс лжеца \cite{book2012})
Если Вы выберете ответ на этот вопрос случайно, какова вероятность правильного ответа?

а) 25{\%}; б) 50{\%}; в) 60{\%}; г) 25{\%}.
\end{problem}

\begin{problem}
Из урны, содержащей $a$ белых, $b$ черных и $c$ красных шаров, последовательно извлекаются три шара. Найдите 
вероятность следующих событий: 
\begin{enumerate}
\item[а)] все три шара разного цвета; 
\item[б)] шары извлечены в последовательности белый, черный, красный; 
\item[в)] шары извлечены в обратной последовательности. 
\end{enumerate}
\end{problem}



\begin{problem}
Из урны, содержащей $a$ белых и $b$ черных шаров, извлекается наугад один шар и откладывается в сторону. Какова вероятность 
того, что извлеченный наугад второй шар окажется белым, если: 
\begin{enumerate}
\item[а)] первый извлеченный шар белый; 
\item[б)] цвет  первого извлеченного шара остается неизвестным? 
\end{enumerate}
\end{problem}


\begin{problem}(Задача Стефана Банаха \cite{2013}, \cite{29})
В двух спичечных коробках имеется по $n$ спичек. На каждом шаге наугад выбирается коробок, и из него удаляется (используется) 
одна спичка. Найдите вероятность того, что в момент, когда один из коробков опустеет, в другом останется $k$ спичек. 
\end{problem}

\begin{ordre}
Событие, удовлетворяющее условию задачи -- из выбранного коробка взяли последнюю спичку, а в другом коробке имеется $k$ спичек. 
\end{ordre}

\begin{problem}
Партия продукции состоит из десяти изделий, среди которых два изделия дефектные. Какова вероятность того, что из пяти отобранных 
наугад и проверенных изделий: 
\begin{enumerate}
\item[а)] ровно одно изделие дефектное; 
\item[б)] ровно два изделия дефектные; 
\item[в)] хотя бы одно изделие дефектное? 
\end{enumerate} 

\begin{ordre}
В пункте в) удобнее искать вероятность противоположного события.
\end{ordre}

\end{problem}

\begin{problem}
Известно, что в результате бросания десяти игральных костей выпала  по крайней мере одна «шестерка». Какова вероятность того, что число выпавших «шестерок» больше единицы?
\end{problem}

\begin{problem}
Найти вероятность того, что из $50$ студентов, присутствующих на лекции по теории вероятностей, хотя бы двое имеют одну и ту же дату рождения. 

\begin{remark}
Для получения приближенного решения можно воспользоваться свойством линейности математического ожидания следующих событий $X_{ij}$ -- у  $i$-го и  $j$-го человека дни рождения совпадают.  
\end{remark}

\end{problem}


\begin{problem}
В урне находится $m$ шаров, из которых $m_1$ белых и $m_2$ черных $(m_1 + m_2 = m)$. 
Производится $n$ извлечений одного шара с возвращением его (после определения его цвета) обратно в урну. Найти вероятность того, 
что ровно $r$ раз из $n$ будет извлечен белый шар. 
\end{problem}


\begin{problem}
Найти вероятность того, что при размещении $n$ различных шаров по $N$ различимым ящикам заданный ящик будет содержать ровно 
$k$: $0\leqslant k\leqslant n$, шаров (все различимые размещения равновероятны). 
\end{problem}


\begin{problem}
В урне находится $m$ шаров, из которых $m_1$ --- первого цвета, $m_2$ --- второго цвета, $\ldots$, $m_s$ --- $s$-го цвета 
$(m_1+m_2+\ldots +m_s=m)$. 
Производится $n$ извлечений одного шара с возвращением его (после определения его цвета) обратно в урну. Найти вероятность того, 
что $r_1$ раз будет извлечен шар первого цвета, $r_2$ раз --- шар второго цвета, $\ldots$, $r_s$ раз --- шар $s$-го цвета 
$(r_1+r_2+\ldots +r_s=n)$. 
\end{problem}


\begin{problem}
\label{sec:clubok}
Из клубка с $n$ разноцветными веревочками выходит $2n$ концов. Свяжем их попарно в случайном порядке (все возможные варианты связок равновероятны); получится несколько (зацепленных друг за друга) веревочных колец разной длины. Найдите математическое ожидание числа колец.
\end{problem}

\begin{ordre}
Задача (взятая из \cite{book2012}) легко сводится к подсчету математического ожидания числа циклов в случайной перестановке (см. задачу \ref{permloop} из раздела \ref{genF}). 
\end{ordre}


\begin{problem}
\label{vkl_iskl}
Пусть $A_1,\ldots,A_n$ -- последовательность событий, для которой известно, что:
\[
\PR_1 = \underset{i = 1}{\overset{n}{\sum}} \PR(A_i), \;
\PR_2 = \underset{i < j}{\sum} \PR(A_i \cap A_j), \;
\PR_k = \underset{i_1 < \ldots < i_k}{\sum} \PR(A_{i_1} \cap \ldots \cap A_{i_k}).
\]

 Покажите, что для всех нечетных $k$ справедливо неравенство
 \[
 \PR(A_{i_1} \cup \ldots \cup A_{i_k}) \leq \underset{j = 1}{\overset{k}{\sum}} (-1)^{j+1}\PR_j,    
 \]
  а для всех четных $k$
 
 \[
 \PR(A_{i_1} \cup \ldots \cup A_{i_k}) \geq \underset{j = 1}{\overset{k}{\sum}} (-1)^{j+1}\PR_j.	
 \]

 
\end{problem}

\begin{remark}
При $k=n$ получим неравенство, известное как \textit{формула включений-исключений}.
\end{remark}

\begin{problem}
В гардеробе случайным образом перепутались $N$ одинаковых шляп посетителей. Какова вероятность того, что хотя бы один посетитель получит свою шляпу (рассмотреть случаи $N=4$, $N=10000$)?  
\end{problem}

\begin{ordre}
Воспользоваться формулой включений-исключений (см. задачу \ref{vkl_iskl}).
\end{ordre}

\begin{problem}
Несколько раз бросается игральная кость. Какое событие более вероятно: 
\begin{enumerate}
\item[а)] сумма выпавших очков четна; 
\item[б)] сумма выпавших очков нечетна? 
\end{enumerate}
\end{problem}


\begin{problem}
Для уменьшения общего количества игр $2n$ команд спортсменов разбиваются на две подгруппы. Определить вероятности того, что 
две наиболее сильные команды окажутся: 
\begin{enumerate}
\item[а)] в одной подгруппе; 
\item[б)] в разных подгруппах. 
\end{enumerate}
\end{problem}


\begin{problem}
В урне находятся $a$ белых и $b$ черных шаров, которые наугад по одному без возвращения извлекаются из урны до тех пор, пока урна не опустеет. 
Какое событие более вероятно: 
\begin{enumerate}
\item[а)] первый извлеченный шар белый; 
\item[б)] последний извлеченный шар белый? 
\end{enumerate}
\end{problem}



\begin{problem}
В урне находятся $a$ белых и $b$ черных шаров. Шары наугад по одному извлекаются из урны без возвращения. Найти вероятность того, 
что $k$-й вынутый шар оказался белым. 
\end{problem}


\begin{problem}
$30$ шаров размещаются по $8$ ящикам так, что для каждого шара одинаково возможно попадание в любой ящик. Найти вероятность 
размещения, при котором будет $3$ пустых ящика, $2$ ящика --- с тремя, $2$ ящика --- с шестью и $1$ ящик --- с двенадцатью шарами. 
\end{problem}


\begin{problem}
$N$ частиц случайно и независимо друг от друга размещаются в $k$ ячейках так, что каждая из них попадает 
в $i$-ую ячейку с вероятностью $p_i$ $(i=1,\ldots,k, \sum\limits_{i=1}^{k} p_i=1)$. Найти вероятность того, что число частиц в ячейках 
примет заданные значения $n_1$, $\ldots$, $n_i$, $\ldots$, $n_k$ (полиномиальное распределение). 
\end{problem}

\begin{problem}
Из $n$ лотерейных билетов $k$ --- выигрышные $(n\geqslant 2k)$. Какова вероятность, что среди $k$ купленных билетов по крайней мере 
один будет выигрышным? 
\end{problem}

\begin{comment}
\begin{problem}
Из совокупности всех подмножеств множества $\{1,2,\ldots,N\}$ по схеме выбора с возвращением выбираются множества $A$ и $B$. 
Найти вероятность, что $A$ и $B$ не пересекаются. 
\end{problem}
\end{comment}





\begin{comment}
\begin{problem}
В самолете $n$ мест. Есть $n$ пассажиров, выстроившихся друг за другом в очередь. Во главе очереди -- <<заяц>>. У всех, 
кроме <<зайца>>, есть билет, на котором указан номер посадочного места. Так как <<заяц>> входит первым, он случайным образом занимает 
некоторое место. Каждый следующий пассажир, входящий в салон самолета, действует по такому принципу: если его место свободно, то 
садится на него, если занято, то занимает с равной вероятностью любое свободное. Найдите вероятность того, что последний пассажир 
сядет на свое место. 
\end{problem}
\end{comment}


\begin{problem}(Какие автобусы переполнены? \cite{book2012}) Двое кончают работу 
одновременно и идут к автобусной остановке вместе. Каждый уезжает на первом 
подъехавшем автобусе: одному нужен 24-й автобус, другому 25-й. При этом 
первый считает, что в среднем 24-е автобусы более полные, а второй -- что 
25-е. Почему так может быть?
\end{problem}

\begin{problem}(Задача Кавалера де Мере) 
Что более вероятно: при одновременном бросании четырех игральных костей получить хотя бы одну единицу или при $24$ бросаниях 
 по две игральные кости одновременно получить хотя бы один раз две единицы? Найти вероятности указанных событий. 
\end{problem}


\begin{problem}
Из $2^N$ множеств: совокупности всех подмножеств множества $\{1,2,\ldots,N\}$ случайно и независимо выбираются 2 множества $A$ и $B$. Найти вероятность, что $A$ и $B$ не пересекаются.
\end{problem}

\begin{problem}(Н.Н. Константинов \cite{book2012})
В самолете $n$ мест. Есть $n$ пассажиров, выстроившихся друг за другом в очередь. Во главе очереди -- ``заяц''. У всех, 
кроме ``зайца'', есть билет, на котором указан номер посадочного места. Так как ``заяц'' входит первым, он случайным образом занимает 
некоторое место. Каждый следующий пассажир, входящий в салон самолета, действует по такому принципу: если его место свободно, то 
садится на него, если занято, то занимает с равной вероятностью любое свободное. Найдите вероятность того, что последний пассажир 
сядет на свое место. 
\end{problem}


