%\subsection{Моментные характеристики}


\begin{problem}(Обобщённое неравенство Чебышёва)
Пусть $g$ -- неотрицательная и неубывающая функция такая, что $\Exp g(|\xi|) < \infty$. Покажите, что для всех $x$, удовлетворяющих условию $g(x) > 0$
\[
\PR(|\xi| \geq x) \leq \frac{ \Exp g(|\xi|) }{ g(x)}.
\]
Докажите следующие следствия (см. \cite{21}):
\begin{enumerate}
\item (Неравенство Маркова): $\Exp |\xi|^p < \infty$, $p > 0$, $x > 0$
\[
\PR(|\xi| \geq x) \leq \frac{ \Exp |\xi|^p }{ x^p}.
\]
\item (Неравенство Чебышёва): $\Var \xi < \infty$, $x > 0$
\[
\PR(|\xi - \Exp \xi| \geq x) \leq \frac{ \Var \xi }{ x^2}.
\]
\item (Экспоненциальное неравенство Чебышёва): $\Exp e^{t \xi} < \infty$, $t > 0$
\[
\PR(\xi \geq x) \leq \frac{ \Exp e^{t \xi} }{ e^{t x} }.
\]
\item (Неравенство Кантелли): $\Var \xi < \infty$, $x > 0$
\[
\PR( \xi - \Exp \xi \geq x) \leq \frac{ \Var \xi }{ x^2 + \Var \xi}.
\]
\end{enumerate}

\end{problem}

\begin{problem}
Докажите \textit{неравенство Ляпунова}: $0 < p < q$ 
\[
(\Exp |\xi|^p)^{1/p} \leq (\Exp |\xi|^q)^{1/q}. 
\]
\end{problem}

\begin{ordre}
Воспользуйтесь неравенством Йенсена: $\Exp |\xi| < \infty$, $g$ -- выпуклая функция
\[
g(\Exp \xi) \leq \Exp g(\xi). 
\]

\end{ordre}

\begin{problem}
Докажите \textit{неравенство Гёльдера}: $1 < p < \infty$, $1/p + 1/q = 1$, $\Exp |\xi|^p < \infty$, $\Exp |\eta|^q < \infty$  
\[
\Exp |\xi \eta | \leq (\Exp |\xi|^p)^{1/p} (\Exp |\eta|^q)^{1/q}. 
\]
\end{problem}

\begin{ordre}
Воспользуйтесь неравенством: $x > 0$, $y > 0$ 
\[
x y \leq \frac{x^p}{p} + \frac{y^q}{q}.  
\]

\end{ordre}

\begin{problem}
Докажите \textit{неравенство Минковского}: $1 \geq p < \infty$, $\Exp |\xi|^p < \infty$, $\Exp |\eta|^p < \infty$  
\[
(\Exp |\xi + \eta |^p)^{1/p} \leq (\Exp |\xi|^p)^{1/p} + (\Exp |\eta|^p)^{1/p}. 
\]
\end{problem}

\begin{ordre}
Воспользуйтесь неравенством: $p > 0$ 
\[
\left| \sum \limits_{k=1}^n x_i \right|^p \leq \max(1, n^{p-1}) \sum \limits_{k=1}^n |x_i|^p  
\]
и неравенством Гёльдера.
\end{ordre}

\begin{problem}
Существует ли случайная величина с конечным вторым моментом и бесконечным первым моментом?
\end{problem}

\begin{problem}
В начале карточной игры принято с помощью жребия определять первого сдающего. Для этого колода хорошо тасуется, и игрокам сдается по одной карте до появления первого туза. Кому выпал туз -- тот и сдает в первой игре. На каком месте в среднем появляется первый туз, если в колоде 32 карты (то есть, нужно найти математическое ожидание случайной величины «Число карт, сданных до первого туза»)?
\end{problem}



\begin{problem}

 В лотерее на выигрыш уходит 40\% от стоимости проданных билетов. Каждый билет стоит 100 рублей. Доказать, что вероятность выиграть 5000 рублей (или больше) меньше 1\%.

Искомая вероятность зависит, конечно, от правил лотереи, но ни при каких условиях она не превосходит $1\%$.
Приведите пример правил лотереи, где искомая вероятность минимальна и максимальна.

\begin{ordre} 
Использовать \textit{неравенство Маркова} (см. \cite{book2012}).
\end{ordre}

\end{problem}

\begin{comment}
\begin{problem}

Покажите, что все моменты распределения

 $p_{\lambda } \left(x\right)=\frac{1}{24} e^{-x^{{1\mathord{\left/ {\vphantom {1 4}} \right. \kern-\nulldelimiterspace} 4} } } \left(1-\lambda \sin x^{{1\mathord{\left/ {\vphantom {1 4}} \right. \kern-\nulldelimiterspace} 4} } \right)$, $x\ge 0$ при любом значении параметра $\lambda \in \left[0,1\right]$ совпадают.

\begin{remark}

Необходимое и достаточное условие того, чтобы моменты однозначно определяли распределение, вообще говоря, комплексной случайной величины $x$, имеет вид:

\noindent $\sum _{n=0}^{\infty }\left(M\left(\left|x\right|^{2n} \right)\right)^{{-1\mathord{\left/ {\vphantom {-1 \left(2n\right)}} \right. \kern-\nulldelimiterspace} \left(2n\right)} } =\infty  $ (условие Карлемана).
\end{remark}

\end{problem} 
\end{comment}

\begin{comment}
\begin{problem}
В каждую $i$-ую единицу времени живая клетка получает случайную дозу облучения $X_i$, причем $\{ X_i\}_{i=1}^{t}$ имеют 
одинаковую функцию распределения $F_X(x)$ и независимы в совокупности $\forall t$. Получив интегральную дозу облучения, 
равную $\nu$, клетка погибает. Оценить среднее время жизни клетки ${\mathbb E}T$. 
\end{problem}

\begin{ordre}

Показать тождество Вальда: 
$$
{\mathbb E}S_T={\mathbb E}X\cdot {\mathbb E}T, 
$$

введя вспомогательную случайную величину

$$
Y_j=\begin{cases}
1, &\text{ если }\quad X_1+\ldots +X_{j-1}=S_{j-1}<\nu, \\
0, &\text{ в остальных случаях }. 
\end{cases}
$$
 

\end{ordre}

\end{comment}


\begin{problem}
Восемь мальчиков и семь девочек купили билеты в кинотеатр на $15$ подряд идущих мест. Все $15!$ возможных способов рассадки равновероятны. Вычислите среднее число пар рядом сидящих мальчика и девочки. Например, (\mars, \female, \mars, \female) содержит три такие пары. 
\end{problem}


\begin{problem}
На первом этаже семнадцатиэтажного общежития МФТИ в лифт вошли десять человек. Предполагая, что каждый из вошедших (независимо от остальных) может с равной вероятностью жить на любом из шестнадцати этажей (со 2-го по 17-й), найдите математическое ожидание числа остановок лифта.
\end{problem}



\begin{problem}
\label{sec:latters}
Имеется $n$ пронумерованных писем и $n$ пронумерованных конвертов. Письма случайным образом раскладываются по конвертам (все $n!$ 
способов равновероятны). 
Найдите математическое ожидание числа совпадений номеров письма и конверта (письмо лежит в конверте с тем же номером). 
\end{problem}

\begin{problem}
\label{mom_ineq}
С.в. $\xi$ имеет ограничение $\PR(\xi \geq h(x)) \leq e^{-x}$ при $x \geq x_0$, $h(x)$ -- абсолютно непрерывная функция. Докажите неравенства для первого и второго моментов $\xi$:
\[
\Exp \xi \leq h(x_0) + \int_{x_0}^{\infty}h'(x)e^{-x}dx,
\]
\[
\Exp \xi^2 \leq h^2(x_0) +  2\int_{x_0}^{\infty}h(x) h'(x)e^{-x}dx.
\]
\end{problem}

\begin{ordre}
Удобно воспользоваться следующей формулой для подсчета математического ожидания 
\[
\Exp \xi = \int_{-\infty}^{\infty} \PR(\xi \geq h(x)) - \PR(\xi < -h(x))d h(x).
\]
\end{ordre}





\begin{comment}

\begin{problem}

Требуется определить начиная с какого этажа брошенный с балкона 100-этажного здания стеклянный шар разбивается. В наличии имеется два таких шара. Предложить метод нахождения граничного этажа, минимизирующий математическое ожидание числа бросков. Рассмотреть случай большего числа шаров.  

\end{problem}

\begin{problem}
На подоконнике лежат $N$ помидоров. У каждого из них предначертана своя судьба - сгнить вечером $i$-го дня ($1 \leqslant i \leqslant N$, причем ни у каких двух помидоров судьба не совпадает). Утром каждого дня приходит человек и случайным образом съедает один свежий помидор из оставшихся. Таким образом, каждый помидор либо сгнивает в свой предначертанный день, либо употребляется ранее в качестве пищи.

\begin{enumerate}
\item Получите рекуррентную формулу для математического ожидания съеденных помидоров от числа $N$.
\item Найдите асимптотическую оценку количества съеденных помидоров при $N \rightarrow \infty$.
\end{enumerate}

\end{problem}

\end{comment}