%\subsection{Независимость и условные вероятности}

\begin{problem}
Приведите пример вероятностного пространства и трёх событий на этом пространстве, которые попарно независимы, но зависимы в совокупности. Предложите обобщение полученной конструкции, в котором любые $n$ из $(n+1)$ событий независимы в совокупности, а все $(n+1)$ -- зависимы в совокупности.
\end{problem}

\begin{ordre}
См. задачу \ref{bernshtein}, \cite{book12}, \cite{stoianov}.
\end{ordre}

\begin{problem}
Показать, что из независимости событий $A$ и $B$ следует независимость событий $A$ и $\overline B$, $\overline A$ и $B$, 
$\overline A$ и $\overline B$. 
\end{problem}


\begin{problem}
Показать, что из равенства ${\mathbb P}(A\, |\, B)={\mathbb P}(A\, |\, \overline B)$ для ненулевых событий $A$ и $B$ следует 
равенство ${\mathbb P}(AB)={\mathbb P}(A){\mathbb P}(B)$, т.е. их независимость. 
\end{problem}


\begin{problem}(Пример Бернштейна, \cite{stoianov})
\label{bernshtein}
Подбрасываются три игральные кости. События $A$, $B$ и $C$ означают выпадение одинакового числа очков (соответственно) на первой и 
второй, на второй и третьей, на первой и третьей костях. Являются ли эти события независимыми 
\begin{enumerate}
\item[а)] попарно, 
\item[б)] в совокупности? 
\end{enumerate}
\end{problem}


\begin{problem}
Имеется три картонки. На одной с обеих сторон нарисована буква $A$, на другой -- $B$, а на третьей с одной стороны -- $A$, с другой -- $B$. Одна из картонок выбирается наугад и кладется на стол. Предположим, что на видимой стороне картонки оказывается буква $A$. Какова вероятность, что на второй стороне тоже $A$?  (См. \cite{2013}).
\end{problem}




\begin{problem}
Пусть $A$, $B$, $C$ --- заданные события. Доказать справедливость неравенств 
\begin{enumerate}
\item ${\mathbb P}(AB)+{\mathbb P}(AC)+{\mathbb P}(BC)\geqslant {\mathbb P}(A)+{\mathbb P}(B)+{\mathbb P}(C)-1$; 
\item ${\mathbb P}(AB)+{\mathbb P}(AC)-{\mathbb P}(BC)\leqslant {\mathbb P}(A)$; 
\item ${\mathbb P}(A\bigtriangleup B)\leqslant {\mathbb P}(A\bigtriangleup C)+{\mathbb P}(C\bigtriangleup B).$ 
\end{enumerate}
\end{problem}

\begin{problem}(Задача A.A. Натана \cite{5})
Вероятности конъюнкций событий $A$, $B$ и $C$ приведены в таблице истинности: 
\vspace{0.3cm}

\begin{tabular}{|c|c|c|c|}
\hline
$\quad A\quad$ & $\quad B\quad$ & $\quad C\quad$ & $\quad {\mathbb P}(ABC)\quad$ \\
\hline
$0$ & $0$ & $0$ & $20/36$ \\
\hline
$0$ & $0$ & $1$ & $5/36$ \\
\hline
$0$ & $1$ & $0$ & $5/36$ \\
\hline
$0$ & $1$ & $1$ & $0$ \\
\hline
$1$ & $0$ & $0$ & $5/36$ \\
\hline
$1$ & $0$ & $1$ & $0$ \\
\hline
$1$ & $1$ & $0$ & $0$ \\
\hline
$1$ & $1$ & $1$ & $1/36$ \\
\hline
\end{tabular}

\vspace{0.3cm}

Можно ли наблюдаемые события $A$ и $B$ использовать как признаки, используемые для обнаружения события $C$? 
\end{problem}

\begin{problem}
Юноша собирается сыграть три теннисных матча со своими родителями, и для победы он должен победить два раза подряд. 
Порядок матчей может быть следующим: отец--мать--отец, мать--отец--мать. Юноше нужно решить, какой порядок для него предпочтительней, 
учитывая, что отец играет лучше матери (См. \cite{book12}). 

\end{problem}


\begin{problem}
Имеются две урны. В одной из них находится один белый шар, в другой --- один черный шар (других шаров урны не содержат). Выбирается 
наугад одна урна. В нее добавляется один белый шар и после перемешивания один из шаров извлекается. Извлеченный шар оказался белым. 
Определить апостериорную вероятность того, что выбранной оказалась урна, которая первоначально содержала белый шар. 
\end{problem}


\begin{problem}
В первой урне содержится $a$ белых и $b$ черных шаров (и только они), во второй --- $c$ белых и $d$ черных шаров 
(и только они). Из выбранной наугад урны извлекается один шар, который обратно не возвращается. Извлеченный шар оказался белым. 
Найти вероятность того, что и второй шар, извлеченный из той же урны, окажется белым. 
\end{problem}

\begin{problem}(Парадокс Монти Холла \cite{book12})
Представьте, что вы стали участником игры, в которой находитесь перед тремя дверями. Ведущий поместил за одной из трех 
пронумерованных дверей автомобиль, а за двумя другими дверями --- по козе (козы тоже пронумерованы) случайным образом --– это значит, 
что все $3! = 6$ вариантов расположения автомобиля и коз за пронумерованными дверями равновероятны). У вас нет никакой информации 
о том, что за какой дверью находится. Ведущий говорит: <<Сначала вы должны выбрать одну из дверей. После этого я открою одну из 
оставшихся дверей (при этом если вы выберете дверь, за которой находится автомобиль, то я с вероятностью $1/2$ выберу дверь, 
за которой находится коза номер $1$, и с вероятностью $1-1/2=1/2$ дверь, за которой находится коза номер $2$). Затем я предложу 
вам изменить свой первоначальный выбор и выбрать оставшуюся закрытую дверь вместо той, которую вы выбрали сначала. Вы можете 
последовать моему совету и выбрать другую дверь, либо подтвердить свой первоначальный выбор. После этого я открою дверь, 
которую вы выбрали, и вы выиграете то, что находится за этой дверью.>> Вы выбираете дверь номер $3$. Ведущий открывает дверь номер $1$ 
и показывает, что за ней находится коза. Затем ведущий предлагает вам выбрать дверь номер $2$. Увеличатся ли ваши шансы 
выиграть автомобиль, если вы последуете его совету? 
\end{problem}


\begin{problem}
Известно, что $96\%$ выпускаемой продукции соответствует стандарту. Упрощенная схема контроля признает годным с вероятностью 
$0.98$ каждый стандартный экземпляр аппаратуры и с вероятностью $0.05$ -- каждый нестандартный экземпляр аппаратуры. Найти вероятность того, 
что изделие, прошедшее контроль, соответствует стандарту. 
\end{problem}


\begin{problem}
Пусть отличник правильно решает задачу с вероятностью 0.95, а двоечник с вероятностью 0.15. Сколько задач нужно дать на зачете и сколько требовать решить, чтоб отличник не сдал зачет с вероятностью не большей 0.01, а двоечник сдал зачет с вероятностью не большей 0.1?
\end{problem}


\begin{problem}(Задача про мужика)
Пусть некоторый мужик говорит правду с вероятностью 75\% и лжет с 
вероятностью 25\%. Он подбрасывает симметричную кость и говорит, 
что ``выпала 6''. C какой вероятностью выпала 6?
\end{problem}

\begin{problem}
Рассмотрим лампочку, которая может ломаться и устройство, которое чинит лампочку, мгновенно, когда та выходит из строя. При этом считается, что времена работы лампочки от поломки до поломки распределены как $\xi \in \text{Exp}(\lambda)$ и не зависят друг от друга. Чинящее устройство, в свою очередь, тоже может ломаться. Распределение времени работы этого устройства, $\eta \in \text{Exp}(\mu)$, и не зависит от времени работы лампочки. Найти распределение времени работы всей системы в целом. Рассмотреть случаи когда $\lambda = \mu$ и $\lambda \neq \mu$.
\end{problem}



\begin{problem}
В $(m+1)$ урне содержится по $m$ шаров, причем урна с номером $n$ содержит $n$ белых и $(m-n)$ черных шаров $(n = 0,1,\ldots,m)$. 
Случайным образом выбирается урна и из нее $k$ раз с возвращением извлекаются шары. Найти: 
\begin{enumerate}
\item[а)] вероятность того, что следующим также будет извлечен белый шар, при условии, что все $k$ шаров оказались белыми, 
\item[б)] ее предел при $m\to\infty$. 
\end{enumerate}
\end{problem}

\begin{ordre}
Применить \textit{формулу полной вероятности} в следующем виде 
$$
{\mathbb P}(B\, |\, A)=\sum\limits_{n=1} {\mathbb P}(B\, |\, H_n A){\mathbb P}(H_n\, |\, A).
$$
\end{ordre}


\begin{problem}
Параметр $p$ (вероятность выпадаения <<орла>>) в схеме испытаний Бернулли является равномерно распределенной случайной величиной на отрезке $[0.1, 0.9]$ и разыгрывается до начала испытаний, причем $p$ не изменяется от опыта к опыту. В серии из $n=1000$ бросаний было подсчитано число успехов $r=777$.  Найдите условную плотность распределения $p(x|r=777)$. Оцените, как изменится ответ, если известно, что точное значение числа успехов $r\in[750, 790]$.
\end{problem}


\begin{comment}
\begin{problem}
\begin{enumerate}
\item Имеется монетка (несимметричная). Несимметричность монетки заключается в том, что либо орел выпадает в два раза чаще решки; 
либо наоборот (априорно (до проведения опытов) оба варианта считаются равновероятными). Монетку бросили $10$ раз. Орел выпал $7$ раз. 
Определите апостериорную вероятность того, что орел выпадает в два раза чаще решки (апостериорная вероятность считается с учетом 
проведенных опытов (иначе говоря, это просто условная вероятность)). 

\item Определите апостериорную вероятность того, что орел выпадает не менее чем в два раза чаще решки. Если несимметричность 
монетки заключается в том, что либо орел выпадает не менее чем в два раза чаще решки; либо наоборот (априорно оба варианта считаются 
равновероятными). 
\end{enumerate}
\end{problem}
\end{comment}

\begin{problem}
Пусть $X$ и $Y$ ~--– независимые случайные величины,  имеющие распределение Пуассона. 
Доказать, что случайная величина $Z = X + Y$ имеет распределение Пуассона. 
Выявить вид условного распределения случайной величины $X$ при фиксированном значении случайной величины $Z$.
\end{problem}
