
%\subsection{Схема испытаний Бернулли}

\begin{problem}
Опыт состоит в подбрасывании симметричной монеты до тех пор, пока два раза подряд она не выпадет одной и той же стороной. Подбрасывания независимы в совокупности. 
Построить пространство элементарных событий и найти вероятности следующих событий: 
\begin{enumerate}
\item опыт окончится до шестого бросания; 
\item для завершения опыта потребуется четное число бросаний. 
\end{enumerate}
\end{problem}

\begin{problem}(Pаспределение Паскаля или отрицательно биномиальное распределение)
Найдите распределение дискретной случайной величины равной количеству произошедших неудач в последовательности испытаний Бернулли с вероятностью успеха $p$, проводимой до $r$-го успеха.

\end{problem}

\begin{problem}
При каждом подбрасывании монета падает вверх орлом с вероятностью $p>0$. Пусть $\pi _{n} $ -- вероятность того, что число орлов после $n\in {\mathbb N}$ независимых подбрасываний будет чётно. Показав, что $\pi _{n+1} =\left(1-p\right)\cdot \pi _{n} +p\cdot \left(1-\pi _{n} \right)$, $n\in {\mathbb N}$, или иным способом найдите~$\pi _{n} $. 
\end{problem}

\begin{comment}
\begin{problem}
Симметричную монету независимо бросили $n$ раз. Результат бросания записали в виде последовательности нулей и единиц. Покажите, что с вероятностью стремящейся к единице при $n\to \infty $ длина максимальной подпоследовательности из подряд идущих единиц лежит в промежутке
\[\left(\log \sqrt{n} ,\; \log n^{2} \right).\] 
\end{problem}
\end{comment}

\begin{problem}

В так называемой ``безобидной игре'' с повторениями участвуют два игрока. Начальный капитал первого игрока равен $z$ руб, второго~--~$\infty$. В каждой партии первый игрок выигрывает или проигрывает 1 руб с вероятностью $0.5$, независимо от предшествующих партий. Обозначим за $z + s_k$ капитал первого игрока после $k$-й итерации, $\eta(z)$ -- число шагов до разорения:
\[
\eta(z) = \min \{k: z + s_k = 0 \}.
\]     
Покажите, что с вероятностью 1 первый игрок разорится $\PR(\eta(z)<\infty) = 1$, но при этом $\Exp \eta(z) = \infty$.  

\end{problem}

\begin{ordre}
Получите рекуррентные соотношения для вероятности разорения и  $\Exp \eta(z)$. Допустив  $\Exp \eta(z) < \infty$, придите к противоречию, доказав $\Exp \eta(z) < 0$ при больших $z$.
\end{ordre}

\begin{problem}
Сколько раз в среднем нужно подбросить монету, чтобы решка выпала два раза подряд?
\end{problem}



\begin{problem} (Честная игра \cite{book2012})
Два человека играют в {\it орлянку} (кидают симметричную монету, и ставят по рублю: один на орла, другой на решку). У каждого есть своя монетка, и каждый подозревает партнера в несимметричности его монетки. Предложите правила (как имея, эти две монеты ``сгенерировать'' симметричную монету), по которым игра будет гарантированно честной?
\end{problem}