\section{Производящие и характеристические \\ функции}
\label{genF}

Помимо источников литературы, указанных в тексте задач, рекомендуется ознакомиться со следующими книгами:
\begin{itemize}
 \item Айгнер М., Комбинаторная теория. -- М.: Мир, 1982. — 561 с.
 \item Гульден Я., Джексон Д. Перечислительная комбинаторика. -- М.: Наука, 1990 -- 504 с.
 \item Рыбников К.А, Комбинаторный анализ. Очерк истории. -- М.: МехМат, 1996. -- 125с.
 \itemСтенли Р. Перечислительная комбинаторика. --  М.: Мир, 1990. -- 440 с.
\end{itemize}

\begin{problem}(Счастливые билеты)
Трамвайные билеты имеют шестизначные номера. Билет называют счастливым, если 
сумма его первых трех цифр равна сумме трех последних. Вычислите приближенно 
вероятность того, что Вам достанется счастливый билет (предполагается, что 
появление билетов равновероятно).

\end{problem}

\begin{ordre}
Покажите, что число счастливых билетов совпадает с числом билетов, 
сумма цифр у которых равна 27. Запишите \textit{производящую функцию (ПФ)} для 
последовательности $\left\{ {a_k } \right\}_{k=0}^{54} $, где $a_k $ -- число 
билетов с суммой цифр равной $k$. Для нахождения коэффициента ПФ $a_{27} $ 
воспользуйтесь \textit{теоремой Коши} (ТФКП). Для оценки полученного интеграла примените 
\textit{метод стационарной фазы}. (см. книгу Федорюк М.В. Метод перевала. – М.: Наука, 1977, а также \cite{lando}. Последнюю книгу можно также рекомендовать и для решения последующих двух задач).
\end{ordre}

\begin{problem}[Числа Каталана]
\label{sec:katalan}
Пусть в очереди в столовой МФТИ за булочками по цене 10 рублей 50 копеек стоят $2n$ студентов. Пусть у $n$ человек нет пятидесятикопеечной монеты, но есть рублевая, 
а у $n$ человек -- есть монета в 50 копеек. Пусть изначально касса пуста. 
Найдите вероятность события, что никто из студентов в очереди не будет ждать 
свою сдачу. Считайте, что все способы расстановки студентов в очереди равновероятны.
\end{problem}

\begin{ordre}
Задачу можно проинтерпретировать в терминах правильных скобочных структур, описываемых числами Каталана. Если обозначить левую скобку буквой $a$, а правую --- $b$, то можно переписать правильные скобочные структуры в виде «слов» в алфавите $\left\{a,b\right\}$ (\textit{язык Дика}). Несложно показать, что «некоммутативный производящий ряд», перечисляющий слова языка (этот ряд представляет собой просто формальную сумму всех слов языка, вкдючая пустое слово $\lambda$, выписанных в порядке возрастания длины):

\[D(a,b)=\lambda +ab+aabb+abab+aaabbb+aababb+\ldots \] 
удовлетворяет уравнению

\[D(a,b)=\lambda +aD(a,b)bD(a,b).\] 

Перейдите от некоммутативного производящего ряда к обычному, сделав подстановку $a=x$, $b=x$, $\lambda =x^{0} =1$.

Можно искать ПФ языка Дика с помощью \textit{теоремы Лагранжа}, связывающую ее с ПФ подъязыка его неразложимых слов.

\textbf{Определение. }Слово $w=\beta _{1} \ldots \beta _{m} $ языка $L$ называется \textit{неразложимым} в $L$, если никакое его непустое подслово 
$$\beta _{i}\beta _{i+1}\dots \beta _{i+l} , 1\le i,\; i+l\le m,\; l\ge 0$$, отличное от самого слова $w$, не принадлежит языку $L$.

В частности, пустое слово в любом языке, содержащим его, неразложимо.

Несложно проверить, что язык Дика удовлетворяет нижеперечисленным свойствам:

1) пустое слово входит в язык $L$;

2) начало всякого неразложимого слова не совпадает с концом другого или того же самого неразложимого слова;

3) если между любыми двумя буквами любого слова языка $L$ вставить слово языка $L$, то получится слово языка $L$;

4) если из любого слова языка $L$ выкинуть подслово, входящее в язык $L$, то получится слово языка $L$.

Обозначим через $n(y)=n_{0} +n_{1} y+n_{2} y^{2} +\ldots $ ПФ для числа неразложимых слов языка $L$.

\textbf{Теорема.} ПФ $l(x)$ для языка $L$, удовлетворяющего свойствам 1)--4), и ПФ $n(x)$ для подъязыка неразложимых слов в нем связаны между собой \textit{уравнением Лагранжа}

$$l(x)=n\left(xl(x)\right).$$

Приведем уравнение к классическому виду. Положим $xl(x)=\tilde{l}(x)$. Тогда уравнение Лагранжа примет вид:

$$\tilde{l}(x)=xn(\tilde{l}(x)).$$



Неразложимые слова в языке Дика --- это $\lambda $ и $ab$. Отсюда немедленно получаем уравнение $l(x)=1+\left(xl(x)\right)^{2} $ на ПФ для языка Дика.

\textit{Замечание:} Уравнение Лагранжа -- функциональное уравнение, связывающее между собой  ПФ для числа слов в языке и числа неразложимых слов в нем. Оказывается, если одна из функций известна, то оно всегда разрешимо. (см.  указания к следующей задаче об уточнении приведенной здесь теоремы Лагранжа)

\end{ordre}


\begin{problem}[Остовные деревья в полном графе]

Пусть имеется полный граф с $n$ вершинами $\{ 1,2,\ldots ,n\} $. Каждое из $C_n^2=\frac{n(n-1)}{2} $ ребер графа с вероятностью $1/2 $ удаляется. Найдите вероятность того, что полученный после удаления ребер граф будет остовным деревом.

\end{problem}

\begin{ordre}

Обозначим $t_{n} $ -- число остовных деревьев на пронумерованных вершинах $\{ 1,2,\ldots ,n\} $. Ясно, что искомая в задаче вероятность есть $t_n/2^{C_n^2}$.

Выделим одну вершину и посмотрим на те связные компоненты или блоки, на которое разобьется остовное дерево, если проигнорировать все ребра, проходящие через выделенную вершину. Если невыделенные вершины образуют $m$ компонент размеров $k_{1} ,k_{2} ,\ldots ,k_{m} $, то их можно соединить с выделенной вершиной $k_{1} k_{2} \cdot\ldots\cdot k_{m} $ способами.

Такие рассуждения приводят к рекуррентному соотношению

$$ t_{n} = \sum _{m>0}\frac{1}{m!}  \; \sum _{\sum_{i=1}^m k_i = n-1}\left(\begin{array}{c} {n-1} \\ {k_{1} ,k_{2} ,\ldots ,k_{m} } \end{array}\right) \, k_{1} k_{2} \cdot\ldots\cdot k_{m} \, t_{k_{1} } t_{k_{2} }\cdot\ldots\cdot t_{k_{m} }, $$

\noindent при любом $n>1$. 

Теперь обозначим $u_{n} =nt_{n} $, тогда рекуррентное соотношение примет следующий вид:

$$ \frac{u_{n} }{n!} =\sum _{m>0}\frac{1}{m!}  \; \sum _{\sum_{i=1}^m k_i = n-1}\frac{u_{k_{1} } }{k_{1} !} \frac{u_{k_{2} } }{k_{2} !} \cdots \frac{u_{k_{m} } }{k_{m} !}  ,\quad n>1.$$

Обозначим за $U(x)$ \textit{экспоненциальную производящую функцию (ЭПФ)} для последовательности $\left\{u_{n} \right\}$ (то есть $U(x)=\sum _{n=0}^{\infty }\frac{u_{n} }{n!} x^{n}  $). Таким образом,

$$U(x)=xe^{U(x)}.$$

Для нахождения явной формулы для этой последовательности, можно воспользоваться следующим уточнением \textit{теоремы Лагранжа}.

\textbf{Теорема.} Пусть функции $\varphi =\varphi (x)$ ($\varphi (0)=0$) и $\psi =\psi (z)$ связаны между собой уравнением Лагранжа 

$$\varphi (x)=x\psi \left(\varphi (x)\right).$$ 

Тогда коэффициент при $x^{n} $ в функции $\varphi $ равен коэффициенту при $z^{n-1} $ в разложении $\frac{1}{n} \psi ^{n} (z)$: $[x^{n}]\varphi(x) = [z^{n-1}]\frac{1}{n} \psi ^{n} (z)$.

\end{ordre}

\begin{problem}[Задача с марками, \cite{13}]
\label{laplas}
Пусть Вы хотите собрать коллекцию из $N$ 
марок, для этого Вы каждый день покупаете конверт со случайной маркой 
(марки появляются на купленных конвертах равновероятно).
\begin{enumerate}
\item Введем дискретную с.в. $X$, равную номеру впервые купленной Вами 
повторной марки. Найдите математическое ожидание с.в. $X$.

\item Покажите, что распределение случайной величины $X$ имеет асимптотически 
\textit{ распределение Релея } при $n=t\sqrt N $ ($N\gg 1)$:
$$ \PR \left ( {X>t\sqrt N } \right ) \sim e^{-\frac{t^2}{2}},
\quad \PR\left ( {X=t\sqrt N } \right ) \sim \frac{1}{\sqrt N }te^{-\frac{t^2}{2}}.$$
\item Определить математическое ожидание номера купленной Вами марки, которая 
станет недостающей в собранной Вами коллекции марок. 
\end{enumerate}
\end{problem}

\begin{ordre}

а) Покажите, что  
$$\PR \left ( X>n \right ) =\frac{n!}{N^n} \left[ {z^n} \right]\left( {1+z} 
\right)^N=n! \; \left[ {z^n} \right] \left( {1+\frac{z}{N}} \right)^N,$$
$$\Exp X
=
\sum\limits_{n=0}^{\infty} 
n! \; \left[ z^n \right] \left( 1+\frac{z}{N} \right)^N
=
\int\limits_0^{\infty} e^{-t} \left( 1+\frac{t}{N} \right)^N dt. $$

Далее для приближенного вычисления интеграла  
$$\int\limits_0^\infty 
{e^{-t}\left( {1+\frac{t}{N}} \right)^Ndt} =N\int\limits_0^\infty 
{e^{N\left( {\ln (1+u)-u} \right)}du} $$
 воспользуемся \textit{методом Лапласа}:
$$
 \int\limits_0^\infty {f(u)e^{NS(u)}du} \approx f(u_0 ) e^{NS(u_0 
)}\int\limits_{u_0 -\delta }^{u_0 +\delta } {e^{\frac{NS''(u_0 )(u-u_0 
)^2}{2}}} du $$
$$\approx \sqrt {2\pi } \frac{f(u_0 )e^{NS(u_0 )}}{\sqrt {-NS''(u_0 )} }, $$
где $u_0 $ -- единственная точка максимума вещественнозначной функции 
$S(u)$ на полубесконечном интервале $(0, +\infty )$. Основная идея 
асимптотического представления интеграла Лапласа заключается в представлении 
функции $S(u)$ в окрестности точки максимума $u_0 $ в виде ряда Тейлора.

б) Согласно \textit{теореме Коши} (из курса ТФКП): 
$$ \PR(X>n)=n!\left[ {z^n} \right]\left( {1+\frac{z}{N}} 
\right)^N=\frac{1}{n!}\frac{1}{2\pi i }\oint\limits_{\vert z\vert =\rho } 
{\left( {1+\frac{z}{N}} \right)^N\frac{dz}{z^{n+1}}}. $$
Перейдя к полярным координатам, получите, что:
$$
\frac{1}{2i\pi }\oint\limits_{\vert z\vert =\rho } {\left( {1+\frac{z}{N}} 
\right)^N \frac{dz}{z^{n+1}}}  =\frac{1}{2\pi }\int\limits_{-\pi }^\pi {\left( 
{1+\frac{\rho e^{i\theta }}{N}} \right)^N\frac{d\theta }{\rho ^ne^{in\theta 
}}} $$
$$=\frac{1}{2\pi }\int\limits_{-\pi }^\pi {e^{f(\rho e^{i\theta })}d\theta 
} ,
$$
где $f(z)=N\ln \left( {1+\frac{z}{N}} \right)-n\ln z$, $z=\rho 
e^{i\theta }$;

Воспользуйтесь \textit{методом перевала}: разложите в ряд Тейлора функцию $f$ в окрестности седловой 
точки: $$f(\rho e^{i\theta })=f(\rho )-\frac{1}{2}\beta (\rho )\theta 
^2+O(\theta ^3)$$ для $\left| \theta \right|<\delta $ (разложение Тейлора), 
где $$\beta (\rho )=\rho ^2\left. {\left[ {\left( {\frac{d}{dz}} 
\right)^2f(z)} \right]} \right|_{z=\rho }. $$

Замените интеграл по всей окружности $\vert z\vert =\rho $ на интеграл по ее 
части: 
$$
\int\limits_{-\delta }^\delta {e^{f(\rho e^{i\theta })}d\theta } 
\approx e^{f(\rho )}\int\limits_{-\delta }^\delta {e^{-\frac{1}{2}\beta 
(\rho )\theta ^2}d\theta } $$
$$=\frac{e^{f(\rho )}}{\sqrt {\beta (\delta )} 
}\int\limits_{-\delta \sqrt {\beta (\rho )} }^{\delta \sqrt {\beta (\rho )} 
} {e^{-\frac{1}{2}u^2} du}\mathop \to \limits_{\beta (\rho )\to \infty } 
\frac{e^{f(\rho )}}{\sqrt {\beta (\delta )} }\int\limits_{-\infty }^\infty 
{e^{-\frac{1}{2}u^2}du}  $$
$$=\sqrt {2\pi }\frac{e^{f(\rho )}}{\sqrt {\beta 
(\delta )} }.
$$

в) Пусть $Y$ -- дискретная с.в., равная номеру купленной Вами 
марки, которая станет недостающей в собранной Вами коллекции марок.

Тогда $$\PR \left ( Y\le n \right ) =\frac{n!}{N^n}\left[ {z^n} \right](e^z-1)^N=n!\left[ 
{z^n} \right](e^{\frac{z}{N}}-1)^N,$$ а следовательно 
$$\Exp Y=\sum\limits_{n=0}^\infty {n!\left[ {z^n} \right]\left( 
{e^z-(e^{\frac{z}{N}}-1)^N} \right)} =\int\limits_0^\infty {\left[ 
{1-(1-e^{-\frac{t}{N}})^N} \right]} dt.$$
Сделав замену переменных 
$y=1-e^{-\frac{t}{N}}$, получите, что $$\Exp Y=N\left( {1+\frac{1}{2}+\cdots 
+\frac{1}{N}} \right). $$

\end{ordre}

\begin{problem}[Урновая схема, \cite{13}]

Рассмотрим случайное размещение $n$ различных шаров по $m$ различным урнам. 
Пусть случайные величины $MIN$, $MAX$ -- размер наименее или наиболее заполненной 
урны в случайном размещении. Получите функции распределения этих случайных 
величин, а именно $$\PR \left ( MAX\le l\right ) =\frac{n!\left[ {z^n} \right]e_l 
(z)^m}{m^n}=n!\left[ {z^n} \right]e_l \left( {\frac{z}{m}} \right)^m,$$

$$\PR\left ( MIN>l\right ) =n!\left[ {z^n} \right]\left( {e^{\frac{z}{m}}-e_l \left( 
{\frac{z}{m}} \right)} \right)^m,
$$
где $e_l (z)=1+z+\frac{z^2}{2!}+\cdots 
+\frac{z^l}{l!}$.

\end{problem}



\begin{problem}[Задача о циклах в случайной перестановке, \cite{13}]
\\
\begin{enumerate}
\item Найдите математическое ожидание числа циклов длины $r$ в 
случайной перестановке длины $n$.

\item Найдите математическое ожидание числа циклов в 
случайной перестановке длины $n$.

\item (Сто заключенных) 
В коридоре находятся 100 человек, у каждого свой номер (от 1 до 100). Их по одному заводят в комнату, в которой 
находится комод со 100 выдвижными ящиками. В ящики случайным образом 
разложены карточки с номерами (от 1 до 100). Каждому разрешается заглянуть в 
не более чем 50 ящиков. Цель каждого -- определить, в каком ящике находится 
его номер. Общаться и передавать друг другу информацию запрещается. 
Предложите стратегию, которая с вероятностью не меньшей $0.3$ (в 
предположении, что все $100!$ способов распределения карточек по ящикам 
равновероятны) приведет к выигрышу всей команды. Команда выигрывает, если 
все 100 участников верно определили ящик с карточкой своего номера.



\end{enumerate}
\end{problem}
\begin{ordre}
 ЭПФ для последовательностей: 
 
---  
числа циклов длины $n$:
\[C(z)=\sum _{n=0}^{\infty }\frac{(n-1)!}{n!} z^{n}  =\log \frac{1}{1-z}; \]

--- числа множеств на $n$ элементах:
\[S(z)=\sum _{n=0}^{\infty }\frac{1}{n!} z^{n}  =e^{z}; \]

--- числа перестановок на $n$ элементах:
\[P(z)=\sum _{n=0}^{\infty }\frac{n!}{n!} z^{n}  =\frac{1}{1-z}. \]

Перестановка есть не что иное, как совокупность циклов:
\[P(z)=S\left(C(z)\right).\]

Если в такой функции ставить метку (переменную $u$) для циклов длины $r$, получим, что ``двойная''  ЭПФ, перечисляющая число циклов длины $r$ в перестановке длины $n$:

$$\exp \left\{(u-1)\frac{z^{r} }{r} +\log \frac{1}{1-z} \right\}=\frac{\exp \left\{(u-1)\frac{z^{r} }{r} \right\}}{1-z} .$$

Заметим, что $\left[z^{n} \right]\left. \frac{\partial }{\partial u} \left(\frac{\exp \left\{(u-1)\frac{z^{r} }{r} \right\}}{1-z} \right)\right|_{u=1} $ -- среднее (математическое ожидание) число циклов длины $r$ в перестановке длины $n$.

\textbf{Стратегия для ста заключенных} Каждый человек вначале открывает ящик под номером, равным его собственному номеру, затем -- под номером, который указан на карточке, лежащей в ящике, и т.д. Среднее число циклов длины $r$ в случайной 
перестановке -- $1/r$. Тогда среднее число циклов длины большей $n/2$
есть $\sum\limits_{k=n \mathord{\left/ {\vphantom {n 2}} \right. 
\kern-\nulldelimiterspace} 2}^n {\frac{1}{k}} $. Это и есть вероятность 
существования цикла длины большей $n/2$. Поэтому вероятность успеха команды -- 
есть $1-\sum\limits_{k=51}^{100} {\frac{1}{k}} \approx 0,31$ (для сравнения, если 
произвольно открывать ящики, то вероятность успеха будет 
$2^{-100}\approx 10^{-30}$). В случае, когда карточки 
разложены не случайным образом, то следует сделать случайной нумерацию 
ящиков, и далее следовать описанной выше стратегии.
\end{ordre}

\begin{problem}[Задача о беспорядках, \cite{lando}]
\label{permloop}
Группа из $n$ фанатов выигрывающей футбольной команды на радостях 
подбрасывают в воздух свои шляпы. Шляпы возвращаются в случайном порядке ---
по одной к каждому болельщику. Какова вероятность того, что никому из 
фанатов не вернется своя шляпа? Найдите математическое ожидание и дисперсию числа шляп, вернувшихся совим хозяевам.
\end{problem}

\begin{ordre}
Формально задача сводится к подсчету числа беспорядков $d_{n} $ на множестве из $n$ элементов (перестановка $\pi $ элементов множества $\left\{1,2,\ldots ,n\right\}$ называется  \textit{беспорядком}, если $\pi (k)\ne k$ ни при каких $k=1,\ldots ,n$).

Получите ЭПФ для числа беспорядков: $$D(x)=\sum _{n=0}^{\infty }\frac{d_{n} }{n!} x^{n}  =\frac{e^{-x} }{1-x}. $$

Пусть $d_{n,k} $ -- число перестановок на множестве из $n$ элементов, оставляющих на месте ровно $k$ элементов (то есть число неподвижных точек равно $k$), тогда $d_{n,0} =d_{n} $. Более того, $d_{n,k} =C_{n}^{k} d_{n-k} $. Из правила суммы, несложно получить:

$$ n!=\sum _{k=0}^{n}d_{n,k}  =\sum _{k=0}^{n}C_{n}^{k} d_{n-k}  =\sum _{k=0}^{n}C_{n}^{n-k} d_{n-k}  =\sum _{k=0}^{n}C_{n}^{k} d_{k}.$$

То есть получилась биномиальная свертка двух рядов:
$$
\sum _{n=0}^{\infty }\frac{n!}{n!} x^{n}  =\sum _{n=0}^{\infty }\frac{d_{n} }{n!} x^{n}  \sum _{n=0}^{\infty }\frac{1}{n!} x^{n}.
$$
Для нахождения среднего и дисперсии числа шляп, вернувшихся к своим хозяевам, удобно воспользоваться указанием к предыдущей задаче о подсчете среднего и дисперсии циклов длины 1 в слуайной перестановке длины $n$.

Сравните подход с помощью производящих функций для этой задачи с вашими решениями задач \ref{sec:clubok} и \ref{sec:latters} из раздела стандартных задач (раздел \ref{standart}).
\end{ordre}


\begin{problem}[Ожерелья]

Найдите вероятность того, что случайная раскраска ожерелья из $n$ бусин в $k$ различных цветов имеет ровно $m\le n$ бусин первого цвета. Ожерелья, получающиеся одно из другого с помощью плоского поворота, считаются эквивалентными. Положите $n=7$, $k=2$, $m=3$.

\end{problem}

\begin{remark}
Ознакомиться с классическими работами по теории перечисления (Пойа, Дж.К. Рота) можно по книгам

Перечислительные задачи комбинаторного анализа / Сборник переводов под редакцией Г. П. Гаврилова.  — М.: Мир, 1979. -- 362 с.

Краснов М.Л. и др. Вся высшая математика. Учебник. Т.7. -- М.: КомКнига. -- 2006. -- 208 с.
\end{remark}

\begin{ordre}

Сопоставим каждой раскраске функцию $f$ как отображение из множества пронумерованных бусин в нераскрашенном ожерелье в множество пронумерованных бусин в раскрашенном ожерелье. Основной нюанс в решении комбинаторных задач такого типа заключается в том, что некоторые функции (раскраски) оказываются эквивалентными, так как получаются одна из другой с помощью некоторой подстановки (в данном случае задающей поворот в плоскости): $f_{1} \sim f_{2} $, если найдется подстановка $g\in G$, что $f_{1} (g)=f_{2} $, т.е. $\forall i=1,\ldots ,n$   $f_{1} \left(g\left(d_{i} \right)\right)=f_{2} \left(d_{i} \right).$

Для решения этой задачи воспользуемся подходом Пойа.
\textit{Цикловым индексом подстановки} называют одночлен
\[x_{1} ^{k_{1} } x_{2} ^{k_{2} } \ldots x_{n} ^{k_{n} } ,\] 
где $\left(k_{1} ,\; k_{2} ,\; \ldots ,\; k_{n} \right)$ -- тип подстановки, т.е. подстановка представима в виде $k_{1} $ цикла длины 1, $k_{2} $ циклов длины 2,  и т.д.

\textit{Цикловым индексом группы подстановок} $G$ называют среднее арифметическое цикловых индексов ее элементов:
\[P_{G} (x_{1} ,x_{2} ,\ldots ,x_{n} )=\frac{1}{|G|} \sum _{g\in G}x_{1} ^{k_{1} } x_{2} ^{k_{2} } \ldots x_{n} ^{k_{n} }  .\] 
Покажите, что цикловой индекс группы поворотов $C_n$ равен: $$P_{G} (x_{1} ,x_{2} ,\ldots ,x_{n} )=\frac{1}{n} \sum _{j=1}^{n}\left(x_{\frac{n}{(n,j)} } \right)^{(n,j)}  ,$$ где $(n,j)$ --- наибольший общий делитель $n$ и $j$. Каждому цвету $r_{i} $ $i=1,\ldots ,k$ придадим некоторый \textit{вес} $w(r_{i} )$. \textit{Весом функции} $f$ назовем произведения весов полученной раскраски:
\[W(f)=\prod _{i=1}^{n}w(f(d_{i} )).\] 
Ясно, что эквивалентные функции имеют одинаковый вес:
\[f_{1} \sim f_{2} \Rightarrow W\left(f_{1} \right)=W\left(f_{2} \right).\] 
Весом класса эквивалентности называется вес любой функции из этого класса; если $F$ -- класс эквивалентности и $f\in F$, то $W(F)=W(f)$. Заметим, что и у неэквивалентных функций могут совпадать веса.

\textbf{Теорема Пойа (1937 г.)}
Сумма весов классов эквивалентности рана
$$\sum _{F}W(F) =P_{G} \left(\sum _{i=1}^{k}w(r_{i} ) ,\; \sum _{i=1}^{k}w^{2} (r_{i} ) ,\; \sum _{i=1}^{k}w^{3} (r_{i} ) ,\; \ldots ,\sum _{i=1}^{k}w^{n} (r_{i} ) \right),$$ где $P_{G} $ -- цикловой индекс группы подстановок $G$.

\end{ordre}

\textbf{Следствие}. 
Число классов эквивалентности равно 
$$P_{G} \left(k,\; k,\; k,\; \ldots ,k\right).$$ 

Согласно следствию теоремы Пойа число различных ожерелий равно $$P_{G} (k,k,\ldots ,k)=\frac{1}{n} \sum _{j=1}^{n}k^{(n,j)}  =\frac{1}{n} \sum _{s|n}k^{s} \phi \left(\frac{n}{s} \right), $$ где $\phi \left(l\right)$ -- число взаимно простых делителей числа $l$, не превосходящих $l$ (\textit{функция Эйлера}), обозначение $s|n$ -- $s$ является делителем $n$.

Для подсчета числа ожерелий с ровно $m$ бусинами первого цвета, положите вес этого цвета $w(r_{1} )=x$, веса остальных цветов -- единицей. В ПФ (полученной согласно теореме Пойа) возьмите коэффициент при $x^{m} $.



\begin{problem}[Изомеры органических молекул]

Рассмотрим математическую модель органической молекулы: в центре тетраэдра поместим атом углерода $C$, в вершинах тетраэдра равновероятно помещаются метил $\left(CH_{3} \right)$, этил $\left(C_{2} H_{5} \right)$, водород $\left(H\right)$ и хлор $\left(Cl\right)$. Найдите вероятность того, что случайная молекула заданной структуры окажется метаном $CH_{4} $. 

\end{problem}


\begin{ordre}
 Воспользуйтесь подходом Пойа, описанным в указаниях предыдущей задачи. Покажите, что цикловой индекс группы вращения тетраэдра $P_{G} (x_{1} ,x_{2} ,x_{3} ,x_{4} )=\frac{1}{12} \left(8x_{1} x_{3} +3x_{2} ^{2} +x_{1} ^{4} \right).$
\end{ordre}

\begin{comment}
\begin{problem}[Marching cubes]
Сколько различных раскрасок вершин кубов в 2 цвета можно получить с учетом вращений и отражений?
\end{problem}

\begin{remark}
Сокращение числа раскрасок используется для сжатого хранения и визуализации изоповерхности на трехмерном скалярном поле при помощи алгоритма marching cubes: вместо хранения массива цветов всех вершин трехмерной сетки достаточно хранить номер раскраски каждого куба сетки (вершина считается окрашенной в положительный цвет, если значение поля в ней не меньше значения поля на изоповерхности; в отрицательный цвет, если значение поля в ней меньше значения поля на изоповерхности).  
\end{remark}

\end{comment}

\begin{remark}
Для решения последующих трех задач рекомендуется ознакомится подходом Егорычева вычисления комбинаторных сумм, изложенных в  книгах

Леонтьев В.К. Избранные задачи комбинаторного анализа. -- М.: МГТУ, -- 2001. -- 184 с.

Егорычев Г.П. Интегральное представление и вычисление комбинаторных сумм. -- Новосибирск, Наука,  1977. -- 284 с.
\end{remark}

\begin{problem}
Обозначим через $E^n$ -- множество бинарных последовательностей длины $n$, или множество вершин 
единичного $n$-мерного куба, а через $E_k^n $ -- $k$-ый слой куба $E^n$, то есть 
подмножество точек $E^n$, имеющих ровно $k$ единичных координат. Пусть 
$X= \langle x, y \rangle$ -- случайная величина, где $ 
x \in E_p^n $, $ y \in E_q^n $ -- независимые и равномерно 
распределенные на $E_p^n $ и $E_q^n $ соответственно векторы. Обозначим 
через $a_{p,q} (k)=\PR\left\{ {X=k} \right\}$. Доказать следующие утверждения:

\begin{enumerate}
\item $\sum\limits_{k=0}^n {a_{p,q} (k)z^k} =\frac{1}{2\pi i}C_n^p 
\oint\limits_{\left| u \right|=\rho } 
{\frac{(1+zu)^p(1+u)^{n-p}}{u^{q+1}}du}; $

\item $a_{p,q} (k)=\frac{C_p^k C_{n-p}^{q-k} }{C_n^q };$

\item $\Exp X=\frac{pq}{n};$

\item $\Var X=\frac{pq}{n(n-1)}\left( {n+\frac{pq}{n}-(p+q)} \right) .$
\end{enumerate}

\end{problem}

\begin{problem}
Пусть $\xi$ -- случайная величина, равномерно распределенная на множестве всех пар векторов 
$(x,y)\in \{ 0,1\}^n\otimes \{ 0,1\}^n$, равная $ \langle x,y \rangle=\sum\limits_{k=1}^{n} x_k y_k$. Найдите: 
$$
{\mathbb P}(\xi=k), \; {\mathbb E}\,\xi, \; \Var \xi . 
$$
\end{problem}

\begin{problem}
Пусть $\xi$ -- с.в., равномерно распределенная на множестве бинарных матриц ($\{0,1\}^{m\times n}$) порядка 
$m\times n$ и равная числу нулевых столбцов матрицы. Доказать, что 
\begin{enumerate}
\item $P_k(m,n)={\mathbb P}(\xi=k)=C_n^k\cdot \left.\bigl( 2^m -1\bigr)^{n-k}\right/ 2^{m\cdot n}$; 

\item ${\mathbb E}\xi=\left. n\right/2^m$; 
\item если $2^m-1=\alpha\cdot n$, где $\alpha$ не зависит от $n$, то 
$$
\lim\limits_{n\to\infty} P_k(m,n)=e^{-\lambda}\, \frac{\lambda^k}{k!} ,\; \text{ где } \lambda=\alpha^{-1} .
$$
\end{enumerate}
\end{problem}

\begin{comment}
\begin{ordre}

Рассмотрим следующие случайные величины: 
$$
\xi_i=\begin{cases}
1, & \text{ $i$-й столбец нулевой}, \\
0, & \text{ иначе }.
\end{cases} 
$$
Тогда $\xi_i\in\Be$, $\xi=\xi_1+\ldots +\xi_n$. 

Имеет место следующее мультипликативное свойство:
$$
\psi_{\xi}(z)=\bigl[\psi_{\xi_i}(z)\bigr]^n
$$

\end{ordre}
\end{comment}

\begin{problem}
Может ли функция $\varphi(t)=\begin{cases}1,\quad t\in[-T,T]\\
0,\quad t\notin[-T,T] \end{cases}$ -- 
быть \textit {характеристической функцией} (х.ф.) некоторой с.в.? Изменится ли ответ, если <<чуть-чуть>> размазать (сгладить) разрывы функции 
$\varphi(t)$ в точках $t=\pm T$? 
\end{problem}

\begin{comment}
\begin{ordre}
Характеристическая функция обладает следующими свойствами:
\begin{fixme}
Iterate the properties 
\end{fixme}
\end{ordre}
\end{comment}


\begin{problem}
Будут ли функции~$\cos(t^2)$, $\exp(-t^4)$, $\arcsin((\cos t)/2) / \arcsin(1/2)$ характеристическими для каких-нибудь случайных величин?
\end{problem}

\begin{problem}

Докажите, что выпуклая линейная комбинация х.ф. есть х.ф.  ``смеси'' слагаемых. Т.е. если $\phi_{k}(t) = \Exp \exp (i\xi_k t)$, $\sum_k a_k = 1$, $a_k \geq 0$,  то $\sum_k a_k \phi_{k}(t)$ есть х.ф. случайной величины
\[
\xi = \sum_k [z = k] \xi_k,
\]
где $z$ не зависящая от $\{\xi_k\}$ c.в. с распределением $\PR(z=k) = a_k$.
\end{problem}


\begin{problem}
Случайные величины $X_1, X_2$ независимы и имеют распределение Коши $\text{Ca}(x_1, d_1), \; \text{Ca}(x_2, d_2)$, где плотность распределения  $\text{Ca}(x_1, d_1)$ определяется как    
\[
f(x) = \frac{d}{\pi(d^2 + (x-x_1)^2)}.
\]
Докажите, что распределением $X_1 + X_2$ является $\text{Ca}(x_1+x_2, d_1+d_2)$.
\end{problem}

\begin{ordre}
Убедитесь, что характеристическая функция $X_1$ имеет вид $\varphi(t) = e^{ix_1t - d|t|}$.
\end{ordre}

\begin{problem}(Квадратичные формы)
$A$ -- симметричная ($n \times n$)-матрица, $X \in \N(0, S)$, покажите что для квадратичной формы $Q = X^TAX$ справедливы равенства 
\[
\Exp e^{tQ} =\det (I - 2t A S)^{-1/2},
\] 
\[
\Exp Q = \text{tr}(AS), \quad \Var Q = 2 \text{tr}((AS)^2),
\]
\[
\Exp [ (WX +m)^{T} A (WX +m) ] = m^T A m + \text{tr}(W^TAWS),
\]
где $W$ и $m$ -- неслучайные матрицы.
\end{problem}


\begin{problem}
Пусть $\varphi_{\xi}$~-- х.ф. абсолютно непрерывной случайной величины~$\xi$ с плотностью~$p_{\xi}$. Рассмотрим $f_1 =\text{Re}(\varphi_{\xi})$ и~$f_2 = \text{Im}(\varphi_{\xi}) $. Существуют ли случайные величины $\eta_1,\,\eta_2$, для которых $f_1,\,f_2$~являются их х.ф.? 
\end{problem}



\begin{problem}
Реализуем $m$~раз схему из $n$~испытаний Бернулли с вероятностью успеха~$p$. Считаем, что все реализации схем взаимно независимы. На выходе получим $m$~случайных векторов $x_1,\,\dots,\,x_m$ с координатами $0$~и~$1$. Некоторые из этих векторов могут совпадать. Скажем, что векторы $x_i,\,x_j,\,x_k$ образуют прямой угол с вершиной в~$x_k$, если~$vx_i-x_k,\,x_j - x_k) = 0$ (помимо обычных прямых углов, под это определение попадают и <<вырожденные>>, т.\,е. образованные совпадающими векторами). Найдите математическое ожидание числа прямых углов во множестве~$\{x_1,\,\dots,\,x_m\}$.
\end{problem}






\begin{problem} 
Найдите вероятность того, что пара случайно выбранных из $\{ 0,1\}^n$ векторов является ортогональной 
\begin{enumerate}
\item над полем $\mathbb{F}_2=\{ 0,1\}$; 

\item над полем действительных чисел. 
\end{enumerate}
\end{problem}

\begin{comment}
\begin{ordre}
a) Пусть 
$$
\xi_{x,y}=\begin{cases}
1, &\text{ если } (x,y)=0,\\
0, &\text{ если } (x,y)=1.
\end{cases}
$$
Для искомой вероятности тогда имеем 
$$
P=\frac{1}{2^n\cdot 2^n}\sum\limits_{x,y}\xi_{x,y}
$$
\end{ordre}
\end{comment}


\begin{problem}[об оценке хвостов]
Пусть $\Psi (z)=\Exp z^X$ --- ПФ случайной величины  $X$. Докажите, что
\[
{\rm\PR}(X\le r)\le x^{-r}\Psi (x),\mbox{ для }0<x\le 1;
\]
\[
{\rm \PR}(X\ge r)\le x^{-r}\Psi (x),\mbox{ для }x\ge 1.
\]
\begin{remark}
Для решения этой и последующих шести задач можно рекомендовать книгу \cite{29}
\end{remark}
\end{problem}









\begin{problem}[Игра У. Пенни, 1969]
Авдотья и Евлампий играют в игру: они 
бросают монету до тех пор, пока не встретится РРО или РОО. Если первой 
появится последовательность РРО, выигрывает Авдотья, если РОО -- Евлампий. Будет ли игра честной (одинаковы ли вероятности выигрыша у обоих игроков)?
\end{problem}

\begin{ordre}

Справедливы следующие равенства:
\[
\mbox{1+N(Р+О)=N+А+В},
\]
\[
\mbox{NРРО=А},
\]
\[
\mbox{NРОО=В+АО},
\]
где $A$ -- конфигурации, выигрышные для Авдотьи, $B$ --  конфигурации, выигрышные для Евлампия, $N$ -- конфигурации последовательностей, для которых ни один из игроков не выиграл.
Заменяя Р и О на $\frac{1}{2}$, найдите значения $A$ и $B$ - вероятностей выигрыша Авдотьи и Евлампия.
\end{ordre}


\begin{problem}
Теперь трое игроков: Авдотья, Евлампий и Компьютер. Играют пока 
не выпадет одна из следующих последовательностей: А=РРОР, В=РОРР, С=ОРРР. Каковы шансы каждого выиграть?
\end{problem}




\begin{problem}
Рассмотрим следующую игру: первый игрок называет одну из восьми комбинаций ООО, ООР, ОРО, РОО, ОРР, РОР, РРО, РРР, второй игрок называет другую, потом они бросают симметричную монету до тех пор, пока в последовательности орлов (О) и решек (Р) -результатов бросания -- не появится одна из выбранных комбинаций. Тот, чья комбинация появится, выиграл. Будет ли игра честной? Найдите для каждой комбинации, выбранной первым игроком, выгодные комбинации второго игрока.
\end{problem}

\begin{ordre} 
Для решения этой задачи, см. также \ref{book12}.Например, если первый игрок выбрал ОРО, то выбираю ООР, второй игрок выигрывает с вероятностью ${2\mathord{\left/ {\vphantom {2 3}} \right. \kern-\nulldelimiterspace} 3} $. Действительно, пусть $p_{OO} $, $p_{OP} $, $p_{PO} $, $p_{PP} $ вероятности выигрыша первого игрока при последних значениях ОО, ОР, РО, РР (в предположении, что раньше никто не выиграл). Их можно найти из системы уравнений:

\[\left\{\begin{array}{l} {p_{OO} =\frac{1}{2} p_{OO} +\frac{1}{2} \cdot 0,} \\ \\ {p_{OP} =\frac{1}{2} \cdot 1+\frac{1}{2} p_{PP} ,} \\ \\ {p_{PO} =\frac{1}{2} p_{OO} +\frac{1}{2} p_{OP} ,} \\ \\ {p_{PP} =\frac{1}{2} p_{PO} +\frac{1}{2} p_{PP} .} \end{array}\right. \] 

Откуда получаем:

\[p_{OO} =0, \; p_{OP} =\frac{2}{3}, \;  p_{PO} =\frac{1}{3}, \; p_{PP} =\frac{1}{3} .\] 

То есть общая вероятность выигрыша первого игрока равна ${1\mathord{\left/ {\vphantom {1 3}} \right. \kern-\nulldelimiterspace} 3}$.
\end{ordre}

\begin{problem}
В вершине пятиугольника $ABCDE$ 
находится яблоко, а на расстоянии двух ребер, в вершине $C$, находится 
червяк. Каждый день червяк переползает в одну из двух соседних вершин с 
равной вероятностью. Так, через один день червяк окажется в вершине $B$ или 
$D$ с вероятностью $1 \mathord{\left/ {\vphantom {1 2}} \right. 
\kern-\nulldelimiterspace} 2$. По прошествии двух дней червяк может снова 
оказаться в $C$, поскольку он не запоминает своих предыдущих положений. 
Достигнув вершины $A$, червячок останавливается пообедать.

\begin{enumerate}
\item Чему равны математическое ожидание и дисперсия числа дней прошедших до обеда?
\item Какую оценку дает неравенство Чебышёва для вероятности $p$ того, что это число дней будет 100 или больше?
\item Что позволяют сказать о величине $p$-оценки из задачи ``об оценке хвостов''.
\end{enumerate}
\end{problem}


\begin{problem}
Пять человек стоят в вершинах пятиугольника $ABCDE$ и 
бросают друг другу диски фрисби. У них имеется два диска, которые в 
начальный момент находятся в соседних вершинах. В очередной момент времени 
диски бросают либо налево, либо направо с одинаковой вероятностью. Процесс 
продолжается до тех пор, пока обе тарелки не окажутся в одной вершине.

\begin{enumerate}
\item Найдите математическое ожидание и дисперсию числа пар бросков.
\item Найдите ``замкнутое'' выражение через числа Фибоначчи для вероятности того, что игра продлится более 100 шагов.
\end{enumerate}
\end{problem}


\begin{problem}
Обобщите предыдущую задачу на случай $m$-угольника и найдите 
математическое ожидание и дисперсию числа пар бросков до столкновения 
дисков. Докажите, что если $m$ нечетно, то ПФ случайной величины  для числа бросаний 
представима в следующем виде:\footnote{ Воспользуйтесь подстановкой $z=1 
\mathord{\left/ {\vphantom {1 {\cos ^2\theta }}} \right. 
\kern-\nulldelimiterspace} {\cos ^2\theta }$.}
\[
G_m (z)=\prod\limits_{k=1}^{(m-1)/2} {\frac{p_k z}{1-q_k z}} ,
\]
где
\[
p_k =\sin ^2\frac{(2k-1)\pi }{2m},
\quad
q_k =\cos ^2\frac{(2k-1)\pi }{2m}.
\]
\end{problem}




\begin{problem}[Загадочный случайный суп]
Студент, решивший отобедать в 
столовой, может обнаружить в своей тарелке с супом случайное число $N$ 
инородных частиц со средним $\mu $ и конечной дисперсией. С 
вероятностью $p$ выбранная частица является мухой, иначе это таракан; типы 
разных частиц независимы. Пусть $F$ -- количество мух и $S$ -- количество тараканов.

\begin{enumerate}
\item Покажите, что ПФ с.в. $F$ удовлетворяет равенству
\[
\psi _F (s)=\psi _N (ps+1-p).
\]

\item Предположим, что с.в. $N$ имеет пуассоновское 
распределение с параметром $\mu$:  $N\in \Po\left( \mu \right))$. Покажите, что $F$ имеет пуассоновское распределение с 
параметром $p\mu $, а с.в. $F$ и $S$ независимы. Покажите, что
\[
\psi _N (s)=\psi _N^2 \Bigl( {\frac{1}{2}(1+s)} \Bigr).
\]
\end{enumerate}

\end{problem}


\begin{problem}[Соболевский А.Н.]
На каждой упаковке овсянки печатается купон одного из $k$ различных цветов. Считая, что при отдельной покупке купон каждого цвета может встретиться с равной вероятностью и различные покупки независимы, требуется найти производящую функцию распределения вероятности, математическое ожидание и дисперсию минимального числа упаковок, которые придется купить для того, чтобы собрать купоны всех $k$ цветов.
\end{problem}

\begin{ordre}
Воспользуйтесь вспомогательными вероятностями: $p(n)$ -- вероятность разделить $n$ одинаковых объектов на $k$ групп так, что в каждой группе будет не менее  одного объекта, $f(m)$ -- вероятность того, что минимальное число купленных  упаковок равно $m$.
\[
p(n) = \sum_{m = k}^{n} f(m), 
\]
\[
P(z) = \sum_{n=k}^{\infty} p(n) z^n = \sum_{n=k}^{\infty} \sum_{m = k}^{n} f(m) z^n = 
\sum_{n=m}^{\infty} \sum_{m = k}^{\infty} f(m) z^m z^{n-m} = 
\]
\[=
F(z) \frac{1}{1-z}.
\]    

См. также задачу \ref{laplas} из этого раздела и указание к ней.
\end{ordre}


\begin{problem}[Задача о предельной форме диаграмм Юнга]
\label{limung}
\textit{Разбиением} $\lambda $ натурального числа $n$ называется набор натуральных 
чисел $(\lambda _1 ,\ldots ,\lambda _N )$, для которого $\lambda _1 \ge 
\ldots \ge \lambda _N >0$ и $\lambda _1 +\ldots +\lambda _N =n$. 
\textit{Диаграммой Юнга} разбиения $\lambda =(\lambda _1 ,\ldots ,\lambda _N )$ называется 
подмножество первого квадранта плоскости, состоящее из единичных 
квадратиков. Квадратики размещаются по строкам, выровненным по левому краю, 
причем число квадратиков в $i$-ой строке равно $\lambda _i $.
Множество всех диаграмм Юнга, 
соответствующих натуральному числу $n$ (или иначе множество всех разбиений 
натурального числа $n$) обозначим через $\mathcal P_n $.
Введем равномерную меру $\mu _n $ на $\mathcal P_n$, то есть $$\mu _n (\lambda 
)=\frac{1}{p(n)}$$ для всех $\lambda \in \mathcal P_n$, где  $p(n)=|\mathcal P_n|$ -- число разбиений натурального $n$ (число диаграмм Юнга, 
соответствующих натуральному $n$).

Пусть $r_k (\lambda )$ -- число слагаемых в разбиении $\lambda$, равных $k$ (иначе говоря, выполнено равенство $n = \sum_k k r_k (\lambda )$). Число слагаемых в разбиении $\lambda$, больших или равных $\lceil t \rceil$ (где $t$ -- неотрицательное действительное чтсло), обозначим через $\phi _\lambda (t)=\sum\limits_{k\ge t} {r_k (\lambda )} $. Заметим, что $\phi _\lambda (t)$, $t \ge 0$ -- 
ступенчатая функция, непрерывная справа, замыкание внутренности подграфика которой и есть диаграмма Юнга, соответствующая разбиению $\lambda $. Сделаем шкалирование (скейлинг) функции $\phi _\lambda (t)$ с множителем $a$: 
$$\tilde {\phi }_\lambda (t)=\frac{a}{n}\sum\limits_{k\ge at} {r_k (\lambda 
)} =\frac{a}{n}\phi _\lambda (at).$$ Заметим, что после шкалирования 
$$\int\limits_0^\infty {\tilde {\phi }_\lambda (t)dt} =1.$$

\textbf{Теорема (А.М.Вершик).} Пусть $a_n =\sqrt n $, тогда для любого $\varepsilon >0$ 
существует $n_\varepsilon $, что для любого $n>n_\varepsilon $
\[
\mu _n \left\{ {\lambda :\quad \mathop {\sup }\limits_t \left| {\tilde {\phi 
}_\lambda (t)-C(t)} \right|<\varepsilon } \right\}>1-\varepsilon ,
\]
где $C(t)=\int\limits_t^\infty {\frac{e^{-\sqrt {\varsigma (2)} 
u}}{1-e^{-\sqrt {\varsigma (2)} u}}du} =-\frac{\sqrt 6 }{\pi }\ln 
(1-e^{-\left( {\pi \mathord{\left/ {\vphantom {\pi {\sqrt 6 }}} \right. 
\kern-\nulldelimiterspace} {\sqrt 6 }} \right)t})$.

Другими словами, типичная (в смысле равномерной меры) диаграмма Юнга после 
шкалирования $\tilde {\phi }_\lambda (t)=\frac{1}{\sqrt n }\sum\limits_{k\ge 
\sqrt n t} {r_k (\lambda )} $ с ростом $n$ имеет \textit {предельную форму (limit shape)}, заданную 
с помощью $C(t)$ или в более симметричной форме: $e^{-\frac{\pi }{\sqrt 6 
}x}+e^{-\frac{\pi }{\sqrt 6 }y}=1$.

Интерпритация этой задачи с точки зрения статистической физики следующая.  $\mathcal P_n $ -- множество всех состояний системы (Бозе-частиц) 
с постоянной энергией $n$, каждое состояние $\lambda \in \mathcal P_n $ однозначно 
характеризуется набором значений $r_k (\lambda )$ (чисел заполнений $k$-го 
уровня энергии). Заметим, что у Бозе-частиц нет ограничений на значения $r_k 
(\lambda )$, в отличие от Ферми-частиц, где $r_k (\lambda )\le 1$. $\mathcal P_n $ в 
статистической физике называется каноническим ансамблем (равномерная 
мера $\mu _n $ соответствует тому, что все состояния равновероятны). Теорема утверждает, что  шкалированная функция распределения по уровням энергии у системы Бозе-частиц (с полной энергией $n$),  имеет в пределе вид функции $C(t)$ (иначе, теорема утверждает, что существует предельное распределение энергий частиц, позволяющее делать заключение типа: какая предельная
доля общей энергии приходится на частицы с данными энергиями).

\begin{remark}
С детальным доказательством теоремы можно ознакомиться в статье А.М. Вершика ``Статистическая механика комбинаторных разбиений и их предельные конфигурации'', Функциональный анализ и его приложения. -- 1996. -- Т. 30. -- Вып. 2. -- С. 19-39.
\end{remark}

Согласно приведенной выше статьи, далее описаны только основные шаги и идеи доказательства. Читателю требуется с учетом приведенных ниже указаниям обосновать результат о предельной форме диаграмм Юнга.

\begin{ordre} Заметим, что для обоснования существования предельной формы шкалированных диаграмм Юнга, необходимо для начала проверить выполнимость следующих пределов:
$$
\mathop {\lim }\limits_{n\to 
\infty } \mathbb E_{n}\tilde {\phi }_\lambda (t) = C(t),
$$
$$\mathop {\lim }\limits_{n\to 
\infty }\mathbb E_{n } \left[ {\left( {\tilde {\phi }_\lambda (t)-C(t)} 
\right)^2} \right]=0,$$
что на самомо деле не так просто.

Основная идея обоснования утверждения теоремы заключается, во-первых, в переходе к пространству всех диаграмм Юнга $\mathcal P=\bigcup\limits_n {\mathcal P_n} $ с семейством (по $x$) мер $\mu _x $:
$$\mu _x (\lambda )=\frac{x^{\sum\limits_i {\lambda _i } 
}}{F(x)}=\frac{x^{n(\lambda )}}{F(x)}, \text{ где } F(x)=\sum\limits_{n=0}^\infty 
{p(n)x^n},$$
которые в свою очередь индуцируют на пространстве $\mathcal P_n $ исходную равномерную меру  $\mu _n $,
во-вторых, обосновании утверждения указанной теоремы для семейства введенных мер $\mu _x $ при $x \to 1-0$ (что существенно проще, см. далее). Представление меры $\mu_x$ как выпуклой комбинации мер $\mu_n$ (см. утверждение 3 ниже) позволяют делать заключение, что тот же предел (найденный для мер  $\mu _x $ при $x \to 1-0$) будет и для мер $\mu _n $ при $n \to \infty$

С точки зрения статистической физики $\mathcal P$ 
соответствует (большому) каноническому ансамблю с мерой Гиббса $\mu _x 
(\lambda )=\frac{x^{n(\lambda )}}{F(x)}$, где $F(x)=\sum\limits_\lambda 
{x^{\sum\limits_i {\lambda _i } }} =\sum\limits_{n=0}^\infty 
{p(n)x^{n(\lambda )}} $ -- статистическая сумма. Покажите, что для 
статистической суммы $F(x)$ (или иначе производящей функции последовательности $p(n)$) справедлива формула Эйлера: 
$$F(x)=\sum\limits_{n=0}^\infty {p(n)x^n} =\prod\limits_{k=1}^\infty 
{(1-x^k)^{-1}}.$$

Переход от малого к большому каноническому ансамблю упрощает решение задачи в силу  мультипликативности мер $\mu _x $ (см. утверждение 2 
ниже).

Проверьте справедливость следующих \textbf{утверждений}:

1) $\left. {\mu _x (\lambda )} \right|_{\mathcal P_n } \equiv \frac{\mu _x (\lambda 
)}{\mu _x (\mathcal P_n)}=\mu _n (\lambda )$, то есть мера на $\mathcal P_n $, индуцированная 
мерой $\mu _x $ совпадает с равномерной мерой $\mu _n $.

2) $r_1 (\lambda ),\;r_2 (\lambda ),\ldots$ как случайные величины на $\mathcal P$ 
независимы относительно мер $\mu _x $.

\textbf{Указание}: Проверьте, что $\mu _x \left( {\lambda :\;r_k (\lambda 
)=s} \right)=x^{ks}(1-x^k)$.

3) Мера $\mu _x $ является выпуклой комбинацией мер $\mu _n $.

\textbf{Указание}: $\mu _x =\sum\limits_{n=0}^\infty 
{\frac{x^np(n)}{F(x)}\mu _n } $.

Согласно описанной выше схемы (идеи) доказательства, проверим выполниммость
$$\mathop {\lim }\limits_{x\to 1-0} \mu_{x} \left\{ {\lambda :\quad \mathop {\sup }\limits_t \left| {\tilde {\phi }_\lambda (t)-C(t)} 
\right|<\varepsilon } \right\}=1.$$
Для этого нужно, во-первых,
$$
\mathop {\lim }\limits_{x\to 1-0} \mathbb E_{x}\tilde {\phi }_\lambda (t)=C(t),
$$
во-вторых,
$$\mathop {\lim }\limits_{x\to 
1-0 } \mathbb E_{x } \left[ {\left( {\tilde {\phi }_\lambda (t)-C(t)} 
\right)^2} \right]=0.$$
Заметим, что в отличие от случая, когда усреднение берется по мере $\mu_n$, $n(\lambda)$ -- случайная величина, воспользоваться линейностью математического ожидания, например, при вычислении $\mathbb E_x [n(\lambda)]^{-1/2} \sum_{k\ge t\sqrt {n(\lambda)}} r_k(\lambda)$ нельзя. Преодоление этой трудности аналогично выбору перевального контура в методе Лапласа: вместо предела $x\to 1-0$ выберем последовательность $x_n$ такую, что с одной стороны $x_n \to 1-0$ при ${n\to\infty}$, с другой стороны мера $\mu_{x_n}$ концентрируется на  $\mathcal P_n$. Для этого будем оптимизировать (по $x$) вероятность того, что случайное (по мере $\mu_x$) разбиение дает фиксированное исходное значение $n$:
$$ \mu_{x} \left\{ \sum_k kr_k(\lambda) = n \right \} = p(n)\frac{x^n}{F(x)}\to\max(x), $$
так что  $x=x_n$ является корнем уравнения
$x\frac{d}{dx}\left[ {\ln 
F(x)} \right]=n$ (покажите это).
Последнее уравнение можно интерпретировать немного иначе, а именно выбирать значение $x=x_n$ из совпадения математического ожидания (по мере $\mu_x$) случайной величины $n(\lambda)$ с исходным значением $n$:
$$ \mathbb E_x \left[ {\sum\limits_k {kr_k (\lambda )} } 
\right]=n. $$
Покажите, что последнее также эквивалентно $x\frac{d}{dx}\left[ {\ln 
F(x)} \right]=n$. Это уравнение имеет единственное решение $x_n \in (0,1)$ и 
при этом $\mathop {\lim }\limits_{n\to \infty } \mathbb E_{x_n } \left[ {\left( 
{\frac{n(\lambda )}{n}-1} \right)^2} \right]=0$. Таким образом, получаем 
$$
\mathbb E_{x_n} \tilde {\phi }_\lambda (t) = \frac{1}{\sqrt n} \sum_{k\ge t\sqtr n} \mathbb E_{x_n} r_k,
$$
что является интегральной суммой для $C(t)$.

Вырожденность предельной формы обосновывается \[\mathop {\lim }\limits_{n\to 
\infty } E_{x_n } \left[ {\left( {\tilde {\phi }_\lambda (t)-C(t)} 
\right)^2} \right]=0.\]

Таким образом, обосновывается
$$\mathop {\lim }\limits_{n\to \infty} \mu _{x_n} \left\{ {\lambda :\quad 
\mathop {\sup }\limits_t \left| {\tilde {\phi }_\lambda (t)-C(t)} 
\right|<\varepsilon } \right\}=1.$$

\begin{comment}
Обозначим 
через $p(n)$ число разбиений натурального $n$ (число диаграмм Юнга, 
соответствующих натуральному $n)$. Условимся считать $p(0)=1$. 

Покажите, что производящая функция последовательности $p(n)$ равна 
$$F(x)=\sum\limits_{n=0}^\infty {p(n)x^n} =\prod\limits_{k=1}^\infty 
{(1-x^k)^{-1}}\text{ (формула Эйлера).}$$
Множество всех диаграмм Юнга, 
соответствующих натуральному числу $n$ (или иначе множество всех разбиений 
натурального числа $n)$ обозначим через $\mathcal P_n $. 

Обозначим через $\phi _\lambda (t)=\sum\limits_{k\ge t} {r_k (\lambda )} $ 
ступенчатую функцию, непрерывную справа, замыкание внутренности подграфика 
которой и есть диаграмма Юнга, соответствующая разбиению $\lambda $. Сделаем 
шкалирование (скейлинг) функции $\phi _\lambda (t)$ с множителем $a$: 
$\tilde {\phi }_\lambda (t)=\frac{a}{n}\sum\limits_{k\ge at} {r_k (\lambda 
)} =\frac{a}{n}\phi _\lambda (at)$. Заметим, что после шкалирования 
$\int\limits_0^\infty {\tilde {\phi }_\lambda (t)dt} =1$. Преобразование, 
сопоставляющее каждой диаграмме $\lambda \in \mathcal P_n $ шкалированную ее границу 
$\tilde {\phi }_\lambda (t)$, обозначим через $\tau _a $. Введем ${\cal D}$ 
{\-} пространство мер с двумя возможными атомами (``зарядами'') в $0$ и 
$\infty $ и непрерывной мерой на $(0,\infty )$, заданной плотностью $p\in 
L_+^1 ({\rm R}_+)$ (неотрицательная монотонно невозрастающая функция), то есть 
$$
{\cal D} = \{ (\alpha_0, \alpha_{\infty}, p(\cdot) ): \alpha_0, \alpha_{\infty} \ge 0, p\in 
L_+^1 ({\rm R}_+), \alpha_0 + \alpha_{\infty} + \int\limits_0^{\infty} {p(t)dt} = 1 \}.
$$
${\cal D}$ является компактом.

Заметим, что образ $\mathcal P_n$ при отображении $\tau _a $ лежит в ${\cal D}$ (это 
меры с нулевыми зарядами $\alpha _0 ,\alpha _\infty $и кусочно-постоянной 
плотностью).

Введем равномерную меру $\mu _n $ на $\mathcal P_n$, то есть $\mu _n (\lambda 
)=\frac{1}{p(n)}$ для всех $\lambda \in \mathcal P_n$.

Обозначим через $\tau _a^\ast (\mu _n )$ образ равномерной меры $\mu _n $ на 
$\mathcal P_n$ при отображении $\tau _a $, то есть $(\tau _a^\ast \mu _n )(A)=\mu _n 
(\tau _a^{-1} A)$, для любого $A\in {\cal D}$.

Следующая теорема говорит о том, что существует последовательность $a_n $, 
что последовательность образов равномерной меры на множестве диаграмм $\tau 
_{a_n }^\ast (\mu _n )$ слабо сходится к вырожденной мере с нетривиальным 
носителем: $\delta _C $, где $C\in {\cal D}$ (не является зарядом) 
называется \textit{предельной формой} (limit shape).

\textbf{Теорема.} Пусть $a_n =\sqrt n $, тогда для любого $\varepsilon >0$ 
существует $n_\varepsilon $, что для любого $n>n_\varepsilon $
\[
\mu _n \left\{ {\lambda :\quad \mathop {\sup }\limits_t \left| {\tilde {\phi 
}_\lambda (t)-C(t)} \right|<\varepsilon } \right\}>1-\varepsilon ,
\]
где $C(t)=\int\limits_t^\infty {\frac{e^{-\sqrt {\varsigma (2)} 
u}}{1-e^{-\sqrt {\varsigma (2)} u}}du} =-\frac{\sqrt 6 }{\pi }\ln 
(1-e^{-\left( {\pi \mathord{\left/ {\vphantom {\pi {\sqrt 6 }}} \right. 
\kern-\nulldelimiterspace} {\sqrt 6 }} \right)t})$.

Другими словами, типичная (в смысле равномерной меры) диаграмма Юнга после 
шкалирования $\tilde {\phi }_\lambda (t)=\frac{1}{\sqrt n }\sum\limits_{k\ge 
\sqrt n t} {r_k (\lambda )} $ с ростом $n$ имеет предельную форму, заданную 
с помощью $C(t)$ или в более симметричной форме: $e^{-\frac{\pi }{\sqrt 6 
}x}+e^{-\frac{\pi }{\sqrt 6 }y}=1$.

Доказательство этой теоремы непростое. Ограничимся обоснованием лишь 
основных идей.

Рассмотрим множество всех диаграмм Юнга $\mathcal P=\bigcup\limits_n {\mathcal P_n} $. Введем 
семейство (по $x)$ мер $\mu _x $ на множестве $\mathcal P$: 

$$\mu _x (\lambda )=\frac{x^{\sum\limits_i {\lambda _i } 
}}{F(x)}=\frac{x^{n(\lambda )}}{F(x)}, \text{ где } F(x)=\sum\limits_{n=0}^\infty 
{p(n)x^n}.$$

Идея перехода к множеству $\mathcal P$ и мерам $\mu _x $ является стандартным в 
статистической физике. Представим $n(\lambda )=\sum\limits_k {kr_k (\lambda 
)} $, где $r_k (\lambda )$ -- число слагаемых в разбиении $\lambda $, 
равных $k$. Тогда $\mathcal P_n $ -- множество всех состояний системы (Бозе-частиц) 
с постоянной энергией $n$, каждое состояние $\lambda \in \mathcal P_n $ однозначно 
характеризуется набором значений $r_k (\lambda )$ (чисел заполнений $k$-го 
уровня энергии). Заметим, что у Бозе-частиц нет ограничений на значения $r_k 
(\lambda )$, в отличие от Ферми-частиц, где $r_k (\lambda )\le 1$. $\mathcal P_n $ в 
статистической физике называется микроканоническим ансамблем (равномерная 
мера $\mu _n $ соответствует тому, что все состояния равновероятны). $\mathcal P$ 
соответствует (большому) каноническому ансамблю с мерой Гиббса $\mu _x 
(\lambda )=\frac{x^{n(\lambda )}}{F(x)}$, где $F(x)=\sum\limits_\lambda 
{x^{\sum\limits_i {\lambda _i } }} =\sum\limits_{n=0}^\infty 
{p(n)x^{n(\lambda )}} $ -- статистическая сумма. Указанная выше теорема 
дает предельное распределение энергий Бозе-частиц.

Переход от малого к большому каноническому ансамблю упрощает решение задачи 
(прежде всего в силу мультипликативности мер $\mu _x $, см. утверждение 2 
ниже).

Проверьте справедливость следующих \textbf{утверждений}:

1) $\left. {\mu _x (\lambda )} \right|_{\mathcal P_n } \equiv \frac{\mu _x (\lambda 
)}{\mu _x (\mathcal P_n)}=\mu _n (\lambda )$, то есть мера на $\mathcal P_n $, индуцированная 
мерой $\mu _x $ совпадает с равномерной мерой $\mu _n $.

2) $r_1 (\lambda ),\;r_2 (\lambda ),\ldots$ как случайные величины на $\mathcal P$ 
независимы относительно мер $\mu _x $.

Проверьте, что $\mu _x \left( {\lambda :\;r_k (\lambda 
)=s} \right)=x^{ks}(1-x^k)$.

3) Мера $\mu _x $ является выпуклой комбинацией мер $\mu _n $.

 $\mu _x =\sum\limits_{n=0}^\infty 
{\frac{x^np(n)}{F(x)}\mu _n } $.

Описанные выше утверждения, а также ``тауберовые'' аргументы позволяют 
доказать слабую эквивалентность малого и большого канонического ансамбля:
\[
\mathop {\lim }\limits_{x\to 1\!-\!0} \tau^* \mu _x =\mathop {\lim 
}\limits_{n\to \infty } \tau^* \mu _n .
\]
Таким образом, теперь доказательство теоремы сводится к обоснованию того, 
что для шкалированых диаграмм Юнга с множителем $\sqrt n $ существует 
вырожденный слабый предел (предельная конфигурация $C)$:
\[
w\!-\!\lim\limits_{x\to 1\!-\!0} \mu _x =\delta _C ,
\]
то есть $$\mathop {\lim }\limits_{x\to 1} \mu _x \left\{ {\lambda :\quad 
\mathop {\sup }\limits_t \left| {\tilde {\phi }_\lambda (t)-C(t)} 
\right|<\varepsilon } \right\}=1.$$
Для этого нужно найти средние мер $\mu _x $ на ${\cal D}$ и оценить дисперсию при $x\to 1\!-\!0$. 
Аналогично выбору перевала в методе Лапласа, выберем последовательность $x_n 
$ так, что для $x=x_n  \quad \mathbb E_x \left[ {\sum\limits_k {kr_k (\lambda )} } 
\right]=n$.
Покажите, что последнее уравнение эквивалентно $x\frac{d}{dx}\left[ {\ln 
F(x)} \right]=n$. Это уравнение имеет единственное решение $x_n \in (0,1)$ и 
при этом $\mathop {\lim }\limits_{n\to \infty } \mathbb E_{x_n } \left[ {\left( 
{\frac{n(\lambda )}{n}-1} \right)^2} \right]=1$. Здесь использовалось 
свойство мультипликативности мер $\mu _x $, другими словами, независимость 
чисел заполнения.
Осталось заметить, что $$\mathop {\lim }\limits_{n\to \infty } \mathbb E_{x_n } 
\left[ {\frac{1}{\sqrt n }\sum\limits_{k\ge t\sqrt n } {kr_k (\lambda )} } 
\right]=C(t).$$
Вырожденность предельной меры обосновывается \[\mathop {\lim }\limits_{n\to 
\infty } E_{x_n } \left[ {\left( {\tilde {\phi }_\lambda (t)-C(t)} 
\right)^2} \right]=0.\]
\end{comment}

Стоит отметить, что предельную форму диаграмм Юнга можно найти и с помощью 
следующих рассуждений.
Пусть в первом квадранте на целочисленной решетке отмечены пары точек $(x_i 
,y_i )$, так что
\[
0=x_1 <x_2 <\ldots <x_k ,\quad y_1 <y_2 <\ldots <y_k =0.
\]
Число возможных диаграмм Юнга, продолженная граница которых проходит через 
заданные пары точек, приближенно выражается 
\[
\exp \left\{ {\sum\limits_i {\left( {\Delta _i x+\Delta _i y} \right)H\left( 
{\frac{\Delta _i x}{\Delta _i x+\Delta _i y}} \right)} } \right\},
\]
где $\Delta _i x=x_{i+1} -x_i $, $\Delta _i y=y_{i+1} -y_i $, $H(p)=-p\ln 
p-(1-p)\ln (1-p)$.

Соедините указанные точки ломаной, продлив ее осями за концевые точки. 
Перейдите в новую систему координат поворотом старой против часовой стрелки 
на $45^o$ и сделав растяжение в $\sqrt 2 $ раза. Ломаная теперь задается как 
график функции $u=f(t)$. Теперь с ростом числа точек
\[
\exp \left\{ {\sum\limits_i {\left( {\Delta _i x+\Delta _i y} \right)H\left( 
{\frac{\Delta _i x}{\Delta _i x+\Delta _i y}} \right)} } \right\}\to \exp 
\left\{ {\int {H\left( {{f}'(t)} \right)dt} } \right\}.
\]
Для нахождения предельной формы нужно найти минимум функционала (с 
энтропийным лагранжианом) при ограничениях $f(t)\ge \vert t\vert $, $\vert 
{f}'\vert \le 1$, $\int {\left( {f(t)-\vert t\vert } \right)dt} =1$. Затем 
вернуться в исходную систему координат.
\end{ordre}
\end{problem}

\begin{problem}\DStar(Статистика выпуклых ломаных \cite{7})
\label{convcurve}
\imgh{50mm}{curves1.pdf}{к задаче "Статистика выпуклых ломаных" ($c=3$).}
Рассмотрим на плоскости выпуклую ломаную $\Gamma $, выходящую из нуля, у которой вершины являются целыми точками и угол наклона каждого звена неотрицателен и не превосходит ${\pi \mathord{\left/ {\vphantom {\pi  2}} \right. \kern-\nulldelimiterspace} 2} $ ($\Gamma $ -- график кусочно-линейной функции $x_2 = \Gamma(x_1)$). Выпуклость понимается в том смысле, что наклон последовательных звеньев ломаной строго возрастает. Пространство всех таких ломаных обозначим через $\Pi $, а через $\Pi (m_{1} ,m_{2} )$ -- множество ломаных, оканчивающихся в точке $(m_{1} ,m_{2} )$. 

Введем равномерную меру на пространстве $\Pi (m_{1} ,m_{2} )$:
\[P_{m_{1} ,m_{2} } (\Gamma )=\frac{1}{\left|\Pi (m_{1} ,m_{2} )\right|} =\frac{1}{N(m_{1} ,m_{2} )} .\] 

Рассмотрим детерминированную кривую $t_2 = L_{c}(t_1) $, $t_1 \in [0,1]$, заданную уравнением $\left(ct_{1} +t_{2} \right)^{2} =4ct_{2} $ (см. Рис. \ref{Fig:curves1.pdf}). Заметим, что точки $(0;0)$ и $(1;c)$ являются соответственно левым и правым концом кривой, также для этой кривой справедливо \[\left. \left({dt_{2} \mathord{\left/ {\vphantom {dt_{2}  dt_{1} }} \right. \kern-\nulldelimiterspace} dt_{1} } \right)\right|_{t_{1} =0} =0, \; \left. \left({dt_{2} \mathord{\left/ {\vphantom {dt_{2}  dt_{1} }} \right. \kern-\nulldelimiterspace} dt_{1} } \right)\right|_{t_{1} =1} =\infty .\] 

Покажите справедливость закона больших чисел для выпуклых ломаных $\Gamma\in \Pi(m_{1} ,m_{2} )$:

\textbf{Теорема (Я.Г.Синай).} Для любого $\delta > 0$ справедливо
$$
\lim_{m_{1},m_2 \to \infty, \frac{m_2}{m_1}\to c} P_{m_{1} ,m_{2} } \left \{ \max_{t_1\in[0;1]}\left | \frac{1}{m_1}\Gamma(m_1 t_1) - L_c(t_1) \right |\le \delta \right\} = 1.
$$


%Возьмем произвольное $\delta >0$ и построим полосу 
%\[U_{\delta ,m_{1} } =\left\{x=(x_{1} ,x_{2} ):\; \left|x_{2} -m_{1} L_{c} \left({x_{1} \mathord{\left/ {\vphantom {x_{1}  m_{1} }} \right. \kern-\nulldelimiterspace} m_{1} } \right)\right|\le \delta m_{1} \right\}.\] 
%Покажите, что для любого $\delta >0$ вероятность $P_{m_{1} ,m_{2} } \left\{\Gamma \subset U_{\delta ,m_{1} } \right\}$ стремится к 1 при $m_{1} \to \infty $, $m_{2} \to \infty $, $\frac{m_2}{m_1}\to c$. 

Иными словами, в результате масштабного преобразования (скейлинга) $t_{1} ={x_{1} \mathord{\left/ {\vphantom {x_{1}  m_{1} }} \right. \kern-\nulldelimiterspace} m_{1} } $ и $t_{2} ={x_{2} \mathord{\left/ {\vphantom {x_{2}  m_{1} }} \right. \kern-\nulldelimiterspace} m_{1} } $ форма случайных ломаных становится при $m_{1} \to \infty $ детерминированной.

\end{problem}

\begin{ordre} 

Согласно работе Я.Г.Синай "Вероятностный подход к анализу статистики выпуклых ломаных". Функциональный анализ и его приложения. -- 1994. -- Т. 28, Вып. 2. -- С. 41-48. (либо см. книгу \cite{7}) рассмотрим подход к исследованию асимптотических вероятностных свойств выпуклых ломаных, основанный на понятиях статистической физики -- микроканонического и большого канонического ансамблей. Согласно такой точки зрения, распределение вероятностей $P_{m_{1} ,m_{2}}$ на $\Pi(m_{1} ,m_{2})$ вводится как условное распределение, индуцированное подходящей мерой $Q$ (см. ниже), заданной на
пространстве $\Pi$. При этом те или иные свойства ломаных (в частности, закон
больших чисел) устанавливаются сначала по отношению к $Q$, а затем переносятся
на случай $P_{m_{1} ,m_{2}}$  с помощью соответствующей локальной предельной теоремы.

Сначала рассмотрим множество $X$ всех пар взаимно простых целых чисел, и пусть $C_{0} (X)$ -- пространство финитных функций на $X$  с неотрицательными целыми значениями. Нетрудно понять, что каждой такой функции естественным образом отвечает некоторая выпуклая (конечнозвенная) ломаная, и наоборот. Введем на $C_{0} (X)$ мультипликативную статистику $Q = Q_{z_1,z_2}$ (параметры $z_1$ и $z_2$, $0<z_1,z_2<1$, определим ниже) -- распределение случайного поля $\nu = \nu(\cdot)$  на $X$ с независимыми значениями при разных $x = (x_1, x_2)\in X$  и распределением:
\[Q_{z_{1} ,z_{2} } (\nu )=\prod _{x=(x_{1} ,x_{2} )\in X}\left[\left(z_{1}^{x_{1} } z_{2}^{x_{2} } \right)^{\nu (x)} \left(1-z_{1}^{x_{1} } z_{2}^{x_{2} } \right)\right] =\] 

\[\prod _{x=(x_{1} ,x_{2} )\in X}\left(z_{1}^{x_{1} } z_{2}^{x_{2} } \right)^{\nu (x)}  \prod _{x=(x_{1} ,x_{2} )\in X}\left(1-z_{1}^{x_{1} } z_{2}^{x_{2} } \right) .\] 
То есть каждая случайная величина $\nu (x)$ имеет геометрическое распределение с параметрами $z_{1}^{x_{1} } z_{2}^{x_{2} } $.

Проверьте, что на пространстве $P_{m_{1} ,m_{2}}$ введенная выше мера $Q_{z_1,z_2}$ индуцирует распределение, не зависящее от параметров $z_1$ и $z_2$, в данном случае равномерное:
\[Q_{z_{1} ,z_{2} } \left(\nu |\Pi (m_{1} ,m_{2} )\right)=\frac{1}{N(m_{1} ,m_{2} )} =P_{m_{1} ,m_{2} } (\nu ),\]

С точки зрения статистической физики, распределение $Q_{z_1,z_2}$ играет роль большого канонического распределения Гиббса, а распределение $P_{m_{1} ,m_{2}}$ -- микроканонического распределения. Если какое-либо событие по отношению к распределению $Q_{z_{1} ,z_{2} } $ имеет малую вероятность, гораздо меньшую, чем вероятность $Q_{z_{1} ,z_{2} } \left(\Pi (m_{1} ,m_{2} )\right)$, то оно имеет и малую вероятность по отношению к распределению $P_{m_{1} ,m_{2} } $. Основная идея заключается в том, чтоб подобрать такие значения параметров $z_{1} ,z_{2} \in \left[0,1\right]$, при которых вероятность $Q_{z_{1} ,z_{2} } \left(\Pi (m_{1} ,m_{2} )\right)$ приняла бы возможно большее значение, то есть чтобы распределение $Q_{z_{1} ,z_{2} }$ концентрировалось на пространстве $\Pi(m_{1} ,m_{2})$:
$$Q_{z_{1} ,z_{2} } \left(\Pi (m_{1} ,m_{2} )\right) = Q_{z_{1} ,z_{2} } \left \{ \sum _{x\in X}\nu (x)x_{1} = m_1, \sum _{x\in X}\nu (x)x_{2} = m_2 \right \} =$$
$$\frac{N(m_1,m_2)z_1^{m_1}z_2^{m_2}}{\prod_{x=(x_{1} ,x_{2} )\in X}(1-z_{1}^{x_{1} } z_{2}^{x_{2} })^{-1}}\to\max(z_{1} ,z_{2} \in \left[0,1\right]).$$
Покажите, что такой выбор согласуется с интуитивным выбором из фиксации "в среднем" по мере $Q_{z_{1} ,z_{2} }$ правого конца случайной ломаной:
\[\begin{array}{l} {\Exp_{z_{1} ,z_{2} } \left(\sum _{x\in X}\nu (x)x_{1}  \right)=\sum _{x\in X}\frac{x_{1}z_1^{x_1}z_2^{x_2}}{1-z_1^{x_1}z_2^{x_2}}=m_{1} ,} \\ {\Exp_{z_{1} ,z_{2} } \left(\sum _{x\in X}\nu (x)x_{2}  \right)=\sum _{x\in X}\frac{x_{2}z_1^{x_1}z_2^{x_2}}{1-z_1^{x_1}z_2^{x_2}}=m_{2}, } \end{array}\] 
\noindent 
что в свою очередь эквивалентно системе уравнений (см. предыдущую задачу):
\[\begin{array}{l} {z_1 \frac{d}{dz_1} \ln \left (\prod_{x=(x_{1} ,x_{2} )\in X}}(1-z_{1}^{x_{1} } z_{2}^{x_{2} })^{-1}\right ) =m_{1} ,} \\ {\z_2 \frac{d}{dz_2} \ln \left ( \prod_{x=(x_{1} ,x_{2} )\in X}}(1-z_{1}^{x_{1} } z_{2}^{x_{2} })^{-1} \right)=m_{2}. } \end{array}\] 

Детали нахождения $z_1$ и $z_2$ см. в приведенной выше литературе. Здесь приводится только результат, что $z_{1}$ и $z_{2} $ равны соответственно 
\[1-\left(\frac{\zeta (3)c}{\zeta (2)m_{1} } \right)^{\frac{1}{3}} (1+o(1)), \; \; 1-\left(\frac{\zeta (3)}{\zeta (2)c^{2} m_{1} } \right)^{\frac{1}{3}}  (1+o(1)).\] 
Дзета-функция Римана появляется в связи с тем, что рассматриваются только пары взаимно простых чисел -- пространство $X$ (см. задачу \ref{sec:z_func_riman} из раздела \ref{hard}).


\begin{comment}
\textbf{Лемма.} Пространство $\Pi $ находится во взаимно однозначном соответствии с пространством $C_{0} (X)$ финитных функций $\nu (x)$, заданных на $X$ - множестве пар взаимно простых положительных чисел, включая пары $(0,1)$, $(1,0)$, и принимающих целые неотрицательные значения. 

Введем на пространстве $\Pi $ (с точки зрения статистической механики представляющего большой канонический ансамбль) распределение вероятностей $Q_{z_{1} ,z_{2} } $, зависящее от двух параметров $z_{1} ,z_{2} \in \left[0;1\right]$, и играющее роль большого канонического распределения Гиббса. Или в силу леммы зададим $Q_{z_{1} ,z_{2} } $ на пространстве функций $C_{0} (X)$:

\[Q_{z_{1} ,z_{2} } (\nu )=\prod _{x=(x_{1} ,x_{2} )\in X}\left[\left(z_{1}^{x_{1} } z_{2}^{x_{2} } \right)^{\nu (x)} \left(1-z_{1}^{x_{1} } z_{2}^{x_{2} } \right)\right] =\] 

\[\prod _{x=(x_{1} ,x_{2} )\in X}\left(z_{1}^{x_{1} } z_{2}^{x_{2} } \right)^{\nu (x)}  \prod _{x=(x_{1} ,x_{2} )\in X}\left(1-z_{1}^{x_{1} } z_{2}^{x_{2} } \right) .\] 

Так каждая случайная величина $\nu (x)$ имеет по отношению к распределению $Q_{z_{1} ,z_{2} } $ геометрическое распределение с параметрами $z_{1}^{x_{1} } z_{2}^{x_{2} } $ и при разных $x$ случайные величины $\nu (x)$ взаимно независимы. Несложно вывести, что

\[Q_{z_{1} ,z_{2} } \left(\Pi (m_{1} ,m_{2} )\right)=\sum _{}Q_{z_{1} ,z_{2} } (\nu ) =N(m_{1} ,m_{2} )z_{1}^{m_{1} } z_{2}^{m_{2} } \prod _{x\in X}\left(1-z_{1}^{x_{1} } z_{2}^{x_{2} } \right) ,\] 
\noindent где суммирование ведется по тем $\nu $, для которых $\sum _{x\in X}\nu (x)x_{1}  =m_{1} $ и $\sum _{x\in X}\nu (x)x_{2}  =m_{2} $.

При этом

\[Q_{z_{1} ,z_{2} } \left(\nu |\Pi (m_{1} ,m_{2} )\right)=\frac{1}{N(m_{1} ,m_{2} )} =P_{m_{1} ,m_{2} } (\nu ),\] 
\noindent то есть равномерное распределение по отношению к распределению $Q_{z_{1} ,z_{2} } $ является микроканоническим распределением. Отсюда будет следовать, что если какое-либо событие по отношению к распределению $Q_{z_{1} ,z_{2} } $ имеет малую вероятность, гораздо меньшую, чем вероятность $Q_{z_{1} ,z_{2} } \left(\Pi (m_{1} ,m_{2} )\right)$, то оно имеет и малую вероятность по отношению к распределению $P_{m_{1} ,m_{2} } $. Основная идея заключается в том, чтоб подобрать такие значения параметров $z_{1} ,z_{2} \in \left[0;1\right]$, при которых вероятность $Q_{z_{1} ,z_{2} } \left(\Pi (m_{1} ,m_{2} )\right)$ приняла бы возможно большее значение. Для этого выберем параметры, удовлетворяющие уравнениям:

\[\begin{array}{l} {\Exp_{z_{1} ,z_{2} } \left(\sum _{x\in X}\nu (x)x_{1}  \right)=m_{1} ,} \\ {\Exp_{z_{1} ,z_{2} } \left(\sum _{x\in X}\nu (x)x_{2}  \right)=m_{2} .} \end{array}\] 

В результате алгебраических преобразований получаем, что при $m_{2} =cm_{1} $ значения $z_{1} ,z_{2} $ равны соответственно 

\[1-\left(\frac{\zeta (3)c}{\zeta (2)m_{1} } \right)^{\frac{1}{3}} (1+o(1)), \; \; 1-\left(\frac{\zeta (3)}{\zeta (2)c^{2} m_{1} } \right)^{\frac{1}{3}}  (1+o(1)).\] 

Дзета-функция Римана появляется в связи с тем, что рассматриваются только пары взаимно простых чисел -- пространство $X$ (см. задачу \ref{sec:z_func_riman} из раздела \ref{hard}).
\end{comment}

Зафиксируем последовательность $0<\tau _{1} <\tau _{2} <\cdots <\tau _{N} $. Впоследствии $\tau _{1} \to 0$, $\tau _{N} \to \infty $, $\mathop{\max }\limits_{j} (\tau _{j+1} -\tau _{j} )\to 0$. Введем случайные величины

\[\zeta _{k}^{(j)} (\nu )=\sum _{x\in X:\; \tau _{j} \le {\frac{x_2}{x_1}} \le \tau _{j+1} }x_{k} \nu (x) ,\quad k=1,2.\] 

\noindent $\zeta _{1}^{(j)} (\nu )$ и $\zeta _{2}^{(j)} (\nu )$ есть приращение по осям $x_{1} $, $x_{2} $ соответственно той части ломаной, где тангенс угла наклона звеньев заключен между $\tau _{j} $ и $\tau _{j+1} $. По отношению к распределению вероятностей $Q_{z_{1} ,z_{2} } $ случайные величины $\zeta _{k}^{(j)} (\nu )$ взаимно независимы при разных $j$. Их математические ожидания:

\[\Exp_{z_{1} ,z_{2} } \zeta _{k}^{(j)} (\nu )=\sum _{\tau _{j} \le {\frac{x_2}{x_1}} \le \tau _{j+1} }x_{k} \frac{z_{1}^{x_{1} } z_{2}^{x_{2} } }{1-z_{1}^{x_{1} } z_{2}^{x_{2} } }  .\] 

С учетом полученных выше параметров $z_{1} ,z_{2} $ это соотношение можно переписать в дифференциальном виде, решением которого и является детерминированная кривая (см. рис.)

\[\left(ct_{1} +t_{2} \right)^{2} =4ct_{2} .\] 

Далее нужно воспользоваться законом больших чисел по отношению к распределениям $Q_{z_{1} ,z_{2} } $ и $P_{m_{1} ,m_{2} } $.

\end{ordre}
\begin{remark}
Помимо уже приведенной выше литературы см. статью А.М.Вершик ``Предельная форма выпуклых целочисленных многоугольников и близкие вопросы'', Функциональный анализ и его приложения. -- 1994. -- Т. 28, Вып.1. -- С. 16–25.
\end{remark}
\begin{comment}

\begin{problem}
\label{sub_exp}
Характеристические функции с.в. $\xi_k$ ограничены согласно выражениям
\[
\log \Exp e^{\lambda \xi_k} \leq \frac{q_k^2 + \lambda^2}{2}, 
\quad |\lambda| < g.
\]
Покажите, что для суммы $S = \sum_k c_k \xi_k$ при условиях $\sum_k c_k = 1$, $\sum_k e^{-q_k} \leq 1$  выполнены следующие утверждения:

\begin{enumerate}
\item если $g = \infty$, то
\[
\log \Exp e^{S} \leq H_1 = \sum_k c_k q_k,
\]
\[
\forall x \geq \frac{1}{2}:  \; \PR(S \geq H_1 + \sqrt{2x}) \leq e^{-x}.
\]

\item если $g < \infty$, то $\forall |\lambda| < g$
\[
\log \Exp e^{\lambda S} \leq \frac{H_2 + \lambda^2}{2}, 
\quad H_2 = \sum_k c_k q_k^2,
\]
\[
\forall x \geq \frac{1}{2}:  \; \PR(S \geq H(x)) \leq e^{-x}, 
\quad H(x) = H_1 + \sqrt{2x} + \frac{g^{-2} x + 1}{g} H_2.  
\]
\item если $g^2 \geq H_2 + 1$, то
\[
\Exp S \leq H_1 + \frac{H_2}{g} + 3,
\quad
\Exp S^2 \leq \left(H_1 + \frac{H_2}{g} + 4 \right)^2.
\]

\end{enumerate}

\end{problem}

\begin{ordre}
\begin{enumerate}
\item Воспользуйтесь неравенством Гельдера в виде 
\[
\log \Exp e^{\sum_k a_k \zeta_k} \leq \sum_k a_k \log \Exp e^{\zeta_k},
\]  
а также соотношением
\[
\PR \left(\sum_k \zeta_k \geq 0 \right) \leq \sum_k \PR (\zeta_k \geq 0).
\]
\item Введем вспомогательную переменную
\[
z_k(\lambda_k) = 
\begin{cases}
\frac{x+q_k}{g} + \frac{g}{2} + \frac{q^2_k}{2g}, \quad \lambda_k \geq g, \\
q_k + \sqrt{2x}, \quad \lambda_k < g.
\end{cases}
\]
Убедитесь, что 
\[
\PR(\xi_k \geq z_k) \leq e^{-x}, 
\quad
\sum_k c_k z_k \leq H(x).
\]
\item Примените результат задачи \ref{mom_ineq} из раздела \ref{standart}.
\end{enumerate}
\end{ordre}

\end{comment}





