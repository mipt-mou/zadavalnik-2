%\subsection{Закон больших чисел}

\begin{problem}
Пусть случайная величина $X_n$ принимает значения 
\begin{enumerate}
\item $(2^n, -2^n)$ с вероятностями $(1/2, 1/2)$;
\item $(-\sqrt{n}$, $\sqrt{n})$ с вероятностями $(1/2, 1/2)$;
\item $(n, 0, -n)$ с вероятностями $(1/4, 1/2, 1/4)$.
\end{enumerate}
Выполняется ли для последовательности независимых случайных величин 
$X_1$, $X_2$, $\ldots$ \textit{закон больших чисел}? 
\end{problem}

\begin{ordre}
 
\[
S_n\xrightarrow{p}0 \,\Leftrightarrow\, S_n\xrightarrow{d}0 \,\Leftrightarrow\, \forall t \; \varphi_{S_n}(t) \to 1,   
\]
\noindent где $S_n=\frac{X_1+\ldots +X_n}{n}$, $\varphi_{S_n}(t)$ -- характеристическая функция. В общем случае соответствие между сходимостью функций распределения $F_n(x)$ и сходимостью соответствующих $\varphi_n(t)$ такое:
\begin{enumerate}
\item Если $F_n(x) \to F(x)$ в точках непрерывности $F(x)$ и $F(x)$ удовлетворяет свойствам функции распределения, то $\forall t \; \varphi_n(t) \to \varphi(t)$, где $\varphi(t)$ х.ф. $F(x)$.
\item Если $\forall t \; \varphi_n(t) \to \varphi(t)$ и $\varphi(t)$ непрерывна в точке $t = 0$, то $\varphi(t)$ является х.ф. некоторого распределения $F(x)$  и $F_n(x) \to F(x)$ в точках непрерывности $F(x)$.

\end{enumerate}

\end{ordre}


\begin{problem}
При каких значениях $\alpha > 0$ к последовательности независимых случайных величин $\{ X_n\}_{n=1}^{\infty}$, 
таких что ${\mathbb P}\{ X_n=n^{\alpha}\}={\mathbb P}\{ X_n=-n^{\alpha}\}=1/2$, применим закон больших чисел? 
\end{problem}

\begin{comment}
\begin{ordre}
Докажите достаточное условие выполнения ЗБЧ:
 \[
Var S_n \xrightarrow {n\to\infty}0
\] 
\end{ordre}
\end{comment}

\begin{problem}
Пусть $\{ X_n\}_{n=1}^{\infty}$ --- последовательность случайных величин с дисперсиями $\sigma_i^2$. Доказать, что если все 
корреляционные моменты (корреляции) $R_{ij}$ случайных величин $X_i$ и $X_j$ неположительны и при  
$\frac{1}{n^2}\sum\limits_{i=1}^{n} \sigma_i^2\to 0$, $n\to\infty$, то для последовательности $\{ X_n\}_{n=1}^{\infty}$ выполняется закон больших чисел. 
\end{problem}

\begin{problem}
Пусть $\{ X_n\}_{n=1}^{\infty}$ --- последовательность случайных величин с равномерно ограниченными дисперсиями, причем каждая 
случайная величина $X_n$ зависит только от $X_{n-1}$ и $X_{n+1}$, но не зависит от остальных $X_i$. Доказать выполнение для этой 
последовательности закона больших чисел.
\end{problem}

\begin{problem} Следует ли из у.з.б.ч. для схемы испытаний Бернулли, что с 
вероятностью 1 по любым сколь угодно маленьким $\varepsilon >0$ и $\delta 
>0$ можно подобрать такое $n\left( {\varepsilon ,\delta } \right)\in {\mathbb N}$, что с вероятностью не меньшей $1-\delta $ для всех $n\ge n\left( 
{\varepsilon ,\delta } \right)$ частота выпадения герба $\nu _n $ отличается 
от вероятности выпадения герба $p$ не больше чем на $\varepsilon $? Следует 
ли это из з.б.ч.?
\end{problem}

\begin{problem} Привести пример последовательности независимых с.в. 
$\left\{ {\xi _n } \right\}_{n\in {\rm N}} $ таких, что предел $\mathop 
{\lim }\limits_{n\to \infty } \frac{\xi _1 +...+\xi _n }{n}$ существует по 
вероятности, но не существует с вероятностью 1.
\end{problem}

\begin{remark} 
См. Стоянов Й. \cite{stoianov}.
\end{remark}

\begin{problem}\Star(Лоэв--Колмогоров) Пусть $\left\{ {\xi _n } \right\}_{n\in 
{\rm N}} $ -- последовательность таких независимых с.в., что 
$\sum\limits_{n=1}^\infty {\frac{\Exp\left| {\xi _n } \right|^{\alpha _n 
}}{n^{\alpha _n }}} <\infty $, где $0<\alpha _n \le 2$, причем $\Exp\xi _n =0$ 
в случае $1\le \alpha _n \le 2$. Покажите, что тогда 
$\frac{1}{n}\sum\limits_{k=1}^n {\xi_k } \buildrel \text{п.н.} \over 
\longrightarrow 0$.
\end{problem}

\begin{problem}\Star(Теорема Колмогорова о трех рядах) 
Пусть $\left\{ {\xi _n } 
\right\}_{n\in {\rm N}} $ -- последовательность таких независимых с.в. 
Покажите, что для сходимости ряда $\sum\limits_{n=1}^\infty {\xi _n } $ с 
вероятностью 1 необходимо, чтобы для любого $c>0$ сходились ряды 
$\sum\limits_{n=1}^\infty {\Exp\xi _n^c } $, $\sum\limits_{n=1}^\infty {\Var\xi 
_n^c } $, $\sum\limits_{n=1}^\infty {\PR\left( {\left| {\xi _n } \right|\ge c} 
\right)} $, где $\xi _n^c =\xi _n I\left( {\xi _n \le c} \right)$, и 
достаточно, чтобы эти ряды сходились при некотором $c>0$.
\end{problem}



\begin{problem} Доказать \textit{локальную теорему Муавра--Лапласа} в общем случае:
\[
C_n^k p^k\left( {1-p} \right)^{n-k}\sim \left( {2\pi np\left( {1-p} \right)} 
\right)^{-1 \mathord{\left/ {\vphantom {1 2}} \right. 
\kern-\nulldelimiterspace} 2}\exp \left( {-\frac{1}{2}\frac{\left( {k-np} 
\right)^2}{np\left( {1-p} \right)}} \right)
\]
равномерно по таким $k$, что $\left| {k-np} \right|\le \varepsilon \left( n 
\right)$, $\varepsilon \left( n \right)=o\left( {n^{2 \mathord{\left/ 
{\vphantom {2 3}} \right. \kern-\nulldelimiterspace} 3}} \right)$, $n\to 
\infty $, $p$ -- фиксировано.
\end{problem}

\begin{ordre}
См. книгу Гнеденко \cite{2}. 
\end{ordre}

\begin{problem}
Докажите, что для с.в. $X_i \in \text{Be}(p)$, $p > 0$ выполнена локальная теорема Муавра--Лапласа
\[
\PR\left(\sum_{i = 1}^n X_i = k \right) \to 
\left( {2\pi np\left( {1-p} \right)} 
\right)^{-1 \mathord{\left/ {\vphantom {1 2}} \right. 
\kern-\nulldelimiterspace} 2}\exp \left( {-\frac{1}{2}\frac{\left( {k-np} 
\right)^2}{np\left( {1-p} \right)}} \right).
\]
Если же $X_1,\ldots,X_n \in \text{Be}(1/n)$, то $\sum_{i = 1}^n X_i \mathop{\longrightarrow}\limits^{d} \text{Po}(1)$. Будет ли иметь место сходимость по вероятности  $\sum_{i = 1}^n X_i \mathop{\longrightarrow}\limits^{p} \text{Po}(1)$?

\end{problem}

\begin{problem}
Рассмотрим машину, называемую ``доска Гальтона'' (см. Рис.\ref{Fig:gam_scheme.JPG}).
\imgh{30mm}{gam_scheme.JPG}{Доска Гальтона}    
Принцип работы этого устройства таков: металлические шарики поступают в самый верхний канал. Наткнувшись на первое острие, они «выбирают» путь направо или налево. Затем происходит второй такой выбор и т.д. При хорошей подгонке деталей выбор оказывается случайным. Как видно, попадание шариков в нижние лунки не равновероятно. В этом случае мы имеем дело с нормальным распределением. Объясните почему?
\end{problem}

\begin{problem}
Книга объемом $500$ страниц содержит $50$ опечаток. Оценить вероятность того, что на случайно выбранной странице 
имеется не менее трех опечаток, используя нормальное и пуассоновское приближения, сравнить результаты. 
\end{problem}

\begin{problem}
В тесто для выпечки булок с изюмом замешано $N$ изюмин. Всего из данного теста выпечено $K$ булок. Оценить вероятность того, 
что в случайно выбранной булке число изюмин находится в пределах от $a$ до $b$. 
\end{problem}

\begin{problem}
В поселке $N$ жителей, каждый из которых в среднем $n$ раз в месяц ездит в город, выбирая дни поездки независимо от остальных. 
Поезд из поселка в город идет один раз в сутки. Какова должна быть вместимость поезда для того, чтобы он переполнился с вероятностью, 
не превышающей заданного числа $\beta$? 
\end{problem}

\begin{problem}
Случайные величины $X_1,\ldots,X_n$ независимы и имеют распределение Коши, т.е. 
\[
f(x) = \frac{d}{\pi(d^2 + x^2)}.
\]
Докажите равенство распределений $(X_1+\ldots+X_n)/n$ и  $X_1$. Противоречит ли данное равенство закону больших чисел или ЦПТ?
\end{problem}


