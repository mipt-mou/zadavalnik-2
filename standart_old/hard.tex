
\section{Задачи повышенной сложности}
\label{hard}

\begin{problem}
Докажите, что при $n\to\infty$ 
$$
X_n\xrightarrow{L_2} X \,\Rightarrow\, X_n\xrightarrow{L_1}X \, \Rightarrow\, X_n\xrightarrow{p}X 
\, \Leftarrow\, X_n\xrightarrow{\text{ п.н. }}X , 
$$
$$
X_n\xrightarrow{p}X \, \Rightarrow\, X_n\xrightarrow{d}X . 
$$
С помощью контрпримеров покажите, что никакие другие стрелки импликации в эту схему в общем случае добавить нельзя. 
При каких дополнительных условиях можно утверждать, что 
$$
X_n\xrightarrow{\text{ п.н. }}X  \, \Rightarrow\, X_n\xrightarrow{L_1}X ?
$$
Кроме того, показать, что 
$$
X_n\xrightarrow{p} X \; (n\to\infty) \,\Leftrightarrow\, \rho_P(X_n,X)={\mathbb E}\Bigl( \frac{|X_n-X|}{1+|X_n-X|}\Bigr)
\xrightarrow{} 0 , 
$$
то есть сходимость по вероятности метризуема.
\end{problem}

\begin{remark} (См. А.Н. Ширяев [\ref{chiraiev}], Т. 1, а также главу 6 \cite{Gupta})
Выполнена импликация 
$$
X_n\xrightarrow{\text{ п.н. }}X  \, \Rightarrow\, X_n\xrightarrow{L_1}X , 
$$
т.е. возможен предельный переход под знаком математического ожидания, если семейство с.в. $\{ X_n\}$ является равномерно интегрируемым: 
$$
\sup\limits_n {\mathbb E}\bigl[ |X_n|\cdot {\mathbb I}_{\{ |X_n|>c\}} \bigr] \rightarrow 0, \quad c \to +\infty; 
$$

Отметим также, что сходимость по распределению также как и по вероятности метризуемы в отличие от сходимости п.н. (см. задачу \ref{limpnnero} из  раздела \ref{standart}). 

\end{remark}

\begin{problem}[Интеграл Лебега-Стилтьеса]
Для неотрицательной случайной величины $\xi$ определим ее математическое $\Exp \xi$ ожидание посредством интеграла Лебега $\int \xi(\omega) d\PR(\omega)$, определяемого как 
\[
\lim \limits_{n \to \infty} \left(
  \sum \limits_{k = 1}^{n 2^n} \frac{k-1}{2^n} \PR\left(  \frac{k-1}{2^n} \leq \xi <  \frac{k}{2^n}  \right) + n \PR(\xi \geq n). 
\right)
\] 

Пусть $\xi: \Omega \to \mathbb{R}$ -- произвольная случайная величина и $\xi^{+} = \max (\xi, 0)$, $\xi^{-} = - \min (\xi, 0)$. Тогда при условии $\Exp \xi^{+} < \infty$ или $\Exp \xi^{-} < \infty$ оператор $\Exp$ определяется как 
$\Exp \xi = \Exp \xi^{+} - \Exp \xi^{-}$. 

Докажите следующие теоремы и вытекающие из них свойства:

\begin{enumerate}
\item Для случайной величины $\xi \geq 0$ существует последовательность \textit{простых} (индикатор измеримого множества) случайных величин, таких что $\xi_{n+1} \geq \xi_n$ и $\xi_n \mathop{\longrightarrow} \limits^{\text{п.н.}} \xi$. Следствие: $\forall \xi \geq 0 \;\; \exists \; \Exp \xi$.
\item Если  $\xi_{n+1} \geq \xi_n \geq 0$ и $\xi_n \mathop{\longrightarrow} \limits^{\text{п.н.}} \xi$, то $\Exp \xi_n \to \Exp \xi$. 
\begin{enumerate}
\item[Следствие 1:] Путь $\xi_n \geq 0$, тогда 
\[ \Exp\left(\sum \limits_{n=1}^{\infty} \xi_n \right) = \sum \limits_{n=1}^{\infty} \Exp \xi_n. \]
\item[Следствие 2:] Пусть $\Exp \vert \xi \vert < \infty$ и $\PR(A) \to 0$, тогда $\Exp (\vert \xi \vert I_A) \to 0$. 
\end{enumerate}
\item Путь $\xi_n \geq 0$, тогда $\Exp (\lim \limits_{n\to\infty} \inf \xi_n) \leq \lim \limits_{n\to\infty} \inf \Exp \xi_n$. 
\item Путь $\vert \xi_n \vert \leq Y$ и $\Exp Y < \infty$, тогда $\Exp (\lim \limits_{n\to\infty} \inf \xi_n) \leq \lim \limits_{n\to\infty} \inf \Exp \xi_n  \leq \lim \limits_{n\to\infty} \sup \Exp \xi_n  \leq \Exp (\lim \limits_{n\to\infty} \sup \xi_n)$.
\begin{enumerate}
\item[Следствие 1:] Путь $\xi_n \geq 0$, $\vert \xi_n \vert \leq Y$, $\Exp Y < \infty$ и $\xi_n \mathop{\longrightarrow} \limits^{\text{п.н.}} \xi$, тогда $\Exp \vert \xi \vert < \infty$, $\Exp \xi_n \to \Exp \xi$ и $\Exp \vert \xi_n - \xi \vert \to 0$.
\item[Следствие 2:] Путь $\vert \xi_n \vert \leq Y$, $\Exp Y^p < \infty$, $p >0$ и $\xi_n \mathop{\longrightarrow} \limits^{\text{п.н.}} \xi$, тогда $\Exp \vert \xi \vert^p < \infty$ и $\Exp \vert \xi_n - \xi \vert^p \to 0$. 
\end{enumerate}
\end{enumerate} 

\end{problem}

\begin{remark}
Положим $g(x): \mathbb{R} \to \Upsilon$ -- борелевская функция ($\forall A \in \mathcal{B}(\Upsilon): \;  g^{-1}(A) \in \mathcal{B}(\mathbb{R}) $). Если существует один из интегралов 
\[
\int \limits_{A} g(x) dF_{\xi}(x) \quad \text{или} \quad \int \limits_{\xi^{-1}(A)} g(\xi(\omega)) d\PR(\omega),
\]
то существует и другой, и они совпадают. Интеграл  $\int \limits_{\mathbb{R}} g(x) dF_{\xi}(x)$ называется интегралом \textit{Лебега--Стилтьеса}.
\end{remark}

\begin{problem}
Из $2^N$ множеств: совокупности всех подмножеств множества $\{1,2,\ldots,N\}$ случайно и независимо выбираются 2 множества $A$ и $B$. Найти вероятность, что $A$ и $B$ не пересекаются.
\end{problem}

\begin{problem} [Н.Н. Константинов]
В самолете $n$ мест. Есть $n$ пассажиров, выстроившихся друг за другом в очередь. Во главе очереди -- <<заяц>>. У всех, 
кроме <<зайца>>, есть билет, на котором указан номер посадочного места. Так как <<заяц>> входит первым, он случайным образом занимает 
некоторое место. Каждый следующий пассажир, входящий в салон самолета, действует по такому принципу: если его место свободно, то 
садится на него, если занято, то занимает с равной вероятностью любое свободное. Найдите вероятность того, что последний пассажир 
сядет на свое место. 
\end{problem}


\begin{problem}
Симметричную монету независимо бросили $n$ раз. Результат бросания записали в виде последовательности нулей и единиц. 
\begin{enumerate}
\item Покажите, что с вероятностью стремящейся к единице при $n\to \infty $ длина максимальной подпоследовательности  из подряд идущих единиц ($l_n$) лежит в промежутке $(\log_2 \sqrt{n} ,\; \log_2 n^{2} )$; 

\item$^{*}$ Покажите, что серия из гербов длины $\log_{2} n$ наблюдается с вероятностью, стремящейся к единице при $n\to\infty$;

\item$^{**}$ Верно ли, что  $\frac{ l_n}{ \log_2 n } \overset{\text{п.н.}}{\longrightarrow} 1$?
\end{enumerate}
\end{problem}

 \begin{ordre}
 В пункте а) для нахождения нижней оценки разобьем всю последовательность бросаний на участки длиной $\log \sqrt{n}$. Оцените вероятность того, что хотя бы один из участков состоит полностью из единиц.    
 \end{ordre}

\begin{remark}
 См. Erdos P., Renyi A. On a new law of large numbers // Journal Analyse Mathematique. 1970. V. 23. P. 103--111.
 
 \verb| http://www.renyi.hu/|$\sim$\verb|p_erdos/1970-07.pdf.| 
\end{remark}


\begin{problem}[Закон 0 и 1]
Пусть $(\Omega,\Xi,{\mathbb P})$ --- вероятностное пространство, $\xi_1,\xi_2,\ldots$ --- некоторая последовательность независимых с.в. 
Обозначим $\Xi_n^{\infty}=\sigma(\xi_{n},\xi_{n+1},\ldots)$ --- $\sigma$-алгебру, порожденную с.в. $\xi_{n},\xi_{n+1},\ldots$ и пусть 
$$
{\mathcal X}=\bigcap\limits_{n=1}^{\infty} \Xi_{n}^{\infty} . 
$$
Поскольку пересечение $\sigma$-алгебр есть снова $\sigma$-алгебра, то ${\mathcal X}$ --- есть $\sigma$-алгебра. Эту $\sigma$-алгебру 
будем называть <<хвостовой>> или <<остаточной>>, в связи с тем, что всякое событие $A\in{\mathcal X}$ не зависит от значений с.в. 
$\xi_1,\xi_2,\ldots,\xi_n$ при любом конечном $n$, а определяется лишь <<поведением бесконечно далеких значений последовательности 
$\xi_1,\xi_2,\ldots$ >>. 

С помощью задачи $\ref{SigmaAlgebra}$ раздела \ref{standart} докажите справедливость следующего утверждения: 

Пусть $\xi_1,\xi_2,\ldots$ --- последовательность независимых в совокупности с.в. и $A\in{\mathcal X}$ 
(событие $A$ принадлежит <<хвостовой>> $\sigma$-алгебре). Тогда ${\mathbb P}(A)$ может принимать лишь два значения $0$ или $1$. 
\end{problem}

\begin{ordre}
Идея доказательства состоит в том, чтобы показать, что каждое <<хвостовое>> событие $A$ не зависит от самого себя и, значит, 
${\mathbb P}(A\cap A)={\mathbb P}(A)\cdot {\mathbb P}(A)$, т.е. ${\mathbb P}(A)={\mathbb P}^2(A)$, откуда 
${\mathbb P}(A)=0$ или $1$. 

Полученный результат, в частности, означает, что $\sum  \xi_k$ сходится или расходится с вероятностью 1.  

Существует также обобщение последнего правила -- закон Хьюитта и Сэвиджа (см. Ширяев Т 2 \cite{21}), где под ${\mathcal X}$ подразумевается $\sigma$-алгебра перестановочных событий:
\[
\mathcal{X} = \{ \xi^{-1}(B), \; B \in \mathcal{B}(\mathbb{R}^{\infty}): \;
\PR (\xi^{-1}(B) \triangle (\pi\xi)^{-1}(B)) = 0, \; \forall \pi \}, 
\] 
где $\pi$ -- перестановка конечного числа элементов, $\triangle$ -- оператор симметрической разности множеств.
\end{ordre}

\begin{problem}
Сто паровозов выехали из города по однополосной линии, каждый с постоянной скоростью. Когда движение установилось, то из-за того, что быстрые догнали идущих впереди более медленных, образовались караваны (группы, движущиеся со скоростью лидера). Найдите математическое ожидание и дисперсию числа караванов. Скорости различных паровозов независимы и одинаково распределены, а функция распределения скорости непрерывна.
\end{problem}

\begin{problem}
Согласно законам о трудоустройстве в городе \textit{М}, наниматели обязаны предоставить всем рабочим выходной, если хотя бы у одного из них день рождения, и принимать на службу рабочих независимо от их дня рождения. За исключением этих выходных рабочие трудятся весь год из 365 дней. Предприниматели хотят максимизировать среднее число человеко-дней в году. Сколько рабочих трудятся на фабрике в городе \textit{М}?

\end{problem}

\begin{problem}
В каждую $i$-ю единицу времени живая клетка получает случайную дозу облучения $X_i$, причем $\{ X_i\}_{i=1}^{t}$ имеют 
одинаковую функцию распределения $F_X(x)$ и независимы в совокупности $\forall t \in \mathbb{N}$. Получив интегральную дозу облучения, 
равную $\nu$, клетка погибает. Оценить среднее время жизни клетки ${\mathbb E}T$. 
\end{problem}

\begin{ordre}

Докажите \textit{тождество Вальда}: 
$$
{\mathbb E}S_T={\mathbb E}X_1\cdot {\mathbb E}T, \; \text{где} \; S_T = \sum_{i=1}^{T}X_i,    
$$

\noindent введя вспомогательную случайную величину

$$
Y_j=\begin{cases}
1, &\text{ если }\quad X_1+\ldots +X_{j-1}=S_{j-1}<\nu, \\
0, &\text{ в остальных случаях }, 
\end{cases}
$$
 
\noindent и записав $S_{T}$ в следующем виде: $S_T = \sum_{i=1}^T Y_i X_i$. 

\end{ordre}

\begin{problem}\Star(Теорема Дуба)
\label{sec:doob}
Пусть $Y_0,\dots,Y_n$ -- последовательность случайных величин, являющаяся мартингалом (см. замечание).
Показать, что для  \textit{момента остановки}
$\tau = \inf \{k\leq n: Y_k\geq \lambda \}$  ($\tau = n$, если $\max_{k\leq n}Y_k <\lambda$), верно
\begin{equation*}
\mathbb{E}{Y_{\tau}} = \mathbb{E}Y_n.
\end{equation*}
\end{problem}
\begin{remark}
Определение мартингала (см. А.Н. Ширяев [\ref{chiraiev}], Т.2). Пусть задано вероятностное пространство $(\Omega,\mathcal{F},\mathbb{P})$ с фильтрацией, т.е. с семейством ($\mathcal{F}_n$) $\sigma$-алгебр $\mathcal{F}_n$, $n\geq 0$ таких ,что $\mathcal{F}_0\subseteq\dots\subseteq\mathcal{F}_n\subseteq \mathcal{F}$ (<<фильтрованное вероятностное пространство>>). 
Пусть $Y_0,\dots,Y_n$ ~--- последовательность случайных величин, заданных на $(\Omega,\mathcal{F},\mathbb{P})$, для каждого $n\geq 0 $ величины $Y_n$ являются $\mathcal{F}_n$-измеримыми, тогда $Y = (Y_n,\mathcal{F}_n)$ образует стохастическую последовательность. Мартингалом назвается  такая стохастическая последовательность $Y = (Y_n,\mathcal{F}_n)$, что 
\begin{equation*}
\mathbb{E}|Y_n|\leq\infty,
\end{equation*}
\begin{equation*}
\mathbb{E}(Y_{n+1}|\mathcal{F}_n)=Y_n \quad \text{почти наверное}.
\end{equation*}

Для доказательства теоремы Дуба необходимо показать, что 
%\begin{equation*}
%\mathbb{E}(Y_{\tau}|\mathcal{F}_n) = Y_n,
%\end{equation*}
%\end{remark}
%то есть
 для любого $A\in \mathcal{F}_n$ 
\begin{equation*}
\int_{A} Y_{\tau} d\mathbb{P} = \int_{A}Y_{n}d\mathbb{P}.
\end{equation*}
\end{remark}

\begin{problem}[Мартингалы и теорема о баллотировке]
Пусть $S_n =\xi _1 +....+\xi _n $, где с.в. $\left\{ {\xi _k } \right\}_{k\in {\mathbb N}} $ -- 
независимы и одинаково распределены \[\xi _k =\left\{ {\begin{array}{l}
 1,\quad p=1 \mathord{\left/ {\vphantom {1 2}} \right. 
\kern-\nulldelimiterspace} 2, \\ 
 -1,\;\,p=1 \mathord{\left/ {\vphantom {1 2}} \right. 
\kern-\nulldelimiterspace} 2. \\ 
 \end{array}} \right.\] Покажите, что тогда

$\PR\left( {S_1 >0,...,S_n >0\left| {S_n =a-b} \right.} 
\right)=\frac{a-b}{a+b},$ где $a>b$ и $a+b=n$.
\end{problem}

\begin{remark}
Проинтерпретируем этот результат: $\xi_k =1$ будем 
интерпретировать как голос, поданный на выборах за кандидата $A$, 
$\xi _k =-1$ -- за кандидата $B$. Тогда $S_n $ есть разность числа 
голосов, поданных за кандидатов $A$ и $B$, если в голосовании 
приняло участие $n$ избирателей, а $\PR\left( {S_1 >0,...,S_n >0\left| {S_n 
=a-b} \right.} \right)$ есть вероятность того, что кандидат $A$ все 
время был впереди кандидата $B$, при условии, что $A$ в общей 
сложности собрал $a$ голосов, а $B$ собрал $b$ голосов и $a>b$, 
$a+b=n$. 
\end{remark}

\begin{problem}[Закон арксинуса]
В условия замечания к предыдущей задаче 
покажите, что при 20 бросаниях один их игроков никогда не будет впереди, и с 
вероятностью 0.54 -- будет впереди не более одного раза.
\end{problem}

\begin{remark} 
Воспользуйтесь \textit{законом арксинуса} 

$\PR\left( {k_n <xn} \right)\simeq 
\frac{2}{\pi }\arcsin \sqrt x $, где с.в. $k_n =\left| {\left\{ 
{k=1,...,n:\;\;S_k \ge 0} \right\}} \right|$.
\end{remark}

\begin{problem}[Случайные блуждания]
Заблудившийся грибник оказался в центре леса имеющего форму круга с радиусом 5 км. Грибник движется по следующим правилам: пройдя 100 метров в случайно выбранном направлении север-юг-запад-восток (на это у грибника уходит 2 минуты), он решает снова случайно выбрать направление движения и т.д. Оцените математическое ожидание времени, через которое грибник выйдет из леса. Как изменится ответ, если выбор направления движения грибника осуществляется равновероятно на $[0,2\pi]$? 
\end{problem}

\begin{problem} 
Покажите, что последовательность дискретных с.в. $\left\{ 
{\xi _n } \right\}_{n\in {\mathbb N}} $, принимающих значения в ${\mathbb N}$ 
сходится по распределению к дискретной с.в. $\xi $ тогда и только тогда, 
когда при $n\to \infty $ для любого $k\in {\mathbb N}$ $\PR\left\{ {\xi _n =k} 
\right\}\to \PR\left\{ {\xi =k} \right\}$.
\end{problem}

\begin{problem}[Сходимость по моментам] 
Пусть последовательность функций 
распределения $\left\{ {F_n \left( x \right)} \right\}_{n\in {\mathbb N}} $, 
имеющих все моменты 
\[M_{n,k} =\int {x^kdF_n \left( x \right)} <\infty. \] 
Пусть для всех $k\in {\mathbb N}$ имеют место следующие сходимости:\\ $M_{n,k} \to 
M_k \ne \pm \infty $. Тогда существуют такая подпоследовательность $\left\{ 
{F_{n_m } \left( x \right)} \right\}_{m\in {\mathbb N}} $ и функция 
распределения $F\left( x \right)$ (с моментами $\left\{ {M_k } 
\right\}_{k\in {\mathbb N}} )$, что $F_{n_m } \left( x \right)\to F\left( x 
\right)$ в точках непрерывности $F\left( x \right)$.

Покажите, что если моменты $\left\{ {M_k } \right\}_{k\in {\mathbb N}} $ 
однозначно определяют функцию $F\left( x \right)$ (достаточным условием для 
этого будет существование такого $\chi >0$, что 
$ M_k^{1/k} / k \leq \chi $, $k\in {\mathbb N})$, то в качестве подпоследовательности можно брать 
саму последовательность.

Проинтерпретируйте полученный результат с точки зрения сходимости по 
распределению соответствующих с.в.
\end{problem}
\begin{remark}
См., например, Сачков В.Н. Вероятностные методы в комбинаторном анализе. М.: Наука, 1978.
\end{remark}

\begin{problem}

Покажите, что все моменты распределения
$$p_{\lambda } \left(x\right)=\frac{1}{24} e^{-x^{{1\mathord{\left/ {\vphantom {1 4}} \right. \kern-\nulldelimiterspace} 4} } } \left(1-\lambda \sin x^{{1\mathord{\left/ {\vphantom {1 4}} \right. \kern-\nulldelimiterspace} 4} } \right),\; x\ge 0$$
при любом значении параметра $\lambda \in \left[0,1\right]$ совпадают.

\begin{remark}

Достаточное \textit{условие Карлемана} того, что моменты однозначно определяют распределение, вообще говоря, комплексной случайной величины $x$, имеет вид:
$$
\sum _{n=0}^{\infty }\left(E\left[\left|x\right|^{2n} \right]\right)^{{-1\mathord{\left/ {\vphantom {-1 \left(2n\right)}} \right. \kern-\nulldelimiterspace} \left(2n\right)} } =\infty.
$$

\end{remark}

\end{problem} 

\begin{problem}
Приведите примеры таких случайных величин $X$, $Y$ и $Z$, что вероятностные распределения сумм $X+Y$ и $X+Z$ совпадают, но распределения случайных величин различны.
\end{problem}
\begin{ordre}
Удобнее сначала подобрать соответствующие характеристические функции.
\end{ordre}
\begin{remark}
См. книгу Г. Секея [\ref{sekei}]. Эту же книгу можно рекомендовать и по двум следующим задачам.
\end{remark}

\begin{problem}
Приведите примеры независимых одинаково распределенных случайных величин $X$ и $Y$, для которых распределение суммы  $X+Y$ не однозначно определяет распределение $X$ и $Y$.
\end{problem}


\begin{problem}
Пусть 
  $X$ стохастически меньше, чем $Y$, т. е. 
 $$\forall t \; F_X(t) \leq F_Y(t), \exists t_0: \; F_X(t_0) < F_Y(t_0).$$
 Парадоксально, но может так случиться, что 
  $\PR(X > Y) \geq 0.99$. Приведите пример таких с.в. $X$ и $Y$.
\end{problem}

\begin{problem}[Биржевой парадокс]
Рассмотрим любопытный экономический пример. Пусть имеется начальный капитал $K_1$, который требуется увеличить. Для этого имеются две возможности: вкладывать деньги в надежный банк и покупать на бирже акции некоторой компании. Пусть $u$ -- доля капитала, вкладываемая в банк, а $v$ -- доля капитала, расходуемая на приобретение акций ($0 \leq u + v \leq 1$). Предположим, что банк гарантирует $b \times 100 \%$ годовых, а акции приносят $X \times 100 \%$ годовых, где $X$ -- случайная величина с математическим ожиданием $m_X > b > 0$.  Таким образом, через год капитал составит величину $K_2 = K_1 (1 + b u + Xv)$. Очевидно, что, если придерживаться стратегии, максимизирующей средний доход за год, то выгодно присвоить следующие значения $u = 0$, $v = 1$. 

Рассмотрите прирост капитала $K_{t+1}$   за  $t$ лет, считая $X_1, \ldots,  X_t$ независимыми с.в.  Покажите, что при ежегодном вложении капитала в акции  
\[
\Exp(K_t) \rightarrow \infty, \; \text{при} \; t  \rightarrow \infty,
\]
\noindent но при этом, в случае $\Exp\left[ \log (1 + X) \right] < 0$     

\[
K_t \overset{\text{п.н.}}{\longrightarrow}  0, \; \text{при} \; t  \rightarrow \infty.
\]

\end{problem}

\begin{ordre}
Воспользуйтесь усиленным законом больших чисел, введя замену $K_{t+1} = K_1 e^{t Y_t}$, где  $Y_t = \frac{1}{t} \sum \limits_{i=1}^{t}\log(1+X_i)$. 
\end{ordre}

\begin{remark}
Рассмотренный парадокс хорошо иллюстрирует особенность поведения случайной последовательности, которая в отличие от детерминированной может сходиться в разных смыслах к разным значениям. 
\end{remark}

Причина рассмотренного биржевого парадокса -- неудачный выбор критерия оптимальности. В качестве альтернативы могут быть выбраны следующие критерии:

\begin{enumerate}
\item \textit{Логарифмическая стратегия} или \textit{стратегия Келли}:
\[
Y_t \rightarrow \lambda(u) = \Exp\left[ \log (1 + b u + X (1-u)) \right],
\]
\[
\lambda(u) \rightarrow \max \limits_{ 0 \leq u \leq 1}
\]
\[
\Updownarrow
\]
\[
\Exp\left[ \log (K_{t+1}) \right] \rightarrow \max \limits_{ u_1 \ldots u_t }.
\]

\item Вероятностный критерий:
\[
\mathbb{P}_\phi (u_1 \ldots u_t ) = \mathbb{P} (K_{t+1} \geq -\phi) \rightarrow \max \limits_{ u_1, \ldots, u_t }.
\]
Используя метод \textit{динамического программирования} (см. D.P. Bertsekas. Dynamic Programming and Optimal Control, Vol. 1.  Athena Scientific, Belmont, Massachusetts, 1995), покажите, что оптимальное управление удовлетворяет следующему соотношению: 

\[
u_t(K_t) = \begin{cases}
\begin{array}{cc}
0, & \phi \geq - K_t(1+b), \\
1, & \phi < - K_t(1+b).
\end{array}\end{cases}
\]

\begin{remark}
Подобный тип управления (стратегии) созвучен исторической практике накопления капитала: в эпоху первичного накопления люди зачастую серьезно рисковали ради денег, но как только они накапливали сумму, достаточную для безбедного существования при ее вложении хотя бы в банк, необходимость в риске для них отпадала.  
\end{remark} 
 
\end{enumerate}


\begin{problem}

Требуется определить, начиная с какого этажа брошенный с балкона 100-этажного здания стеклянный шар разбивается. В наличии имеется два таких шара. Предложить метод нахождения граничного этажа, минимизирующий математическое ожидание числа бросков. Рассмотреть случай большего числа шаров.  

\end{problem}

\begin{problem}\Star
На подоконнике лежит $N$ помидоров. Вечером $i$-го дня ($1 \leqslant i \leqslant N$) портится один помидор. Каждое утро человек съедает один (случайно выбранный) свежий помидор из оставшихся. Таким образом, каждый помидор либо испортился, либо был съеден.

\begin{enumerate}
\item Получите рекуррентную формулу для математического ожидания количества съеденных помидоров от числа $N$.
\item Найдите асимптотическую оценку количества съеденных помидоров при $N \rightarrow \infty$.
\end{enumerate}

\end{problem}

\begin{problem}
\begin{enumerate}
\item Имеется монетка (несимметричная). Несимметричность монетки заключается в том, что либо орел выпадает в два раза чаще решки; 
либо наоборот (априорно, до проведения опытов, оба варианта считаются равновероятными). Монетку бросили $10$ раз. Орел выпал $7$ раз. 
Определите апостериорную вероятность того, что орел выпадает в два раза чаще решки (апостериорная вероятность считается с учетом 
проведенных опытов; иначе говоря, это просто условная вероятность). 

\item Определите апостериорную вероятность того, что орел выпадает не менее чем в два раза чаще решки. Если несимметричность 
монетки заключается в том, что либо орел выпадает не менее чем в два раза чаще решки; либо наоборот (априорно оба варианта считаются 
равновероятными). 
\end{enumerate}
\end{problem}





\begin{problem}[Распределение Гумбеля или двойное экспоненциальное распределение] 
\label{gumbel}
Пусть $\varsigma _{1} ,...,\varsigma _{n} $ -- независимые одинаково распределенные случайные величины, и существуют такие константы $\alpha ,T>0$, что
\[\mathop{\lim }\limits_{y\to \infty } e^{{y\mathord{\left/ {\vphantom {y T}} \right. \kern-\nulldelimiterspace} T} } \left[1-\PR\left(\varsigma _{k} <y\right)\right]=\alpha , \] 
тогда
\[
\max \left\{\varsigma _{1} ,...,\varsigma _{n} \right\}-T\ln \left(\alpha n\right)\xrightarrow[{n\to \infty }]{d} \xi,  
\]
\noindent где $\PR\left(\xi <\tau \right)=\exp \left\{-e^{-{\tau \mathord{\left/ {\vphantom {\tau  T}} \right. \kern-\nulldelimiterspace} T} } \right\}.$

\end{problem}


\begin{problem}\Star(Logit-распределение или распределение Гиббса)
\label{gibbs}
Пусть известно, что $\xi_{1} ,\ldots,\xi_{n}$~--- независимые одинаково распределенные по \textit{закону Гумбеля} случайные величины. Пусть  $x_{k} =C_{k} +\xi _{k} ,$ $k=1,...,n$, где $C_{k} $ -- не случайные величины. Положим $q_n=\arg \mathop{\max }\limits_{k=1,\ldots,n} \left\{C_{k} +\xi _{k} \right\}$. Тогда с.в. $q_n$ имеет распределение:
\[\PR\left(q_n=k\right)=\frac{e^{{C_{k} \mathord{\left/ {\vphantom {C_{k}  T}} \right. \kern-\nulldelimiterspace} T} } }{\sum _{l=1}^{n}e^{{C_{l} \mathord{\left/ {\vphantom {C_{l}  T}} \right. \kern-\nulldelimiterspace} T} }  } , \quad k=1,\ldots,n.\] 
Докажите, что результат останется верным, если $n \to \infty$, a также $\xi_k$ -- независимые одинаково распределенные, причем 
\[
\mathop{\lim }\limits_{y\to \infty } e^{{y\mathord{\left/ {\vphantom {y T}} \right. \kern-\nulldelimiterspace} T} } \left[1-\PR\left(\xi _{k} <y\right)\right]=\alpha \geq 0. 
\]
\end{problem}
\begin{remark}
См. монографию Andersen S.P., de Palma A., Thisse J.-F. Discrete choice theory of product differentiation. MIT Press, Cambridge, 1992; см. также \cite{222}.
\end{remark}


\begin{problem}[Рекорды]
Пусть $X_1 ,X_2 ,\ldots $ - независимые 
случайные величины с одной и той же плотностью распределения вероятностей 
$p(x)$. Будем говорить, что наблюдается рекордное значение в момент времени 
$n>1$, если $X_n >\max \left[ {X_1 ,...,X_{n-1} } \right]$. Докажите 
следующие утверждения.

\begin{enumerate}
\item Вероятность того, что рекорд зафиксирован в момент времени $n$, 
равна $1/n$;

\item Математическое ожидание числа рекордов до момента времени $n$ 
равно 
\[
\sum\limits_{1<k\le n} {\frac{1}{k}} \sim \ln n;
\]

\item Пусть $Y_n $ --- случайная величина, принимающая значение $1$, если 
в момент времени $n$ зафиксирован рекорд, и значение $0$ -- в противном случае. 
Тогда случайные величины $Y_1 ,Y_2 ,\ldots$ независимы в совокупности;

\item Дисперсия числа рекордов до момента времени $n$ равна
\[
\sum\limits_{1<k\le n} {\frac{k-1}{k^2}} \sim \ln n;
\]

\item Если $T$ -- момент появления первого рекорда после момента времени $1$, то $\Exp T= \infty$.
\end{enumerate}
\end{problem}

\begin{ordre}
См. Кельберт--Сухов Т 1 \cite{4}.
\end{ordre}

\begin{remark}
Приведем для справки следующую теорему. Пусть $\eta_0,\eta_1,\dots$ ~--- последовательность независимых случайных величин с одной и той же непрерывной функцией распределения. Для каждого $n\in \mathbb{Z}_{+}$ по случайным величинам $\eta_0,\eta_1,\dots,\eta_n$ построим вариационный ряд 
$$\eta_{0,n}\leq \eta_{1,n}\leq\dots\leq\eta_{n,n}.$$
Рекордные моменты $\{\nu(n),n\in\mathbb{Z}_{+}\}$ определяются следующим образом: $\nu(0) = 0$ и
$$\nu(n+1)=\min\{j>\nu(n): \eta_j>\eta_{j-1,j-1}\},\quad n\in \mathbb{Z}_{+}.$$
Верен следующий результат (см. Якымив А.Л. Вероятностные приложения тауберовых теорем МАИК Наука, 2005, с. 256):
$$\mathbb{P}\{\nu(n)>t\}\equiv \frac{t^{-1}\ln^{n-1}(t)}{(n-1)!}.$$

\medskip

Для доказательства приведенной теоремы существенно используются тауберовы теоремы.
Тауберовыми теоремами называют теоремы, выводящие из асимптотических свойств производящих функций и преобразований Лапласа функций и последовательностей  (а также других интегральных преобразований) асимптотики этих функиций и последовательностей (то есть эти теоремы обратные к абелевым). 
\end{remark}

\begin{problem}[Распределение Коши]
Радиоактивный источник испускает 
частицы в случайном направлении (при этом все направления равновероятны). 
Источник находится на расстоянии $d$ от фотопластины, которая представляет 
собой бесконечную вертикальную плоскость.

\begin{enumerate}
\item При условии, что частица попадает в плоскость, покажите, что 
горизонтальная координата точки попадания (если начало координат выбирается 
в точке, ближайшей к источнику) имеет плотность распределения:
\[
p\left( x \right)=\frac{d}{\pi \left( {d^2+x^2} \right)}.
\]
Это распределение известно как \textit{распределение Коши}.

\item Можно ли вычислить среднее (математическое ожидание) этого 
распределения?

\item Предположим, что параметр $d$ неизвестен, но есть $n \gg 1$ независимых реализаций рассматриваемой с.в. $X_1, \ldots, X_n$. Предложите способ оценивания $d$. То есть необходимо указать функцию от выборки (статистику) $\widehat d_n=\widehat d_n(X^n)$,
значение которой будет рассматриваться в качестве приближения к неизвестному истинному значению~d. 
\begin{remark}
К такого вида оценке, как правило, предъявляются следующие требования:
\begin{enumerate}
\item Состоятельность: $\widehat d_n \rightarrow d$ почти наверное или по вероятности;
\item Несмещенность: $\Exp(\widehat d_n) = d$.
\end{enumerate}

Для сравнения между собой различных оценок одного и того же параметра выбирают некоторую \textit{функцию риска}, которая измеряет отклонение оценки от истинного значения параметра.
Чаще всего в качестве функции риска рассматривают дисперсию статистики. 
\end{remark}

\end{enumerate}
\end{problem}

\begin{problem}
\label{condExp1}
Предположим, что с.в. $X\in L_2$, это означает ${\mathbb E}(X^2)<\infty$. Докажите, что 
\begin{equation*}
\label{UMO}\tag{E}
\| X-{\mathbb E}(X|Y_1,\ldots,Y_n)\|_{2}=\min\limits_{\varphi\in H} \| X-\varphi(Y_1,\ldots,Y_n)\|_{2} , 
\end{equation*}
где $H$ --- подпространство пространства $L_2$ всевозможных борелевских функций $\varphi(Y_1,\ldots,Y_n)\in L_2$; 
${\mathbb E}(X|Y_1,\ldots,Y_n)$ --- условное математическое ожидание с.в. $X$ относительно $\sigma$-алгебры, порожденной с.в. 
$Y_1,\ldots,Y_n$, часто говорят просто относительно с.в. $Y_1,\ldots,Y_n$; 
$$
\| X\|_{2}=\sqrt{\langle X,X\rangle_{2}}=\sqrt{{\mathbb E}(X\cdot X)}=\sqrt{{\mathbb E}(X^2)} . 
$$
Обобщите утверждение задачи на случай, когда $X$ -- случайный вектор.
\end{problem}


\begin{ordre}
Покажите, что $(X-{\mathbb E}[X|H]) \bot \xi,\; \forall\xi\in H$, т.е. ${\mathbb E}(\cdot|H)$ 
является проектором на подпространство $H$ в $L_2$. Детали см., например, в учебнике Розанов Ю.А. Теория вероятностей. Случайные процессы. Математическая статистика. М.: Наука, 1985.
\end{ordre}

\begin{problem}
Используя выражение (\ref{UMO}) в качестве определения условного математического ожидания, докажите справедливость основных свойств математического ожидания, в частности
\[
\Exp(\Exp(X|Y)) = \Exp(X). 
\]
\end{problem}

\begin{ordre}
$\langle 1, X \rangle = \langle 1, \Exp(X|Y) \rangle$.
\end{ordre}


\begin{problem}
\label{condExp3}
Докажите, что если в условиях предыдущей задачи вектор $(X,Y_1,\ldots,Y_n)^T$ --- является нормальным случайным вектором (без ограничения 
общности можно также считать, что $(Y_1,\ldots,Y_n)^T$  --- невырожденный нормальный случайный вектор), то в качестве $H$ можно взять 
подпространство всевозможных линейных комбинаций с.в. $Y_1,\ldots,Y_n$. Т.е. можно более конкретно сказать, на каком именно 
классе борелевских функций достигается минимум в задаче $\ref{UMO}$. 
\end{problem}

\begin{ordre}
Будем искать 
${\mathbb E}(X|Y_1,\ldots,Y_n)$ в виде 
$$
\label{Gauss}
{\mathbb E}(X|Y_1,\ldots,Y_n)=c_1 Y_1+\ldots +c_n Y_n . 
$$
Докажите следующие утверждения:

\begin{enumerate}
\item $X-c_1 Y_1-\ldots-c_n Y_n, Y_1,\ldots, Y_n$ -- независимы.
\item $X-c_1 Y_1-\ldots-c_n Y_n$ ортогонален подпространству $H$ пространства $L_2$ всевозможных борелевских функций $\varphi(Y_1,\ldots,Y_n)\in L_2$.
\end{enumerate}
 
\end{ordre}


\begin{comment}
\begin{problem}
Некто обладает одной облигацией, которую намеревается продать в один из последующих четырех дней, в которых цена облигации 
принимает различные значения, априори неизвестные, но становящиеся известными в начале каждого дня продаж. Предполагается, что 
цены облигации независимы и их перестановки по торговым дням равновозможны. Какова стратегия продавца, состоящая в выборе дня 
продажи облигации и гарантирующая максимальную вероятность того, что он продаст облигацию в день ее наибольшей цены? 
\end{problem}

\begin{ordre}
Рассмотреть следующие возможные стратегии и сравнить вероятности продажи облигации в день наибольшей цены: 
\begin{enumerate}
\item на первом шаге (в первый день торгов) запомним имевшую место цену облигации, не продавая ее, а затем продадим 
облигацию в тот день, когда ее цена окажется большей цены, зафиксированной в первый день, или (когда такого дня не окажется) в 
последний (четвертый) день, независимо от цены этого дня (стратегия $S_1$); 

\item не продавая облигацию в первом и втором торговых днях, зафиксируем  максимальную цену из двух, имевших место для этих дней, 
и продадим облигацию в третьем торговом дне, если цена облигации в нем будет выше, чем указанная зафиксированная максимальная цена, 
или, в противном случае, в четвертом дне (стратегия $S_2$). 
\end{enumerate}
\end{ordre}
\end{comment}

\begin{problem}
Ведущий приносит два одинаковых конверта и говорит, что в них лежат деньги, причем в одном вдвое больше, чем в другом. Двое участников берут конверты и тайком друг от друга смотрят, сколько в них денег. Затем один говорит другому: «Махнемся не глядя?» (предлагая поменяться конвертами). Стоит ли соглашаться?
\end{problem}
\begin{remark}
См. А. Шень. Вероятность: примеры и задачи; Г. Секей \cite{book12} (эту книгу полезно посмотреть и в следующей задаче). В книги Г. Секея поднимаются в связи с аналогичной задачей вопросы о аксиоматике теории вероятностей. Отметим, что аксимоматика А.Н. Колмогорова, построеная на теории меры не единственный способ ввести случайные величины. Интересные материалы на эту тему имеются в статье D. Mumford'а "На заре эры стохастичности" в сборнике "Математика: граница и перспективы. Под ред. Д.В. Аносова и А.Н. Паршина. М. Фазис, 2005" (также отметим в этом сборнике статью W.T. Gowers'а и аналогичный сборник "Математические события XX века. М.: Фазис, 2003"). Также в этой связи можно отметить книжку Janes E.T. \cite{200} и Кановей В.Г., Любецкий В.А. Современная теория множеств: абсолютно неразрешимые классические проблемы. М.: МЦНМО, 2013.  
\end{remark}

\begin{problem}[Спящая красавица]
В воскресенье с красавицей обговаривается схема эксперимента, согласно которой в вечером в воскресенье красавица усыпляется. Далее подкидывается симметричная монетка. Если монетка выпадает орлом, то красавицу будят в понедельник (потом снова дают снотворное), затем будят еще раз во вторник (потом снова дают снотворное). Если решкой, то будят только в понедельник (потом снова дают снотворное). В среду красавицу пробуждают окончательно в любом случае. Снотворное стирает красавице память в том смысле, что она помнит правила, оговоренные с ней в воскресенья, но не помнит сколько раз уже ее будили и не знает какой сегодня день недели. Каждый раз когда красавицу будят ей предлагают оценить вероятность того, что монетка выпала решкой. Что может ответить красавица?
\end{problem}

\begin{problem}[Как играть в проигрышную игру] 
Предположим, что в некоторой игре четное число розыгрышей $2n$. Игрок  $A$ в одном розыгрыше выигрывает с вероятностью 0.45, $B$ --  с вероятностью 0.55. Чтобы выиграть в игру, игрок должен набрать более половины всех очков. Если у  $A$ есть возможность выбрать  $2n$, то, как ни странно $2n=2$ не является лучшим выбором. Найдите оптимальное $2n$.    
\end{problem}

\begin{problem}[А.Н. Соболевский]
В лесах дремучих стоит дом не дом, чертог не чертог, а дворец зверя лесного, чуда морского, весь в огне, в серебре и золоте и каменьях самоцветных. Красная девица входит на широкий двор, в ворота растворенные, и находит там три двери, а за ними три горницы красоты несказанной, а в каждой из тех горниц еще по три двери, ведущие в горницы краше прежних.
Походив по горницам, красная девица начинает догадываться, что дворец построен ярусами: двери со двора ведут в горницы первого яруса, из тех -- в горницы второго яруса, и так далее. В каждой горнице есть вход и три выхода, ведущие в три горницы следующего яруса. В горницах последнего, n-го яруса растворены окна широкие во сады диковинные, плодовитые, а в садах птицы поют и цветы растут.
Вернувшись на широкий двор и отдохнув, красная девица видит, что произошла перемена: ворота, через которые она вошла, и часть дверей внутри дворца сами собой затворились,
да не просто так, а каждая дверь с вероятностью 1/3 независимо от других. Немного обеспокоенная, красная девица начинает метаться из горницы в горницу сквозь оставшиеся незатворенными двери в поисках выхода. Покажите, что при больших $n$ вероятность, что
она сможет добраться до окон, растворенных в сады, близка к $(9 - \sqrt{27}) / 4 \approx 0.95$.
\end{problem}


 \begin{problem} [Задача М. Гарднера о разборчивой невесте]
 В аудитории находится невеста, которая хочет выбрать себе жениха. За дверью выстроилась очередь из $N$ женихов. Относительно любых 
 двух женихов невеста может сделать вывод, какой из них для неё предпочтительнее. Таким образом, невеста задает на множестве женихов 
 отношение порядка (естественно считать, что если $A$ предпочтительнее $B$, а $B$ предпочтительнее $C$, то $A$ предпочтительнее $C$). 
 Предположим, что все $N!$ вариантов очередей равновероятны и невеста об этом знает (равно, как и число $N$). Женихи запускаются 
 в аудиторию по очереди. Невеста видит каждого из них в первый раз! Если на каком-то женихе невеста остановится (сделает свой выбор), 
 то оставшаяся очередь расходится. Невеста хочет выбрать наилучшего жениха (исследуя $k$–го по очереди жениха, невеста лишь может 
 сравнить его со всеми предыдущими, которых она уже просмотрела и пропустила). Оцените (при $N\to\infty$) вероятность того, что невесте 
 удастся выбрать наилучшего жениха, если она придерживается следующей стратегии: просмотреть (пропустить) первых по очереди $[N/e]$ 
 кандидатов и затем выбрать первого кандидата, который лучше всех предыдущих (впрочем, такого кандидата может и не оказаться, тогда, 
 очевидно, невеста не смогла выбрать наилучшего жениха). 
 \end{problem}
 \begin{remark}
 Можно показать, что описанная стратегия будет асимптотически оптимальной. Популярное изложение имеется у С.М. Гусейн-Заде. Однако полезно обратить внимание, что эта задача является ярким примером целого направления: оптимальной остановки стохастических процессов, перетекающего в оптимальное управление процессами Маркова. Наиболее полезным для приложений (особенно в области "Исследование операций") во всем этом является распространение принципа динамического программирования Беллмана на стохастический случай ({\itпринцип Вальда--Беллмана}). Об этом можно прочитать, например, у Дынкина--Юшкевича, Ширяева, Аркина--Евстигнеева. Для введения можно посмотреть книги Е.С. Вентцель или Ю.А. Розанова \cite{6}. Часть студентов ФУПМ также столкнется с этим в курсе стохастических дифференциальных уравнений. В таком контексте полезно будет посмотреть книгу Б. Оксендаля Стохастические дифференциальные уравнения: Введение в теорию и приложения -- М.: МИР, 2003. 
 \end{remark}

\begin{problem}[Равновесие Нэша]
Один игрок прячет (зажимает в кулаке) одну или две монеты достоинством 10 рублей. Другой игрок должен отгадать, сколько денег у первого спрятано. Если отгадывает, то получает деньги, если нет -- платит 15 рублей. Каковы  должны быть стратегии игроков при многократном повторении игры?

\end{problem}

\begin{problem}
В аудитории находится 100 человек (игроков). Каждого просят написать целое число от 1 до 100. Победителем окажется тот участник, который написал число, наиболее близкое к 2/3 от среднего арифметического всех чисел. Требуется найти  \textit{наилучший ответ} при фиксированных стратегиях соперников: каждый соперник, просчитав на $x \in \mathrm{Po}(2)$  хода  вперед, выбирает наугад (равномерно) число от 1 до $100 \cdot(2/3)^x$. К примеру, возможен следующий ход мыслей соперника: поскольку все догадались, что не стоит писать число большее $100 \cdot(2/3)$, то не стоит писать число большее $100 \cdot(2/3)^2$.
\end{problem}

\begin{remark}
Стратегия (действие игрока $i$ в зависимости от своего типа $t_i \in T_i$) $s_i^*: T_i, S_{-i} \rightarrow A_i$ называется \textit{наилучшим ответом} на заданные стратегии соперников $s_{-i}(t_{-i})$, если она является решением задачи максимизации ожидаемого выигрыша рассматриваемого игрока. При этом усреднение производится по всем неизвестным переменным, относящимся к соперникам (типам соперников $t_{-i}$): 
 \[
 \underset{t_{-i}}{\sum} u_i(s_i^*, s_{-i}(t_{-i}), t_i, t_{-i}) \cdot  \PR(t_{-i} | t_i) = \]\[ \underset{a_i \in A_i}{\max} \underset{t_{-i}}{\sum} u_i(a_i, s_{-i}(t_{-i}), t_i, t_{-i}) \cdot \PR(t_{-i} | t_i), 
 \]
 \noindent где $u_i$ -- выигрыш игрока $i$  при заданных действиях и типах всех игроков, $A_i$ -- множество действий игрока $i$, $\PR(t_{-i} | t_i)$ -- представление о типах остальных игроков при известном своем типе. 
 
Профиль стратегий $\{s_i^*\}$ -- есть множество \textit{равновесных} стратегий (\textit{равновесие по Нэшу}), где каждая стратегия является наилучшим ответом на фиксированные стратегии соперников. Поиск оптимальной стратегии зачастую подразумевает поиск равновесной стратегии. Равновесие по Нэшу является неподвижной точкой отображения $S_1 \times \ldots \times S_n \rightarrow S_1^* \times \ldots \times S_n^*$, где $S_i$ -- множество отображений $T_i \rightarrow A_i$, $S_i$ -- множество наилучших ответов $i$-го игрока, причем профиль стратегий $(s_1, \ldots, s_n)$ отображается во множество ответов каждого из игроков (имеет место многозначное отображение). Такая точка не всегда существует, но всегда существует ее аналог в случае, когда игрок может смешивать несколько стратегий (т.е. смешанной стратегией будет дискретное распределение $(p_1,\ldots,p_{|S_i|}),\; \PR(s_i = k) = p_k$).     
\end{remark}

 \begin{comment}
\begin{problem}[сублинейный приближенный вероятностный алгоритм для 
матричных игр; Григориадис -- Хачиян, 1995]
Рассматривается симметричная 
антагонистическая игра двух лиц X и Y. Смешанные стратегии X и Y будем 
обозначать соответственно $\vec {x}$ и $\vec {y}$. При этом $x_k $ - 
вероятность того, что игрок X выберет стратегию с номером k, аналогично 
определяется $y_k $. Таким образом, $\vec {x},\vec {y}\in S=\left\{ {\vec 
{x}\in {\mathbb R}^n:\;\;\vec {e}^T\vec {x}=1,\;\vec {x}\ge \vec {0}} \right\}$, 
где $\vec {e}=\left( {1,...,1} \right)^T$. Выигрыш игрока X: $V_X \left( 
{\vec {x},\vec {y}} \right)=\vec {y}^TA\vec {x}$, а выигрыш игрока Y: $V_Y 
\left( {\vec {x},\vec {y}} \right)=-\vec {y}^TA\vec {x}$ (игра 
антагонистическая). Каждый игрок стремится максимизировать свой выигрыш, при 
заданном ходе оппонента. Равновесием Нэша (в смешанных стратегиях) 
называется такая пара стратегий $\left( {\vec {x}^\ast ,\;\vec {y}^\ast } 
\right)$, что
\[
\vec {x}^\ast \in \mbox{Arg}\mathop {\max }\limits_{\vec {x}\in S} \vec 
{y}^{\ast T} A\vec {x},
\quad
\vec {y}^\ast \in \mbox{Arg}\mathop {\min }\limits_{\vec {y}\in S} \vec 
{y}^TA\vec {x}^\ast .
\]
Ценой игры называют $\mathop {\max }\limits_{\vec {x}\in S} \mathop {\min 
}\limits_{\vec {y}\in S} \vec {y}^TA\vec {x}=\mathop {\min }\limits_{\vec 
{y}\in S} \mathop {\max }\limits_{\vec {x}\in S} \vec {y}^TA\vec {x}=\vec 
{y}^{\ast T}A\vec {x}^\ast $. Поскольку, по условию, игра также симметричная, 
то $A=-A^T$ - матрица$n\times n$. С помощью стандартной редукции можно 
свести к этому случаю общий случай произвольной матричной игры. В 
рассматриваемом же случае цена игры (выигрыш игроков в положении равновесия 
Нэша) есть 0, а множества оптимальных стратегий игроков совпадают. Требуется 
найти с точностью $\varepsilon >0$ положение равновесия Нэша (оптимальную 
стратегию), т.е. требуется найти такой вектор $\vec {x}$, что $A\vec {x}\le 
\varepsilon \vec {e}$, $\vec {x}\in S$. Покажите, считая элементы матрицы 
$A$ равномерно ограниченными, скажем, единицей, что приводимый ниже алгоритм 
находит с вероятностью не меньшей $1 \mathord{\left/ {\vphantom {1 2}} 
\right. \kern-\nulldelimiterspace} 2$ (вместо $1 \mathord{\left/ {\vphantom 
{1 2}} \right. \kern-\nulldelimiterspace} 2$ можно взять любое положительное 
число меньшее единицы) такой $\vec {x}$ за время ${\rm O}\left( {\varepsilon 
^{-2}n\log ^2n} \right)$, т.е. в определенном смысле даже не вся матрица (из 
$n^2$ элементов) просматривается. Отметим также, что в классе 
детерминированных алгоритмов, время работы растет с ростом $n$ не медленнее 
чем $\sim n^2$ (эта нижняя оценка получается из информационных соображений). 
Другими словами, никакой детерминированный алгоритм не может также 
асимптотически быстро находить приближенно равновесие Нэша. Точнее говоря, 
описанный ниже вероятностный алгоритм дает почти квадратичное ускорение по 
сравнению с детерминированными.

\underline {\textbf{Алгоритм}}

\begin{enumerate}
\item \textbf{Инициализация:} $\vec {x}=\vec {U}=\vec {0}$, $\vec {p}={\vec {e}} \mathord{\left/ {\vphantom {{\vec {e}} n}} \right. \kern-\nulldelimiterspace} n$, $t=0$.
\item \textbf{Повторить:}
\item \textbf{Счетчик итераций: }$t:=t+1$.
\item \textbf{Датчик случайных чисел:} выбираем $k\in \left\{ {1,...,n} \right\}$ с вероятностью $p_k $.
\item \textbf{Модификация }$\vec {X}$\textbf{: }$X_k :=X_k +1$.
\item \textbf{Модификация }$\vec {U}$\textbf{:} $U_i :=U_i +a_{ik} $, $i=1,...,n$.
\item \textbf{Модификация }$\vec {p}$\textbf{:} $p_i :={p_i \exp \left( {\varepsilon {a_{ik} } \mathord{\left/ {\vphantom {{a_{ik} } 2}} \right. \kern-\nulldelimiterspace} 2} \right)} \mathord{\left/ {\vphantom {{p_i \exp \left( {\varepsilon {a_{ik} } \mathord{\left/ {\vphantom {{a_{ik} } 2}} \right. \kern-\nulldelimiterspace} 2} \right)} {\left( {\sum\limits_{j=1}^n {p_j \exp \left( {\varepsilon {a_{jk} } \mathord{\left/ {\vphantom {{a_{jk} } 2}} \right. \kern-\nulldelimiterspace} 2} \right)} } \right)}}} \right. \kern-\nulldelimiterspace} {\left( {\sum\limits_{j=1}^n {p_j \exp \left( {\varepsilon {a_{jk} } \mathord{\left/ {\vphantom {{a_{jk} } 2}} \right. \kern-\nulldelimiterspace} 2} \right)} } \right)}$, $i=1,...,n$.
\item \textbf{Критерий останова:} если ${\vec {U}} \mathord{\left/ {\vphantom {{\vec {U}} t}} \right. \kern-\nulldelimiterspace} t\le 
\varepsilon \vec {e}$, то останавливаемся и печатаем ${\vec {x}=\vec {X}} \mathord{\left/ {\vphantom {{\vec {x}=\vec {X}} t}} \right. \kern-\nulldelimiterspace} t$.
\end{enumerate}
\textbf{Указание.} Покажите, что с вероятностью не меньшей, чем $1 
\mathord{\left/ {\vphantom {1 2}} \right. \kern-\nulldelimiterspace} 2$ 
алгоритм остановится через $t^\ast =4\varepsilon ^{-2}\ln n$ итераций. Для 
этого введите $P_i \left( t \right)=\exp \left( {{\varepsilon U_i \left( t 
\right)} \mathord{\left/ {\vphantom {{\varepsilon U_i \left( t \right)} 2}} 
\right. \kern-\nulldelimiterspace} 2} \right)$ и $\Phi \left( t 
\right)=\sum\limits_{i=1}^n {P_i \left( t \right)} $. Покажите, что

$M\left[ {\left. {\Phi \left( {t+1} \right)} \right|\vec {P}\left( t \right)} 
\right]=\Phi \left( t \right)\sum\limits_{i,k=1}^n {p_i \left( t \right)} 
p_k \left( t \right)\exp \left( {{\varepsilon a_{ik} } \mathord{\left/ 
{\vphantom {{\varepsilon a_{ik} } 2}} \right. \kern-\nulldelimiterspace} 2} 
\right)$ и $\exp \left( {{\varepsilon a_{ik} } \mathord{\left/ {\vphantom 
{{\varepsilon a_{ik} } 2}} \right. \kern-\nulldelimiterspace} 2} \right)\le 
1+{\varepsilon a_{ik} } \mathord{\left/ {\vphantom {{\varepsilon a_{ik} } 
2}} \right. \kern-\nulldelimiterspace} 2+{\varepsilon ^2} \mathord{\left/ 
{\vphantom {{\varepsilon ^2} 6}} \right. \kern-\nulldelimiterspace} 6$.

Используя это и кососимметричность матрицы $A$, покажите
\[
M\left[ {\Phi \left( {t+1} \right)} \right]\le M\left[ {\Phi \left( t 
\right)} \right]\left( {1+{\varepsilon ^2} \mathord{\left/ {\vphantom 
{{\varepsilon ^2} 6}} \right. \kern-\nulldelimiterspace} 6} \right).
\]
Следовательно, $M\left[ {\Phi \left( t \right)} \right]\le n\exp \left( 
{{t\varepsilon ^2} \mathord{\left/ {\vphantom {{t\varepsilon ^2} 6}} \right. 
\kern-\nulldelimiterspace} 6} \right)$ и $M\left[ {\Phi \left( {t^\ast } 
\right)} \right]\le n^{5 \mathord{\left/ {\vphantom {5 3}} \right. 
\kern-\nulldelimiterspace} 3}$. Отсюда по неравенству Маркова имеем, что 
($n\ge 8)$
\[
P\left( {\Phi \left( {t^\ast } \right)\le n^2} \right)\ge P\left( {\Phi 
\left( {t^\ast } \right)\le 2n^{5 \mathord{\left/ {\vphantom {5 3}} \right. 
\kern-\nulldelimiterspace} 3}} \right)\ge 1 \mathord{\left/ {\vphantom {1 
2}} \right. \kern-\nulldelimiterspace} 2.
\]
Тогда $P\left( {{\varepsilon U_i \left( {t^\ast } \right)} \mathord{\left/ 
{\vphantom {{\varepsilon U_i \left( {t^\ast } \right)} 2}} \right. 
\kern-\nulldelimiterspace} 2\le 2\ln n,\;i=1,...,n} \right)\ge 1 
\mathord{\left/ {\vphantom {1 2}} \right. \kern-\nulldelimiterspace} 2$. 
Откуда уже следует, что $P\left( {\vec {x}\left( {t^\ast } \right)\le 
\varepsilon \vec {e}} \right)\ge 1 \mathord{\left/ {\vphantom {1 2}} \right. 
\kern-\nulldelimiterspace} 2$.
\end{problem}
 \end{comment}

\begin{problem}[Аукцион первой цены]
Два участника аукциона конкурируют за покупку некоторого объекта. Ценности объекта    $v_1$ и $v_2$ для участников являются независимыми случайными величинами, равномерно распределенными на отрезке [0, 1]. Участник  имеет точную информацию о своем значении $v_i$, но не знает $v_j$. Участники делают ставки из диапазона [0, 1] одновременно и независимо друг от друга. В данном аукционе побеждает тот, кто поставил большую ставку. При равенстве ставок бросается жребий. Каждый обязан заплатить по средней ставке, даже если ему объект не достается! Отказаться от участия в этом аукционе нельзя.
\begin{enumerate}
\item Выпишите функции выигрыша игороков в данной игре;
\item Найдите наилучший ответ игрока в данной игре в классе квадратичных стратегий: $b_i(v_i) = cv_i^2$, где $c > 0$. Зависит ли наилучший ответ от стратегии соперника в данном случае?
\item Покажите, что в этой игре нет других (кроме найденных в пункте б)) оптимальных решений (наилучших ответов) с гладкими  возрастающими симметричными стратегиями. \textit{Симметричными} стратегиями является пара $(b_1(v_1), b_2(v_2)) = (b(v_1), b(v_2))$.
\end{enumerate}
\end{problem}

\begin{ordre}
Стратегией игрока $i$ в данной игре является функция $b_i: [0, 1] \rightarrow [0, 1]$ равная величине ставки при ценности объекта  $v_i$.
\end{ordre}


\begin{problem}[Распределения канторовского типа]
Рассмотрим в сумме $\sum_{k=1}^\infty 2^{-k} X_{k}$, где $X_{k} $ -- взаимно независимые с.в., имеющие распределение Бернулли с параметром ${1\mathord{\left/ {\vphantom {1 2}} \right. \kern-\nulldelimiterspace} 2} $, только слагаемые с четными номерами, или, что с точностью до множителя 3 (в дальнейшем потребуется для удобства) есть $Y=3\sum _{s=1}^{\infty }4^{-s} X_{s}  $. Покажите, что функция распределения $F(x)=P\left\{Y\le x\right\}$ является сингулярной (когда не оговаривается относительно какой меры, подразумевается, что относительно \textit{меры Лебега}, т.е. равномерной).


\begin{ordre}
Можно рассматривать $Y$ как выигрыш игрока, который получает $3\cdot 4^{-k} $, когда $k$-е бросание симметричной монеты дает в результате решку. Ясно, что полный выигрыш лежит между 0 и $3\left(4^{-1} +4^{-2} +\ldots \right)=1$. Если первое подбрасывание монеты привело к решке, то полный выигрыш $\ge {3\mathord{\left/ {\vphantom {3 4}} \right. \kern-\nulldelimiterspace} 4} $, тогда как в противоположном случае $Y\le 3\left(4^{-2} +4^{-3} +\ldots \right)=4^{-1} $. То есть неравенство ${1\mathord{\left/ {\vphantom {1 4}} \right. \kern-\nulldelimiterspace} 4} <Y<{3\mathord{\left/ {\vphantom {3 4}} \right. \kern-\nulldelimiterspace} 4} $ не может быть осуществлено ни при каких обстоятельствах, значит $F(x)={1\mathord{\left/ {\vphantom {1 2}} \right. \kern-\nulldelimiterspace} 2} $ в интервале $x\in \left({1\mathord{\left/ {\vphantom {1 4}} \right. \kern-\nulldelimiterspace} 4} ,{3\mathord{\left/ {\vphantom {3 4}} \right. \kern-\nulldelimiterspace} 4} \right)$. Чтобы определить, как ведет себя функция распределения на интервале $x\in \left(0,{1\mathord{\left/ {\vphantom {1 4}} \right. \kern-\nulldelimiterspace} 4} \right)$, покажите, что на этом интервале график отличается только преобразованием подобия $F(x)={1\mathord{\left/ {\vphantom {1 2}} \right. \kern-\nulldelimiterspace} 2} F(4x)$.

\end{ordre}

\begin{remark}
Пример, когда свертка двух сингулярных распределений имеет непрерывную плотность: с.в. $X=\sum _{k=1}^{\infty }2^{-k} X_{k}  $ имеет равномерное распределение на $\left(0;1\right)$. Обозначим сумму членов ряда с четными и нечетными номерами через $U$ и $V$ соответственно. Ясно, что $U$ и $2V$ имеют одинаковое распределение и их распределение относится к канторовскому типу.
\end{remark}

\end{problem}


\begin{problem}
Напомним, что \textit{сингулярными} мерами называются меры, функции распределения $F(x)$ которых непрерывны, но точки их роста ($x$ -- точка роста $F(x)$, если для любого $\varepsilon >0$ выполняется: $F(x+\varepsilon )-F(x-\varepsilon )>0$) образуют множество нулевой меры Лебега. Покажите, что мера, соответствующая функции Кантора, сингулярна по отношению к мере Лебега.

\end{problem}

\begin{remark} (Соболевский А.Н. Конкретная теория вероятностей)
Любая вероятностная мера может быть представлена в виде суммы абсолютно непрерывной, дискретной и сингулярной мер. 

Сингулярные распределения вероятности возникают в эргодической теории и математической статистической физике как инвариантные меры диссипативных динамических систем, обладающих т.н. ``странными аттракторами''. Количественное изучение таких мер относится к геометрической теории меры и известно под названием ``фрактальной геометрии''. Основные импульсы развития этой дисциплины исходили из работ К. Каратеодори, Ф. Хаусдорфа, А. Безиковича 1920-х годов, а позднее -- Б. Мандельброта и многочисленных физиков, которые занимались ``динамическим хаосом'' в 1980-х годах (П. Грассбергер, И. Прокачча, Дж. Паризи, У. Фриш). Подробнее о мультифрактальных мерах см., например, книги: Е. Федер, Фракталы -- М.: Мир, 1991; K. Falconer. Fractal Geometry: Mathematical Foundations and Applications, John Wiley  Sons, 1990; Я. Б. Песин, Теория размерности и динамические системы -- М.-Ижевск: Ин-т компьютерных исследований, 2002.

\end{remark} 


\begin{problem}[Модель Эрдёша-Реньи]
\label{sec:erdRenyi}
 Пусть есть конечное множество (в дальнейшем множество вершин) $V$, $\xi _{vv'} $ -- независимые с.в., занумерованные парами $\left\{v,v'\right\}\in V\times V$, $\vert V \vert = N$, $\xi _{vv'} \in \Be(p)$. 
Таким образом, можно задать абстрактный случайный граф на фиксированном множестве вершин. Покажите, что 
 
\begin{enumerate}
\item  При $p=\frac{1}{N^{1+\varepsilon } } $, $\varepsilon >0$ среднее число не изолированных вершин в случайном графе $o\left(N\right)$.
 
\item  При $p=\frac{1}{N^{1-\varepsilon } } $, $\varepsilon >0$ с вероятностью, близкой к 1, существует связная компонента порядка $N$ ($N \gg 1$).
\end{enumerate}
\end{problem}
 \begin{remark}
 См. брошюру В.А. Малышева \cite{27}, аналогично для следующей задачи.
 \end{remark}
 
 
\begin{problem} 
Рассматривается конфигурация спинов $\omega =\left\{x_{mn} \right\}$ (где $x_{mn} \in Be(p) $ - независимые с.в.) на двумерной решетке $\left\{(m,n) \in {\mathbb Z}^{2} \right\}$. Вершину $(m,n)$ назовем занятой, если $x_{mn} =1$. Соединим ребром все соседние (находящиеся на расстоянии 1) занятые вершины. Получится случайный граф $G=G\left(\omega \right)$. Назовем кластером графа $G$ максимальное подмножество $A$ вершин решетки такое, что для любых двух $v,v'\in A$ существует связывающий их путь по ребрам графа $G$. Докажите, что существует такое $0<\bar{p}<1$, что при $p<\bar{p}$ все кластеры конечны с вероятностью 1, а при $p>\bar{p}$ с положительной вероятностью есть хотя бы один бесконечный кластер.
 
 
\begin{ordre}
Покажите, что при достаточно малых значения $p$ вероятность события, что все кластеры конечны, равна 1. Покажите, что вероятность того, что кластер, содержащий начало координат и имеющий не менее $N$ вершин, не превосходит $\left(Cp\right)^{N} \mathop{\to }\limits_{N\to \infty } 0$, где $C$ -- некоторая константа. А значит и событие: бесконечный кластер содержит начало координат -- имеет нулевую вероятность.
\end{ordre}
 
\end{problem}

\begin{problem}

На некоторой реке имеется 6 островов (см. Рис. \ref{Fig:graphs_bridges.pdf}), соединенных между собой системой мостов. Во время летнего наводнения часть мостов была разрушена. При этом каждый мост разрушается с вероятностью ${1\mathord{\left/ {\vphantom {1 2}} \right. \kern-\nulldelimiterspace} 2} $, независимо от других мостов. Какова вероятность того, что после наводнения можно будет перейти с одного берега на другой, используя не разрушенные мосты?

\imgh{70mm}{graphs_bridges.pdf}{Схема мостов}

\end{problem}

\imgh{70mm}{guk.jpg}{``Иллюстрация к задаче Перколяция''. Анастасия Ковылина, 2014.}

\begin{problem}\Star(Перколяция)
В квадратном пруду (со стороной равной 1) 
выросли (случайным образом) $N\gg 1$ цветков лотоса, имеющих форму круга 
радиуса $r>0$. Назовем $r_N $ -- \textit{радиусом перколяции}, если с вероятностью не меньшей 0.99 не 
любящий воду жук сможет переползти по цветкам лотоса с северного берега на 
южный, не замочившись. Покажите, что $r_N \sim C / \sqrt{N}$. Оцените $C$.



\end{problem}

\begin{remark} 
См. Кестен Х. Теория просачивания для математиков. --М.: Мир, 1986, Grimmett G. Percolation. --М.:  Springer, 1999.
\end{remark}



\begin{problem}
В квадратной таблице задан процесс окрашивания ячеек: с вероятностью $p$ ячейка независимо окрашивается в черный цвет, иначе остается белой. Исследуем в такой системе размеры связных компонент из черных ячеек. При небольших значениях $p$ образуется много отделимых клеточных областей, в то время как при $p \sim 1$ c большой вероятностью получится несколько  компонент соизмеримых со всей таблицей. Установлено, что значение $p_c = 0.5927462\ldots$ является критическим, при превышении которого в результате раскраски  среднее значение размера компоненты растет вместе с увеличением размера таблицы (таблица достаточно большая). Определим $p(s)$ как плотность распределения площадей компонент, где площадь одной клетки равна $a$. Допустим, что $p(s)$ зависит только от параметров $a$ и $\Exp s$. Так как исследуемая плотность не должна зависеть от размерности, то 
\[
p(s) = c_1 f\left( \frac{s}{a}, \frac{a}{\Exp s} \right).
\]      
Изменим площадь одной клетки $a \to a/b$, тогда также ввиду независимости от размерности в области ($p < p_c$) плотность должна принять аналогичный вид  
\[
p(s) = c_2(b) f\left( \frac{s}{a/b}, \frac{a/b}{\Exp s} \right) = c_2 f\left( \frac{sb}{a}, \frac{a}{\Exp sb} \right) .
\]
Приблизившись к значению $p_c$, где $\Exp s \to \infty$, получим соотношение
\[
p(s) = c_2 f\left( \frac{sb}{a}, 0 \right) = \frac{c_2}{c_1} p(sb).
\]
Докажите, что распределение со свойством $p(s) = g(b) p(sb)$  является степенным ($p(s) \propto s^{-\beta}$).
\end{problem}

\begin{remark}
В контексте этой и последующих задач рекомендуем ознакомится с работами M. E. J. Newman, Power laws, Pareto distributions and Zipf’s law, Contemporary Physics, 2005; M. Mitzenmacher
A Brief History of Generative Models for Power Law and Lognormal Distributions, Internet Mathematics, vol 1, No. 2. 
\end{remark}

\begin{problem}
В популяции каждый индивидуум имеет  порог чувствительности к депрессии, являющийся независимой от других индивидуумов с.в. 
На популяцию периодически нагоняют депрессию заданного уровня, после чего индивидуумы с порогом ниже уровня покидают популяцию, а на их место генерируются новые индивидуумы. Используя рассуждения предыдущей задачи, установите степенное распределение для количества заменяемых индивидуумов за одно воздействие депрессии.     
\end{problem}

\begin{problem}
Докажите, что итерационные интервалы между рекордными значениями при генерации независимых одинаково распределенных случайных величин имеют степенное распределение $p(x) \propto \frac{1}{x}.$ Элемент в последовательности называется  рекордным если все   предшествующие элементы последовательности меньше него.
\end{problem}

\begin{problem}
Типичность возникновения степенного распределения в ряде естественных манипуляций (к примеру, суммирование или умножение членов последовательности) со случайными величинами частично обусловлена схожестью с лог-нормальным распределением (см. задачу \ref{lognorm} из раздела \ref{zb4}) с плотностью
\[
p(x) = - \frac{(\ln x)^2}{2 \sigma^2} + \left( \frac{m}{\sigma^2} - 1 \right) \ln x  - \frac{m^2}{2 \sigma^2}.
\]    
При каких значениях $ \sigma$ распределения  похожи на большем участке значений $x$?
Покажите, что для произведения случайных величин характерна сходимость к лог-нормальному распределению. Докажите это же свойство для длин отрезков, получающихся в результате деления единичного отрезка с равномерным многократным выбором точек раздела.     
\end{problem}



\begin{problem}[Growth model with preferential attachment; J. Hopcroft, R. Kannan ``Computer science Theory for the Information Age'']

Рассмотрим следующую модель роста графа: пусть в каждый момент времени в 
графе добавляется новая вершина, и с вероятностью $\delta $ добавляется новое 
ребро, соединяющее новую вершину со случайно выбранной имеющейся вершиной 
(вероятность выбора пропорционально степени вершины).

Обозначим через $d_i (t)$ -- среднее значение степени i-й вершины в момент 
времени $t$. Покажите, что для $d_i (t)$ справедливо следующее 
дифференциальное уравнение:
\[
\frac{\partial }{\partial t}d_i (t)=\frac{d_i (t)}{2t}
\]
с условием $d_i (i)=\delta $. Откуда следует, что $d_i (t)=\delta \sqrt 
{\frac{t}{i}} $.

Покажите, что в этой модели справедлив степенной закон для степени вершин, а 
именно, что функция плотности распределения для степени вершины в точке $d$ 
равна $2\frac{\delta ^2}{d^3}$.
\end{problem}





 
\begin{problem}[Обобщенная схема размещений]
\noindent
\begin{enumerate}
\item  Пусть для целочисленных  неотрицательных с.в. $\eta _1,\ldots,\eta _N$ существуют независимые одинаково распределенные с.в. 
$\xi_1,\ldots,\xi_N$ такие, что
\[ 
\PR\left( {\eta _1 =k_1,\ldots,\eta _N =k_N } \right) =  \]\[ \PR\left( {\left. {\xi _1 
=k_1 ,\ldots,\xi _N =k_N } \right|\xi _1 +\ldots+\xi _N =n} \right) \;\;\;\; (*) \]
 
Введем независимые одинаково распределенные с.в. $\xi _1^{\left( r \right)} 
$,{\ldots},$\xi _N^{\left( r \right)} $, где $r$ целое неотрицательное число 
и
\[
\PR\left( {\xi _1^{\left( r \right)} =k} \right)=\PR\left( {\left. {\xi _1 =k} 
\right|\xi _1 \ne r} \right),
\quad
k=0,1,\ldots
\]
Пусть $p_r = \PR\left( {\xi _1 =r} \right)$ и $S_N =\xi _1 +\ldots+\xi _N$,
$S_N^{\left( r \right)} =\xi _1^{\left( r \right)} +\ldots+\xi _N^{\left( r \right)} $. Пусть $\mu _r \left( {n,N} \right)$ -- число с.в. $\eta _1$,{\ldots},$\eta_N $, принявших значение $r$. Покажите, что с.в. типа $\mu_r \left( {n,N} \right)$ можно изучать с помощью \textit{обобщенной схемы размещений}: для любого $k=0,\ldots,N$
\[
\PR\left( {\mu _r \left( {n,N} \right)=k} \right)=C_n^k p_r^k \left( {1-p_r } 
\right)^{N-k}\frac{\PR\left( {S_{N-k}^{\left( r \right)} =n-kr} 
\right)}{\PR\left( {S_N =n} \right)}.
\]
Напомним, что в классической схеме размещений $n$ различных частиц по $N$ 
различным ячейкам было доказано, что распределение заполнений ячеек $\eta _1 
$,{\ldots},$\eta _N $ имеет вид (\textit{полиномиальное распределение}):
\[
\PR\left( {\eta _1 =k_1 ,\ldots,\eta _N =k_N } \right)=\frac{n!}{k_1!\ldots k_N! N^n},
\]
где $k_1,\ldots,k_N$ -- неотрицательные целые числа такие, что  $k_1 +\ldots+k_N =n$. Если положить $\xi_1,\ldots,\xi_N \in \Po \left( \lambda 
\right)$ -- независимые с.в. ($\lambda >0$ - произвольно), то получим (*).
 
\item ** Дан случайный граф (модель Эрдеша--Реньи) $G\left( {n,\;p} \right)$. Пусть 
$p=c\frac{\ln n}{n}$. Покажите, что при $n\to\infty$ и $c>1$ граф $G\left( {n,\;p} \right)$ 
почти наверное связен, а при $c>1$ почти наверное не связен.
 
 
\end{enumerate}
\begin{remark}
См. монографию Колчин В.Ф. Случайные графы. -- М.:~Физматлит, 2004. Пункт б) подробно разобран в монографии Алона--Спенсера. Мы также рекомендуем смотреть популярные тексты А.М.~Райгородского по тематике случайных графов и их приложений.
\end{remark}
\end{problem}


\begin{problem}[Устойчивые системы большой размерности; В.И. Опойцев]
Из курсов функционального анализа и вычислительной математики хорошо известно, что если спектральный радиус матрицы 
$A=\| a_{ij}\|_{i,j=1}^{n}$ меньше единицы: $\rho(A)<1$, то итерационный процесс ${\vec x}^{n+1}=A{\vec x}^n +{\vec b}$ 
(СОДУ $\dot{\vec x}=-{\vec x}+A{\vec x}+{\vec b}$), вне зависимости от точки старта ${\vec x}^0$, 
сходится к единственному решению уравнения ${\vec x}^*=A{\vec x}^*+{\vec b}$. 
Скажем, если $\| A\|=\max\limits_{i} \sum\limits_j |a_{ij}|<1$, то и $\rho(A)<1$ (обратное, конечно, не верно). Предположим, что 
существует такое $\varepsilon>0$, что 
$$
\frac{1}{n}\sum\limits_{i,j} |a_{ij}|<1-\varepsilon . 
\quad (S)
$$
Очевидно, что отсюда не следует: $\rho(A)<1$. 
Тем не менее, введя на множестве матриц, удовлетворяющих условию $(S)$, равномерную меру, покажите, что относительная мера тех матриц 
(удовлетворяющих условию $(S)$), для которых спектральный радиус не меньше единицы, стремится к нулю 
с ростом $n$ ($\varepsilon$ --- фиксировано и от $n$ не зависит). 
\end{problem}
\begin{ordre}

1. Покажите, что  достаточно рассматривать матрицы с неотрицательными элементами. 

2.  Покажите, что достаточно доказать утверждение задачи на множестве матриц, удовлетворяющих условию 
$$
\frac{1}{n}\sum\limits_{i,j} a_{ij}=1-\varepsilon . 
\quad (SE)
$$

3. Далее положим $a_{ij}\in \mathrm{Exp} \bigl( n/(1-\varepsilon)\bigr)$ --- независимые одинаково распределенные случайные величины. Покажите, что при $n\to\infty$ распределение элементов случайной матрицы $A=\| a_{ij}\|_{i,j=1}^n$ 
будет сходиться к равномерному распределению на множестве матриц, удовлетворяющих ($SE$). 

4. Введя обозначения
$P_n={\mathbb P}(\| A\|\ge 1)\ge {\mathbb P}(\rho(A)\ge 1)$, воспользуйтесь неравенством Чебышёва

$$
P_n\le n {\mathbb P}\Bigl( \sum\limits_j a_{1j}\ge 1 \Bigr)=n {\mathbb P}\Bigl( X\ge 1 \Bigr)\le 
n {\mathbb P}\Bigl( |X-(1-\varepsilon)|\ge \varepsilon \Bigr)=
$$
$$
=n {\mathbb P}\Bigl( |X-{\mathbb E}X|\ge \varepsilon \Bigr)\le \frac{n}{\varepsilon^4} {\mathbb E}(X-{\mathbb E}X)^4=
O\Bigl(\frac{1}{n}\Bigr) \xrightarrow{n\to\infty} 0 . 
$$
\end{ordre}




\begin{problem}[Вероятностное доказательство формулы Эйлера] 
Пусть X -- 
целочисленная случайная величина с распределением

\[\PR\left( {X=n} \right)=\frac{1}{\varsigma (s)n^s},\;\]
\noindent где $\varsigma 
(s)=\sum\limits_{n\in {\mathbb N}} {n^{-s}} ,\quad s>1$.

Пусть $1<p_1 <p_2 <p_3 <\ldots $ -- простые числа, и пусть $A_k $ -- событие 
= {\{}X делится на $p_k ${\}}.

\begin{enumerate}

\item Найдите $\PR\{A_k \}$ и покажите, что события $A_1 ,A_2 ,\ldots $ 
независимы;

\item Покажите, что
\begin{center}
$\prod\limits_{k=1}^\infty {(1-p_k ^{-s})} =\frac{1}{\varsigma (s)}$ (формула 
Эйлера).
\end{center}

\end{enumerate}
\end{problem}

\begin{problem}\Star(Статистика теоретико-числовых функций)
\label{sec:z_func_riman}
Довольно часто 
вероятностные соображения (например, независимость) используются в теории 
чисел не совсем строго, но зато весьма часто они позволяют угадать 
правильный ответ. Поясним сказанное, пожалуй, наиболее популярным примером (Дирихле, 1849 г.) 
из книги  Марка Каца 
``Статистическая независимость в теории вероятностей, анализе и теории 
чисел''. М.: ИЛ, 1963.

Пусть $A$ -- некоторое множество положительных целых чисел. Обозначим через 
$A\left( n \right)$ количество тех его элементов, которые содержатся среди 
первых $n$ чисел натурального ряда. Если существует предел $\mathop {\lim 
}\limits_{n\to \infty } {A\left( n \right)} \mathord{\left/ {\vphantom 
{{A\left( n \right)} n}} \right. \kern-\nulldelimiterspace} n=\PR\left( A 
\right)$, то он называется плотностью $A$. К сожалению, вероятностная мера 
$\PR\left( A \right)$ не является вполне аддитивной (счетно-аддитивной).

Рассмотрим целые числа, делящиеся на простое число $p$. Плотность множества 
таких чисел, очевидно, равна $1 \mathord{\left/ {\vphantom {1 p}} \right. 
\kern-\nulldelimiterspace} p$. Возьмем теперь множество целых чисел, которые 
делятся одновременно на $p$ и $q$ ($q$ -- другое простое число). Делимость 
на $p$ и $q$ эквивалентна делимости на $pq$, и, следовательно, плотность 
нового множества равна $1 \mathord{\left/ {\vphantom {1 {pq}}} \right. 
\kern-\nulldelimiterspace} {pq}$.Так как $1 \mathord{\left/ {\vphantom {1 
{pq}}} \right. \kern-\nulldelimiterspace} {pq}=\left( {1 \mathord{\left/ 
{\vphantom {1 p}} \right. \kern-\nulldelimiterspace} p} \right)\cdot \left( 
{1 \mathord{\left/ {\vphantom {1 q}} \right. \kern-\nulldelimiterspace} q} 
\right)$, то мы можем истолковать это так: ``события'', заключающиеся в 
делимости на $p$ и $q$, независимы. Это, конечно, выполняется для любого 
количества простых чисел.

Поставим теперь задачу посчитать долю несократимых дробей или, другими 
словами, ``вероятность'' несократимости дроби (фиксируется знаменатель дроби 
$n$, а затем случайно, с равной вероятностью $1 \mathord{\left/ {\vphantom 
{1 n}} \right. \kern-\nulldelimiterspace} n$ выбирается любое число от 1 до 
$n$ в качестве числителя, и подсчитывается доля случаев, в которых 
полученная дробь оказывалась несократимой) в следующем (чезаровском) смысле 
(здесь и далее индекс $p$ может пробегать только простые числа):
\[
\mathop {\lim }\limits_{N\to \infty } \frac{1}{N}\sum\limits_{n=1}^N 
{\frac{\# \left\{ {k<n:\;\;\text{Н.О.Д.}\left( {n,k} \right)=1} \right\}}{n}} 
=\mathop {\lim }\limits_{N\to \infty } \frac{1}{N}\sum\limits_{n=1}^N 
{\frac{\phi \left( n \right)}{n}} = \]\[\mathop {\lim }\limits_{N\to \infty } 
\frac{1}{N}\sum\limits_{n=1}^N {\prod\limits_p {\left( {1-\frac{\rho _p 
\left( n \right)}{p}} \right)} } ,
\]
где $\phi \left( n \right)$ -- функция Эйлера, $\rho _p \left( n 
\right)=\left\{ {\begin{array}{l}
 1,\quad n\mbox{ делится на }p, \\ 
 0,\quad \mbox{иначе}. \\ 
 \end{array}} \right.$.

\noindent Согласно введенному выше определению плотности:

\[
\Exp\left\{ {\prod\limits_{p\le p_k } {\left( {1-\frac{\rho _p \left( n 
\right)}{p}} \right)} } \right\}=\prod\limits_{p\le p_k } {\Exp\left\{ {\left( 
{1-\frac{\rho _p \left( n \right)}{p}} \right)} \right\}} 
=\prod\limits_{p\le p_k } {\left( {1-\frac{1}{p^2}} \right)} .
\]
С учетом этого хочется написать следующее:
\[
\mathop {\lim }\limits_{N\to \infty } \frac{1}{N}\sum\limits_{n=1}^N 
{\frac{\phi \left( n \right)}{n}} =\Exp\left\{ {\frac{\phi \left( n 
\right)}{n}} \right\}=\Exp\left\{ {\prod\limits_p {\left( {1-\frac{\rho _p 
\left( n \right)}{p}} \right)} } \right\}\mathop =\limits^? 
\]
\[
\mathop =\limits^? \prod\limits_p {\Exp\left\{ {\left( {1-\frac{\rho _p \left( 
n \right)}{p}} \right)} \right\}} =\prod\limits_p {\left( {1-\frac{1}{p^2}} 
\right)} =\frac{1}{\varsigma \left( 2 \right)}=\frac{6}{\pi ^2}.
\]
Будь введенная вероятностная мера, по которой считается это математическое 
ожидание, счетно-аддитивной, то можно было бы поставить точку, получив 
ответ. Однако, это не так. Несмотря на правильность ответа, 
приведенное выше рассуждение не может считаться доказательством. Впрочем, 
часто вероятностные рассуждения удается пополнить, используя их в качестве 
основы. Так в разобранном нами примере все сводится к обоснованию равенства 
``$?$''.

Легко понять, что полученный ответ несет определенную информацию о 
статистических свойствах функции Эйлера.

В теории чисел такого типа задачи занимают крайне важное место. Достаточно 
сказать, что гипотеза Римана ``на миллион'' о распределении нетривиальных 
нулей дзета-функции Римана $\varsigma \left( z \right)$ равносильна 
следующему свойству функции Мёбиуса $\mu(n)$:

\begin{center}
$\left| {\sum\limits_{n=1}^N {\mu \left( n \right)} } \right|\le \sqrt N $ 
(Одлыжко--Риэль),
\end{center}
где $\mu(n)$ определяется уравнением
\[
\sum\limits_{d \vert n} \mu(d) = 
\begin{cases}
1, & n = 1; \\
0, & n > 1.
\end{cases}
\quad
\Leftrightarrow
\quad
\mu(n) = 
\begin{cases}
(-1)^r, & n = p_1\ldots p_r; \\
0, & n \;  \mod \; p^2 = 0.
\end{cases}
\]
Что в свою очередь (Х.М. Эдвардс) в определенном смысле ``завязано'' на 
случайности последовательности $\left\{ {\mu \left( n \right)} 
\right\}_{n\in {\mathbb N}} $.

Используя указанный выше формализм, найдите долю чисел натурального ряда, 
``свободных от квадратов'', т.е. не делящихся на квадрат любого простого 
числа.

\end{problem}

\begin{remark} 
Известный российский математик Владимир Игоревич Арнольд 
последние десять лет жизни активно развивал описанное направление, которое 
он называл ``Экспериментальной математикой'' (помимо популярных книжек и 
статей, осталось и несколько видеолекций на эту тему на mathnet.ru
с выступлениями на семинаре МИАНа, в летней школе Современная математика и 
на мехмате). Получая схожим образом ``ответы'', их далее можно проверять, 
ставя численные эксперименты. При современных возможностях вычислительных 
машин, можно отслеживать логарифмические функции в асимптотике (т.е. 
проверять гипотезы с логарифмами), но не с повторными логарифмами, которые 
также как и в случайных процессах встречаются в теории чисел. Таким образом, 
у В.И. Арнольда получалось довольно много теорем (десятки, а возможно, даже сотни). 
Часть теорем, конечно, была известна ранее (см., например, книгу Карацубы 
А.А. Основы аналитической теория чисел. М.: Наука, 1975), но удавалось 
получать и новые формулировки.

Следующий пример, взятый из другой книги М. Каца \cite{20}, демонстрирует, что отмеченным выше 
способом можно получить и неверный результат.

\begin{example}[Из журнала Nature, 1940] Из изложенного выше следует, что количество целых чисел, не превосходящих $N$ и не делящихся ни на одно из простых 
чисел $p_1 $, $p_2 $, {\ldots}, $p_k $, равно приблизительно 
$N\prod\limits_{j=1}^k {\left( {1-\frac{1}{p_j }} \right)} $. Рассмотрим 
теперь количество целых чисел, не превосходящих $N$ и не делящихся ни на одно из 
простых чисел, меньших $\sqrt N $. Такими числами могут быть только простые 
числа, лежащие между $\sqrt N $ и $N$, число которых
\[
\pi \left( N \right)-\pi \left( {\sqrt N } \right)\sim N\prod\limits_{p_j 
<\sqrt N } {\left( {1-\frac{1}{p_j }} \right)} .
\]
Но из теории чисел известно, что $\pi \left( N \right)\sim N \mathord{\left/ 
{\vphantom {N {\ln N}}} \right. \kern-\nulldelimiterspace} {\ln N}$ и
\[
\prod\limits_{p_j <\sqrt N } {\left( {1-\frac{1}{p_j }} \right)} \sim {\exp 
\left( {-\gamma } \right)} \mathord{\left/ {\vphantom {{\exp \left( {-\gamma 
} \right)} {\ln \sqrt N }}} \right. \kern-\nulldelimiterspace} {\ln \sqrt N 
}={2\exp \left( {-\gamma } \right)} \mathord{\left/ {\vphantom {{2\exp 
\left( {-\gamma } \right)} {\ln N}}} \right. \kern-\nulldelimiterspace} {\ln 
N},
\]
где $\gamma $ -- константа Эйлера. Следовательно, $2\exp \left( {-\gamma } 
\right)=1$. Пришли к неверному соотношению!
\end{example}

Интересно в этой связи отметить также вероятностный способ получения 
правильной асимптотической формулы $\pi \left( N \right)\sim N 
\mathord{\left/ {\vphantom {N {\ln N}}} \right. \kern-\nulldelimiterspace} 
{\ln N}$ для количества простых чисел, не превосходящих $N$, приведенный в книге 
Куранта Р., Роббинса Г. Что такое математика. М.: МЦНМО, 2007.

В заключение заметим, что применение вероятностных соображений в теории 
чисел продолжает привлекать ведущих математиков и по сей день (см. 
выступление Я.Г. Синая  на mathnet.ru).


\end{remark}

\begin{problem}\Star(Парадокс Банаха--Тарского)
С. Банах показал, что если предполагать только аддитивность меры, то в одно- и двумерном пространствах любое ограниченное множество становится измеримым (имеет длину и площадь). Таким образом, в одно- и двумерном случаях равномерное распределение можно задать на любом (ограниченном) множестве, если от вероятности требовать только аддитивность. Приведите пример, показывающий, что в трехмерном пространстве это сделать невозможно. 
\end{problem}
\begin{remark}
Базируясь на аксиоме выбора, шар в трехмерном пространстве допускает такое разбиение на конечное число непересекающихся множеств, из которых можно составить передвижением (как твердых тел = перенос + поворот) два шара того же радиуса (см., например, В. Босс, т. 6, 12, 16 \cite{2013}).
\end{remark}

\begin{problem}

Пусть $\xi _{ij}^{n} $, $1\le i,j\le n$, $n=1,2,\ldots $ -- совокупность одинаково распределенных случайных величин, удовлетворяющих условиям:

\begin{enumerate}
\item \label{zero}  при каждом $n$ случайные величины $\xi _{ij}^{n} $, $1\le i\le j\le n$ независимы;

\item \label{first}  матрица $\left(A^{n} \right)_{ij} =\frac{1}{2\sqrt{n} } \left(\xi _{ij}^{n} \right)$ симметричная, т.е. $\xi _{ij}^{n} =\xi _{ji}^{n} $;

\item \label{second} случайная величина $\xi _{ij}^{n} $ имеет симметричное распределение, т.е. для всякого борелевского множества $B\in B({\mathbb R})$ выполнено равенство $\PR\left(\xi _{ij}^{n} \in B\right)=\PR\left(\xi _{ij}^{n} \in -B\right)$;

\item \label{third} все моменты с.в. $\xi _{ij}^{n} $ конечны, т.е. $\Exp\left[\left(\xi _{ij}^{n} \right)^{k} \right]<\infty $ при всех $k\ge~1$, причем дисперсия равна единице, $\Var\left[\xi _{ij}^{n} \right]=1$.
\end{enumerate}

Рассмотрим дискретную вероятностную меру $\mu ^{n} $ собственных значений $\lambda _{1}^{(n)} ,\ldots ,\lambda _{n}^{(n)} $ случайной матрицы $A^{n} $: для всякого борелевского множества $B\in B({\mathbb R})$
\begin{center}
$\mu ^{n} (B)=\frac{1}{n} \sum _{i=1}^{n}I\left\{\lambda _{i}^{(n)} \in B\right\} $. 
\end{center}
Ясно, что такая мера сама по себе случайна, так как зависит от собственных значений случайной матрицы. Пусть $L_{k}^{n} $ -- $k$-й момент меры $\mu ^{n} $ матрицы $A^{n} $, т.е. 
\begin{center}
$L_{k}^{n} =\int _{-\infty }^{\infty }x^{k} \mu ^{n} (dx) =\frac{1}{n} \sum _{i=1}^{n}\left(\lambda _{i}^{(n)} \right)^{k}  $ (также случайная величина).
\end{center}
Докажите, что $\mathop{\lim }\limits_{n\to \infty } \Exp\left[L_{k}^{n} \right]=m_{k} $, $\mathop{\lim }\limits_{n\to \infty } \Var\left[L_{k}^{n} \right]=0$, где
\begin{center}
 $m_{k} =\frac{2}{\pi } \int _{-1}^{1}\lambda ^{k} \sqrt{1-\lambda ^{2} } d\lambda  $,
\end{center}
то есть случайные меры собственных значений в некотором смысле сходятся к неслучайной мере на действительной прямой с плотностью, задаваемой \textit{полукруговым законом Вигнера}: 

\[p(\lambda )=\left\{\begin{array}{cc} {\frac{2}{\pi } \int _{-1}^{1}\sqrt{1-\lambda ^{2} } d\lambda , } & {-1\le \lambda \le 1;} \\ {0,} & {\text{иначе.}} \end{array}\right. \] 

\noindent Применив неравенство Чебышёва, покажите, что $L_{k}^{n} \mathop{\to }\limits_{n\to \infty }^{p} m_{k} .$


\end{problem}


\begin{remark}
Если $\eta ^{n} $ -- последовательность мер, моменты которых сходятся к соответствующим моментам меры $\eta $, то (при дополнительных условиях на рост моментов мер $\eta ^{n} $) сами эти меры слабо сходятся к мере $\eta $.

См. по данной тематике Мета М.Л., Случайные матрицы -- М.:  МЦНМО, 2012, а также T. Tao, Topics in random matrix theory.Graduate Studies in Mathematics, vol. 132, 2012. 
\end{remark}

\begin{ordre} (См. Коралов-Синай \cite{7}). Покажите, что 

\[\Exp\left[L_{k}^{n} \right]=\frac{1}{n} \Exp\left[\sum _{i=1}^{n}\left(\lambda _{i}^{(n)} \right)^{k}  \right]=\frac{1}{n} \Exp\left[\mathrm{tr}\left[\left(A^{n} \right)^{k} \right]\right] = \]\[ \frac{1}{n} \left(\frac{1}{2\sqrt{n} } \right)^{k} \Exp\left[\sum _{i_{1} ,\ldots ,i_{k} =1}^{n}\xi _{i_{1} i_{2} }^{n} \xi _{i_{2} i_{3} }^{n} \cdots \xi _{i_{k} i_{1} }^{n}  \right].\] 
\noindent где $\lambda _{i}^{(n)} $ -- собственные значения матрицы $A^{n} $ и $\mathrm{tr}\left[\left(A^{n} \right)^{k} \right]$ - след ее $k$-й степени.
С учетом условий \[\Exp\left[\xi _{i_{1} i_{2} }^{n} \xi _{i_{2} i_{3} }^{n} \cdots \xi _{i_{k} i_{1} }^{n} \right]=\prod _{\begin{array}{l} {\small J_k} \end{array}}\Exp\left[\left(\xi _{ij}^{n} \right)^{p(i.j)} \right], \]\[J_k = \{ i,j: i \leq j, \;
\sum p(i,j) = k\}. \] причем в силу условия \ref{second} нечетные моменты равны нулю: $$\Exp\left[\left(\xi _{ij}^{n} \right)^{p(i.j)} \right]=0,$$ если $p(i,j)$ -- нечетное. Таким образом, для четных моментов $k=2s$ имеем:

\[\Exp\left[L_{2r}^{n} \right]=\frac{1}{2^{2r} n^{r+1} } \Exp\left[\sum _{i_{1} ,\ldots ,i_{2r} =1}^{n}\xi _{i_{1} i_{2} }^{n} \xi _{i_{2} i_{3} }^{n} \cdots \xi _{i_{2r} i_{1} }^{n}  \right]= \]\[ \frac{1}{2^{2r} n^{r+1} } \sum _{i_{1} ,\ldots ,i_{2r} =1}^{n}\prod _{\begin{array}{c} {\scriptsize J_r }\end{array}}\Exp\left[\left(\xi _{ij}^{n} \right)^{2p(i.j)} \right]  .\] 

 Задачу можно свести к комбинаторному подсчету соответствующих путей (сопоставленных ненулевым слагаемым $\xi _{i_{1} i_{2} }^{n} \xi _{i_{2} i_{3} }^{n}, \ldots, \xi _{i_{2r} i_{1} }^{n} $) на множестве $\left\{1,2,\ldots ,n\right\}$, где каждое ребро (петли при этом не запрещаются) проходится четное число раз (без учета направления). Совокупность таких путей можно разбить на два класса: пути, накрывающие дерево из $r$ ребер, каждое из которых проходится дважды и остальные пути, для которых либо есть петли, либо циклы, либо есть ребра, через которые путь проходит по крайней мере четыре раза. Для путей второго класса (с учетом условия \ref{third}) получите оценку: \[
 \Exp\left[
    \sum  \limits_{ i_{1} < \ldots < i_{2r} } \xi _{i_{1} i_{2} }^{n} \xi _{i_{2} i_{3} }^{n} \cdots           \xi _{i_{2r} i_{1} }^{n}  
\right]\le C_{r} n^{r} ,
 \] \noindent где $C_{r} $ - некоторая константа.

Как следствие, вклад таких путей в искомое математическое ожидание стремится к нулю при $n\to \infty $: \[\mathop{\lim }\limits_{n\to \infty } \frac{1}{2^{2r} n^{r+1} } C_{r} n^{r} k_{r} =0.\] Для путей первого класса в силу условия \ref{third} на дисперсию получите $\Exp \left(\xi _{i_{1} i_{2} }^{n} \xi _{i_{2} i_{3} }^{n} \cdots \xi _{i_{2r} i_{1} }^{n}\right) =1$. Покажите, что число путей первого класса равно \[\frac{n!(2r)!}{(n-r-1)!r!(r+1)!}. \] Для этого сопоставьте таким путям неотрицательные траектории одномерного симметричного случайного блуждания: $\left(\omega _{0} \omega _{1} \ldots \omega _{2r} \right)$, где $\omega _{0} =\omega _{2r} =0$, $\omega _{i} \ge 0$ при всех $i=1,\ldots ,2r$. Каждой фиксированной такой траектории $\left(\omega _{0} \omega _{1} \ldots \omega _{2r} \right)$ соответствует $n(n-1)\ldots (n-r)$ допустимых путей первого класса. Число неотрицательных траекторий с начальной и конечной нулевой точкой равно $\frac{(2r)!}{r!(r+1)!} $ (см. задачу \ref{sec:katalan} из раздела \ref{genF}).
Итак, 

\[\mathop{\lim }\limits_{n\to \infty } \Exp\left[L_{2r}^{n} \right]=\mathop{\lim }\limits_{n\to \infty } \frac{1}{2^{2r} n^{r+1} } \frac{n!(2r)!}{(n-r-1)!r!(r+1)!} =\frac{(2r)!}{2^{2r} r!(r+1)!} =m_{2r} .\] 
Утверждение касательно дисперсии $L_{k}^{n} $ доказывается аналогично.

\end{ordre}