%\subsection{Вероятностное пространство, $\sigma$–алгебра}

\begin{problem}
Показать, что борелевская $\sigma$-алгебра в ${\mathbb R}^1$, содержащая все числовые промежутки вида $[a,b)$, 
содержит все промежутки вида $(a,b)$, $(a,b]$, $[a,b]$ и отдельные точки прямой. 

\begin{ordre}
Учесть свойство замкнутости $\sigma$-алгебры относительно операций 
объединения, пересечения и вычитания. 
\end{ordre}

\end{problem}


\begin{problem}
Пусть $\Omega = [a,b]\times...\times[a,b]$, $\mathcal{F}$ --- $\sigma$–алгебра, содержащая все отрезки 
$[\alpha_1,\beta_1]\times...\times[\alpha_n,\beta_n]$ $(a \leqslant \alpha_1 < \beta_1 \leqslant b, \ldots, a \leqslant \alpha_n < \beta_n \leqslant b)$ с вероятностной мерой 
${\mathbb P}\{ \omega_1  \in [\alpha_1,\beta_1], ..., \omega_n  \in [\alpha_n,\beta_n]  \}=\dfrac{\mes[\alpha_1, \beta_1]}{\mes[a,b]}  \ldots  \dfrac{\mes[\alpha_n, \beta_n]}{\mes[a,b]}$. 
Показать, что 
\begin{enumerate}
\item ${\mathbb P}\{ \omega_1=c=\const\}=0$; 
\item ${\mathbb P}\{ \omega_1=\omega_2\}=0$;
\item найти вероятность ${\mathbb P}\{ \omega_1 \leq \omega_3 \leq \omega_2\}$. 
\end{enumerate}

\end{problem}


\begin{problem}
Число элементарных событий некоторого вероятностного пространства равно $n$. Указать минимальное и максимальное возможные значения 
для числа событий. 
\end{problem}

\begin{problem}
Может ли число всех событий какого-либо вероятностного пространства быть равным $129$; $130$; $128$? 
\end{problem}


\begin{problem}
\label{L_extension}
Два лыжника условились о встрече между $10$ и $11$ часами утра у подножья тягуна, причем договорились ждать друг друга не более $10$ минут, чтобы не замерзнуть. Считая, что момент прихода на встречу каждым выбирается ``наудачу'' в пределах указанного часа, найти вероятность того, что встреча состоится. 
\end{problem}

\begin{remark}
В подобного рода задачах, где явно не определена вероятностная тройка $\langle \Omega, \mathcal{F}, \PR \rangle$ в общем случае следует придерживаться такой последовательности действий: 
\begin{enumerate}
\item Задать вероятностную меру на множествах простой структуры;
\item Продолжить вероятностную меру на множества более сложной структуры, рассматриваемые в задаче;  
\item Проверить условия, гарантирующие $\sigma$-аддитивность продолженной меры.   
\end{enumerate}
Рассмотрим каждый из этапов в отдельности, придерживаясь контекста задачи о встрече. Возьмем $\Omega = [0,1] \times [0,1]$. В качестве множеств простой структуры $\text{П}$ удобно воспользоваться \textit{полукольцом с единицей} $\Leftrightarrow$ $C_i, A, B \in \text{П}$, $C_i \cap C_j  = \emptyset$:
\[
\Omega \in \text{П},  \quad A\cap B \in \text{П}, \quad A_{\supset B} \setminus B = \bigcup \limits_{i=1}^n C_i.
\]
Зададим меру $\PR$ на $\text{П}$, $\Omega \subseteq \mathbb{R}^n$ при помощи формулы 
\[
\PR\{ [x_1^{(0)}, x_1^{(1)})\times \ldots \times [x_n^{(0)},  x_n^{(1)})  \} = 
\]
\[\tag{1} =
\begin{cases}
\sum \limits_{\alpha_1, \ldots, \alpha_n} (-1)^{\sum \alpha_i} F\left(x_1^{(\alpha_1)} ,\ldots, x_n^{(\alpha_n)}\right),  \; n \; \text{четно}, \\
\sum \limits_{\alpha_1, \ldots, \alpha_n} (-1)^{\sum \alpha_i - 1} F\left(x_1^{(\alpha_1)} ,\ldots, x_n^{(\alpha_n)} \right) ,  \; n \; \text{нечетно},\\
\end{cases}
\]
тогда выполнимость свойств аддитивной вероятностной меры $\PR$ равносильна  выполнимости свойств (1-6) для функции $F$, а $\sigma$-аддитивность равносильна свойству 7: 

\begin{enumerate}
\item[1.] Неотрицательность;
\item[2.] Монотонность по каждой координате;
\item[3.] $F(x_1, \ldots, x_n) \to 0$ при $x_i \to -\infty$;
\item[4.] $F(x_1, \ldots, x_n) \to 1$ при $x_1, \ldots, x_n  \to \infty$;
\item[5.] $\PR\{ [x_1^{(0)}, x_1^{(1)})\times \ldots \times [x_n^{(0)},  x_n^{(1)})  \} \geq 0$;
\item[6.] $F(x_1, \ldots, x_n) \to  F(x_1, \ldots x_{i-1}, x_{i+1}  \ldots , x_n)$ при $x_i \to \infty$;
\item[7.] Непрерывность слева в точках разрыва.
\end{enumerate} 

Заметим также, что если $\PR$ абсолютно непрерывна, то существует функция плотности распределения  $f(x_1 ,\ldots, x_n ) \geq 0$, что гарантирует выполнение свойств (1-7), т.е. $\PR$ является  $\sigma$–аддитивной вероятностной мерой на  $\text{П}$.
Продолжим вероятностную меру на класс множеств 
\[\mu \tilde{A} \supset \{t_1, t_2 \in \Omega: |t_1 - t_2| =\tau \}\]
посредством \textit{пополнения по Лебегу}:
\[
\PR^{*}(B) = \inf \limits_{B \subseteq \bigcup \limits_{i=1}^{\infty} A_i} \sum \limits_{i=1}^{\infty} \PR(A_i)
\]
 \[
 A \in \mu \tilde{A} \Leftrightarrow  \forall \varepsilon >0 \; \exists A_\varepsilon \in \tilde{A}:  \PR^{*}(A_\varepsilon \triangle A) < \varepsilon,
 \] 
где $\tilde{A}$ --  минимальная по мощности алгебра, содержащая $\text{П}$ ($B \in \tilde{A} \Leftrightarrow B = \bigcup \limits_{i=1}^n C_i, \; C_i \cap C_j  = \emptyset$).

Предложенное продолжение сохраняет $\sigma$-аддитивность и другие свойства меры.  

\end{remark}

\begin{problem}
Имеется монетка с вероятностью выпадение орла $p=1/2$. Покажите, что событие $A$, заключающееся в том, что отношение числа успехов к
общему числу бросаний $n$ стремится к $1/2$ при $n \to \infty$, принадлежит $\mu \tilde{A}$ (см. замечание к задаче \ref{L_extension}). Считать бросания независимыми.
\end{problem}

\begin{ordre}
Воспользуйтесь вспомогательными событиями 
\[
A_n^\varepsilon = \Bigg\{ (x_1, x_2, x_3, \ldots):\Bigg| \frac{1}{n}\sum \limits_{i=1}^n x_i -\frac{1}{2} \Bigg| \geq \varepsilon \Bigg\}.
\]
\end{ordre}

\begin{remark}
При $\Omega = \mathbb{R}^\infty$ аналогом выражения (1) в задаче \ref{L_extension} будет
\[
\PR\{ [x_{k_1}^{(0)}, x_{k_1}^{(1)})\times \ldots \times [x_{k_n}^{(0)},  x_{k_n}^{(1)} )\times \prod \limits_{k \in \mathbb{N} \setminus \{k_1, \ldots, k_n\} }  \mathbb{R}_k  \} = 
\]
\[
\begin{cases}
\sum \limits_{\alpha_1, \ldots, \alpha_n} (-1)^{\sum \alpha_i} F_{k_1 \ldots k_n}\left(x_{k_1}^{(\alpha_1)} ,\ldots, x_{k_n}^{(\alpha_n)} \right) ,  \; n \; \text{четно}, \\
\sum \limits_{\alpha_1, \ldots, \alpha_n} (-1)^{\sum \alpha_i - 1} F_{k_1 \ldots k_n}\left( x_{k_1}^{(\alpha_1)} ,\ldots, x_{k_n}^{(\alpha_n)} \right)  ,  \; n \; \text{нечетно}.\\
\end{cases}
\]

К свойствам (1-7) в этом случае необходимо добавить еще одно свойство:
 для любой перестановки  $\pi$ 
\[
 F_{\pi(k_1) \ldots \pi(k_n)} \left(x_{\pi(k_1)} ,\ldots, x_{\pi(k_n)} \right) =  F_{k_1 \ldots k_n}\left(x_{k_1} ,\ldots, x_{k_n}\right). 
\]

\end{remark}


\begin{problem}
Вы приходите на станцию метро в случайный момент времени $T$(например, $T$ имеет равномерное распределение на интервале между 10 и 11 часами) и садитесь в первый пришедший поезд (в ту или другую сторону). Будем считать, что поезда в обе стороны ходят одинаково часто (например, каждые 10 минут) согласно расписанию. Одинаковы ли вероятности событий, что Вы поедете в одну или в другую стороны? Рассмотрите случай, когда в одну стороны поезда идут в 10:00, 10:10, 10:20,\dots , а в другую -- в 10:02, 10:12, 10:22,\dots 
\end{problem}


\begin{remark} Казалось бы, интуиция подсказывает, что оба направления ``равноправны'' -- ведь мы приходим в ``случайный'' момент времени. ``Парадокс'' легко разрешается введением вероятностного пространства (конкретизацией ``случайности'' времени прихода на станцию).
\end{remark}

\begin{problem}
В урне находится $3$ белых и $2$ черных шара. 
Эксперимент состоит в последовательном извлечении из урны всех шаров по одному наугад без возвращения. Построить вероятностное пространство. 
Описать $\sigma$-алгебру, порожденную случайной величиной $X$, если: 
\begin{enumerate}
\item $X$ --- число белых шаров, предшествующих первому черному шару; 
\item $X$ --- число черных шаров среди извлеченных; 
\item $X=X_1+X_2$, где $X_1$ --- число белых шаров, предшествующих первому черному шару, 
$X_2$ --- число черных шаров, предшествующих белому шару. 
\end{enumerate}
\end{problem}


\begin{problem}\Star
\label{SigmaAlgebra}
Пусть $(\Omega,\Sigma,{\mathbb P})$ --- некоторое вероятностное пространство и $A$ --- алгебра подмножеств $\Omega$ такая, что 
$\sigma(A)=\Sigma$ ($\sigma(A)$ --- наименьшая $\sigma$-алгебра, содержащая алгебру $A$). Доказать, что 
$$
\forall\varepsilon>0,\, B\in\Sigma\quad \exists A_{\varepsilon}\in A:\quad {\mathbb P}(A_{\varepsilon}\bigtriangleup B)
\leqslant\varepsilon . 
$$
\end{problem}

\begin{ordre}
Рассмотрите совокупность множеств 
$$
{\mathcal B}=\bigl\{ B\in\Sigma\, | \, \forall\varepsilon>0 \; \exists A_B\in A:\; {\mathbb P}(A_B\bigtriangleup B)
\leqslant\varepsilon \bigr\} . 
$$

\noindent Покажите, что ${\mathcal B}$ является минимальной $\sigma$-алгеброй 

\end{ordre}



 \begin{problem}
 \begin{enumerate}
 
 \item Может ли приведенная ниже функция
\[
F\left( {x_1 ,x_2 } \right)=\left\{ {\begin{array}{l}
 1,\quad \min \left( {x_1 ,x_2 } \right)>1 \\ 
 0,\quad \min \left( {x_1 ,x_2 } \right)\le 1 \\ 
 \end{array}} \right.
\]
быть функцией распределения некоторого двумерного случайного вектора? 
Приведите необходимое и достаточное условие того, чтобы $F\left( {x_1 
,...,x_n } \right)$ была функцией распределения некоторого случайного 
вектора.

\item На ${\mathbb R}^n$ задана функция $F\left( {x_1 ,...,x_n } \right)$, 
неубывающая по каждому аргументу, такая, что $F\left( {x_1 ,...,-\infty 
,...,x_n } \right)=0$, $F\left( {\infty ,...,\infty } \right)=1$. Что еще 
надо потребовать от этой функции, чтобы с помощью неё можно было бы 
естественным образом задать счетно-аддитивную вероятностную меру на 
минимальной $\sigma $-алгебре, содержащей всевозможные прямоугольные 
параллелепипеды в ${\mathbb R}^n$?

\item Может ли функция распределения $F_X \left( x \right)$ с.в. $X$ 
иметь более чем счетное множество точек разрыва?

\end{enumerate}
\end{problem}

\begin{comment}
\begin{problem}[Пример о сходимости ряда]
Пусть $(\Omega,\Xi,{\mathbb P})$ --- вероятностное пространство, $\xi_1,\xi_2,\ldots$ --- некоторая последовательность с.в.. 
Обозначим $\Xi_n^{\infty}=\sigma(\xi_{n},\xi_{n+1},\ldots)$ --- $\sigma$-алгебру, порожденную с.в. $\xi_{n},\xi_{n+1},\ldots$ и пусть 
$$
{\mathcal X}=\bigcap\limits_{n=1}^{\infty} \Xi_{n}^{\infty}. 
$$
Поскольку пересечение $\sigma$-алгебр есть снова $\sigma$-алгебра, то ${\mathcal X}$ --- есть $\sigma$-алгебра. Эту $\sigma$-алгебру 
будем называть <<хвостовой>> или <<остаточной>>, в связи с тем, что всякое событие $A\in{\mathcal X}$ не зависит от значений с.в. 
$\xi_1,\xi_2,\ldots,\xi_n$ при любом конечном $n$, а определяется лишь <<поведением бесконечно далеких значений последовательности 
$\xi_1,\xi_2,\ldots$ >>. 

С помощью задачи $\ref{SigmaAlgebra}$ докажите справедливость следующего утверждения: 

Пусть $\xi_1,\xi_2,\ldots$ --- последовательность независимых в совокупности с.в. и $A\in{\mathcal X}$ 
(событие $A$ принадлежит <<хвостовой>> $\sigma$-алгебре). Тогда ${\mathbb P}(A)$ может принимать лишь два значения $0$ или $1$. 
\end{problem}

\begin{ordre}
Идея доказательства состоит в том, чтобы показать, что каждое <<хвостовое>> событие $A$ не зависит от самого себя и, значит, 
${\mathbb P}(A\cap A)={\mathbb P}(A)\cdot {\mathbb P}(A)$, т.е. ${\mathbb P}(A)={\mathbb P}^2(A)$, откуда 
${\mathbb P}(A)=0$ или $1$. 
\end{ordre}
\end{comment}